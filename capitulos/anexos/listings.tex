\chapter{Códigos}\label{chapter:anex_codes}
{ % para scope de cambio de settings y etc

\captionsetup[lstlisting]{font={Large}, justification=justified, singlelinecheck=false, format=plain}
\lstinputlisting[style=CPP, captionpos=t, label={code:pluginc},caption={Archivo \texttt{src/plugin.c} en su totalidad.}]{codigo/plugin.c}

\newpage

\lstinputlisting[style=CPP, captionpos=t, label={cod:traciserver_prestep}, caption={Método \texttt{preStep()} en \texttt{TraCIServer}}]{codigo/traciserver_prestep.cpp}

\newpage

\lstinputlisting[style=CPP, captionpos=t, label={cod:traciserver_poststep}, caption={Método \texttt{postStep()} en \texttt{TraCIServer}}]{codigo/traciserver_poststep.cpp}

\newpage

\lstinputlisting[style=CPP, captionpos=t, label={code:variablesubscription_main},caption={Métodos base de todas las suscripciones.}]{codigo/variablesubscription_mainmethods.cpp}

\newpage

\lstinputlisting[style=CPP, captionpos=t, label={code:addsub},caption={Método de actualización y creación de suscripciones en \texttt{TraCIServer} en su totalidad.}]{codigo/traciserver_addsubscription.cpp}

\newpage

\lstinputlisting[style=CPP, captionpos=t, label={code:simvar},caption={Obtención de variables en \texttt{Simulation}. \texttt{VehicleManager} y \texttt{Network} cuentan con métodos análogos a los presentados aquí, por lo que no se expondrán en este documento.}]{codigo/simulation_getsimvar.cpp}

\newpage

\lstinputlisting[style=CPP, captionpos=t, label={code:getrealnetworkbounds},caption={Obtención de los límites del escenario de transporte.}]{codigo/getrealnetworkbounds.cpp}

\newpage

\lstinputlisting[style=CPP, captionpos=t, label={code:speedcontrollers},caption={Implementación de los controladores de velocidad.}]{codigo/speedcontrollers.cpp}

} % end scope