\section{Simulation}\label{sec:simulation}

La principal funcionalidad de este módulo es abstraer y encapsular el acceso a los parámetros de la simulación vehicular de Paramics. Se implementó como una clase de C++ utilizando el patrón de diseño \emph{singleton}; esto quiere decir que sólo se permite la instanciación de un único objeto de este tipo en la ejecución del programa. Esto ya que, por razones lógicas, cada ejecución del \emph{plugin} está asociada a una única simulación en Paramics, y por ende no tiene sentido que pueda existir más de un objeto de acceso a ésta. Este patrón de diseño tiene además la ventaja que simplifica el acceso a la instancia global de la clase en el sistema, desde cualquier otro objeto u función.

\subsection{Obtención de variables}\label{sec:simulation:vars}

Las variables obtenibles desde este módulo son todas aquellas que se relacionan con la simulación como ente abstracto, enumeradas en el ítem \ref{item:simvars} de la sección \ref{sec:comandos}. La implementación de los métodos \texttt{packSimulationVariable()} y \texttt{getSimulationVariable()}, encargados de facilitar el acceso a las variables representadas por este módulo, pueden observarse en el código \ref{code:simvar} en los apéndices. Cabe destacar que los módulos \texttt{VehicleManager} y \texttt{Network} cuentan con métodos análogos \emph{muy} similares, por lo que no se incluirá el código de éstos últimos en el documento.

Se debe mencionar también la especial implementación de la obtención de algunas de las variables anteriormente mencionadas. En específico, las variables referentes a los vehículos que comenzaron o terminaron su viaje en el último paso de simulación son accesibles desde este módulo, pero su obtención fue implementada en el módulo \texttt{VehicleManager}. Esto ya que dicho módulo debe mantener una lista interna de todos los vehículos de la simulación en todo instante de tiempo, por lo que obtener estos valores era mucho más directo de implementar allá. Ver la sección sobre \texttt{VehicleManager}, \ref{sec:vehiclemanager}, para más detalles.

De las variables efectivamente implementadas en este módulo, vale destacar un par de detalles. En primer lugar, existe una diferencia entre cómo VEINS y OMNeT++ manejan el tiempo de simulación, y cómo lo hace Paramics; los primeros ocupan mili-segundos, mientras que este último ocupa segundos. Esto implicó realizar las respectivas conversiones necesarias.

En segundo lugar se hablará del comando de obtención de las coordenadas de los límites de la simulación. Este es de extrema importancia para VEINS, ya que con estos valores se crea el escenario de comunicación inalámbrica en OMNeT++; de ser erróneos, tarde o temprano la posición de un vehículo (representado por un nodo de comunicación en OMNeT++) quedará fuera del escenario, gatillando un error fatal en la simulación. 
Desafortunadamente, si bien la API de Paramics cuenta con un comando para, supuestamente, obtener estas coordenadas, por razones que no se lograron dilucidar, este comando retorna valores altamente erróneos (esto se verificó con múltiples redes de transporte).
Se debió entonces implementar el cálculo correcto de éstos límites en el módulo mismo, en el método, apropiadamente nombrado, \texttt{getRealNetworkBounds()} (expuesto en el código \ref{code:getrealnetworkbounds} en los anexos). Este cálculo se hace prácticamente a fuerza bruta, recorriendo todos los elementos que definen el alcance de la red (calles, intersecciones y zonas de emisión de vehículos), obteniendo sus coordenadas y luego obteniendo el rectángulo que las contiene (más un cierto margen de error).
Si bien este método no escala bien con redes más grandes, su impacto en la eficiencia del sistema se estimó como mínimo ya que se accede una única vez por simulación a este valor.