\subsection{plugin.c}\label{sec:plugin.c}

Si bien en estricto rigor no es un módulo del \emph{framework}, merece ser mencionado al ser el archivo principal del \emph{plugin} desarrollado. En este archivo se definen las funciones de extensión y \emph{override} (prefijos \texttt{QPX} y \texttt{QPO}, ver apéndice \ref{anex:paramics_api} para un detalle sobre la API de Paramics) a ser invocadas por Paramics. A continuación se describirán brevemente las más importantes de estas funciones, mientras que el archivo \texttt{plugin.c} puede estudiarse en su totalidad en el código \ref{code:pluginc} en los anexos.

\subsubsection{\texttt{void qpx\_NET\_postOpen()}}\label{sec:qpx_postopen}

Invocada inmediatamente luego de que Paramics carga la red y el \emph{plugin}, esta función inicializa el servidor TraCI. Para esto, crea un \emph{thread} donde corre una función auxiliar \texttt{runner\_fn()}, la cual se encarga de:

\begin{enumerate}
    \item Obtener el puerto en el cual esperar conexiones entrantes desde los parámetros de ejecución de Paramics. De no haberse especificado puerto, utiliza uno por defecto.
    \item Inicializar un objeto \texttt{TraCIServer} (ver sección \ref{sec:traciserver}) encargado de las conexiones entrantes en el puerto anteriormente definido.
\end{enumerate}

\subsubsection{\texttt{void qpx\_CLK\_startOfSimLoop()} y \texttt{void qpx\_CLK\_endOfSimLoop()}}

Estas funciones se ejecutan antes y después de cada paso de simulación respectivamente, y llaman a los procedimientos correspondientes en el servidor, los método \texttt{preStep()} y \texttt{postStep()}. Ver sección \ref{sec:traciserver} para más detalle sobre estos métodos y el avance de simulación en general.

\subsubsection{\texttt{void qpx\_VHC\_release(\dots)} y \texttt{void qpx\_VHC\_arrive(\dots)}}

\texttt{qpx\_VHC\_release(VEHICLE* vehicle)} es invocada por Paramics cada vez que un vehículo es liberado a la red de transporte. Simplemente se encarga de notificar al \texttt{VehicleManager} (ver sección \ref{sec:vehiclemanager}) para su inclusión en el modelo interno del \emph{plugin}.

Por otro lado, \texttt{qpx\_VHC\_arrive(VEHICLE* vehicle, LINK* link, ZONE* zone)} es invocada cuando un vehículo alcanza su destino final, y notifica al \texttt{VehicleManager} para eliminar el vehículo en cuestión de la representación interna.

\subsubsection{\texttt{int qpo\_RTM\_decision(\dots)}}\label{sec:plugin.c:decision}

Esta función de \emph{override} es llamada por el núcleo de simulación de Paramics cada vez que un vehículo necesita evaluar su elección de ruta, y debe retornar el índice de la siguiente salida que el vehículo debe tomar (o 0 si se desea mantener la ruta por defecto). En el \emph{plugin} se utiliza para aplicar rutas personalizadas otorgadas por el cliente TraCI.

\subsubsection{\texttt{void qpx\_VHC\_transfer(\dots)}}

Este método es ejecutado por Paramics cada vez que un vehículo pasa de una calle a otra, y se utiliza para determinar si es necesario recalcular la ruta del vehículo en cuestión.

\subsubsection{\texttt{float qpo\_CFM\_leadSpeed(\dots)} y \texttt{float qpo\_CFM\_followSpeed(\dots)}}\label{sec:plugin.c:speed}

Estas funciones se invocan en cada paso de simulación para cada vehículo en la simulación de tráfico de Paramics -- \texttt{leadSpeed()} se invoca para aquellos vehículos que no tienen otro vehículo delante, mientras \texttt{followSpeed()} es invocada los vehículos que se encuentran detrás de otro.

Estas funciones deben retornar la rapidez que se le deberá aplicar al vehículo en cuestión en el siguiente paso de simulación. En el \emph{framework}, se utilizan para aplicar cambios de velocidad dictados por comandos TraCI.