\section{Resultados}\label{sec:results}

\subsection{Análisis}

Los resultados obtenidos de cada simulación fuero exportados desde OMNeT++ a archivos CSV, los cuales luego se analizaron utilizando Python 3.6.
Se utilizaron las librerías Pandas \autocite{pandas}, para el manejo de los datos de manera eficiente, Numpy \autocite{numpy}, para cálculos, y Matplotlib \autocite{matplotlib} para la generación de gráficos que permitiesen analizar los datos de manera más efectiva e intuitiva.

\subsection{Eficiencia}

La figura \ref{fig:vehiclesvstime} presenta los resultados obtenidos del tiempo de duración de la simulación versus la cantidad de vehículos promedio por instante de tiempo, mientras que la tabla \ref{table:vehiclesvstime} presenta estos mismos datos de manera resumida. De estos resultados se puede concluir que el \emph{framework}, como era de esperarse dada la naturaleza de la simulación, presenta una relación levemente exponencial entre la cantidad promedio de nodos en la red vehicular con el tiempo que efectivamente demora una simulación en completar su ejecución.

Se utilizó numpy para calcular un ajuste polinomial a los datos obtenidos, obteniendo la siguiente relación entre cantidad de vehículos promedio en la simulación $N_{v}$ y tiempo de ejecución de ésta en segundos, $T(N_{v})$:

\[ T(N_{v}) = (5.829 \times 10^{-4})N_{v}^{2} + (2.586 \times 10^{-1})N_{v} + 6.317 \]

Esto indica que a pesar de ser exponencial, la relación presenta una curva bastante suavizada. Es factible entonces simular escenarios de escala aún mayor que la presentada en esta memoria (la cual, cabe notar, no es menor), sin mayores dificultades.

\begin{figure}[htpb]
    \centering
    \begin{tabular}{@{}rrr@{}}
        \textbf{Factor de Demanda} & \textbf{Nro. Prom. Vehículos} & \textbf{Tiempo Promedio [s]} \\ \midrule
        100\%           & 1379.9          & 1471.5              \\ %\midrule
        75\%            & 868.75          & 683.5               \\ %\midrule
        50\%            & 514.5825        & 275.75              \\ %\midrule
        25\%            & 246.5675        & 113.25              \\ \bottomrule
    \end{tabular}
    \captionof{table}[Cantidad de vehículos vs. tiempo real de simulación]{Promedio cantidad de vehículos en simulación (por instante de tiempo) vs. tiempo promedio de simulación, 15 minutos de tiempo simulado.}
    \label{table:vehiclesvstime}
    
    
    %% Creator: Matplotlib, PGF backend
%%
%% To include the figure in your LaTeX document, write
%%   \input{<filename>.pgf}
%%
%% Make sure the required packages are loaded in your preamble
%%   \usepackage{pgf}
%%
%% Figures using additional raster images can only be included by \input if
%% they are in the same directory as the main LaTeX file. For loading figures
%% from other directories you can use the `import` package
%%   \usepackage{import}
%% and then include the figures with
%%   \import{<path to file>}{<filename>.pgf}
%%
%% Matplotlib used the following preamble
%%   \usepackage[utf8x]{inputenc}
%%   \usepackage[T1]{fontenc}
%%   \usepackage{cmbright}
%%
\begingroup%
\makeatletter%
\begin{pgfpicture}%
\pgfpathrectangle{\pgfpointorigin}{\pgfqpoint{6.400000in}{4.800000in}}%
\pgfusepath{use as bounding box, clip}%
\begin{pgfscope}%
\pgfsetbuttcap%
\pgfsetmiterjoin%
\definecolor{currentfill}{rgb}{1.000000,1.000000,1.000000}%
\pgfsetfillcolor{currentfill}%
\pgfsetlinewidth{0.000000pt}%
\definecolor{currentstroke}{rgb}{1.000000,1.000000,1.000000}%
\pgfsetstrokecolor{currentstroke}%
\pgfsetdash{}{0pt}%
\pgfpathmoveto{\pgfqpoint{0.000000in}{0.000000in}}%
\pgfpathlineto{\pgfqpoint{6.400000in}{0.000000in}}%
\pgfpathlineto{\pgfqpoint{6.400000in}{4.800000in}}%
\pgfpathlineto{\pgfqpoint{0.000000in}{4.800000in}}%
\pgfpathclose%
\pgfusepath{fill}%
\end{pgfscope}%
\begin{pgfscope}%
\pgfsetbuttcap%
\pgfsetmiterjoin%
\definecolor{currentfill}{rgb}{1.000000,1.000000,1.000000}%
\pgfsetfillcolor{currentfill}%
\pgfsetlinewidth{0.000000pt}%
\definecolor{currentstroke}{rgb}{0.000000,0.000000,0.000000}%
\pgfsetstrokecolor{currentstroke}%
\pgfsetstrokeopacity{0.000000}%
\pgfsetdash{}{0pt}%
\pgfpathmoveto{\pgfqpoint{0.800000in}{0.528000in}}%
\pgfpathlineto{\pgfqpoint{5.760000in}{0.528000in}}%
\pgfpathlineto{\pgfqpoint{5.760000in}{4.224000in}}%
\pgfpathlineto{\pgfqpoint{0.800000in}{4.224000in}}%
\pgfpathclose%
\pgfusepath{fill}%
\end{pgfscope}%
\begin{pgfscope}%
\pgfpathrectangle{\pgfqpoint{0.800000in}{0.528000in}}{\pgfqpoint{4.960000in}{3.696000in}} %
\pgfusepath{clip}%
\pgfsetrectcap%
\pgfsetroundjoin%
\pgfsetlinewidth{0.803000pt}%
\definecolor{currentstroke}{rgb}{0.631373,0.631373,0.631373}%
\pgfsetstrokecolor{currentstroke}%
\pgfsetstrokeopacity{0.100000}%
\pgfsetdash{}{0pt}%
\pgfpathmoveto{\pgfqpoint{0.985075in}{0.528000in}}%
\pgfpathlineto{\pgfqpoint{0.985075in}{4.224000in}}%
\pgfusepath{stroke}%
\end{pgfscope}%
\begin{pgfscope}%
\pgfsetbuttcap%
\pgfsetroundjoin%
\definecolor{currentfill}{rgb}{0.333333,0.333333,0.333333}%
\pgfsetfillcolor{currentfill}%
\pgfsetlinewidth{0.803000pt}%
\definecolor{currentstroke}{rgb}{0.333333,0.333333,0.333333}%
\pgfsetstrokecolor{currentstroke}%
\pgfsetdash{}{0pt}%
\pgfsys@defobject{currentmarker}{\pgfqpoint{0.000000in}{-0.048611in}}{\pgfqpoint{0.000000in}{0.000000in}}{%
\pgfpathmoveto{\pgfqpoint{0.000000in}{0.000000in}}%
\pgfpathlineto{\pgfqpoint{0.000000in}{-0.048611in}}%
\pgfusepath{stroke,fill}%
}%
\begin{pgfscope}%
\pgfsys@transformshift{0.985075in}{0.528000in}%
\pgfsys@useobject{currentmarker}{}%
\end{pgfscope}%
\end{pgfscope}%
\begin{pgfscope}%
\definecolor{textcolor}{rgb}{0.333333,0.333333,0.333333}%
\pgfsetstrokecolor{textcolor}%
\pgfsetfillcolor{textcolor}%
\pgftext[x=0.985075in,y=0.430778in,,top]{\color{textcolor}\sffamily\fontsize{10.000000}{12.000000}\selectfont 200}%
\end{pgfscope}%
\begin{pgfscope}%
\pgfpathrectangle{\pgfqpoint{0.800000in}{0.528000in}}{\pgfqpoint{4.960000in}{3.696000in}} %
\pgfusepath{clip}%
\pgfsetrectcap%
\pgfsetroundjoin%
\pgfsetlinewidth{0.803000pt}%
\definecolor{currentstroke}{rgb}{0.631373,0.631373,0.631373}%
\pgfsetstrokecolor{currentstroke}%
\pgfsetstrokeopacity{0.100000}%
\pgfsetdash{}{0pt}%
\pgfpathmoveto{\pgfqpoint{1.355224in}{0.528000in}}%
\pgfpathlineto{\pgfqpoint{1.355224in}{4.224000in}}%
\pgfusepath{stroke}%
\end{pgfscope}%
\begin{pgfscope}%
\pgfsetbuttcap%
\pgfsetroundjoin%
\definecolor{currentfill}{rgb}{0.333333,0.333333,0.333333}%
\pgfsetfillcolor{currentfill}%
\pgfsetlinewidth{0.803000pt}%
\definecolor{currentstroke}{rgb}{0.333333,0.333333,0.333333}%
\pgfsetstrokecolor{currentstroke}%
\pgfsetdash{}{0pt}%
\pgfsys@defobject{currentmarker}{\pgfqpoint{0.000000in}{-0.048611in}}{\pgfqpoint{0.000000in}{0.000000in}}{%
\pgfpathmoveto{\pgfqpoint{0.000000in}{0.000000in}}%
\pgfpathlineto{\pgfqpoint{0.000000in}{-0.048611in}}%
\pgfusepath{stroke,fill}%
}%
\begin{pgfscope}%
\pgfsys@transformshift{1.355224in}{0.528000in}%
\pgfsys@useobject{currentmarker}{}%
\end{pgfscope}%
\end{pgfscope}%
\begin{pgfscope}%
\definecolor{textcolor}{rgb}{0.333333,0.333333,0.333333}%
\pgfsetstrokecolor{textcolor}%
\pgfsetfillcolor{textcolor}%
\pgftext[x=1.355224in,y=0.430778in,,top]{\color{textcolor}\sffamily\fontsize{10.000000}{12.000000}\selectfont 300}%
\end{pgfscope}%
\begin{pgfscope}%
\pgfpathrectangle{\pgfqpoint{0.800000in}{0.528000in}}{\pgfqpoint{4.960000in}{3.696000in}} %
\pgfusepath{clip}%
\pgfsetrectcap%
\pgfsetroundjoin%
\pgfsetlinewidth{0.803000pt}%
\definecolor{currentstroke}{rgb}{0.631373,0.631373,0.631373}%
\pgfsetstrokecolor{currentstroke}%
\pgfsetstrokeopacity{0.100000}%
\pgfsetdash{}{0pt}%
\pgfpathmoveto{\pgfqpoint{1.725373in}{0.528000in}}%
\pgfpathlineto{\pgfqpoint{1.725373in}{4.224000in}}%
\pgfusepath{stroke}%
\end{pgfscope}%
\begin{pgfscope}%
\pgfsetbuttcap%
\pgfsetroundjoin%
\definecolor{currentfill}{rgb}{0.333333,0.333333,0.333333}%
\pgfsetfillcolor{currentfill}%
\pgfsetlinewidth{0.803000pt}%
\definecolor{currentstroke}{rgb}{0.333333,0.333333,0.333333}%
\pgfsetstrokecolor{currentstroke}%
\pgfsetdash{}{0pt}%
\pgfsys@defobject{currentmarker}{\pgfqpoint{0.000000in}{-0.048611in}}{\pgfqpoint{0.000000in}{0.000000in}}{%
\pgfpathmoveto{\pgfqpoint{0.000000in}{0.000000in}}%
\pgfpathlineto{\pgfqpoint{0.000000in}{-0.048611in}}%
\pgfusepath{stroke,fill}%
}%
\begin{pgfscope}%
\pgfsys@transformshift{1.725373in}{0.528000in}%
\pgfsys@useobject{currentmarker}{}%
\end{pgfscope}%
\end{pgfscope}%
\begin{pgfscope}%
\definecolor{textcolor}{rgb}{0.333333,0.333333,0.333333}%
\pgfsetstrokecolor{textcolor}%
\pgfsetfillcolor{textcolor}%
\pgftext[x=1.725373in,y=0.430778in,,top]{\color{textcolor}\sffamily\fontsize{10.000000}{12.000000}\selectfont 400}%
\end{pgfscope}%
\begin{pgfscope}%
\pgfpathrectangle{\pgfqpoint{0.800000in}{0.528000in}}{\pgfqpoint{4.960000in}{3.696000in}} %
\pgfusepath{clip}%
\pgfsetrectcap%
\pgfsetroundjoin%
\pgfsetlinewidth{0.803000pt}%
\definecolor{currentstroke}{rgb}{0.631373,0.631373,0.631373}%
\pgfsetstrokecolor{currentstroke}%
\pgfsetstrokeopacity{0.100000}%
\pgfsetdash{}{0pt}%
\pgfpathmoveto{\pgfqpoint{2.095522in}{0.528000in}}%
\pgfpathlineto{\pgfqpoint{2.095522in}{4.224000in}}%
\pgfusepath{stroke}%
\end{pgfscope}%
\begin{pgfscope}%
\pgfsetbuttcap%
\pgfsetroundjoin%
\definecolor{currentfill}{rgb}{0.333333,0.333333,0.333333}%
\pgfsetfillcolor{currentfill}%
\pgfsetlinewidth{0.803000pt}%
\definecolor{currentstroke}{rgb}{0.333333,0.333333,0.333333}%
\pgfsetstrokecolor{currentstroke}%
\pgfsetdash{}{0pt}%
\pgfsys@defobject{currentmarker}{\pgfqpoint{0.000000in}{-0.048611in}}{\pgfqpoint{0.000000in}{0.000000in}}{%
\pgfpathmoveto{\pgfqpoint{0.000000in}{0.000000in}}%
\pgfpathlineto{\pgfqpoint{0.000000in}{-0.048611in}}%
\pgfusepath{stroke,fill}%
}%
\begin{pgfscope}%
\pgfsys@transformshift{2.095522in}{0.528000in}%
\pgfsys@useobject{currentmarker}{}%
\end{pgfscope}%
\end{pgfscope}%
\begin{pgfscope}%
\definecolor{textcolor}{rgb}{0.333333,0.333333,0.333333}%
\pgfsetstrokecolor{textcolor}%
\pgfsetfillcolor{textcolor}%
\pgftext[x=2.095522in,y=0.430778in,,top]{\color{textcolor}\sffamily\fontsize{10.000000}{12.000000}\selectfont 500}%
\end{pgfscope}%
\begin{pgfscope}%
\pgfpathrectangle{\pgfqpoint{0.800000in}{0.528000in}}{\pgfqpoint{4.960000in}{3.696000in}} %
\pgfusepath{clip}%
\pgfsetrectcap%
\pgfsetroundjoin%
\pgfsetlinewidth{0.803000pt}%
\definecolor{currentstroke}{rgb}{0.631373,0.631373,0.631373}%
\pgfsetstrokecolor{currentstroke}%
\pgfsetstrokeopacity{0.100000}%
\pgfsetdash{}{0pt}%
\pgfpathmoveto{\pgfqpoint{2.465672in}{0.528000in}}%
\pgfpathlineto{\pgfqpoint{2.465672in}{4.224000in}}%
\pgfusepath{stroke}%
\end{pgfscope}%
\begin{pgfscope}%
\pgfsetbuttcap%
\pgfsetroundjoin%
\definecolor{currentfill}{rgb}{0.333333,0.333333,0.333333}%
\pgfsetfillcolor{currentfill}%
\pgfsetlinewidth{0.803000pt}%
\definecolor{currentstroke}{rgb}{0.333333,0.333333,0.333333}%
\pgfsetstrokecolor{currentstroke}%
\pgfsetdash{}{0pt}%
\pgfsys@defobject{currentmarker}{\pgfqpoint{0.000000in}{-0.048611in}}{\pgfqpoint{0.000000in}{0.000000in}}{%
\pgfpathmoveto{\pgfqpoint{0.000000in}{0.000000in}}%
\pgfpathlineto{\pgfqpoint{0.000000in}{-0.048611in}}%
\pgfusepath{stroke,fill}%
}%
\begin{pgfscope}%
\pgfsys@transformshift{2.465672in}{0.528000in}%
\pgfsys@useobject{currentmarker}{}%
\end{pgfscope}%
\end{pgfscope}%
\begin{pgfscope}%
\definecolor{textcolor}{rgb}{0.333333,0.333333,0.333333}%
\pgfsetstrokecolor{textcolor}%
\pgfsetfillcolor{textcolor}%
\pgftext[x=2.465672in,y=0.430778in,,top]{\color{textcolor}\sffamily\fontsize{10.000000}{12.000000}\selectfont 600}%
\end{pgfscope}%
\begin{pgfscope}%
\pgfpathrectangle{\pgfqpoint{0.800000in}{0.528000in}}{\pgfqpoint{4.960000in}{3.696000in}} %
\pgfusepath{clip}%
\pgfsetrectcap%
\pgfsetroundjoin%
\pgfsetlinewidth{0.803000pt}%
\definecolor{currentstroke}{rgb}{0.631373,0.631373,0.631373}%
\pgfsetstrokecolor{currentstroke}%
\pgfsetstrokeopacity{0.100000}%
\pgfsetdash{}{0pt}%
\pgfpathmoveto{\pgfqpoint{2.835821in}{0.528000in}}%
\pgfpathlineto{\pgfqpoint{2.835821in}{4.224000in}}%
\pgfusepath{stroke}%
\end{pgfscope}%
\begin{pgfscope}%
\pgfsetbuttcap%
\pgfsetroundjoin%
\definecolor{currentfill}{rgb}{0.333333,0.333333,0.333333}%
\pgfsetfillcolor{currentfill}%
\pgfsetlinewidth{0.803000pt}%
\definecolor{currentstroke}{rgb}{0.333333,0.333333,0.333333}%
\pgfsetstrokecolor{currentstroke}%
\pgfsetdash{}{0pt}%
\pgfsys@defobject{currentmarker}{\pgfqpoint{0.000000in}{-0.048611in}}{\pgfqpoint{0.000000in}{0.000000in}}{%
\pgfpathmoveto{\pgfqpoint{0.000000in}{0.000000in}}%
\pgfpathlineto{\pgfqpoint{0.000000in}{-0.048611in}}%
\pgfusepath{stroke,fill}%
}%
\begin{pgfscope}%
\pgfsys@transformshift{2.835821in}{0.528000in}%
\pgfsys@useobject{currentmarker}{}%
\end{pgfscope}%
\end{pgfscope}%
\begin{pgfscope}%
\definecolor{textcolor}{rgb}{0.333333,0.333333,0.333333}%
\pgfsetstrokecolor{textcolor}%
\pgfsetfillcolor{textcolor}%
\pgftext[x=2.835821in,y=0.430778in,,top]{\color{textcolor}\sffamily\fontsize{10.000000}{12.000000}\selectfont 700}%
\end{pgfscope}%
\begin{pgfscope}%
\pgfpathrectangle{\pgfqpoint{0.800000in}{0.528000in}}{\pgfqpoint{4.960000in}{3.696000in}} %
\pgfusepath{clip}%
\pgfsetrectcap%
\pgfsetroundjoin%
\pgfsetlinewidth{0.803000pt}%
\definecolor{currentstroke}{rgb}{0.631373,0.631373,0.631373}%
\pgfsetstrokecolor{currentstroke}%
\pgfsetstrokeopacity{0.100000}%
\pgfsetdash{}{0pt}%
\pgfpathmoveto{\pgfqpoint{3.205970in}{0.528000in}}%
\pgfpathlineto{\pgfqpoint{3.205970in}{4.224000in}}%
\pgfusepath{stroke}%
\end{pgfscope}%
\begin{pgfscope}%
\pgfsetbuttcap%
\pgfsetroundjoin%
\definecolor{currentfill}{rgb}{0.333333,0.333333,0.333333}%
\pgfsetfillcolor{currentfill}%
\pgfsetlinewidth{0.803000pt}%
\definecolor{currentstroke}{rgb}{0.333333,0.333333,0.333333}%
\pgfsetstrokecolor{currentstroke}%
\pgfsetdash{}{0pt}%
\pgfsys@defobject{currentmarker}{\pgfqpoint{0.000000in}{-0.048611in}}{\pgfqpoint{0.000000in}{0.000000in}}{%
\pgfpathmoveto{\pgfqpoint{0.000000in}{0.000000in}}%
\pgfpathlineto{\pgfqpoint{0.000000in}{-0.048611in}}%
\pgfusepath{stroke,fill}%
}%
\begin{pgfscope}%
\pgfsys@transformshift{3.205970in}{0.528000in}%
\pgfsys@useobject{currentmarker}{}%
\end{pgfscope}%
\end{pgfscope}%
\begin{pgfscope}%
\definecolor{textcolor}{rgb}{0.333333,0.333333,0.333333}%
\pgfsetstrokecolor{textcolor}%
\pgfsetfillcolor{textcolor}%
\pgftext[x=3.205970in,y=0.430778in,,top]{\color{textcolor}\sffamily\fontsize{10.000000}{12.000000}\selectfont 800}%
\end{pgfscope}%
\begin{pgfscope}%
\pgfpathrectangle{\pgfqpoint{0.800000in}{0.528000in}}{\pgfqpoint{4.960000in}{3.696000in}} %
\pgfusepath{clip}%
\pgfsetrectcap%
\pgfsetroundjoin%
\pgfsetlinewidth{0.803000pt}%
\definecolor{currentstroke}{rgb}{0.631373,0.631373,0.631373}%
\pgfsetstrokecolor{currentstroke}%
\pgfsetstrokeopacity{0.100000}%
\pgfsetdash{}{0pt}%
\pgfpathmoveto{\pgfqpoint{3.576119in}{0.528000in}}%
\pgfpathlineto{\pgfqpoint{3.576119in}{4.224000in}}%
\pgfusepath{stroke}%
\end{pgfscope}%
\begin{pgfscope}%
\pgfsetbuttcap%
\pgfsetroundjoin%
\definecolor{currentfill}{rgb}{0.333333,0.333333,0.333333}%
\pgfsetfillcolor{currentfill}%
\pgfsetlinewidth{0.803000pt}%
\definecolor{currentstroke}{rgb}{0.333333,0.333333,0.333333}%
\pgfsetstrokecolor{currentstroke}%
\pgfsetdash{}{0pt}%
\pgfsys@defobject{currentmarker}{\pgfqpoint{0.000000in}{-0.048611in}}{\pgfqpoint{0.000000in}{0.000000in}}{%
\pgfpathmoveto{\pgfqpoint{0.000000in}{0.000000in}}%
\pgfpathlineto{\pgfqpoint{0.000000in}{-0.048611in}}%
\pgfusepath{stroke,fill}%
}%
\begin{pgfscope}%
\pgfsys@transformshift{3.576119in}{0.528000in}%
\pgfsys@useobject{currentmarker}{}%
\end{pgfscope}%
\end{pgfscope}%
\begin{pgfscope}%
\definecolor{textcolor}{rgb}{0.333333,0.333333,0.333333}%
\pgfsetstrokecolor{textcolor}%
\pgfsetfillcolor{textcolor}%
\pgftext[x=3.576119in,y=0.430778in,,top]{\color{textcolor}\sffamily\fontsize{10.000000}{12.000000}\selectfont 900}%
\end{pgfscope}%
\begin{pgfscope}%
\pgfpathrectangle{\pgfqpoint{0.800000in}{0.528000in}}{\pgfqpoint{4.960000in}{3.696000in}} %
\pgfusepath{clip}%
\pgfsetrectcap%
\pgfsetroundjoin%
\pgfsetlinewidth{0.803000pt}%
\definecolor{currentstroke}{rgb}{0.631373,0.631373,0.631373}%
\pgfsetstrokecolor{currentstroke}%
\pgfsetstrokeopacity{0.100000}%
\pgfsetdash{}{0pt}%
\pgfpathmoveto{\pgfqpoint{3.946269in}{0.528000in}}%
\pgfpathlineto{\pgfqpoint{3.946269in}{4.224000in}}%
\pgfusepath{stroke}%
\end{pgfscope}%
\begin{pgfscope}%
\pgfsetbuttcap%
\pgfsetroundjoin%
\definecolor{currentfill}{rgb}{0.333333,0.333333,0.333333}%
\pgfsetfillcolor{currentfill}%
\pgfsetlinewidth{0.803000pt}%
\definecolor{currentstroke}{rgb}{0.333333,0.333333,0.333333}%
\pgfsetstrokecolor{currentstroke}%
\pgfsetdash{}{0pt}%
\pgfsys@defobject{currentmarker}{\pgfqpoint{0.000000in}{-0.048611in}}{\pgfqpoint{0.000000in}{0.000000in}}{%
\pgfpathmoveto{\pgfqpoint{0.000000in}{0.000000in}}%
\pgfpathlineto{\pgfqpoint{0.000000in}{-0.048611in}}%
\pgfusepath{stroke,fill}%
}%
\begin{pgfscope}%
\pgfsys@transformshift{3.946269in}{0.528000in}%
\pgfsys@useobject{currentmarker}{}%
\end{pgfscope}%
\end{pgfscope}%
\begin{pgfscope}%
\definecolor{textcolor}{rgb}{0.333333,0.333333,0.333333}%
\pgfsetstrokecolor{textcolor}%
\pgfsetfillcolor{textcolor}%
\pgftext[x=3.946269in,y=0.430778in,,top]{\color{textcolor}\sffamily\fontsize{10.000000}{12.000000}\selectfont 1000}%
\end{pgfscope}%
\begin{pgfscope}%
\pgfpathrectangle{\pgfqpoint{0.800000in}{0.528000in}}{\pgfqpoint{4.960000in}{3.696000in}} %
\pgfusepath{clip}%
\pgfsetrectcap%
\pgfsetroundjoin%
\pgfsetlinewidth{0.803000pt}%
\definecolor{currentstroke}{rgb}{0.631373,0.631373,0.631373}%
\pgfsetstrokecolor{currentstroke}%
\pgfsetstrokeopacity{0.100000}%
\pgfsetdash{}{0pt}%
\pgfpathmoveto{\pgfqpoint{4.316418in}{0.528000in}}%
\pgfpathlineto{\pgfqpoint{4.316418in}{4.224000in}}%
\pgfusepath{stroke}%
\end{pgfscope}%
\begin{pgfscope}%
\pgfsetbuttcap%
\pgfsetroundjoin%
\definecolor{currentfill}{rgb}{0.333333,0.333333,0.333333}%
\pgfsetfillcolor{currentfill}%
\pgfsetlinewidth{0.803000pt}%
\definecolor{currentstroke}{rgb}{0.333333,0.333333,0.333333}%
\pgfsetstrokecolor{currentstroke}%
\pgfsetdash{}{0pt}%
\pgfsys@defobject{currentmarker}{\pgfqpoint{0.000000in}{-0.048611in}}{\pgfqpoint{0.000000in}{0.000000in}}{%
\pgfpathmoveto{\pgfqpoint{0.000000in}{0.000000in}}%
\pgfpathlineto{\pgfqpoint{0.000000in}{-0.048611in}}%
\pgfusepath{stroke,fill}%
}%
\begin{pgfscope}%
\pgfsys@transformshift{4.316418in}{0.528000in}%
\pgfsys@useobject{currentmarker}{}%
\end{pgfscope}%
\end{pgfscope}%
\begin{pgfscope}%
\definecolor{textcolor}{rgb}{0.333333,0.333333,0.333333}%
\pgfsetstrokecolor{textcolor}%
\pgfsetfillcolor{textcolor}%
\pgftext[x=4.316418in,y=0.430778in,,top]{\color{textcolor}\sffamily\fontsize{10.000000}{12.000000}\selectfont 1100}%
\end{pgfscope}%
\begin{pgfscope}%
\pgfpathrectangle{\pgfqpoint{0.800000in}{0.528000in}}{\pgfqpoint{4.960000in}{3.696000in}} %
\pgfusepath{clip}%
\pgfsetrectcap%
\pgfsetroundjoin%
\pgfsetlinewidth{0.803000pt}%
\definecolor{currentstroke}{rgb}{0.631373,0.631373,0.631373}%
\pgfsetstrokecolor{currentstroke}%
\pgfsetstrokeopacity{0.100000}%
\pgfsetdash{}{0pt}%
\pgfpathmoveto{\pgfqpoint{4.686567in}{0.528000in}}%
\pgfpathlineto{\pgfqpoint{4.686567in}{4.224000in}}%
\pgfusepath{stroke}%
\end{pgfscope}%
\begin{pgfscope}%
\pgfsetbuttcap%
\pgfsetroundjoin%
\definecolor{currentfill}{rgb}{0.333333,0.333333,0.333333}%
\pgfsetfillcolor{currentfill}%
\pgfsetlinewidth{0.803000pt}%
\definecolor{currentstroke}{rgb}{0.333333,0.333333,0.333333}%
\pgfsetstrokecolor{currentstroke}%
\pgfsetdash{}{0pt}%
\pgfsys@defobject{currentmarker}{\pgfqpoint{0.000000in}{-0.048611in}}{\pgfqpoint{0.000000in}{0.000000in}}{%
\pgfpathmoveto{\pgfqpoint{0.000000in}{0.000000in}}%
\pgfpathlineto{\pgfqpoint{0.000000in}{-0.048611in}}%
\pgfusepath{stroke,fill}%
}%
\begin{pgfscope}%
\pgfsys@transformshift{4.686567in}{0.528000in}%
\pgfsys@useobject{currentmarker}{}%
\end{pgfscope}%
\end{pgfscope}%
\begin{pgfscope}%
\definecolor{textcolor}{rgb}{0.333333,0.333333,0.333333}%
\pgfsetstrokecolor{textcolor}%
\pgfsetfillcolor{textcolor}%
\pgftext[x=4.686567in,y=0.430778in,,top]{\color{textcolor}\sffamily\fontsize{10.000000}{12.000000}\selectfont 1200}%
\end{pgfscope}%
\begin{pgfscope}%
\pgfpathrectangle{\pgfqpoint{0.800000in}{0.528000in}}{\pgfqpoint{4.960000in}{3.696000in}} %
\pgfusepath{clip}%
\pgfsetrectcap%
\pgfsetroundjoin%
\pgfsetlinewidth{0.803000pt}%
\definecolor{currentstroke}{rgb}{0.631373,0.631373,0.631373}%
\pgfsetstrokecolor{currentstroke}%
\pgfsetstrokeopacity{0.100000}%
\pgfsetdash{}{0pt}%
\pgfpathmoveto{\pgfqpoint{5.056716in}{0.528000in}}%
\pgfpathlineto{\pgfqpoint{5.056716in}{4.224000in}}%
\pgfusepath{stroke}%
\end{pgfscope}%
\begin{pgfscope}%
\pgfsetbuttcap%
\pgfsetroundjoin%
\definecolor{currentfill}{rgb}{0.333333,0.333333,0.333333}%
\pgfsetfillcolor{currentfill}%
\pgfsetlinewidth{0.803000pt}%
\definecolor{currentstroke}{rgb}{0.333333,0.333333,0.333333}%
\pgfsetstrokecolor{currentstroke}%
\pgfsetdash{}{0pt}%
\pgfsys@defobject{currentmarker}{\pgfqpoint{0.000000in}{-0.048611in}}{\pgfqpoint{0.000000in}{0.000000in}}{%
\pgfpathmoveto{\pgfqpoint{0.000000in}{0.000000in}}%
\pgfpathlineto{\pgfqpoint{0.000000in}{-0.048611in}}%
\pgfusepath{stroke,fill}%
}%
\begin{pgfscope}%
\pgfsys@transformshift{5.056716in}{0.528000in}%
\pgfsys@useobject{currentmarker}{}%
\end{pgfscope}%
\end{pgfscope}%
\begin{pgfscope}%
\definecolor{textcolor}{rgb}{0.333333,0.333333,0.333333}%
\pgfsetstrokecolor{textcolor}%
\pgfsetfillcolor{textcolor}%
\pgftext[x=5.056716in,y=0.430778in,,top]{\color{textcolor}\sffamily\fontsize{10.000000}{12.000000}\selectfont 1300}%
\end{pgfscope}%
\begin{pgfscope}%
\pgfpathrectangle{\pgfqpoint{0.800000in}{0.528000in}}{\pgfqpoint{4.960000in}{3.696000in}} %
\pgfusepath{clip}%
\pgfsetrectcap%
\pgfsetroundjoin%
\pgfsetlinewidth{0.803000pt}%
\definecolor{currentstroke}{rgb}{0.631373,0.631373,0.631373}%
\pgfsetstrokecolor{currentstroke}%
\pgfsetstrokeopacity{0.100000}%
\pgfsetdash{}{0pt}%
\pgfpathmoveto{\pgfqpoint{5.426866in}{0.528000in}}%
\pgfpathlineto{\pgfqpoint{5.426866in}{4.224000in}}%
\pgfusepath{stroke}%
\end{pgfscope}%
\begin{pgfscope}%
\pgfsetbuttcap%
\pgfsetroundjoin%
\definecolor{currentfill}{rgb}{0.333333,0.333333,0.333333}%
\pgfsetfillcolor{currentfill}%
\pgfsetlinewidth{0.803000pt}%
\definecolor{currentstroke}{rgb}{0.333333,0.333333,0.333333}%
\pgfsetstrokecolor{currentstroke}%
\pgfsetdash{}{0pt}%
\pgfsys@defobject{currentmarker}{\pgfqpoint{0.000000in}{-0.048611in}}{\pgfqpoint{0.000000in}{0.000000in}}{%
\pgfpathmoveto{\pgfqpoint{0.000000in}{0.000000in}}%
\pgfpathlineto{\pgfqpoint{0.000000in}{-0.048611in}}%
\pgfusepath{stroke,fill}%
}%
\begin{pgfscope}%
\pgfsys@transformshift{5.426866in}{0.528000in}%
\pgfsys@useobject{currentmarker}{}%
\end{pgfscope}%
\end{pgfscope}%
\begin{pgfscope}%
\definecolor{textcolor}{rgb}{0.333333,0.333333,0.333333}%
\pgfsetstrokecolor{textcolor}%
\pgfsetfillcolor{textcolor}%
\pgftext[x=5.426866in,y=0.430778in,,top]{\color{textcolor}\sffamily\fontsize{10.000000}{12.000000}\selectfont 1400}%
\end{pgfscope}%
\begin{pgfscope}%
\pgfsetbuttcap%
\pgfsetroundjoin%
\definecolor{currentfill}{rgb}{0.333333,0.333333,0.333333}%
\pgfsetfillcolor{currentfill}%
\pgfsetlinewidth{0.602250pt}%
\definecolor{currentstroke}{rgb}{0.333333,0.333333,0.333333}%
\pgfsetstrokecolor{currentstroke}%
\pgfsetdash{}{0pt}%
\pgfsys@defobject{currentmarker}{\pgfqpoint{0.000000in}{-0.027778in}}{\pgfqpoint{0.000000in}{0.000000in}}{%
\pgfpathmoveto{\pgfqpoint{0.000000in}{0.000000in}}%
\pgfpathlineto{\pgfqpoint{0.000000in}{-0.027778in}}%
\pgfusepath{stroke,fill}%
}%
\begin{pgfscope}%
\pgfsys@transformshift{0.800000in}{0.528000in}%
\pgfsys@useobject{currentmarker}{}%
\end{pgfscope}%
\end{pgfscope}%
\begin{pgfscope}%
\pgfsetbuttcap%
\pgfsetroundjoin%
\definecolor{currentfill}{rgb}{0.333333,0.333333,0.333333}%
\pgfsetfillcolor{currentfill}%
\pgfsetlinewidth{0.602250pt}%
\definecolor{currentstroke}{rgb}{0.333333,0.333333,0.333333}%
\pgfsetstrokecolor{currentstroke}%
\pgfsetdash{}{0pt}%
\pgfsys@defobject{currentmarker}{\pgfqpoint{0.000000in}{-0.027778in}}{\pgfqpoint{0.000000in}{0.000000in}}{%
\pgfpathmoveto{\pgfqpoint{0.000000in}{0.000000in}}%
\pgfpathlineto{\pgfqpoint{0.000000in}{-0.027778in}}%
\pgfusepath{stroke,fill}%
}%
\begin{pgfscope}%
\pgfsys@transformshift{0.837015in}{0.528000in}%
\pgfsys@useobject{currentmarker}{}%
\end{pgfscope}%
\end{pgfscope}%
\begin{pgfscope}%
\pgfsetbuttcap%
\pgfsetroundjoin%
\definecolor{currentfill}{rgb}{0.333333,0.333333,0.333333}%
\pgfsetfillcolor{currentfill}%
\pgfsetlinewidth{0.602250pt}%
\definecolor{currentstroke}{rgb}{0.333333,0.333333,0.333333}%
\pgfsetstrokecolor{currentstroke}%
\pgfsetdash{}{0pt}%
\pgfsys@defobject{currentmarker}{\pgfqpoint{0.000000in}{-0.027778in}}{\pgfqpoint{0.000000in}{0.000000in}}{%
\pgfpathmoveto{\pgfqpoint{0.000000in}{0.000000in}}%
\pgfpathlineto{\pgfqpoint{0.000000in}{-0.027778in}}%
\pgfusepath{stroke,fill}%
}%
\begin{pgfscope}%
\pgfsys@transformshift{0.874030in}{0.528000in}%
\pgfsys@useobject{currentmarker}{}%
\end{pgfscope}%
\end{pgfscope}%
\begin{pgfscope}%
\pgfsetbuttcap%
\pgfsetroundjoin%
\definecolor{currentfill}{rgb}{0.333333,0.333333,0.333333}%
\pgfsetfillcolor{currentfill}%
\pgfsetlinewidth{0.602250pt}%
\definecolor{currentstroke}{rgb}{0.333333,0.333333,0.333333}%
\pgfsetstrokecolor{currentstroke}%
\pgfsetdash{}{0pt}%
\pgfsys@defobject{currentmarker}{\pgfqpoint{0.000000in}{-0.027778in}}{\pgfqpoint{0.000000in}{0.000000in}}{%
\pgfpathmoveto{\pgfqpoint{0.000000in}{0.000000in}}%
\pgfpathlineto{\pgfqpoint{0.000000in}{-0.027778in}}%
\pgfusepath{stroke,fill}%
}%
\begin{pgfscope}%
\pgfsys@transformshift{0.911045in}{0.528000in}%
\pgfsys@useobject{currentmarker}{}%
\end{pgfscope}%
\end{pgfscope}%
\begin{pgfscope}%
\pgfsetbuttcap%
\pgfsetroundjoin%
\definecolor{currentfill}{rgb}{0.333333,0.333333,0.333333}%
\pgfsetfillcolor{currentfill}%
\pgfsetlinewidth{0.602250pt}%
\definecolor{currentstroke}{rgb}{0.333333,0.333333,0.333333}%
\pgfsetstrokecolor{currentstroke}%
\pgfsetdash{}{0pt}%
\pgfsys@defobject{currentmarker}{\pgfqpoint{0.000000in}{-0.027778in}}{\pgfqpoint{0.000000in}{0.000000in}}{%
\pgfpathmoveto{\pgfqpoint{0.000000in}{0.000000in}}%
\pgfpathlineto{\pgfqpoint{0.000000in}{-0.027778in}}%
\pgfusepath{stroke,fill}%
}%
\begin{pgfscope}%
\pgfsys@transformshift{0.948060in}{0.528000in}%
\pgfsys@useobject{currentmarker}{}%
\end{pgfscope}%
\end{pgfscope}%
\begin{pgfscope}%
\pgfsetbuttcap%
\pgfsetroundjoin%
\definecolor{currentfill}{rgb}{0.333333,0.333333,0.333333}%
\pgfsetfillcolor{currentfill}%
\pgfsetlinewidth{0.602250pt}%
\definecolor{currentstroke}{rgb}{0.333333,0.333333,0.333333}%
\pgfsetstrokecolor{currentstroke}%
\pgfsetdash{}{0pt}%
\pgfsys@defobject{currentmarker}{\pgfqpoint{0.000000in}{-0.027778in}}{\pgfqpoint{0.000000in}{0.000000in}}{%
\pgfpathmoveto{\pgfqpoint{0.000000in}{0.000000in}}%
\pgfpathlineto{\pgfqpoint{0.000000in}{-0.027778in}}%
\pgfusepath{stroke,fill}%
}%
\begin{pgfscope}%
\pgfsys@transformshift{0.985075in}{0.528000in}%
\pgfsys@useobject{currentmarker}{}%
\end{pgfscope}%
\end{pgfscope}%
\begin{pgfscope}%
\pgfsetbuttcap%
\pgfsetroundjoin%
\definecolor{currentfill}{rgb}{0.333333,0.333333,0.333333}%
\pgfsetfillcolor{currentfill}%
\pgfsetlinewidth{0.602250pt}%
\definecolor{currentstroke}{rgb}{0.333333,0.333333,0.333333}%
\pgfsetstrokecolor{currentstroke}%
\pgfsetdash{}{0pt}%
\pgfsys@defobject{currentmarker}{\pgfqpoint{0.000000in}{-0.027778in}}{\pgfqpoint{0.000000in}{0.000000in}}{%
\pgfpathmoveto{\pgfqpoint{0.000000in}{0.000000in}}%
\pgfpathlineto{\pgfqpoint{0.000000in}{-0.027778in}}%
\pgfusepath{stroke,fill}%
}%
\begin{pgfscope}%
\pgfsys@transformshift{1.022090in}{0.528000in}%
\pgfsys@useobject{currentmarker}{}%
\end{pgfscope}%
\end{pgfscope}%
\begin{pgfscope}%
\pgfsetbuttcap%
\pgfsetroundjoin%
\definecolor{currentfill}{rgb}{0.333333,0.333333,0.333333}%
\pgfsetfillcolor{currentfill}%
\pgfsetlinewidth{0.602250pt}%
\definecolor{currentstroke}{rgb}{0.333333,0.333333,0.333333}%
\pgfsetstrokecolor{currentstroke}%
\pgfsetdash{}{0pt}%
\pgfsys@defobject{currentmarker}{\pgfqpoint{0.000000in}{-0.027778in}}{\pgfqpoint{0.000000in}{0.000000in}}{%
\pgfpathmoveto{\pgfqpoint{0.000000in}{0.000000in}}%
\pgfpathlineto{\pgfqpoint{0.000000in}{-0.027778in}}%
\pgfusepath{stroke,fill}%
}%
\begin{pgfscope}%
\pgfsys@transformshift{1.059104in}{0.528000in}%
\pgfsys@useobject{currentmarker}{}%
\end{pgfscope}%
\end{pgfscope}%
\begin{pgfscope}%
\pgfsetbuttcap%
\pgfsetroundjoin%
\definecolor{currentfill}{rgb}{0.333333,0.333333,0.333333}%
\pgfsetfillcolor{currentfill}%
\pgfsetlinewidth{0.602250pt}%
\definecolor{currentstroke}{rgb}{0.333333,0.333333,0.333333}%
\pgfsetstrokecolor{currentstroke}%
\pgfsetdash{}{0pt}%
\pgfsys@defobject{currentmarker}{\pgfqpoint{0.000000in}{-0.027778in}}{\pgfqpoint{0.000000in}{0.000000in}}{%
\pgfpathmoveto{\pgfqpoint{0.000000in}{0.000000in}}%
\pgfpathlineto{\pgfqpoint{0.000000in}{-0.027778in}}%
\pgfusepath{stroke,fill}%
}%
\begin{pgfscope}%
\pgfsys@transformshift{1.096119in}{0.528000in}%
\pgfsys@useobject{currentmarker}{}%
\end{pgfscope}%
\end{pgfscope}%
\begin{pgfscope}%
\pgfsetbuttcap%
\pgfsetroundjoin%
\definecolor{currentfill}{rgb}{0.333333,0.333333,0.333333}%
\pgfsetfillcolor{currentfill}%
\pgfsetlinewidth{0.602250pt}%
\definecolor{currentstroke}{rgb}{0.333333,0.333333,0.333333}%
\pgfsetstrokecolor{currentstroke}%
\pgfsetdash{}{0pt}%
\pgfsys@defobject{currentmarker}{\pgfqpoint{0.000000in}{-0.027778in}}{\pgfqpoint{0.000000in}{0.000000in}}{%
\pgfpathmoveto{\pgfqpoint{0.000000in}{0.000000in}}%
\pgfpathlineto{\pgfqpoint{0.000000in}{-0.027778in}}%
\pgfusepath{stroke,fill}%
}%
\begin{pgfscope}%
\pgfsys@transformshift{1.133134in}{0.528000in}%
\pgfsys@useobject{currentmarker}{}%
\end{pgfscope}%
\end{pgfscope}%
\begin{pgfscope}%
\pgfsetbuttcap%
\pgfsetroundjoin%
\definecolor{currentfill}{rgb}{0.333333,0.333333,0.333333}%
\pgfsetfillcolor{currentfill}%
\pgfsetlinewidth{0.602250pt}%
\definecolor{currentstroke}{rgb}{0.333333,0.333333,0.333333}%
\pgfsetstrokecolor{currentstroke}%
\pgfsetdash{}{0pt}%
\pgfsys@defobject{currentmarker}{\pgfqpoint{0.000000in}{-0.027778in}}{\pgfqpoint{0.000000in}{0.000000in}}{%
\pgfpathmoveto{\pgfqpoint{0.000000in}{0.000000in}}%
\pgfpathlineto{\pgfqpoint{0.000000in}{-0.027778in}}%
\pgfusepath{stroke,fill}%
}%
\begin{pgfscope}%
\pgfsys@transformshift{1.170149in}{0.528000in}%
\pgfsys@useobject{currentmarker}{}%
\end{pgfscope}%
\end{pgfscope}%
\begin{pgfscope}%
\pgfsetbuttcap%
\pgfsetroundjoin%
\definecolor{currentfill}{rgb}{0.333333,0.333333,0.333333}%
\pgfsetfillcolor{currentfill}%
\pgfsetlinewidth{0.602250pt}%
\definecolor{currentstroke}{rgb}{0.333333,0.333333,0.333333}%
\pgfsetstrokecolor{currentstroke}%
\pgfsetdash{}{0pt}%
\pgfsys@defobject{currentmarker}{\pgfqpoint{0.000000in}{-0.027778in}}{\pgfqpoint{0.000000in}{0.000000in}}{%
\pgfpathmoveto{\pgfqpoint{0.000000in}{0.000000in}}%
\pgfpathlineto{\pgfqpoint{0.000000in}{-0.027778in}}%
\pgfusepath{stroke,fill}%
}%
\begin{pgfscope}%
\pgfsys@transformshift{1.207164in}{0.528000in}%
\pgfsys@useobject{currentmarker}{}%
\end{pgfscope}%
\end{pgfscope}%
\begin{pgfscope}%
\pgfsetbuttcap%
\pgfsetroundjoin%
\definecolor{currentfill}{rgb}{0.333333,0.333333,0.333333}%
\pgfsetfillcolor{currentfill}%
\pgfsetlinewidth{0.602250pt}%
\definecolor{currentstroke}{rgb}{0.333333,0.333333,0.333333}%
\pgfsetstrokecolor{currentstroke}%
\pgfsetdash{}{0pt}%
\pgfsys@defobject{currentmarker}{\pgfqpoint{0.000000in}{-0.027778in}}{\pgfqpoint{0.000000in}{0.000000in}}{%
\pgfpathmoveto{\pgfqpoint{0.000000in}{0.000000in}}%
\pgfpathlineto{\pgfqpoint{0.000000in}{-0.027778in}}%
\pgfusepath{stroke,fill}%
}%
\begin{pgfscope}%
\pgfsys@transformshift{1.244179in}{0.528000in}%
\pgfsys@useobject{currentmarker}{}%
\end{pgfscope}%
\end{pgfscope}%
\begin{pgfscope}%
\pgfsetbuttcap%
\pgfsetroundjoin%
\definecolor{currentfill}{rgb}{0.333333,0.333333,0.333333}%
\pgfsetfillcolor{currentfill}%
\pgfsetlinewidth{0.602250pt}%
\definecolor{currentstroke}{rgb}{0.333333,0.333333,0.333333}%
\pgfsetstrokecolor{currentstroke}%
\pgfsetdash{}{0pt}%
\pgfsys@defobject{currentmarker}{\pgfqpoint{0.000000in}{-0.027778in}}{\pgfqpoint{0.000000in}{0.000000in}}{%
\pgfpathmoveto{\pgfqpoint{0.000000in}{0.000000in}}%
\pgfpathlineto{\pgfqpoint{0.000000in}{-0.027778in}}%
\pgfusepath{stroke,fill}%
}%
\begin{pgfscope}%
\pgfsys@transformshift{1.281194in}{0.528000in}%
\pgfsys@useobject{currentmarker}{}%
\end{pgfscope}%
\end{pgfscope}%
\begin{pgfscope}%
\pgfsetbuttcap%
\pgfsetroundjoin%
\definecolor{currentfill}{rgb}{0.333333,0.333333,0.333333}%
\pgfsetfillcolor{currentfill}%
\pgfsetlinewidth{0.602250pt}%
\definecolor{currentstroke}{rgb}{0.333333,0.333333,0.333333}%
\pgfsetstrokecolor{currentstroke}%
\pgfsetdash{}{0pt}%
\pgfsys@defobject{currentmarker}{\pgfqpoint{0.000000in}{-0.027778in}}{\pgfqpoint{0.000000in}{0.000000in}}{%
\pgfpathmoveto{\pgfqpoint{0.000000in}{0.000000in}}%
\pgfpathlineto{\pgfqpoint{0.000000in}{-0.027778in}}%
\pgfusepath{stroke,fill}%
}%
\begin{pgfscope}%
\pgfsys@transformshift{1.318209in}{0.528000in}%
\pgfsys@useobject{currentmarker}{}%
\end{pgfscope}%
\end{pgfscope}%
\begin{pgfscope}%
\pgfsetbuttcap%
\pgfsetroundjoin%
\definecolor{currentfill}{rgb}{0.333333,0.333333,0.333333}%
\pgfsetfillcolor{currentfill}%
\pgfsetlinewidth{0.602250pt}%
\definecolor{currentstroke}{rgb}{0.333333,0.333333,0.333333}%
\pgfsetstrokecolor{currentstroke}%
\pgfsetdash{}{0pt}%
\pgfsys@defobject{currentmarker}{\pgfqpoint{0.000000in}{-0.027778in}}{\pgfqpoint{0.000000in}{0.000000in}}{%
\pgfpathmoveto{\pgfqpoint{0.000000in}{0.000000in}}%
\pgfpathlineto{\pgfqpoint{0.000000in}{-0.027778in}}%
\pgfusepath{stroke,fill}%
}%
\begin{pgfscope}%
\pgfsys@transformshift{1.355224in}{0.528000in}%
\pgfsys@useobject{currentmarker}{}%
\end{pgfscope}%
\end{pgfscope}%
\begin{pgfscope}%
\pgfsetbuttcap%
\pgfsetroundjoin%
\definecolor{currentfill}{rgb}{0.333333,0.333333,0.333333}%
\pgfsetfillcolor{currentfill}%
\pgfsetlinewidth{0.602250pt}%
\definecolor{currentstroke}{rgb}{0.333333,0.333333,0.333333}%
\pgfsetstrokecolor{currentstroke}%
\pgfsetdash{}{0pt}%
\pgfsys@defobject{currentmarker}{\pgfqpoint{0.000000in}{-0.027778in}}{\pgfqpoint{0.000000in}{0.000000in}}{%
\pgfpathmoveto{\pgfqpoint{0.000000in}{0.000000in}}%
\pgfpathlineto{\pgfqpoint{0.000000in}{-0.027778in}}%
\pgfusepath{stroke,fill}%
}%
\begin{pgfscope}%
\pgfsys@transformshift{1.392239in}{0.528000in}%
\pgfsys@useobject{currentmarker}{}%
\end{pgfscope}%
\end{pgfscope}%
\begin{pgfscope}%
\pgfsetbuttcap%
\pgfsetroundjoin%
\definecolor{currentfill}{rgb}{0.333333,0.333333,0.333333}%
\pgfsetfillcolor{currentfill}%
\pgfsetlinewidth{0.602250pt}%
\definecolor{currentstroke}{rgb}{0.333333,0.333333,0.333333}%
\pgfsetstrokecolor{currentstroke}%
\pgfsetdash{}{0pt}%
\pgfsys@defobject{currentmarker}{\pgfqpoint{0.000000in}{-0.027778in}}{\pgfqpoint{0.000000in}{0.000000in}}{%
\pgfpathmoveto{\pgfqpoint{0.000000in}{0.000000in}}%
\pgfpathlineto{\pgfqpoint{0.000000in}{-0.027778in}}%
\pgfusepath{stroke,fill}%
}%
\begin{pgfscope}%
\pgfsys@transformshift{1.429254in}{0.528000in}%
\pgfsys@useobject{currentmarker}{}%
\end{pgfscope}%
\end{pgfscope}%
\begin{pgfscope}%
\pgfsetbuttcap%
\pgfsetroundjoin%
\definecolor{currentfill}{rgb}{0.333333,0.333333,0.333333}%
\pgfsetfillcolor{currentfill}%
\pgfsetlinewidth{0.602250pt}%
\definecolor{currentstroke}{rgb}{0.333333,0.333333,0.333333}%
\pgfsetstrokecolor{currentstroke}%
\pgfsetdash{}{0pt}%
\pgfsys@defobject{currentmarker}{\pgfqpoint{0.000000in}{-0.027778in}}{\pgfqpoint{0.000000in}{0.000000in}}{%
\pgfpathmoveto{\pgfqpoint{0.000000in}{0.000000in}}%
\pgfpathlineto{\pgfqpoint{0.000000in}{-0.027778in}}%
\pgfusepath{stroke,fill}%
}%
\begin{pgfscope}%
\pgfsys@transformshift{1.466269in}{0.528000in}%
\pgfsys@useobject{currentmarker}{}%
\end{pgfscope}%
\end{pgfscope}%
\begin{pgfscope}%
\pgfsetbuttcap%
\pgfsetroundjoin%
\definecolor{currentfill}{rgb}{0.333333,0.333333,0.333333}%
\pgfsetfillcolor{currentfill}%
\pgfsetlinewidth{0.602250pt}%
\definecolor{currentstroke}{rgb}{0.333333,0.333333,0.333333}%
\pgfsetstrokecolor{currentstroke}%
\pgfsetdash{}{0pt}%
\pgfsys@defobject{currentmarker}{\pgfqpoint{0.000000in}{-0.027778in}}{\pgfqpoint{0.000000in}{0.000000in}}{%
\pgfpathmoveto{\pgfqpoint{0.000000in}{0.000000in}}%
\pgfpathlineto{\pgfqpoint{0.000000in}{-0.027778in}}%
\pgfusepath{stroke,fill}%
}%
\begin{pgfscope}%
\pgfsys@transformshift{1.503284in}{0.528000in}%
\pgfsys@useobject{currentmarker}{}%
\end{pgfscope}%
\end{pgfscope}%
\begin{pgfscope}%
\pgfsetbuttcap%
\pgfsetroundjoin%
\definecolor{currentfill}{rgb}{0.333333,0.333333,0.333333}%
\pgfsetfillcolor{currentfill}%
\pgfsetlinewidth{0.602250pt}%
\definecolor{currentstroke}{rgb}{0.333333,0.333333,0.333333}%
\pgfsetstrokecolor{currentstroke}%
\pgfsetdash{}{0pt}%
\pgfsys@defobject{currentmarker}{\pgfqpoint{0.000000in}{-0.027778in}}{\pgfqpoint{0.000000in}{0.000000in}}{%
\pgfpathmoveto{\pgfqpoint{0.000000in}{0.000000in}}%
\pgfpathlineto{\pgfqpoint{0.000000in}{-0.027778in}}%
\pgfusepath{stroke,fill}%
}%
\begin{pgfscope}%
\pgfsys@transformshift{1.540299in}{0.528000in}%
\pgfsys@useobject{currentmarker}{}%
\end{pgfscope}%
\end{pgfscope}%
\begin{pgfscope}%
\pgfsetbuttcap%
\pgfsetroundjoin%
\definecolor{currentfill}{rgb}{0.333333,0.333333,0.333333}%
\pgfsetfillcolor{currentfill}%
\pgfsetlinewidth{0.602250pt}%
\definecolor{currentstroke}{rgb}{0.333333,0.333333,0.333333}%
\pgfsetstrokecolor{currentstroke}%
\pgfsetdash{}{0pt}%
\pgfsys@defobject{currentmarker}{\pgfqpoint{0.000000in}{-0.027778in}}{\pgfqpoint{0.000000in}{0.000000in}}{%
\pgfpathmoveto{\pgfqpoint{0.000000in}{0.000000in}}%
\pgfpathlineto{\pgfqpoint{0.000000in}{-0.027778in}}%
\pgfusepath{stroke,fill}%
}%
\begin{pgfscope}%
\pgfsys@transformshift{1.577313in}{0.528000in}%
\pgfsys@useobject{currentmarker}{}%
\end{pgfscope}%
\end{pgfscope}%
\begin{pgfscope}%
\pgfsetbuttcap%
\pgfsetroundjoin%
\definecolor{currentfill}{rgb}{0.333333,0.333333,0.333333}%
\pgfsetfillcolor{currentfill}%
\pgfsetlinewidth{0.602250pt}%
\definecolor{currentstroke}{rgb}{0.333333,0.333333,0.333333}%
\pgfsetstrokecolor{currentstroke}%
\pgfsetdash{}{0pt}%
\pgfsys@defobject{currentmarker}{\pgfqpoint{0.000000in}{-0.027778in}}{\pgfqpoint{0.000000in}{0.000000in}}{%
\pgfpathmoveto{\pgfqpoint{0.000000in}{0.000000in}}%
\pgfpathlineto{\pgfqpoint{0.000000in}{-0.027778in}}%
\pgfusepath{stroke,fill}%
}%
\begin{pgfscope}%
\pgfsys@transformshift{1.614328in}{0.528000in}%
\pgfsys@useobject{currentmarker}{}%
\end{pgfscope}%
\end{pgfscope}%
\begin{pgfscope}%
\pgfsetbuttcap%
\pgfsetroundjoin%
\definecolor{currentfill}{rgb}{0.333333,0.333333,0.333333}%
\pgfsetfillcolor{currentfill}%
\pgfsetlinewidth{0.602250pt}%
\definecolor{currentstroke}{rgb}{0.333333,0.333333,0.333333}%
\pgfsetstrokecolor{currentstroke}%
\pgfsetdash{}{0pt}%
\pgfsys@defobject{currentmarker}{\pgfqpoint{0.000000in}{-0.027778in}}{\pgfqpoint{0.000000in}{0.000000in}}{%
\pgfpathmoveto{\pgfqpoint{0.000000in}{0.000000in}}%
\pgfpathlineto{\pgfqpoint{0.000000in}{-0.027778in}}%
\pgfusepath{stroke,fill}%
}%
\begin{pgfscope}%
\pgfsys@transformshift{1.651343in}{0.528000in}%
\pgfsys@useobject{currentmarker}{}%
\end{pgfscope}%
\end{pgfscope}%
\begin{pgfscope}%
\pgfsetbuttcap%
\pgfsetroundjoin%
\definecolor{currentfill}{rgb}{0.333333,0.333333,0.333333}%
\pgfsetfillcolor{currentfill}%
\pgfsetlinewidth{0.602250pt}%
\definecolor{currentstroke}{rgb}{0.333333,0.333333,0.333333}%
\pgfsetstrokecolor{currentstroke}%
\pgfsetdash{}{0pt}%
\pgfsys@defobject{currentmarker}{\pgfqpoint{0.000000in}{-0.027778in}}{\pgfqpoint{0.000000in}{0.000000in}}{%
\pgfpathmoveto{\pgfqpoint{0.000000in}{0.000000in}}%
\pgfpathlineto{\pgfqpoint{0.000000in}{-0.027778in}}%
\pgfusepath{stroke,fill}%
}%
\begin{pgfscope}%
\pgfsys@transformshift{1.688358in}{0.528000in}%
\pgfsys@useobject{currentmarker}{}%
\end{pgfscope}%
\end{pgfscope}%
\begin{pgfscope}%
\pgfsetbuttcap%
\pgfsetroundjoin%
\definecolor{currentfill}{rgb}{0.333333,0.333333,0.333333}%
\pgfsetfillcolor{currentfill}%
\pgfsetlinewidth{0.602250pt}%
\definecolor{currentstroke}{rgb}{0.333333,0.333333,0.333333}%
\pgfsetstrokecolor{currentstroke}%
\pgfsetdash{}{0pt}%
\pgfsys@defobject{currentmarker}{\pgfqpoint{0.000000in}{-0.027778in}}{\pgfqpoint{0.000000in}{0.000000in}}{%
\pgfpathmoveto{\pgfqpoint{0.000000in}{0.000000in}}%
\pgfpathlineto{\pgfqpoint{0.000000in}{-0.027778in}}%
\pgfusepath{stroke,fill}%
}%
\begin{pgfscope}%
\pgfsys@transformshift{1.725373in}{0.528000in}%
\pgfsys@useobject{currentmarker}{}%
\end{pgfscope}%
\end{pgfscope}%
\begin{pgfscope}%
\pgfsetbuttcap%
\pgfsetroundjoin%
\definecolor{currentfill}{rgb}{0.333333,0.333333,0.333333}%
\pgfsetfillcolor{currentfill}%
\pgfsetlinewidth{0.602250pt}%
\definecolor{currentstroke}{rgb}{0.333333,0.333333,0.333333}%
\pgfsetstrokecolor{currentstroke}%
\pgfsetdash{}{0pt}%
\pgfsys@defobject{currentmarker}{\pgfqpoint{0.000000in}{-0.027778in}}{\pgfqpoint{0.000000in}{0.000000in}}{%
\pgfpathmoveto{\pgfqpoint{0.000000in}{0.000000in}}%
\pgfpathlineto{\pgfqpoint{0.000000in}{-0.027778in}}%
\pgfusepath{stroke,fill}%
}%
\begin{pgfscope}%
\pgfsys@transformshift{1.762388in}{0.528000in}%
\pgfsys@useobject{currentmarker}{}%
\end{pgfscope}%
\end{pgfscope}%
\begin{pgfscope}%
\pgfsetbuttcap%
\pgfsetroundjoin%
\definecolor{currentfill}{rgb}{0.333333,0.333333,0.333333}%
\pgfsetfillcolor{currentfill}%
\pgfsetlinewidth{0.602250pt}%
\definecolor{currentstroke}{rgb}{0.333333,0.333333,0.333333}%
\pgfsetstrokecolor{currentstroke}%
\pgfsetdash{}{0pt}%
\pgfsys@defobject{currentmarker}{\pgfqpoint{0.000000in}{-0.027778in}}{\pgfqpoint{0.000000in}{0.000000in}}{%
\pgfpathmoveto{\pgfqpoint{0.000000in}{0.000000in}}%
\pgfpathlineto{\pgfqpoint{0.000000in}{-0.027778in}}%
\pgfusepath{stroke,fill}%
}%
\begin{pgfscope}%
\pgfsys@transformshift{1.799403in}{0.528000in}%
\pgfsys@useobject{currentmarker}{}%
\end{pgfscope}%
\end{pgfscope}%
\begin{pgfscope}%
\pgfsetbuttcap%
\pgfsetroundjoin%
\definecolor{currentfill}{rgb}{0.333333,0.333333,0.333333}%
\pgfsetfillcolor{currentfill}%
\pgfsetlinewidth{0.602250pt}%
\definecolor{currentstroke}{rgb}{0.333333,0.333333,0.333333}%
\pgfsetstrokecolor{currentstroke}%
\pgfsetdash{}{0pt}%
\pgfsys@defobject{currentmarker}{\pgfqpoint{0.000000in}{-0.027778in}}{\pgfqpoint{0.000000in}{0.000000in}}{%
\pgfpathmoveto{\pgfqpoint{0.000000in}{0.000000in}}%
\pgfpathlineto{\pgfqpoint{0.000000in}{-0.027778in}}%
\pgfusepath{stroke,fill}%
}%
\begin{pgfscope}%
\pgfsys@transformshift{1.836418in}{0.528000in}%
\pgfsys@useobject{currentmarker}{}%
\end{pgfscope}%
\end{pgfscope}%
\begin{pgfscope}%
\pgfsetbuttcap%
\pgfsetroundjoin%
\definecolor{currentfill}{rgb}{0.333333,0.333333,0.333333}%
\pgfsetfillcolor{currentfill}%
\pgfsetlinewidth{0.602250pt}%
\definecolor{currentstroke}{rgb}{0.333333,0.333333,0.333333}%
\pgfsetstrokecolor{currentstroke}%
\pgfsetdash{}{0pt}%
\pgfsys@defobject{currentmarker}{\pgfqpoint{0.000000in}{-0.027778in}}{\pgfqpoint{0.000000in}{0.000000in}}{%
\pgfpathmoveto{\pgfqpoint{0.000000in}{0.000000in}}%
\pgfpathlineto{\pgfqpoint{0.000000in}{-0.027778in}}%
\pgfusepath{stroke,fill}%
}%
\begin{pgfscope}%
\pgfsys@transformshift{1.873433in}{0.528000in}%
\pgfsys@useobject{currentmarker}{}%
\end{pgfscope}%
\end{pgfscope}%
\begin{pgfscope}%
\pgfsetbuttcap%
\pgfsetroundjoin%
\definecolor{currentfill}{rgb}{0.333333,0.333333,0.333333}%
\pgfsetfillcolor{currentfill}%
\pgfsetlinewidth{0.602250pt}%
\definecolor{currentstroke}{rgb}{0.333333,0.333333,0.333333}%
\pgfsetstrokecolor{currentstroke}%
\pgfsetdash{}{0pt}%
\pgfsys@defobject{currentmarker}{\pgfqpoint{0.000000in}{-0.027778in}}{\pgfqpoint{0.000000in}{0.000000in}}{%
\pgfpathmoveto{\pgfqpoint{0.000000in}{0.000000in}}%
\pgfpathlineto{\pgfqpoint{0.000000in}{-0.027778in}}%
\pgfusepath{stroke,fill}%
}%
\begin{pgfscope}%
\pgfsys@transformshift{1.910448in}{0.528000in}%
\pgfsys@useobject{currentmarker}{}%
\end{pgfscope}%
\end{pgfscope}%
\begin{pgfscope}%
\pgfsetbuttcap%
\pgfsetroundjoin%
\definecolor{currentfill}{rgb}{0.333333,0.333333,0.333333}%
\pgfsetfillcolor{currentfill}%
\pgfsetlinewidth{0.602250pt}%
\definecolor{currentstroke}{rgb}{0.333333,0.333333,0.333333}%
\pgfsetstrokecolor{currentstroke}%
\pgfsetdash{}{0pt}%
\pgfsys@defobject{currentmarker}{\pgfqpoint{0.000000in}{-0.027778in}}{\pgfqpoint{0.000000in}{0.000000in}}{%
\pgfpathmoveto{\pgfqpoint{0.000000in}{0.000000in}}%
\pgfpathlineto{\pgfqpoint{0.000000in}{-0.027778in}}%
\pgfusepath{stroke,fill}%
}%
\begin{pgfscope}%
\pgfsys@transformshift{1.947463in}{0.528000in}%
\pgfsys@useobject{currentmarker}{}%
\end{pgfscope}%
\end{pgfscope}%
\begin{pgfscope}%
\pgfsetbuttcap%
\pgfsetroundjoin%
\definecolor{currentfill}{rgb}{0.333333,0.333333,0.333333}%
\pgfsetfillcolor{currentfill}%
\pgfsetlinewidth{0.602250pt}%
\definecolor{currentstroke}{rgb}{0.333333,0.333333,0.333333}%
\pgfsetstrokecolor{currentstroke}%
\pgfsetdash{}{0pt}%
\pgfsys@defobject{currentmarker}{\pgfqpoint{0.000000in}{-0.027778in}}{\pgfqpoint{0.000000in}{0.000000in}}{%
\pgfpathmoveto{\pgfqpoint{0.000000in}{0.000000in}}%
\pgfpathlineto{\pgfqpoint{0.000000in}{-0.027778in}}%
\pgfusepath{stroke,fill}%
}%
\begin{pgfscope}%
\pgfsys@transformshift{1.984478in}{0.528000in}%
\pgfsys@useobject{currentmarker}{}%
\end{pgfscope}%
\end{pgfscope}%
\begin{pgfscope}%
\pgfsetbuttcap%
\pgfsetroundjoin%
\definecolor{currentfill}{rgb}{0.333333,0.333333,0.333333}%
\pgfsetfillcolor{currentfill}%
\pgfsetlinewidth{0.602250pt}%
\definecolor{currentstroke}{rgb}{0.333333,0.333333,0.333333}%
\pgfsetstrokecolor{currentstroke}%
\pgfsetdash{}{0pt}%
\pgfsys@defobject{currentmarker}{\pgfqpoint{0.000000in}{-0.027778in}}{\pgfqpoint{0.000000in}{0.000000in}}{%
\pgfpathmoveto{\pgfqpoint{0.000000in}{0.000000in}}%
\pgfpathlineto{\pgfqpoint{0.000000in}{-0.027778in}}%
\pgfusepath{stroke,fill}%
}%
\begin{pgfscope}%
\pgfsys@transformshift{2.021493in}{0.528000in}%
\pgfsys@useobject{currentmarker}{}%
\end{pgfscope}%
\end{pgfscope}%
\begin{pgfscope}%
\pgfsetbuttcap%
\pgfsetroundjoin%
\definecolor{currentfill}{rgb}{0.333333,0.333333,0.333333}%
\pgfsetfillcolor{currentfill}%
\pgfsetlinewidth{0.602250pt}%
\definecolor{currentstroke}{rgb}{0.333333,0.333333,0.333333}%
\pgfsetstrokecolor{currentstroke}%
\pgfsetdash{}{0pt}%
\pgfsys@defobject{currentmarker}{\pgfqpoint{0.000000in}{-0.027778in}}{\pgfqpoint{0.000000in}{0.000000in}}{%
\pgfpathmoveto{\pgfqpoint{0.000000in}{0.000000in}}%
\pgfpathlineto{\pgfqpoint{0.000000in}{-0.027778in}}%
\pgfusepath{stroke,fill}%
}%
\begin{pgfscope}%
\pgfsys@transformshift{2.058507in}{0.528000in}%
\pgfsys@useobject{currentmarker}{}%
\end{pgfscope}%
\end{pgfscope}%
\begin{pgfscope}%
\pgfsetbuttcap%
\pgfsetroundjoin%
\definecolor{currentfill}{rgb}{0.333333,0.333333,0.333333}%
\pgfsetfillcolor{currentfill}%
\pgfsetlinewidth{0.602250pt}%
\definecolor{currentstroke}{rgb}{0.333333,0.333333,0.333333}%
\pgfsetstrokecolor{currentstroke}%
\pgfsetdash{}{0pt}%
\pgfsys@defobject{currentmarker}{\pgfqpoint{0.000000in}{-0.027778in}}{\pgfqpoint{0.000000in}{0.000000in}}{%
\pgfpathmoveto{\pgfqpoint{0.000000in}{0.000000in}}%
\pgfpathlineto{\pgfqpoint{0.000000in}{-0.027778in}}%
\pgfusepath{stroke,fill}%
}%
\begin{pgfscope}%
\pgfsys@transformshift{2.095522in}{0.528000in}%
\pgfsys@useobject{currentmarker}{}%
\end{pgfscope}%
\end{pgfscope}%
\begin{pgfscope}%
\pgfsetbuttcap%
\pgfsetroundjoin%
\definecolor{currentfill}{rgb}{0.333333,0.333333,0.333333}%
\pgfsetfillcolor{currentfill}%
\pgfsetlinewidth{0.602250pt}%
\definecolor{currentstroke}{rgb}{0.333333,0.333333,0.333333}%
\pgfsetstrokecolor{currentstroke}%
\pgfsetdash{}{0pt}%
\pgfsys@defobject{currentmarker}{\pgfqpoint{0.000000in}{-0.027778in}}{\pgfqpoint{0.000000in}{0.000000in}}{%
\pgfpathmoveto{\pgfqpoint{0.000000in}{0.000000in}}%
\pgfpathlineto{\pgfqpoint{0.000000in}{-0.027778in}}%
\pgfusepath{stroke,fill}%
}%
\begin{pgfscope}%
\pgfsys@transformshift{2.132537in}{0.528000in}%
\pgfsys@useobject{currentmarker}{}%
\end{pgfscope}%
\end{pgfscope}%
\begin{pgfscope}%
\pgfsetbuttcap%
\pgfsetroundjoin%
\definecolor{currentfill}{rgb}{0.333333,0.333333,0.333333}%
\pgfsetfillcolor{currentfill}%
\pgfsetlinewidth{0.602250pt}%
\definecolor{currentstroke}{rgb}{0.333333,0.333333,0.333333}%
\pgfsetstrokecolor{currentstroke}%
\pgfsetdash{}{0pt}%
\pgfsys@defobject{currentmarker}{\pgfqpoint{0.000000in}{-0.027778in}}{\pgfqpoint{0.000000in}{0.000000in}}{%
\pgfpathmoveto{\pgfqpoint{0.000000in}{0.000000in}}%
\pgfpathlineto{\pgfqpoint{0.000000in}{-0.027778in}}%
\pgfusepath{stroke,fill}%
}%
\begin{pgfscope}%
\pgfsys@transformshift{2.169552in}{0.528000in}%
\pgfsys@useobject{currentmarker}{}%
\end{pgfscope}%
\end{pgfscope}%
\begin{pgfscope}%
\pgfsetbuttcap%
\pgfsetroundjoin%
\definecolor{currentfill}{rgb}{0.333333,0.333333,0.333333}%
\pgfsetfillcolor{currentfill}%
\pgfsetlinewidth{0.602250pt}%
\definecolor{currentstroke}{rgb}{0.333333,0.333333,0.333333}%
\pgfsetstrokecolor{currentstroke}%
\pgfsetdash{}{0pt}%
\pgfsys@defobject{currentmarker}{\pgfqpoint{0.000000in}{-0.027778in}}{\pgfqpoint{0.000000in}{0.000000in}}{%
\pgfpathmoveto{\pgfqpoint{0.000000in}{0.000000in}}%
\pgfpathlineto{\pgfqpoint{0.000000in}{-0.027778in}}%
\pgfusepath{stroke,fill}%
}%
\begin{pgfscope}%
\pgfsys@transformshift{2.206567in}{0.528000in}%
\pgfsys@useobject{currentmarker}{}%
\end{pgfscope}%
\end{pgfscope}%
\begin{pgfscope}%
\pgfsetbuttcap%
\pgfsetroundjoin%
\definecolor{currentfill}{rgb}{0.333333,0.333333,0.333333}%
\pgfsetfillcolor{currentfill}%
\pgfsetlinewidth{0.602250pt}%
\definecolor{currentstroke}{rgb}{0.333333,0.333333,0.333333}%
\pgfsetstrokecolor{currentstroke}%
\pgfsetdash{}{0pt}%
\pgfsys@defobject{currentmarker}{\pgfqpoint{0.000000in}{-0.027778in}}{\pgfqpoint{0.000000in}{0.000000in}}{%
\pgfpathmoveto{\pgfqpoint{0.000000in}{0.000000in}}%
\pgfpathlineto{\pgfqpoint{0.000000in}{-0.027778in}}%
\pgfusepath{stroke,fill}%
}%
\begin{pgfscope}%
\pgfsys@transformshift{2.243582in}{0.528000in}%
\pgfsys@useobject{currentmarker}{}%
\end{pgfscope}%
\end{pgfscope}%
\begin{pgfscope}%
\pgfsetbuttcap%
\pgfsetroundjoin%
\definecolor{currentfill}{rgb}{0.333333,0.333333,0.333333}%
\pgfsetfillcolor{currentfill}%
\pgfsetlinewidth{0.602250pt}%
\definecolor{currentstroke}{rgb}{0.333333,0.333333,0.333333}%
\pgfsetstrokecolor{currentstroke}%
\pgfsetdash{}{0pt}%
\pgfsys@defobject{currentmarker}{\pgfqpoint{0.000000in}{-0.027778in}}{\pgfqpoint{0.000000in}{0.000000in}}{%
\pgfpathmoveto{\pgfqpoint{0.000000in}{0.000000in}}%
\pgfpathlineto{\pgfqpoint{0.000000in}{-0.027778in}}%
\pgfusepath{stroke,fill}%
}%
\begin{pgfscope}%
\pgfsys@transformshift{2.280597in}{0.528000in}%
\pgfsys@useobject{currentmarker}{}%
\end{pgfscope}%
\end{pgfscope}%
\begin{pgfscope}%
\pgfsetbuttcap%
\pgfsetroundjoin%
\definecolor{currentfill}{rgb}{0.333333,0.333333,0.333333}%
\pgfsetfillcolor{currentfill}%
\pgfsetlinewidth{0.602250pt}%
\definecolor{currentstroke}{rgb}{0.333333,0.333333,0.333333}%
\pgfsetstrokecolor{currentstroke}%
\pgfsetdash{}{0pt}%
\pgfsys@defobject{currentmarker}{\pgfqpoint{0.000000in}{-0.027778in}}{\pgfqpoint{0.000000in}{0.000000in}}{%
\pgfpathmoveto{\pgfqpoint{0.000000in}{0.000000in}}%
\pgfpathlineto{\pgfqpoint{0.000000in}{-0.027778in}}%
\pgfusepath{stroke,fill}%
}%
\begin{pgfscope}%
\pgfsys@transformshift{2.317612in}{0.528000in}%
\pgfsys@useobject{currentmarker}{}%
\end{pgfscope}%
\end{pgfscope}%
\begin{pgfscope}%
\pgfsetbuttcap%
\pgfsetroundjoin%
\definecolor{currentfill}{rgb}{0.333333,0.333333,0.333333}%
\pgfsetfillcolor{currentfill}%
\pgfsetlinewidth{0.602250pt}%
\definecolor{currentstroke}{rgb}{0.333333,0.333333,0.333333}%
\pgfsetstrokecolor{currentstroke}%
\pgfsetdash{}{0pt}%
\pgfsys@defobject{currentmarker}{\pgfqpoint{0.000000in}{-0.027778in}}{\pgfqpoint{0.000000in}{0.000000in}}{%
\pgfpathmoveto{\pgfqpoint{0.000000in}{0.000000in}}%
\pgfpathlineto{\pgfqpoint{0.000000in}{-0.027778in}}%
\pgfusepath{stroke,fill}%
}%
\begin{pgfscope}%
\pgfsys@transformshift{2.354627in}{0.528000in}%
\pgfsys@useobject{currentmarker}{}%
\end{pgfscope}%
\end{pgfscope}%
\begin{pgfscope}%
\pgfsetbuttcap%
\pgfsetroundjoin%
\definecolor{currentfill}{rgb}{0.333333,0.333333,0.333333}%
\pgfsetfillcolor{currentfill}%
\pgfsetlinewidth{0.602250pt}%
\definecolor{currentstroke}{rgb}{0.333333,0.333333,0.333333}%
\pgfsetstrokecolor{currentstroke}%
\pgfsetdash{}{0pt}%
\pgfsys@defobject{currentmarker}{\pgfqpoint{0.000000in}{-0.027778in}}{\pgfqpoint{0.000000in}{0.000000in}}{%
\pgfpathmoveto{\pgfqpoint{0.000000in}{0.000000in}}%
\pgfpathlineto{\pgfqpoint{0.000000in}{-0.027778in}}%
\pgfusepath{stroke,fill}%
}%
\begin{pgfscope}%
\pgfsys@transformshift{2.391642in}{0.528000in}%
\pgfsys@useobject{currentmarker}{}%
\end{pgfscope}%
\end{pgfscope}%
\begin{pgfscope}%
\pgfsetbuttcap%
\pgfsetroundjoin%
\definecolor{currentfill}{rgb}{0.333333,0.333333,0.333333}%
\pgfsetfillcolor{currentfill}%
\pgfsetlinewidth{0.602250pt}%
\definecolor{currentstroke}{rgb}{0.333333,0.333333,0.333333}%
\pgfsetstrokecolor{currentstroke}%
\pgfsetdash{}{0pt}%
\pgfsys@defobject{currentmarker}{\pgfqpoint{0.000000in}{-0.027778in}}{\pgfqpoint{0.000000in}{0.000000in}}{%
\pgfpathmoveto{\pgfqpoint{0.000000in}{0.000000in}}%
\pgfpathlineto{\pgfqpoint{0.000000in}{-0.027778in}}%
\pgfusepath{stroke,fill}%
}%
\begin{pgfscope}%
\pgfsys@transformshift{2.428657in}{0.528000in}%
\pgfsys@useobject{currentmarker}{}%
\end{pgfscope}%
\end{pgfscope}%
\begin{pgfscope}%
\pgfsetbuttcap%
\pgfsetroundjoin%
\definecolor{currentfill}{rgb}{0.333333,0.333333,0.333333}%
\pgfsetfillcolor{currentfill}%
\pgfsetlinewidth{0.602250pt}%
\definecolor{currentstroke}{rgb}{0.333333,0.333333,0.333333}%
\pgfsetstrokecolor{currentstroke}%
\pgfsetdash{}{0pt}%
\pgfsys@defobject{currentmarker}{\pgfqpoint{0.000000in}{-0.027778in}}{\pgfqpoint{0.000000in}{0.000000in}}{%
\pgfpathmoveto{\pgfqpoint{0.000000in}{0.000000in}}%
\pgfpathlineto{\pgfqpoint{0.000000in}{-0.027778in}}%
\pgfusepath{stroke,fill}%
}%
\begin{pgfscope}%
\pgfsys@transformshift{2.465672in}{0.528000in}%
\pgfsys@useobject{currentmarker}{}%
\end{pgfscope}%
\end{pgfscope}%
\begin{pgfscope}%
\pgfsetbuttcap%
\pgfsetroundjoin%
\definecolor{currentfill}{rgb}{0.333333,0.333333,0.333333}%
\pgfsetfillcolor{currentfill}%
\pgfsetlinewidth{0.602250pt}%
\definecolor{currentstroke}{rgb}{0.333333,0.333333,0.333333}%
\pgfsetstrokecolor{currentstroke}%
\pgfsetdash{}{0pt}%
\pgfsys@defobject{currentmarker}{\pgfqpoint{0.000000in}{-0.027778in}}{\pgfqpoint{0.000000in}{0.000000in}}{%
\pgfpathmoveto{\pgfqpoint{0.000000in}{0.000000in}}%
\pgfpathlineto{\pgfqpoint{0.000000in}{-0.027778in}}%
\pgfusepath{stroke,fill}%
}%
\begin{pgfscope}%
\pgfsys@transformshift{2.502687in}{0.528000in}%
\pgfsys@useobject{currentmarker}{}%
\end{pgfscope}%
\end{pgfscope}%
\begin{pgfscope}%
\pgfsetbuttcap%
\pgfsetroundjoin%
\definecolor{currentfill}{rgb}{0.333333,0.333333,0.333333}%
\pgfsetfillcolor{currentfill}%
\pgfsetlinewidth{0.602250pt}%
\definecolor{currentstroke}{rgb}{0.333333,0.333333,0.333333}%
\pgfsetstrokecolor{currentstroke}%
\pgfsetdash{}{0pt}%
\pgfsys@defobject{currentmarker}{\pgfqpoint{0.000000in}{-0.027778in}}{\pgfqpoint{0.000000in}{0.000000in}}{%
\pgfpathmoveto{\pgfqpoint{0.000000in}{0.000000in}}%
\pgfpathlineto{\pgfqpoint{0.000000in}{-0.027778in}}%
\pgfusepath{stroke,fill}%
}%
\begin{pgfscope}%
\pgfsys@transformshift{2.539701in}{0.528000in}%
\pgfsys@useobject{currentmarker}{}%
\end{pgfscope}%
\end{pgfscope}%
\begin{pgfscope}%
\pgfsetbuttcap%
\pgfsetroundjoin%
\definecolor{currentfill}{rgb}{0.333333,0.333333,0.333333}%
\pgfsetfillcolor{currentfill}%
\pgfsetlinewidth{0.602250pt}%
\definecolor{currentstroke}{rgb}{0.333333,0.333333,0.333333}%
\pgfsetstrokecolor{currentstroke}%
\pgfsetdash{}{0pt}%
\pgfsys@defobject{currentmarker}{\pgfqpoint{0.000000in}{-0.027778in}}{\pgfqpoint{0.000000in}{0.000000in}}{%
\pgfpathmoveto{\pgfqpoint{0.000000in}{0.000000in}}%
\pgfpathlineto{\pgfqpoint{0.000000in}{-0.027778in}}%
\pgfusepath{stroke,fill}%
}%
\begin{pgfscope}%
\pgfsys@transformshift{2.576716in}{0.528000in}%
\pgfsys@useobject{currentmarker}{}%
\end{pgfscope}%
\end{pgfscope}%
\begin{pgfscope}%
\pgfsetbuttcap%
\pgfsetroundjoin%
\definecolor{currentfill}{rgb}{0.333333,0.333333,0.333333}%
\pgfsetfillcolor{currentfill}%
\pgfsetlinewidth{0.602250pt}%
\definecolor{currentstroke}{rgb}{0.333333,0.333333,0.333333}%
\pgfsetstrokecolor{currentstroke}%
\pgfsetdash{}{0pt}%
\pgfsys@defobject{currentmarker}{\pgfqpoint{0.000000in}{-0.027778in}}{\pgfqpoint{0.000000in}{0.000000in}}{%
\pgfpathmoveto{\pgfqpoint{0.000000in}{0.000000in}}%
\pgfpathlineto{\pgfqpoint{0.000000in}{-0.027778in}}%
\pgfusepath{stroke,fill}%
}%
\begin{pgfscope}%
\pgfsys@transformshift{2.613731in}{0.528000in}%
\pgfsys@useobject{currentmarker}{}%
\end{pgfscope}%
\end{pgfscope}%
\begin{pgfscope}%
\pgfsetbuttcap%
\pgfsetroundjoin%
\definecolor{currentfill}{rgb}{0.333333,0.333333,0.333333}%
\pgfsetfillcolor{currentfill}%
\pgfsetlinewidth{0.602250pt}%
\definecolor{currentstroke}{rgb}{0.333333,0.333333,0.333333}%
\pgfsetstrokecolor{currentstroke}%
\pgfsetdash{}{0pt}%
\pgfsys@defobject{currentmarker}{\pgfqpoint{0.000000in}{-0.027778in}}{\pgfqpoint{0.000000in}{0.000000in}}{%
\pgfpathmoveto{\pgfqpoint{0.000000in}{0.000000in}}%
\pgfpathlineto{\pgfqpoint{0.000000in}{-0.027778in}}%
\pgfusepath{stroke,fill}%
}%
\begin{pgfscope}%
\pgfsys@transformshift{2.650746in}{0.528000in}%
\pgfsys@useobject{currentmarker}{}%
\end{pgfscope}%
\end{pgfscope}%
\begin{pgfscope}%
\pgfsetbuttcap%
\pgfsetroundjoin%
\definecolor{currentfill}{rgb}{0.333333,0.333333,0.333333}%
\pgfsetfillcolor{currentfill}%
\pgfsetlinewidth{0.602250pt}%
\definecolor{currentstroke}{rgb}{0.333333,0.333333,0.333333}%
\pgfsetstrokecolor{currentstroke}%
\pgfsetdash{}{0pt}%
\pgfsys@defobject{currentmarker}{\pgfqpoint{0.000000in}{-0.027778in}}{\pgfqpoint{0.000000in}{0.000000in}}{%
\pgfpathmoveto{\pgfqpoint{0.000000in}{0.000000in}}%
\pgfpathlineto{\pgfqpoint{0.000000in}{-0.027778in}}%
\pgfusepath{stroke,fill}%
}%
\begin{pgfscope}%
\pgfsys@transformshift{2.687761in}{0.528000in}%
\pgfsys@useobject{currentmarker}{}%
\end{pgfscope}%
\end{pgfscope}%
\begin{pgfscope}%
\pgfsetbuttcap%
\pgfsetroundjoin%
\definecolor{currentfill}{rgb}{0.333333,0.333333,0.333333}%
\pgfsetfillcolor{currentfill}%
\pgfsetlinewidth{0.602250pt}%
\definecolor{currentstroke}{rgb}{0.333333,0.333333,0.333333}%
\pgfsetstrokecolor{currentstroke}%
\pgfsetdash{}{0pt}%
\pgfsys@defobject{currentmarker}{\pgfqpoint{0.000000in}{-0.027778in}}{\pgfqpoint{0.000000in}{0.000000in}}{%
\pgfpathmoveto{\pgfqpoint{0.000000in}{0.000000in}}%
\pgfpathlineto{\pgfqpoint{0.000000in}{-0.027778in}}%
\pgfusepath{stroke,fill}%
}%
\begin{pgfscope}%
\pgfsys@transformshift{2.724776in}{0.528000in}%
\pgfsys@useobject{currentmarker}{}%
\end{pgfscope}%
\end{pgfscope}%
\begin{pgfscope}%
\pgfsetbuttcap%
\pgfsetroundjoin%
\definecolor{currentfill}{rgb}{0.333333,0.333333,0.333333}%
\pgfsetfillcolor{currentfill}%
\pgfsetlinewidth{0.602250pt}%
\definecolor{currentstroke}{rgb}{0.333333,0.333333,0.333333}%
\pgfsetstrokecolor{currentstroke}%
\pgfsetdash{}{0pt}%
\pgfsys@defobject{currentmarker}{\pgfqpoint{0.000000in}{-0.027778in}}{\pgfqpoint{0.000000in}{0.000000in}}{%
\pgfpathmoveto{\pgfqpoint{0.000000in}{0.000000in}}%
\pgfpathlineto{\pgfqpoint{0.000000in}{-0.027778in}}%
\pgfusepath{stroke,fill}%
}%
\begin{pgfscope}%
\pgfsys@transformshift{2.761791in}{0.528000in}%
\pgfsys@useobject{currentmarker}{}%
\end{pgfscope}%
\end{pgfscope}%
\begin{pgfscope}%
\pgfsetbuttcap%
\pgfsetroundjoin%
\definecolor{currentfill}{rgb}{0.333333,0.333333,0.333333}%
\pgfsetfillcolor{currentfill}%
\pgfsetlinewidth{0.602250pt}%
\definecolor{currentstroke}{rgb}{0.333333,0.333333,0.333333}%
\pgfsetstrokecolor{currentstroke}%
\pgfsetdash{}{0pt}%
\pgfsys@defobject{currentmarker}{\pgfqpoint{0.000000in}{-0.027778in}}{\pgfqpoint{0.000000in}{0.000000in}}{%
\pgfpathmoveto{\pgfqpoint{0.000000in}{0.000000in}}%
\pgfpathlineto{\pgfqpoint{0.000000in}{-0.027778in}}%
\pgfusepath{stroke,fill}%
}%
\begin{pgfscope}%
\pgfsys@transformshift{2.798806in}{0.528000in}%
\pgfsys@useobject{currentmarker}{}%
\end{pgfscope}%
\end{pgfscope}%
\begin{pgfscope}%
\pgfsetbuttcap%
\pgfsetroundjoin%
\definecolor{currentfill}{rgb}{0.333333,0.333333,0.333333}%
\pgfsetfillcolor{currentfill}%
\pgfsetlinewidth{0.602250pt}%
\definecolor{currentstroke}{rgb}{0.333333,0.333333,0.333333}%
\pgfsetstrokecolor{currentstroke}%
\pgfsetdash{}{0pt}%
\pgfsys@defobject{currentmarker}{\pgfqpoint{0.000000in}{-0.027778in}}{\pgfqpoint{0.000000in}{0.000000in}}{%
\pgfpathmoveto{\pgfqpoint{0.000000in}{0.000000in}}%
\pgfpathlineto{\pgfqpoint{0.000000in}{-0.027778in}}%
\pgfusepath{stroke,fill}%
}%
\begin{pgfscope}%
\pgfsys@transformshift{2.835821in}{0.528000in}%
\pgfsys@useobject{currentmarker}{}%
\end{pgfscope}%
\end{pgfscope}%
\begin{pgfscope}%
\pgfsetbuttcap%
\pgfsetroundjoin%
\definecolor{currentfill}{rgb}{0.333333,0.333333,0.333333}%
\pgfsetfillcolor{currentfill}%
\pgfsetlinewidth{0.602250pt}%
\definecolor{currentstroke}{rgb}{0.333333,0.333333,0.333333}%
\pgfsetstrokecolor{currentstroke}%
\pgfsetdash{}{0pt}%
\pgfsys@defobject{currentmarker}{\pgfqpoint{0.000000in}{-0.027778in}}{\pgfqpoint{0.000000in}{0.000000in}}{%
\pgfpathmoveto{\pgfqpoint{0.000000in}{0.000000in}}%
\pgfpathlineto{\pgfqpoint{0.000000in}{-0.027778in}}%
\pgfusepath{stroke,fill}%
}%
\begin{pgfscope}%
\pgfsys@transformshift{2.872836in}{0.528000in}%
\pgfsys@useobject{currentmarker}{}%
\end{pgfscope}%
\end{pgfscope}%
\begin{pgfscope}%
\pgfsetbuttcap%
\pgfsetroundjoin%
\definecolor{currentfill}{rgb}{0.333333,0.333333,0.333333}%
\pgfsetfillcolor{currentfill}%
\pgfsetlinewidth{0.602250pt}%
\definecolor{currentstroke}{rgb}{0.333333,0.333333,0.333333}%
\pgfsetstrokecolor{currentstroke}%
\pgfsetdash{}{0pt}%
\pgfsys@defobject{currentmarker}{\pgfqpoint{0.000000in}{-0.027778in}}{\pgfqpoint{0.000000in}{0.000000in}}{%
\pgfpathmoveto{\pgfqpoint{0.000000in}{0.000000in}}%
\pgfpathlineto{\pgfqpoint{0.000000in}{-0.027778in}}%
\pgfusepath{stroke,fill}%
}%
\begin{pgfscope}%
\pgfsys@transformshift{2.909851in}{0.528000in}%
\pgfsys@useobject{currentmarker}{}%
\end{pgfscope}%
\end{pgfscope}%
\begin{pgfscope}%
\pgfsetbuttcap%
\pgfsetroundjoin%
\definecolor{currentfill}{rgb}{0.333333,0.333333,0.333333}%
\pgfsetfillcolor{currentfill}%
\pgfsetlinewidth{0.602250pt}%
\definecolor{currentstroke}{rgb}{0.333333,0.333333,0.333333}%
\pgfsetstrokecolor{currentstroke}%
\pgfsetdash{}{0pt}%
\pgfsys@defobject{currentmarker}{\pgfqpoint{0.000000in}{-0.027778in}}{\pgfqpoint{0.000000in}{0.000000in}}{%
\pgfpathmoveto{\pgfqpoint{0.000000in}{0.000000in}}%
\pgfpathlineto{\pgfqpoint{0.000000in}{-0.027778in}}%
\pgfusepath{stroke,fill}%
}%
\begin{pgfscope}%
\pgfsys@transformshift{2.946866in}{0.528000in}%
\pgfsys@useobject{currentmarker}{}%
\end{pgfscope}%
\end{pgfscope}%
\begin{pgfscope}%
\pgfsetbuttcap%
\pgfsetroundjoin%
\definecolor{currentfill}{rgb}{0.333333,0.333333,0.333333}%
\pgfsetfillcolor{currentfill}%
\pgfsetlinewidth{0.602250pt}%
\definecolor{currentstroke}{rgb}{0.333333,0.333333,0.333333}%
\pgfsetstrokecolor{currentstroke}%
\pgfsetdash{}{0pt}%
\pgfsys@defobject{currentmarker}{\pgfqpoint{0.000000in}{-0.027778in}}{\pgfqpoint{0.000000in}{0.000000in}}{%
\pgfpathmoveto{\pgfqpoint{0.000000in}{0.000000in}}%
\pgfpathlineto{\pgfqpoint{0.000000in}{-0.027778in}}%
\pgfusepath{stroke,fill}%
}%
\begin{pgfscope}%
\pgfsys@transformshift{2.983881in}{0.528000in}%
\pgfsys@useobject{currentmarker}{}%
\end{pgfscope}%
\end{pgfscope}%
\begin{pgfscope}%
\pgfsetbuttcap%
\pgfsetroundjoin%
\definecolor{currentfill}{rgb}{0.333333,0.333333,0.333333}%
\pgfsetfillcolor{currentfill}%
\pgfsetlinewidth{0.602250pt}%
\definecolor{currentstroke}{rgb}{0.333333,0.333333,0.333333}%
\pgfsetstrokecolor{currentstroke}%
\pgfsetdash{}{0pt}%
\pgfsys@defobject{currentmarker}{\pgfqpoint{0.000000in}{-0.027778in}}{\pgfqpoint{0.000000in}{0.000000in}}{%
\pgfpathmoveto{\pgfqpoint{0.000000in}{0.000000in}}%
\pgfpathlineto{\pgfqpoint{0.000000in}{-0.027778in}}%
\pgfusepath{stroke,fill}%
}%
\begin{pgfscope}%
\pgfsys@transformshift{3.020896in}{0.528000in}%
\pgfsys@useobject{currentmarker}{}%
\end{pgfscope}%
\end{pgfscope}%
\begin{pgfscope}%
\pgfsetbuttcap%
\pgfsetroundjoin%
\definecolor{currentfill}{rgb}{0.333333,0.333333,0.333333}%
\pgfsetfillcolor{currentfill}%
\pgfsetlinewidth{0.602250pt}%
\definecolor{currentstroke}{rgb}{0.333333,0.333333,0.333333}%
\pgfsetstrokecolor{currentstroke}%
\pgfsetdash{}{0pt}%
\pgfsys@defobject{currentmarker}{\pgfqpoint{0.000000in}{-0.027778in}}{\pgfqpoint{0.000000in}{0.000000in}}{%
\pgfpathmoveto{\pgfqpoint{0.000000in}{0.000000in}}%
\pgfpathlineto{\pgfqpoint{0.000000in}{-0.027778in}}%
\pgfusepath{stroke,fill}%
}%
\begin{pgfscope}%
\pgfsys@transformshift{3.057910in}{0.528000in}%
\pgfsys@useobject{currentmarker}{}%
\end{pgfscope}%
\end{pgfscope}%
\begin{pgfscope}%
\pgfsetbuttcap%
\pgfsetroundjoin%
\definecolor{currentfill}{rgb}{0.333333,0.333333,0.333333}%
\pgfsetfillcolor{currentfill}%
\pgfsetlinewidth{0.602250pt}%
\definecolor{currentstroke}{rgb}{0.333333,0.333333,0.333333}%
\pgfsetstrokecolor{currentstroke}%
\pgfsetdash{}{0pt}%
\pgfsys@defobject{currentmarker}{\pgfqpoint{0.000000in}{-0.027778in}}{\pgfqpoint{0.000000in}{0.000000in}}{%
\pgfpathmoveto{\pgfqpoint{0.000000in}{0.000000in}}%
\pgfpathlineto{\pgfqpoint{0.000000in}{-0.027778in}}%
\pgfusepath{stroke,fill}%
}%
\begin{pgfscope}%
\pgfsys@transformshift{3.094925in}{0.528000in}%
\pgfsys@useobject{currentmarker}{}%
\end{pgfscope}%
\end{pgfscope}%
\begin{pgfscope}%
\pgfsetbuttcap%
\pgfsetroundjoin%
\definecolor{currentfill}{rgb}{0.333333,0.333333,0.333333}%
\pgfsetfillcolor{currentfill}%
\pgfsetlinewidth{0.602250pt}%
\definecolor{currentstroke}{rgb}{0.333333,0.333333,0.333333}%
\pgfsetstrokecolor{currentstroke}%
\pgfsetdash{}{0pt}%
\pgfsys@defobject{currentmarker}{\pgfqpoint{0.000000in}{-0.027778in}}{\pgfqpoint{0.000000in}{0.000000in}}{%
\pgfpathmoveto{\pgfqpoint{0.000000in}{0.000000in}}%
\pgfpathlineto{\pgfqpoint{0.000000in}{-0.027778in}}%
\pgfusepath{stroke,fill}%
}%
\begin{pgfscope}%
\pgfsys@transformshift{3.131940in}{0.528000in}%
\pgfsys@useobject{currentmarker}{}%
\end{pgfscope}%
\end{pgfscope}%
\begin{pgfscope}%
\pgfsetbuttcap%
\pgfsetroundjoin%
\definecolor{currentfill}{rgb}{0.333333,0.333333,0.333333}%
\pgfsetfillcolor{currentfill}%
\pgfsetlinewidth{0.602250pt}%
\definecolor{currentstroke}{rgb}{0.333333,0.333333,0.333333}%
\pgfsetstrokecolor{currentstroke}%
\pgfsetdash{}{0pt}%
\pgfsys@defobject{currentmarker}{\pgfqpoint{0.000000in}{-0.027778in}}{\pgfqpoint{0.000000in}{0.000000in}}{%
\pgfpathmoveto{\pgfqpoint{0.000000in}{0.000000in}}%
\pgfpathlineto{\pgfqpoint{0.000000in}{-0.027778in}}%
\pgfusepath{stroke,fill}%
}%
\begin{pgfscope}%
\pgfsys@transformshift{3.168955in}{0.528000in}%
\pgfsys@useobject{currentmarker}{}%
\end{pgfscope}%
\end{pgfscope}%
\begin{pgfscope}%
\pgfsetbuttcap%
\pgfsetroundjoin%
\definecolor{currentfill}{rgb}{0.333333,0.333333,0.333333}%
\pgfsetfillcolor{currentfill}%
\pgfsetlinewidth{0.602250pt}%
\definecolor{currentstroke}{rgb}{0.333333,0.333333,0.333333}%
\pgfsetstrokecolor{currentstroke}%
\pgfsetdash{}{0pt}%
\pgfsys@defobject{currentmarker}{\pgfqpoint{0.000000in}{-0.027778in}}{\pgfqpoint{0.000000in}{0.000000in}}{%
\pgfpathmoveto{\pgfqpoint{0.000000in}{0.000000in}}%
\pgfpathlineto{\pgfqpoint{0.000000in}{-0.027778in}}%
\pgfusepath{stroke,fill}%
}%
\begin{pgfscope}%
\pgfsys@transformshift{3.205970in}{0.528000in}%
\pgfsys@useobject{currentmarker}{}%
\end{pgfscope}%
\end{pgfscope}%
\begin{pgfscope}%
\pgfsetbuttcap%
\pgfsetroundjoin%
\definecolor{currentfill}{rgb}{0.333333,0.333333,0.333333}%
\pgfsetfillcolor{currentfill}%
\pgfsetlinewidth{0.602250pt}%
\definecolor{currentstroke}{rgb}{0.333333,0.333333,0.333333}%
\pgfsetstrokecolor{currentstroke}%
\pgfsetdash{}{0pt}%
\pgfsys@defobject{currentmarker}{\pgfqpoint{0.000000in}{-0.027778in}}{\pgfqpoint{0.000000in}{0.000000in}}{%
\pgfpathmoveto{\pgfqpoint{0.000000in}{0.000000in}}%
\pgfpathlineto{\pgfqpoint{0.000000in}{-0.027778in}}%
\pgfusepath{stroke,fill}%
}%
\begin{pgfscope}%
\pgfsys@transformshift{3.242985in}{0.528000in}%
\pgfsys@useobject{currentmarker}{}%
\end{pgfscope}%
\end{pgfscope}%
\begin{pgfscope}%
\pgfsetbuttcap%
\pgfsetroundjoin%
\definecolor{currentfill}{rgb}{0.333333,0.333333,0.333333}%
\pgfsetfillcolor{currentfill}%
\pgfsetlinewidth{0.602250pt}%
\definecolor{currentstroke}{rgb}{0.333333,0.333333,0.333333}%
\pgfsetstrokecolor{currentstroke}%
\pgfsetdash{}{0pt}%
\pgfsys@defobject{currentmarker}{\pgfqpoint{0.000000in}{-0.027778in}}{\pgfqpoint{0.000000in}{0.000000in}}{%
\pgfpathmoveto{\pgfqpoint{0.000000in}{0.000000in}}%
\pgfpathlineto{\pgfqpoint{0.000000in}{-0.027778in}}%
\pgfusepath{stroke,fill}%
}%
\begin{pgfscope}%
\pgfsys@transformshift{3.280000in}{0.528000in}%
\pgfsys@useobject{currentmarker}{}%
\end{pgfscope}%
\end{pgfscope}%
\begin{pgfscope}%
\pgfsetbuttcap%
\pgfsetroundjoin%
\definecolor{currentfill}{rgb}{0.333333,0.333333,0.333333}%
\pgfsetfillcolor{currentfill}%
\pgfsetlinewidth{0.602250pt}%
\definecolor{currentstroke}{rgb}{0.333333,0.333333,0.333333}%
\pgfsetstrokecolor{currentstroke}%
\pgfsetdash{}{0pt}%
\pgfsys@defobject{currentmarker}{\pgfqpoint{0.000000in}{-0.027778in}}{\pgfqpoint{0.000000in}{0.000000in}}{%
\pgfpathmoveto{\pgfqpoint{0.000000in}{0.000000in}}%
\pgfpathlineto{\pgfqpoint{0.000000in}{-0.027778in}}%
\pgfusepath{stroke,fill}%
}%
\begin{pgfscope}%
\pgfsys@transformshift{3.317015in}{0.528000in}%
\pgfsys@useobject{currentmarker}{}%
\end{pgfscope}%
\end{pgfscope}%
\begin{pgfscope}%
\pgfsetbuttcap%
\pgfsetroundjoin%
\definecolor{currentfill}{rgb}{0.333333,0.333333,0.333333}%
\pgfsetfillcolor{currentfill}%
\pgfsetlinewidth{0.602250pt}%
\definecolor{currentstroke}{rgb}{0.333333,0.333333,0.333333}%
\pgfsetstrokecolor{currentstroke}%
\pgfsetdash{}{0pt}%
\pgfsys@defobject{currentmarker}{\pgfqpoint{0.000000in}{-0.027778in}}{\pgfqpoint{0.000000in}{0.000000in}}{%
\pgfpathmoveto{\pgfqpoint{0.000000in}{0.000000in}}%
\pgfpathlineto{\pgfqpoint{0.000000in}{-0.027778in}}%
\pgfusepath{stroke,fill}%
}%
\begin{pgfscope}%
\pgfsys@transformshift{3.354030in}{0.528000in}%
\pgfsys@useobject{currentmarker}{}%
\end{pgfscope}%
\end{pgfscope}%
\begin{pgfscope}%
\pgfsetbuttcap%
\pgfsetroundjoin%
\definecolor{currentfill}{rgb}{0.333333,0.333333,0.333333}%
\pgfsetfillcolor{currentfill}%
\pgfsetlinewidth{0.602250pt}%
\definecolor{currentstroke}{rgb}{0.333333,0.333333,0.333333}%
\pgfsetstrokecolor{currentstroke}%
\pgfsetdash{}{0pt}%
\pgfsys@defobject{currentmarker}{\pgfqpoint{0.000000in}{-0.027778in}}{\pgfqpoint{0.000000in}{0.000000in}}{%
\pgfpathmoveto{\pgfqpoint{0.000000in}{0.000000in}}%
\pgfpathlineto{\pgfqpoint{0.000000in}{-0.027778in}}%
\pgfusepath{stroke,fill}%
}%
\begin{pgfscope}%
\pgfsys@transformshift{3.391045in}{0.528000in}%
\pgfsys@useobject{currentmarker}{}%
\end{pgfscope}%
\end{pgfscope}%
\begin{pgfscope}%
\pgfsetbuttcap%
\pgfsetroundjoin%
\definecolor{currentfill}{rgb}{0.333333,0.333333,0.333333}%
\pgfsetfillcolor{currentfill}%
\pgfsetlinewidth{0.602250pt}%
\definecolor{currentstroke}{rgb}{0.333333,0.333333,0.333333}%
\pgfsetstrokecolor{currentstroke}%
\pgfsetdash{}{0pt}%
\pgfsys@defobject{currentmarker}{\pgfqpoint{0.000000in}{-0.027778in}}{\pgfqpoint{0.000000in}{0.000000in}}{%
\pgfpathmoveto{\pgfqpoint{0.000000in}{0.000000in}}%
\pgfpathlineto{\pgfqpoint{0.000000in}{-0.027778in}}%
\pgfusepath{stroke,fill}%
}%
\begin{pgfscope}%
\pgfsys@transformshift{3.428060in}{0.528000in}%
\pgfsys@useobject{currentmarker}{}%
\end{pgfscope}%
\end{pgfscope}%
\begin{pgfscope}%
\pgfsetbuttcap%
\pgfsetroundjoin%
\definecolor{currentfill}{rgb}{0.333333,0.333333,0.333333}%
\pgfsetfillcolor{currentfill}%
\pgfsetlinewidth{0.602250pt}%
\definecolor{currentstroke}{rgb}{0.333333,0.333333,0.333333}%
\pgfsetstrokecolor{currentstroke}%
\pgfsetdash{}{0pt}%
\pgfsys@defobject{currentmarker}{\pgfqpoint{0.000000in}{-0.027778in}}{\pgfqpoint{0.000000in}{0.000000in}}{%
\pgfpathmoveto{\pgfqpoint{0.000000in}{0.000000in}}%
\pgfpathlineto{\pgfqpoint{0.000000in}{-0.027778in}}%
\pgfusepath{stroke,fill}%
}%
\begin{pgfscope}%
\pgfsys@transformshift{3.465075in}{0.528000in}%
\pgfsys@useobject{currentmarker}{}%
\end{pgfscope}%
\end{pgfscope}%
\begin{pgfscope}%
\pgfsetbuttcap%
\pgfsetroundjoin%
\definecolor{currentfill}{rgb}{0.333333,0.333333,0.333333}%
\pgfsetfillcolor{currentfill}%
\pgfsetlinewidth{0.602250pt}%
\definecolor{currentstroke}{rgb}{0.333333,0.333333,0.333333}%
\pgfsetstrokecolor{currentstroke}%
\pgfsetdash{}{0pt}%
\pgfsys@defobject{currentmarker}{\pgfqpoint{0.000000in}{-0.027778in}}{\pgfqpoint{0.000000in}{0.000000in}}{%
\pgfpathmoveto{\pgfqpoint{0.000000in}{0.000000in}}%
\pgfpathlineto{\pgfqpoint{0.000000in}{-0.027778in}}%
\pgfusepath{stroke,fill}%
}%
\begin{pgfscope}%
\pgfsys@transformshift{3.502090in}{0.528000in}%
\pgfsys@useobject{currentmarker}{}%
\end{pgfscope}%
\end{pgfscope}%
\begin{pgfscope}%
\pgfsetbuttcap%
\pgfsetroundjoin%
\definecolor{currentfill}{rgb}{0.333333,0.333333,0.333333}%
\pgfsetfillcolor{currentfill}%
\pgfsetlinewidth{0.602250pt}%
\definecolor{currentstroke}{rgb}{0.333333,0.333333,0.333333}%
\pgfsetstrokecolor{currentstroke}%
\pgfsetdash{}{0pt}%
\pgfsys@defobject{currentmarker}{\pgfqpoint{0.000000in}{-0.027778in}}{\pgfqpoint{0.000000in}{0.000000in}}{%
\pgfpathmoveto{\pgfqpoint{0.000000in}{0.000000in}}%
\pgfpathlineto{\pgfqpoint{0.000000in}{-0.027778in}}%
\pgfusepath{stroke,fill}%
}%
\begin{pgfscope}%
\pgfsys@transformshift{3.539104in}{0.528000in}%
\pgfsys@useobject{currentmarker}{}%
\end{pgfscope}%
\end{pgfscope}%
\begin{pgfscope}%
\pgfsetbuttcap%
\pgfsetroundjoin%
\definecolor{currentfill}{rgb}{0.333333,0.333333,0.333333}%
\pgfsetfillcolor{currentfill}%
\pgfsetlinewidth{0.602250pt}%
\definecolor{currentstroke}{rgb}{0.333333,0.333333,0.333333}%
\pgfsetstrokecolor{currentstroke}%
\pgfsetdash{}{0pt}%
\pgfsys@defobject{currentmarker}{\pgfqpoint{0.000000in}{-0.027778in}}{\pgfqpoint{0.000000in}{0.000000in}}{%
\pgfpathmoveto{\pgfqpoint{0.000000in}{0.000000in}}%
\pgfpathlineto{\pgfqpoint{0.000000in}{-0.027778in}}%
\pgfusepath{stroke,fill}%
}%
\begin{pgfscope}%
\pgfsys@transformshift{3.576119in}{0.528000in}%
\pgfsys@useobject{currentmarker}{}%
\end{pgfscope}%
\end{pgfscope}%
\begin{pgfscope}%
\pgfsetbuttcap%
\pgfsetroundjoin%
\definecolor{currentfill}{rgb}{0.333333,0.333333,0.333333}%
\pgfsetfillcolor{currentfill}%
\pgfsetlinewidth{0.602250pt}%
\definecolor{currentstroke}{rgb}{0.333333,0.333333,0.333333}%
\pgfsetstrokecolor{currentstroke}%
\pgfsetdash{}{0pt}%
\pgfsys@defobject{currentmarker}{\pgfqpoint{0.000000in}{-0.027778in}}{\pgfqpoint{0.000000in}{0.000000in}}{%
\pgfpathmoveto{\pgfqpoint{0.000000in}{0.000000in}}%
\pgfpathlineto{\pgfqpoint{0.000000in}{-0.027778in}}%
\pgfusepath{stroke,fill}%
}%
\begin{pgfscope}%
\pgfsys@transformshift{3.613134in}{0.528000in}%
\pgfsys@useobject{currentmarker}{}%
\end{pgfscope}%
\end{pgfscope}%
\begin{pgfscope}%
\pgfsetbuttcap%
\pgfsetroundjoin%
\definecolor{currentfill}{rgb}{0.333333,0.333333,0.333333}%
\pgfsetfillcolor{currentfill}%
\pgfsetlinewidth{0.602250pt}%
\definecolor{currentstroke}{rgb}{0.333333,0.333333,0.333333}%
\pgfsetstrokecolor{currentstroke}%
\pgfsetdash{}{0pt}%
\pgfsys@defobject{currentmarker}{\pgfqpoint{0.000000in}{-0.027778in}}{\pgfqpoint{0.000000in}{0.000000in}}{%
\pgfpathmoveto{\pgfqpoint{0.000000in}{0.000000in}}%
\pgfpathlineto{\pgfqpoint{0.000000in}{-0.027778in}}%
\pgfusepath{stroke,fill}%
}%
\begin{pgfscope}%
\pgfsys@transformshift{3.650149in}{0.528000in}%
\pgfsys@useobject{currentmarker}{}%
\end{pgfscope}%
\end{pgfscope}%
\begin{pgfscope}%
\pgfsetbuttcap%
\pgfsetroundjoin%
\definecolor{currentfill}{rgb}{0.333333,0.333333,0.333333}%
\pgfsetfillcolor{currentfill}%
\pgfsetlinewidth{0.602250pt}%
\definecolor{currentstroke}{rgb}{0.333333,0.333333,0.333333}%
\pgfsetstrokecolor{currentstroke}%
\pgfsetdash{}{0pt}%
\pgfsys@defobject{currentmarker}{\pgfqpoint{0.000000in}{-0.027778in}}{\pgfqpoint{0.000000in}{0.000000in}}{%
\pgfpathmoveto{\pgfqpoint{0.000000in}{0.000000in}}%
\pgfpathlineto{\pgfqpoint{0.000000in}{-0.027778in}}%
\pgfusepath{stroke,fill}%
}%
\begin{pgfscope}%
\pgfsys@transformshift{3.687164in}{0.528000in}%
\pgfsys@useobject{currentmarker}{}%
\end{pgfscope}%
\end{pgfscope}%
\begin{pgfscope}%
\pgfsetbuttcap%
\pgfsetroundjoin%
\definecolor{currentfill}{rgb}{0.333333,0.333333,0.333333}%
\pgfsetfillcolor{currentfill}%
\pgfsetlinewidth{0.602250pt}%
\definecolor{currentstroke}{rgb}{0.333333,0.333333,0.333333}%
\pgfsetstrokecolor{currentstroke}%
\pgfsetdash{}{0pt}%
\pgfsys@defobject{currentmarker}{\pgfqpoint{0.000000in}{-0.027778in}}{\pgfqpoint{0.000000in}{0.000000in}}{%
\pgfpathmoveto{\pgfqpoint{0.000000in}{0.000000in}}%
\pgfpathlineto{\pgfqpoint{0.000000in}{-0.027778in}}%
\pgfusepath{stroke,fill}%
}%
\begin{pgfscope}%
\pgfsys@transformshift{3.724179in}{0.528000in}%
\pgfsys@useobject{currentmarker}{}%
\end{pgfscope}%
\end{pgfscope}%
\begin{pgfscope}%
\pgfsetbuttcap%
\pgfsetroundjoin%
\definecolor{currentfill}{rgb}{0.333333,0.333333,0.333333}%
\pgfsetfillcolor{currentfill}%
\pgfsetlinewidth{0.602250pt}%
\definecolor{currentstroke}{rgb}{0.333333,0.333333,0.333333}%
\pgfsetstrokecolor{currentstroke}%
\pgfsetdash{}{0pt}%
\pgfsys@defobject{currentmarker}{\pgfqpoint{0.000000in}{-0.027778in}}{\pgfqpoint{0.000000in}{0.000000in}}{%
\pgfpathmoveto{\pgfqpoint{0.000000in}{0.000000in}}%
\pgfpathlineto{\pgfqpoint{0.000000in}{-0.027778in}}%
\pgfusepath{stroke,fill}%
}%
\begin{pgfscope}%
\pgfsys@transformshift{3.761194in}{0.528000in}%
\pgfsys@useobject{currentmarker}{}%
\end{pgfscope}%
\end{pgfscope}%
\begin{pgfscope}%
\pgfsetbuttcap%
\pgfsetroundjoin%
\definecolor{currentfill}{rgb}{0.333333,0.333333,0.333333}%
\pgfsetfillcolor{currentfill}%
\pgfsetlinewidth{0.602250pt}%
\definecolor{currentstroke}{rgb}{0.333333,0.333333,0.333333}%
\pgfsetstrokecolor{currentstroke}%
\pgfsetdash{}{0pt}%
\pgfsys@defobject{currentmarker}{\pgfqpoint{0.000000in}{-0.027778in}}{\pgfqpoint{0.000000in}{0.000000in}}{%
\pgfpathmoveto{\pgfqpoint{0.000000in}{0.000000in}}%
\pgfpathlineto{\pgfqpoint{0.000000in}{-0.027778in}}%
\pgfusepath{stroke,fill}%
}%
\begin{pgfscope}%
\pgfsys@transformshift{3.798209in}{0.528000in}%
\pgfsys@useobject{currentmarker}{}%
\end{pgfscope}%
\end{pgfscope}%
\begin{pgfscope}%
\pgfsetbuttcap%
\pgfsetroundjoin%
\definecolor{currentfill}{rgb}{0.333333,0.333333,0.333333}%
\pgfsetfillcolor{currentfill}%
\pgfsetlinewidth{0.602250pt}%
\definecolor{currentstroke}{rgb}{0.333333,0.333333,0.333333}%
\pgfsetstrokecolor{currentstroke}%
\pgfsetdash{}{0pt}%
\pgfsys@defobject{currentmarker}{\pgfqpoint{0.000000in}{-0.027778in}}{\pgfqpoint{0.000000in}{0.000000in}}{%
\pgfpathmoveto{\pgfqpoint{0.000000in}{0.000000in}}%
\pgfpathlineto{\pgfqpoint{0.000000in}{-0.027778in}}%
\pgfusepath{stroke,fill}%
}%
\begin{pgfscope}%
\pgfsys@transformshift{3.835224in}{0.528000in}%
\pgfsys@useobject{currentmarker}{}%
\end{pgfscope}%
\end{pgfscope}%
\begin{pgfscope}%
\pgfsetbuttcap%
\pgfsetroundjoin%
\definecolor{currentfill}{rgb}{0.333333,0.333333,0.333333}%
\pgfsetfillcolor{currentfill}%
\pgfsetlinewidth{0.602250pt}%
\definecolor{currentstroke}{rgb}{0.333333,0.333333,0.333333}%
\pgfsetstrokecolor{currentstroke}%
\pgfsetdash{}{0pt}%
\pgfsys@defobject{currentmarker}{\pgfqpoint{0.000000in}{-0.027778in}}{\pgfqpoint{0.000000in}{0.000000in}}{%
\pgfpathmoveto{\pgfqpoint{0.000000in}{0.000000in}}%
\pgfpathlineto{\pgfqpoint{0.000000in}{-0.027778in}}%
\pgfusepath{stroke,fill}%
}%
\begin{pgfscope}%
\pgfsys@transformshift{3.872239in}{0.528000in}%
\pgfsys@useobject{currentmarker}{}%
\end{pgfscope}%
\end{pgfscope}%
\begin{pgfscope}%
\pgfsetbuttcap%
\pgfsetroundjoin%
\definecolor{currentfill}{rgb}{0.333333,0.333333,0.333333}%
\pgfsetfillcolor{currentfill}%
\pgfsetlinewidth{0.602250pt}%
\definecolor{currentstroke}{rgb}{0.333333,0.333333,0.333333}%
\pgfsetstrokecolor{currentstroke}%
\pgfsetdash{}{0pt}%
\pgfsys@defobject{currentmarker}{\pgfqpoint{0.000000in}{-0.027778in}}{\pgfqpoint{0.000000in}{0.000000in}}{%
\pgfpathmoveto{\pgfqpoint{0.000000in}{0.000000in}}%
\pgfpathlineto{\pgfqpoint{0.000000in}{-0.027778in}}%
\pgfusepath{stroke,fill}%
}%
\begin{pgfscope}%
\pgfsys@transformshift{3.909254in}{0.528000in}%
\pgfsys@useobject{currentmarker}{}%
\end{pgfscope}%
\end{pgfscope}%
\begin{pgfscope}%
\pgfsetbuttcap%
\pgfsetroundjoin%
\definecolor{currentfill}{rgb}{0.333333,0.333333,0.333333}%
\pgfsetfillcolor{currentfill}%
\pgfsetlinewidth{0.602250pt}%
\definecolor{currentstroke}{rgb}{0.333333,0.333333,0.333333}%
\pgfsetstrokecolor{currentstroke}%
\pgfsetdash{}{0pt}%
\pgfsys@defobject{currentmarker}{\pgfqpoint{0.000000in}{-0.027778in}}{\pgfqpoint{0.000000in}{0.000000in}}{%
\pgfpathmoveto{\pgfqpoint{0.000000in}{0.000000in}}%
\pgfpathlineto{\pgfqpoint{0.000000in}{-0.027778in}}%
\pgfusepath{stroke,fill}%
}%
\begin{pgfscope}%
\pgfsys@transformshift{3.946269in}{0.528000in}%
\pgfsys@useobject{currentmarker}{}%
\end{pgfscope}%
\end{pgfscope}%
\begin{pgfscope}%
\pgfsetbuttcap%
\pgfsetroundjoin%
\definecolor{currentfill}{rgb}{0.333333,0.333333,0.333333}%
\pgfsetfillcolor{currentfill}%
\pgfsetlinewidth{0.602250pt}%
\definecolor{currentstroke}{rgb}{0.333333,0.333333,0.333333}%
\pgfsetstrokecolor{currentstroke}%
\pgfsetdash{}{0pt}%
\pgfsys@defobject{currentmarker}{\pgfqpoint{0.000000in}{-0.027778in}}{\pgfqpoint{0.000000in}{0.000000in}}{%
\pgfpathmoveto{\pgfqpoint{0.000000in}{0.000000in}}%
\pgfpathlineto{\pgfqpoint{0.000000in}{-0.027778in}}%
\pgfusepath{stroke,fill}%
}%
\begin{pgfscope}%
\pgfsys@transformshift{3.983284in}{0.528000in}%
\pgfsys@useobject{currentmarker}{}%
\end{pgfscope}%
\end{pgfscope}%
\begin{pgfscope}%
\pgfsetbuttcap%
\pgfsetroundjoin%
\definecolor{currentfill}{rgb}{0.333333,0.333333,0.333333}%
\pgfsetfillcolor{currentfill}%
\pgfsetlinewidth{0.602250pt}%
\definecolor{currentstroke}{rgb}{0.333333,0.333333,0.333333}%
\pgfsetstrokecolor{currentstroke}%
\pgfsetdash{}{0pt}%
\pgfsys@defobject{currentmarker}{\pgfqpoint{0.000000in}{-0.027778in}}{\pgfqpoint{0.000000in}{0.000000in}}{%
\pgfpathmoveto{\pgfqpoint{0.000000in}{0.000000in}}%
\pgfpathlineto{\pgfqpoint{0.000000in}{-0.027778in}}%
\pgfusepath{stroke,fill}%
}%
\begin{pgfscope}%
\pgfsys@transformshift{4.020299in}{0.528000in}%
\pgfsys@useobject{currentmarker}{}%
\end{pgfscope}%
\end{pgfscope}%
\begin{pgfscope}%
\pgfsetbuttcap%
\pgfsetroundjoin%
\definecolor{currentfill}{rgb}{0.333333,0.333333,0.333333}%
\pgfsetfillcolor{currentfill}%
\pgfsetlinewidth{0.602250pt}%
\definecolor{currentstroke}{rgb}{0.333333,0.333333,0.333333}%
\pgfsetstrokecolor{currentstroke}%
\pgfsetdash{}{0pt}%
\pgfsys@defobject{currentmarker}{\pgfqpoint{0.000000in}{-0.027778in}}{\pgfqpoint{0.000000in}{0.000000in}}{%
\pgfpathmoveto{\pgfqpoint{0.000000in}{0.000000in}}%
\pgfpathlineto{\pgfqpoint{0.000000in}{-0.027778in}}%
\pgfusepath{stroke,fill}%
}%
\begin{pgfscope}%
\pgfsys@transformshift{4.057313in}{0.528000in}%
\pgfsys@useobject{currentmarker}{}%
\end{pgfscope}%
\end{pgfscope}%
\begin{pgfscope}%
\pgfsetbuttcap%
\pgfsetroundjoin%
\definecolor{currentfill}{rgb}{0.333333,0.333333,0.333333}%
\pgfsetfillcolor{currentfill}%
\pgfsetlinewidth{0.602250pt}%
\definecolor{currentstroke}{rgb}{0.333333,0.333333,0.333333}%
\pgfsetstrokecolor{currentstroke}%
\pgfsetdash{}{0pt}%
\pgfsys@defobject{currentmarker}{\pgfqpoint{0.000000in}{-0.027778in}}{\pgfqpoint{0.000000in}{0.000000in}}{%
\pgfpathmoveto{\pgfqpoint{0.000000in}{0.000000in}}%
\pgfpathlineto{\pgfqpoint{0.000000in}{-0.027778in}}%
\pgfusepath{stroke,fill}%
}%
\begin{pgfscope}%
\pgfsys@transformshift{4.094328in}{0.528000in}%
\pgfsys@useobject{currentmarker}{}%
\end{pgfscope}%
\end{pgfscope}%
\begin{pgfscope}%
\pgfsetbuttcap%
\pgfsetroundjoin%
\definecolor{currentfill}{rgb}{0.333333,0.333333,0.333333}%
\pgfsetfillcolor{currentfill}%
\pgfsetlinewidth{0.602250pt}%
\definecolor{currentstroke}{rgb}{0.333333,0.333333,0.333333}%
\pgfsetstrokecolor{currentstroke}%
\pgfsetdash{}{0pt}%
\pgfsys@defobject{currentmarker}{\pgfqpoint{0.000000in}{-0.027778in}}{\pgfqpoint{0.000000in}{0.000000in}}{%
\pgfpathmoveto{\pgfqpoint{0.000000in}{0.000000in}}%
\pgfpathlineto{\pgfqpoint{0.000000in}{-0.027778in}}%
\pgfusepath{stroke,fill}%
}%
\begin{pgfscope}%
\pgfsys@transformshift{4.131343in}{0.528000in}%
\pgfsys@useobject{currentmarker}{}%
\end{pgfscope}%
\end{pgfscope}%
\begin{pgfscope}%
\pgfsetbuttcap%
\pgfsetroundjoin%
\definecolor{currentfill}{rgb}{0.333333,0.333333,0.333333}%
\pgfsetfillcolor{currentfill}%
\pgfsetlinewidth{0.602250pt}%
\definecolor{currentstroke}{rgb}{0.333333,0.333333,0.333333}%
\pgfsetstrokecolor{currentstroke}%
\pgfsetdash{}{0pt}%
\pgfsys@defobject{currentmarker}{\pgfqpoint{0.000000in}{-0.027778in}}{\pgfqpoint{0.000000in}{0.000000in}}{%
\pgfpathmoveto{\pgfqpoint{0.000000in}{0.000000in}}%
\pgfpathlineto{\pgfqpoint{0.000000in}{-0.027778in}}%
\pgfusepath{stroke,fill}%
}%
\begin{pgfscope}%
\pgfsys@transformshift{4.168358in}{0.528000in}%
\pgfsys@useobject{currentmarker}{}%
\end{pgfscope}%
\end{pgfscope}%
\begin{pgfscope}%
\pgfsetbuttcap%
\pgfsetroundjoin%
\definecolor{currentfill}{rgb}{0.333333,0.333333,0.333333}%
\pgfsetfillcolor{currentfill}%
\pgfsetlinewidth{0.602250pt}%
\definecolor{currentstroke}{rgb}{0.333333,0.333333,0.333333}%
\pgfsetstrokecolor{currentstroke}%
\pgfsetdash{}{0pt}%
\pgfsys@defobject{currentmarker}{\pgfqpoint{0.000000in}{-0.027778in}}{\pgfqpoint{0.000000in}{0.000000in}}{%
\pgfpathmoveto{\pgfqpoint{0.000000in}{0.000000in}}%
\pgfpathlineto{\pgfqpoint{0.000000in}{-0.027778in}}%
\pgfusepath{stroke,fill}%
}%
\begin{pgfscope}%
\pgfsys@transformshift{4.205373in}{0.528000in}%
\pgfsys@useobject{currentmarker}{}%
\end{pgfscope}%
\end{pgfscope}%
\begin{pgfscope}%
\pgfsetbuttcap%
\pgfsetroundjoin%
\definecolor{currentfill}{rgb}{0.333333,0.333333,0.333333}%
\pgfsetfillcolor{currentfill}%
\pgfsetlinewidth{0.602250pt}%
\definecolor{currentstroke}{rgb}{0.333333,0.333333,0.333333}%
\pgfsetstrokecolor{currentstroke}%
\pgfsetdash{}{0pt}%
\pgfsys@defobject{currentmarker}{\pgfqpoint{0.000000in}{-0.027778in}}{\pgfqpoint{0.000000in}{0.000000in}}{%
\pgfpathmoveto{\pgfqpoint{0.000000in}{0.000000in}}%
\pgfpathlineto{\pgfqpoint{0.000000in}{-0.027778in}}%
\pgfusepath{stroke,fill}%
}%
\begin{pgfscope}%
\pgfsys@transformshift{4.242388in}{0.528000in}%
\pgfsys@useobject{currentmarker}{}%
\end{pgfscope}%
\end{pgfscope}%
\begin{pgfscope}%
\pgfsetbuttcap%
\pgfsetroundjoin%
\definecolor{currentfill}{rgb}{0.333333,0.333333,0.333333}%
\pgfsetfillcolor{currentfill}%
\pgfsetlinewidth{0.602250pt}%
\definecolor{currentstroke}{rgb}{0.333333,0.333333,0.333333}%
\pgfsetstrokecolor{currentstroke}%
\pgfsetdash{}{0pt}%
\pgfsys@defobject{currentmarker}{\pgfqpoint{0.000000in}{-0.027778in}}{\pgfqpoint{0.000000in}{0.000000in}}{%
\pgfpathmoveto{\pgfqpoint{0.000000in}{0.000000in}}%
\pgfpathlineto{\pgfqpoint{0.000000in}{-0.027778in}}%
\pgfusepath{stroke,fill}%
}%
\begin{pgfscope}%
\pgfsys@transformshift{4.279403in}{0.528000in}%
\pgfsys@useobject{currentmarker}{}%
\end{pgfscope}%
\end{pgfscope}%
\begin{pgfscope}%
\pgfsetbuttcap%
\pgfsetroundjoin%
\definecolor{currentfill}{rgb}{0.333333,0.333333,0.333333}%
\pgfsetfillcolor{currentfill}%
\pgfsetlinewidth{0.602250pt}%
\definecolor{currentstroke}{rgb}{0.333333,0.333333,0.333333}%
\pgfsetstrokecolor{currentstroke}%
\pgfsetdash{}{0pt}%
\pgfsys@defobject{currentmarker}{\pgfqpoint{0.000000in}{-0.027778in}}{\pgfqpoint{0.000000in}{0.000000in}}{%
\pgfpathmoveto{\pgfqpoint{0.000000in}{0.000000in}}%
\pgfpathlineto{\pgfqpoint{0.000000in}{-0.027778in}}%
\pgfusepath{stroke,fill}%
}%
\begin{pgfscope}%
\pgfsys@transformshift{4.316418in}{0.528000in}%
\pgfsys@useobject{currentmarker}{}%
\end{pgfscope}%
\end{pgfscope}%
\begin{pgfscope}%
\pgfsetbuttcap%
\pgfsetroundjoin%
\definecolor{currentfill}{rgb}{0.333333,0.333333,0.333333}%
\pgfsetfillcolor{currentfill}%
\pgfsetlinewidth{0.602250pt}%
\definecolor{currentstroke}{rgb}{0.333333,0.333333,0.333333}%
\pgfsetstrokecolor{currentstroke}%
\pgfsetdash{}{0pt}%
\pgfsys@defobject{currentmarker}{\pgfqpoint{0.000000in}{-0.027778in}}{\pgfqpoint{0.000000in}{0.000000in}}{%
\pgfpathmoveto{\pgfqpoint{0.000000in}{0.000000in}}%
\pgfpathlineto{\pgfqpoint{0.000000in}{-0.027778in}}%
\pgfusepath{stroke,fill}%
}%
\begin{pgfscope}%
\pgfsys@transformshift{4.353433in}{0.528000in}%
\pgfsys@useobject{currentmarker}{}%
\end{pgfscope}%
\end{pgfscope}%
\begin{pgfscope}%
\pgfsetbuttcap%
\pgfsetroundjoin%
\definecolor{currentfill}{rgb}{0.333333,0.333333,0.333333}%
\pgfsetfillcolor{currentfill}%
\pgfsetlinewidth{0.602250pt}%
\definecolor{currentstroke}{rgb}{0.333333,0.333333,0.333333}%
\pgfsetstrokecolor{currentstroke}%
\pgfsetdash{}{0pt}%
\pgfsys@defobject{currentmarker}{\pgfqpoint{0.000000in}{-0.027778in}}{\pgfqpoint{0.000000in}{0.000000in}}{%
\pgfpathmoveto{\pgfqpoint{0.000000in}{0.000000in}}%
\pgfpathlineto{\pgfqpoint{0.000000in}{-0.027778in}}%
\pgfusepath{stroke,fill}%
}%
\begin{pgfscope}%
\pgfsys@transformshift{4.390448in}{0.528000in}%
\pgfsys@useobject{currentmarker}{}%
\end{pgfscope}%
\end{pgfscope}%
\begin{pgfscope}%
\pgfsetbuttcap%
\pgfsetroundjoin%
\definecolor{currentfill}{rgb}{0.333333,0.333333,0.333333}%
\pgfsetfillcolor{currentfill}%
\pgfsetlinewidth{0.602250pt}%
\definecolor{currentstroke}{rgb}{0.333333,0.333333,0.333333}%
\pgfsetstrokecolor{currentstroke}%
\pgfsetdash{}{0pt}%
\pgfsys@defobject{currentmarker}{\pgfqpoint{0.000000in}{-0.027778in}}{\pgfqpoint{0.000000in}{0.000000in}}{%
\pgfpathmoveto{\pgfqpoint{0.000000in}{0.000000in}}%
\pgfpathlineto{\pgfqpoint{0.000000in}{-0.027778in}}%
\pgfusepath{stroke,fill}%
}%
\begin{pgfscope}%
\pgfsys@transformshift{4.427463in}{0.528000in}%
\pgfsys@useobject{currentmarker}{}%
\end{pgfscope}%
\end{pgfscope}%
\begin{pgfscope}%
\pgfsetbuttcap%
\pgfsetroundjoin%
\definecolor{currentfill}{rgb}{0.333333,0.333333,0.333333}%
\pgfsetfillcolor{currentfill}%
\pgfsetlinewidth{0.602250pt}%
\definecolor{currentstroke}{rgb}{0.333333,0.333333,0.333333}%
\pgfsetstrokecolor{currentstroke}%
\pgfsetdash{}{0pt}%
\pgfsys@defobject{currentmarker}{\pgfqpoint{0.000000in}{-0.027778in}}{\pgfqpoint{0.000000in}{0.000000in}}{%
\pgfpathmoveto{\pgfqpoint{0.000000in}{0.000000in}}%
\pgfpathlineto{\pgfqpoint{0.000000in}{-0.027778in}}%
\pgfusepath{stroke,fill}%
}%
\begin{pgfscope}%
\pgfsys@transformshift{4.464478in}{0.528000in}%
\pgfsys@useobject{currentmarker}{}%
\end{pgfscope}%
\end{pgfscope}%
\begin{pgfscope}%
\pgfsetbuttcap%
\pgfsetroundjoin%
\definecolor{currentfill}{rgb}{0.333333,0.333333,0.333333}%
\pgfsetfillcolor{currentfill}%
\pgfsetlinewidth{0.602250pt}%
\definecolor{currentstroke}{rgb}{0.333333,0.333333,0.333333}%
\pgfsetstrokecolor{currentstroke}%
\pgfsetdash{}{0pt}%
\pgfsys@defobject{currentmarker}{\pgfqpoint{0.000000in}{-0.027778in}}{\pgfqpoint{0.000000in}{0.000000in}}{%
\pgfpathmoveto{\pgfqpoint{0.000000in}{0.000000in}}%
\pgfpathlineto{\pgfqpoint{0.000000in}{-0.027778in}}%
\pgfusepath{stroke,fill}%
}%
\begin{pgfscope}%
\pgfsys@transformshift{4.501493in}{0.528000in}%
\pgfsys@useobject{currentmarker}{}%
\end{pgfscope}%
\end{pgfscope}%
\begin{pgfscope}%
\pgfsetbuttcap%
\pgfsetroundjoin%
\definecolor{currentfill}{rgb}{0.333333,0.333333,0.333333}%
\pgfsetfillcolor{currentfill}%
\pgfsetlinewidth{0.602250pt}%
\definecolor{currentstroke}{rgb}{0.333333,0.333333,0.333333}%
\pgfsetstrokecolor{currentstroke}%
\pgfsetdash{}{0pt}%
\pgfsys@defobject{currentmarker}{\pgfqpoint{0.000000in}{-0.027778in}}{\pgfqpoint{0.000000in}{0.000000in}}{%
\pgfpathmoveto{\pgfqpoint{0.000000in}{0.000000in}}%
\pgfpathlineto{\pgfqpoint{0.000000in}{-0.027778in}}%
\pgfusepath{stroke,fill}%
}%
\begin{pgfscope}%
\pgfsys@transformshift{4.538507in}{0.528000in}%
\pgfsys@useobject{currentmarker}{}%
\end{pgfscope}%
\end{pgfscope}%
\begin{pgfscope}%
\pgfsetbuttcap%
\pgfsetroundjoin%
\definecolor{currentfill}{rgb}{0.333333,0.333333,0.333333}%
\pgfsetfillcolor{currentfill}%
\pgfsetlinewidth{0.602250pt}%
\definecolor{currentstroke}{rgb}{0.333333,0.333333,0.333333}%
\pgfsetstrokecolor{currentstroke}%
\pgfsetdash{}{0pt}%
\pgfsys@defobject{currentmarker}{\pgfqpoint{0.000000in}{-0.027778in}}{\pgfqpoint{0.000000in}{0.000000in}}{%
\pgfpathmoveto{\pgfqpoint{0.000000in}{0.000000in}}%
\pgfpathlineto{\pgfqpoint{0.000000in}{-0.027778in}}%
\pgfusepath{stroke,fill}%
}%
\begin{pgfscope}%
\pgfsys@transformshift{4.575522in}{0.528000in}%
\pgfsys@useobject{currentmarker}{}%
\end{pgfscope}%
\end{pgfscope}%
\begin{pgfscope}%
\pgfsetbuttcap%
\pgfsetroundjoin%
\definecolor{currentfill}{rgb}{0.333333,0.333333,0.333333}%
\pgfsetfillcolor{currentfill}%
\pgfsetlinewidth{0.602250pt}%
\definecolor{currentstroke}{rgb}{0.333333,0.333333,0.333333}%
\pgfsetstrokecolor{currentstroke}%
\pgfsetdash{}{0pt}%
\pgfsys@defobject{currentmarker}{\pgfqpoint{0.000000in}{-0.027778in}}{\pgfqpoint{0.000000in}{0.000000in}}{%
\pgfpathmoveto{\pgfqpoint{0.000000in}{0.000000in}}%
\pgfpathlineto{\pgfqpoint{0.000000in}{-0.027778in}}%
\pgfusepath{stroke,fill}%
}%
\begin{pgfscope}%
\pgfsys@transformshift{4.612537in}{0.528000in}%
\pgfsys@useobject{currentmarker}{}%
\end{pgfscope}%
\end{pgfscope}%
\begin{pgfscope}%
\pgfsetbuttcap%
\pgfsetroundjoin%
\definecolor{currentfill}{rgb}{0.333333,0.333333,0.333333}%
\pgfsetfillcolor{currentfill}%
\pgfsetlinewidth{0.602250pt}%
\definecolor{currentstroke}{rgb}{0.333333,0.333333,0.333333}%
\pgfsetstrokecolor{currentstroke}%
\pgfsetdash{}{0pt}%
\pgfsys@defobject{currentmarker}{\pgfqpoint{0.000000in}{-0.027778in}}{\pgfqpoint{0.000000in}{0.000000in}}{%
\pgfpathmoveto{\pgfqpoint{0.000000in}{0.000000in}}%
\pgfpathlineto{\pgfqpoint{0.000000in}{-0.027778in}}%
\pgfusepath{stroke,fill}%
}%
\begin{pgfscope}%
\pgfsys@transformshift{4.649552in}{0.528000in}%
\pgfsys@useobject{currentmarker}{}%
\end{pgfscope}%
\end{pgfscope}%
\begin{pgfscope}%
\pgfsetbuttcap%
\pgfsetroundjoin%
\definecolor{currentfill}{rgb}{0.333333,0.333333,0.333333}%
\pgfsetfillcolor{currentfill}%
\pgfsetlinewidth{0.602250pt}%
\definecolor{currentstroke}{rgb}{0.333333,0.333333,0.333333}%
\pgfsetstrokecolor{currentstroke}%
\pgfsetdash{}{0pt}%
\pgfsys@defobject{currentmarker}{\pgfqpoint{0.000000in}{-0.027778in}}{\pgfqpoint{0.000000in}{0.000000in}}{%
\pgfpathmoveto{\pgfqpoint{0.000000in}{0.000000in}}%
\pgfpathlineto{\pgfqpoint{0.000000in}{-0.027778in}}%
\pgfusepath{stroke,fill}%
}%
\begin{pgfscope}%
\pgfsys@transformshift{4.686567in}{0.528000in}%
\pgfsys@useobject{currentmarker}{}%
\end{pgfscope}%
\end{pgfscope}%
\begin{pgfscope}%
\pgfsetbuttcap%
\pgfsetroundjoin%
\definecolor{currentfill}{rgb}{0.333333,0.333333,0.333333}%
\pgfsetfillcolor{currentfill}%
\pgfsetlinewidth{0.602250pt}%
\definecolor{currentstroke}{rgb}{0.333333,0.333333,0.333333}%
\pgfsetstrokecolor{currentstroke}%
\pgfsetdash{}{0pt}%
\pgfsys@defobject{currentmarker}{\pgfqpoint{0.000000in}{-0.027778in}}{\pgfqpoint{0.000000in}{0.000000in}}{%
\pgfpathmoveto{\pgfqpoint{0.000000in}{0.000000in}}%
\pgfpathlineto{\pgfqpoint{0.000000in}{-0.027778in}}%
\pgfusepath{stroke,fill}%
}%
\begin{pgfscope}%
\pgfsys@transformshift{4.723582in}{0.528000in}%
\pgfsys@useobject{currentmarker}{}%
\end{pgfscope}%
\end{pgfscope}%
\begin{pgfscope}%
\pgfsetbuttcap%
\pgfsetroundjoin%
\definecolor{currentfill}{rgb}{0.333333,0.333333,0.333333}%
\pgfsetfillcolor{currentfill}%
\pgfsetlinewidth{0.602250pt}%
\definecolor{currentstroke}{rgb}{0.333333,0.333333,0.333333}%
\pgfsetstrokecolor{currentstroke}%
\pgfsetdash{}{0pt}%
\pgfsys@defobject{currentmarker}{\pgfqpoint{0.000000in}{-0.027778in}}{\pgfqpoint{0.000000in}{0.000000in}}{%
\pgfpathmoveto{\pgfqpoint{0.000000in}{0.000000in}}%
\pgfpathlineto{\pgfqpoint{0.000000in}{-0.027778in}}%
\pgfusepath{stroke,fill}%
}%
\begin{pgfscope}%
\pgfsys@transformshift{4.760597in}{0.528000in}%
\pgfsys@useobject{currentmarker}{}%
\end{pgfscope}%
\end{pgfscope}%
\begin{pgfscope}%
\pgfsetbuttcap%
\pgfsetroundjoin%
\definecolor{currentfill}{rgb}{0.333333,0.333333,0.333333}%
\pgfsetfillcolor{currentfill}%
\pgfsetlinewidth{0.602250pt}%
\definecolor{currentstroke}{rgb}{0.333333,0.333333,0.333333}%
\pgfsetstrokecolor{currentstroke}%
\pgfsetdash{}{0pt}%
\pgfsys@defobject{currentmarker}{\pgfqpoint{0.000000in}{-0.027778in}}{\pgfqpoint{0.000000in}{0.000000in}}{%
\pgfpathmoveto{\pgfqpoint{0.000000in}{0.000000in}}%
\pgfpathlineto{\pgfqpoint{0.000000in}{-0.027778in}}%
\pgfusepath{stroke,fill}%
}%
\begin{pgfscope}%
\pgfsys@transformshift{4.797612in}{0.528000in}%
\pgfsys@useobject{currentmarker}{}%
\end{pgfscope}%
\end{pgfscope}%
\begin{pgfscope}%
\pgfsetbuttcap%
\pgfsetroundjoin%
\definecolor{currentfill}{rgb}{0.333333,0.333333,0.333333}%
\pgfsetfillcolor{currentfill}%
\pgfsetlinewidth{0.602250pt}%
\definecolor{currentstroke}{rgb}{0.333333,0.333333,0.333333}%
\pgfsetstrokecolor{currentstroke}%
\pgfsetdash{}{0pt}%
\pgfsys@defobject{currentmarker}{\pgfqpoint{0.000000in}{-0.027778in}}{\pgfqpoint{0.000000in}{0.000000in}}{%
\pgfpathmoveto{\pgfqpoint{0.000000in}{0.000000in}}%
\pgfpathlineto{\pgfqpoint{0.000000in}{-0.027778in}}%
\pgfusepath{stroke,fill}%
}%
\begin{pgfscope}%
\pgfsys@transformshift{4.834627in}{0.528000in}%
\pgfsys@useobject{currentmarker}{}%
\end{pgfscope}%
\end{pgfscope}%
\begin{pgfscope}%
\pgfsetbuttcap%
\pgfsetroundjoin%
\definecolor{currentfill}{rgb}{0.333333,0.333333,0.333333}%
\pgfsetfillcolor{currentfill}%
\pgfsetlinewidth{0.602250pt}%
\definecolor{currentstroke}{rgb}{0.333333,0.333333,0.333333}%
\pgfsetstrokecolor{currentstroke}%
\pgfsetdash{}{0pt}%
\pgfsys@defobject{currentmarker}{\pgfqpoint{0.000000in}{-0.027778in}}{\pgfqpoint{0.000000in}{0.000000in}}{%
\pgfpathmoveto{\pgfqpoint{0.000000in}{0.000000in}}%
\pgfpathlineto{\pgfqpoint{0.000000in}{-0.027778in}}%
\pgfusepath{stroke,fill}%
}%
\begin{pgfscope}%
\pgfsys@transformshift{4.871642in}{0.528000in}%
\pgfsys@useobject{currentmarker}{}%
\end{pgfscope}%
\end{pgfscope}%
\begin{pgfscope}%
\pgfsetbuttcap%
\pgfsetroundjoin%
\definecolor{currentfill}{rgb}{0.333333,0.333333,0.333333}%
\pgfsetfillcolor{currentfill}%
\pgfsetlinewidth{0.602250pt}%
\definecolor{currentstroke}{rgb}{0.333333,0.333333,0.333333}%
\pgfsetstrokecolor{currentstroke}%
\pgfsetdash{}{0pt}%
\pgfsys@defobject{currentmarker}{\pgfqpoint{0.000000in}{-0.027778in}}{\pgfqpoint{0.000000in}{0.000000in}}{%
\pgfpathmoveto{\pgfqpoint{0.000000in}{0.000000in}}%
\pgfpathlineto{\pgfqpoint{0.000000in}{-0.027778in}}%
\pgfusepath{stroke,fill}%
}%
\begin{pgfscope}%
\pgfsys@transformshift{4.908657in}{0.528000in}%
\pgfsys@useobject{currentmarker}{}%
\end{pgfscope}%
\end{pgfscope}%
\begin{pgfscope}%
\pgfsetbuttcap%
\pgfsetroundjoin%
\definecolor{currentfill}{rgb}{0.333333,0.333333,0.333333}%
\pgfsetfillcolor{currentfill}%
\pgfsetlinewidth{0.602250pt}%
\definecolor{currentstroke}{rgb}{0.333333,0.333333,0.333333}%
\pgfsetstrokecolor{currentstroke}%
\pgfsetdash{}{0pt}%
\pgfsys@defobject{currentmarker}{\pgfqpoint{0.000000in}{-0.027778in}}{\pgfqpoint{0.000000in}{0.000000in}}{%
\pgfpathmoveto{\pgfqpoint{0.000000in}{0.000000in}}%
\pgfpathlineto{\pgfqpoint{0.000000in}{-0.027778in}}%
\pgfusepath{stroke,fill}%
}%
\begin{pgfscope}%
\pgfsys@transformshift{4.945672in}{0.528000in}%
\pgfsys@useobject{currentmarker}{}%
\end{pgfscope}%
\end{pgfscope}%
\begin{pgfscope}%
\pgfsetbuttcap%
\pgfsetroundjoin%
\definecolor{currentfill}{rgb}{0.333333,0.333333,0.333333}%
\pgfsetfillcolor{currentfill}%
\pgfsetlinewidth{0.602250pt}%
\definecolor{currentstroke}{rgb}{0.333333,0.333333,0.333333}%
\pgfsetstrokecolor{currentstroke}%
\pgfsetdash{}{0pt}%
\pgfsys@defobject{currentmarker}{\pgfqpoint{0.000000in}{-0.027778in}}{\pgfqpoint{0.000000in}{0.000000in}}{%
\pgfpathmoveto{\pgfqpoint{0.000000in}{0.000000in}}%
\pgfpathlineto{\pgfqpoint{0.000000in}{-0.027778in}}%
\pgfusepath{stroke,fill}%
}%
\begin{pgfscope}%
\pgfsys@transformshift{4.982687in}{0.528000in}%
\pgfsys@useobject{currentmarker}{}%
\end{pgfscope}%
\end{pgfscope}%
\begin{pgfscope}%
\pgfsetbuttcap%
\pgfsetroundjoin%
\definecolor{currentfill}{rgb}{0.333333,0.333333,0.333333}%
\pgfsetfillcolor{currentfill}%
\pgfsetlinewidth{0.602250pt}%
\definecolor{currentstroke}{rgb}{0.333333,0.333333,0.333333}%
\pgfsetstrokecolor{currentstroke}%
\pgfsetdash{}{0pt}%
\pgfsys@defobject{currentmarker}{\pgfqpoint{0.000000in}{-0.027778in}}{\pgfqpoint{0.000000in}{0.000000in}}{%
\pgfpathmoveto{\pgfqpoint{0.000000in}{0.000000in}}%
\pgfpathlineto{\pgfqpoint{0.000000in}{-0.027778in}}%
\pgfusepath{stroke,fill}%
}%
\begin{pgfscope}%
\pgfsys@transformshift{5.019701in}{0.528000in}%
\pgfsys@useobject{currentmarker}{}%
\end{pgfscope}%
\end{pgfscope}%
\begin{pgfscope}%
\pgfsetbuttcap%
\pgfsetroundjoin%
\definecolor{currentfill}{rgb}{0.333333,0.333333,0.333333}%
\pgfsetfillcolor{currentfill}%
\pgfsetlinewidth{0.602250pt}%
\definecolor{currentstroke}{rgb}{0.333333,0.333333,0.333333}%
\pgfsetstrokecolor{currentstroke}%
\pgfsetdash{}{0pt}%
\pgfsys@defobject{currentmarker}{\pgfqpoint{0.000000in}{-0.027778in}}{\pgfqpoint{0.000000in}{0.000000in}}{%
\pgfpathmoveto{\pgfqpoint{0.000000in}{0.000000in}}%
\pgfpathlineto{\pgfqpoint{0.000000in}{-0.027778in}}%
\pgfusepath{stroke,fill}%
}%
\begin{pgfscope}%
\pgfsys@transformshift{5.056716in}{0.528000in}%
\pgfsys@useobject{currentmarker}{}%
\end{pgfscope}%
\end{pgfscope}%
\begin{pgfscope}%
\pgfsetbuttcap%
\pgfsetroundjoin%
\definecolor{currentfill}{rgb}{0.333333,0.333333,0.333333}%
\pgfsetfillcolor{currentfill}%
\pgfsetlinewidth{0.602250pt}%
\definecolor{currentstroke}{rgb}{0.333333,0.333333,0.333333}%
\pgfsetstrokecolor{currentstroke}%
\pgfsetdash{}{0pt}%
\pgfsys@defobject{currentmarker}{\pgfqpoint{0.000000in}{-0.027778in}}{\pgfqpoint{0.000000in}{0.000000in}}{%
\pgfpathmoveto{\pgfqpoint{0.000000in}{0.000000in}}%
\pgfpathlineto{\pgfqpoint{0.000000in}{-0.027778in}}%
\pgfusepath{stroke,fill}%
}%
\begin{pgfscope}%
\pgfsys@transformshift{5.093731in}{0.528000in}%
\pgfsys@useobject{currentmarker}{}%
\end{pgfscope}%
\end{pgfscope}%
\begin{pgfscope}%
\pgfsetbuttcap%
\pgfsetroundjoin%
\definecolor{currentfill}{rgb}{0.333333,0.333333,0.333333}%
\pgfsetfillcolor{currentfill}%
\pgfsetlinewidth{0.602250pt}%
\definecolor{currentstroke}{rgb}{0.333333,0.333333,0.333333}%
\pgfsetstrokecolor{currentstroke}%
\pgfsetdash{}{0pt}%
\pgfsys@defobject{currentmarker}{\pgfqpoint{0.000000in}{-0.027778in}}{\pgfqpoint{0.000000in}{0.000000in}}{%
\pgfpathmoveto{\pgfqpoint{0.000000in}{0.000000in}}%
\pgfpathlineto{\pgfqpoint{0.000000in}{-0.027778in}}%
\pgfusepath{stroke,fill}%
}%
\begin{pgfscope}%
\pgfsys@transformshift{5.130746in}{0.528000in}%
\pgfsys@useobject{currentmarker}{}%
\end{pgfscope}%
\end{pgfscope}%
\begin{pgfscope}%
\pgfsetbuttcap%
\pgfsetroundjoin%
\definecolor{currentfill}{rgb}{0.333333,0.333333,0.333333}%
\pgfsetfillcolor{currentfill}%
\pgfsetlinewidth{0.602250pt}%
\definecolor{currentstroke}{rgb}{0.333333,0.333333,0.333333}%
\pgfsetstrokecolor{currentstroke}%
\pgfsetdash{}{0pt}%
\pgfsys@defobject{currentmarker}{\pgfqpoint{0.000000in}{-0.027778in}}{\pgfqpoint{0.000000in}{0.000000in}}{%
\pgfpathmoveto{\pgfqpoint{0.000000in}{0.000000in}}%
\pgfpathlineto{\pgfqpoint{0.000000in}{-0.027778in}}%
\pgfusepath{stroke,fill}%
}%
\begin{pgfscope}%
\pgfsys@transformshift{5.167761in}{0.528000in}%
\pgfsys@useobject{currentmarker}{}%
\end{pgfscope}%
\end{pgfscope}%
\begin{pgfscope}%
\pgfsetbuttcap%
\pgfsetroundjoin%
\definecolor{currentfill}{rgb}{0.333333,0.333333,0.333333}%
\pgfsetfillcolor{currentfill}%
\pgfsetlinewidth{0.602250pt}%
\definecolor{currentstroke}{rgb}{0.333333,0.333333,0.333333}%
\pgfsetstrokecolor{currentstroke}%
\pgfsetdash{}{0pt}%
\pgfsys@defobject{currentmarker}{\pgfqpoint{0.000000in}{-0.027778in}}{\pgfqpoint{0.000000in}{0.000000in}}{%
\pgfpathmoveto{\pgfqpoint{0.000000in}{0.000000in}}%
\pgfpathlineto{\pgfqpoint{0.000000in}{-0.027778in}}%
\pgfusepath{stroke,fill}%
}%
\begin{pgfscope}%
\pgfsys@transformshift{5.204776in}{0.528000in}%
\pgfsys@useobject{currentmarker}{}%
\end{pgfscope}%
\end{pgfscope}%
\begin{pgfscope}%
\pgfsetbuttcap%
\pgfsetroundjoin%
\definecolor{currentfill}{rgb}{0.333333,0.333333,0.333333}%
\pgfsetfillcolor{currentfill}%
\pgfsetlinewidth{0.602250pt}%
\definecolor{currentstroke}{rgb}{0.333333,0.333333,0.333333}%
\pgfsetstrokecolor{currentstroke}%
\pgfsetdash{}{0pt}%
\pgfsys@defobject{currentmarker}{\pgfqpoint{0.000000in}{-0.027778in}}{\pgfqpoint{0.000000in}{0.000000in}}{%
\pgfpathmoveto{\pgfqpoint{0.000000in}{0.000000in}}%
\pgfpathlineto{\pgfqpoint{0.000000in}{-0.027778in}}%
\pgfusepath{stroke,fill}%
}%
\begin{pgfscope}%
\pgfsys@transformshift{5.241791in}{0.528000in}%
\pgfsys@useobject{currentmarker}{}%
\end{pgfscope}%
\end{pgfscope}%
\begin{pgfscope}%
\pgfsetbuttcap%
\pgfsetroundjoin%
\definecolor{currentfill}{rgb}{0.333333,0.333333,0.333333}%
\pgfsetfillcolor{currentfill}%
\pgfsetlinewidth{0.602250pt}%
\definecolor{currentstroke}{rgb}{0.333333,0.333333,0.333333}%
\pgfsetstrokecolor{currentstroke}%
\pgfsetdash{}{0pt}%
\pgfsys@defobject{currentmarker}{\pgfqpoint{0.000000in}{-0.027778in}}{\pgfqpoint{0.000000in}{0.000000in}}{%
\pgfpathmoveto{\pgfqpoint{0.000000in}{0.000000in}}%
\pgfpathlineto{\pgfqpoint{0.000000in}{-0.027778in}}%
\pgfusepath{stroke,fill}%
}%
\begin{pgfscope}%
\pgfsys@transformshift{5.278806in}{0.528000in}%
\pgfsys@useobject{currentmarker}{}%
\end{pgfscope}%
\end{pgfscope}%
\begin{pgfscope}%
\pgfsetbuttcap%
\pgfsetroundjoin%
\definecolor{currentfill}{rgb}{0.333333,0.333333,0.333333}%
\pgfsetfillcolor{currentfill}%
\pgfsetlinewidth{0.602250pt}%
\definecolor{currentstroke}{rgb}{0.333333,0.333333,0.333333}%
\pgfsetstrokecolor{currentstroke}%
\pgfsetdash{}{0pt}%
\pgfsys@defobject{currentmarker}{\pgfqpoint{0.000000in}{-0.027778in}}{\pgfqpoint{0.000000in}{0.000000in}}{%
\pgfpathmoveto{\pgfqpoint{0.000000in}{0.000000in}}%
\pgfpathlineto{\pgfqpoint{0.000000in}{-0.027778in}}%
\pgfusepath{stroke,fill}%
}%
\begin{pgfscope}%
\pgfsys@transformshift{5.315821in}{0.528000in}%
\pgfsys@useobject{currentmarker}{}%
\end{pgfscope}%
\end{pgfscope}%
\begin{pgfscope}%
\pgfsetbuttcap%
\pgfsetroundjoin%
\definecolor{currentfill}{rgb}{0.333333,0.333333,0.333333}%
\pgfsetfillcolor{currentfill}%
\pgfsetlinewidth{0.602250pt}%
\definecolor{currentstroke}{rgb}{0.333333,0.333333,0.333333}%
\pgfsetstrokecolor{currentstroke}%
\pgfsetdash{}{0pt}%
\pgfsys@defobject{currentmarker}{\pgfqpoint{0.000000in}{-0.027778in}}{\pgfqpoint{0.000000in}{0.000000in}}{%
\pgfpathmoveto{\pgfqpoint{0.000000in}{0.000000in}}%
\pgfpathlineto{\pgfqpoint{0.000000in}{-0.027778in}}%
\pgfusepath{stroke,fill}%
}%
\begin{pgfscope}%
\pgfsys@transformshift{5.352836in}{0.528000in}%
\pgfsys@useobject{currentmarker}{}%
\end{pgfscope}%
\end{pgfscope}%
\begin{pgfscope}%
\pgfsetbuttcap%
\pgfsetroundjoin%
\definecolor{currentfill}{rgb}{0.333333,0.333333,0.333333}%
\pgfsetfillcolor{currentfill}%
\pgfsetlinewidth{0.602250pt}%
\definecolor{currentstroke}{rgb}{0.333333,0.333333,0.333333}%
\pgfsetstrokecolor{currentstroke}%
\pgfsetdash{}{0pt}%
\pgfsys@defobject{currentmarker}{\pgfqpoint{0.000000in}{-0.027778in}}{\pgfqpoint{0.000000in}{0.000000in}}{%
\pgfpathmoveto{\pgfqpoint{0.000000in}{0.000000in}}%
\pgfpathlineto{\pgfqpoint{0.000000in}{-0.027778in}}%
\pgfusepath{stroke,fill}%
}%
\begin{pgfscope}%
\pgfsys@transformshift{5.389851in}{0.528000in}%
\pgfsys@useobject{currentmarker}{}%
\end{pgfscope}%
\end{pgfscope}%
\begin{pgfscope}%
\pgfsetbuttcap%
\pgfsetroundjoin%
\definecolor{currentfill}{rgb}{0.333333,0.333333,0.333333}%
\pgfsetfillcolor{currentfill}%
\pgfsetlinewidth{0.602250pt}%
\definecolor{currentstroke}{rgb}{0.333333,0.333333,0.333333}%
\pgfsetstrokecolor{currentstroke}%
\pgfsetdash{}{0pt}%
\pgfsys@defobject{currentmarker}{\pgfqpoint{0.000000in}{-0.027778in}}{\pgfqpoint{0.000000in}{0.000000in}}{%
\pgfpathmoveto{\pgfqpoint{0.000000in}{0.000000in}}%
\pgfpathlineto{\pgfqpoint{0.000000in}{-0.027778in}}%
\pgfusepath{stroke,fill}%
}%
\begin{pgfscope}%
\pgfsys@transformshift{5.426866in}{0.528000in}%
\pgfsys@useobject{currentmarker}{}%
\end{pgfscope}%
\end{pgfscope}%
\begin{pgfscope}%
\pgfsetbuttcap%
\pgfsetroundjoin%
\definecolor{currentfill}{rgb}{0.333333,0.333333,0.333333}%
\pgfsetfillcolor{currentfill}%
\pgfsetlinewidth{0.602250pt}%
\definecolor{currentstroke}{rgb}{0.333333,0.333333,0.333333}%
\pgfsetstrokecolor{currentstroke}%
\pgfsetdash{}{0pt}%
\pgfsys@defobject{currentmarker}{\pgfqpoint{0.000000in}{-0.027778in}}{\pgfqpoint{0.000000in}{0.000000in}}{%
\pgfpathmoveto{\pgfqpoint{0.000000in}{0.000000in}}%
\pgfpathlineto{\pgfqpoint{0.000000in}{-0.027778in}}%
\pgfusepath{stroke,fill}%
}%
\begin{pgfscope}%
\pgfsys@transformshift{5.463881in}{0.528000in}%
\pgfsys@useobject{currentmarker}{}%
\end{pgfscope}%
\end{pgfscope}%
\begin{pgfscope}%
\pgfsetbuttcap%
\pgfsetroundjoin%
\definecolor{currentfill}{rgb}{0.333333,0.333333,0.333333}%
\pgfsetfillcolor{currentfill}%
\pgfsetlinewidth{0.602250pt}%
\definecolor{currentstroke}{rgb}{0.333333,0.333333,0.333333}%
\pgfsetstrokecolor{currentstroke}%
\pgfsetdash{}{0pt}%
\pgfsys@defobject{currentmarker}{\pgfqpoint{0.000000in}{-0.027778in}}{\pgfqpoint{0.000000in}{0.000000in}}{%
\pgfpathmoveto{\pgfqpoint{0.000000in}{0.000000in}}%
\pgfpathlineto{\pgfqpoint{0.000000in}{-0.027778in}}%
\pgfusepath{stroke,fill}%
}%
\begin{pgfscope}%
\pgfsys@transformshift{5.500896in}{0.528000in}%
\pgfsys@useobject{currentmarker}{}%
\end{pgfscope}%
\end{pgfscope}%
\begin{pgfscope}%
\pgfsetbuttcap%
\pgfsetroundjoin%
\definecolor{currentfill}{rgb}{0.333333,0.333333,0.333333}%
\pgfsetfillcolor{currentfill}%
\pgfsetlinewidth{0.602250pt}%
\definecolor{currentstroke}{rgb}{0.333333,0.333333,0.333333}%
\pgfsetstrokecolor{currentstroke}%
\pgfsetdash{}{0pt}%
\pgfsys@defobject{currentmarker}{\pgfqpoint{0.000000in}{-0.027778in}}{\pgfqpoint{0.000000in}{0.000000in}}{%
\pgfpathmoveto{\pgfqpoint{0.000000in}{0.000000in}}%
\pgfpathlineto{\pgfqpoint{0.000000in}{-0.027778in}}%
\pgfusepath{stroke,fill}%
}%
\begin{pgfscope}%
\pgfsys@transformshift{5.537910in}{0.528000in}%
\pgfsys@useobject{currentmarker}{}%
\end{pgfscope}%
\end{pgfscope}%
\begin{pgfscope}%
\pgfsetbuttcap%
\pgfsetroundjoin%
\definecolor{currentfill}{rgb}{0.333333,0.333333,0.333333}%
\pgfsetfillcolor{currentfill}%
\pgfsetlinewidth{0.602250pt}%
\definecolor{currentstroke}{rgb}{0.333333,0.333333,0.333333}%
\pgfsetstrokecolor{currentstroke}%
\pgfsetdash{}{0pt}%
\pgfsys@defobject{currentmarker}{\pgfqpoint{0.000000in}{-0.027778in}}{\pgfqpoint{0.000000in}{0.000000in}}{%
\pgfpathmoveto{\pgfqpoint{0.000000in}{0.000000in}}%
\pgfpathlineto{\pgfqpoint{0.000000in}{-0.027778in}}%
\pgfusepath{stroke,fill}%
}%
\begin{pgfscope}%
\pgfsys@transformshift{5.574925in}{0.528000in}%
\pgfsys@useobject{currentmarker}{}%
\end{pgfscope}%
\end{pgfscope}%
\begin{pgfscope}%
\pgfsetbuttcap%
\pgfsetroundjoin%
\definecolor{currentfill}{rgb}{0.333333,0.333333,0.333333}%
\pgfsetfillcolor{currentfill}%
\pgfsetlinewidth{0.602250pt}%
\definecolor{currentstroke}{rgb}{0.333333,0.333333,0.333333}%
\pgfsetstrokecolor{currentstroke}%
\pgfsetdash{}{0pt}%
\pgfsys@defobject{currentmarker}{\pgfqpoint{0.000000in}{-0.027778in}}{\pgfqpoint{0.000000in}{0.000000in}}{%
\pgfpathmoveto{\pgfqpoint{0.000000in}{0.000000in}}%
\pgfpathlineto{\pgfqpoint{0.000000in}{-0.027778in}}%
\pgfusepath{stroke,fill}%
}%
\begin{pgfscope}%
\pgfsys@transformshift{5.611940in}{0.528000in}%
\pgfsys@useobject{currentmarker}{}%
\end{pgfscope}%
\end{pgfscope}%
\begin{pgfscope}%
\pgfsetbuttcap%
\pgfsetroundjoin%
\definecolor{currentfill}{rgb}{0.333333,0.333333,0.333333}%
\pgfsetfillcolor{currentfill}%
\pgfsetlinewidth{0.602250pt}%
\definecolor{currentstroke}{rgb}{0.333333,0.333333,0.333333}%
\pgfsetstrokecolor{currentstroke}%
\pgfsetdash{}{0pt}%
\pgfsys@defobject{currentmarker}{\pgfqpoint{0.000000in}{-0.027778in}}{\pgfqpoint{0.000000in}{0.000000in}}{%
\pgfpathmoveto{\pgfqpoint{0.000000in}{0.000000in}}%
\pgfpathlineto{\pgfqpoint{0.000000in}{-0.027778in}}%
\pgfusepath{stroke,fill}%
}%
\begin{pgfscope}%
\pgfsys@transformshift{5.648955in}{0.528000in}%
\pgfsys@useobject{currentmarker}{}%
\end{pgfscope}%
\end{pgfscope}%
\begin{pgfscope}%
\pgfsetbuttcap%
\pgfsetroundjoin%
\definecolor{currentfill}{rgb}{0.333333,0.333333,0.333333}%
\pgfsetfillcolor{currentfill}%
\pgfsetlinewidth{0.602250pt}%
\definecolor{currentstroke}{rgb}{0.333333,0.333333,0.333333}%
\pgfsetstrokecolor{currentstroke}%
\pgfsetdash{}{0pt}%
\pgfsys@defobject{currentmarker}{\pgfqpoint{0.000000in}{-0.027778in}}{\pgfqpoint{0.000000in}{0.000000in}}{%
\pgfpathmoveto{\pgfqpoint{0.000000in}{0.000000in}}%
\pgfpathlineto{\pgfqpoint{0.000000in}{-0.027778in}}%
\pgfusepath{stroke,fill}%
}%
\begin{pgfscope}%
\pgfsys@transformshift{5.685970in}{0.528000in}%
\pgfsys@useobject{currentmarker}{}%
\end{pgfscope}%
\end{pgfscope}%
\begin{pgfscope}%
\pgfsetbuttcap%
\pgfsetroundjoin%
\definecolor{currentfill}{rgb}{0.333333,0.333333,0.333333}%
\pgfsetfillcolor{currentfill}%
\pgfsetlinewidth{0.602250pt}%
\definecolor{currentstroke}{rgb}{0.333333,0.333333,0.333333}%
\pgfsetstrokecolor{currentstroke}%
\pgfsetdash{}{0pt}%
\pgfsys@defobject{currentmarker}{\pgfqpoint{0.000000in}{-0.027778in}}{\pgfqpoint{0.000000in}{0.000000in}}{%
\pgfpathmoveto{\pgfqpoint{0.000000in}{0.000000in}}%
\pgfpathlineto{\pgfqpoint{0.000000in}{-0.027778in}}%
\pgfusepath{stroke,fill}%
}%
\begin{pgfscope}%
\pgfsys@transformshift{5.722985in}{0.528000in}%
\pgfsys@useobject{currentmarker}{}%
\end{pgfscope}%
\end{pgfscope}%
\begin{pgfscope}%
\pgfsetbuttcap%
\pgfsetroundjoin%
\definecolor{currentfill}{rgb}{0.333333,0.333333,0.333333}%
\pgfsetfillcolor{currentfill}%
\pgfsetlinewidth{0.602250pt}%
\definecolor{currentstroke}{rgb}{0.333333,0.333333,0.333333}%
\pgfsetstrokecolor{currentstroke}%
\pgfsetdash{}{0pt}%
\pgfsys@defobject{currentmarker}{\pgfqpoint{0.000000in}{-0.027778in}}{\pgfqpoint{0.000000in}{0.000000in}}{%
\pgfpathmoveto{\pgfqpoint{0.000000in}{0.000000in}}%
\pgfpathlineto{\pgfqpoint{0.000000in}{-0.027778in}}%
\pgfusepath{stroke,fill}%
}%
\begin{pgfscope}%
\pgfsys@transformshift{5.760000in}{0.528000in}%
\pgfsys@useobject{currentmarker}{}%
\end{pgfscope}%
\end{pgfscope}%
\begin{pgfscope}%
\definecolor{textcolor}{rgb}{0.333333,0.333333,0.333333}%
\pgfsetstrokecolor{textcolor}%
\pgfsetfillcolor{textcolor}%
\pgftext[x=3.280000in,y=0.255624in,,top]{\color{textcolor}\sffamily\fontsize{12.000000}{14.400000}\selectfont Cantidad promedio vehículos en simulación}%
\end{pgfscope}%
\begin{pgfscope}%
\pgfpathrectangle{\pgfqpoint{0.800000in}{0.528000in}}{\pgfqpoint{4.960000in}{3.696000in}} %
\pgfusepath{clip}%
\pgfsetrectcap%
\pgfsetroundjoin%
\pgfsetlinewidth{0.803000pt}%
\definecolor{currentstroke}{rgb}{0.631373,0.631373,0.631373}%
\pgfsetstrokecolor{currentstroke}%
\pgfsetstrokeopacity{0.100000}%
\pgfsetdash{}{0pt}%
\pgfpathmoveto{\pgfqpoint{0.800000in}{0.528000in}}%
\pgfpathlineto{\pgfqpoint{5.760000in}{0.528000in}}%
\pgfusepath{stroke}%
\end{pgfscope}%
\begin{pgfscope}%
\pgfsetbuttcap%
\pgfsetroundjoin%
\definecolor{currentfill}{rgb}{0.333333,0.333333,0.333333}%
\pgfsetfillcolor{currentfill}%
\pgfsetlinewidth{0.803000pt}%
\definecolor{currentstroke}{rgb}{0.333333,0.333333,0.333333}%
\pgfsetstrokecolor{currentstroke}%
\pgfsetdash{}{0pt}%
\pgfsys@defobject{currentmarker}{\pgfqpoint{-0.048611in}{0.000000in}}{\pgfqpoint{0.000000in}{0.000000in}}{%
\pgfpathmoveto{\pgfqpoint{0.000000in}{0.000000in}}%
\pgfpathlineto{\pgfqpoint{-0.048611in}{0.000000in}}%
\pgfusepath{stroke,fill}%
}%
\begin{pgfscope}%
\pgfsys@transformshift{0.800000in}{0.528000in}%
\pgfsys@useobject{currentmarker}{}%
\end{pgfscope}%
\end{pgfscope}%
\begin{pgfscope}%
\definecolor{textcolor}{rgb}{0.333333,0.333333,0.333333}%
\pgfsetstrokecolor{textcolor}%
\pgfsetfillcolor{textcolor}%
\pgftext[x=0.370605in,y=0.479775in,left,base]{\color{textcolor}\sffamily\fontsize{10.000000}{12.000000}\selectfont 00:00}%
\end{pgfscope}%
\begin{pgfscope}%
\pgfpathrectangle{\pgfqpoint{0.800000in}{0.528000in}}{\pgfqpoint{4.960000in}{3.696000in}} %
\pgfusepath{clip}%
\pgfsetrectcap%
\pgfsetroundjoin%
\pgfsetlinewidth{0.803000pt}%
\definecolor{currentstroke}{rgb}{0.631373,0.631373,0.631373}%
\pgfsetstrokecolor{currentstroke}%
\pgfsetstrokeopacity{0.100000}%
\pgfsetdash{}{0pt}%
\pgfpathmoveto{\pgfqpoint{0.800000in}{0.816914in}}%
\pgfpathlineto{\pgfqpoint{5.760000in}{0.816914in}}%
\pgfusepath{stroke}%
\end{pgfscope}%
\begin{pgfscope}%
\pgfsetbuttcap%
\pgfsetroundjoin%
\definecolor{currentfill}{rgb}{0.333333,0.333333,0.333333}%
\pgfsetfillcolor{currentfill}%
\pgfsetlinewidth{0.803000pt}%
\definecolor{currentstroke}{rgb}{0.333333,0.333333,0.333333}%
\pgfsetstrokecolor{currentstroke}%
\pgfsetdash{}{0pt}%
\pgfsys@defobject{currentmarker}{\pgfqpoint{-0.048611in}{0.000000in}}{\pgfqpoint{0.000000in}{0.000000in}}{%
\pgfpathmoveto{\pgfqpoint{0.000000in}{0.000000in}}%
\pgfpathlineto{\pgfqpoint{-0.048611in}{0.000000in}}%
\pgfusepath{stroke,fill}%
}%
\begin{pgfscope}%
\pgfsys@transformshift{0.800000in}{0.816914in}%
\pgfsys@useobject{currentmarker}{}%
\end{pgfscope}%
\end{pgfscope}%
\begin{pgfscope}%
\definecolor{textcolor}{rgb}{0.333333,0.333333,0.333333}%
\pgfsetstrokecolor{textcolor}%
\pgfsetfillcolor{textcolor}%
\pgftext[x=0.370605in,y=0.768688in,left,base]{\color{textcolor}\sffamily\fontsize{10.000000}{12.000000}\selectfont 02:00}%
\end{pgfscope}%
\begin{pgfscope}%
\pgfpathrectangle{\pgfqpoint{0.800000in}{0.528000in}}{\pgfqpoint{4.960000in}{3.696000in}} %
\pgfusepath{clip}%
\pgfsetrectcap%
\pgfsetroundjoin%
\pgfsetlinewidth{0.803000pt}%
\definecolor{currentstroke}{rgb}{0.631373,0.631373,0.631373}%
\pgfsetstrokecolor{currentstroke}%
\pgfsetstrokeopacity{0.100000}%
\pgfsetdash{}{0pt}%
\pgfpathmoveto{\pgfqpoint{0.800000in}{1.105827in}}%
\pgfpathlineto{\pgfqpoint{5.760000in}{1.105827in}}%
\pgfusepath{stroke}%
\end{pgfscope}%
\begin{pgfscope}%
\pgfsetbuttcap%
\pgfsetroundjoin%
\definecolor{currentfill}{rgb}{0.333333,0.333333,0.333333}%
\pgfsetfillcolor{currentfill}%
\pgfsetlinewidth{0.803000pt}%
\definecolor{currentstroke}{rgb}{0.333333,0.333333,0.333333}%
\pgfsetstrokecolor{currentstroke}%
\pgfsetdash{}{0pt}%
\pgfsys@defobject{currentmarker}{\pgfqpoint{-0.048611in}{0.000000in}}{\pgfqpoint{0.000000in}{0.000000in}}{%
\pgfpathmoveto{\pgfqpoint{0.000000in}{0.000000in}}%
\pgfpathlineto{\pgfqpoint{-0.048611in}{0.000000in}}%
\pgfusepath{stroke,fill}%
}%
\begin{pgfscope}%
\pgfsys@transformshift{0.800000in}{1.105827in}%
\pgfsys@useobject{currentmarker}{}%
\end{pgfscope}%
\end{pgfscope}%
\begin{pgfscope}%
\definecolor{textcolor}{rgb}{0.333333,0.333333,0.333333}%
\pgfsetstrokecolor{textcolor}%
\pgfsetfillcolor{textcolor}%
\pgftext[x=0.370605in,y=1.057602in,left,base]{\color{textcolor}\sffamily\fontsize{10.000000}{12.000000}\selectfont 04:00}%
\end{pgfscope}%
\begin{pgfscope}%
\pgfpathrectangle{\pgfqpoint{0.800000in}{0.528000in}}{\pgfqpoint{4.960000in}{3.696000in}} %
\pgfusepath{clip}%
\pgfsetrectcap%
\pgfsetroundjoin%
\pgfsetlinewidth{0.803000pt}%
\definecolor{currentstroke}{rgb}{0.631373,0.631373,0.631373}%
\pgfsetstrokecolor{currentstroke}%
\pgfsetstrokeopacity{0.100000}%
\pgfsetdash{}{0pt}%
\pgfpathmoveto{\pgfqpoint{0.800000in}{1.394741in}}%
\pgfpathlineto{\pgfqpoint{5.760000in}{1.394741in}}%
\pgfusepath{stroke}%
\end{pgfscope}%
\begin{pgfscope}%
\pgfsetbuttcap%
\pgfsetroundjoin%
\definecolor{currentfill}{rgb}{0.333333,0.333333,0.333333}%
\pgfsetfillcolor{currentfill}%
\pgfsetlinewidth{0.803000pt}%
\definecolor{currentstroke}{rgb}{0.333333,0.333333,0.333333}%
\pgfsetstrokecolor{currentstroke}%
\pgfsetdash{}{0pt}%
\pgfsys@defobject{currentmarker}{\pgfqpoint{-0.048611in}{0.000000in}}{\pgfqpoint{0.000000in}{0.000000in}}{%
\pgfpathmoveto{\pgfqpoint{0.000000in}{0.000000in}}%
\pgfpathlineto{\pgfqpoint{-0.048611in}{0.000000in}}%
\pgfusepath{stroke,fill}%
}%
\begin{pgfscope}%
\pgfsys@transformshift{0.800000in}{1.394741in}%
\pgfsys@useobject{currentmarker}{}%
\end{pgfscope}%
\end{pgfscope}%
\begin{pgfscope}%
\definecolor{textcolor}{rgb}{0.333333,0.333333,0.333333}%
\pgfsetstrokecolor{textcolor}%
\pgfsetfillcolor{textcolor}%
\pgftext[x=0.370605in,y=1.346516in,left,base]{\color{textcolor}\sffamily\fontsize{10.000000}{12.000000}\selectfont 06:00}%
\end{pgfscope}%
\begin{pgfscope}%
\pgfpathrectangle{\pgfqpoint{0.800000in}{0.528000in}}{\pgfqpoint{4.960000in}{3.696000in}} %
\pgfusepath{clip}%
\pgfsetrectcap%
\pgfsetroundjoin%
\pgfsetlinewidth{0.803000pt}%
\definecolor{currentstroke}{rgb}{0.631373,0.631373,0.631373}%
\pgfsetstrokecolor{currentstroke}%
\pgfsetstrokeopacity{0.100000}%
\pgfsetdash{}{0pt}%
\pgfpathmoveto{\pgfqpoint{0.800000in}{1.683655in}}%
\pgfpathlineto{\pgfqpoint{5.760000in}{1.683655in}}%
\pgfusepath{stroke}%
\end{pgfscope}%
\begin{pgfscope}%
\pgfsetbuttcap%
\pgfsetroundjoin%
\definecolor{currentfill}{rgb}{0.333333,0.333333,0.333333}%
\pgfsetfillcolor{currentfill}%
\pgfsetlinewidth{0.803000pt}%
\definecolor{currentstroke}{rgb}{0.333333,0.333333,0.333333}%
\pgfsetstrokecolor{currentstroke}%
\pgfsetdash{}{0pt}%
\pgfsys@defobject{currentmarker}{\pgfqpoint{-0.048611in}{0.000000in}}{\pgfqpoint{0.000000in}{0.000000in}}{%
\pgfpathmoveto{\pgfqpoint{0.000000in}{0.000000in}}%
\pgfpathlineto{\pgfqpoint{-0.048611in}{0.000000in}}%
\pgfusepath{stroke,fill}%
}%
\begin{pgfscope}%
\pgfsys@transformshift{0.800000in}{1.683655in}%
\pgfsys@useobject{currentmarker}{}%
\end{pgfscope}%
\end{pgfscope}%
\begin{pgfscope}%
\definecolor{textcolor}{rgb}{0.333333,0.333333,0.333333}%
\pgfsetstrokecolor{textcolor}%
\pgfsetfillcolor{textcolor}%
\pgftext[x=0.370605in,y=1.635429in,left,base]{\color{textcolor}\sffamily\fontsize{10.000000}{12.000000}\selectfont 08:00}%
\end{pgfscope}%
\begin{pgfscope}%
\pgfpathrectangle{\pgfqpoint{0.800000in}{0.528000in}}{\pgfqpoint{4.960000in}{3.696000in}} %
\pgfusepath{clip}%
\pgfsetrectcap%
\pgfsetroundjoin%
\pgfsetlinewidth{0.803000pt}%
\definecolor{currentstroke}{rgb}{0.631373,0.631373,0.631373}%
\pgfsetstrokecolor{currentstroke}%
\pgfsetstrokeopacity{0.100000}%
\pgfsetdash{}{0pt}%
\pgfpathmoveto{\pgfqpoint{0.800000in}{1.972568in}}%
\pgfpathlineto{\pgfqpoint{5.760000in}{1.972568in}}%
\pgfusepath{stroke}%
\end{pgfscope}%
\begin{pgfscope}%
\pgfsetbuttcap%
\pgfsetroundjoin%
\definecolor{currentfill}{rgb}{0.333333,0.333333,0.333333}%
\pgfsetfillcolor{currentfill}%
\pgfsetlinewidth{0.803000pt}%
\definecolor{currentstroke}{rgb}{0.333333,0.333333,0.333333}%
\pgfsetstrokecolor{currentstroke}%
\pgfsetdash{}{0pt}%
\pgfsys@defobject{currentmarker}{\pgfqpoint{-0.048611in}{0.000000in}}{\pgfqpoint{0.000000in}{0.000000in}}{%
\pgfpathmoveto{\pgfqpoint{0.000000in}{0.000000in}}%
\pgfpathlineto{\pgfqpoint{-0.048611in}{0.000000in}}%
\pgfusepath{stroke,fill}%
}%
\begin{pgfscope}%
\pgfsys@transformshift{0.800000in}{1.972568in}%
\pgfsys@useobject{currentmarker}{}%
\end{pgfscope}%
\end{pgfscope}%
\begin{pgfscope}%
\definecolor{textcolor}{rgb}{0.333333,0.333333,0.333333}%
\pgfsetstrokecolor{textcolor}%
\pgfsetfillcolor{textcolor}%
\pgftext[x=0.370605in,y=1.924343in,left,base]{\color{textcolor}\sffamily\fontsize{10.000000}{12.000000}\selectfont 10:00}%
\end{pgfscope}%
\begin{pgfscope}%
\pgfpathrectangle{\pgfqpoint{0.800000in}{0.528000in}}{\pgfqpoint{4.960000in}{3.696000in}} %
\pgfusepath{clip}%
\pgfsetrectcap%
\pgfsetroundjoin%
\pgfsetlinewidth{0.803000pt}%
\definecolor{currentstroke}{rgb}{0.631373,0.631373,0.631373}%
\pgfsetstrokecolor{currentstroke}%
\pgfsetstrokeopacity{0.100000}%
\pgfsetdash{}{0pt}%
\pgfpathmoveto{\pgfqpoint{0.800000in}{2.261482in}}%
\pgfpathlineto{\pgfqpoint{5.760000in}{2.261482in}}%
\pgfusepath{stroke}%
\end{pgfscope}%
\begin{pgfscope}%
\pgfsetbuttcap%
\pgfsetroundjoin%
\definecolor{currentfill}{rgb}{0.333333,0.333333,0.333333}%
\pgfsetfillcolor{currentfill}%
\pgfsetlinewidth{0.803000pt}%
\definecolor{currentstroke}{rgb}{0.333333,0.333333,0.333333}%
\pgfsetstrokecolor{currentstroke}%
\pgfsetdash{}{0pt}%
\pgfsys@defobject{currentmarker}{\pgfqpoint{-0.048611in}{0.000000in}}{\pgfqpoint{0.000000in}{0.000000in}}{%
\pgfpathmoveto{\pgfqpoint{0.000000in}{0.000000in}}%
\pgfpathlineto{\pgfqpoint{-0.048611in}{0.000000in}}%
\pgfusepath{stroke,fill}%
}%
\begin{pgfscope}%
\pgfsys@transformshift{0.800000in}{2.261482in}%
\pgfsys@useobject{currentmarker}{}%
\end{pgfscope}%
\end{pgfscope}%
\begin{pgfscope}%
\definecolor{textcolor}{rgb}{0.333333,0.333333,0.333333}%
\pgfsetstrokecolor{textcolor}%
\pgfsetfillcolor{textcolor}%
\pgftext[x=0.370605in,y=2.213257in,left,base]{\color{textcolor}\sffamily\fontsize{10.000000}{12.000000}\selectfont 12:00}%
\end{pgfscope}%
\begin{pgfscope}%
\pgfpathrectangle{\pgfqpoint{0.800000in}{0.528000in}}{\pgfqpoint{4.960000in}{3.696000in}} %
\pgfusepath{clip}%
\pgfsetrectcap%
\pgfsetroundjoin%
\pgfsetlinewidth{0.803000pt}%
\definecolor{currentstroke}{rgb}{0.631373,0.631373,0.631373}%
\pgfsetstrokecolor{currentstroke}%
\pgfsetstrokeopacity{0.100000}%
\pgfsetdash{}{0pt}%
\pgfpathmoveto{\pgfqpoint{0.800000in}{2.550395in}}%
\pgfpathlineto{\pgfqpoint{5.760000in}{2.550395in}}%
\pgfusepath{stroke}%
\end{pgfscope}%
\begin{pgfscope}%
\pgfsetbuttcap%
\pgfsetroundjoin%
\definecolor{currentfill}{rgb}{0.333333,0.333333,0.333333}%
\pgfsetfillcolor{currentfill}%
\pgfsetlinewidth{0.803000pt}%
\definecolor{currentstroke}{rgb}{0.333333,0.333333,0.333333}%
\pgfsetstrokecolor{currentstroke}%
\pgfsetdash{}{0pt}%
\pgfsys@defobject{currentmarker}{\pgfqpoint{-0.048611in}{0.000000in}}{\pgfqpoint{0.000000in}{0.000000in}}{%
\pgfpathmoveto{\pgfqpoint{0.000000in}{0.000000in}}%
\pgfpathlineto{\pgfqpoint{-0.048611in}{0.000000in}}%
\pgfusepath{stroke,fill}%
}%
\begin{pgfscope}%
\pgfsys@transformshift{0.800000in}{2.550395in}%
\pgfsys@useobject{currentmarker}{}%
\end{pgfscope}%
\end{pgfscope}%
\begin{pgfscope}%
\definecolor{textcolor}{rgb}{0.333333,0.333333,0.333333}%
\pgfsetstrokecolor{textcolor}%
\pgfsetfillcolor{textcolor}%
\pgftext[x=0.370605in,y=2.502170in,left,base]{\color{textcolor}\sffamily\fontsize{10.000000}{12.000000}\selectfont 14:00}%
\end{pgfscope}%
\begin{pgfscope}%
\pgfpathrectangle{\pgfqpoint{0.800000in}{0.528000in}}{\pgfqpoint{4.960000in}{3.696000in}} %
\pgfusepath{clip}%
\pgfsetrectcap%
\pgfsetroundjoin%
\pgfsetlinewidth{0.803000pt}%
\definecolor{currentstroke}{rgb}{0.631373,0.631373,0.631373}%
\pgfsetstrokecolor{currentstroke}%
\pgfsetstrokeopacity{0.100000}%
\pgfsetdash{}{0pt}%
\pgfpathmoveto{\pgfqpoint{0.800000in}{2.839309in}}%
\pgfpathlineto{\pgfqpoint{5.760000in}{2.839309in}}%
\pgfusepath{stroke}%
\end{pgfscope}%
\begin{pgfscope}%
\pgfsetbuttcap%
\pgfsetroundjoin%
\definecolor{currentfill}{rgb}{0.333333,0.333333,0.333333}%
\pgfsetfillcolor{currentfill}%
\pgfsetlinewidth{0.803000pt}%
\definecolor{currentstroke}{rgb}{0.333333,0.333333,0.333333}%
\pgfsetstrokecolor{currentstroke}%
\pgfsetdash{}{0pt}%
\pgfsys@defobject{currentmarker}{\pgfqpoint{-0.048611in}{0.000000in}}{\pgfqpoint{0.000000in}{0.000000in}}{%
\pgfpathmoveto{\pgfqpoint{0.000000in}{0.000000in}}%
\pgfpathlineto{\pgfqpoint{-0.048611in}{0.000000in}}%
\pgfusepath{stroke,fill}%
}%
\begin{pgfscope}%
\pgfsys@transformshift{0.800000in}{2.839309in}%
\pgfsys@useobject{currentmarker}{}%
\end{pgfscope}%
\end{pgfscope}%
\begin{pgfscope}%
\definecolor{textcolor}{rgb}{0.333333,0.333333,0.333333}%
\pgfsetstrokecolor{textcolor}%
\pgfsetfillcolor{textcolor}%
\pgftext[x=0.370605in,y=2.791084in,left,base]{\color{textcolor}\sffamily\fontsize{10.000000}{12.000000}\selectfont 16:00}%
\end{pgfscope}%
\begin{pgfscope}%
\pgfpathrectangle{\pgfqpoint{0.800000in}{0.528000in}}{\pgfqpoint{4.960000in}{3.696000in}} %
\pgfusepath{clip}%
\pgfsetrectcap%
\pgfsetroundjoin%
\pgfsetlinewidth{0.803000pt}%
\definecolor{currentstroke}{rgb}{0.631373,0.631373,0.631373}%
\pgfsetstrokecolor{currentstroke}%
\pgfsetstrokeopacity{0.100000}%
\pgfsetdash{}{0pt}%
\pgfpathmoveto{\pgfqpoint{0.800000in}{3.128223in}}%
\pgfpathlineto{\pgfqpoint{5.760000in}{3.128223in}}%
\pgfusepath{stroke}%
\end{pgfscope}%
\begin{pgfscope}%
\pgfsetbuttcap%
\pgfsetroundjoin%
\definecolor{currentfill}{rgb}{0.333333,0.333333,0.333333}%
\pgfsetfillcolor{currentfill}%
\pgfsetlinewidth{0.803000pt}%
\definecolor{currentstroke}{rgb}{0.333333,0.333333,0.333333}%
\pgfsetstrokecolor{currentstroke}%
\pgfsetdash{}{0pt}%
\pgfsys@defobject{currentmarker}{\pgfqpoint{-0.048611in}{0.000000in}}{\pgfqpoint{0.000000in}{0.000000in}}{%
\pgfpathmoveto{\pgfqpoint{0.000000in}{0.000000in}}%
\pgfpathlineto{\pgfqpoint{-0.048611in}{0.000000in}}%
\pgfusepath{stroke,fill}%
}%
\begin{pgfscope}%
\pgfsys@transformshift{0.800000in}{3.128223in}%
\pgfsys@useobject{currentmarker}{}%
\end{pgfscope}%
\end{pgfscope}%
\begin{pgfscope}%
\definecolor{textcolor}{rgb}{0.333333,0.333333,0.333333}%
\pgfsetstrokecolor{textcolor}%
\pgfsetfillcolor{textcolor}%
\pgftext[x=0.370605in,y=3.079998in,left,base]{\color{textcolor}\sffamily\fontsize{10.000000}{12.000000}\selectfont 18:00}%
\end{pgfscope}%
\begin{pgfscope}%
\pgfpathrectangle{\pgfqpoint{0.800000in}{0.528000in}}{\pgfqpoint{4.960000in}{3.696000in}} %
\pgfusepath{clip}%
\pgfsetrectcap%
\pgfsetroundjoin%
\pgfsetlinewidth{0.803000pt}%
\definecolor{currentstroke}{rgb}{0.631373,0.631373,0.631373}%
\pgfsetstrokecolor{currentstroke}%
\pgfsetstrokeopacity{0.100000}%
\pgfsetdash{}{0pt}%
\pgfpathmoveto{\pgfqpoint{0.800000in}{3.417136in}}%
\pgfpathlineto{\pgfqpoint{5.760000in}{3.417136in}}%
\pgfusepath{stroke}%
\end{pgfscope}%
\begin{pgfscope}%
\pgfsetbuttcap%
\pgfsetroundjoin%
\definecolor{currentfill}{rgb}{0.333333,0.333333,0.333333}%
\pgfsetfillcolor{currentfill}%
\pgfsetlinewidth{0.803000pt}%
\definecolor{currentstroke}{rgb}{0.333333,0.333333,0.333333}%
\pgfsetstrokecolor{currentstroke}%
\pgfsetdash{}{0pt}%
\pgfsys@defobject{currentmarker}{\pgfqpoint{-0.048611in}{0.000000in}}{\pgfqpoint{0.000000in}{0.000000in}}{%
\pgfpathmoveto{\pgfqpoint{0.000000in}{0.000000in}}%
\pgfpathlineto{\pgfqpoint{-0.048611in}{0.000000in}}%
\pgfusepath{stroke,fill}%
}%
\begin{pgfscope}%
\pgfsys@transformshift{0.800000in}{3.417136in}%
\pgfsys@useobject{currentmarker}{}%
\end{pgfscope}%
\end{pgfscope}%
\begin{pgfscope}%
\definecolor{textcolor}{rgb}{0.333333,0.333333,0.333333}%
\pgfsetstrokecolor{textcolor}%
\pgfsetfillcolor{textcolor}%
\pgftext[x=0.370605in,y=3.368911in,left,base]{\color{textcolor}\sffamily\fontsize{10.000000}{12.000000}\selectfont 20:00}%
\end{pgfscope}%
\begin{pgfscope}%
\pgfpathrectangle{\pgfqpoint{0.800000in}{0.528000in}}{\pgfqpoint{4.960000in}{3.696000in}} %
\pgfusepath{clip}%
\pgfsetrectcap%
\pgfsetroundjoin%
\pgfsetlinewidth{0.803000pt}%
\definecolor{currentstroke}{rgb}{0.631373,0.631373,0.631373}%
\pgfsetstrokecolor{currentstroke}%
\pgfsetstrokeopacity{0.100000}%
\pgfsetdash{}{0pt}%
\pgfpathmoveto{\pgfqpoint{0.800000in}{3.706050in}}%
\pgfpathlineto{\pgfqpoint{5.760000in}{3.706050in}}%
\pgfusepath{stroke}%
\end{pgfscope}%
\begin{pgfscope}%
\pgfsetbuttcap%
\pgfsetroundjoin%
\definecolor{currentfill}{rgb}{0.333333,0.333333,0.333333}%
\pgfsetfillcolor{currentfill}%
\pgfsetlinewidth{0.803000pt}%
\definecolor{currentstroke}{rgb}{0.333333,0.333333,0.333333}%
\pgfsetstrokecolor{currentstroke}%
\pgfsetdash{}{0pt}%
\pgfsys@defobject{currentmarker}{\pgfqpoint{-0.048611in}{0.000000in}}{\pgfqpoint{0.000000in}{0.000000in}}{%
\pgfpathmoveto{\pgfqpoint{0.000000in}{0.000000in}}%
\pgfpathlineto{\pgfqpoint{-0.048611in}{0.000000in}}%
\pgfusepath{stroke,fill}%
}%
\begin{pgfscope}%
\pgfsys@transformshift{0.800000in}{3.706050in}%
\pgfsys@useobject{currentmarker}{}%
\end{pgfscope}%
\end{pgfscope}%
\begin{pgfscope}%
\definecolor{textcolor}{rgb}{0.333333,0.333333,0.333333}%
\pgfsetstrokecolor{textcolor}%
\pgfsetfillcolor{textcolor}%
\pgftext[x=0.370605in,y=3.657825in,left,base]{\color{textcolor}\sffamily\fontsize{10.000000}{12.000000}\selectfont 22:00}%
\end{pgfscope}%
\begin{pgfscope}%
\pgfpathrectangle{\pgfqpoint{0.800000in}{0.528000in}}{\pgfqpoint{4.960000in}{3.696000in}} %
\pgfusepath{clip}%
\pgfsetrectcap%
\pgfsetroundjoin%
\pgfsetlinewidth{0.803000pt}%
\definecolor{currentstroke}{rgb}{0.631373,0.631373,0.631373}%
\pgfsetstrokecolor{currentstroke}%
\pgfsetstrokeopacity{0.100000}%
\pgfsetdash{}{0pt}%
\pgfpathmoveto{\pgfqpoint{0.800000in}{3.994964in}}%
\pgfpathlineto{\pgfqpoint{5.760000in}{3.994964in}}%
\pgfusepath{stroke}%
\end{pgfscope}%
\begin{pgfscope}%
\pgfsetbuttcap%
\pgfsetroundjoin%
\definecolor{currentfill}{rgb}{0.333333,0.333333,0.333333}%
\pgfsetfillcolor{currentfill}%
\pgfsetlinewidth{0.803000pt}%
\definecolor{currentstroke}{rgb}{0.333333,0.333333,0.333333}%
\pgfsetstrokecolor{currentstroke}%
\pgfsetdash{}{0pt}%
\pgfsys@defobject{currentmarker}{\pgfqpoint{-0.048611in}{0.000000in}}{\pgfqpoint{0.000000in}{0.000000in}}{%
\pgfpathmoveto{\pgfqpoint{0.000000in}{0.000000in}}%
\pgfpathlineto{\pgfqpoint{-0.048611in}{0.000000in}}%
\pgfusepath{stroke,fill}%
}%
\begin{pgfscope}%
\pgfsys@transformshift{0.800000in}{3.994964in}%
\pgfsys@useobject{currentmarker}{}%
\end{pgfscope}%
\end{pgfscope}%
\begin{pgfscope}%
\definecolor{textcolor}{rgb}{0.333333,0.333333,0.333333}%
\pgfsetstrokecolor{textcolor}%
\pgfsetfillcolor{textcolor}%
\pgftext[x=0.370605in,y=3.946738in,left,base]{\color{textcolor}\sffamily\fontsize{10.000000}{12.000000}\selectfont 24:00}%
\end{pgfscope}%
\begin{pgfscope}%
\pgfsetbuttcap%
\pgfsetroundjoin%
\definecolor{currentfill}{rgb}{0.333333,0.333333,0.333333}%
\pgfsetfillcolor{currentfill}%
\pgfsetlinewidth{0.602250pt}%
\definecolor{currentstroke}{rgb}{0.333333,0.333333,0.333333}%
\pgfsetstrokecolor{currentstroke}%
\pgfsetdash{}{0pt}%
\pgfsys@defobject{currentmarker}{\pgfqpoint{-0.027778in}{0.000000in}}{\pgfqpoint{0.000000in}{0.000000in}}{%
\pgfpathmoveto{\pgfqpoint{0.000000in}{0.000000in}}%
\pgfpathlineto{\pgfqpoint{-0.027778in}{0.000000in}}%
\pgfusepath{stroke,fill}%
}%
\begin{pgfscope}%
\pgfsys@transformshift{0.800000in}{0.528000in}%
\pgfsys@useobject{currentmarker}{}%
\end{pgfscope}%
\end{pgfscope}%
\begin{pgfscope}%
\pgfsetbuttcap%
\pgfsetroundjoin%
\definecolor{currentfill}{rgb}{0.333333,0.333333,0.333333}%
\pgfsetfillcolor{currentfill}%
\pgfsetlinewidth{0.602250pt}%
\definecolor{currentstroke}{rgb}{0.333333,0.333333,0.333333}%
\pgfsetstrokecolor{currentstroke}%
\pgfsetdash{}{0pt}%
\pgfsys@defobject{currentmarker}{\pgfqpoint{-0.027778in}{0.000000in}}{\pgfqpoint{0.000000in}{0.000000in}}{%
\pgfpathmoveto{\pgfqpoint{0.000000in}{0.000000in}}%
\pgfpathlineto{\pgfqpoint{-0.027778in}{0.000000in}}%
\pgfusepath{stroke,fill}%
}%
\begin{pgfscope}%
\pgfsys@transformshift{0.800000in}{0.672457in}%
\pgfsys@useobject{currentmarker}{}%
\end{pgfscope}%
\end{pgfscope}%
\begin{pgfscope}%
\pgfsetbuttcap%
\pgfsetroundjoin%
\definecolor{currentfill}{rgb}{0.333333,0.333333,0.333333}%
\pgfsetfillcolor{currentfill}%
\pgfsetlinewidth{0.602250pt}%
\definecolor{currentstroke}{rgb}{0.333333,0.333333,0.333333}%
\pgfsetstrokecolor{currentstroke}%
\pgfsetdash{}{0pt}%
\pgfsys@defobject{currentmarker}{\pgfqpoint{-0.027778in}{0.000000in}}{\pgfqpoint{0.000000in}{0.000000in}}{%
\pgfpathmoveto{\pgfqpoint{0.000000in}{0.000000in}}%
\pgfpathlineto{\pgfqpoint{-0.027778in}{0.000000in}}%
\pgfusepath{stroke,fill}%
}%
\begin{pgfscope}%
\pgfsys@transformshift{0.800000in}{0.816914in}%
\pgfsys@useobject{currentmarker}{}%
\end{pgfscope}%
\end{pgfscope}%
\begin{pgfscope}%
\pgfsetbuttcap%
\pgfsetroundjoin%
\definecolor{currentfill}{rgb}{0.333333,0.333333,0.333333}%
\pgfsetfillcolor{currentfill}%
\pgfsetlinewidth{0.602250pt}%
\definecolor{currentstroke}{rgb}{0.333333,0.333333,0.333333}%
\pgfsetstrokecolor{currentstroke}%
\pgfsetdash{}{0pt}%
\pgfsys@defobject{currentmarker}{\pgfqpoint{-0.027778in}{0.000000in}}{\pgfqpoint{0.000000in}{0.000000in}}{%
\pgfpathmoveto{\pgfqpoint{0.000000in}{0.000000in}}%
\pgfpathlineto{\pgfqpoint{-0.027778in}{0.000000in}}%
\pgfusepath{stroke,fill}%
}%
\begin{pgfscope}%
\pgfsys@transformshift{0.800000in}{0.961370in}%
\pgfsys@useobject{currentmarker}{}%
\end{pgfscope}%
\end{pgfscope}%
\begin{pgfscope}%
\pgfsetbuttcap%
\pgfsetroundjoin%
\definecolor{currentfill}{rgb}{0.333333,0.333333,0.333333}%
\pgfsetfillcolor{currentfill}%
\pgfsetlinewidth{0.602250pt}%
\definecolor{currentstroke}{rgb}{0.333333,0.333333,0.333333}%
\pgfsetstrokecolor{currentstroke}%
\pgfsetdash{}{0pt}%
\pgfsys@defobject{currentmarker}{\pgfqpoint{-0.027778in}{0.000000in}}{\pgfqpoint{0.000000in}{0.000000in}}{%
\pgfpathmoveto{\pgfqpoint{0.000000in}{0.000000in}}%
\pgfpathlineto{\pgfqpoint{-0.027778in}{0.000000in}}%
\pgfusepath{stroke,fill}%
}%
\begin{pgfscope}%
\pgfsys@transformshift{0.800000in}{1.105827in}%
\pgfsys@useobject{currentmarker}{}%
\end{pgfscope}%
\end{pgfscope}%
\begin{pgfscope}%
\pgfsetbuttcap%
\pgfsetroundjoin%
\definecolor{currentfill}{rgb}{0.333333,0.333333,0.333333}%
\pgfsetfillcolor{currentfill}%
\pgfsetlinewidth{0.602250pt}%
\definecolor{currentstroke}{rgb}{0.333333,0.333333,0.333333}%
\pgfsetstrokecolor{currentstroke}%
\pgfsetdash{}{0pt}%
\pgfsys@defobject{currentmarker}{\pgfqpoint{-0.027778in}{0.000000in}}{\pgfqpoint{0.000000in}{0.000000in}}{%
\pgfpathmoveto{\pgfqpoint{0.000000in}{0.000000in}}%
\pgfpathlineto{\pgfqpoint{-0.027778in}{0.000000in}}%
\pgfusepath{stroke,fill}%
}%
\begin{pgfscope}%
\pgfsys@transformshift{0.800000in}{1.250284in}%
\pgfsys@useobject{currentmarker}{}%
\end{pgfscope}%
\end{pgfscope}%
\begin{pgfscope}%
\pgfsetbuttcap%
\pgfsetroundjoin%
\definecolor{currentfill}{rgb}{0.333333,0.333333,0.333333}%
\pgfsetfillcolor{currentfill}%
\pgfsetlinewidth{0.602250pt}%
\definecolor{currentstroke}{rgb}{0.333333,0.333333,0.333333}%
\pgfsetstrokecolor{currentstroke}%
\pgfsetdash{}{0pt}%
\pgfsys@defobject{currentmarker}{\pgfqpoint{-0.027778in}{0.000000in}}{\pgfqpoint{0.000000in}{0.000000in}}{%
\pgfpathmoveto{\pgfqpoint{0.000000in}{0.000000in}}%
\pgfpathlineto{\pgfqpoint{-0.027778in}{0.000000in}}%
\pgfusepath{stroke,fill}%
}%
\begin{pgfscope}%
\pgfsys@transformshift{0.800000in}{1.394741in}%
\pgfsys@useobject{currentmarker}{}%
\end{pgfscope}%
\end{pgfscope}%
\begin{pgfscope}%
\pgfsetbuttcap%
\pgfsetroundjoin%
\definecolor{currentfill}{rgb}{0.333333,0.333333,0.333333}%
\pgfsetfillcolor{currentfill}%
\pgfsetlinewidth{0.602250pt}%
\definecolor{currentstroke}{rgb}{0.333333,0.333333,0.333333}%
\pgfsetstrokecolor{currentstroke}%
\pgfsetdash{}{0pt}%
\pgfsys@defobject{currentmarker}{\pgfqpoint{-0.027778in}{0.000000in}}{\pgfqpoint{0.000000in}{0.000000in}}{%
\pgfpathmoveto{\pgfqpoint{0.000000in}{0.000000in}}%
\pgfpathlineto{\pgfqpoint{-0.027778in}{0.000000in}}%
\pgfusepath{stroke,fill}%
}%
\begin{pgfscope}%
\pgfsys@transformshift{0.800000in}{1.539198in}%
\pgfsys@useobject{currentmarker}{}%
\end{pgfscope}%
\end{pgfscope}%
\begin{pgfscope}%
\pgfsetbuttcap%
\pgfsetroundjoin%
\definecolor{currentfill}{rgb}{0.333333,0.333333,0.333333}%
\pgfsetfillcolor{currentfill}%
\pgfsetlinewidth{0.602250pt}%
\definecolor{currentstroke}{rgb}{0.333333,0.333333,0.333333}%
\pgfsetstrokecolor{currentstroke}%
\pgfsetdash{}{0pt}%
\pgfsys@defobject{currentmarker}{\pgfqpoint{-0.027778in}{0.000000in}}{\pgfqpoint{0.000000in}{0.000000in}}{%
\pgfpathmoveto{\pgfqpoint{0.000000in}{0.000000in}}%
\pgfpathlineto{\pgfqpoint{-0.027778in}{0.000000in}}%
\pgfusepath{stroke,fill}%
}%
\begin{pgfscope}%
\pgfsys@transformshift{0.800000in}{1.683655in}%
\pgfsys@useobject{currentmarker}{}%
\end{pgfscope}%
\end{pgfscope}%
\begin{pgfscope}%
\pgfsetbuttcap%
\pgfsetroundjoin%
\definecolor{currentfill}{rgb}{0.333333,0.333333,0.333333}%
\pgfsetfillcolor{currentfill}%
\pgfsetlinewidth{0.602250pt}%
\definecolor{currentstroke}{rgb}{0.333333,0.333333,0.333333}%
\pgfsetstrokecolor{currentstroke}%
\pgfsetdash{}{0pt}%
\pgfsys@defobject{currentmarker}{\pgfqpoint{-0.027778in}{0.000000in}}{\pgfqpoint{0.000000in}{0.000000in}}{%
\pgfpathmoveto{\pgfqpoint{0.000000in}{0.000000in}}%
\pgfpathlineto{\pgfqpoint{-0.027778in}{0.000000in}}%
\pgfusepath{stroke,fill}%
}%
\begin{pgfscope}%
\pgfsys@transformshift{0.800000in}{1.828111in}%
\pgfsys@useobject{currentmarker}{}%
\end{pgfscope}%
\end{pgfscope}%
\begin{pgfscope}%
\pgfsetbuttcap%
\pgfsetroundjoin%
\definecolor{currentfill}{rgb}{0.333333,0.333333,0.333333}%
\pgfsetfillcolor{currentfill}%
\pgfsetlinewidth{0.602250pt}%
\definecolor{currentstroke}{rgb}{0.333333,0.333333,0.333333}%
\pgfsetstrokecolor{currentstroke}%
\pgfsetdash{}{0pt}%
\pgfsys@defobject{currentmarker}{\pgfqpoint{-0.027778in}{0.000000in}}{\pgfqpoint{0.000000in}{0.000000in}}{%
\pgfpathmoveto{\pgfqpoint{0.000000in}{0.000000in}}%
\pgfpathlineto{\pgfqpoint{-0.027778in}{0.000000in}}%
\pgfusepath{stroke,fill}%
}%
\begin{pgfscope}%
\pgfsys@transformshift{0.800000in}{1.972568in}%
\pgfsys@useobject{currentmarker}{}%
\end{pgfscope}%
\end{pgfscope}%
\begin{pgfscope}%
\pgfsetbuttcap%
\pgfsetroundjoin%
\definecolor{currentfill}{rgb}{0.333333,0.333333,0.333333}%
\pgfsetfillcolor{currentfill}%
\pgfsetlinewidth{0.602250pt}%
\definecolor{currentstroke}{rgb}{0.333333,0.333333,0.333333}%
\pgfsetstrokecolor{currentstroke}%
\pgfsetdash{}{0pt}%
\pgfsys@defobject{currentmarker}{\pgfqpoint{-0.027778in}{0.000000in}}{\pgfqpoint{0.000000in}{0.000000in}}{%
\pgfpathmoveto{\pgfqpoint{0.000000in}{0.000000in}}%
\pgfpathlineto{\pgfqpoint{-0.027778in}{0.000000in}}%
\pgfusepath{stroke,fill}%
}%
\begin{pgfscope}%
\pgfsys@transformshift{0.800000in}{2.117025in}%
\pgfsys@useobject{currentmarker}{}%
\end{pgfscope}%
\end{pgfscope}%
\begin{pgfscope}%
\pgfsetbuttcap%
\pgfsetroundjoin%
\definecolor{currentfill}{rgb}{0.333333,0.333333,0.333333}%
\pgfsetfillcolor{currentfill}%
\pgfsetlinewidth{0.602250pt}%
\definecolor{currentstroke}{rgb}{0.333333,0.333333,0.333333}%
\pgfsetstrokecolor{currentstroke}%
\pgfsetdash{}{0pt}%
\pgfsys@defobject{currentmarker}{\pgfqpoint{-0.027778in}{0.000000in}}{\pgfqpoint{0.000000in}{0.000000in}}{%
\pgfpathmoveto{\pgfqpoint{0.000000in}{0.000000in}}%
\pgfpathlineto{\pgfqpoint{-0.027778in}{0.000000in}}%
\pgfusepath{stroke,fill}%
}%
\begin{pgfscope}%
\pgfsys@transformshift{0.800000in}{2.261482in}%
\pgfsys@useobject{currentmarker}{}%
\end{pgfscope}%
\end{pgfscope}%
\begin{pgfscope}%
\pgfsetbuttcap%
\pgfsetroundjoin%
\definecolor{currentfill}{rgb}{0.333333,0.333333,0.333333}%
\pgfsetfillcolor{currentfill}%
\pgfsetlinewidth{0.602250pt}%
\definecolor{currentstroke}{rgb}{0.333333,0.333333,0.333333}%
\pgfsetstrokecolor{currentstroke}%
\pgfsetdash{}{0pt}%
\pgfsys@defobject{currentmarker}{\pgfqpoint{-0.027778in}{0.000000in}}{\pgfqpoint{0.000000in}{0.000000in}}{%
\pgfpathmoveto{\pgfqpoint{0.000000in}{0.000000in}}%
\pgfpathlineto{\pgfqpoint{-0.027778in}{0.000000in}}%
\pgfusepath{stroke,fill}%
}%
\begin{pgfscope}%
\pgfsys@transformshift{0.800000in}{2.405939in}%
\pgfsys@useobject{currentmarker}{}%
\end{pgfscope}%
\end{pgfscope}%
\begin{pgfscope}%
\pgfsetbuttcap%
\pgfsetroundjoin%
\definecolor{currentfill}{rgb}{0.333333,0.333333,0.333333}%
\pgfsetfillcolor{currentfill}%
\pgfsetlinewidth{0.602250pt}%
\definecolor{currentstroke}{rgb}{0.333333,0.333333,0.333333}%
\pgfsetstrokecolor{currentstroke}%
\pgfsetdash{}{0pt}%
\pgfsys@defobject{currentmarker}{\pgfqpoint{-0.027778in}{0.000000in}}{\pgfqpoint{0.000000in}{0.000000in}}{%
\pgfpathmoveto{\pgfqpoint{0.000000in}{0.000000in}}%
\pgfpathlineto{\pgfqpoint{-0.027778in}{0.000000in}}%
\pgfusepath{stroke,fill}%
}%
\begin{pgfscope}%
\pgfsys@transformshift{0.800000in}{2.550395in}%
\pgfsys@useobject{currentmarker}{}%
\end{pgfscope}%
\end{pgfscope}%
\begin{pgfscope}%
\pgfsetbuttcap%
\pgfsetroundjoin%
\definecolor{currentfill}{rgb}{0.333333,0.333333,0.333333}%
\pgfsetfillcolor{currentfill}%
\pgfsetlinewidth{0.602250pt}%
\definecolor{currentstroke}{rgb}{0.333333,0.333333,0.333333}%
\pgfsetstrokecolor{currentstroke}%
\pgfsetdash{}{0pt}%
\pgfsys@defobject{currentmarker}{\pgfqpoint{-0.027778in}{0.000000in}}{\pgfqpoint{0.000000in}{0.000000in}}{%
\pgfpathmoveto{\pgfqpoint{0.000000in}{0.000000in}}%
\pgfpathlineto{\pgfqpoint{-0.027778in}{0.000000in}}%
\pgfusepath{stroke,fill}%
}%
\begin{pgfscope}%
\pgfsys@transformshift{0.800000in}{2.694852in}%
\pgfsys@useobject{currentmarker}{}%
\end{pgfscope}%
\end{pgfscope}%
\begin{pgfscope}%
\pgfsetbuttcap%
\pgfsetroundjoin%
\definecolor{currentfill}{rgb}{0.333333,0.333333,0.333333}%
\pgfsetfillcolor{currentfill}%
\pgfsetlinewidth{0.602250pt}%
\definecolor{currentstroke}{rgb}{0.333333,0.333333,0.333333}%
\pgfsetstrokecolor{currentstroke}%
\pgfsetdash{}{0pt}%
\pgfsys@defobject{currentmarker}{\pgfqpoint{-0.027778in}{0.000000in}}{\pgfqpoint{0.000000in}{0.000000in}}{%
\pgfpathmoveto{\pgfqpoint{0.000000in}{0.000000in}}%
\pgfpathlineto{\pgfqpoint{-0.027778in}{0.000000in}}%
\pgfusepath{stroke,fill}%
}%
\begin{pgfscope}%
\pgfsys@transformshift{0.800000in}{2.839309in}%
\pgfsys@useobject{currentmarker}{}%
\end{pgfscope}%
\end{pgfscope}%
\begin{pgfscope}%
\pgfsetbuttcap%
\pgfsetroundjoin%
\definecolor{currentfill}{rgb}{0.333333,0.333333,0.333333}%
\pgfsetfillcolor{currentfill}%
\pgfsetlinewidth{0.602250pt}%
\definecolor{currentstroke}{rgb}{0.333333,0.333333,0.333333}%
\pgfsetstrokecolor{currentstroke}%
\pgfsetdash{}{0pt}%
\pgfsys@defobject{currentmarker}{\pgfqpoint{-0.027778in}{0.000000in}}{\pgfqpoint{0.000000in}{0.000000in}}{%
\pgfpathmoveto{\pgfqpoint{0.000000in}{0.000000in}}%
\pgfpathlineto{\pgfqpoint{-0.027778in}{0.000000in}}%
\pgfusepath{stroke,fill}%
}%
\begin{pgfscope}%
\pgfsys@transformshift{0.800000in}{2.983766in}%
\pgfsys@useobject{currentmarker}{}%
\end{pgfscope}%
\end{pgfscope}%
\begin{pgfscope}%
\pgfsetbuttcap%
\pgfsetroundjoin%
\definecolor{currentfill}{rgb}{0.333333,0.333333,0.333333}%
\pgfsetfillcolor{currentfill}%
\pgfsetlinewidth{0.602250pt}%
\definecolor{currentstroke}{rgb}{0.333333,0.333333,0.333333}%
\pgfsetstrokecolor{currentstroke}%
\pgfsetdash{}{0pt}%
\pgfsys@defobject{currentmarker}{\pgfqpoint{-0.027778in}{0.000000in}}{\pgfqpoint{0.000000in}{0.000000in}}{%
\pgfpathmoveto{\pgfqpoint{0.000000in}{0.000000in}}%
\pgfpathlineto{\pgfqpoint{-0.027778in}{0.000000in}}%
\pgfusepath{stroke,fill}%
}%
\begin{pgfscope}%
\pgfsys@transformshift{0.800000in}{3.128223in}%
\pgfsys@useobject{currentmarker}{}%
\end{pgfscope}%
\end{pgfscope}%
\begin{pgfscope}%
\pgfsetbuttcap%
\pgfsetroundjoin%
\definecolor{currentfill}{rgb}{0.333333,0.333333,0.333333}%
\pgfsetfillcolor{currentfill}%
\pgfsetlinewidth{0.602250pt}%
\definecolor{currentstroke}{rgb}{0.333333,0.333333,0.333333}%
\pgfsetstrokecolor{currentstroke}%
\pgfsetdash{}{0pt}%
\pgfsys@defobject{currentmarker}{\pgfqpoint{-0.027778in}{0.000000in}}{\pgfqpoint{0.000000in}{0.000000in}}{%
\pgfpathmoveto{\pgfqpoint{0.000000in}{0.000000in}}%
\pgfpathlineto{\pgfqpoint{-0.027778in}{0.000000in}}%
\pgfusepath{stroke,fill}%
}%
\begin{pgfscope}%
\pgfsys@transformshift{0.800000in}{3.272680in}%
\pgfsys@useobject{currentmarker}{}%
\end{pgfscope}%
\end{pgfscope}%
\begin{pgfscope}%
\pgfsetbuttcap%
\pgfsetroundjoin%
\definecolor{currentfill}{rgb}{0.333333,0.333333,0.333333}%
\pgfsetfillcolor{currentfill}%
\pgfsetlinewidth{0.602250pt}%
\definecolor{currentstroke}{rgb}{0.333333,0.333333,0.333333}%
\pgfsetstrokecolor{currentstroke}%
\pgfsetdash{}{0pt}%
\pgfsys@defobject{currentmarker}{\pgfqpoint{-0.027778in}{0.000000in}}{\pgfqpoint{0.000000in}{0.000000in}}{%
\pgfpathmoveto{\pgfqpoint{0.000000in}{0.000000in}}%
\pgfpathlineto{\pgfqpoint{-0.027778in}{0.000000in}}%
\pgfusepath{stroke,fill}%
}%
\begin{pgfscope}%
\pgfsys@transformshift{0.800000in}{3.417136in}%
\pgfsys@useobject{currentmarker}{}%
\end{pgfscope}%
\end{pgfscope}%
\begin{pgfscope}%
\pgfsetbuttcap%
\pgfsetroundjoin%
\definecolor{currentfill}{rgb}{0.333333,0.333333,0.333333}%
\pgfsetfillcolor{currentfill}%
\pgfsetlinewidth{0.602250pt}%
\definecolor{currentstroke}{rgb}{0.333333,0.333333,0.333333}%
\pgfsetstrokecolor{currentstroke}%
\pgfsetdash{}{0pt}%
\pgfsys@defobject{currentmarker}{\pgfqpoint{-0.027778in}{0.000000in}}{\pgfqpoint{0.000000in}{0.000000in}}{%
\pgfpathmoveto{\pgfqpoint{0.000000in}{0.000000in}}%
\pgfpathlineto{\pgfqpoint{-0.027778in}{0.000000in}}%
\pgfusepath{stroke,fill}%
}%
\begin{pgfscope}%
\pgfsys@transformshift{0.800000in}{3.561593in}%
\pgfsys@useobject{currentmarker}{}%
\end{pgfscope}%
\end{pgfscope}%
\begin{pgfscope}%
\pgfsetbuttcap%
\pgfsetroundjoin%
\definecolor{currentfill}{rgb}{0.333333,0.333333,0.333333}%
\pgfsetfillcolor{currentfill}%
\pgfsetlinewidth{0.602250pt}%
\definecolor{currentstroke}{rgb}{0.333333,0.333333,0.333333}%
\pgfsetstrokecolor{currentstroke}%
\pgfsetdash{}{0pt}%
\pgfsys@defobject{currentmarker}{\pgfqpoint{-0.027778in}{0.000000in}}{\pgfqpoint{0.000000in}{0.000000in}}{%
\pgfpathmoveto{\pgfqpoint{0.000000in}{0.000000in}}%
\pgfpathlineto{\pgfqpoint{-0.027778in}{0.000000in}}%
\pgfusepath{stroke,fill}%
}%
\begin{pgfscope}%
\pgfsys@transformshift{0.800000in}{3.706050in}%
\pgfsys@useobject{currentmarker}{}%
\end{pgfscope}%
\end{pgfscope}%
\begin{pgfscope}%
\pgfsetbuttcap%
\pgfsetroundjoin%
\definecolor{currentfill}{rgb}{0.333333,0.333333,0.333333}%
\pgfsetfillcolor{currentfill}%
\pgfsetlinewidth{0.602250pt}%
\definecolor{currentstroke}{rgb}{0.333333,0.333333,0.333333}%
\pgfsetstrokecolor{currentstroke}%
\pgfsetdash{}{0pt}%
\pgfsys@defobject{currentmarker}{\pgfqpoint{-0.027778in}{0.000000in}}{\pgfqpoint{0.000000in}{0.000000in}}{%
\pgfpathmoveto{\pgfqpoint{0.000000in}{0.000000in}}%
\pgfpathlineto{\pgfqpoint{-0.027778in}{0.000000in}}%
\pgfusepath{stroke,fill}%
}%
\begin{pgfscope}%
\pgfsys@transformshift{0.800000in}{3.850507in}%
\pgfsys@useobject{currentmarker}{}%
\end{pgfscope}%
\end{pgfscope}%
\begin{pgfscope}%
\pgfsetbuttcap%
\pgfsetroundjoin%
\definecolor{currentfill}{rgb}{0.333333,0.333333,0.333333}%
\pgfsetfillcolor{currentfill}%
\pgfsetlinewidth{0.602250pt}%
\definecolor{currentstroke}{rgb}{0.333333,0.333333,0.333333}%
\pgfsetstrokecolor{currentstroke}%
\pgfsetdash{}{0pt}%
\pgfsys@defobject{currentmarker}{\pgfqpoint{-0.027778in}{0.000000in}}{\pgfqpoint{0.000000in}{0.000000in}}{%
\pgfpathmoveto{\pgfqpoint{0.000000in}{0.000000in}}%
\pgfpathlineto{\pgfqpoint{-0.027778in}{0.000000in}}%
\pgfusepath{stroke,fill}%
}%
\begin{pgfscope}%
\pgfsys@transformshift{0.800000in}{3.994964in}%
\pgfsys@useobject{currentmarker}{}%
\end{pgfscope}%
\end{pgfscope}%
\begin{pgfscope}%
\pgfsetbuttcap%
\pgfsetroundjoin%
\definecolor{currentfill}{rgb}{0.333333,0.333333,0.333333}%
\pgfsetfillcolor{currentfill}%
\pgfsetlinewidth{0.602250pt}%
\definecolor{currentstroke}{rgb}{0.333333,0.333333,0.333333}%
\pgfsetstrokecolor{currentstroke}%
\pgfsetdash{}{0pt}%
\pgfsys@defobject{currentmarker}{\pgfqpoint{-0.027778in}{0.000000in}}{\pgfqpoint{0.000000in}{0.000000in}}{%
\pgfpathmoveto{\pgfqpoint{0.000000in}{0.000000in}}%
\pgfpathlineto{\pgfqpoint{-0.027778in}{0.000000in}}%
\pgfusepath{stroke,fill}%
}%
\begin{pgfscope}%
\pgfsys@transformshift{0.800000in}{4.139421in}%
\pgfsys@useobject{currentmarker}{}%
\end{pgfscope}%
\end{pgfscope}%
\begin{pgfscope}%
\definecolor{textcolor}{rgb}{0.333333,0.333333,0.333333}%
\pgfsetstrokecolor{textcolor}%
\pgfsetfillcolor{textcolor}%
\pgftext[x=0.315049in,y=2.376000in,,bottom,rotate=90.000000]{\color{textcolor}\sffamily\fontsize{12.000000}{14.400000}\selectfont Tiempo (MM:SS)}%
\end{pgfscope}%
\begin{pgfscope}%
\pgfpathrectangle{\pgfqpoint{0.800000in}{0.528000in}}{\pgfqpoint{4.960000in}{3.696000in}} %
\pgfusepath{clip}%
\pgfsetbuttcap%
\pgfsetroundjoin%
\pgfsetlinewidth{1.505625pt}%
\definecolor{currentstroke}{rgb}{0.941176,0.392157,0.286275}%
\pgfsetstrokecolor{currentstroke}%
\pgfsetstrokeopacity{0.300000}%
\pgfsetdash{{9.600000pt}{2.400000pt}{1.500000pt}{2.400000pt}}{0.000000pt}%
\pgfpathmoveto{\pgfqpoint{0.796299in}{0.886734in}}%
\pgfpathlineto{\pgfqpoint{5.763701in}{4.117752in}}%
\pgfpathlineto{\pgfqpoint{5.763701in}{4.117752in}}%
\pgfusepath{stroke}%
\end{pgfscope}%
\begin{pgfscope}%
\pgfpathrectangle{\pgfqpoint{0.800000in}{0.528000in}}{\pgfqpoint{4.960000in}{3.696000in}} %
\pgfusepath{clip}%
\pgfsetbuttcap%
\pgfsetroundjoin%
\definecolor{currentfill}{rgb}{0.090196,0.745098,0.733333}%
\pgfsetfillcolor{currentfill}%
\pgfsetlinewidth{1.003750pt}%
\definecolor{currentstroke}{rgb}{0.090196,0.745098,0.733333}%
\pgfsetstrokecolor{currentstroke}%
\pgfsetdash{}{0pt}%
\pgfsys@defobject{currentmarker}{\pgfqpoint{-0.041667in}{-0.041667in}}{\pgfqpoint{0.041667in}{0.041667in}}{%
\pgfpathmoveto{\pgfqpoint{0.000000in}{-0.041667in}}%
\pgfpathcurveto{\pgfqpoint{0.011050in}{-0.041667in}}{\pgfqpoint{0.021649in}{-0.037276in}}{\pgfqpoint{0.029463in}{-0.029463in}}%
\pgfpathcurveto{\pgfqpoint{0.037276in}{-0.021649in}}{\pgfqpoint{0.041667in}{-0.011050in}}{\pgfqpoint{0.041667in}{0.000000in}}%
\pgfpathcurveto{\pgfqpoint{0.041667in}{0.011050in}}{\pgfqpoint{0.037276in}{0.021649in}}{\pgfqpoint{0.029463in}{0.029463in}}%
\pgfpathcurveto{\pgfqpoint{0.021649in}{0.037276in}}{\pgfqpoint{0.011050in}{0.041667in}}{\pgfqpoint{0.000000in}{0.041667in}}%
\pgfpathcurveto{\pgfqpoint{-0.011050in}{0.041667in}}{\pgfqpoint{-0.021649in}{0.037276in}}{\pgfqpoint{-0.029463in}{0.029463in}}%
\pgfpathcurveto{\pgfqpoint{-0.037276in}{0.021649in}}{\pgfqpoint{-0.041667in}{0.011050in}}{\pgfqpoint{-0.041667in}{0.000000in}}%
\pgfpathcurveto{\pgfqpoint{-0.041667in}{-0.011050in}}{\pgfqpoint{-0.037276in}{-0.021649in}}{\pgfqpoint{-0.029463in}{-0.029463in}}%
\pgfpathcurveto{\pgfqpoint{-0.021649in}{-0.037276in}}{\pgfqpoint{-0.011050in}{-0.041667in}}{\pgfqpoint{0.000000in}{-0.041667in}}%
\pgfpathclose%
\pgfusepath{stroke,fill}%
}%
\begin{pgfscope}%
\pgfsys@transformshift{5.420943in}{3.894807in}%
\pgfsys@useobject{currentmarker}{}%
\end{pgfscope}%
\begin{pgfscope}%
\pgfsys@transformshift{5.246973in}{3.781649in}%
\pgfsys@useobject{currentmarker}{}%
\end{pgfscope}%
\begin{pgfscope}%
\pgfsys@transformshift{5.259077in}{3.789522in}%
\pgfsys@useobject{currentmarker}{}%
\end{pgfscope}%
\begin{pgfscope}%
\pgfsys@transformshift{5.482869in}{3.935086in}%
\pgfsys@useobject{currentmarker}{}%
\end{pgfscope}%
\end{pgfscope}%
\begin{pgfscope}%
\pgfpathrectangle{\pgfqpoint{0.800000in}{0.528000in}}{\pgfqpoint{4.960000in}{3.696000in}} %
\pgfusepath{clip}%
\pgfsetbuttcap%
\pgfsetroundjoin%
\definecolor{currentfill}{rgb}{0.937255,0.176471,0.337255}%
\pgfsetfillcolor{currentfill}%
\pgfsetlinewidth{1.003750pt}%
\definecolor{currentstroke}{rgb}{0.937255,0.176471,0.337255}%
\pgfsetstrokecolor{currentstroke}%
\pgfsetdash{}{0pt}%
\pgfsys@defobject{currentmarker}{\pgfqpoint{-0.041667in}{-0.041667in}}{\pgfqpoint{0.041667in}{0.041667in}}{%
\pgfpathmoveto{\pgfqpoint{0.000000in}{-0.041667in}}%
\pgfpathcurveto{\pgfqpoint{0.011050in}{-0.041667in}}{\pgfqpoint{0.021649in}{-0.037276in}}{\pgfqpoint{0.029463in}{-0.029463in}}%
\pgfpathcurveto{\pgfqpoint{0.037276in}{-0.021649in}}{\pgfqpoint{0.041667in}{-0.011050in}}{\pgfqpoint{0.041667in}{0.000000in}}%
\pgfpathcurveto{\pgfqpoint{0.041667in}{0.011050in}}{\pgfqpoint{0.037276in}{0.021649in}}{\pgfqpoint{0.029463in}{0.029463in}}%
\pgfpathcurveto{\pgfqpoint{0.021649in}{0.037276in}}{\pgfqpoint{0.011050in}{0.041667in}}{\pgfqpoint{0.000000in}{0.041667in}}%
\pgfpathcurveto{\pgfqpoint{-0.011050in}{0.041667in}}{\pgfqpoint{-0.021649in}{0.037276in}}{\pgfqpoint{-0.029463in}{0.029463in}}%
\pgfpathcurveto{\pgfqpoint{-0.037276in}{0.021649in}}{\pgfqpoint{-0.041667in}{0.011050in}}{\pgfqpoint{-0.041667in}{0.000000in}}%
\pgfpathcurveto{\pgfqpoint{-0.041667in}{-0.011050in}}{\pgfqpoint{-0.037276in}{-0.021649in}}{\pgfqpoint{-0.029463in}{-0.029463in}}%
\pgfpathcurveto{\pgfqpoint{-0.021649in}{-0.037276in}}{\pgfqpoint{-0.011050in}{-0.041667in}}{\pgfqpoint{0.000000in}{-0.041667in}}%
\pgfpathclose%
\pgfusepath{stroke,fill}%
}%
\begin{pgfscope}%
\pgfsys@transformshift{3.465815in}{2.623105in}%
\pgfsys@useobject{currentmarker}{}%
\end{pgfscope}%
\begin{pgfscope}%
\pgfsys@transformshift{3.473477in}{2.628089in}%
\pgfsys@useobject{currentmarker}{}%
\end{pgfscope}%
\begin{pgfscope}%
\pgfsys@transformshift{3.361914in}{2.555524in}%
\pgfsys@useobject{currentmarker}{}%
\end{pgfscope}%
\begin{pgfscope}%
\pgfsys@transformshift{3.540585in}{2.671739in}%
\pgfsys@useobject{currentmarker}{}%
\end{pgfscope}%
\end{pgfscope}%
\begin{pgfscope}%
\pgfpathrectangle{\pgfqpoint{0.800000in}{0.528000in}}{\pgfqpoint{4.960000in}{3.696000in}} %
\pgfusepath{clip}%
\pgfsetbuttcap%
\pgfsetroundjoin%
\definecolor{currentfill}{rgb}{0.549020,0.847059,0.403922}%
\pgfsetfillcolor{currentfill}%
\pgfsetlinewidth{1.003750pt}%
\definecolor{currentstroke}{rgb}{0.549020,0.847059,0.403922}%
\pgfsetstrokecolor{currentstroke}%
\pgfsetdash{}{0pt}%
\pgfsys@defobject{currentmarker}{\pgfqpoint{-0.041667in}{-0.041667in}}{\pgfqpoint{0.041667in}{0.041667in}}{%
\pgfpathmoveto{\pgfqpoint{0.000000in}{-0.041667in}}%
\pgfpathcurveto{\pgfqpoint{0.011050in}{-0.041667in}}{\pgfqpoint{0.021649in}{-0.037276in}}{\pgfqpoint{0.029463in}{-0.029463in}}%
\pgfpathcurveto{\pgfqpoint{0.037276in}{-0.021649in}}{\pgfqpoint{0.041667in}{-0.011050in}}{\pgfqpoint{0.041667in}{0.000000in}}%
\pgfpathcurveto{\pgfqpoint{0.041667in}{0.011050in}}{\pgfqpoint{0.037276in}{0.021649in}}{\pgfqpoint{0.029463in}{0.029463in}}%
\pgfpathcurveto{\pgfqpoint{0.021649in}{0.037276in}}{\pgfqpoint{0.011050in}{0.041667in}}{\pgfqpoint{0.000000in}{0.041667in}}%
\pgfpathcurveto{\pgfqpoint{-0.011050in}{0.041667in}}{\pgfqpoint{-0.021649in}{0.037276in}}{\pgfqpoint{-0.029463in}{0.029463in}}%
\pgfpathcurveto{\pgfqpoint{-0.037276in}{0.021649in}}{\pgfqpoint{-0.041667in}{0.011050in}}{\pgfqpoint{-0.041667in}{0.000000in}}%
\pgfpathcurveto{\pgfqpoint{-0.041667in}{-0.011050in}}{\pgfqpoint{-0.037276in}{-0.021649in}}{\pgfqpoint{-0.029463in}{-0.029463in}}%
\pgfpathcurveto{\pgfqpoint{-0.021649in}{-0.037276in}}{\pgfqpoint{-0.011050in}{-0.041667in}}{\pgfqpoint{0.000000in}{-0.041667in}}%
\pgfpathclose%
\pgfusepath{stroke,fill}%
}%
\begin{pgfscope}%
\pgfsys@transformshift{2.173735in}{1.782680in}%
\pgfsys@useobject{currentmarker}{}%
\end{pgfscope}%
\begin{pgfscope}%
\pgfsys@transformshift{2.111809in}{1.742400in}%
\pgfsys@useobject{currentmarker}{}%
\end{pgfscope}%
\begin{pgfscope}%
\pgfsys@transformshift{2.140681in}{1.761180in}%
\pgfsys@useobject{currentmarker}{}%
\end{pgfscope}%
\begin{pgfscope}%
\pgfsys@transformshift{2.171773in}{1.781404in}%
\pgfsys@useobject{currentmarker}{}%
\end{pgfscope}%
\end{pgfscope}%
\begin{pgfscope}%
\pgfpathrectangle{\pgfqpoint{0.800000in}{0.528000in}}{\pgfqpoint{4.960000in}{3.696000in}} %
\pgfusepath{clip}%
\pgfsetbuttcap%
\pgfsetroundjoin%
\definecolor{currentfill}{rgb}{0.184314,0.141176,0.227451}%
\pgfsetfillcolor{currentfill}%
\pgfsetlinewidth{1.003750pt}%
\definecolor{currentstroke}{rgb}{0.184314,0.141176,0.227451}%
\pgfsetstrokecolor{currentstroke}%
\pgfsetdash{}{0pt}%
\pgfsys@defobject{currentmarker}{\pgfqpoint{-0.041667in}{-0.041667in}}{\pgfqpoint{0.041667in}{0.041667in}}{%
\pgfpathmoveto{\pgfqpoint{0.000000in}{-0.041667in}}%
\pgfpathcurveto{\pgfqpoint{0.011050in}{-0.041667in}}{\pgfqpoint{0.021649in}{-0.037276in}}{\pgfqpoint{0.029463in}{-0.029463in}}%
\pgfpathcurveto{\pgfqpoint{0.037276in}{-0.021649in}}{\pgfqpoint{0.041667in}{-0.011050in}}{\pgfqpoint{0.041667in}{0.000000in}}%
\pgfpathcurveto{\pgfqpoint{0.041667in}{0.011050in}}{\pgfqpoint{0.037276in}{0.021649in}}{\pgfqpoint{0.029463in}{0.029463in}}%
\pgfpathcurveto{\pgfqpoint{0.021649in}{0.037276in}}{\pgfqpoint{0.011050in}{0.041667in}}{\pgfqpoint{0.000000in}{0.041667in}}%
\pgfpathcurveto{\pgfqpoint{-0.011050in}{0.041667in}}{\pgfqpoint{-0.021649in}{0.037276in}}{\pgfqpoint{-0.029463in}{0.029463in}}%
\pgfpathcurveto{\pgfqpoint{-0.037276in}{0.021649in}}{\pgfqpoint{-0.041667in}{0.011050in}}{\pgfqpoint{-0.041667in}{0.000000in}}%
\pgfpathcurveto{\pgfqpoint{-0.041667in}{-0.011050in}}{\pgfqpoint{-0.037276in}{-0.021649in}}{\pgfqpoint{-0.029463in}{-0.029463in}}%
\pgfpathcurveto{\pgfqpoint{-0.021649in}{-0.037276in}}{\pgfqpoint{-0.011050in}{-0.041667in}}{\pgfqpoint{0.000000in}{-0.041667in}}%
\pgfpathclose%
\pgfusepath{stroke,fill}%
}%
\begin{pgfscope}%
\pgfsys@transformshift{1.168928in}{1.129109in}%
\pgfsys@useobject{currentmarker}{}%
\end{pgfscope}%
\begin{pgfscope}%
\pgfsys@transformshift{1.187916in}{1.141460in}%
\pgfsys@useobject{currentmarker}{}%
\end{pgfscope}%
\begin{pgfscope}%
\pgfsys@transformshift{1.136355in}{1.107922in}%
\pgfsys@useobject{currentmarker}{}%
\end{pgfscope}%
\begin{pgfscope}%
\pgfsys@transformshift{1.136577in}{1.108066in}%
\pgfsys@useobject{currentmarker}{}%
\end{pgfscope}%
\end{pgfscope}%
\begin{pgfscope}%
\pgfsetrectcap%
\pgfsetmiterjoin%
\pgfsetlinewidth{1.003750pt}%
\definecolor{currentstroke}{rgb}{1.000000,1.000000,1.000000}%
\pgfsetstrokecolor{currentstroke}%
\pgfsetdash{}{0pt}%
\pgfpathmoveto{\pgfqpoint{0.800000in}{0.528000in}}%
\pgfpathlineto{\pgfqpoint{0.800000in}{4.224000in}}%
\pgfusepath{stroke}%
\end{pgfscope}%
\begin{pgfscope}%
\pgfsetrectcap%
\pgfsetmiterjoin%
\pgfsetlinewidth{1.003750pt}%
\definecolor{currentstroke}{rgb}{1.000000,1.000000,1.000000}%
\pgfsetstrokecolor{currentstroke}%
\pgfsetdash{}{0pt}%
\pgfpathmoveto{\pgfqpoint{5.760000in}{0.528000in}}%
\pgfpathlineto{\pgfqpoint{5.760000in}{4.224000in}}%
\pgfusepath{stroke}%
\end{pgfscope}%
\begin{pgfscope}%
\pgfsetrectcap%
\pgfsetmiterjoin%
\pgfsetlinewidth{1.003750pt}%
\definecolor{currentstroke}{rgb}{1.000000,1.000000,1.000000}%
\pgfsetstrokecolor{currentstroke}%
\pgfsetdash{}{0pt}%
\pgfpathmoveto{\pgfqpoint{0.800000in}{0.528000in}}%
\pgfpathlineto{\pgfqpoint{5.760000in}{0.528000in}}%
\pgfusepath{stroke}%
\end{pgfscope}%
\begin{pgfscope}%
\pgfsetrectcap%
\pgfsetmiterjoin%
\pgfsetlinewidth{1.003750pt}%
\definecolor{currentstroke}{rgb}{1.000000,1.000000,1.000000}%
\pgfsetstrokecolor{currentstroke}%
\pgfsetdash{}{0pt}%
\pgfpathmoveto{\pgfqpoint{0.800000in}{4.224000in}}%
\pgfpathlineto{\pgfqpoint{5.760000in}{4.224000in}}%
\pgfusepath{stroke}%
\end{pgfscope}%
\begin{pgfscope}%
\pgfsetbuttcap%
\pgfsetmiterjoin%
\definecolor{currentfill}{rgb}{0.898039,0.898039,0.898039}%
\pgfsetfillcolor{currentfill}%
\pgfsetfillopacity{0.800000}%
\pgfsetlinewidth{0.501875pt}%
\definecolor{currentstroke}{rgb}{0.800000,0.800000,0.800000}%
\pgfsetstrokecolor{currentstroke}%
\pgfsetstrokeopacity{0.800000}%
\pgfsetdash{}{0pt}%
\pgfpathmoveto{\pgfqpoint{0.897222in}{3.136038in}}%
\pgfpathlineto{\pgfqpoint{2.073604in}{3.136038in}}%
\pgfpathquadraticcurveto{\pgfqpoint{2.101382in}{3.136038in}}{\pgfqpoint{2.101382in}{3.163815in}}%
\pgfpathlineto{\pgfqpoint{2.101382in}{4.126778in}}%
\pgfpathquadraticcurveto{\pgfqpoint{2.101382in}{4.154556in}}{\pgfqpoint{2.073604in}{4.154556in}}%
\pgfpathlineto{\pgfqpoint{0.897222in}{4.154556in}}%
\pgfpathquadraticcurveto{\pgfqpoint{0.869444in}{4.154556in}}{\pgfqpoint{0.869444in}{4.126778in}}%
\pgfpathlineto{\pgfqpoint{0.869444in}{3.163815in}}%
\pgfpathquadraticcurveto{\pgfqpoint{0.869444in}{3.136038in}}{\pgfqpoint{0.897222in}{3.136038in}}%
\pgfpathclose%
\pgfusepath{stroke,fill}%
\end{pgfscope}%
\begin{pgfscope}%
\pgfsetbuttcap%
\pgfsetroundjoin%
\pgfsetlinewidth{1.505625pt}%
\definecolor{currentstroke}{rgb}{0.941176,0.392157,0.286275}%
\pgfsetstrokecolor{currentstroke}%
\pgfsetstrokeopacity{0.300000}%
\pgfsetdash{{9.600000pt}{2.400000pt}{1.500000pt}{2.400000pt}}{0.000000pt}%
\pgfpathmoveto{\pgfqpoint{0.925000in}{4.050389in}}%
\pgfpathlineto{\pgfqpoint{1.202778in}{4.050389in}}%
\pgfusepath{stroke}%
\end{pgfscope}%
\begin{pgfscope}%
\pgftext[x=1.313889in,y=4.001778in,left,base]{\sffamily\fontsize{10.000000}{12.000000}\selectfont Ajuste lineal}%
\end{pgfscope}%
\begin{pgfscope}%
\pgfsetbuttcap%
\pgfsetroundjoin%
\definecolor{currentfill}{rgb}{0.090196,0.745098,0.733333}%
\pgfsetfillcolor{currentfill}%
\pgfsetlinewidth{1.003750pt}%
\definecolor{currentstroke}{rgb}{0.090196,0.745098,0.733333}%
\pgfsetstrokecolor{currentstroke}%
\pgfsetdash{}{0pt}%
\pgfsys@defobject{currentmarker}{\pgfqpoint{-0.041667in}{-0.041667in}}{\pgfqpoint{0.041667in}{0.041667in}}{%
\pgfpathmoveto{\pgfqpoint{0.000000in}{-0.041667in}}%
\pgfpathcurveto{\pgfqpoint{0.011050in}{-0.041667in}}{\pgfqpoint{0.021649in}{-0.037276in}}{\pgfqpoint{0.029463in}{-0.029463in}}%
\pgfpathcurveto{\pgfqpoint{0.037276in}{-0.021649in}}{\pgfqpoint{0.041667in}{-0.011050in}}{\pgfqpoint{0.041667in}{0.000000in}}%
\pgfpathcurveto{\pgfqpoint{0.041667in}{0.011050in}}{\pgfqpoint{0.037276in}{0.021649in}}{\pgfqpoint{0.029463in}{0.029463in}}%
\pgfpathcurveto{\pgfqpoint{0.021649in}{0.037276in}}{\pgfqpoint{0.011050in}{0.041667in}}{\pgfqpoint{0.000000in}{0.041667in}}%
\pgfpathcurveto{\pgfqpoint{-0.011050in}{0.041667in}}{\pgfqpoint{-0.021649in}{0.037276in}}{\pgfqpoint{-0.029463in}{0.029463in}}%
\pgfpathcurveto{\pgfqpoint{-0.037276in}{0.021649in}}{\pgfqpoint{-0.041667in}{0.011050in}}{\pgfqpoint{-0.041667in}{0.000000in}}%
\pgfpathcurveto{\pgfqpoint{-0.041667in}{-0.011050in}}{\pgfqpoint{-0.037276in}{-0.021649in}}{\pgfqpoint{-0.029463in}{-0.029463in}}%
\pgfpathcurveto{\pgfqpoint{-0.021649in}{-0.037276in}}{\pgfqpoint{-0.011050in}{-0.041667in}}{\pgfqpoint{0.000000in}{-0.041667in}}%
\pgfpathclose%
\pgfusepath{stroke,fill}%
}%
\begin{pgfscope}%
\pgfsys@transformshift{1.063889in}{3.853630in}%
\pgfsys@useobject{currentmarker}{}%
\end{pgfscope}%
\end{pgfscope}%
\begin{pgfscope}%
\pgftext[x=1.313889in,y=3.805019in,left,base]{\sffamily\fontsize{10.000000}{12.000000}\selectfont Carga 100\%}%
\end{pgfscope}%
\begin{pgfscope}%
\pgfsetbuttcap%
\pgfsetroundjoin%
\definecolor{currentfill}{rgb}{0.937255,0.176471,0.337255}%
\pgfsetfillcolor{currentfill}%
\pgfsetlinewidth{1.003750pt}%
\definecolor{currentstroke}{rgb}{0.937255,0.176471,0.337255}%
\pgfsetstrokecolor{currentstroke}%
\pgfsetdash{}{0pt}%
\pgfsys@defobject{currentmarker}{\pgfqpoint{-0.041667in}{-0.041667in}}{\pgfqpoint{0.041667in}{0.041667in}}{%
\pgfpathmoveto{\pgfqpoint{0.000000in}{-0.041667in}}%
\pgfpathcurveto{\pgfqpoint{0.011050in}{-0.041667in}}{\pgfqpoint{0.021649in}{-0.037276in}}{\pgfqpoint{0.029463in}{-0.029463in}}%
\pgfpathcurveto{\pgfqpoint{0.037276in}{-0.021649in}}{\pgfqpoint{0.041667in}{-0.011050in}}{\pgfqpoint{0.041667in}{0.000000in}}%
\pgfpathcurveto{\pgfqpoint{0.041667in}{0.011050in}}{\pgfqpoint{0.037276in}{0.021649in}}{\pgfqpoint{0.029463in}{0.029463in}}%
\pgfpathcurveto{\pgfqpoint{0.021649in}{0.037276in}}{\pgfqpoint{0.011050in}{0.041667in}}{\pgfqpoint{0.000000in}{0.041667in}}%
\pgfpathcurveto{\pgfqpoint{-0.011050in}{0.041667in}}{\pgfqpoint{-0.021649in}{0.037276in}}{\pgfqpoint{-0.029463in}{0.029463in}}%
\pgfpathcurveto{\pgfqpoint{-0.037276in}{0.021649in}}{\pgfqpoint{-0.041667in}{0.011050in}}{\pgfqpoint{-0.041667in}{0.000000in}}%
\pgfpathcurveto{\pgfqpoint{-0.041667in}{-0.011050in}}{\pgfqpoint{-0.037276in}{-0.021649in}}{\pgfqpoint{-0.029463in}{-0.029463in}}%
\pgfpathcurveto{\pgfqpoint{-0.021649in}{-0.037276in}}{\pgfqpoint{-0.011050in}{-0.041667in}}{\pgfqpoint{0.000000in}{-0.041667in}}%
\pgfpathclose%
\pgfusepath{stroke,fill}%
}%
\begin{pgfscope}%
\pgfsys@transformshift{1.063889in}{3.656871in}%
\pgfsys@useobject{currentmarker}{}%
\end{pgfscope}%
\end{pgfscope}%
\begin{pgfscope}%
\pgftext[x=1.313889in,y=3.608259in,left,base]{\sffamily\fontsize{10.000000}{12.000000}\selectfont Carga 75\%}%
\end{pgfscope}%
\begin{pgfscope}%
\pgfsetbuttcap%
\pgfsetroundjoin%
\definecolor{currentfill}{rgb}{0.549020,0.847059,0.403922}%
\pgfsetfillcolor{currentfill}%
\pgfsetlinewidth{1.003750pt}%
\definecolor{currentstroke}{rgb}{0.549020,0.847059,0.403922}%
\pgfsetstrokecolor{currentstroke}%
\pgfsetdash{}{0pt}%
\pgfsys@defobject{currentmarker}{\pgfqpoint{-0.041667in}{-0.041667in}}{\pgfqpoint{0.041667in}{0.041667in}}{%
\pgfpathmoveto{\pgfqpoint{0.000000in}{-0.041667in}}%
\pgfpathcurveto{\pgfqpoint{0.011050in}{-0.041667in}}{\pgfqpoint{0.021649in}{-0.037276in}}{\pgfqpoint{0.029463in}{-0.029463in}}%
\pgfpathcurveto{\pgfqpoint{0.037276in}{-0.021649in}}{\pgfqpoint{0.041667in}{-0.011050in}}{\pgfqpoint{0.041667in}{0.000000in}}%
\pgfpathcurveto{\pgfqpoint{0.041667in}{0.011050in}}{\pgfqpoint{0.037276in}{0.021649in}}{\pgfqpoint{0.029463in}{0.029463in}}%
\pgfpathcurveto{\pgfqpoint{0.021649in}{0.037276in}}{\pgfqpoint{0.011050in}{0.041667in}}{\pgfqpoint{0.000000in}{0.041667in}}%
\pgfpathcurveto{\pgfqpoint{-0.011050in}{0.041667in}}{\pgfqpoint{-0.021649in}{0.037276in}}{\pgfqpoint{-0.029463in}{0.029463in}}%
\pgfpathcurveto{\pgfqpoint{-0.037276in}{0.021649in}}{\pgfqpoint{-0.041667in}{0.011050in}}{\pgfqpoint{-0.041667in}{0.000000in}}%
\pgfpathcurveto{\pgfqpoint{-0.041667in}{-0.011050in}}{\pgfqpoint{-0.037276in}{-0.021649in}}{\pgfqpoint{-0.029463in}{-0.029463in}}%
\pgfpathcurveto{\pgfqpoint{-0.021649in}{-0.037276in}}{\pgfqpoint{-0.011050in}{-0.041667in}}{\pgfqpoint{0.000000in}{-0.041667in}}%
\pgfpathclose%
\pgfusepath{stroke,fill}%
}%
\begin{pgfscope}%
\pgfsys@transformshift{1.063889in}{3.460111in}%
\pgfsys@useobject{currentmarker}{}%
\end{pgfscope}%
\end{pgfscope}%
\begin{pgfscope}%
\pgftext[x=1.313889in,y=3.411500in,left,base]{\sffamily\fontsize{10.000000}{12.000000}\selectfont Carga 50\%}%
\end{pgfscope}%
\begin{pgfscope}%
\pgfsetbuttcap%
\pgfsetroundjoin%
\definecolor{currentfill}{rgb}{0.184314,0.141176,0.227451}%
\pgfsetfillcolor{currentfill}%
\pgfsetlinewidth{1.003750pt}%
\definecolor{currentstroke}{rgb}{0.184314,0.141176,0.227451}%
\pgfsetstrokecolor{currentstroke}%
\pgfsetdash{}{0pt}%
\pgfsys@defobject{currentmarker}{\pgfqpoint{-0.041667in}{-0.041667in}}{\pgfqpoint{0.041667in}{0.041667in}}{%
\pgfpathmoveto{\pgfqpoint{0.000000in}{-0.041667in}}%
\pgfpathcurveto{\pgfqpoint{0.011050in}{-0.041667in}}{\pgfqpoint{0.021649in}{-0.037276in}}{\pgfqpoint{0.029463in}{-0.029463in}}%
\pgfpathcurveto{\pgfqpoint{0.037276in}{-0.021649in}}{\pgfqpoint{0.041667in}{-0.011050in}}{\pgfqpoint{0.041667in}{0.000000in}}%
\pgfpathcurveto{\pgfqpoint{0.041667in}{0.011050in}}{\pgfqpoint{0.037276in}{0.021649in}}{\pgfqpoint{0.029463in}{0.029463in}}%
\pgfpathcurveto{\pgfqpoint{0.021649in}{0.037276in}}{\pgfqpoint{0.011050in}{0.041667in}}{\pgfqpoint{0.000000in}{0.041667in}}%
\pgfpathcurveto{\pgfqpoint{-0.011050in}{0.041667in}}{\pgfqpoint{-0.021649in}{0.037276in}}{\pgfqpoint{-0.029463in}{0.029463in}}%
\pgfpathcurveto{\pgfqpoint{-0.037276in}{0.021649in}}{\pgfqpoint{-0.041667in}{0.011050in}}{\pgfqpoint{-0.041667in}{0.000000in}}%
\pgfpathcurveto{\pgfqpoint{-0.041667in}{-0.011050in}}{\pgfqpoint{-0.037276in}{-0.021649in}}{\pgfqpoint{-0.029463in}{-0.029463in}}%
\pgfpathcurveto{\pgfqpoint{-0.021649in}{-0.037276in}}{\pgfqpoint{-0.011050in}{-0.041667in}}{\pgfqpoint{0.000000in}{-0.041667in}}%
\pgfpathclose%
\pgfusepath{stroke,fill}%
}%
\begin{pgfscope}%
\pgfsys@transformshift{1.063889in}{3.263352in}%
\pgfsys@useobject{currentmarker}{}%
\end{pgfscope}%
\end{pgfscope}%
\begin{pgfscope}%
\pgftext[x=1.313889in,y=3.214741in,left,base]{\sffamily\fontsize{10.000000}{12.000000}\selectfont Carga 25\%}%
\end{pgfscope}%
\end{pgfpicture}%
\makeatother%
\endgroup%

    \captionof{figure}[Cantidad de vehículos vs. tiempo real de simulación]{Gráfico de dispersión del promedio de vehículos en simulación por instante de tiempo vs. tiempo total de simulación, para una simulación de 15 minutos de tiempo simulado.}
    \label{fig:vehiclesvstime}
\end{figure}

La figura \ref{fig:timevsvehicles_evolution} presenta además la evolución de la red vehicular en términos de cantidad de vehículos para un \emph{run} con factor de demanda de 100\%, tanto en tiempo real como en tiempo simulado. 

\begin{figure}[h]
    \centering
    %% Creator: Matplotlib, PGF backend
%%
%% To include the figure in your LaTeX document, write
%%   \input{<filename>.pgf}
%%
%% Make sure the required packages are loaded in your preamble
%%   \usepackage{pgf}
%%
%% Figures using additional raster images can only be included by \input if
%% they are in the same directory as the main LaTeX file. For loading figures
%% from other directories you can use the `import` package
%%   \usepackage{import}
%% and then include the figures with
%%   \import{<path to file>}{<filename>.pgf}
%%
%% Matplotlib used the following preamble
%%   \usepackage[utf8x]{inputenc}
%%   \usepackage[T1]{fontenc}
%%   \usepackage{cmbright}
%%
\begingroup%
\makeatletter%
\begin{pgfpicture}%
\pgfpathrectangle{\pgfpointorigin}{\pgfqpoint{6.400000in}{4.800000in}}%
\pgfusepath{use as bounding box, clip}%
\begin{pgfscope}%
\pgfsetbuttcap%
\pgfsetmiterjoin%
\definecolor{currentfill}{rgb}{1.000000,1.000000,1.000000}%
\pgfsetfillcolor{currentfill}%
\pgfsetlinewidth{0.000000pt}%
\definecolor{currentstroke}{rgb}{1.000000,1.000000,1.000000}%
\pgfsetstrokecolor{currentstroke}%
\pgfsetdash{}{0pt}%
\pgfpathmoveto{\pgfqpoint{0.000000in}{0.000000in}}%
\pgfpathlineto{\pgfqpoint{6.400000in}{0.000000in}}%
\pgfpathlineto{\pgfqpoint{6.400000in}{4.800000in}}%
\pgfpathlineto{\pgfqpoint{0.000000in}{4.800000in}}%
\pgfpathclose%
\pgfusepath{fill}%
\end{pgfscope}%
\begin{pgfscope}%
\pgfsetbuttcap%
\pgfsetmiterjoin%
\definecolor{currentfill}{rgb}{1.000000,1.000000,1.000000}%
\pgfsetfillcolor{currentfill}%
\pgfsetlinewidth{0.000000pt}%
\definecolor{currentstroke}{rgb}{0.000000,0.000000,0.000000}%
\pgfsetstrokecolor{currentstroke}%
\pgfsetstrokeopacity{0.000000}%
\pgfsetdash{}{0pt}%
\pgfpathmoveto{\pgfqpoint{0.800000in}{0.528000in}}%
\pgfpathlineto{\pgfqpoint{5.760000in}{0.528000in}}%
\pgfpathlineto{\pgfqpoint{5.760000in}{4.224000in}}%
\pgfpathlineto{\pgfqpoint{0.800000in}{4.224000in}}%
\pgfpathclose%
\pgfusepath{fill}%
\end{pgfscope}%
\begin{pgfscope}%
\pgfpathrectangle{\pgfqpoint{0.800000in}{0.528000in}}{\pgfqpoint{4.960000in}{3.696000in}} %
\pgfusepath{clip}%
\pgfsetrectcap%
\pgfsetroundjoin%
\pgfsetlinewidth{0.803000pt}%
\definecolor{currentstroke}{rgb}{0.631373,0.631373,0.631373}%
\pgfsetstrokecolor{currentstroke}%
\pgfsetstrokeopacity{0.100000}%
\pgfsetdash{}{0pt}%
\pgfpathmoveto{\pgfqpoint{1.025455in}{0.528000in}}%
\pgfpathlineto{\pgfqpoint{1.025455in}{4.224000in}}%
\pgfusepath{stroke}%
\end{pgfscope}%
\begin{pgfscope}%
\pgfsetbuttcap%
\pgfsetroundjoin%
\definecolor{currentfill}{rgb}{0.333333,0.333333,0.333333}%
\pgfsetfillcolor{currentfill}%
\pgfsetlinewidth{0.803000pt}%
\definecolor{currentstroke}{rgb}{0.333333,0.333333,0.333333}%
\pgfsetstrokecolor{currentstroke}%
\pgfsetdash{}{0pt}%
\pgfsys@defobject{currentmarker}{\pgfqpoint{0.000000in}{-0.048611in}}{\pgfqpoint{0.000000in}{0.000000in}}{%
\pgfpathmoveto{\pgfqpoint{0.000000in}{0.000000in}}%
\pgfpathlineto{\pgfqpoint{0.000000in}{-0.048611in}}%
\pgfusepath{stroke,fill}%
}%
\begin{pgfscope}%
\pgfsys@transformshift{1.025455in}{0.528000in}%
\pgfsys@useobject{currentmarker}{}%
\end{pgfscope}%
\end{pgfscope}%
\begin{pgfscope}%
\definecolor{textcolor}{rgb}{0.333333,0.333333,0.333333}%
\pgfsetstrokecolor{textcolor}%
\pgfsetfillcolor{textcolor}%
\pgftext[x=1.025455in,y=0.430778in,,top]{\color{textcolor}\sffamily\fontsize{10.000000}{12.000000}\selectfont 00:00}%
\end{pgfscope}%
\begin{pgfscope}%
\pgfpathrectangle{\pgfqpoint{0.800000in}{0.528000in}}{\pgfqpoint{4.960000in}{3.696000in}} %
\pgfusepath{clip}%
\pgfsetrectcap%
\pgfsetroundjoin%
\pgfsetlinewidth{0.803000pt}%
\definecolor{currentstroke}{rgb}{0.631373,0.631373,0.631373}%
\pgfsetstrokecolor{currentstroke}%
\pgfsetstrokeopacity{0.100000}%
\pgfsetdash{}{0pt}%
\pgfpathmoveto{\pgfqpoint{1.660091in}{0.528000in}}%
\pgfpathlineto{\pgfqpoint{1.660091in}{4.224000in}}%
\pgfusepath{stroke}%
\end{pgfscope}%
\begin{pgfscope}%
\pgfsetbuttcap%
\pgfsetroundjoin%
\definecolor{currentfill}{rgb}{0.333333,0.333333,0.333333}%
\pgfsetfillcolor{currentfill}%
\pgfsetlinewidth{0.803000pt}%
\definecolor{currentstroke}{rgb}{0.333333,0.333333,0.333333}%
\pgfsetstrokecolor{currentstroke}%
\pgfsetdash{}{0pt}%
\pgfsys@defobject{currentmarker}{\pgfqpoint{0.000000in}{-0.048611in}}{\pgfqpoint{0.000000in}{0.000000in}}{%
\pgfpathmoveto{\pgfqpoint{0.000000in}{0.000000in}}%
\pgfpathlineto{\pgfqpoint{0.000000in}{-0.048611in}}%
\pgfusepath{stroke,fill}%
}%
\begin{pgfscope}%
\pgfsys@transformshift{1.660091in}{0.528000in}%
\pgfsys@useobject{currentmarker}{}%
\end{pgfscope}%
\end{pgfscope}%
\begin{pgfscope}%
\definecolor{textcolor}{rgb}{0.333333,0.333333,0.333333}%
\pgfsetstrokecolor{textcolor}%
\pgfsetfillcolor{textcolor}%
\pgftext[x=1.660091in,y=0.430778in,,top]{\color{textcolor}\sffamily\fontsize{10.000000}{12.000000}\selectfont 03:20}%
\end{pgfscope}%
\begin{pgfscope}%
\pgfpathrectangle{\pgfqpoint{0.800000in}{0.528000in}}{\pgfqpoint{4.960000in}{3.696000in}} %
\pgfusepath{clip}%
\pgfsetrectcap%
\pgfsetroundjoin%
\pgfsetlinewidth{0.803000pt}%
\definecolor{currentstroke}{rgb}{0.631373,0.631373,0.631373}%
\pgfsetstrokecolor{currentstroke}%
\pgfsetstrokeopacity{0.100000}%
\pgfsetdash{}{0pt}%
\pgfpathmoveto{\pgfqpoint{2.294727in}{0.528000in}}%
\pgfpathlineto{\pgfqpoint{2.294727in}{4.224000in}}%
\pgfusepath{stroke}%
\end{pgfscope}%
\begin{pgfscope}%
\pgfsetbuttcap%
\pgfsetroundjoin%
\definecolor{currentfill}{rgb}{0.333333,0.333333,0.333333}%
\pgfsetfillcolor{currentfill}%
\pgfsetlinewidth{0.803000pt}%
\definecolor{currentstroke}{rgb}{0.333333,0.333333,0.333333}%
\pgfsetstrokecolor{currentstroke}%
\pgfsetdash{}{0pt}%
\pgfsys@defobject{currentmarker}{\pgfqpoint{0.000000in}{-0.048611in}}{\pgfqpoint{0.000000in}{0.000000in}}{%
\pgfpathmoveto{\pgfqpoint{0.000000in}{0.000000in}}%
\pgfpathlineto{\pgfqpoint{0.000000in}{-0.048611in}}%
\pgfusepath{stroke,fill}%
}%
\begin{pgfscope}%
\pgfsys@transformshift{2.294727in}{0.528000in}%
\pgfsys@useobject{currentmarker}{}%
\end{pgfscope}%
\end{pgfscope}%
\begin{pgfscope}%
\definecolor{textcolor}{rgb}{0.333333,0.333333,0.333333}%
\pgfsetstrokecolor{textcolor}%
\pgfsetfillcolor{textcolor}%
\pgftext[x=2.294727in,y=0.430778in,,top]{\color{textcolor}\sffamily\fontsize{10.000000}{12.000000}\selectfont 06:40}%
\end{pgfscope}%
\begin{pgfscope}%
\pgfpathrectangle{\pgfqpoint{0.800000in}{0.528000in}}{\pgfqpoint{4.960000in}{3.696000in}} %
\pgfusepath{clip}%
\pgfsetrectcap%
\pgfsetroundjoin%
\pgfsetlinewidth{0.803000pt}%
\definecolor{currentstroke}{rgb}{0.631373,0.631373,0.631373}%
\pgfsetstrokecolor{currentstroke}%
\pgfsetstrokeopacity{0.100000}%
\pgfsetdash{}{0pt}%
\pgfpathmoveto{\pgfqpoint{2.929363in}{0.528000in}}%
\pgfpathlineto{\pgfqpoint{2.929363in}{4.224000in}}%
\pgfusepath{stroke}%
\end{pgfscope}%
\begin{pgfscope}%
\pgfsetbuttcap%
\pgfsetroundjoin%
\definecolor{currentfill}{rgb}{0.333333,0.333333,0.333333}%
\pgfsetfillcolor{currentfill}%
\pgfsetlinewidth{0.803000pt}%
\definecolor{currentstroke}{rgb}{0.333333,0.333333,0.333333}%
\pgfsetstrokecolor{currentstroke}%
\pgfsetdash{}{0pt}%
\pgfsys@defobject{currentmarker}{\pgfqpoint{0.000000in}{-0.048611in}}{\pgfqpoint{0.000000in}{0.000000in}}{%
\pgfpathmoveto{\pgfqpoint{0.000000in}{0.000000in}}%
\pgfpathlineto{\pgfqpoint{0.000000in}{-0.048611in}}%
\pgfusepath{stroke,fill}%
}%
\begin{pgfscope}%
\pgfsys@transformshift{2.929363in}{0.528000in}%
\pgfsys@useobject{currentmarker}{}%
\end{pgfscope}%
\end{pgfscope}%
\begin{pgfscope}%
\definecolor{textcolor}{rgb}{0.333333,0.333333,0.333333}%
\pgfsetstrokecolor{textcolor}%
\pgfsetfillcolor{textcolor}%
\pgftext[x=2.929363in,y=0.430778in,,top]{\color{textcolor}\sffamily\fontsize{10.000000}{12.000000}\selectfont 10:00}%
\end{pgfscope}%
\begin{pgfscope}%
\pgfpathrectangle{\pgfqpoint{0.800000in}{0.528000in}}{\pgfqpoint{4.960000in}{3.696000in}} %
\pgfusepath{clip}%
\pgfsetrectcap%
\pgfsetroundjoin%
\pgfsetlinewidth{0.803000pt}%
\definecolor{currentstroke}{rgb}{0.631373,0.631373,0.631373}%
\pgfsetstrokecolor{currentstroke}%
\pgfsetstrokeopacity{0.100000}%
\pgfsetdash{}{0pt}%
\pgfpathmoveto{\pgfqpoint{3.564000in}{0.528000in}}%
\pgfpathlineto{\pgfqpoint{3.564000in}{4.224000in}}%
\pgfusepath{stroke}%
\end{pgfscope}%
\begin{pgfscope}%
\pgfsetbuttcap%
\pgfsetroundjoin%
\definecolor{currentfill}{rgb}{0.333333,0.333333,0.333333}%
\pgfsetfillcolor{currentfill}%
\pgfsetlinewidth{0.803000pt}%
\definecolor{currentstroke}{rgb}{0.333333,0.333333,0.333333}%
\pgfsetstrokecolor{currentstroke}%
\pgfsetdash{}{0pt}%
\pgfsys@defobject{currentmarker}{\pgfqpoint{0.000000in}{-0.048611in}}{\pgfqpoint{0.000000in}{0.000000in}}{%
\pgfpathmoveto{\pgfqpoint{0.000000in}{0.000000in}}%
\pgfpathlineto{\pgfqpoint{0.000000in}{-0.048611in}}%
\pgfusepath{stroke,fill}%
}%
\begin{pgfscope}%
\pgfsys@transformshift{3.564000in}{0.528000in}%
\pgfsys@useobject{currentmarker}{}%
\end{pgfscope}%
\end{pgfscope}%
\begin{pgfscope}%
\definecolor{textcolor}{rgb}{0.333333,0.333333,0.333333}%
\pgfsetstrokecolor{textcolor}%
\pgfsetfillcolor{textcolor}%
\pgftext[x=3.564000in,y=0.430778in,,top]{\color{textcolor}\sffamily\fontsize{10.000000}{12.000000}\selectfont 13:20}%
\end{pgfscope}%
\begin{pgfscope}%
\pgfpathrectangle{\pgfqpoint{0.800000in}{0.528000in}}{\pgfqpoint{4.960000in}{3.696000in}} %
\pgfusepath{clip}%
\pgfsetrectcap%
\pgfsetroundjoin%
\pgfsetlinewidth{0.803000pt}%
\definecolor{currentstroke}{rgb}{0.631373,0.631373,0.631373}%
\pgfsetstrokecolor{currentstroke}%
\pgfsetstrokeopacity{0.100000}%
\pgfsetdash{}{0pt}%
\pgfpathmoveto{\pgfqpoint{4.198636in}{0.528000in}}%
\pgfpathlineto{\pgfqpoint{4.198636in}{4.224000in}}%
\pgfusepath{stroke}%
\end{pgfscope}%
\begin{pgfscope}%
\pgfsetbuttcap%
\pgfsetroundjoin%
\definecolor{currentfill}{rgb}{0.333333,0.333333,0.333333}%
\pgfsetfillcolor{currentfill}%
\pgfsetlinewidth{0.803000pt}%
\definecolor{currentstroke}{rgb}{0.333333,0.333333,0.333333}%
\pgfsetstrokecolor{currentstroke}%
\pgfsetdash{}{0pt}%
\pgfsys@defobject{currentmarker}{\pgfqpoint{0.000000in}{-0.048611in}}{\pgfqpoint{0.000000in}{0.000000in}}{%
\pgfpathmoveto{\pgfqpoint{0.000000in}{0.000000in}}%
\pgfpathlineto{\pgfqpoint{0.000000in}{-0.048611in}}%
\pgfusepath{stroke,fill}%
}%
\begin{pgfscope}%
\pgfsys@transformshift{4.198636in}{0.528000in}%
\pgfsys@useobject{currentmarker}{}%
\end{pgfscope}%
\end{pgfscope}%
\begin{pgfscope}%
\definecolor{textcolor}{rgb}{0.333333,0.333333,0.333333}%
\pgfsetstrokecolor{textcolor}%
\pgfsetfillcolor{textcolor}%
\pgftext[x=4.198636in,y=0.430778in,,top]{\color{textcolor}\sffamily\fontsize{10.000000}{12.000000}\selectfont 16:40}%
\end{pgfscope}%
\begin{pgfscope}%
\pgfpathrectangle{\pgfqpoint{0.800000in}{0.528000in}}{\pgfqpoint{4.960000in}{3.696000in}} %
\pgfusepath{clip}%
\pgfsetrectcap%
\pgfsetroundjoin%
\pgfsetlinewidth{0.803000pt}%
\definecolor{currentstroke}{rgb}{0.631373,0.631373,0.631373}%
\pgfsetstrokecolor{currentstroke}%
\pgfsetstrokeopacity{0.100000}%
\pgfsetdash{}{0pt}%
\pgfpathmoveto{\pgfqpoint{4.833272in}{0.528000in}}%
\pgfpathlineto{\pgfqpoint{4.833272in}{4.224000in}}%
\pgfusepath{stroke}%
\end{pgfscope}%
\begin{pgfscope}%
\pgfsetbuttcap%
\pgfsetroundjoin%
\definecolor{currentfill}{rgb}{0.333333,0.333333,0.333333}%
\pgfsetfillcolor{currentfill}%
\pgfsetlinewidth{0.803000pt}%
\definecolor{currentstroke}{rgb}{0.333333,0.333333,0.333333}%
\pgfsetstrokecolor{currentstroke}%
\pgfsetdash{}{0pt}%
\pgfsys@defobject{currentmarker}{\pgfqpoint{0.000000in}{-0.048611in}}{\pgfqpoint{0.000000in}{0.000000in}}{%
\pgfpathmoveto{\pgfqpoint{0.000000in}{0.000000in}}%
\pgfpathlineto{\pgfqpoint{0.000000in}{-0.048611in}}%
\pgfusepath{stroke,fill}%
}%
\begin{pgfscope}%
\pgfsys@transformshift{4.833272in}{0.528000in}%
\pgfsys@useobject{currentmarker}{}%
\end{pgfscope}%
\end{pgfscope}%
\begin{pgfscope}%
\definecolor{textcolor}{rgb}{0.333333,0.333333,0.333333}%
\pgfsetstrokecolor{textcolor}%
\pgfsetfillcolor{textcolor}%
\pgftext[x=4.833272in,y=0.430778in,,top]{\color{textcolor}\sffamily\fontsize{10.000000}{12.000000}\selectfont 20:00}%
\end{pgfscope}%
\begin{pgfscope}%
\pgfpathrectangle{\pgfqpoint{0.800000in}{0.528000in}}{\pgfqpoint{4.960000in}{3.696000in}} %
\pgfusepath{clip}%
\pgfsetrectcap%
\pgfsetroundjoin%
\pgfsetlinewidth{0.803000pt}%
\definecolor{currentstroke}{rgb}{0.631373,0.631373,0.631373}%
\pgfsetstrokecolor{currentstroke}%
\pgfsetstrokeopacity{0.100000}%
\pgfsetdash{}{0pt}%
\pgfpathmoveto{\pgfqpoint{5.467909in}{0.528000in}}%
\pgfpathlineto{\pgfqpoint{5.467909in}{4.224000in}}%
\pgfusepath{stroke}%
\end{pgfscope}%
\begin{pgfscope}%
\pgfsetbuttcap%
\pgfsetroundjoin%
\definecolor{currentfill}{rgb}{0.333333,0.333333,0.333333}%
\pgfsetfillcolor{currentfill}%
\pgfsetlinewidth{0.803000pt}%
\definecolor{currentstroke}{rgb}{0.333333,0.333333,0.333333}%
\pgfsetstrokecolor{currentstroke}%
\pgfsetdash{}{0pt}%
\pgfsys@defobject{currentmarker}{\pgfqpoint{0.000000in}{-0.048611in}}{\pgfqpoint{0.000000in}{0.000000in}}{%
\pgfpathmoveto{\pgfqpoint{0.000000in}{0.000000in}}%
\pgfpathlineto{\pgfqpoint{0.000000in}{-0.048611in}}%
\pgfusepath{stroke,fill}%
}%
\begin{pgfscope}%
\pgfsys@transformshift{5.467909in}{0.528000in}%
\pgfsys@useobject{currentmarker}{}%
\end{pgfscope}%
\end{pgfscope}%
\begin{pgfscope}%
\definecolor{textcolor}{rgb}{0.333333,0.333333,0.333333}%
\pgfsetstrokecolor{textcolor}%
\pgfsetfillcolor{textcolor}%
\pgftext[x=5.467909in,y=0.430778in,,top]{\color{textcolor}\sffamily\fontsize{10.000000}{12.000000}\selectfont 23:20}%
\end{pgfscope}%
\begin{pgfscope}%
\definecolor{textcolor}{rgb}{0.333333,0.333333,0.333333}%
\pgfsetstrokecolor{textcolor}%
\pgfsetfillcolor{textcolor}%
\pgftext[x=3.280000in,y=0.255624in,,top]{\color{textcolor}\sffamily\fontsize{12.000000}{14.400000}\selectfont Tiempo [MM:SS]}%
\end{pgfscope}%
\begin{pgfscope}%
\pgfpathrectangle{\pgfqpoint{0.800000in}{0.528000in}}{\pgfqpoint{4.960000in}{3.696000in}} %
\pgfusepath{clip}%
\pgfsetrectcap%
\pgfsetroundjoin%
\pgfsetlinewidth{0.803000pt}%
\definecolor{currentstroke}{rgb}{0.631373,0.631373,0.631373}%
\pgfsetstrokecolor{currentstroke}%
\pgfsetstrokeopacity{0.100000}%
\pgfsetdash{}{0pt}%
\pgfpathmoveto{\pgfqpoint{0.800000in}{0.696000in}}%
\pgfpathlineto{\pgfqpoint{5.760000in}{0.696000in}}%
\pgfusepath{stroke}%
\end{pgfscope}%
\begin{pgfscope}%
\pgfsetbuttcap%
\pgfsetroundjoin%
\definecolor{currentfill}{rgb}{0.333333,0.333333,0.333333}%
\pgfsetfillcolor{currentfill}%
\pgfsetlinewidth{0.803000pt}%
\definecolor{currentstroke}{rgb}{0.333333,0.333333,0.333333}%
\pgfsetstrokecolor{currentstroke}%
\pgfsetdash{}{0pt}%
\pgfsys@defobject{currentmarker}{\pgfqpoint{-0.048611in}{0.000000in}}{\pgfqpoint{0.000000in}{0.000000in}}{%
\pgfpathmoveto{\pgfqpoint{0.000000in}{0.000000in}}%
\pgfpathlineto{\pgfqpoint{-0.048611in}{0.000000in}}%
\pgfusepath{stroke,fill}%
}%
\begin{pgfscope}%
\pgfsys@transformshift{0.800000in}{0.696000in}%
\pgfsys@useobject{currentmarker}{}%
\end{pgfscope}%
\end{pgfscope}%
\begin{pgfscope}%
\definecolor{textcolor}{rgb}{0.333333,0.333333,0.333333}%
\pgfsetstrokecolor{textcolor}%
\pgfsetfillcolor{textcolor}%
\pgftext[x=0.629862in,y=0.647775in,left,base]{\color{textcolor}\sffamily\fontsize{10.000000}{12.000000}\selectfont 0}%
\end{pgfscope}%
\begin{pgfscope}%
\pgfpathrectangle{\pgfqpoint{0.800000in}{0.528000in}}{\pgfqpoint{4.960000in}{3.696000in}} %
\pgfusepath{clip}%
\pgfsetrectcap%
\pgfsetroundjoin%
\pgfsetlinewidth{0.803000pt}%
\definecolor{currentstroke}{rgb}{0.631373,0.631373,0.631373}%
\pgfsetstrokecolor{currentstroke}%
\pgfsetstrokeopacity{0.100000}%
\pgfsetdash{}{0pt}%
\pgfpathmoveto{\pgfqpoint{0.800000in}{1.131911in}}%
\pgfpathlineto{\pgfqpoint{5.760000in}{1.131911in}}%
\pgfusepath{stroke}%
\end{pgfscope}%
\begin{pgfscope}%
\pgfsetbuttcap%
\pgfsetroundjoin%
\definecolor{currentfill}{rgb}{0.333333,0.333333,0.333333}%
\pgfsetfillcolor{currentfill}%
\pgfsetlinewidth{0.803000pt}%
\definecolor{currentstroke}{rgb}{0.333333,0.333333,0.333333}%
\pgfsetstrokecolor{currentstroke}%
\pgfsetdash{}{0pt}%
\pgfsys@defobject{currentmarker}{\pgfqpoint{-0.048611in}{0.000000in}}{\pgfqpoint{0.000000in}{0.000000in}}{%
\pgfpathmoveto{\pgfqpoint{0.000000in}{0.000000in}}%
\pgfpathlineto{\pgfqpoint{-0.048611in}{0.000000in}}%
\pgfusepath{stroke,fill}%
}%
\begin{pgfscope}%
\pgfsys@transformshift{0.800000in}{1.131911in}%
\pgfsys@useobject{currentmarker}{}%
\end{pgfscope}%
\end{pgfscope}%
\begin{pgfscope}%
\definecolor{textcolor}{rgb}{0.333333,0.333333,0.333333}%
\pgfsetstrokecolor{textcolor}%
\pgfsetfillcolor{textcolor}%
\pgftext[x=0.484030in,y=1.083685in,left,base]{\color{textcolor}\sffamily\fontsize{10.000000}{12.000000}\selectfont 250}%
\end{pgfscope}%
\begin{pgfscope}%
\pgfpathrectangle{\pgfqpoint{0.800000in}{0.528000in}}{\pgfqpoint{4.960000in}{3.696000in}} %
\pgfusepath{clip}%
\pgfsetrectcap%
\pgfsetroundjoin%
\pgfsetlinewidth{0.803000pt}%
\definecolor{currentstroke}{rgb}{0.631373,0.631373,0.631373}%
\pgfsetstrokecolor{currentstroke}%
\pgfsetstrokeopacity{0.100000}%
\pgfsetdash{}{0pt}%
\pgfpathmoveto{\pgfqpoint{0.800000in}{1.567821in}}%
\pgfpathlineto{\pgfqpoint{5.760000in}{1.567821in}}%
\pgfusepath{stroke}%
\end{pgfscope}%
\begin{pgfscope}%
\pgfsetbuttcap%
\pgfsetroundjoin%
\definecolor{currentfill}{rgb}{0.333333,0.333333,0.333333}%
\pgfsetfillcolor{currentfill}%
\pgfsetlinewidth{0.803000pt}%
\definecolor{currentstroke}{rgb}{0.333333,0.333333,0.333333}%
\pgfsetstrokecolor{currentstroke}%
\pgfsetdash{}{0pt}%
\pgfsys@defobject{currentmarker}{\pgfqpoint{-0.048611in}{0.000000in}}{\pgfqpoint{0.000000in}{0.000000in}}{%
\pgfpathmoveto{\pgfqpoint{0.000000in}{0.000000in}}%
\pgfpathlineto{\pgfqpoint{-0.048611in}{0.000000in}}%
\pgfusepath{stroke,fill}%
}%
\begin{pgfscope}%
\pgfsys@transformshift{0.800000in}{1.567821in}%
\pgfsys@useobject{currentmarker}{}%
\end{pgfscope}%
\end{pgfscope}%
\begin{pgfscope}%
\definecolor{textcolor}{rgb}{0.333333,0.333333,0.333333}%
\pgfsetstrokecolor{textcolor}%
\pgfsetfillcolor{textcolor}%
\pgftext[x=0.484030in,y=1.519596in,left,base]{\color{textcolor}\sffamily\fontsize{10.000000}{12.000000}\selectfont 500}%
\end{pgfscope}%
\begin{pgfscope}%
\pgfpathrectangle{\pgfqpoint{0.800000in}{0.528000in}}{\pgfqpoint{4.960000in}{3.696000in}} %
\pgfusepath{clip}%
\pgfsetrectcap%
\pgfsetroundjoin%
\pgfsetlinewidth{0.803000pt}%
\definecolor{currentstroke}{rgb}{0.631373,0.631373,0.631373}%
\pgfsetstrokecolor{currentstroke}%
\pgfsetstrokeopacity{0.100000}%
\pgfsetdash{}{0pt}%
\pgfpathmoveto{\pgfqpoint{0.800000in}{2.003732in}}%
\pgfpathlineto{\pgfqpoint{5.760000in}{2.003732in}}%
\pgfusepath{stroke}%
\end{pgfscope}%
\begin{pgfscope}%
\pgfsetbuttcap%
\pgfsetroundjoin%
\definecolor{currentfill}{rgb}{0.333333,0.333333,0.333333}%
\pgfsetfillcolor{currentfill}%
\pgfsetlinewidth{0.803000pt}%
\definecolor{currentstroke}{rgb}{0.333333,0.333333,0.333333}%
\pgfsetstrokecolor{currentstroke}%
\pgfsetdash{}{0pt}%
\pgfsys@defobject{currentmarker}{\pgfqpoint{-0.048611in}{0.000000in}}{\pgfqpoint{0.000000in}{0.000000in}}{%
\pgfpathmoveto{\pgfqpoint{0.000000in}{0.000000in}}%
\pgfpathlineto{\pgfqpoint{-0.048611in}{0.000000in}}%
\pgfusepath{stroke,fill}%
}%
\begin{pgfscope}%
\pgfsys@transformshift{0.800000in}{2.003732in}%
\pgfsys@useobject{currentmarker}{}%
\end{pgfscope}%
\end{pgfscope}%
\begin{pgfscope}%
\definecolor{textcolor}{rgb}{0.333333,0.333333,0.333333}%
\pgfsetstrokecolor{textcolor}%
\pgfsetfillcolor{textcolor}%
\pgftext[x=0.484030in,y=1.955507in,left,base]{\color{textcolor}\sffamily\fontsize{10.000000}{12.000000}\selectfont 750}%
\end{pgfscope}%
\begin{pgfscope}%
\pgfpathrectangle{\pgfqpoint{0.800000in}{0.528000in}}{\pgfqpoint{4.960000in}{3.696000in}} %
\pgfusepath{clip}%
\pgfsetrectcap%
\pgfsetroundjoin%
\pgfsetlinewidth{0.803000pt}%
\definecolor{currentstroke}{rgb}{0.631373,0.631373,0.631373}%
\pgfsetstrokecolor{currentstroke}%
\pgfsetstrokeopacity{0.100000}%
\pgfsetdash{}{0pt}%
\pgfpathmoveto{\pgfqpoint{0.800000in}{2.439643in}}%
\pgfpathlineto{\pgfqpoint{5.760000in}{2.439643in}}%
\pgfusepath{stroke}%
\end{pgfscope}%
\begin{pgfscope}%
\pgfsetbuttcap%
\pgfsetroundjoin%
\definecolor{currentfill}{rgb}{0.333333,0.333333,0.333333}%
\pgfsetfillcolor{currentfill}%
\pgfsetlinewidth{0.803000pt}%
\definecolor{currentstroke}{rgb}{0.333333,0.333333,0.333333}%
\pgfsetstrokecolor{currentstroke}%
\pgfsetdash{}{0pt}%
\pgfsys@defobject{currentmarker}{\pgfqpoint{-0.048611in}{0.000000in}}{\pgfqpoint{0.000000in}{0.000000in}}{%
\pgfpathmoveto{\pgfqpoint{0.000000in}{0.000000in}}%
\pgfpathlineto{\pgfqpoint{-0.048611in}{0.000000in}}%
\pgfusepath{stroke,fill}%
}%
\begin{pgfscope}%
\pgfsys@transformshift{0.800000in}{2.439643in}%
\pgfsys@useobject{currentmarker}{}%
\end{pgfscope}%
\end{pgfscope}%
\begin{pgfscope}%
\definecolor{textcolor}{rgb}{0.333333,0.333333,0.333333}%
\pgfsetstrokecolor{textcolor}%
\pgfsetfillcolor{textcolor}%
\pgftext[x=0.411114in,y=2.391418in,left,base]{\color{textcolor}\sffamily\fontsize{10.000000}{12.000000}\selectfont 1000}%
\end{pgfscope}%
\begin{pgfscope}%
\pgfpathrectangle{\pgfqpoint{0.800000in}{0.528000in}}{\pgfqpoint{4.960000in}{3.696000in}} %
\pgfusepath{clip}%
\pgfsetrectcap%
\pgfsetroundjoin%
\pgfsetlinewidth{0.803000pt}%
\definecolor{currentstroke}{rgb}{0.631373,0.631373,0.631373}%
\pgfsetstrokecolor{currentstroke}%
\pgfsetstrokeopacity{0.100000}%
\pgfsetdash{}{0pt}%
\pgfpathmoveto{\pgfqpoint{0.800000in}{2.875554in}}%
\pgfpathlineto{\pgfqpoint{5.760000in}{2.875554in}}%
\pgfusepath{stroke}%
\end{pgfscope}%
\begin{pgfscope}%
\pgfsetbuttcap%
\pgfsetroundjoin%
\definecolor{currentfill}{rgb}{0.333333,0.333333,0.333333}%
\pgfsetfillcolor{currentfill}%
\pgfsetlinewidth{0.803000pt}%
\definecolor{currentstroke}{rgb}{0.333333,0.333333,0.333333}%
\pgfsetstrokecolor{currentstroke}%
\pgfsetdash{}{0pt}%
\pgfsys@defobject{currentmarker}{\pgfqpoint{-0.048611in}{0.000000in}}{\pgfqpoint{0.000000in}{0.000000in}}{%
\pgfpathmoveto{\pgfqpoint{0.000000in}{0.000000in}}%
\pgfpathlineto{\pgfqpoint{-0.048611in}{0.000000in}}%
\pgfusepath{stroke,fill}%
}%
\begin{pgfscope}%
\pgfsys@transformshift{0.800000in}{2.875554in}%
\pgfsys@useobject{currentmarker}{}%
\end{pgfscope}%
\end{pgfscope}%
\begin{pgfscope}%
\definecolor{textcolor}{rgb}{0.333333,0.333333,0.333333}%
\pgfsetstrokecolor{textcolor}%
\pgfsetfillcolor{textcolor}%
\pgftext[x=0.411114in,y=2.827328in,left,base]{\color{textcolor}\sffamily\fontsize{10.000000}{12.000000}\selectfont 1250}%
\end{pgfscope}%
\begin{pgfscope}%
\pgfpathrectangle{\pgfqpoint{0.800000in}{0.528000in}}{\pgfqpoint{4.960000in}{3.696000in}} %
\pgfusepath{clip}%
\pgfsetrectcap%
\pgfsetroundjoin%
\pgfsetlinewidth{0.803000pt}%
\definecolor{currentstroke}{rgb}{0.631373,0.631373,0.631373}%
\pgfsetstrokecolor{currentstroke}%
\pgfsetstrokeopacity{0.100000}%
\pgfsetdash{}{0pt}%
\pgfpathmoveto{\pgfqpoint{0.800000in}{3.311464in}}%
\pgfpathlineto{\pgfqpoint{5.760000in}{3.311464in}}%
\pgfusepath{stroke}%
\end{pgfscope}%
\begin{pgfscope}%
\pgfsetbuttcap%
\pgfsetroundjoin%
\definecolor{currentfill}{rgb}{0.333333,0.333333,0.333333}%
\pgfsetfillcolor{currentfill}%
\pgfsetlinewidth{0.803000pt}%
\definecolor{currentstroke}{rgb}{0.333333,0.333333,0.333333}%
\pgfsetstrokecolor{currentstroke}%
\pgfsetdash{}{0pt}%
\pgfsys@defobject{currentmarker}{\pgfqpoint{-0.048611in}{0.000000in}}{\pgfqpoint{0.000000in}{0.000000in}}{%
\pgfpathmoveto{\pgfqpoint{0.000000in}{0.000000in}}%
\pgfpathlineto{\pgfqpoint{-0.048611in}{0.000000in}}%
\pgfusepath{stroke,fill}%
}%
\begin{pgfscope}%
\pgfsys@transformshift{0.800000in}{3.311464in}%
\pgfsys@useobject{currentmarker}{}%
\end{pgfscope}%
\end{pgfscope}%
\begin{pgfscope}%
\definecolor{textcolor}{rgb}{0.333333,0.333333,0.333333}%
\pgfsetstrokecolor{textcolor}%
\pgfsetfillcolor{textcolor}%
\pgftext[x=0.411114in,y=3.263239in,left,base]{\color{textcolor}\sffamily\fontsize{10.000000}{12.000000}\selectfont 1500}%
\end{pgfscope}%
\begin{pgfscope}%
\pgfpathrectangle{\pgfqpoint{0.800000in}{0.528000in}}{\pgfqpoint{4.960000in}{3.696000in}} %
\pgfusepath{clip}%
\pgfsetrectcap%
\pgfsetroundjoin%
\pgfsetlinewidth{0.803000pt}%
\definecolor{currentstroke}{rgb}{0.631373,0.631373,0.631373}%
\pgfsetstrokecolor{currentstroke}%
\pgfsetstrokeopacity{0.100000}%
\pgfsetdash{}{0pt}%
\pgfpathmoveto{\pgfqpoint{0.800000in}{3.747375in}}%
\pgfpathlineto{\pgfqpoint{5.760000in}{3.747375in}}%
\pgfusepath{stroke}%
\end{pgfscope}%
\begin{pgfscope}%
\pgfsetbuttcap%
\pgfsetroundjoin%
\definecolor{currentfill}{rgb}{0.333333,0.333333,0.333333}%
\pgfsetfillcolor{currentfill}%
\pgfsetlinewidth{0.803000pt}%
\definecolor{currentstroke}{rgb}{0.333333,0.333333,0.333333}%
\pgfsetstrokecolor{currentstroke}%
\pgfsetdash{}{0pt}%
\pgfsys@defobject{currentmarker}{\pgfqpoint{-0.048611in}{0.000000in}}{\pgfqpoint{0.000000in}{0.000000in}}{%
\pgfpathmoveto{\pgfqpoint{0.000000in}{0.000000in}}%
\pgfpathlineto{\pgfqpoint{-0.048611in}{0.000000in}}%
\pgfusepath{stroke,fill}%
}%
\begin{pgfscope}%
\pgfsys@transformshift{0.800000in}{3.747375in}%
\pgfsys@useobject{currentmarker}{}%
\end{pgfscope}%
\end{pgfscope}%
\begin{pgfscope}%
\definecolor{textcolor}{rgb}{0.333333,0.333333,0.333333}%
\pgfsetstrokecolor{textcolor}%
\pgfsetfillcolor{textcolor}%
\pgftext[x=0.411114in,y=3.699150in,left,base]{\color{textcolor}\sffamily\fontsize{10.000000}{12.000000}\selectfont 1750}%
\end{pgfscope}%
\begin{pgfscope}%
\pgfpathrectangle{\pgfqpoint{0.800000in}{0.528000in}}{\pgfqpoint{4.960000in}{3.696000in}} %
\pgfusepath{clip}%
\pgfsetrectcap%
\pgfsetroundjoin%
\pgfsetlinewidth{0.803000pt}%
\definecolor{currentstroke}{rgb}{0.631373,0.631373,0.631373}%
\pgfsetstrokecolor{currentstroke}%
\pgfsetstrokeopacity{0.100000}%
\pgfsetdash{}{0pt}%
\pgfpathmoveto{\pgfqpoint{0.800000in}{4.183286in}}%
\pgfpathlineto{\pgfqpoint{5.760000in}{4.183286in}}%
\pgfusepath{stroke}%
\end{pgfscope}%
\begin{pgfscope}%
\pgfsetbuttcap%
\pgfsetroundjoin%
\definecolor{currentfill}{rgb}{0.333333,0.333333,0.333333}%
\pgfsetfillcolor{currentfill}%
\pgfsetlinewidth{0.803000pt}%
\definecolor{currentstroke}{rgb}{0.333333,0.333333,0.333333}%
\pgfsetstrokecolor{currentstroke}%
\pgfsetdash{}{0pt}%
\pgfsys@defobject{currentmarker}{\pgfqpoint{-0.048611in}{0.000000in}}{\pgfqpoint{0.000000in}{0.000000in}}{%
\pgfpathmoveto{\pgfqpoint{0.000000in}{0.000000in}}%
\pgfpathlineto{\pgfqpoint{-0.048611in}{0.000000in}}%
\pgfusepath{stroke,fill}%
}%
\begin{pgfscope}%
\pgfsys@transformshift{0.800000in}{4.183286in}%
\pgfsys@useobject{currentmarker}{}%
\end{pgfscope}%
\end{pgfscope}%
\begin{pgfscope}%
\definecolor{textcolor}{rgb}{0.333333,0.333333,0.333333}%
\pgfsetstrokecolor{textcolor}%
\pgfsetfillcolor{textcolor}%
\pgftext[x=0.411114in,y=4.135061in,left,base]{\color{textcolor}\sffamily\fontsize{10.000000}{12.000000}\selectfont 2000}%
\end{pgfscope}%
\begin{pgfscope}%
\definecolor{textcolor}{rgb}{0.333333,0.333333,0.333333}%
\pgfsetstrokecolor{textcolor}%
\pgfsetfillcolor{textcolor}%
\pgftext[x=0.355558in,y=2.376000in,,bottom,rotate=90.000000]{\color{textcolor}\sffamily\fontsize{12.000000}{14.400000}\selectfont Número de Vehículos en Simulación}%
\end{pgfscope}%
\begin{pgfscope}%
\pgfpathrectangle{\pgfqpoint{0.800000in}{0.528000in}}{\pgfqpoint{4.960000in}{3.696000in}} %
\pgfusepath{clip}%
\pgfsetrectcap%
\pgfsetroundjoin%
\pgfsetlinewidth{1.505625pt}%
\definecolor{currentstroke}{rgb}{0.886275,0.290196,0.200000}%
\pgfsetstrokecolor{currentstroke}%
\pgfsetdash{}{0pt}%
\pgfpathmoveto{\pgfqpoint{1.025455in}{0.696000in}}%
\pgfpathlineto{\pgfqpoint{1.136516in}{1.346379in}}%
\pgfpathlineto{\pgfqpoint{1.203153in}{1.820650in}}%
\pgfpathlineto{\pgfqpoint{1.301521in}{2.169378in}}%
\pgfpathlineto{\pgfqpoint{1.453834in}{2.535543in}}%
\pgfpathlineto{\pgfqpoint{1.656918in}{2.668060in}}%
\pgfpathlineto{\pgfqpoint{1.898079in}{2.896477in}}%
\pgfpathlineto{\pgfqpoint{2.183666in}{3.062124in}}%
\pgfpathlineto{\pgfqpoint{2.504157in}{3.213820in}}%
\pgfpathlineto{\pgfqpoint{2.853207in}{3.280079in}}%
\pgfpathlineto{\pgfqpoint{3.224469in}{3.447469in}}%
\pgfpathlineto{\pgfqpoint{3.621117in}{3.478854in}}%
\pgfpathlineto{\pgfqpoint{4.055843in}{3.726451in}}%
\pgfpathlineto{\pgfqpoint{4.512781in}{3.728195in}}%
\pgfpathlineto{\pgfqpoint{4.991931in}{3.981023in}}%
\pgfpathlineto{\pgfqpoint{5.534545in}{4.056000in}}%
\pgfusepath{stroke}%
\end{pgfscope}%
\begin{pgfscope}%
\pgfpathrectangle{\pgfqpoint{0.800000in}{0.528000in}}{\pgfqpoint{4.960000in}{3.696000in}} %
\pgfusepath{clip}%
\pgfsetbuttcap%
\pgfsetroundjoin%
\definecolor{currentfill}{rgb}{0.886275,0.290196,0.200000}%
\pgfsetfillcolor{currentfill}%
\pgfsetlinewidth{1.003750pt}%
\definecolor{currentstroke}{rgb}{0.886275,0.290196,0.200000}%
\pgfsetstrokecolor{currentstroke}%
\pgfsetdash{}{0pt}%
\pgfsys@defobject{currentmarker}{\pgfqpoint{-0.020833in}{-0.020833in}}{\pgfqpoint{0.020833in}{0.020833in}}{%
\pgfpathmoveto{\pgfqpoint{0.000000in}{-0.020833in}}%
\pgfpathcurveto{\pgfqpoint{0.005525in}{-0.020833in}}{\pgfqpoint{0.010825in}{-0.018638in}}{\pgfqpoint{0.014731in}{-0.014731in}}%
\pgfpathcurveto{\pgfqpoint{0.018638in}{-0.010825in}}{\pgfqpoint{0.020833in}{-0.005525in}}{\pgfqpoint{0.020833in}{0.000000in}}%
\pgfpathcurveto{\pgfqpoint{0.020833in}{0.005525in}}{\pgfqpoint{0.018638in}{0.010825in}}{\pgfqpoint{0.014731in}{0.014731in}}%
\pgfpathcurveto{\pgfqpoint{0.010825in}{0.018638in}}{\pgfqpoint{0.005525in}{0.020833in}}{\pgfqpoint{0.000000in}{0.020833in}}%
\pgfpathcurveto{\pgfqpoint{-0.005525in}{0.020833in}}{\pgfqpoint{-0.010825in}{0.018638in}}{\pgfqpoint{-0.014731in}{0.014731in}}%
\pgfpathcurveto{\pgfqpoint{-0.018638in}{0.010825in}}{\pgfqpoint{-0.020833in}{0.005525in}}{\pgfqpoint{-0.020833in}{0.000000in}}%
\pgfpathcurveto{\pgfqpoint{-0.020833in}{-0.005525in}}{\pgfqpoint{-0.018638in}{-0.010825in}}{\pgfqpoint{-0.014731in}{-0.014731in}}%
\pgfpathcurveto{\pgfqpoint{-0.010825in}{-0.018638in}}{\pgfqpoint{-0.005525in}{-0.020833in}}{\pgfqpoint{0.000000in}{-0.020833in}}%
\pgfpathclose%
\pgfusepath{stroke,fill}%
}%
\begin{pgfscope}%
\pgfsys@transformshift{1.025455in}{0.696000in}%
\pgfsys@useobject{currentmarker}{}%
\end{pgfscope}%
\begin{pgfscope}%
\pgfsys@transformshift{1.136516in}{1.346379in}%
\pgfsys@useobject{currentmarker}{}%
\end{pgfscope}%
\begin{pgfscope}%
\pgfsys@transformshift{1.203153in}{1.820650in}%
\pgfsys@useobject{currentmarker}{}%
\end{pgfscope}%
\begin{pgfscope}%
\pgfsys@transformshift{1.301521in}{2.169378in}%
\pgfsys@useobject{currentmarker}{}%
\end{pgfscope}%
\begin{pgfscope}%
\pgfsys@transformshift{1.453834in}{2.535543in}%
\pgfsys@useobject{currentmarker}{}%
\end{pgfscope}%
\begin{pgfscope}%
\pgfsys@transformshift{1.656918in}{2.668060in}%
\pgfsys@useobject{currentmarker}{}%
\end{pgfscope}%
\begin{pgfscope}%
\pgfsys@transformshift{1.898079in}{2.896477in}%
\pgfsys@useobject{currentmarker}{}%
\end{pgfscope}%
\begin{pgfscope}%
\pgfsys@transformshift{2.183666in}{3.062124in}%
\pgfsys@useobject{currentmarker}{}%
\end{pgfscope}%
\begin{pgfscope}%
\pgfsys@transformshift{2.504157in}{3.213820in}%
\pgfsys@useobject{currentmarker}{}%
\end{pgfscope}%
\begin{pgfscope}%
\pgfsys@transformshift{2.853207in}{3.280079in}%
\pgfsys@useobject{currentmarker}{}%
\end{pgfscope}%
\begin{pgfscope}%
\pgfsys@transformshift{3.224469in}{3.447469in}%
\pgfsys@useobject{currentmarker}{}%
\end{pgfscope}%
\begin{pgfscope}%
\pgfsys@transformshift{3.621117in}{3.478854in}%
\pgfsys@useobject{currentmarker}{}%
\end{pgfscope}%
\begin{pgfscope}%
\pgfsys@transformshift{4.055843in}{3.726451in}%
\pgfsys@useobject{currentmarker}{}%
\end{pgfscope}%
\begin{pgfscope}%
\pgfsys@transformshift{4.512781in}{3.728195in}%
\pgfsys@useobject{currentmarker}{}%
\end{pgfscope}%
\begin{pgfscope}%
\pgfsys@transformshift{4.991931in}{3.981023in}%
\pgfsys@useobject{currentmarker}{}%
\end{pgfscope}%
\begin{pgfscope}%
\pgfsys@transformshift{5.534545in}{4.056000in}%
\pgfsys@useobject{currentmarker}{}%
\end{pgfscope}%
\end{pgfscope}%
\begin{pgfscope}%
\pgfpathrectangle{\pgfqpoint{0.800000in}{0.528000in}}{\pgfqpoint{4.960000in}{3.696000in}} %
\pgfusepath{clip}%
\pgfsetrectcap%
\pgfsetroundjoin%
\pgfsetlinewidth{1.505625pt}%
\definecolor{currentstroke}{rgb}{0.203922,0.541176,0.741176}%
\pgfsetstrokecolor{currentstroke}%
\pgfsetdash{}{0pt}%
\pgfpathmoveto{\pgfqpoint{1.025455in}{0.696000in}}%
\pgfpathlineto{\pgfqpoint{1.215845in}{1.346379in}}%
\pgfpathlineto{\pgfqpoint{1.406236in}{1.820650in}}%
\pgfpathlineto{\pgfqpoint{1.596627in}{2.169378in}}%
\pgfpathlineto{\pgfqpoint{1.787018in}{2.535543in}}%
\pgfpathlineto{\pgfqpoint{1.977409in}{2.668060in}}%
\pgfpathlineto{\pgfqpoint{2.167800in}{2.896477in}}%
\pgfpathlineto{\pgfqpoint{2.358191in}{3.062124in}}%
\pgfpathlineto{\pgfqpoint{2.548582in}{3.213820in}}%
\pgfpathlineto{\pgfqpoint{2.738973in}{3.280079in}}%
\pgfpathlineto{\pgfqpoint{2.929363in}{3.447469in}}%
\pgfpathlineto{\pgfqpoint{3.119754in}{3.478854in}}%
\pgfpathlineto{\pgfqpoint{3.310145in}{3.726451in}}%
\pgfpathlineto{\pgfqpoint{3.500536in}{3.728195in}}%
\pgfpathlineto{\pgfqpoint{3.690927in}{3.981023in}}%
\pgfpathlineto{\pgfqpoint{3.881318in}{4.056000in}}%
\pgfusepath{stroke}%
\end{pgfscope}%
\begin{pgfscope}%
\pgfpathrectangle{\pgfqpoint{0.800000in}{0.528000in}}{\pgfqpoint{4.960000in}{3.696000in}} %
\pgfusepath{clip}%
\pgfsetbuttcap%
\pgfsetroundjoin%
\definecolor{currentfill}{rgb}{0.203922,0.541176,0.741176}%
\pgfsetfillcolor{currentfill}%
\pgfsetlinewidth{1.003750pt}%
\definecolor{currentstroke}{rgb}{0.203922,0.541176,0.741176}%
\pgfsetstrokecolor{currentstroke}%
\pgfsetdash{}{0pt}%
\pgfsys@defobject{currentmarker}{\pgfqpoint{-0.020833in}{-0.020833in}}{\pgfqpoint{0.020833in}{0.020833in}}{%
\pgfpathmoveto{\pgfqpoint{0.000000in}{-0.020833in}}%
\pgfpathcurveto{\pgfqpoint{0.005525in}{-0.020833in}}{\pgfqpoint{0.010825in}{-0.018638in}}{\pgfqpoint{0.014731in}{-0.014731in}}%
\pgfpathcurveto{\pgfqpoint{0.018638in}{-0.010825in}}{\pgfqpoint{0.020833in}{-0.005525in}}{\pgfqpoint{0.020833in}{0.000000in}}%
\pgfpathcurveto{\pgfqpoint{0.020833in}{0.005525in}}{\pgfqpoint{0.018638in}{0.010825in}}{\pgfqpoint{0.014731in}{0.014731in}}%
\pgfpathcurveto{\pgfqpoint{0.010825in}{0.018638in}}{\pgfqpoint{0.005525in}{0.020833in}}{\pgfqpoint{0.000000in}{0.020833in}}%
\pgfpathcurveto{\pgfqpoint{-0.005525in}{0.020833in}}{\pgfqpoint{-0.010825in}{0.018638in}}{\pgfqpoint{-0.014731in}{0.014731in}}%
\pgfpathcurveto{\pgfqpoint{-0.018638in}{0.010825in}}{\pgfqpoint{-0.020833in}{0.005525in}}{\pgfqpoint{-0.020833in}{0.000000in}}%
\pgfpathcurveto{\pgfqpoint{-0.020833in}{-0.005525in}}{\pgfqpoint{-0.018638in}{-0.010825in}}{\pgfqpoint{-0.014731in}{-0.014731in}}%
\pgfpathcurveto{\pgfqpoint{-0.010825in}{-0.018638in}}{\pgfqpoint{-0.005525in}{-0.020833in}}{\pgfqpoint{0.000000in}{-0.020833in}}%
\pgfpathclose%
\pgfusepath{stroke,fill}%
}%
\begin{pgfscope}%
\pgfsys@transformshift{1.025455in}{0.696000in}%
\pgfsys@useobject{currentmarker}{}%
\end{pgfscope}%
\begin{pgfscope}%
\pgfsys@transformshift{1.215845in}{1.346379in}%
\pgfsys@useobject{currentmarker}{}%
\end{pgfscope}%
\begin{pgfscope}%
\pgfsys@transformshift{1.406236in}{1.820650in}%
\pgfsys@useobject{currentmarker}{}%
\end{pgfscope}%
\begin{pgfscope}%
\pgfsys@transformshift{1.596627in}{2.169378in}%
\pgfsys@useobject{currentmarker}{}%
\end{pgfscope}%
\begin{pgfscope}%
\pgfsys@transformshift{1.787018in}{2.535543in}%
\pgfsys@useobject{currentmarker}{}%
\end{pgfscope}%
\begin{pgfscope}%
\pgfsys@transformshift{1.977409in}{2.668060in}%
\pgfsys@useobject{currentmarker}{}%
\end{pgfscope}%
\begin{pgfscope}%
\pgfsys@transformshift{2.167800in}{2.896477in}%
\pgfsys@useobject{currentmarker}{}%
\end{pgfscope}%
\begin{pgfscope}%
\pgfsys@transformshift{2.358191in}{3.062124in}%
\pgfsys@useobject{currentmarker}{}%
\end{pgfscope}%
\begin{pgfscope}%
\pgfsys@transformshift{2.548582in}{3.213820in}%
\pgfsys@useobject{currentmarker}{}%
\end{pgfscope}%
\begin{pgfscope}%
\pgfsys@transformshift{2.738973in}{3.280079in}%
\pgfsys@useobject{currentmarker}{}%
\end{pgfscope}%
\begin{pgfscope}%
\pgfsys@transformshift{2.929363in}{3.447469in}%
\pgfsys@useobject{currentmarker}{}%
\end{pgfscope}%
\begin{pgfscope}%
\pgfsys@transformshift{3.119754in}{3.478854in}%
\pgfsys@useobject{currentmarker}{}%
\end{pgfscope}%
\begin{pgfscope}%
\pgfsys@transformshift{3.310145in}{3.726451in}%
\pgfsys@useobject{currentmarker}{}%
\end{pgfscope}%
\begin{pgfscope}%
\pgfsys@transformshift{3.500536in}{3.728195in}%
\pgfsys@useobject{currentmarker}{}%
\end{pgfscope}%
\begin{pgfscope}%
\pgfsys@transformshift{3.690927in}{3.981023in}%
\pgfsys@useobject{currentmarker}{}%
\end{pgfscope}%
\begin{pgfscope}%
\pgfsys@transformshift{3.881318in}{4.056000in}%
\pgfsys@useobject{currentmarker}{}%
\end{pgfscope}%
\end{pgfscope}%
\begin{pgfscope}%
\pgfsetrectcap%
\pgfsetmiterjoin%
\pgfsetlinewidth{1.003750pt}%
\definecolor{currentstroke}{rgb}{1.000000,1.000000,1.000000}%
\pgfsetstrokecolor{currentstroke}%
\pgfsetdash{}{0pt}%
\pgfpathmoveto{\pgfqpoint{0.800000in}{0.528000in}}%
\pgfpathlineto{\pgfqpoint{0.800000in}{4.224000in}}%
\pgfusepath{stroke}%
\end{pgfscope}%
\begin{pgfscope}%
\pgfsetrectcap%
\pgfsetmiterjoin%
\pgfsetlinewidth{1.003750pt}%
\definecolor{currentstroke}{rgb}{1.000000,1.000000,1.000000}%
\pgfsetstrokecolor{currentstroke}%
\pgfsetdash{}{0pt}%
\pgfpathmoveto{\pgfqpoint{5.760000in}{0.528000in}}%
\pgfpathlineto{\pgfqpoint{5.760000in}{4.224000in}}%
\pgfusepath{stroke}%
\end{pgfscope}%
\begin{pgfscope}%
\pgfsetrectcap%
\pgfsetmiterjoin%
\pgfsetlinewidth{1.003750pt}%
\definecolor{currentstroke}{rgb}{1.000000,1.000000,1.000000}%
\pgfsetstrokecolor{currentstroke}%
\pgfsetdash{}{0pt}%
\pgfpathmoveto{\pgfqpoint{0.800000in}{0.528000in}}%
\pgfpathlineto{\pgfqpoint{5.760000in}{0.528000in}}%
\pgfusepath{stroke}%
\end{pgfscope}%
\begin{pgfscope}%
\pgfsetrectcap%
\pgfsetmiterjoin%
\pgfsetlinewidth{1.003750pt}%
\definecolor{currentstroke}{rgb}{1.000000,1.000000,1.000000}%
\pgfsetstrokecolor{currentstroke}%
\pgfsetdash{}{0pt}%
\pgfpathmoveto{\pgfqpoint{0.800000in}{4.224000in}}%
\pgfpathlineto{\pgfqpoint{5.760000in}{4.224000in}}%
\pgfusepath{stroke}%
\end{pgfscope}%
\begin{pgfscope}%
\pgfsetbuttcap%
\pgfsetmiterjoin%
\definecolor{currentfill}{rgb}{0.898039,0.898039,0.898039}%
\pgfsetfillcolor{currentfill}%
\pgfsetfillopacity{0.800000}%
\pgfsetlinewidth{0.501875pt}%
\definecolor{currentstroke}{rgb}{0.800000,0.800000,0.800000}%
\pgfsetstrokecolor{currentstroke}%
\pgfsetstrokeopacity{0.800000}%
\pgfsetdash{}{0pt}%
\pgfpathmoveto{\pgfqpoint{4.177949in}{0.597444in}}%
\pgfpathlineto{\pgfqpoint{5.662778in}{0.597444in}}%
\pgfpathquadraticcurveto{\pgfqpoint{5.690556in}{0.597444in}}{\pgfqpoint{5.690556in}{0.625222in}}%
\pgfpathlineto{\pgfqpoint{5.690556in}{0.990963in}}%
\pgfpathquadraticcurveto{\pgfqpoint{5.690556in}{1.018741in}}{\pgfqpoint{5.662778in}{1.018741in}}%
\pgfpathlineto{\pgfqpoint{4.177949in}{1.018741in}}%
\pgfpathquadraticcurveto{\pgfqpoint{4.150171in}{1.018741in}}{\pgfqpoint{4.150171in}{0.990963in}}%
\pgfpathlineto{\pgfqpoint{4.150171in}{0.625222in}}%
\pgfpathquadraticcurveto{\pgfqpoint{4.150171in}{0.597444in}}{\pgfqpoint{4.177949in}{0.597444in}}%
\pgfpathclose%
\pgfusepath{stroke,fill}%
\end{pgfscope}%
\begin{pgfscope}%
\pgfsetrectcap%
\pgfsetroundjoin%
\pgfsetlinewidth{1.505625pt}%
\definecolor{currentstroke}{rgb}{0.886275,0.290196,0.200000}%
\pgfsetstrokecolor{currentstroke}%
\pgfsetdash{}{0pt}%
\pgfpathmoveto{\pgfqpoint{4.205727in}{0.914574in}}%
\pgfpathlineto{\pgfqpoint{4.483504in}{0.914574in}}%
\pgfusepath{stroke}%
\end{pgfscope}%
\begin{pgfscope}%
\pgfsetbuttcap%
\pgfsetroundjoin%
\definecolor{currentfill}{rgb}{0.886275,0.290196,0.200000}%
\pgfsetfillcolor{currentfill}%
\pgfsetlinewidth{1.003750pt}%
\definecolor{currentstroke}{rgb}{0.886275,0.290196,0.200000}%
\pgfsetstrokecolor{currentstroke}%
\pgfsetdash{}{0pt}%
\pgfsys@defobject{currentmarker}{\pgfqpoint{-0.020833in}{-0.020833in}}{\pgfqpoint{0.020833in}{0.020833in}}{%
\pgfpathmoveto{\pgfqpoint{0.000000in}{-0.020833in}}%
\pgfpathcurveto{\pgfqpoint{0.005525in}{-0.020833in}}{\pgfqpoint{0.010825in}{-0.018638in}}{\pgfqpoint{0.014731in}{-0.014731in}}%
\pgfpathcurveto{\pgfqpoint{0.018638in}{-0.010825in}}{\pgfqpoint{0.020833in}{-0.005525in}}{\pgfqpoint{0.020833in}{0.000000in}}%
\pgfpathcurveto{\pgfqpoint{0.020833in}{0.005525in}}{\pgfqpoint{0.018638in}{0.010825in}}{\pgfqpoint{0.014731in}{0.014731in}}%
\pgfpathcurveto{\pgfqpoint{0.010825in}{0.018638in}}{\pgfqpoint{0.005525in}{0.020833in}}{\pgfqpoint{0.000000in}{0.020833in}}%
\pgfpathcurveto{\pgfqpoint{-0.005525in}{0.020833in}}{\pgfqpoint{-0.010825in}{0.018638in}}{\pgfqpoint{-0.014731in}{0.014731in}}%
\pgfpathcurveto{\pgfqpoint{-0.018638in}{0.010825in}}{\pgfqpoint{-0.020833in}{0.005525in}}{\pgfqpoint{-0.020833in}{0.000000in}}%
\pgfpathcurveto{\pgfqpoint{-0.020833in}{-0.005525in}}{\pgfqpoint{-0.018638in}{-0.010825in}}{\pgfqpoint{-0.014731in}{-0.014731in}}%
\pgfpathcurveto{\pgfqpoint{-0.010825in}{-0.018638in}}{\pgfqpoint{-0.005525in}{-0.020833in}}{\pgfqpoint{0.000000in}{-0.020833in}}%
\pgfpathclose%
\pgfusepath{stroke,fill}%
}%
\begin{pgfscope}%
\pgfsys@transformshift{4.344616in}{0.914574in}%
\pgfsys@useobject{currentmarker}{}%
\end{pgfscope}%
\end{pgfscope}%
\begin{pgfscope}%
\pgftext[x=4.594616in,y=0.865963in,left,base]{\sffamily\fontsize{10.000000}{12.000000}\selectfont Tiempo Real}%
\end{pgfscope}%
\begin{pgfscope}%
\pgfsetrectcap%
\pgfsetroundjoin%
\pgfsetlinewidth{1.505625pt}%
\definecolor{currentstroke}{rgb}{0.203922,0.541176,0.741176}%
\pgfsetstrokecolor{currentstroke}%
\pgfsetdash{}{0pt}%
\pgfpathmoveto{\pgfqpoint{4.205727in}{0.724759in}}%
\pgfpathlineto{\pgfqpoint{4.483504in}{0.724759in}}%
\pgfusepath{stroke}%
\end{pgfscope}%
\begin{pgfscope}%
\pgfsetbuttcap%
\pgfsetroundjoin%
\definecolor{currentfill}{rgb}{0.203922,0.541176,0.741176}%
\pgfsetfillcolor{currentfill}%
\pgfsetlinewidth{1.003750pt}%
\definecolor{currentstroke}{rgb}{0.203922,0.541176,0.741176}%
\pgfsetstrokecolor{currentstroke}%
\pgfsetdash{}{0pt}%
\pgfsys@defobject{currentmarker}{\pgfqpoint{-0.020833in}{-0.020833in}}{\pgfqpoint{0.020833in}{0.020833in}}{%
\pgfpathmoveto{\pgfqpoint{0.000000in}{-0.020833in}}%
\pgfpathcurveto{\pgfqpoint{0.005525in}{-0.020833in}}{\pgfqpoint{0.010825in}{-0.018638in}}{\pgfqpoint{0.014731in}{-0.014731in}}%
\pgfpathcurveto{\pgfqpoint{0.018638in}{-0.010825in}}{\pgfqpoint{0.020833in}{-0.005525in}}{\pgfqpoint{0.020833in}{0.000000in}}%
\pgfpathcurveto{\pgfqpoint{0.020833in}{0.005525in}}{\pgfqpoint{0.018638in}{0.010825in}}{\pgfqpoint{0.014731in}{0.014731in}}%
\pgfpathcurveto{\pgfqpoint{0.010825in}{0.018638in}}{\pgfqpoint{0.005525in}{0.020833in}}{\pgfqpoint{0.000000in}{0.020833in}}%
\pgfpathcurveto{\pgfqpoint{-0.005525in}{0.020833in}}{\pgfqpoint{-0.010825in}{0.018638in}}{\pgfqpoint{-0.014731in}{0.014731in}}%
\pgfpathcurveto{\pgfqpoint{-0.018638in}{0.010825in}}{\pgfqpoint{-0.020833in}{0.005525in}}{\pgfqpoint{-0.020833in}{0.000000in}}%
\pgfpathcurveto{\pgfqpoint{-0.020833in}{-0.005525in}}{\pgfqpoint{-0.018638in}{-0.010825in}}{\pgfqpoint{-0.014731in}{-0.014731in}}%
\pgfpathcurveto{\pgfqpoint{-0.010825in}{-0.018638in}}{\pgfqpoint{-0.005525in}{-0.020833in}}{\pgfqpoint{0.000000in}{-0.020833in}}%
\pgfpathclose%
\pgfusepath{stroke,fill}%
}%
\begin{pgfscope}%
\pgfsys@transformshift{4.344616in}{0.724759in}%
\pgfsys@useobject{currentmarker}{}%
\end{pgfscope}%
\end{pgfscope}%
\begin{pgfscope}%
\pgftext[x=4.594616in,y=0.676148in,left,base]{\sffamily\fontsize{10.000000}{12.000000}\selectfont Tiempo Simulado}%
\end{pgfscope}%
\end{pgfpicture}%
\makeatother%
\endgroup%

    \caption[Evolución temporal de la cantidad de vehículos en la simulación.]{Evolución de la cantidad de vehículos en una simulación con factor de demanda 100\%, para tiempo real y simulado.}
    \label{fig:timevsvehicles_evolution}
\end{figure}

Por otro lado, en términos de carga sobre el entorno de simulación, se pueden observar los resultados obtenidos en las figuras \ref{fig:systemload:cpuram} y \ref{fig:systemload:io}. 

La figura \ref{fig:systemload:cpuram} ilustra la carga sobre el sistema en términos porcentuales. En específico, se puede observar como el uso promedio del procesador aumenta en aproximadamente un 20\% durante la simulación, situación fácilmente manejable para cualquier procesador moderno. Además, el uso de memoria aumenta en menos de un 5\% -- en términos numéricos, el sistema utiliza menos de 600 MB para simular un escenario con un promedio de 1400 nodos presentes en cualquier instante, lo cual es un valor muy razonable si se considera que el estándar de memoria RAM para computadores personales hoy en día es por lo menos 4 GB \autocite{steamhwsurvey, unityhardwaresurvey}.

\begin{figure}[h]
    \centering
    %% Creator: Matplotlib, PGF backend
%%
%% To include the figure in your LaTeX document, write
%%   \input{<filename>.pgf}
%%
%% Make sure the required packages are loaded in your preamble
%%   \usepackage{pgf}
%%
%% Figures using additional raster images can only be included by \input if
%% they are in the same directory as the main LaTeX file. For loading figures
%% from other directories you can use the `import` package
%%   \usepackage{import}
%% and then include the figures with
%%   \import{<path to file>}{<filename>.pgf}
%%
%% Matplotlib used the following preamble
%%   \usepackage[utf8x]{inputenc}
%%   \usepackage[T1]{fontenc}
%%   \usepackage{cmbright}
%%
\begingroup%
\makeatletter%
\begin{pgfpicture}%
\pgfpathrectangle{\pgfpointorigin}{\pgfqpoint{6.400000in}{4.800000in}}%
\pgfusepath{use as bounding box, clip}%
\begin{pgfscope}%
\pgfsetbuttcap%
\pgfsetmiterjoin%
\definecolor{currentfill}{rgb}{1.000000,1.000000,1.000000}%
\pgfsetfillcolor{currentfill}%
\pgfsetlinewidth{0.000000pt}%
\definecolor{currentstroke}{rgb}{1.000000,1.000000,1.000000}%
\pgfsetstrokecolor{currentstroke}%
\pgfsetdash{}{0pt}%
\pgfpathmoveto{\pgfqpoint{0.000000in}{0.000000in}}%
\pgfpathlineto{\pgfqpoint{6.400000in}{0.000000in}}%
\pgfpathlineto{\pgfqpoint{6.400000in}{4.800000in}}%
\pgfpathlineto{\pgfqpoint{0.000000in}{4.800000in}}%
\pgfpathclose%
\pgfusepath{fill}%
\end{pgfscope}%
\begin{pgfscope}%
\pgfsetbuttcap%
\pgfsetmiterjoin%
\definecolor{currentfill}{rgb}{1.000000,1.000000,1.000000}%
\pgfsetfillcolor{currentfill}%
\pgfsetlinewidth{0.000000pt}%
\definecolor{currentstroke}{rgb}{0.000000,0.000000,0.000000}%
\pgfsetstrokecolor{currentstroke}%
\pgfsetstrokeopacity{0.000000}%
\pgfsetdash{}{0pt}%
\pgfpathmoveto{\pgfqpoint{0.800000in}{0.528000in}}%
\pgfpathlineto{\pgfqpoint{5.760000in}{0.528000in}}%
\pgfpathlineto{\pgfqpoint{5.760000in}{4.224000in}}%
\pgfpathlineto{\pgfqpoint{0.800000in}{4.224000in}}%
\pgfpathclose%
\pgfusepath{fill}%
\end{pgfscope}%
\begin{pgfscope}%
\pgfpathrectangle{\pgfqpoint{0.800000in}{0.528000in}}{\pgfqpoint{4.960000in}{3.696000in}} %
\pgfusepath{clip}%
\pgfsetrectcap%
\pgfsetroundjoin%
\pgfsetlinewidth{0.803000pt}%
\definecolor{currentstroke}{rgb}{0.631373,0.631373,0.631373}%
\pgfsetstrokecolor{currentstroke}%
\pgfsetstrokeopacity{0.100000}%
\pgfsetdash{}{0pt}%
\pgfpathmoveto{\pgfqpoint{0.800000in}{0.528000in}}%
\pgfpathlineto{\pgfqpoint{0.800000in}{4.224000in}}%
\pgfusepath{stroke}%
\end{pgfscope}%
\begin{pgfscope}%
\pgfsetbuttcap%
\pgfsetroundjoin%
\definecolor{currentfill}{rgb}{0.333333,0.333333,0.333333}%
\pgfsetfillcolor{currentfill}%
\pgfsetlinewidth{0.803000pt}%
\definecolor{currentstroke}{rgb}{0.333333,0.333333,0.333333}%
\pgfsetstrokecolor{currentstroke}%
\pgfsetdash{}{0pt}%
\pgfsys@defobject{currentmarker}{\pgfqpoint{0.000000in}{-0.048611in}}{\pgfqpoint{0.000000in}{0.000000in}}{%
\pgfpathmoveto{\pgfqpoint{0.000000in}{0.000000in}}%
\pgfpathlineto{\pgfqpoint{0.000000in}{-0.048611in}}%
\pgfusepath{stroke,fill}%
}%
\begin{pgfscope}%
\pgfsys@transformshift{0.800000in}{0.528000in}%
\pgfsys@useobject{currentmarker}{}%
\end{pgfscope}%
\end{pgfscope}%
\begin{pgfscope}%
\definecolor{textcolor}{rgb}{0.333333,0.333333,0.333333}%
\pgfsetstrokecolor{textcolor}%
\pgfsetfillcolor{textcolor}%
\pgftext[x=0.800000in,y=0.430778in,,top]{\color{textcolor}\sffamily\fontsize{10.000000}{12.000000}\selectfont 00:00}%
\end{pgfscope}%
\begin{pgfscope}%
\pgfpathrectangle{\pgfqpoint{0.800000in}{0.528000in}}{\pgfqpoint{4.960000in}{3.696000in}} %
\pgfusepath{clip}%
\pgfsetrectcap%
\pgfsetroundjoin%
\pgfsetlinewidth{0.803000pt}%
\definecolor{currentstroke}{rgb}{0.631373,0.631373,0.631373}%
\pgfsetstrokecolor{currentstroke}%
\pgfsetstrokeopacity{0.100000}%
\pgfsetdash{}{0pt}%
\pgfpathmoveto{\pgfqpoint{1.420000in}{0.528000in}}%
\pgfpathlineto{\pgfqpoint{1.420000in}{4.224000in}}%
\pgfusepath{stroke}%
\end{pgfscope}%
\begin{pgfscope}%
\pgfsetbuttcap%
\pgfsetroundjoin%
\definecolor{currentfill}{rgb}{0.333333,0.333333,0.333333}%
\pgfsetfillcolor{currentfill}%
\pgfsetlinewidth{0.803000pt}%
\definecolor{currentstroke}{rgb}{0.333333,0.333333,0.333333}%
\pgfsetstrokecolor{currentstroke}%
\pgfsetdash{}{0pt}%
\pgfsys@defobject{currentmarker}{\pgfqpoint{0.000000in}{-0.048611in}}{\pgfqpoint{0.000000in}{0.000000in}}{%
\pgfpathmoveto{\pgfqpoint{0.000000in}{0.000000in}}%
\pgfpathlineto{\pgfqpoint{0.000000in}{-0.048611in}}%
\pgfusepath{stroke,fill}%
}%
\begin{pgfscope}%
\pgfsys@transformshift{1.420000in}{0.528000in}%
\pgfsys@useobject{currentmarker}{}%
\end{pgfscope}%
\end{pgfscope}%
\begin{pgfscope}%
\definecolor{textcolor}{rgb}{0.333333,0.333333,0.333333}%
\pgfsetstrokecolor{textcolor}%
\pgfsetfillcolor{textcolor}%
\pgftext[x=1.420000in,y=0.430778in,,top]{\color{textcolor}\sffamily\fontsize{10.000000}{12.000000}\selectfont 03:20}%
\end{pgfscope}%
\begin{pgfscope}%
\pgfpathrectangle{\pgfqpoint{0.800000in}{0.528000in}}{\pgfqpoint{4.960000in}{3.696000in}} %
\pgfusepath{clip}%
\pgfsetrectcap%
\pgfsetroundjoin%
\pgfsetlinewidth{0.803000pt}%
\definecolor{currentstroke}{rgb}{0.631373,0.631373,0.631373}%
\pgfsetstrokecolor{currentstroke}%
\pgfsetstrokeopacity{0.100000}%
\pgfsetdash{}{0pt}%
\pgfpathmoveto{\pgfqpoint{2.040000in}{0.528000in}}%
\pgfpathlineto{\pgfqpoint{2.040000in}{4.224000in}}%
\pgfusepath{stroke}%
\end{pgfscope}%
\begin{pgfscope}%
\pgfsetbuttcap%
\pgfsetroundjoin%
\definecolor{currentfill}{rgb}{0.333333,0.333333,0.333333}%
\pgfsetfillcolor{currentfill}%
\pgfsetlinewidth{0.803000pt}%
\definecolor{currentstroke}{rgb}{0.333333,0.333333,0.333333}%
\pgfsetstrokecolor{currentstroke}%
\pgfsetdash{}{0pt}%
\pgfsys@defobject{currentmarker}{\pgfqpoint{0.000000in}{-0.048611in}}{\pgfqpoint{0.000000in}{0.000000in}}{%
\pgfpathmoveto{\pgfqpoint{0.000000in}{0.000000in}}%
\pgfpathlineto{\pgfqpoint{0.000000in}{-0.048611in}}%
\pgfusepath{stroke,fill}%
}%
\begin{pgfscope}%
\pgfsys@transformshift{2.040000in}{0.528000in}%
\pgfsys@useobject{currentmarker}{}%
\end{pgfscope}%
\end{pgfscope}%
\begin{pgfscope}%
\definecolor{textcolor}{rgb}{0.333333,0.333333,0.333333}%
\pgfsetstrokecolor{textcolor}%
\pgfsetfillcolor{textcolor}%
\pgftext[x=2.040000in,y=0.430778in,,top]{\color{textcolor}\sffamily\fontsize{10.000000}{12.000000}\selectfont 06:40}%
\end{pgfscope}%
\begin{pgfscope}%
\pgfpathrectangle{\pgfqpoint{0.800000in}{0.528000in}}{\pgfqpoint{4.960000in}{3.696000in}} %
\pgfusepath{clip}%
\pgfsetrectcap%
\pgfsetroundjoin%
\pgfsetlinewidth{0.803000pt}%
\definecolor{currentstroke}{rgb}{0.631373,0.631373,0.631373}%
\pgfsetstrokecolor{currentstroke}%
\pgfsetstrokeopacity{0.100000}%
\pgfsetdash{}{0pt}%
\pgfpathmoveto{\pgfqpoint{2.660000in}{0.528000in}}%
\pgfpathlineto{\pgfqpoint{2.660000in}{4.224000in}}%
\pgfusepath{stroke}%
\end{pgfscope}%
\begin{pgfscope}%
\pgfsetbuttcap%
\pgfsetroundjoin%
\definecolor{currentfill}{rgb}{0.333333,0.333333,0.333333}%
\pgfsetfillcolor{currentfill}%
\pgfsetlinewidth{0.803000pt}%
\definecolor{currentstroke}{rgb}{0.333333,0.333333,0.333333}%
\pgfsetstrokecolor{currentstroke}%
\pgfsetdash{}{0pt}%
\pgfsys@defobject{currentmarker}{\pgfqpoint{0.000000in}{-0.048611in}}{\pgfqpoint{0.000000in}{0.000000in}}{%
\pgfpathmoveto{\pgfqpoint{0.000000in}{0.000000in}}%
\pgfpathlineto{\pgfqpoint{0.000000in}{-0.048611in}}%
\pgfusepath{stroke,fill}%
}%
\begin{pgfscope}%
\pgfsys@transformshift{2.660000in}{0.528000in}%
\pgfsys@useobject{currentmarker}{}%
\end{pgfscope}%
\end{pgfscope}%
\begin{pgfscope}%
\definecolor{textcolor}{rgb}{0.333333,0.333333,0.333333}%
\pgfsetstrokecolor{textcolor}%
\pgfsetfillcolor{textcolor}%
\pgftext[x=2.660000in,y=0.430778in,,top]{\color{textcolor}\sffamily\fontsize{10.000000}{12.000000}\selectfont 10:00}%
\end{pgfscope}%
\begin{pgfscope}%
\pgfpathrectangle{\pgfqpoint{0.800000in}{0.528000in}}{\pgfqpoint{4.960000in}{3.696000in}} %
\pgfusepath{clip}%
\pgfsetrectcap%
\pgfsetroundjoin%
\pgfsetlinewidth{0.803000pt}%
\definecolor{currentstroke}{rgb}{0.631373,0.631373,0.631373}%
\pgfsetstrokecolor{currentstroke}%
\pgfsetstrokeopacity{0.100000}%
\pgfsetdash{}{0pt}%
\pgfpathmoveto{\pgfqpoint{3.280000in}{0.528000in}}%
\pgfpathlineto{\pgfqpoint{3.280000in}{4.224000in}}%
\pgfusepath{stroke}%
\end{pgfscope}%
\begin{pgfscope}%
\pgfsetbuttcap%
\pgfsetroundjoin%
\definecolor{currentfill}{rgb}{0.333333,0.333333,0.333333}%
\pgfsetfillcolor{currentfill}%
\pgfsetlinewidth{0.803000pt}%
\definecolor{currentstroke}{rgb}{0.333333,0.333333,0.333333}%
\pgfsetstrokecolor{currentstroke}%
\pgfsetdash{}{0pt}%
\pgfsys@defobject{currentmarker}{\pgfqpoint{0.000000in}{-0.048611in}}{\pgfqpoint{0.000000in}{0.000000in}}{%
\pgfpathmoveto{\pgfqpoint{0.000000in}{0.000000in}}%
\pgfpathlineto{\pgfqpoint{0.000000in}{-0.048611in}}%
\pgfusepath{stroke,fill}%
}%
\begin{pgfscope}%
\pgfsys@transformshift{3.280000in}{0.528000in}%
\pgfsys@useobject{currentmarker}{}%
\end{pgfscope}%
\end{pgfscope}%
\begin{pgfscope}%
\definecolor{textcolor}{rgb}{0.333333,0.333333,0.333333}%
\pgfsetstrokecolor{textcolor}%
\pgfsetfillcolor{textcolor}%
\pgftext[x=3.280000in,y=0.430778in,,top]{\color{textcolor}\sffamily\fontsize{10.000000}{12.000000}\selectfont 13:20}%
\end{pgfscope}%
\begin{pgfscope}%
\pgfpathrectangle{\pgfqpoint{0.800000in}{0.528000in}}{\pgfqpoint{4.960000in}{3.696000in}} %
\pgfusepath{clip}%
\pgfsetrectcap%
\pgfsetroundjoin%
\pgfsetlinewidth{0.803000pt}%
\definecolor{currentstroke}{rgb}{0.631373,0.631373,0.631373}%
\pgfsetstrokecolor{currentstroke}%
\pgfsetstrokeopacity{0.100000}%
\pgfsetdash{}{0pt}%
\pgfpathmoveto{\pgfqpoint{3.900000in}{0.528000in}}%
\pgfpathlineto{\pgfqpoint{3.900000in}{4.224000in}}%
\pgfusepath{stroke}%
\end{pgfscope}%
\begin{pgfscope}%
\pgfsetbuttcap%
\pgfsetroundjoin%
\definecolor{currentfill}{rgb}{0.333333,0.333333,0.333333}%
\pgfsetfillcolor{currentfill}%
\pgfsetlinewidth{0.803000pt}%
\definecolor{currentstroke}{rgb}{0.333333,0.333333,0.333333}%
\pgfsetstrokecolor{currentstroke}%
\pgfsetdash{}{0pt}%
\pgfsys@defobject{currentmarker}{\pgfqpoint{0.000000in}{-0.048611in}}{\pgfqpoint{0.000000in}{0.000000in}}{%
\pgfpathmoveto{\pgfqpoint{0.000000in}{0.000000in}}%
\pgfpathlineto{\pgfqpoint{0.000000in}{-0.048611in}}%
\pgfusepath{stroke,fill}%
}%
\begin{pgfscope}%
\pgfsys@transformshift{3.900000in}{0.528000in}%
\pgfsys@useobject{currentmarker}{}%
\end{pgfscope}%
\end{pgfscope}%
\begin{pgfscope}%
\definecolor{textcolor}{rgb}{0.333333,0.333333,0.333333}%
\pgfsetstrokecolor{textcolor}%
\pgfsetfillcolor{textcolor}%
\pgftext[x=3.900000in,y=0.430778in,,top]{\color{textcolor}\sffamily\fontsize{10.000000}{12.000000}\selectfont 16:40}%
\end{pgfscope}%
\begin{pgfscope}%
\pgfpathrectangle{\pgfqpoint{0.800000in}{0.528000in}}{\pgfqpoint{4.960000in}{3.696000in}} %
\pgfusepath{clip}%
\pgfsetrectcap%
\pgfsetroundjoin%
\pgfsetlinewidth{0.803000pt}%
\definecolor{currentstroke}{rgb}{0.631373,0.631373,0.631373}%
\pgfsetstrokecolor{currentstroke}%
\pgfsetstrokeopacity{0.100000}%
\pgfsetdash{}{0pt}%
\pgfpathmoveto{\pgfqpoint{4.520000in}{0.528000in}}%
\pgfpathlineto{\pgfqpoint{4.520000in}{4.224000in}}%
\pgfusepath{stroke}%
\end{pgfscope}%
\begin{pgfscope}%
\pgfsetbuttcap%
\pgfsetroundjoin%
\definecolor{currentfill}{rgb}{0.333333,0.333333,0.333333}%
\pgfsetfillcolor{currentfill}%
\pgfsetlinewidth{0.803000pt}%
\definecolor{currentstroke}{rgb}{0.333333,0.333333,0.333333}%
\pgfsetstrokecolor{currentstroke}%
\pgfsetdash{}{0pt}%
\pgfsys@defobject{currentmarker}{\pgfqpoint{0.000000in}{-0.048611in}}{\pgfqpoint{0.000000in}{0.000000in}}{%
\pgfpathmoveto{\pgfqpoint{0.000000in}{0.000000in}}%
\pgfpathlineto{\pgfqpoint{0.000000in}{-0.048611in}}%
\pgfusepath{stroke,fill}%
}%
\begin{pgfscope}%
\pgfsys@transformshift{4.520000in}{0.528000in}%
\pgfsys@useobject{currentmarker}{}%
\end{pgfscope}%
\end{pgfscope}%
\begin{pgfscope}%
\definecolor{textcolor}{rgb}{0.333333,0.333333,0.333333}%
\pgfsetstrokecolor{textcolor}%
\pgfsetfillcolor{textcolor}%
\pgftext[x=4.520000in,y=0.430778in,,top]{\color{textcolor}\sffamily\fontsize{10.000000}{12.000000}\selectfont 20:00}%
\end{pgfscope}%
\begin{pgfscope}%
\pgfpathrectangle{\pgfqpoint{0.800000in}{0.528000in}}{\pgfqpoint{4.960000in}{3.696000in}} %
\pgfusepath{clip}%
\pgfsetrectcap%
\pgfsetroundjoin%
\pgfsetlinewidth{0.803000pt}%
\definecolor{currentstroke}{rgb}{0.631373,0.631373,0.631373}%
\pgfsetstrokecolor{currentstroke}%
\pgfsetstrokeopacity{0.100000}%
\pgfsetdash{}{0pt}%
\pgfpathmoveto{\pgfqpoint{5.140000in}{0.528000in}}%
\pgfpathlineto{\pgfqpoint{5.140000in}{4.224000in}}%
\pgfusepath{stroke}%
\end{pgfscope}%
\begin{pgfscope}%
\pgfsetbuttcap%
\pgfsetroundjoin%
\definecolor{currentfill}{rgb}{0.333333,0.333333,0.333333}%
\pgfsetfillcolor{currentfill}%
\pgfsetlinewidth{0.803000pt}%
\definecolor{currentstroke}{rgb}{0.333333,0.333333,0.333333}%
\pgfsetstrokecolor{currentstroke}%
\pgfsetdash{}{0pt}%
\pgfsys@defobject{currentmarker}{\pgfqpoint{0.000000in}{-0.048611in}}{\pgfqpoint{0.000000in}{0.000000in}}{%
\pgfpathmoveto{\pgfqpoint{0.000000in}{0.000000in}}%
\pgfpathlineto{\pgfqpoint{0.000000in}{-0.048611in}}%
\pgfusepath{stroke,fill}%
}%
\begin{pgfscope}%
\pgfsys@transformshift{5.140000in}{0.528000in}%
\pgfsys@useobject{currentmarker}{}%
\end{pgfscope}%
\end{pgfscope}%
\begin{pgfscope}%
\definecolor{textcolor}{rgb}{0.333333,0.333333,0.333333}%
\pgfsetstrokecolor{textcolor}%
\pgfsetfillcolor{textcolor}%
\pgftext[x=5.140000in,y=0.430778in,,top]{\color{textcolor}\sffamily\fontsize{10.000000}{12.000000}\selectfont 23:20}%
\end{pgfscope}%
\begin{pgfscope}%
\pgfpathrectangle{\pgfqpoint{0.800000in}{0.528000in}}{\pgfqpoint{4.960000in}{3.696000in}} %
\pgfusepath{clip}%
\pgfsetrectcap%
\pgfsetroundjoin%
\pgfsetlinewidth{0.803000pt}%
\definecolor{currentstroke}{rgb}{0.631373,0.631373,0.631373}%
\pgfsetstrokecolor{currentstroke}%
\pgfsetstrokeopacity{0.100000}%
\pgfsetdash{}{0pt}%
\pgfpathmoveto{\pgfqpoint{5.760000in}{0.528000in}}%
\pgfpathlineto{\pgfqpoint{5.760000in}{4.224000in}}%
\pgfusepath{stroke}%
\end{pgfscope}%
\begin{pgfscope}%
\pgfsetbuttcap%
\pgfsetroundjoin%
\definecolor{currentfill}{rgb}{0.333333,0.333333,0.333333}%
\pgfsetfillcolor{currentfill}%
\pgfsetlinewidth{0.803000pt}%
\definecolor{currentstroke}{rgb}{0.333333,0.333333,0.333333}%
\pgfsetstrokecolor{currentstroke}%
\pgfsetdash{}{0pt}%
\pgfsys@defobject{currentmarker}{\pgfqpoint{0.000000in}{-0.048611in}}{\pgfqpoint{0.000000in}{0.000000in}}{%
\pgfpathmoveto{\pgfqpoint{0.000000in}{0.000000in}}%
\pgfpathlineto{\pgfqpoint{0.000000in}{-0.048611in}}%
\pgfusepath{stroke,fill}%
}%
\begin{pgfscope}%
\pgfsys@transformshift{5.760000in}{0.528000in}%
\pgfsys@useobject{currentmarker}{}%
\end{pgfscope}%
\end{pgfscope}%
\begin{pgfscope}%
\definecolor{textcolor}{rgb}{0.333333,0.333333,0.333333}%
\pgfsetstrokecolor{textcolor}%
\pgfsetfillcolor{textcolor}%
\pgftext[x=5.760000in,y=0.430778in,,top]{\color{textcolor}\sffamily\fontsize{10.000000}{12.000000}\selectfont 26:40}%
\end{pgfscope}%
\begin{pgfscope}%
\definecolor{textcolor}{rgb}{0.333333,0.333333,0.333333}%
\pgfsetstrokecolor{textcolor}%
\pgfsetfillcolor{textcolor}%
\pgftext[x=3.280000in,y=0.255624in,,top]{\color{textcolor}\sffamily\fontsize{12.000000}{14.400000}\selectfont Tiempo (MM:SS)}%
\end{pgfscope}%
\begin{pgfscope}%
\pgfpathrectangle{\pgfqpoint{0.800000in}{0.528000in}}{\pgfqpoint{4.960000in}{3.696000in}} %
\pgfusepath{clip}%
\pgfsetrectcap%
\pgfsetroundjoin%
\pgfsetlinewidth{0.803000pt}%
\definecolor{currentstroke}{rgb}{0.631373,0.631373,0.631373}%
\pgfsetstrokecolor{currentstroke}%
\pgfsetstrokeopacity{0.100000}%
\pgfsetdash{}{0pt}%
\pgfpathmoveto{\pgfqpoint{0.800000in}{0.528000in}}%
\pgfpathlineto{\pgfqpoint{5.760000in}{0.528000in}}%
\pgfusepath{stroke}%
\end{pgfscope}%
\begin{pgfscope}%
\pgfsetbuttcap%
\pgfsetroundjoin%
\definecolor{currentfill}{rgb}{0.333333,0.333333,0.333333}%
\pgfsetfillcolor{currentfill}%
\pgfsetlinewidth{0.803000pt}%
\definecolor{currentstroke}{rgb}{0.333333,0.333333,0.333333}%
\pgfsetstrokecolor{currentstroke}%
\pgfsetdash{}{0pt}%
\pgfsys@defobject{currentmarker}{\pgfqpoint{-0.048611in}{0.000000in}}{\pgfqpoint{0.000000in}{0.000000in}}{%
\pgfpathmoveto{\pgfqpoint{0.000000in}{0.000000in}}%
\pgfpathlineto{\pgfqpoint{-0.048611in}{0.000000in}}%
\pgfusepath{stroke,fill}%
}%
\begin{pgfscope}%
\pgfsys@transformshift{0.800000in}{0.528000in}%
\pgfsys@useobject{currentmarker}{}%
\end{pgfscope}%
\end{pgfscope}%
\begin{pgfscope}%
\definecolor{textcolor}{rgb}{0.333333,0.333333,0.333333}%
\pgfsetstrokecolor{textcolor}%
\pgfsetfillcolor{textcolor}%
\pgftext[x=0.629862in,y=0.479775in,left,base]{\color{textcolor}\sffamily\fontsize{10.000000}{12.000000}\selectfont 0}%
\end{pgfscope}%
\begin{pgfscope}%
\pgfpathrectangle{\pgfqpoint{0.800000in}{0.528000in}}{\pgfqpoint{4.960000in}{3.696000in}} %
\pgfusepath{clip}%
\pgfsetrectcap%
\pgfsetroundjoin%
\pgfsetlinewidth{0.803000pt}%
\definecolor{currentstroke}{rgb}{0.631373,0.631373,0.631373}%
\pgfsetstrokecolor{currentstroke}%
\pgfsetstrokeopacity{0.100000}%
\pgfsetdash{}{0pt}%
\pgfpathmoveto{\pgfqpoint{0.800000in}{1.144000in}}%
\pgfpathlineto{\pgfqpoint{5.760000in}{1.144000in}}%
\pgfusepath{stroke}%
\end{pgfscope}%
\begin{pgfscope}%
\pgfsetbuttcap%
\pgfsetroundjoin%
\definecolor{currentfill}{rgb}{0.333333,0.333333,0.333333}%
\pgfsetfillcolor{currentfill}%
\pgfsetlinewidth{0.803000pt}%
\definecolor{currentstroke}{rgb}{0.333333,0.333333,0.333333}%
\pgfsetstrokecolor{currentstroke}%
\pgfsetdash{}{0pt}%
\pgfsys@defobject{currentmarker}{\pgfqpoint{-0.048611in}{0.000000in}}{\pgfqpoint{0.000000in}{0.000000in}}{%
\pgfpathmoveto{\pgfqpoint{0.000000in}{0.000000in}}%
\pgfpathlineto{\pgfqpoint{-0.048611in}{0.000000in}}%
\pgfusepath{stroke,fill}%
}%
\begin{pgfscope}%
\pgfsys@transformshift{0.800000in}{1.144000in}%
\pgfsys@useobject{currentmarker}{}%
\end{pgfscope}%
\end{pgfscope}%
\begin{pgfscope}%
\definecolor{textcolor}{rgb}{0.333333,0.333333,0.333333}%
\pgfsetstrokecolor{textcolor}%
\pgfsetfillcolor{textcolor}%
\pgftext[x=0.556946in,y=1.095775in,left,base]{\color{textcolor}\sffamily\fontsize{10.000000}{12.000000}\selectfont 10}%
\end{pgfscope}%
\begin{pgfscope}%
\pgfpathrectangle{\pgfqpoint{0.800000in}{0.528000in}}{\pgfqpoint{4.960000in}{3.696000in}} %
\pgfusepath{clip}%
\pgfsetrectcap%
\pgfsetroundjoin%
\pgfsetlinewidth{0.803000pt}%
\definecolor{currentstroke}{rgb}{0.631373,0.631373,0.631373}%
\pgfsetstrokecolor{currentstroke}%
\pgfsetstrokeopacity{0.100000}%
\pgfsetdash{}{0pt}%
\pgfpathmoveto{\pgfqpoint{0.800000in}{1.760000in}}%
\pgfpathlineto{\pgfqpoint{5.760000in}{1.760000in}}%
\pgfusepath{stroke}%
\end{pgfscope}%
\begin{pgfscope}%
\pgfsetbuttcap%
\pgfsetroundjoin%
\definecolor{currentfill}{rgb}{0.333333,0.333333,0.333333}%
\pgfsetfillcolor{currentfill}%
\pgfsetlinewidth{0.803000pt}%
\definecolor{currentstroke}{rgb}{0.333333,0.333333,0.333333}%
\pgfsetstrokecolor{currentstroke}%
\pgfsetdash{}{0pt}%
\pgfsys@defobject{currentmarker}{\pgfqpoint{-0.048611in}{0.000000in}}{\pgfqpoint{0.000000in}{0.000000in}}{%
\pgfpathmoveto{\pgfqpoint{0.000000in}{0.000000in}}%
\pgfpathlineto{\pgfqpoint{-0.048611in}{0.000000in}}%
\pgfusepath{stroke,fill}%
}%
\begin{pgfscope}%
\pgfsys@transformshift{0.800000in}{1.760000in}%
\pgfsys@useobject{currentmarker}{}%
\end{pgfscope}%
\end{pgfscope}%
\begin{pgfscope}%
\definecolor{textcolor}{rgb}{0.333333,0.333333,0.333333}%
\pgfsetstrokecolor{textcolor}%
\pgfsetfillcolor{textcolor}%
\pgftext[x=0.556946in,y=1.711775in,left,base]{\color{textcolor}\sffamily\fontsize{10.000000}{12.000000}\selectfont 20}%
\end{pgfscope}%
\begin{pgfscope}%
\pgfpathrectangle{\pgfqpoint{0.800000in}{0.528000in}}{\pgfqpoint{4.960000in}{3.696000in}} %
\pgfusepath{clip}%
\pgfsetrectcap%
\pgfsetroundjoin%
\pgfsetlinewidth{0.803000pt}%
\definecolor{currentstroke}{rgb}{0.631373,0.631373,0.631373}%
\pgfsetstrokecolor{currentstroke}%
\pgfsetstrokeopacity{0.100000}%
\pgfsetdash{}{0pt}%
\pgfpathmoveto{\pgfqpoint{0.800000in}{2.376000in}}%
\pgfpathlineto{\pgfqpoint{5.760000in}{2.376000in}}%
\pgfusepath{stroke}%
\end{pgfscope}%
\begin{pgfscope}%
\pgfsetbuttcap%
\pgfsetroundjoin%
\definecolor{currentfill}{rgb}{0.333333,0.333333,0.333333}%
\pgfsetfillcolor{currentfill}%
\pgfsetlinewidth{0.803000pt}%
\definecolor{currentstroke}{rgb}{0.333333,0.333333,0.333333}%
\pgfsetstrokecolor{currentstroke}%
\pgfsetdash{}{0pt}%
\pgfsys@defobject{currentmarker}{\pgfqpoint{-0.048611in}{0.000000in}}{\pgfqpoint{0.000000in}{0.000000in}}{%
\pgfpathmoveto{\pgfqpoint{0.000000in}{0.000000in}}%
\pgfpathlineto{\pgfqpoint{-0.048611in}{0.000000in}}%
\pgfusepath{stroke,fill}%
}%
\begin{pgfscope}%
\pgfsys@transformshift{0.800000in}{2.376000in}%
\pgfsys@useobject{currentmarker}{}%
\end{pgfscope}%
\end{pgfscope}%
\begin{pgfscope}%
\definecolor{textcolor}{rgb}{0.333333,0.333333,0.333333}%
\pgfsetstrokecolor{textcolor}%
\pgfsetfillcolor{textcolor}%
\pgftext[x=0.556946in,y=2.327775in,left,base]{\color{textcolor}\sffamily\fontsize{10.000000}{12.000000}\selectfont 30}%
\end{pgfscope}%
\begin{pgfscope}%
\pgfpathrectangle{\pgfqpoint{0.800000in}{0.528000in}}{\pgfqpoint{4.960000in}{3.696000in}} %
\pgfusepath{clip}%
\pgfsetrectcap%
\pgfsetroundjoin%
\pgfsetlinewidth{0.803000pt}%
\definecolor{currentstroke}{rgb}{0.631373,0.631373,0.631373}%
\pgfsetstrokecolor{currentstroke}%
\pgfsetstrokeopacity{0.100000}%
\pgfsetdash{}{0pt}%
\pgfpathmoveto{\pgfqpoint{0.800000in}{2.992000in}}%
\pgfpathlineto{\pgfqpoint{5.760000in}{2.992000in}}%
\pgfusepath{stroke}%
\end{pgfscope}%
\begin{pgfscope}%
\pgfsetbuttcap%
\pgfsetroundjoin%
\definecolor{currentfill}{rgb}{0.333333,0.333333,0.333333}%
\pgfsetfillcolor{currentfill}%
\pgfsetlinewidth{0.803000pt}%
\definecolor{currentstroke}{rgb}{0.333333,0.333333,0.333333}%
\pgfsetstrokecolor{currentstroke}%
\pgfsetdash{}{0pt}%
\pgfsys@defobject{currentmarker}{\pgfqpoint{-0.048611in}{0.000000in}}{\pgfqpoint{0.000000in}{0.000000in}}{%
\pgfpathmoveto{\pgfqpoint{0.000000in}{0.000000in}}%
\pgfpathlineto{\pgfqpoint{-0.048611in}{0.000000in}}%
\pgfusepath{stroke,fill}%
}%
\begin{pgfscope}%
\pgfsys@transformshift{0.800000in}{2.992000in}%
\pgfsys@useobject{currentmarker}{}%
\end{pgfscope}%
\end{pgfscope}%
\begin{pgfscope}%
\definecolor{textcolor}{rgb}{0.333333,0.333333,0.333333}%
\pgfsetstrokecolor{textcolor}%
\pgfsetfillcolor{textcolor}%
\pgftext[x=0.556946in,y=2.943775in,left,base]{\color{textcolor}\sffamily\fontsize{10.000000}{12.000000}\selectfont 40}%
\end{pgfscope}%
\begin{pgfscope}%
\pgfpathrectangle{\pgfqpoint{0.800000in}{0.528000in}}{\pgfqpoint{4.960000in}{3.696000in}} %
\pgfusepath{clip}%
\pgfsetrectcap%
\pgfsetroundjoin%
\pgfsetlinewidth{0.803000pt}%
\definecolor{currentstroke}{rgb}{0.631373,0.631373,0.631373}%
\pgfsetstrokecolor{currentstroke}%
\pgfsetstrokeopacity{0.100000}%
\pgfsetdash{}{0pt}%
\pgfpathmoveto{\pgfqpoint{0.800000in}{3.608000in}}%
\pgfpathlineto{\pgfqpoint{5.760000in}{3.608000in}}%
\pgfusepath{stroke}%
\end{pgfscope}%
\begin{pgfscope}%
\pgfsetbuttcap%
\pgfsetroundjoin%
\definecolor{currentfill}{rgb}{0.333333,0.333333,0.333333}%
\pgfsetfillcolor{currentfill}%
\pgfsetlinewidth{0.803000pt}%
\definecolor{currentstroke}{rgb}{0.333333,0.333333,0.333333}%
\pgfsetstrokecolor{currentstroke}%
\pgfsetdash{}{0pt}%
\pgfsys@defobject{currentmarker}{\pgfqpoint{-0.048611in}{0.000000in}}{\pgfqpoint{0.000000in}{0.000000in}}{%
\pgfpathmoveto{\pgfqpoint{0.000000in}{0.000000in}}%
\pgfpathlineto{\pgfqpoint{-0.048611in}{0.000000in}}%
\pgfusepath{stroke,fill}%
}%
\begin{pgfscope}%
\pgfsys@transformshift{0.800000in}{3.608000in}%
\pgfsys@useobject{currentmarker}{}%
\end{pgfscope}%
\end{pgfscope}%
\begin{pgfscope}%
\definecolor{textcolor}{rgb}{0.333333,0.333333,0.333333}%
\pgfsetstrokecolor{textcolor}%
\pgfsetfillcolor{textcolor}%
\pgftext[x=0.556946in,y=3.559775in,left,base]{\color{textcolor}\sffamily\fontsize{10.000000}{12.000000}\selectfont 50}%
\end{pgfscope}%
\begin{pgfscope}%
\pgfpathrectangle{\pgfqpoint{0.800000in}{0.528000in}}{\pgfqpoint{4.960000in}{3.696000in}} %
\pgfusepath{clip}%
\pgfsetrectcap%
\pgfsetroundjoin%
\pgfsetlinewidth{0.803000pt}%
\definecolor{currentstroke}{rgb}{0.631373,0.631373,0.631373}%
\pgfsetstrokecolor{currentstroke}%
\pgfsetstrokeopacity{0.100000}%
\pgfsetdash{}{0pt}%
\pgfpathmoveto{\pgfqpoint{0.800000in}{4.224000in}}%
\pgfpathlineto{\pgfqpoint{5.760000in}{4.224000in}}%
\pgfusepath{stroke}%
\end{pgfscope}%
\begin{pgfscope}%
\pgfsetbuttcap%
\pgfsetroundjoin%
\definecolor{currentfill}{rgb}{0.333333,0.333333,0.333333}%
\pgfsetfillcolor{currentfill}%
\pgfsetlinewidth{0.803000pt}%
\definecolor{currentstroke}{rgb}{0.333333,0.333333,0.333333}%
\pgfsetstrokecolor{currentstroke}%
\pgfsetdash{}{0pt}%
\pgfsys@defobject{currentmarker}{\pgfqpoint{-0.048611in}{0.000000in}}{\pgfqpoint{0.000000in}{0.000000in}}{%
\pgfpathmoveto{\pgfqpoint{0.000000in}{0.000000in}}%
\pgfpathlineto{\pgfqpoint{-0.048611in}{0.000000in}}%
\pgfusepath{stroke,fill}%
}%
\begin{pgfscope}%
\pgfsys@transformshift{0.800000in}{4.224000in}%
\pgfsys@useobject{currentmarker}{}%
\end{pgfscope}%
\end{pgfscope}%
\begin{pgfscope}%
\definecolor{textcolor}{rgb}{0.333333,0.333333,0.333333}%
\pgfsetstrokecolor{textcolor}%
\pgfsetfillcolor{textcolor}%
\pgftext[x=0.556946in,y=4.175775in,left,base]{\color{textcolor}\sffamily\fontsize{10.000000}{12.000000}\selectfont 60}%
\end{pgfscope}%
\begin{pgfscope}%
\definecolor{textcolor}{rgb}{0.333333,0.333333,0.333333}%
\pgfsetstrokecolor{textcolor}%
\pgfsetfillcolor{textcolor}%
\pgftext[x=0.501390in,y=2.376000in,,bottom,rotate=90.000000]{\color{textcolor}\sffamily\fontsize{12.000000}{14.400000}\selectfont Porcentaje}%
\end{pgfscope}%
\begin{pgfscope}%
\pgfpathrectangle{\pgfqpoint{0.800000in}{0.528000in}}{\pgfqpoint{4.960000in}{3.696000in}} %
\pgfusepath{clip}%
\pgfsetrectcap%
\pgfsetroundjoin%
\pgfsetlinewidth{1.505625pt}%
\definecolor{currentstroke}{rgb}{0.886275,0.290196,0.200000}%
\pgfsetstrokecolor{currentstroke}%
\pgfsetdash{}{0pt}%
\pgfpathmoveto{\pgfqpoint{0.815500in}{0.843946in}}%
\pgfpathlineto{\pgfqpoint{0.831000in}{0.839430in}}%
\pgfpathlineto{\pgfqpoint{0.846500in}{0.818396in}}%
\pgfpathlineto{\pgfqpoint{0.862000in}{0.789492in}}%
\pgfpathlineto{\pgfqpoint{0.877500in}{1.025653in}}%
\pgfpathlineto{\pgfqpoint{0.893000in}{1.941364in}}%
\pgfpathlineto{\pgfqpoint{0.908500in}{1.197674in}}%
\pgfpathlineto{\pgfqpoint{0.924000in}{2.299359in}}%
\pgfpathlineto{\pgfqpoint{0.939500in}{2.140643in}}%
\pgfpathlineto{\pgfqpoint{0.955000in}{2.385239in}}%
\pgfpathlineto{\pgfqpoint{0.970500in}{2.513413in}}%
\pgfpathlineto{\pgfqpoint{0.986000in}{2.002642in}}%
\pgfpathlineto{\pgfqpoint{1.001500in}{1.988940in}}%
\pgfpathlineto{\pgfqpoint{1.017000in}{1.999353in}}%
\pgfpathlineto{\pgfqpoint{1.032500in}{1.974242in}}%
\pgfpathlineto{\pgfqpoint{1.048000in}{1.859969in}}%
\pgfpathlineto{\pgfqpoint{1.063500in}{1.760237in}}%
\pgfpathlineto{\pgfqpoint{1.079000in}{1.797368in}}%
\pgfpathlineto{\pgfqpoint{1.094500in}{1.783851in}}%
\pgfpathlineto{\pgfqpoint{1.110000in}{1.756240in}}%
\pgfpathlineto{\pgfqpoint{1.125500in}{1.794786in}}%
\pgfpathlineto{\pgfqpoint{1.141000in}{1.764596in}}%
\pgfpathlineto{\pgfqpoint{1.156500in}{1.748872in}}%
\pgfpathlineto{\pgfqpoint{1.172000in}{1.752246in}}%
\pgfpathlineto{\pgfqpoint{1.187500in}{1.765678in}}%
\pgfpathlineto{\pgfqpoint{1.203000in}{1.776240in}}%
\pgfpathlineto{\pgfqpoint{1.218500in}{1.917339in}}%
\pgfpathlineto{\pgfqpoint{1.234000in}{1.923973in}}%
\pgfpathlineto{\pgfqpoint{1.249500in}{1.775572in}}%
\pgfpathlineto{\pgfqpoint{1.265000in}{1.901517in}}%
\pgfpathlineto{\pgfqpoint{1.280500in}{2.265541in}}%
\pgfpathlineto{\pgfqpoint{1.296000in}{1.773509in}}%
\pgfpathlineto{\pgfqpoint{1.311500in}{1.786744in}}%
\pgfpathlineto{\pgfqpoint{1.327000in}{1.781254in}}%
\pgfpathlineto{\pgfqpoint{1.342500in}{1.817042in}}%
\pgfpathlineto{\pgfqpoint{1.358000in}{1.749401in}}%
\pgfpathlineto{\pgfqpoint{1.373500in}{1.743925in}}%
\pgfpathlineto{\pgfqpoint{1.389000in}{1.749643in}}%
\pgfpathlineto{\pgfqpoint{1.404500in}{2.010126in}}%
\pgfpathlineto{\pgfqpoint{1.420000in}{2.445195in}}%
\pgfpathlineto{\pgfqpoint{1.435500in}{2.467466in}}%
\pgfpathlineto{\pgfqpoint{1.451000in}{4.014794in}}%
\pgfpathlineto{\pgfqpoint{1.466500in}{2.388109in}}%
\pgfpathlineto{\pgfqpoint{1.482000in}{2.639154in}}%
\pgfpathlineto{\pgfqpoint{1.497500in}{1.794294in}}%
\pgfpathlineto{\pgfqpoint{1.513000in}{1.941373in}}%
\pgfpathlineto{\pgfqpoint{1.528500in}{1.972998in}}%
\pgfpathlineto{\pgfqpoint{1.544000in}{1.768262in}}%
\pgfpathlineto{\pgfqpoint{1.559500in}{1.784255in}}%
\pgfpathlineto{\pgfqpoint{1.575000in}{1.772583in}}%
\pgfpathlineto{\pgfqpoint{1.590500in}{1.794532in}}%
\pgfpathlineto{\pgfqpoint{1.606000in}{2.721929in}}%
\pgfpathlineto{\pgfqpoint{1.621500in}{1.712477in}}%
\pgfpathlineto{\pgfqpoint{1.637000in}{2.079817in}}%
\pgfpathlineto{\pgfqpoint{1.652500in}{2.746211in}}%
\pgfpathlineto{\pgfqpoint{1.668000in}{2.388423in}}%
\pgfpathlineto{\pgfqpoint{1.683500in}{1.872736in}}%
\pgfpathlineto{\pgfqpoint{1.699000in}{1.995088in}}%
\pgfpathlineto{\pgfqpoint{1.714500in}{3.324274in}}%
\pgfpathlineto{\pgfqpoint{1.730000in}{1.993567in}}%
\pgfpathlineto{\pgfqpoint{1.745500in}{2.537485in}}%
\pgfpathlineto{\pgfqpoint{1.761000in}{1.985536in}}%
\pgfpathlineto{\pgfqpoint{1.776500in}{1.922600in}}%
\pgfpathlineto{\pgfqpoint{1.792000in}{2.191287in}}%
\pgfpathlineto{\pgfqpoint{1.807500in}{2.569363in}}%
\pgfpathlineto{\pgfqpoint{1.823000in}{2.496214in}}%
\pgfpathlineto{\pgfqpoint{1.838500in}{1.961383in}}%
\pgfpathlineto{\pgfqpoint{1.869500in}{2.012748in}}%
\pgfpathlineto{\pgfqpoint{1.885000in}{2.324913in}}%
\pgfpathlineto{\pgfqpoint{1.900500in}{1.769637in}}%
\pgfpathlineto{\pgfqpoint{1.916000in}{1.781340in}}%
\pgfpathlineto{\pgfqpoint{1.931500in}{1.858577in}}%
\pgfpathlineto{\pgfqpoint{1.947000in}{1.754627in}}%
\pgfpathlineto{\pgfqpoint{1.962500in}{1.794935in}}%
\pgfpathlineto{\pgfqpoint{1.978000in}{1.765459in}}%
\pgfpathlineto{\pgfqpoint{1.993500in}{1.786237in}}%
\pgfpathlineto{\pgfqpoint{2.009000in}{1.783526in}}%
\pgfpathlineto{\pgfqpoint{2.024500in}{1.802919in}}%
\pgfpathlineto{\pgfqpoint{2.040000in}{1.744912in}}%
\pgfpathlineto{\pgfqpoint{2.055500in}{1.770575in}}%
\pgfpathlineto{\pgfqpoint{2.071000in}{1.791724in}}%
\pgfpathlineto{\pgfqpoint{2.086500in}{1.742083in}}%
\pgfpathlineto{\pgfqpoint{2.102000in}{1.828571in}}%
\pgfpathlineto{\pgfqpoint{2.117500in}{1.739866in}}%
\pgfpathlineto{\pgfqpoint{2.133000in}{1.831376in}}%
\pgfpathlineto{\pgfqpoint{2.148500in}{1.763145in}}%
\pgfpathlineto{\pgfqpoint{2.164000in}{1.795741in}}%
\pgfpathlineto{\pgfqpoint{2.179500in}{1.741184in}}%
\pgfpathlineto{\pgfqpoint{2.195000in}{1.793852in}}%
\pgfpathlineto{\pgfqpoint{2.210500in}{1.740866in}}%
\pgfpathlineto{\pgfqpoint{2.226000in}{1.737248in}}%
\pgfpathlineto{\pgfqpoint{2.241500in}{1.789801in}}%
\pgfpathlineto{\pgfqpoint{2.257000in}{1.741257in}}%
\pgfpathlineto{\pgfqpoint{2.272500in}{1.931173in}}%
\pgfpathlineto{\pgfqpoint{2.288000in}{2.794843in}}%
\pgfpathlineto{\pgfqpoint{2.303500in}{2.381779in}}%
\pgfpathlineto{\pgfqpoint{2.319000in}{1.867384in}}%
\pgfpathlineto{\pgfqpoint{2.334500in}{1.710728in}}%
\pgfpathlineto{\pgfqpoint{2.350000in}{1.723718in}}%
\pgfpathlineto{\pgfqpoint{2.365500in}{1.701340in}}%
\pgfpathlineto{\pgfqpoint{2.381000in}{2.106413in}}%
\pgfpathlineto{\pgfqpoint{2.396500in}{2.308510in}}%
\pgfpathlineto{\pgfqpoint{2.412000in}{2.403074in}}%
\pgfpathlineto{\pgfqpoint{2.427500in}{1.786766in}}%
\pgfpathlineto{\pgfqpoint{2.443000in}{2.029771in}}%
\pgfpathlineto{\pgfqpoint{2.458500in}{2.497187in}}%
\pgfpathlineto{\pgfqpoint{2.474000in}{1.880326in}}%
\pgfpathlineto{\pgfqpoint{2.489500in}{1.851844in}}%
\pgfpathlineto{\pgfqpoint{2.520500in}{1.767708in}}%
\pgfpathlineto{\pgfqpoint{2.536000in}{1.754106in}}%
\pgfpathlineto{\pgfqpoint{2.551500in}{1.719021in}}%
\pgfpathlineto{\pgfqpoint{2.567000in}{1.698690in}}%
\pgfpathlineto{\pgfqpoint{2.582500in}{1.690379in}}%
\pgfpathlineto{\pgfqpoint{2.598000in}{1.701137in}}%
\pgfpathlineto{\pgfqpoint{2.613500in}{1.724367in}}%
\pgfpathlineto{\pgfqpoint{2.629000in}{1.727422in}}%
\pgfpathlineto{\pgfqpoint{2.660000in}{1.709174in}}%
\pgfpathlineto{\pgfqpoint{2.675500in}{1.721505in}}%
\pgfpathlineto{\pgfqpoint{2.691000in}{1.714849in}}%
\pgfpathlineto{\pgfqpoint{2.706500in}{1.714083in}}%
\pgfpathlineto{\pgfqpoint{2.722000in}{1.729371in}}%
\pgfpathlineto{\pgfqpoint{2.737500in}{1.712558in}}%
\pgfpathlineto{\pgfqpoint{2.753000in}{1.720689in}}%
\pgfpathlineto{\pgfqpoint{2.768500in}{1.693950in}}%
\pgfpathlineto{\pgfqpoint{2.784000in}{1.704038in}}%
\pgfpathlineto{\pgfqpoint{2.799500in}{1.731150in}}%
\pgfpathlineto{\pgfqpoint{2.815000in}{1.901288in}}%
\pgfpathlineto{\pgfqpoint{2.830500in}{1.712867in}}%
\pgfpathlineto{\pgfqpoint{2.846000in}{1.718525in}}%
\pgfpathlineto{\pgfqpoint{2.861500in}{1.687528in}}%
\pgfpathlineto{\pgfqpoint{2.877000in}{1.746762in}}%
\pgfpathlineto{\pgfqpoint{2.892500in}{1.711823in}}%
\pgfpathlineto{\pgfqpoint{2.908000in}{2.803971in}}%
\pgfpathlineto{\pgfqpoint{2.923500in}{1.918770in}}%
\pgfpathlineto{\pgfqpoint{2.939000in}{1.848640in}}%
\pgfpathlineto{\pgfqpoint{2.954500in}{1.731268in}}%
\pgfpathlineto{\pgfqpoint{2.970000in}{1.744144in}}%
\pgfpathlineto{\pgfqpoint{2.985500in}{1.720471in}}%
\pgfpathlineto{\pgfqpoint{3.001000in}{1.716456in}}%
\pgfpathlineto{\pgfqpoint{3.016500in}{1.741263in}}%
\pgfpathlineto{\pgfqpoint{3.032000in}{1.721067in}}%
\pgfpathlineto{\pgfqpoint{3.047500in}{1.740281in}}%
\pgfpathlineto{\pgfqpoint{3.063000in}{1.767428in}}%
\pgfpathlineto{\pgfqpoint{3.078500in}{1.806978in}}%
\pgfpathlineto{\pgfqpoint{3.094000in}{1.741258in}}%
\pgfpathlineto{\pgfqpoint{3.109500in}{1.731436in}}%
\pgfpathlineto{\pgfqpoint{3.125000in}{1.726528in}}%
\pgfpathlineto{\pgfqpoint{3.140500in}{1.717249in}}%
\pgfpathlineto{\pgfqpoint{3.156000in}{1.740768in}}%
\pgfpathlineto{\pgfqpoint{3.171500in}{1.805215in}}%
\pgfpathlineto{\pgfqpoint{3.187000in}{1.739552in}}%
\pgfpathlineto{\pgfqpoint{3.202500in}{1.719929in}}%
\pgfpathlineto{\pgfqpoint{3.218000in}{1.731832in}}%
\pgfpathlineto{\pgfqpoint{3.233500in}{1.731523in}}%
\pgfpathlineto{\pgfqpoint{3.249000in}{1.764046in}}%
\pgfpathlineto{\pgfqpoint{3.264500in}{1.746872in}}%
\pgfpathlineto{\pgfqpoint{3.280000in}{1.753419in}}%
\pgfpathlineto{\pgfqpoint{3.295500in}{1.711126in}}%
\pgfpathlineto{\pgfqpoint{3.311000in}{1.747822in}}%
\pgfpathlineto{\pgfqpoint{3.326500in}{1.722195in}}%
\pgfpathlineto{\pgfqpoint{3.342000in}{1.729798in}}%
\pgfpathlineto{\pgfqpoint{3.357500in}{1.716158in}}%
\pgfpathlineto{\pgfqpoint{3.388500in}{1.902375in}}%
\pgfpathlineto{\pgfqpoint{3.404000in}{1.731745in}}%
\pgfpathlineto{\pgfqpoint{3.419500in}{1.737666in}}%
\pgfpathlineto{\pgfqpoint{3.435000in}{1.733773in}}%
\pgfpathlineto{\pgfqpoint{3.450500in}{1.768307in}}%
\pgfpathlineto{\pgfqpoint{3.466000in}{1.732938in}}%
\pgfpathlineto{\pgfqpoint{3.481500in}{1.735779in}}%
\pgfpathlineto{\pgfqpoint{3.497000in}{1.722845in}}%
\pgfpathlineto{\pgfqpoint{3.512500in}{1.721963in}}%
\pgfpathlineto{\pgfqpoint{3.528000in}{1.730910in}}%
\pgfpathlineto{\pgfqpoint{3.543500in}{1.731113in}}%
\pgfpathlineto{\pgfqpoint{3.559000in}{1.741212in}}%
\pgfpathlineto{\pgfqpoint{3.574500in}{1.720532in}}%
\pgfpathlineto{\pgfqpoint{3.590000in}{1.730390in}}%
\pgfpathlineto{\pgfqpoint{3.605500in}{1.718211in}}%
\pgfpathlineto{\pgfqpoint{3.621000in}{1.733789in}}%
\pgfpathlineto{\pgfqpoint{3.636500in}{1.729109in}}%
\pgfpathlineto{\pgfqpoint{3.652000in}{2.522506in}}%
\pgfpathlineto{\pgfqpoint{3.667500in}{2.750342in}}%
\pgfpathlineto{\pgfqpoint{3.683000in}{2.707354in}}%
\pgfpathlineto{\pgfqpoint{3.698500in}{2.717491in}}%
\pgfpathlineto{\pgfqpoint{3.714000in}{2.840928in}}%
\pgfpathlineto{\pgfqpoint{3.729500in}{2.865663in}}%
\pgfpathlineto{\pgfqpoint{3.745000in}{3.109238in}}%
\pgfpathlineto{\pgfqpoint{3.760500in}{2.649440in}}%
\pgfpathlineto{\pgfqpoint{3.776000in}{1.720966in}}%
\pgfpathlineto{\pgfqpoint{3.791500in}{1.736925in}}%
\pgfpathlineto{\pgfqpoint{3.807000in}{1.728770in}}%
\pgfpathlineto{\pgfqpoint{3.822500in}{1.709900in}}%
\pgfpathlineto{\pgfqpoint{3.838000in}{1.777839in}}%
\pgfpathlineto{\pgfqpoint{3.853500in}{1.733973in}}%
\pgfpathlineto{\pgfqpoint{3.869000in}{1.725422in}}%
\pgfpathlineto{\pgfqpoint{3.884500in}{1.731252in}}%
\pgfpathlineto{\pgfqpoint{3.900000in}{1.726264in}}%
\pgfpathlineto{\pgfqpoint{3.915500in}{1.718586in}}%
\pgfpathlineto{\pgfqpoint{3.931000in}{1.730583in}}%
\pgfpathlineto{\pgfqpoint{3.946500in}{1.720116in}}%
\pgfpathlineto{\pgfqpoint{3.962000in}{1.735521in}}%
\pgfpathlineto{\pgfqpoint{3.977500in}{1.720753in}}%
\pgfpathlineto{\pgfqpoint{4.008500in}{1.755650in}}%
\pgfpathlineto{\pgfqpoint{4.024000in}{1.764361in}}%
\pgfpathlineto{\pgfqpoint{4.039500in}{1.723415in}}%
\pgfpathlineto{\pgfqpoint{4.055000in}{1.729909in}}%
\pgfpathlineto{\pgfqpoint{4.070500in}{1.948418in}}%
\pgfpathlineto{\pgfqpoint{4.086000in}{1.766058in}}%
\pgfpathlineto{\pgfqpoint{4.101500in}{1.731231in}}%
\pgfpathlineto{\pgfqpoint{4.117000in}{1.733369in}}%
\pgfpathlineto{\pgfqpoint{4.132500in}{1.712657in}}%
\pgfpathlineto{\pgfqpoint{4.148000in}{1.718959in}}%
\pgfpathlineto{\pgfqpoint{4.163500in}{1.727745in}}%
\pgfpathlineto{\pgfqpoint{4.179000in}{1.719066in}}%
\pgfpathlineto{\pgfqpoint{4.194500in}{1.728261in}}%
\pgfpathlineto{\pgfqpoint{4.210000in}{1.747492in}}%
\pgfpathlineto{\pgfqpoint{4.225500in}{1.799382in}}%
\pgfpathlineto{\pgfqpoint{4.241000in}{1.738973in}}%
\pgfpathlineto{\pgfqpoint{4.256500in}{1.715857in}}%
\pgfpathlineto{\pgfqpoint{4.272000in}{2.228470in}}%
\pgfpathlineto{\pgfqpoint{4.287500in}{1.852145in}}%
\pgfpathlineto{\pgfqpoint{4.303000in}{1.789565in}}%
\pgfpathlineto{\pgfqpoint{4.318500in}{1.804034in}}%
\pgfpathlineto{\pgfqpoint{4.334000in}{1.748145in}}%
\pgfpathlineto{\pgfqpoint{4.349500in}{1.730707in}}%
\pgfpathlineto{\pgfqpoint{4.365000in}{1.738290in}}%
\pgfpathlineto{\pgfqpoint{4.380500in}{1.713394in}}%
\pgfpathlineto{\pgfqpoint{4.396000in}{1.754060in}}%
\pgfpathlineto{\pgfqpoint{4.411500in}{1.873285in}}%
\pgfpathlineto{\pgfqpoint{4.427000in}{1.732093in}}%
\pgfpathlineto{\pgfqpoint{4.442500in}{1.719748in}}%
\pgfpathlineto{\pgfqpoint{4.458000in}{1.714386in}}%
\pgfpathlineto{\pgfqpoint{4.473500in}{1.731806in}}%
\pgfpathlineto{\pgfqpoint{4.489000in}{1.736977in}}%
\pgfpathlineto{\pgfqpoint{4.504500in}{1.720338in}}%
\pgfpathlineto{\pgfqpoint{4.520000in}{1.729150in}}%
\pgfpathlineto{\pgfqpoint{4.535500in}{1.733966in}}%
\pgfpathlineto{\pgfqpoint{4.551000in}{1.734713in}}%
\pgfpathlineto{\pgfqpoint{4.566500in}{1.726439in}}%
\pgfpathlineto{\pgfqpoint{4.582000in}{1.731048in}}%
\pgfpathlineto{\pgfqpoint{4.613000in}{1.760129in}}%
\pgfpathlineto{\pgfqpoint{4.628500in}{1.727099in}}%
\pgfpathlineto{\pgfqpoint{4.644000in}{1.706958in}}%
\pgfpathlineto{\pgfqpoint{4.659500in}{1.719560in}}%
\pgfpathlineto{\pgfqpoint{4.675000in}{1.742530in}}%
\pgfpathlineto{\pgfqpoint{4.690500in}{1.863712in}}%
\pgfpathlineto{\pgfqpoint{4.706000in}{1.748889in}}%
\pgfpathlineto{\pgfqpoint{4.721500in}{1.779717in}}%
\pgfpathlineto{\pgfqpoint{4.737000in}{1.737799in}}%
\pgfpathlineto{\pgfqpoint{4.752500in}{1.891348in}}%
\pgfpathlineto{\pgfqpoint{4.768000in}{1.775736in}}%
\pgfpathlineto{\pgfqpoint{4.783500in}{2.477302in}}%
\pgfpathlineto{\pgfqpoint{4.799000in}{1.779510in}}%
\pgfpathlineto{\pgfqpoint{4.814500in}{1.729839in}}%
\pgfpathlineto{\pgfqpoint{4.830000in}{1.726950in}}%
\pgfpathlineto{\pgfqpoint{4.845500in}{1.721494in}}%
\pgfpathlineto{\pgfqpoint{4.861000in}{1.712303in}}%
\pgfpathlineto{\pgfqpoint{4.876500in}{1.775068in}}%
\pgfpathlineto{\pgfqpoint{4.892000in}{1.722267in}}%
\pgfpathlineto{\pgfqpoint{4.907500in}{1.724380in}}%
\pgfpathlineto{\pgfqpoint{4.923000in}{1.749203in}}%
\pgfpathlineto{\pgfqpoint{4.938500in}{1.734779in}}%
\pgfpathlineto{\pgfqpoint{4.954000in}{1.773869in}}%
\pgfpathlineto{\pgfqpoint{4.969500in}{1.725076in}}%
\pgfpathlineto{\pgfqpoint{4.985000in}{1.754020in}}%
\pgfpathlineto{\pgfqpoint{5.000500in}{1.708101in}}%
\pgfpathlineto{\pgfqpoint{5.016000in}{1.703497in}}%
\pgfpathlineto{\pgfqpoint{5.031500in}{1.706220in}}%
\pgfpathlineto{\pgfqpoint{5.047000in}{1.704905in}}%
\pgfpathlineto{\pgfqpoint{5.062500in}{1.742220in}}%
\pgfpathlineto{\pgfqpoint{5.078000in}{1.711820in}}%
\pgfpathlineto{\pgfqpoint{5.093500in}{1.730073in}}%
\pgfpathlineto{\pgfqpoint{5.109000in}{1.705983in}}%
\pgfpathlineto{\pgfqpoint{5.124500in}{1.700705in}}%
\pgfpathlineto{\pgfqpoint{5.140000in}{1.724055in}}%
\pgfpathlineto{\pgfqpoint{5.155500in}{1.708646in}}%
\pgfpathlineto{\pgfqpoint{5.171000in}{1.706094in}}%
\pgfpathlineto{\pgfqpoint{5.186500in}{1.741892in}}%
\pgfpathlineto{\pgfqpoint{5.202000in}{1.707047in}}%
\pgfpathlineto{\pgfqpoint{5.217500in}{1.713277in}}%
\pgfpathlineto{\pgfqpoint{5.233000in}{1.779661in}}%
\pgfpathlineto{\pgfqpoint{5.248500in}{2.513947in}}%
\pgfpathlineto{\pgfqpoint{5.264000in}{1.719381in}}%
\pgfpathlineto{\pgfqpoint{5.279500in}{1.699164in}}%
\pgfpathlineto{\pgfqpoint{5.295000in}{1.723126in}}%
\pgfpathlineto{\pgfqpoint{5.310500in}{1.792121in}}%
\pgfpathlineto{\pgfqpoint{5.326000in}{1.700730in}}%
\pgfpathlineto{\pgfqpoint{5.341500in}{2.034740in}}%
\pgfpathlineto{\pgfqpoint{5.357000in}{2.135864in}}%
\pgfpathlineto{\pgfqpoint{5.372500in}{2.162001in}}%
\pgfpathlineto{\pgfqpoint{5.388000in}{2.138386in}}%
\pgfpathlineto{\pgfqpoint{5.403500in}{2.120695in}}%
\pgfpathlineto{\pgfqpoint{5.419000in}{2.139231in}}%
\pgfpathlineto{\pgfqpoint{5.434500in}{2.146125in}}%
\pgfpathlineto{\pgfqpoint{5.450000in}{2.135977in}}%
\pgfpathlineto{\pgfqpoint{5.465500in}{2.149607in}}%
\pgfpathlineto{\pgfqpoint{5.481000in}{2.144264in}}%
\pgfpathlineto{\pgfqpoint{5.496500in}{2.269644in}}%
\pgfpathlineto{\pgfqpoint{5.512000in}{2.270262in}}%
\pgfpathlineto{\pgfqpoint{5.527500in}{1.600592in}}%
\pgfpathlineto{\pgfqpoint{5.543000in}{0.783506in}}%
\pgfpathlineto{\pgfqpoint{5.558500in}{0.782423in}}%
\pgfpathlineto{\pgfqpoint{5.574000in}{0.752673in}}%
\pgfpathlineto{\pgfqpoint{5.589500in}{0.737021in}}%
\pgfpathlineto{\pgfqpoint{5.605000in}{0.946776in}}%
\pgfpathlineto{\pgfqpoint{5.620500in}{0.781728in}}%
\pgfpathlineto{\pgfqpoint{5.636000in}{0.736311in}}%
\pgfpathlineto{\pgfqpoint{5.651500in}{0.753726in}}%
\pgfpathlineto{\pgfqpoint{5.667000in}{0.746174in}}%
\pgfpathlineto{\pgfqpoint{5.682500in}{0.753541in}}%
\pgfpathlineto{\pgfqpoint{5.698000in}{0.738856in}}%
\pgfpathlineto{\pgfqpoint{5.729000in}{0.733991in}}%
\pgfpathlineto{\pgfqpoint{5.744500in}{0.750518in}}%
\pgfpathlineto{\pgfqpoint{5.760000in}{0.738376in}}%
\pgfpathlineto{\pgfqpoint{5.770000in}{0.740342in}}%
\pgfpathlineto{\pgfqpoint{5.770000in}{0.740342in}}%
\pgfusepath{stroke}%
\end{pgfscope}%
\begin{pgfscope}%
\pgfpathrectangle{\pgfqpoint{0.800000in}{0.528000in}}{\pgfqpoint{4.960000in}{3.696000in}} %
\pgfusepath{clip}%
\pgfsetrectcap%
\pgfsetroundjoin%
\pgfsetlinewidth{1.505625pt}%
\definecolor{currentstroke}{rgb}{0.203922,0.541176,0.741176}%
\pgfsetstrokecolor{currentstroke}%
\pgfsetdash{}{0pt}%
\pgfpathmoveto{\pgfqpoint{0.800000in}{3.503531in}}%
\pgfpathlineto{\pgfqpoint{0.815500in}{3.496955in}}%
\pgfpathlineto{\pgfqpoint{0.831000in}{3.501002in}}%
\pgfpathlineto{\pgfqpoint{0.862000in}{3.495943in}}%
\pgfpathlineto{\pgfqpoint{0.877500in}{3.533378in}}%
\pgfpathlineto{\pgfqpoint{0.893000in}{3.694751in}}%
\pgfpathlineto{\pgfqpoint{0.908500in}{3.675528in}}%
\pgfpathlineto{\pgfqpoint{0.924000in}{3.689187in}}%
\pgfpathlineto{\pgfqpoint{0.939500in}{3.689187in}}%
\pgfpathlineto{\pgfqpoint{0.955000in}{3.617353in}}%
\pgfpathlineto{\pgfqpoint{0.970500in}{3.645176in}}%
\pgfpathlineto{\pgfqpoint{0.986000in}{3.644670in}}%
\pgfpathlineto{\pgfqpoint{1.001500in}{3.619376in}}%
\pgfpathlineto{\pgfqpoint{1.032500in}{3.619376in}}%
\pgfpathlineto{\pgfqpoint{1.048000in}{3.622411in}}%
\pgfpathlineto{\pgfqpoint{1.063500in}{3.626964in}}%
\pgfpathlineto{\pgfqpoint{1.079000in}{3.628988in}}%
\pgfpathlineto{\pgfqpoint{1.094500in}{3.637588in}}%
\pgfpathlineto{\pgfqpoint{1.110000in}{3.633541in}}%
\pgfpathlineto{\pgfqpoint{1.125500in}{3.641635in}}%
\pgfpathlineto{\pgfqpoint{1.141000in}{3.643152in}}%
\pgfpathlineto{\pgfqpoint{1.156500in}{3.637082in}}%
\pgfpathlineto{\pgfqpoint{1.172000in}{3.635564in}}%
\pgfpathlineto{\pgfqpoint{1.203000in}{3.641129in}}%
\pgfpathlineto{\pgfqpoint{1.218500in}{3.615835in}}%
\pgfpathlineto{\pgfqpoint{1.234000in}{3.632023in}}%
\pgfpathlineto{\pgfqpoint{1.249500in}{3.629494in}}%
\pgfpathlineto{\pgfqpoint{1.265000in}{3.636576in}}%
\pgfpathlineto{\pgfqpoint{1.280500in}{3.641635in}}%
\pgfpathlineto{\pgfqpoint{1.296000in}{3.642140in}}%
\pgfpathlineto{\pgfqpoint{1.311500in}{3.630505in}}%
\pgfpathlineto{\pgfqpoint{1.358000in}{3.626964in}}%
\pgfpathlineto{\pgfqpoint{1.373500in}{3.622411in}}%
\pgfpathlineto{\pgfqpoint{1.389000in}{3.625953in}}%
\pgfpathlineto{\pgfqpoint{1.420000in}{3.645682in}}%
\pgfpathlineto{\pgfqpoint{1.435500in}{3.679575in}}%
\pgfpathlineto{\pgfqpoint{1.451000in}{3.752421in}}%
\pgfpathlineto{\pgfqpoint{1.466500in}{3.776197in}}%
\pgfpathlineto{\pgfqpoint{1.482000in}{3.771138in}}%
\pgfpathlineto{\pgfqpoint{1.497500in}{3.753432in}}%
\pgfpathlineto{\pgfqpoint{1.513000in}{3.717010in}}%
\pgfpathlineto{\pgfqpoint{1.528500in}{3.716504in}}%
\pgfpathlineto{\pgfqpoint{1.544000in}{3.711951in}}%
\pgfpathlineto{\pgfqpoint{1.559500in}{3.711951in}}%
\pgfpathlineto{\pgfqpoint{1.575000in}{3.693234in}}%
\pgfpathlineto{\pgfqpoint{1.590500in}{3.697281in}}%
\pgfpathlineto{\pgfqpoint{1.606000in}{3.745844in}}%
\pgfpathlineto{\pgfqpoint{1.621500in}{3.730162in}}%
\pgfpathlineto{\pgfqpoint{1.637000in}{3.716504in}}%
\pgfpathlineto{\pgfqpoint{1.652500in}{3.773667in}}%
\pgfpathlineto{\pgfqpoint{1.668000in}{3.817678in}}%
\pgfpathlineto{\pgfqpoint{1.683500in}{3.785302in}}%
\pgfpathlineto{\pgfqpoint{1.699000in}{3.757479in}}%
\pgfpathlineto{\pgfqpoint{1.714500in}{3.835890in}}%
\pgfpathlineto{\pgfqpoint{1.730000in}{3.825772in}}%
\pgfpathlineto{\pgfqpoint{1.745500in}{3.821725in}}%
\pgfpathlineto{\pgfqpoint{1.761000in}{3.816161in}}%
\pgfpathlineto{\pgfqpoint{1.776500in}{3.784797in}}%
\pgfpathlineto{\pgfqpoint{1.792000in}{3.781761in}}%
\pgfpathlineto{\pgfqpoint{1.807500in}{3.795926in}}%
\pgfpathlineto{\pgfqpoint{1.823000in}{3.813631in}}%
\pgfpathlineto{\pgfqpoint{1.838500in}{3.794408in}}%
\pgfpathlineto{\pgfqpoint{1.854000in}{3.796432in}}%
\pgfpathlineto{\pgfqpoint{1.869500in}{3.781761in}}%
\pgfpathlineto{\pgfqpoint{1.885000in}{3.688681in}}%
\pgfpathlineto{\pgfqpoint{1.900500in}{3.687163in}}%
\pgfpathlineto{\pgfqpoint{1.916000in}{3.688175in}}%
\pgfpathlineto{\pgfqpoint{1.931500in}{3.697281in}}%
\pgfpathlineto{\pgfqpoint{1.947000in}{3.693234in}}%
\pgfpathlineto{\pgfqpoint{1.962500in}{3.690704in}}%
\pgfpathlineto{\pgfqpoint{1.978000in}{3.691210in}}%
\pgfpathlineto{\pgfqpoint{1.993500in}{3.693739in}}%
\pgfpathlineto{\pgfqpoint{2.009000in}{3.713974in}}%
\pgfpathlineto{\pgfqpoint{2.024500in}{3.706386in}}%
\pgfpathlineto{\pgfqpoint{2.040000in}{3.696775in}}%
\pgfpathlineto{\pgfqpoint{2.055500in}{3.694751in}}%
\pgfpathlineto{\pgfqpoint{2.086500in}{3.695763in}}%
\pgfpathlineto{\pgfqpoint{2.102000in}{3.697281in}}%
\pgfpathlineto{\pgfqpoint{2.117500in}{3.697281in}}%
\pgfpathlineto{\pgfqpoint{2.133000in}{3.704363in}}%
\pgfpathlineto{\pgfqpoint{2.164000in}{3.698798in}}%
\pgfpathlineto{\pgfqpoint{2.179500in}{3.699304in}}%
\pgfpathlineto{\pgfqpoint{2.195000in}{3.711445in}}%
\pgfpathlineto{\pgfqpoint{2.210500in}{3.701833in}}%
\pgfpathlineto{\pgfqpoint{2.257000in}{3.702339in}}%
\pgfpathlineto{\pgfqpoint{2.272500in}{3.712457in}}%
\pgfpathlineto{\pgfqpoint{2.288000in}{3.779738in}}%
\pgfpathlineto{\pgfqpoint{2.303500in}{3.793396in}}%
\pgfpathlineto{\pgfqpoint{2.319000in}{3.751409in}}%
\pgfpathlineto{\pgfqpoint{2.334500in}{3.742303in}}%
\pgfpathlineto{\pgfqpoint{2.350000in}{3.747868in}}%
\pgfpathlineto{\pgfqpoint{2.365500in}{3.741292in}}%
\pgfpathlineto{\pgfqpoint{2.381000in}{3.737245in}}%
\pgfpathlineto{\pgfqpoint{2.396500in}{3.742809in}}%
\pgfpathlineto{\pgfqpoint{2.412000in}{3.762538in}}%
\pgfpathlineto{\pgfqpoint{2.427500in}{3.751915in}}%
\pgfpathlineto{\pgfqpoint{2.443000in}{3.776703in}}%
\pgfpathlineto{\pgfqpoint{2.458500in}{3.820714in}}%
\pgfpathlineto{\pgfqpoint{2.474000in}{3.788338in}}%
\pgfpathlineto{\pgfqpoint{2.489500in}{3.786820in}}%
\pgfpathlineto{\pgfqpoint{2.505000in}{3.733703in}}%
\pgfpathlineto{\pgfqpoint{2.520500in}{3.731174in}}%
\pgfpathlineto{\pgfqpoint{2.536000in}{3.733198in}}%
\pgfpathlineto{\pgfqpoint{2.567000in}{3.733703in}}%
\pgfpathlineto{\pgfqpoint{2.582500in}{3.740786in}}%
\pgfpathlineto{\pgfqpoint{2.598000in}{3.738762in}}%
\pgfpathlineto{\pgfqpoint{2.613500in}{3.740786in}}%
\pgfpathlineto{\pgfqpoint{2.629000in}{3.739268in}}%
\pgfpathlineto{\pgfqpoint{2.644500in}{3.742303in}}%
\pgfpathlineto{\pgfqpoint{2.660000in}{3.736233in}}%
\pgfpathlineto{\pgfqpoint{2.675500in}{3.735221in}}%
\pgfpathlineto{\pgfqpoint{2.691000in}{3.738762in}}%
\pgfpathlineto{\pgfqpoint{2.706500in}{3.737245in}}%
\pgfpathlineto{\pgfqpoint{2.722000in}{3.744327in}}%
\pgfpathlineto{\pgfqpoint{2.737500in}{3.739774in}}%
\pgfpathlineto{\pgfqpoint{2.768500in}{3.739774in}}%
\pgfpathlineto{\pgfqpoint{2.784000in}{3.739774in}}%
\pgfpathlineto{\pgfqpoint{2.799500in}{3.728645in}}%
\pgfpathlineto{\pgfqpoint{2.815000in}{3.666928in}}%
\pgfpathlineto{\pgfqpoint{2.830500in}{3.670975in}}%
\pgfpathlineto{\pgfqpoint{2.846000in}{3.661870in}}%
\pgfpathlineto{\pgfqpoint{2.861500in}{3.662881in}}%
\pgfpathlineto{\pgfqpoint{2.877000in}{3.670975in}}%
\pgfpathlineto{\pgfqpoint{2.892500in}{3.666928in}}%
\pgfpathlineto{\pgfqpoint{2.908000in}{3.747868in}}%
\pgfpathlineto{\pgfqpoint{2.923500in}{3.747362in}}%
\pgfpathlineto{\pgfqpoint{2.939000in}{3.733703in}}%
\pgfpathlineto{\pgfqpoint{2.954500in}{3.710433in}}%
\pgfpathlineto{\pgfqpoint{3.001000in}{3.708410in}}%
\pgfpathlineto{\pgfqpoint{3.016500in}{3.712457in}}%
\pgfpathlineto{\pgfqpoint{3.032000in}{3.712457in}}%
\pgfpathlineto{\pgfqpoint{3.047500in}{3.709927in}}%
\pgfpathlineto{\pgfqpoint{3.078500in}{3.708916in}}%
\pgfpathlineto{\pgfqpoint{3.094000in}{3.716504in}}%
\pgfpathlineto{\pgfqpoint{3.109500in}{3.719033in}}%
\pgfpathlineto{\pgfqpoint{3.125000in}{3.704869in}}%
\pgfpathlineto{\pgfqpoint{3.140500in}{3.704869in}}%
\pgfpathlineto{\pgfqpoint{3.156000in}{3.700822in}}%
\pgfpathlineto{\pgfqpoint{3.171500in}{3.678563in}}%
\pgfpathlineto{\pgfqpoint{3.187000in}{3.687669in}}%
\pgfpathlineto{\pgfqpoint{3.218000in}{3.680081in}}%
\pgfpathlineto{\pgfqpoint{3.233500in}{3.686151in}}%
\pgfpathlineto{\pgfqpoint{3.249000in}{3.682610in}}%
\pgfpathlineto{\pgfqpoint{3.264500in}{3.664905in}}%
\pgfpathlineto{\pgfqpoint{3.280000in}{3.672493in}}%
\pgfpathlineto{\pgfqpoint{3.295500in}{3.670469in}}%
\pgfpathlineto{\pgfqpoint{3.311000in}{3.663893in}}%
\pgfpathlineto{\pgfqpoint{3.357500in}{3.663387in}}%
\pgfpathlineto{\pgfqpoint{3.373000in}{3.688175in}}%
\pgfpathlineto{\pgfqpoint{3.388500in}{3.689187in}}%
\pgfpathlineto{\pgfqpoint{3.419500in}{3.687163in}}%
\pgfpathlineto{\pgfqpoint{3.435000in}{3.666422in}}%
\pgfpathlineto{\pgfqpoint{3.450500in}{3.666422in}}%
\pgfpathlineto{\pgfqpoint{3.466000in}{3.678057in}}%
\pgfpathlineto{\pgfqpoint{3.481500in}{3.677046in}}%
\pgfpathlineto{\pgfqpoint{3.497000in}{3.674516in}}%
\pgfpathlineto{\pgfqpoint{3.512500in}{3.669963in}}%
\pgfpathlineto{\pgfqpoint{3.543500in}{3.667940in}}%
\pgfpathlineto{\pgfqpoint{3.574500in}{3.668446in}}%
\pgfpathlineto{\pgfqpoint{3.621000in}{3.667434in}}%
\pgfpathlineto{\pgfqpoint{3.636500in}{3.669963in}}%
\pgfpathlineto{\pgfqpoint{3.652000in}{3.769115in}}%
\pgfpathlineto{\pgfqpoint{3.667500in}{3.770632in}}%
\pgfpathlineto{\pgfqpoint{3.683000in}{3.768609in}}%
\pgfpathlineto{\pgfqpoint{3.698500in}{3.774679in}}%
\pgfpathlineto{\pgfqpoint{3.714000in}{3.739268in}}%
\pgfpathlineto{\pgfqpoint{3.729500in}{3.696775in}}%
\pgfpathlineto{\pgfqpoint{3.745000in}{3.697281in}}%
\pgfpathlineto{\pgfqpoint{3.760500in}{3.656811in}}%
\pgfpathlineto{\pgfqpoint{3.791500in}{3.653776in}}%
\pgfpathlineto{\pgfqpoint{3.807000in}{3.658328in}}%
\pgfpathlineto{\pgfqpoint{3.822500in}{3.661364in}}%
\pgfpathlineto{\pgfqpoint{3.838000in}{3.667434in}}%
\pgfpathlineto{\pgfqpoint{3.853500in}{3.666928in}}%
\pgfpathlineto{\pgfqpoint{3.869000in}{3.656811in}}%
\pgfpathlineto{\pgfqpoint{3.884500in}{3.655293in}}%
\pgfpathlineto{\pgfqpoint{3.962000in}{3.654787in}}%
\pgfpathlineto{\pgfqpoint{3.993000in}{3.653776in}}%
\pgfpathlineto{\pgfqpoint{4.008500in}{3.658328in}}%
\pgfpathlineto{\pgfqpoint{4.024000in}{3.668952in}}%
\pgfpathlineto{\pgfqpoint{4.039500in}{3.670469in}}%
\pgfpathlineto{\pgfqpoint{4.055000in}{3.661870in}}%
\pgfpathlineto{\pgfqpoint{4.086000in}{3.661364in}}%
\pgfpathlineto{\pgfqpoint{4.132500in}{3.663387in}}%
\pgfpathlineto{\pgfqpoint{4.148000in}{3.671987in}}%
\pgfpathlineto{\pgfqpoint{4.163500in}{3.667940in}}%
\pgfpathlineto{\pgfqpoint{4.194500in}{3.663893in}}%
\pgfpathlineto{\pgfqpoint{4.210000in}{3.670975in}}%
\pgfpathlineto{\pgfqpoint{4.225500in}{3.674516in}}%
\pgfpathlineto{\pgfqpoint{4.241000in}{3.664905in}}%
\pgfpathlineto{\pgfqpoint{4.256500in}{3.664399in}}%
\pgfpathlineto{\pgfqpoint{4.272000in}{3.667434in}}%
\pgfpathlineto{\pgfqpoint{4.287500in}{3.666422in}}%
\pgfpathlineto{\pgfqpoint{4.303000in}{3.677552in}}%
\pgfpathlineto{\pgfqpoint{4.318500in}{3.682104in}}%
\pgfpathlineto{\pgfqpoint{4.349500in}{3.684128in}}%
\pgfpathlineto{\pgfqpoint{4.365000in}{3.666422in}}%
\pgfpathlineto{\pgfqpoint{4.380500in}{3.663387in}}%
\pgfpathlineto{\pgfqpoint{4.396000in}{3.674010in}}%
\pgfpathlineto{\pgfqpoint{4.411500in}{3.677552in}}%
\pgfpathlineto{\pgfqpoint{4.442500in}{3.668446in}}%
\pgfpathlineto{\pgfqpoint{4.458000in}{3.676540in}}%
\pgfpathlineto{\pgfqpoint{4.473500in}{3.680587in}}%
\pgfpathlineto{\pgfqpoint{4.489000in}{3.679069in}}%
\pgfpathlineto{\pgfqpoint{4.520000in}{3.670975in}}%
\pgfpathlineto{\pgfqpoint{4.535500in}{3.668952in}}%
\pgfpathlineto{\pgfqpoint{4.566500in}{3.668952in}}%
\pgfpathlineto{\pgfqpoint{4.582000in}{3.678057in}}%
\pgfpathlineto{\pgfqpoint{4.597500in}{3.678057in}}%
\pgfpathlineto{\pgfqpoint{4.613000in}{3.667940in}}%
\pgfpathlineto{\pgfqpoint{4.659500in}{3.667434in}}%
\pgfpathlineto{\pgfqpoint{4.675000in}{3.658328in}}%
\pgfpathlineto{\pgfqpoint{4.690500in}{3.602682in}}%
\pgfpathlineto{\pgfqpoint{4.737000in}{3.610776in}}%
\pgfpathlineto{\pgfqpoint{4.752500in}{3.607235in}}%
\pgfpathlineto{\pgfqpoint{4.768000in}{3.621906in}}%
\pgfpathlineto{\pgfqpoint{4.783500in}{3.710433in}}%
\pgfpathlineto{\pgfqpoint{4.799000in}{3.705375in}}%
\pgfpathlineto{\pgfqpoint{4.814500in}{3.703351in}}%
\pgfpathlineto{\pgfqpoint{4.830000in}{3.695763in}}%
\pgfpathlineto{\pgfqpoint{4.861000in}{3.694751in}}%
\pgfpathlineto{\pgfqpoint{4.876500in}{3.704363in}}%
\pgfpathlineto{\pgfqpoint{4.892000in}{3.703857in}}%
\pgfpathlineto{\pgfqpoint{4.907500in}{3.698292in}}%
\pgfpathlineto{\pgfqpoint{4.923000in}{3.706386in}}%
\pgfpathlineto{\pgfqpoint{4.938500in}{3.705880in}}%
\pgfpathlineto{\pgfqpoint{4.954000in}{3.698798in}}%
\pgfpathlineto{\pgfqpoint{4.969500in}{3.698292in}}%
\pgfpathlineto{\pgfqpoint{4.985000in}{3.705375in}}%
\pgfpathlineto{\pgfqpoint{5.000500in}{3.695763in}}%
\pgfpathlineto{\pgfqpoint{5.031500in}{3.695257in}}%
\pgfpathlineto{\pgfqpoint{5.078000in}{3.696775in}}%
\pgfpathlineto{\pgfqpoint{5.093500in}{3.687669in}}%
\pgfpathlineto{\pgfqpoint{5.109000in}{3.692222in}}%
\pgfpathlineto{\pgfqpoint{5.124500in}{3.698798in}}%
\pgfpathlineto{\pgfqpoint{5.155500in}{3.684634in}}%
\pgfpathlineto{\pgfqpoint{5.171000in}{3.684128in}}%
\pgfpathlineto{\pgfqpoint{5.186500in}{3.685140in}}%
\pgfpathlineto{\pgfqpoint{5.202000in}{3.696269in}}%
\pgfpathlineto{\pgfqpoint{5.217500in}{3.684128in}}%
\pgfpathlineto{\pgfqpoint{5.233000in}{3.707904in}}%
\pgfpathlineto{\pgfqpoint{5.248500in}{3.738256in}}%
\pgfpathlineto{\pgfqpoint{5.264000in}{3.731680in}}%
\pgfpathlineto{\pgfqpoint{5.279500in}{3.731680in}}%
\pgfpathlineto{\pgfqpoint{5.310500in}{3.671987in}}%
\pgfpathlineto{\pgfqpoint{5.326000in}{3.670975in}}%
\pgfpathlineto{\pgfqpoint{5.341500in}{3.660352in}}%
\pgfpathlineto{\pgfqpoint{5.357000in}{3.651752in}}%
\pgfpathlineto{\pgfqpoint{5.372500in}{3.651246in}}%
\pgfpathlineto{\pgfqpoint{5.388000in}{3.660858in}}%
\pgfpathlineto{\pgfqpoint{5.403500in}{3.652764in}}%
\pgfpathlineto{\pgfqpoint{5.419000in}{3.652258in}}%
\pgfpathlineto{\pgfqpoint{5.434500in}{3.642646in}}%
\pgfpathlineto{\pgfqpoint{5.481000in}{3.641129in}}%
\pgfpathlineto{\pgfqpoint{5.496500in}{3.603188in}}%
\pgfpathlineto{\pgfqpoint{5.512000in}{3.618870in}}%
\pgfpathlineto{\pgfqpoint{5.527500in}{3.537931in}}%
\pgfpathlineto{\pgfqpoint{5.543000in}{3.535907in}}%
\pgfpathlineto{\pgfqpoint{5.558500in}{3.542484in}}%
\pgfpathlineto{\pgfqpoint{5.574000in}{3.553613in}}%
\pgfpathlineto{\pgfqpoint{5.589500in}{3.545519in}}%
\pgfpathlineto{\pgfqpoint{5.620500in}{3.544507in}}%
\pgfpathlineto{\pgfqpoint{5.651500in}{3.544001in}}%
\pgfpathlineto{\pgfqpoint{5.682500in}{3.544507in}}%
\pgfpathlineto{\pgfqpoint{5.698000in}{3.544001in}}%
\pgfpathlineto{\pgfqpoint{5.713500in}{3.541978in}}%
\pgfpathlineto{\pgfqpoint{5.744500in}{3.541978in}}%
\pgfpathlineto{\pgfqpoint{5.760000in}{3.550578in}}%
\pgfpathlineto{\pgfqpoint{5.770000in}{3.545682in}}%
\pgfpathlineto{\pgfqpoint{5.770000in}{3.545682in}}%
\pgfusepath{stroke}%
\end{pgfscope}%
\begin{pgfscope}%
\pgfsetrectcap%
\pgfsetmiterjoin%
\pgfsetlinewidth{1.003750pt}%
\definecolor{currentstroke}{rgb}{1.000000,1.000000,1.000000}%
\pgfsetstrokecolor{currentstroke}%
\pgfsetdash{}{0pt}%
\pgfpathmoveto{\pgfqpoint{0.800000in}{0.528000in}}%
\pgfpathlineto{\pgfqpoint{0.800000in}{4.224000in}}%
\pgfusepath{stroke}%
\end{pgfscope}%
\begin{pgfscope}%
\pgfsetrectcap%
\pgfsetmiterjoin%
\pgfsetlinewidth{1.003750pt}%
\definecolor{currentstroke}{rgb}{1.000000,1.000000,1.000000}%
\pgfsetstrokecolor{currentstroke}%
\pgfsetdash{}{0pt}%
\pgfpathmoveto{\pgfqpoint{5.760000in}{0.528000in}}%
\pgfpathlineto{\pgfqpoint{5.760000in}{4.224000in}}%
\pgfusepath{stroke}%
\end{pgfscope}%
\begin{pgfscope}%
\pgfsetrectcap%
\pgfsetmiterjoin%
\pgfsetlinewidth{1.003750pt}%
\definecolor{currentstroke}{rgb}{1.000000,1.000000,1.000000}%
\pgfsetstrokecolor{currentstroke}%
\pgfsetdash{}{0pt}%
\pgfpathmoveto{\pgfqpoint{0.800000in}{0.528000in}}%
\pgfpathlineto{\pgfqpoint{5.760000in}{0.528000in}}%
\pgfusepath{stroke}%
\end{pgfscope}%
\begin{pgfscope}%
\pgfsetrectcap%
\pgfsetmiterjoin%
\pgfsetlinewidth{1.003750pt}%
\definecolor{currentstroke}{rgb}{1.000000,1.000000,1.000000}%
\pgfsetstrokecolor{currentstroke}%
\pgfsetdash{}{0pt}%
\pgfpathmoveto{\pgfqpoint{0.800000in}{4.224000in}}%
\pgfpathlineto{\pgfqpoint{5.760000in}{4.224000in}}%
\pgfusepath{stroke}%
\end{pgfscope}%
\begin{pgfscope}%
\pgfsetbuttcap%
\pgfsetmiterjoin%
\definecolor{currentfill}{rgb}{0.898039,0.898039,0.898039}%
\pgfsetfillcolor{currentfill}%
\pgfsetfillopacity{0.800000}%
\pgfsetlinewidth{0.501875pt}%
\definecolor{currentstroke}{rgb}{0.800000,0.800000,0.800000}%
\pgfsetstrokecolor{currentstroke}%
\pgfsetstrokeopacity{0.800000}%
\pgfsetdash{}{0pt}%
\pgfpathmoveto{\pgfqpoint{2.534982in}{0.597444in}}%
\pgfpathlineto{\pgfqpoint{4.025018in}{0.597444in}}%
\pgfpathquadraticcurveto{\pgfqpoint{4.052796in}{0.597444in}}{\pgfqpoint{4.052796in}{0.625222in}}%
\pgfpathlineto{\pgfqpoint{4.052796in}{0.997907in}}%
\pgfpathquadraticcurveto{\pgfqpoint{4.052796in}{1.025685in}}{\pgfqpoint{4.025018in}{1.025685in}}%
\pgfpathlineto{\pgfqpoint{2.534982in}{1.025685in}}%
\pgfpathquadraticcurveto{\pgfqpoint{2.507204in}{1.025685in}}{\pgfqpoint{2.507204in}{0.997907in}}%
\pgfpathlineto{\pgfqpoint{2.507204in}{0.625222in}}%
\pgfpathquadraticcurveto{\pgfqpoint{2.507204in}{0.597444in}}{\pgfqpoint{2.534982in}{0.597444in}}%
\pgfpathclose%
\pgfusepath{stroke,fill}%
\end{pgfscope}%
\begin{pgfscope}%
\pgfsetrectcap%
\pgfsetroundjoin%
\pgfsetlinewidth{1.505625pt}%
\definecolor{currentstroke}{rgb}{0.886275,0.290196,0.200000}%
\pgfsetstrokecolor{currentstroke}%
\pgfsetdash{}{0pt}%
\pgfpathmoveto{\pgfqpoint{2.562760in}{0.914574in}}%
\pgfpathlineto{\pgfqpoint{2.840537in}{0.914574in}}%
\pgfusepath{stroke}%
\end{pgfscope}%
\begin{pgfscope}%
\pgftext[x=2.951648in,y=0.865963in,left,base]{\sffamily\fontsize{10.000000}{12.000000}\selectfont \% uso procesador}%
\end{pgfscope}%
\begin{pgfscope}%
\pgfsetrectcap%
\pgfsetroundjoin%
\pgfsetlinewidth{1.505625pt}%
\definecolor{currentstroke}{rgb}{0.203922,0.541176,0.741176}%
\pgfsetstrokecolor{currentstroke}%
\pgfsetdash{}{0pt}%
\pgfpathmoveto{\pgfqpoint{2.562760in}{0.724759in}}%
\pgfpathlineto{\pgfqpoint{2.840537in}{0.724759in}}%
\pgfusepath{stroke}%
\end{pgfscope}%
\begin{pgfscope}%
\pgftext[x=2.951648in,y=0.676148in,left,base]{\sffamily\fontsize{10.000000}{12.000000}\selectfont \% uso RAM}%
\end{pgfscope}%
\end{pgfpicture}%
\makeatother%
\endgroup%

    \caption{Carga sobre el sistema durante una simulación con factor de demanda 100\%.}
    \label{fig:systemload:cpuram}
\end{figure}
\begin{figure}[h]
    \centering
    %% Creator: Matplotlib, PGF backend
%%
%% To include the figure in your LaTeX document, write
%%   \input{<filename>.pgf}
%%
%% Make sure the required packages are loaded in your preamble
%%   \usepackage{pgf}
%%
%% Figures using additional raster images can only be included by \input if
%% they are in the same directory as the main LaTeX file. For loading figures
%% from other directories you can use the `import` package
%%   \usepackage{import}
%% and then include the figures with
%%   \import{<path to file>}{<filename>.pgf}
%%
%% Matplotlib used the following preamble
%%   \usepackage[utf8x]{inputenc}
%%   \usepackage[T1]{fontenc}
%%   \usepackage{cmbright}
%%
\begingroup%
\makeatletter%
\begin{pgfpicture}%
\pgfpathrectangle{\pgfpointorigin}{\pgfqpoint{6.400000in}{4.800000in}}%
\pgfusepath{use as bounding box, clip}%
\begin{pgfscope}%
\pgfsetbuttcap%
\pgfsetmiterjoin%
\definecolor{currentfill}{rgb}{1.000000,1.000000,1.000000}%
\pgfsetfillcolor{currentfill}%
\pgfsetlinewidth{0.000000pt}%
\definecolor{currentstroke}{rgb}{1.000000,1.000000,1.000000}%
\pgfsetstrokecolor{currentstroke}%
\pgfsetdash{}{0pt}%
\pgfpathmoveto{\pgfqpoint{0.000000in}{0.000000in}}%
\pgfpathlineto{\pgfqpoint{6.400000in}{0.000000in}}%
\pgfpathlineto{\pgfqpoint{6.400000in}{4.800000in}}%
\pgfpathlineto{\pgfqpoint{0.000000in}{4.800000in}}%
\pgfpathclose%
\pgfusepath{fill}%
\end{pgfscope}%
\begin{pgfscope}%
\pgfsetbuttcap%
\pgfsetmiterjoin%
\definecolor{currentfill}{rgb}{1.000000,1.000000,1.000000}%
\pgfsetfillcolor{currentfill}%
\pgfsetlinewidth{0.000000pt}%
\definecolor{currentstroke}{rgb}{0.000000,0.000000,0.000000}%
\pgfsetstrokecolor{currentstroke}%
\pgfsetstrokeopacity{0.000000}%
\pgfsetdash{}{0pt}%
\pgfpathmoveto{\pgfqpoint{0.800000in}{0.528000in}}%
\pgfpathlineto{\pgfqpoint{5.760000in}{0.528000in}}%
\pgfpathlineto{\pgfqpoint{5.760000in}{4.224000in}}%
\pgfpathlineto{\pgfqpoint{0.800000in}{4.224000in}}%
\pgfpathclose%
\pgfusepath{fill}%
\end{pgfscope}%
\begin{pgfscope}%
\pgfpathrectangle{\pgfqpoint{0.800000in}{0.528000in}}{\pgfqpoint{4.960000in}{3.696000in}} %
\pgfusepath{clip}%
\pgfsetrectcap%
\pgfsetroundjoin%
\pgfsetlinewidth{0.803000pt}%
\definecolor{currentstroke}{rgb}{0.631373,0.631373,0.631373}%
\pgfsetstrokecolor{currentstroke}%
\pgfsetstrokeopacity{0.100000}%
\pgfsetdash{}{0pt}%
\pgfpathmoveto{\pgfqpoint{0.800000in}{0.528000in}}%
\pgfpathlineto{\pgfqpoint{0.800000in}{4.224000in}}%
\pgfusepath{stroke}%
\end{pgfscope}%
\begin{pgfscope}%
\pgfsetbuttcap%
\pgfsetroundjoin%
\definecolor{currentfill}{rgb}{0.333333,0.333333,0.333333}%
\pgfsetfillcolor{currentfill}%
\pgfsetlinewidth{0.803000pt}%
\definecolor{currentstroke}{rgb}{0.333333,0.333333,0.333333}%
\pgfsetstrokecolor{currentstroke}%
\pgfsetdash{}{0pt}%
\pgfsys@defobject{currentmarker}{\pgfqpoint{0.000000in}{-0.048611in}}{\pgfqpoint{0.000000in}{0.000000in}}{%
\pgfpathmoveto{\pgfqpoint{0.000000in}{0.000000in}}%
\pgfpathlineto{\pgfqpoint{0.000000in}{-0.048611in}}%
\pgfusepath{stroke,fill}%
}%
\begin{pgfscope}%
\pgfsys@transformshift{0.800000in}{0.528000in}%
\pgfsys@useobject{currentmarker}{}%
\end{pgfscope}%
\end{pgfscope}%
\begin{pgfscope}%
\definecolor{textcolor}{rgb}{0.333333,0.333333,0.333333}%
\pgfsetstrokecolor{textcolor}%
\pgfsetfillcolor{textcolor}%
\pgftext[x=0.800000in,y=0.430778in,,top]{\color{textcolor}\sffamily\fontsize{10.000000}{12.000000}\selectfont 00:00}%
\end{pgfscope}%
\begin{pgfscope}%
\pgfpathrectangle{\pgfqpoint{0.800000in}{0.528000in}}{\pgfqpoint{4.960000in}{3.696000in}} %
\pgfusepath{clip}%
\pgfsetrectcap%
\pgfsetroundjoin%
\pgfsetlinewidth{0.803000pt}%
\definecolor{currentstroke}{rgb}{0.631373,0.631373,0.631373}%
\pgfsetstrokecolor{currentstroke}%
\pgfsetstrokeopacity{0.100000}%
\pgfsetdash{}{0pt}%
\pgfpathmoveto{\pgfqpoint{1.420000in}{0.528000in}}%
\pgfpathlineto{\pgfqpoint{1.420000in}{4.224000in}}%
\pgfusepath{stroke}%
\end{pgfscope}%
\begin{pgfscope}%
\pgfsetbuttcap%
\pgfsetroundjoin%
\definecolor{currentfill}{rgb}{0.333333,0.333333,0.333333}%
\pgfsetfillcolor{currentfill}%
\pgfsetlinewidth{0.803000pt}%
\definecolor{currentstroke}{rgb}{0.333333,0.333333,0.333333}%
\pgfsetstrokecolor{currentstroke}%
\pgfsetdash{}{0pt}%
\pgfsys@defobject{currentmarker}{\pgfqpoint{0.000000in}{-0.048611in}}{\pgfqpoint{0.000000in}{0.000000in}}{%
\pgfpathmoveto{\pgfqpoint{0.000000in}{0.000000in}}%
\pgfpathlineto{\pgfqpoint{0.000000in}{-0.048611in}}%
\pgfusepath{stroke,fill}%
}%
\begin{pgfscope}%
\pgfsys@transformshift{1.420000in}{0.528000in}%
\pgfsys@useobject{currentmarker}{}%
\end{pgfscope}%
\end{pgfscope}%
\begin{pgfscope}%
\definecolor{textcolor}{rgb}{0.333333,0.333333,0.333333}%
\pgfsetstrokecolor{textcolor}%
\pgfsetfillcolor{textcolor}%
\pgftext[x=1.420000in,y=0.430778in,,top]{\color{textcolor}\sffamily\fontsize{10.000000}{12.000000}\selectfont 03:20}%
\end{pgfscope}%
\begin{pgfscope}%
\pgfpathrectangle{\pgfqpoint{0.800000in}{0.528000in}}{\pgfqpoint{4.960000in}{3.696000in}} %
\pgfusepath{clip}%
\pgfsetrectcap%
\pgfsetroundjoin%
\pgfsetlinewidth{0.803000pt}%
\definecolor{currentstroke}{rgb}{0.631373,0.631373,0.631373}%
\pgfsetstrokecolor{currentstroke}%
\pgfsetstrokeopacity{0.100000}%
\pgfsetdash{}{0pt}%
\pgfpathmoveto{\pgfqpoint{2.040000in}{0.528000in}}%
\pgfpathlineto{\pgfqpoint{2.040000in}{4.224000in}}%
\pgfusepath{stroke}%
\end{pgfscope}%
\begin{pgfscope}%
\pgfsetbuttcap%
\pgfsetroundjoin%
\definecolor{currentfill}{rgb}{0.333333,0.333333,0.333333}%
\pgfsetfillcolor{currentfill}%
\pgfsetlinewidth{0.803000pt}%
\definecolor{currentstroke}{rgb}{0.333333,0.333333,0.333333}%
\pgfsetstrokecolor{currentstroke}%
\pgfsetdash{}{0pt}%
\pgfsys@defobject{currentmarker}{\pgfqpoint{0.000000in}{-0.048611in}}{\pgfqpoint{0.000000in}{0.000000in}}{%
\pgfpathmoveto{\pgfqpoint{0.000000in}{0.000000in}}%
\pgfpathlineto{\pgfqpoint{0.000000in}{-0.048611in}}%
\pgfusepath{stroke,fill}%
}%
\begin{pgfscope}%
\pgfsys@transformshift{2.040000in}{0.528000in}%
\pgfsys@useobject{currentmarker}{}%
\end{pgfscope}%
\end{pgfscope}%
\begin{pgfscope}%
\definecolor{textcolor}{rgb}{0.333333,0.333333,0.333333}%
\pgfsetstrokecolor{textcolor}%
\pgfsetfillcolor{textcolor}%
\pgftext[x=2.040000in,y=0.430778in,,top]{\color{textcolor}\sffamily\fontsize{10.000000}{12.000000}\selectfont 06:40}%
\end{pgfscope}%
\begin{pgfscope}%
\pgfpathrectangle{\pgfqpoint{0.800000in}{0.528000in}}{\pgfqpoint{4.960000in}{3.696000in}} %
\pgfusepath{clip}%
\pgfsetrectcap%
\pgfsetroundjoin%
\pgfsetlinewidth{0.803000pt}%
\definecolor{currentstroke}{rgb}{0.631373,0.631373,0.631373}%
\pgfsetstrokecolor{currentstroke}%
\pgfsetstrokeopacity{0.100000}%
\pgfsetdash{}{0pt}%
\pgfpathmoveto{\pgfqpoint{2.660000in}{0.528000in}}%
\pgfpathlineto{\pgfqpoint{2.660000in}{4.224000in}}%
\pgfusepath{stroke}%
\end{pgfscope}%
\begin{pgfscope}%
\pgfsetbuttcap%
\pgfsetroundjoin%
\definecolor{currentfill}{rgb}{0.333333,0.333333,0.333333}%
\pgfsetfillcolor{currentfill}%
\pgfsetlinewidth{0.803000pt}%
\definecolor{currentstroke}{rgb}{0.333333,0.333333,0.333333}%
\pgfsetstrokecolor{currentstroke}%
\pgfsetdash{}{0pt}%
\pgfsys@defobject{currentmarker}{\pgfqpoint{0.000000in}{-0.048611in}}{\pgfqpoint{0.000000in}{0.000000in}}{%
\pgfpathmoveto{\pgfqpoint{0.000000in}{0.000000in}}%
\pgfpathlineto{\pgfqpoint{0.000000in}{-0.048611in}}%
\pgfusepath{stroke,fill}%
}%
\begin{pgfscope}%
\pgfsys@transformshift{2.660000in}{0.528000in}%
\pgfsys@useobject{currentmarker}{}%
\end{pgfscope}%
\end{pgfscope}%
\begin{pgfscope}%
\definecolor{textcolor}{rgb}{0.333333,0.333333,0.333333}%
\pgfsetstrokecolor{textcolor}%
\pgfsetfillcolor{textcolor}%
\pgftext[x=2.660000in,y=0.430778in,,top]{\color{textcolor}\sffamily\fontsize{10.000000}{12.000000}\selectfont 10:00}%
\end{pgfscope}%
\begin{pgfscope}%
\pgfpathrectangle{\pgfqpoint{0.800000in}{0.528000in}}{\pgfqpoint{4.960000in}{3.696000in}} %
\pgfusepath{clip}%
\pgfsetrectcap%
\pgfsetroundjoin%
\pgfsetlinewidth{0.803000pt}%
\definecolor{currentstroke}{rgb}{0.631373,0.631373,0.631373}%
\pgfsetstrokecolor{currentstroke}%
\pgfsetstrokeopacity{0.100000}%
\pgfsetdash{}{0pt}%
\pgfpathmoveto{\pgfqpoint{3.280000in}{0.528000in}}%
\pgfpathlineto{\pgfqpoint{3.280000in}{4.224000in}}%
\pgfusepath{stroke}%
\end{pgfscope}%
\begin{pgfscope}%
\pgfsetbuttcap%
\pgfsetroundjoin%
\definecolor{currentfill}{rgb}{0.333333,0.333333,0.333333}%
\pgfsetfillcolor{currentfill}%
\pgfsetlinewidth{0.803000pt}%
\definecolor{currentstroke}{rgb}{0.333333,0.333333,0.333333}%
\pgfsetstrokecolor{currentstroke}%
\pgfsetdash{}{0pt}%
\pgfsys@defobject{currentmarker}{\pgfqpoint{0.000000in}{-0.048611in}}{\pgfqpoint{0.000000in}{0.000000in}}{%
\pgfpathmoveto{\pgfqpoint{0.000000in}{0.000000in}}%
\pgfpathlineto{\pgfqpoint{0.000000in}{-0.048611in}}%
\pgfusepath{stroke,fill}%
}%
\begin{pgfscope}%
\pgfsys@transformshift{3.280000in}{0.528000in}%
\pgfsys@useobject{currentmarker}{}%
\end{pgfscope}%
\end{pgfscope}%
\begin{pgfscope}%
\definecolor{textcolor}{rgb}{0.333333,0.333333,0.333333}%
\pgfsetstrokecolor{textcolor}%
\pgfsetfillcolor{textcolor}%
\pgftext[x=3.280000in,y=0.430778in,,top]{\color{textcolor}\sffamily\fontsize{10.000000}{12.000000}\selectfont 13:20}%
\end{pgfscope}%
\begin{pgfscope}%
\pgfpathrectangle{\pgfqpoint{0.800000in}{0.528000in}}{\pgfqpoint{4.960000in}{3.696000in}} %
\pgfusepath{clip}%
\pgfsetrectcap%
\pgfsetroundjoin%
\pgfsetlinewidth{0.803000pt}%
\definecolor{currentstroke}{rgb}{0.631373,0.631373,0.631373}%
\pgfsetstrokecolor{currentstroke}%
\pgfsetstrokeopacity{0.100000}%
\pgfsetdash{}{0pt}%
\pgfpathmoveto{\pgfqpoint{3.900000in}{0.528000in}}%
\pgfpathlineto{\pgfqpoint{3.900000in}{4.224000in}}%
\pgfusepath{stroke}%
\end{pgfscope}%
\begin{pgfscope}%
\pgfsetbuttcap%
\pgfsetroundjoin%
\definecolor{currentfill}{rgb}{0.333333,0.333333,0.333333}%
\pgfsetfillcolor{currentfill}%
\pgfsetlinewidth{0.803000pt}%
\definecolor{currentstroke}{rgb}{0.333333,0.333333,0.333333}%
\pgfsetstrokecolor{currentstroke}%
\pgfsetdash{}{0pt}%
\pgfsys@defobject{currentmarker}{\pgfqpoint{0.000000in}{-0.048611in}}{\pgfqpoint{0.000000in}{0.000000in}}{%
\pgfpathmoveto{\pgfqpoint{0.000000in}{0.000000in}}%
\pgfpathlineto{\pgfqpoint{0.000000in}{-0.048611in}}%
\pgfusepath{stroke,fill}%
}%
\begin{pgfscope}%
\pgfsys@transformshift{3.900000in}{0.528000in}%
\pgfsys@useobject{currentmarker}{}%
\end{pgfscope}%
\end{pgfscope}%
\begin{pgfscope}%
\definecolor{textcolor}{rgb}{0.333333,0.333333,0.333333}%
\pgfsetstrokecolor{textcolor}%
\pgfsetfillcolor{textcolor}%
\pgftext[x=3.900000in,y=0.430778in,,top]{\color{textcolor}\sffamily\fontsize{10.000000}{12.000000}\selectfont 16:40}%
\end{pgfscope}%
\begin{pgfscope}%
\pgfpathrectangle{\pgfqpoint{0.800000in}{0.528000in}}{\pgfqpoint{4.960000in}{3.696000in}} %
\pgfusepath{clip}%
\pgfsetrectcap%
\pgfsetroundjoin%
\pgfsetlinewidth{0.803000pt}%
\definecolor{currentstroke}{rgb}{0.631373,0.631373,0.631373}%
\pgfsetstrokecolor{currentstroke}%
\pgfsetstrokeopacity{0.100000}%
\pgfsetdash{}{0pt}%
\pgfpathmoveto{\pgfqpoint{4.520000in}{0.528000in}}%
\pgfpathlineto{\pgfqpoint{4.520000in}{4.224000in}}%
\pgfusepath{stroke}%
\end{pgfscope}%
\begin{pgfscope}%
\pgfsetbuttcap%
\pgfsetroundjoin%
\definecolor{currentfill}{rgb}{0.333333,0.333333,0.333333}%
\pgfsetfillcolor{currentfill}%
\pgfsetlinewidth{0.803000pt}%
\definecolor{currentstroke}{rgb}{0.333333,0.333333,0.333333}%
\pgfsetstrokecolor{currentstroke}%
\pgfsetdash{}{0pt}%
\pgfsys@defobject{currentmarker}{\pgfqpoint{0.000000in}{-0.048611in}}{\pgfqpoint{0.000000in}{0.000000in}}{%
\pgfpathmoveto{\pgfqpoint{0.000000in}{0.000000in}}%
\pgfpathlineto{\pgfqpoint{0.000000in}{-0.048611in}}%
\pgfusepath{stroke,fill}%
}%
\begin{pgfscope}%
\pgfsys@transformshift{4.520000in}{0.528000in}%
\pgfsys@useobject{currentmarker}{}%
\end{pgfscope}%
\end{pgfscope}%
\begin{pgfscope}%
\definecolor{textcolor}{rgb}{0.333333,0.333333,0.333333}%
\pgfsetstrokecolor{textcolor}%
\pgfsetfillcolor{textcolor}%
\pgftext[x=4.520000in,y=0.430778in,,top]{\color{textcolor}\sffamily\fontsize{10.000000}{12.000000}\selectfont 20:00}%
\end{pgfscope}%
\begin{pgfscope}%
\pgfpathrectangle{\pgfqpoint{0.800000in}{0.528000in}}{\pgfqpoint{4.960000in}{3.696000in}} %
\pgfusepath{clip}%
\pgfsetrectcap%
\pgfsetroundjoin%
\pgfsetlinewidth{0.803000pt}%
\definecolor{currentstroke}{rgb}{0.631373,0.631373,0.631373}%
\pgfsetstrokecolor{currentstroke}%
\pgfsetstrokeopacity{0.100000}%
\pgfsetdash{}{0pt}%
\pgfpathmoveto{\pgfqpoint{5.140000in}{0.528000in}}%
\pgfpathlineto{\pgfqpoint{5.140000in}{4.224000in}}%
\pgfusepath{stroke}%
\end{pgfscope}%
\begin{pgfscope}%
\pgfsetbuttcap%
\pgfsetroundjoin%
\definecolor{currentfill}{rgb}{0.333333,0.333333,0.333333}%
\pgfsetfillcolor{currentfill}%
\pgfsetlinewidth{0.803000pt}%
\definecolor{currentstroke}{rgb}{0.333333,0.333333,0.333333}%
\pgfsetstrokecolor{currentstroke}%
\pgfsetdash{}{0pt}%
\pgfsys@defobject{currentmarker}{\pgfqpoint{0.000000in}{-0.048611in}}{\pgfqpoint{0.000000in}{0.000000in}}{%
\pgfpathmoveto{\pgfqpoint{0.000000in}{0.000000in}}%
\pgfpathlineto{\pgfqpoint{0.000000in}{-0.048611in}}%
\pgfusepath{stroke,fill}%
}%
\begin{pgfscope}%
\pgfsys@transformshift{5.140000in}{0.528000in}%
\pgfsys@useobject{currentmarker}{}%
\end{pgfscope}%
\end{pgfscope}%
\begin{pgfscope}%
\definecolor{textcolor}{rgb}{0.333333,0.333333,0.333333}%
\pgfsetstrokecolor{textcolor}%
\pgfsetfillcolor{textcolor}%
\pgftext[x=5.140000in,y=0.430778in,,top]{\color{textcolor}\sffamily\fontsize{10.000000}{12.000000}\selectfont 23:20}%
\end{pgfscope}%
\begin{pgfscope}%
\pgfpathrectangle{\pgfqpoint{0.800000in}{0.528000in}}{\pgfqpoint{4.960000in}{3.696000in}} %
\pgfusepath{clip}%
\pgfsetrectcap%
\pgfsetroundjoin%
\pgfsetlinewidth{0.803000pt}%
\definecolor{currentstroke}{rgb}{0.631373,0.631373,0.631373}%
\pgfsetstrokecolor{currentstroke}%
\pgfsetstrokeopacity{0.100000}%
\pgfsetdash{}{0pt}%
\pgfpathmoveto{\pgfqpoint{5.760000in}{0.528000in}}%
\pgfpathlineto{\pgfqpoint{5.760000in}{4.224000in}}%
\pgfusepath{stroke}%
\end{pgfscope}%
\begin{pgfscope}%
\pgfsetbuttcap%
\pgfsetroundjoin%
\definecolor{currentfill}{rgb}{0.333333,0.333333,0.333333}%
\pgfsetfillcolor{currentfill}%
\pgfsetlinewidth{0.803000pt}%
\definecolor{currentstroke}{rgb}{0.333333,0.333333,0.333333}%
\pgfsetstrokecolor{currentstroke}%
\pgfsetdash{}{0pt}%
\pgfsys@defobject{currentmarker}{\pgfqpoint{0.000000in}{-0.048611in}}{\pgfqpoint{0.000000in}{0.000000in}}{%
\pgfpathmoveto{\pgfqpoint{0.000000in}{0.000000in}}%
\pgfpathlineto{\pgfqpoint{0.000000in}{-0.048611in}}%
\pgfusepath{stroke,fill}%
}%
\begin{pgfscope}%
\pgfsys@transformshift{5.760000in}{0.528000in}%
\pgfsys@useobject{currentmarker}{}%
\end{pgfscope}%
\end{pgfscope}%
\begin{pgfscope}%
\definecolor{textcolor}{rgb}{0.333333,0.333333,0.333333}%
\pgfsetstrokecolor{textcolor}%
\pgfsetfillcolor{textcolor}%
\pgftext[x=5.760000in,y=0.430778in,,top]{\color{textcolor}\sffamily\fontsize{10.000000}{12.000000}\selectfont 26:40}%
\end{pgfscope}%
\begin{pgfscope}%
\definecolor{textcolor}{rgb}{0.333333,0.333333,0.333333}%
\pgfsetstrokecolor{textcolor}%
\pgfsetfillcolor{textcolor}%
\pgftext[x=3.280000in,y=0.255624in,,top]{\color{textcolor}\sffamily\fontsize{12.000000}{14.400000}\selectfont Tiempo (MM:SS)}%
\end{pgfscope}%
\begin{pgfscope}%
\pgfpathrectangle{\pgfqpoint{0.800000in}{0.528000in}}{\pgfqpoint{4.960000in}{3.696000in}} %
\pgfusepath{clip}%
\pgfsetrectcap%
\pgfsetroundjoin%
\pgfsetlinewidth{0.803000pt}%
\definecolor{currentstroke}{rgb}{0.631373,0.631373,0.631373}%
\pgfsetstrokecolor{currentstroke}%
\pgfsetstrokeopacity{0.100000}%
\pgfsetdash{}{0pt}%
\pgfpathmoveto{\pgfqpoint{0.800000in}{0.654081in}}%
\pgfpathlineto{\pgfqpoint{5.760000in}{0.654081in}}%
\pgfusepath{stroke}%
\end{pgfscope}%
\begin{pgfscope}%
\pgfsetbuttcap%
\pgfsetroundjoin%
\definecolor{currentfill}{rgb}{0.333333,0.333333,0.333333}%
\pgfsetfillcolor{currentfill}%
\pgfsetlinewidth{0.803000pt}%
\definecolor{currentstroke}{rgb}{0.333333,0.333333,0.333333}%
\pgfsetstrokecolor{currentstroke}%
\pgfsetdash{}{0pt}%
\pgfsys@defobject{currentmarker}{\pgfqpoint{-0.048611in}{0.000000in}}{\pgfqpoint{0.000000in}{0.000000in}}{%
\pgfpathmoveto{\pgfqpoint{0.000000in}{0.000000in}}%
\pgfpathlineto{\pgfqpoint{-0.048611in}{0.000000in}}%
\pgfusepath{stroke,fill}%
}%
\begin{pgfscope}%
\pgfsys@transformshift{0.800000in}{0.654081in}%
\pgfsys@useobject{currentmarker}{}%
\end{pgfscope}%
\end{pgfscope}%
\begin{pgfscope}%
\definecolor{textcolor}{rgb}{0.333333,0.333333,0.333333}%
\pgfsetstrokecolor{textcolor}%
\pgfsetfillcolor{textcolor}%
\pgftext[x=0.629862in,y=0.605856in,left,base]{\color{textcolor}\sffamily\fontsize{10.000000}{12.000000}\selectfont 0}%
\end{pgfscope}%
\begin{pgfscope}%
\pgfpathrectangle{\pgfqpoint{0.800000in}{0.528000in}}{\pgfqpoint{4.960000in}{3.696000in}} %
\pgfusepath{clip}%
\pgfsetrectcap%
\pgfsetroundjoin%
\pgfsetlinewidth{0.803000pt}%
\definecolor{currentstroke}{rgb}{0.631373,0.631373,0.631373}%
\pgfsetstrokecolor{currentstroke}%
\pgfsetstrokeopacity{0.100000}%
\pgfsetdash{}{0pt}%
\pgfpathmoveto{\pgfqpoint{0.800000in}{1.072905in}}%
\pgfpathlineto{\pgfqpoint{5.760000in}{1.072905in}}%
\pgfusepath{stroke}%
\end{pgfscope}%
\begin{pgfscope}%
\pgfsetbuttcap%
\pgfsetroundjoin%
\definecolor{currentfill}{rgb}{0.333333,0.333333,0.333333}%
\pgfsetfillcolor{currentfill}%
\pgfsetlinewidth{0.803000pt}%
\definecolor{currentstroke}{rgb}{0.333333,0.333333,0.333333}%
\pgfsetstrokecolor{currentstroke}%
\pgfsetdash{}{0pt}%
\pgfsys@defobject{currentmarker}{\pgfqpoint{-0.048611in}{0.000000in}}{\pgfqpoint{0.000000in}{0.000000in}}{%
\pgfpathmoveto{\pgfqpoint{0.000000in}{0.000000in}}%
\pgfpathlineto{\pgfqpoint{-0.048611in}{0.000000in}}%
\pgfusepath{stroke,fill}%
}%
\begin{pgfscope}%
\pgfsys@transformshift{0.800000in}{1.072905in}%
\pgfsys@useobject{currentmarker}{}%
\end{pgfscope}%
\end{pgfscope}%
\begin{pgfscope}%
\definecolor{textcolor}{rgb}{0.333333,0.333333,0.333333}%
\pgfsetstrokecolor{textcolor}%
\pgfsetfillcolor{textcolor}%
\pgftext[x=0.556946in,y=1.024680in,left,base]{\color{textcolor}\sffamily\fontsize{10.000000}{12.000000}\selectfont 10}%
\end{pgfscope}%
\begin{pgfscope}%
\pgfpathrectangle{\pgfqpoint{0.800000in}{0.528000in}}{\pgfqpoint{4.960000in}{3.696000in}} %
\pgfusepath{clip}%
\pgfsetrectcap%
\pgfsetroundjoin%
\pgfsetlinewidth{0.803000pt}%
\definecolor{currentstroke}{rgb}{0.631373,0.631373,0.631373}%
\pgfsetstrokecolor{currentstroke}%
\pgfsetstrokeopacity{0.100000}%
\pgfsetdash{}{0pt}%
\pgfpathmoveto{\pgfqpoint{0.800000in}{1.491729in}}%
\pgfpathlineto{\pgfqpoint{5.760000in}{1.491729in}}%
\pgfusepath{stroke}%
\end{pgfscope}%
\begin{pgfscope}%
\pgfsetbuttcap%
\pgfsetroundjoin%
\definecolor{currentfill}{rgb}{0.333333,0.333333,0.333333}%
\pgfsetfillcolor{currentfill}%
\pgfsetlinewidth{0.803000pt}%
\definecolor{currentstroke}{rgb}{0.333333,0.333333,0.333333}%
\pgfsetstrokecolor{currentstroke}%
\pgfsetdash{}{0pt}%
\pgfsys@defobject{currentmarker}{\pgfqpoint{-0.048611in}{0.000000in}}{\pgfqpoint{0.000000in}{0.000000in}}{%
\pgfpathmoveto{\pgfqpoint{0.000000in}{0.000000in}}%
\pgfpathlineto{\pgfqpoint{-0.048611in}{0.000000in}}%
\pgfusepath{stroke,fill}%
}%
\begin{pgfscope}%
\pgfsys@transformshift{0.800000in}{1.491729in}%
\pgfsys@useobject{currentmarker}{}%
\end{pgfscope}%
\end{pgfscope}%
\begin{pgfscope}%
\definecolor{textcolor}{rgb}{0.333333,0.333333,0.333333}%
\pgfsetstrokecolor{textcolor}%
\pgfsetfillcolor{textcolor}%
\pgftext[x=0.556946in,y=1.443504in,left,base]{\color{textcolor}\sffamily\fontsize{10.000000}{12.000000}\selectfont 20}%
\end{pgfscope}%
\begin{pgfscope}%
\pgfpathrectangle{\pgfqpoint{0.800000in}{0.528000in}}{\pgfqpoint{4.960000in}{3.696000in}} %
\pgfusepath{clip}%
\pgfsetrectcap%
\pgfsetroundjoin%
\pgfsetlinewidth{0.803000pt}%
\definecolor{currentstroke}{rgb}{0.631373,0.631373,0.631373}%
\pgfsetstrokecolor{currentstroke}%
\pgfsetstrokeopacity{0.100000}%
\pgfsetdash{}{0pt}%
\pgfpathmoveto{\pgfqpoint{0.800000in}{1.910553in}}%
\pgfpathlineto{\pgfqpoint{5.760000in}{1.910553in}}%
\pgfusepath{stroke}%
\end{pgfscope}%
\begin{pgfscope}%
\pgfsetbuttcap%
\pgfsetroundjoin%
\definecolor{currentfill}{rgb}{0.333333,0.333333,0.333333}%
\pgfsetfillcolor{currentfill}%
\pgfsetlinewidth{0.803000pt}%
\definecolor{currentstroke}{rgb}{0.333333,0.333333,0.333333}%
\pgfsetstrokecolor{currentstroke}%
\pgfsetdash{}{0pt}%
\pgfsys@defobject{currentmarker}{\pgfqpoint{-0.048611in}{0.000000in}}{\pgfqpoint{0.000000in}{0.000000in}}{%
\pgfpathmoveto{\pgfqpoint{0.000000in}{0.000000in}}%
\pgfpathlineto{\pgfqpoint{-0.048611in}{0.000000in}}%
\pgfusepath{stroke,fill}%
}%
\begin{pgfscope}%
\pgfsys@transformshift{0.800000in}{1.910553in}%
\pgfsys@useobject{currentmarker}{}%
\end{pgfscope}%
\end{pgfscope}%
\begin{pgfscope}%
\definecolor{textcolor}{rgb}{0.333333,0.333333,0.333333}%
\pgfsetstrokecolor{textcolor}%
\pgfsetfillcolor{textcolor}%
\pgftext[x=0.556946in,y=1.862328in,left,base]{\color{textcolor}\sffamily\fontsize{10.000000}{12.000000}\selectfont 30}%
\end{pgfscope}%
\begin{pgfscope}%
\pgfpathrectangle{\pgfqpoint{0.800000in}{0.528000in}}{\pgfqpoint{4.960000in}{3.696000in}} %
\pgfusepath{clip}%
\pgfsetrectcap%
\pgfsetroundjoin%
\pgfsetlinewidth{0.803000pt}%
\definecolor{currentstroke}{rgb}{0.631373,0.631373,0.631373}%
\pgfsetstrokecolor{currentstroke}%
\pgfsetstrokeopacity{0.100000}%
\pgfsetdash{}{0pt}%
\pgfpathmoveto{\pgfqpoint{0.800000in}{2.329377in}}%
\pgfpathlineto{\pgfqpoint{5.760000in}{2.329377in}}%
\pgfusepath{stroke}%
\end{pgfscope}%
\begin{pgfscope}%
\pgfsetbuttcap%
\pgfsetroundjoin%
\definecolor{currentfill}{rgb}{0.333333,0.333333,0.333333}%
\pgfsetfillcolor{currentfill}%
\pgfsetlinewidth{0.803000pt}%
\definecolor{currentstroke}{rgb}{0.333333,0.333333,0.333333}%
\pgfsetstrokecolor{currentstroke}%
\pgfsetdash{}{0pt}%
\pgfsys@defobject{currentmarker}{\pgfqpoint{-0.048611in}{0.000000in}}{\pgfqpoint{0.000000in}{0.000000in}}{%
\pgfpathmoveto{\pgfqpoint{0.000000in}{0.000000in}}%
\pgfpathlineto{\pgfqpoint{-0.048611in}{0.000000in}}%
\pgfusepath{stroke,fill}%
}%
\begin{pgfscope}%
\pgfsys@transformshift{0.800000in}{2.329377in}%
\pgfsys@useobject{currentmarker}{}%
\end{pgfscope}%
\end{pgfscope}%
\begin{pgfscope}%
\definecolor{textcolor}{rgb}{0.333333,0.333333,0.333333}%
\pgfsetstrokecolor{textcolor}%
\pgfsetfillcolor{textcolor}%
\pgftext[x=0.556946in,y=2.281152in,left,base]{\color{textcolor}\sffamily\fontsize{10.000000}{12.000000}\selectfont 40}%
\end{pgfscope}%
\begin{pgfscope}%
\pgfpathrectangle{\pgfqpoint{0.800000in}{0.528000in}}{\pgfqpoint{4.960000in}{3.696000in}} %
\pgfusepath{clip}%
\pgfsetrectcap%
\pgfsetroundjoin%
\pgfsetlinewidth{0.803000pt}%
\definecolor{currentstroke}{rgb}{0.631373,0.631373,0.631373}%
\pgfsetstrokecolor{currentstroke}%
\pgfsetstrokeopacity{0.100000}%
\pgfsetdash{}{0pt}%
\pgfpathmoveto{\pgfqpoint{0.800000in}{2.748201in}}%
\pgfpathlineto{\pgfqpoint{5.760000in}{2.748201in}}%
\pgfusepath{stroke}%
\end{pgfscope}%
\begin{pgfscope}%
\pgfsetbuttcap%
\pgfsetroundjoin%
\definecolor{currentfill}{rgb}{0.333333,0.333333,0.333333}%
\pgfsetfillcolor{currentfill}%
\pgfsetlinewidth{0.803000pt}%
\definecolor{currentstroke}{rgb}{0.333333,0.333333,0.333333}%
\pgfsetstrokecolor{currentstroke}%
\pgfsetdash{}{0pt}%
\pgfsys@defobject{currentmarker}{\pgfqpoint{-0.048611in}{0.000000in}}{\pgfqpoint{0.000000in}{0.000000in}}{%
\pgfpathmoveto{\pgfqpoint{0.000000in}{0.000000in}}%
\pgfpathlineto{\pgfqpoint{-0.048611in}{0.000000in}}%
\pgfusepath{stroke,fill}%
}%
\begin{pgfscope}%
\pgfsys@transformshift{0.800000in}{2.748201in}%
\pgfsys@useobject{currentmarker}{}%
\end{pgfscope}%
\end{pgfscope}%
\begin{pgfscope}%
\definecolor{textcolor}{rgb}{0.333333,0.333333,0.333333}%
\pgfsetstrokecolor{textcolor}%
\pgfsetfillcolor{textcolor}%
\pgftext[x=0.556946in,y=2.699976in,left,base]{\color{textcolor}\sffamily\fontsize{10.000000}{12.000000}\selectfont 50}%
\end{pgfscope}%
\begin{pgfscope}%
\pgfpathrectangle{\pgfqpoint{0.800000in}{0.528000in}}{\pgfqpoint{4.960000in}{3.696000in}} %
\pgfusepath{clip}%
\pgfsetrectcap%
\pgfsetroundjoin%
\pgfsetlinewidth{0.803000pt}%
\definecolor{currentstroke}{rgb}{0.631373,0.631373,0.631373}%
\pgfsetstrokecolor{currentstroke}%
\pgfsetstrokeopacity{0.100000}%
\pgfsetdash{}{0pt}%
\pgfpathmoveto{\pgfqpoint{0.800000in}{3.167025in}}%
\pgfpathlineto{\pgfqpoint{5.760000in}{3.167025in}}%
\pgfusepath{stroke}%
\end{pgfscope}%
\begin{pgfscope}%
\pgfsetbuttcap%
\pgfsetroundjoin%
\definecolor{currentfill}{rgb}{0.333333,0.333333,0.333333}%
\pgfsetfillcolor{currentfill}%
\pgfsetlinewidth{0.803000pt}%
\definecolor{currentstroke}{rgb}{0.333333,0.333333,0.333333}%
\pgfsetstrokecolor{currentstroke}%
\pgfsetdash{}{0pt}%
\pgfsys@defobject{currentmarker}{\pgfqpoint{-0.048611in}{0.000000in}}{\pgfqpoint{0.000000in}{0.000000in}}{%
\pgfpathmoveto{\pgfqpoint{0.000000in}{0.000000in}}%
\pgfpathlineto{\pgfqpoint{-0.048611in}{0.000000in}}%
\pgfusepath{stroke,fill}%
}%
\begin{pgfscope}%
\pgfsys@transformshift{0.800000in}{3.167025in}%
\pgfsys@useobject{currentmarker}{}%
\end{pgfscope}%
\end{pgfscope}%
\begin{pgfscope}%
\definecolor{textcolor}{rgb}{0.333333,0.333333,0.333333}%
\pgfsetstrokecolor{textcolor}%
\pgfsetfillcolor{textcolor}%
\pgftext[x=0.556946in,y=3.118800in,left,base]{\color{textcolor}\sffamily\fontsize{10.000000}{12.000000}\selectfont 60}%
\end{pgfscope}%
\begin{pgfscope}%
\pgfpathrectangle{\pgfqpoint{0.800000in}{0.528000in}}{\pgfqpoint{4.960000in}{3.696000in}} %
\pgfusepath{clip}%
\pgfsetrectcap%
\pgfsetroundjoin%
\pgfsetlinewidth{0.803000pt}%
\definecolor{currentstroke}{rgb}{0.631373,0.631373,0.631373}%
\pgfsetstrokecolor{currentstroke}%
\pgfsetstrokeopacity{0.100000}%
\pgfsetdash{}{0pt}%
\pgfpathmoveto{\pgfqpoint{0.800000in}{3.585849in}}%
\pgfpathlineto{\pgfqpoint{5.760000in}{3.585849in}}%
\pgfusepath{stroke}%
\end{pgfscope}%
\begin{pgfscope}%
\pgfsetbuttcap%
\pgfsetroundjoin%
\definecolor{currentfill}{rgb}{0.333333,0.333333,0.333333}%
\pgfsetfillcolor{currentfill}%
\pgfsetlinewidth{0.803000pt}%
\definecolor{currentstroke}{rgb}{0.333333,0.333333,0.333333}%
\pgfsetstrokecolor{currentstroke}%
\pgfsetdash{}{0pt}%
\pgfsys@defobject{currentmarker}{\pgfqpoint{-0.048611in}{0.000000in}}{\pgfqpoint{0.000000in}{0.000000in}}{%
\pgfpathmoveto{\pgfqpoint{0.000000in}{0.000000in}}%
\pgfpathlineto{\pgfqpoint{-0.048611in}{0.000000in}}%
\pgfusepath{stroke,fill}%
}%
\begin{pgfscope}%
\pgfsys@transformshift{0.800000in}{3.585849in}%
\pgfsys@useobject{currentmarker}{}%
\end{pgfscope}%
\end{pgfscope}%
\begin{pgfscope}%
\definecolor{textcolor}{rgb}{0.333333,0.333333,0.333333}%
\pgfsetstrokecolor{textcolor}%
\pgfsetfillcolor{textcolor}%
\pgftext[x=0.556946in,y=3.537624in,left,base]{\color{textcolor}\sffamily\fontsize{10.000000}{12.000000}\selectfont 70}%
\end{pgfscope}%
\begin{pgfscope}%
\pgfpathrectangle{\pgfqpoint{0.800000in}{0.528000in}}{\pgfqpoint{4.960000in}{3.696000in}} %
\pgfusepath{clip}%
\pgfsetrectcap%
\pgfsetroundjoin%
\pgfsetlinewidth{0.803000pt}%
\definecolor{currentstroke}{rgb}{0.631373,0.631373,0.631373}%
\pgfsetstrokecolor{currentstroke}%
\pgfsetstrokeopacity{0.100000}%
\pgfsetdash{}{0pt}%
\pgfpathmoveto{\pgfqpoint{0.800000in}{4.004673in}}%
\pgfpathlineto{\pgfqpoint{5.760000in}{4.004673in}}%
\pgfusepath{stroke}%
\end{pgfscope}%
\begin{pgfscope}%
\pgfsetbuttcap%
\pgfsetroundjoin%
\definecolor{currentfill}{rgb}{0.333333,0.333333,0.333333}%
\pgfsetfillcolor{currentfill}%
\pgfsetlinewidth{0.803000pt}%
\definecolor{currentstroke}{rgb}{0.333333,0.333333,0.333333}%
\pgfsetstrokecolor{currentstroke}%
\pgfsetdash{}{0pt}%
\pgfsys@defobject{currentmarker}{\pgfqpoint{-0.048611in}{0.000000in}}{\pgfqpoint{0.000000in}{0.000000in}}{%
\pgfpathmoveto{\pgfqpoint{0.000000in}{0.000000in}}%
\pgfpathlineto{\pgfqpoint{-0.048611in}{0.000000in}}%
\pgfusepath{stroke,fill}%
}%
\begin{pgfscope}%
\pgfsys@transformshift{0.800000in}{4.004673in}%
\pgfsys@useobject{currentmarker}{}%
\end{pgfscope}%
\end{pgfscope}%
\begin{pgfscope}%
\definecolor{textcolor}{rgb}{0.333333,0.333333,0.333333}%
\pgfsetstrokecolor{textcolor}%
\pgfsetfillcolor{textcolor}%
\pgftext[x=0.556946in,y=3.956448in,left,base]{\color{textcolor}\sffamily\fontsize{10.000000}{12.000000}\selectfont 80}%
\end{pgfscope}%
\begin{pgfscope}%
\definecolor{textcolor}{rgb}{0.333333,0.333333,0.333333}%
\pgfsetstrokecolor{textcolor}%
\pgfsetfillcolor{textcolor}%
\pgftext[x=0.501390in,y=2.376000in,,bottom,rotate=90.000000]{\color{textcolor}\sffamily\fontsize{12.000000}{14.400000}\selectfont Operaciones I/O por segundo}%
\end{pgfscope}%
\begin{pgfscope}%
\pgfpathrectangle{\pgfqpoint{0.800000in}{0.528000in}}{\pgfqpoint{4.960000in}{3.696000in}} %
\pgfusepath{clip}%
\pgfsetrectcap%
\pgfsetroundjoin%
\pgfsetlinewidth{1.505625pt}%
\definecolor{currentstroke}{rgb}{0.886275,0.290196,0.200000}%
\pgfsetstrokecolor{currentstroke}%
\pgfsetdash{}{0pt}%
\pgfpathmoveto{\pgfqpoint{0.815500in}{1.180524in}}%
\pgfpathlineto{\pgfqpoint{0.831000in}{1.582414in}}%
\pgfpathlineto{\pgfqpoint{0.846500in}{1.055971in}}%
\pgfpathlineto{\pgfqpoint{0.862000in}{0.746212in}}%
\pgfpathlineto{\pgfqpoint{0.877500in}{1.005904in}}%
\pgfpathlineto{\pgfqpoint{0.893000in}{0.955949in}}%
\pgfpathlineto{\pgfqpoint{0.908500in}{2.603830in}}%
\pgfpathlineto{\pgfqpoint{0.924000in}{1.290635in}}%
\pgfpathlineto{\pgfqpoint{0.939500in}{1.508901in}}%
\pgfpathlineto{\pgfqpoint{0.955000in}{1.401156in}}%
\pgfpathlineto{\pgfqpoint{0.970500in}{1.348002in}}%
\pgfpathlineto{\pgfqpoint{0.986000in}{1.198059in}}%
\pgfpathlineto{\pgfqpoint{1.001500in}{1.223964in}}%
\pgfpathlineto{\pgfqpoint{1.017000in}{0.889208in}}%
\pgfpathlineto{\pgfqpoint{1.032500in}{0.938506in}}%
\pgfpathlineto{\pgfqpoint{1.048000in}{1.969593in}}%
\pgfpathlineto{\pgfqpoint{1.063500in}{0.947403in}}%
\pgfpathlineto{\pgfqpoint{1.079000in}{1.089613in}}%
\pgfpathlineto{\pgfqpoint{1.094500in}{1.381577in}}%
\pgfpathlineto{\pgfqpoint{1.110000in}{0.981585in}}%
\pgfpathlineto{\pgfqpoint{1.125500in}{1.180374in}}%
\pgfpathlineto{\pgfqpoint{1.141000in}{0.779941in}}%
\pgfpathlineto{\pgfqpoint{1.156500in}{1.364730in}}%
\pgfpathlineto{\pgfqpoint{1.172000in}{0.939477in}}%
\pgfpathlineto{\pgfqpoint{1.187500in}{1.197401in}}%
\pgfpathlineto{\pgfqpoint{1.203000in}{1.359183in}}%
\pgfpathlineto{\pgfqpoint{1.218500in}{1.801096in}}%
\pgfpathlineto{\pgfqpoint{1.234000in}{0.921991in}}%
\pgfpathlineto{\pgfqpoint{1.249500in}{1.173264in}}%
\pgfpathlineto{\pgfqpoint{1.265000in}{0.804872in}}%
\pgfpathlineto{\pgfqpoint{1.280500in}{2.087154in}}%
\pgfpathlineto{\pgfqpoint{1.296000in}{1.123587in}}%
\pgfpathlineto{\pgfqpoint{1.311500in}{1.139086in}}%
\pgfpathlineto{\pgfqpoint{1.327000in}{1.350091in}}%
\pgfpathlineto{\pgfqpoint{1.342500in}{1.030529in}}%
\pgfpathlineto{\pgfqpoint{1.358000in}{0.871935in}}%
\pgfpathlineto{\pgfqpoint{1.373500in}{0.989102in}}%
\pgfpathlineto{\pgfqpoint{1.389000in}{1.298997in}}%
\pgfpathlineto{\pgfqpoint{1.404500in}{1.064726in}}%
\pgfpathlineto{\pgfqpoint{1.420000in}{1.156416in}}%
\pgfpathlineto{\pgfqpoint{1.435500in}{1.743549in}}%
\pgfpathlineto{\pgfqpoint{1.451000in}{2.070986in}}%
\pgfpathlineto{\pgfqpoint{1.466500in}{2.086346in}}%
\pgfpathlineto{\pgfqpoint{1.482000in}{1.585028in}}%
\pgfpathlineto{\pgfqpoint{1.497500in}{1.522836in}}%
\pgfpathlineto{\pgfqpoint{1.513000in}{1.031131in}}%
\pgfpathlineto{\pgfqpoint{1.528500in}{2.221250in}}%
\pgfpathlineto{\pgfqpoint{1.544000in}{1.090402in}}%
\pgfpathlineto{\pgfqpoint{1.559500in}{1.389951in}}%
\pgfpathlineto{\pgfqpoint{1.575000in}{1.348566in}}%
\pgfpathlineto{\pgfqpoint{1.590500in}{1.098249in}}%
\pgfpathlineto{\pgfqpoint{1.606000in}{1.407590in}}%
\pgfpathlineto{\pgfqpoint{1.621500in}{1.508733in}}%
\pgfpathlineto{\pgfqpoint{1.637000in}{1.215694in}}%
\pgfpathlineto{\pgfqpoint{1.652500in}{1.123862in}}%
\pgfpathlineto{\pgfqpoint{1.668000in}{2.085756in}}%
\pgfpathlineto{\pgfqpoint{1.683500in}{1.449089in}}%
\pgfpathlineto{\pgfqpoint{1.699000in}{0.955836in}}%
\pgfpathlineto{\pgfqpoint{1.714500in}{1.592799in}}%
\pgfpathlineto{\pgfqpoint{1.730000in}{2.050940in}}%
\pgfpathlineto{\pgfqpoint{1.745500in}{1.518790in}}%
\pgfpathlineto{\pgfqpoint{1.761000in}{1.088552in}}%
\pgfpathlineto{\pgfqpoint{1.776500in}{1.642204in}}%
\pgfpathlineto{\pgfqpoint{1.792000in}{1.149331in}}%
\pgfpathlineto{\pgfqpoint{1.807500in}{1.540998in}}%
\pgfpathlineto{\pgfqpoint{1.838500in}{1.390230in}}%
\pgfpathlineto{\pgfqpoint{1.854000in}{1.232087in}}%
\pgfpathlineto{\pgfqpoint{1.869500in}{1.559087in}}%
\pgfpathlineto{\pgfqpoint{1.900500in}{0.955785in}}%
\pgfpathlineto{\pgfqpoint{1.916000in}{1.190436in}}%
\pgfpathlineto{\pgfqpoint{1.931500in}{1.323860in}}%
\pgfpathlineto{\pgfqpoint{1.947000in}{1.047734in}}%
\pgfpathlineto{\pgfqpoint{1.962500in}{1.106091in}}%
\pgfpathlineto{\pgfqpoint{1.978000in}{1.224304in}}%
\pgfpathlineto{\pgfqpoint{1.993500in}{0.981188in}}%
\pgfpathlineto{\pgfqpoint{2.009000in}{0.913670in}}%
\pgfpathlineto{\pgfqpoint{2.024500in}{1.013443in}}%
\pgfpathlineto{\pgfqpoint{2.040000in}{0.779739in}}%
\pgfpathlineto{\pgfqpoint{2.055500in}{1.258700in}}%
\pgfpathlineto{\pgfqpoint{2.071000in}{1.198379in}}%
\pgfpathlineto{\pgfqpoint{2.086500in}{1.473911in}}%
\pgfpathlineto{\pgfqpoint{2.102000in}{0.997136in}}%
\pgfpathlineto{\pgfqpoint{2.117500in}{1.249097in}}%
\pgfpathlineto{\pgfqpoint{2.133000in}{1.031212in}}%
\pgfpathlineto{\pgfqpoint{2.148500in}{1.342108in}}%
\pgfpathlineto{\pgfqpoint{2.164000in}{0.846722in}}%
\pgfpathlineto{\pgfqpoint{2.179500in}{1.265128in}}%
\pgfpathlineto{\pgfqpoint{2.195000in}{1.157020in}}%
\pgfpathlineto{\pgfqpoint{2.210500in}{1.131391in}}%
\pgfpathlineto{\pgfqpoint{2.226000in}{0.863556in}}%
\pgfpathlineto{\pgfqpoint{2.241500in}{1.315091in}}%
\pgfpathlineto{\pgfqpoint{2.257000in}{0.938962in}}%
\pgfpathlineto{\pgfqpoint{2.272500in}{1.164168in}}%
\pgfpathlineto{\pgfqpoint{2.288000in}{1.274257in}}%
\pgfpathlineto{\pgfqpoint{2.303500in}{1.509770in}}%
\pgfpathlineto{\pgfqpoint{2.334500in}{0.855060in}}%
\pgfpathlineto{\pgfqpoint{2.350000in}{1.584143in}}%
\pgfpathlineto{\pgfqpoint{2.365500in}{0.988905in}}%
\pgfpathlineto{\pgfqpoint{2.381000in}{1.157520in}}%
\pgfpathlineto{\pgfqpoint{2.396500in}{1.415194in}}%
\pgfpathlineto{\pgfqpoint{2.412000in}{1.081664in}}%
\pgfpathlineto{\pgfqpoint{2.427500in}{1.097189in}}%
\pgfpathlineto{\pgfqpoint{2.443000in}{0.947753in}}%
\pgfpathlineto{\pgfqpoint{2.458500in}{1.591042in}}%
\pgfpathlineto{\pgfqpoint{2.474000in}{1.451421in}}%
\pgfpathlineto{\pgfqpoint{2.489500in}{1.440164in}}%
\pgfpathlineto{\pgfqpoint{2.505000in}{1.265115in}}%
\pgfpathlineto{\pgfqpoint{2.520500in}{1.265989in}}%
\pgfpathlineto{\pgfqpoint{2.536000in}{3.396186in}}%
\pgfpathlineto{\pgfqpoint{2.551500in}{1.172536in}}%
\pgfpathlineto{\pgfqpoint{2.567000in}{0.813537in}}%
\pgfpathlineto{\pgfqpoint{2.582500in}{1.348280in}}%
\pgfpathlineto{\pgfqpoint{2.598000in}{0.922468in}}%
\pgfpathlineto{\pgfqpoint{2.613500in}{1.098340in}}%
\pgfpathlineto{\pgfqpoint{2.629000in}{0.712605in}}%
\pgfpathlineto{\pgfqpoint{2.644500in}{2.105949in}}%
\pgfpathlineto{\pgfqpoint{2.660000in}{0.846759in}}%
\pgfpathlineto{\pgfqpoint{2.675500in}{1.096918in}}%
\pgfpathlineto{\pgfqpoint{2.691000in}{2.021759in}}%
\pgfpathlineto{\pgfqpoint{2.706500in}{1.272880in}}%
\pgfpathlineto{\pgfqpoint{2.722000in}{0.922156in}}%
\pgfpathlineto{\pgfqpoint{2.737500in}{0.981646in}}%
\pgfpathlineto{\pgfqpoint{2.753000in}{1.223392in}}%
\pgfpathlineto{\pgfqpoint{2.768500in}{1.047968in}}%
\pgfpathlineto{\pgfqpoint{2.784000in}{0.963979in}}%
\pgfpathlineto{\pgfqpoint{2.799500in}{1.750625in}}%
\pgfpathlineto{\pgfqpoint{2.815000in}{0.897069in}}%
\pgfpathlineto{\pgfqpoint{2.830500in}{0.871541in}}%
\pgfpathlineto{\pgfqpoint{2.846000in}{1.282253in}}%
\pgfpathlineto{\pgfqpoint{2.861500in}{1.022680in}}%
\pgfpathlineto{\pgfqpoint{2.877000in}{0.880085in}}%
\pgfpathlineto{\pgfqpoint{2.892500in}{1.040287in}}%
\pgfpathlineto{\pgfqpoint{2.908000in}{2.135918in}}%
\pgfpathlineto{\pgfqpoint{2.923500in}{0.980311in}}%
\pgfpathlineto{\pgfqpoint{2.939000in}{0.922717in}}%
\pgfpathlineto{\pgfqpoint{2.954500in}{1.817302in}}%
\pgfpathlineto{\pgfqpoint{2.970000in}{0.813153in}}%
\pgfpathlineto{\pgfqpoint{2.985500in}{1.224190in}}%
\pgfpathlineto{\pgfqpoint{3.001000in}{0.913569in}}%
\pgfpathlineto{\pgfqpoint{3.016500in}{1.617854in}}%
\pgfpathlineto{\pgfqpoint{3.032000in}{0.813151in}}%
\pgfpathlineto{\pgfqpoint{3.047500in}{1.115017in}}%
\pgfpathlineto{\pgfqpoint{3.063000in}{1.139864in}}%
\pgfpathlineto{\pgfqpoint{3.078500in}{1.156315in}}%
\pgfpathlineto{\pgfqpoint{3.094000in}{0.821627in}}%
\pgfpathlineto{\pgfqpoint{3.109500in}{1.577233in}}%
\pgfpathlineto{\pgfqpoint{3.125000in}{0.921566in}}%
\pgfpathlineto{\pgfqpoint{3.140500in}{0.897116in}}%
\pgfpathlineto{\pgfqpoint{3.156000in}{0.838309in}}%
\pgfpathlineto{\pgfqpoint{3.171500in}{1.550630in}}%
\pgfpathlineto{\pgfqpoint{3.187000in}{0.930309in}}%
\pgfpathlineto{\pgfqpoint{3.202500in}{1.148239in}}%
\pgfpathlineto{\pgfqpoint{3.218000in}{0.871934in}}%
\pgfpathlineto{\pgfqpoint{3.233500in}{0.922204in}}%
\pgfpathlineto{\pgfqpoint{3.249000in}{1.474890in}}%
\pgfpathlineto{\pgfqpoint{3.264500in}{0.947401in}}%
\pgfpathlineto{\pgfqpoint{3.280000in}{1.022316in}}%
\pgfpathlineto{\pgfqpoint{3.295500in}{1.233040in}}%
\pgfpathlineto{\pgfqpoint{3.311000in}{0.796522in}}%
\pgfpathlineto{\pgfqpoint{3.326500in}{1.307508in}}%
\pgfpathlineto{\pgfqpoint{3.342000in}{0.788144in}}%
\pgfpathlineto{\pgfqpoint{3.357500in}{1.432389in}}%
\pgfpathlineto{\pgfqpoint{3.373000in}{0.922206in}}%
\pgfpathlineto{\pgfqpoint{3.388500in}{1.206646in}}%
\pgfpathlineto{\pgfqpoint{3.404000in}{1.399808in}}%
\pgfpathlineto{\pgfqpoint{3.419500in}{1.265379in}}%
\pgfpathlineto{\pgfqpoint{3.435000in}{0.762789in}}%
\pgfpathlineto{\pgfqpoint{3.450500in}{1.510607in}}%
\pgfpathlineto{\pgfqpoint{3.466000in}{0.863180in}}%
\pgfpathlineto{\pgfqpoint{3.481500in}{1.140157in}}%
\pgfpathlineto{\pgfqpoint{3.497000in}{0.829722in}}%
\pgfpathlineto{\pgfqpoint{3.512500in}{0.931139in}}%
\pgfpathlineto{\pgfqpoint{3.528000in}{0.963979in}}%
\pgfpathlineto{\pgfqpoint{3.543500in}{1.340328in}}%
\pgfpathlineto{\pgfqpoint{3.559000in}{0.964164in}}%
\pgfpathlineto{\pgfqpoint{3.574500in}{1.232460in}}%
\pgfpathlineto{\pgfqpoint{3.590000in}{1.189372in}}%
\pgfpathlineto{\pgfqpoint{3.605500in}{0.938907in}}%
\pgfpathlineto{\pgfqpoint{3.621000in}{0.796580in}}%
\pgfpathlineto{\pgfqpoint{3.636500in}{1.272881in}}%
\pgfpathlineto{\pgfqpoint{3.652000in}{1.921086in}}%
\pgfpathlineto{\pgfqpoint{3.667500in}{1.055466in}}%
\pgfpathlineto{\pgfqpoint{3.683000in}{1.309079in}}%
\pgfpathlineto{\pgfqpoint{3.698500in}{1.297327in}}%
\pgfpathlineto{\pgfqpoint{3.714000in}{0.872414in}}%
\pgfpathlineto{\pgfqpoint{3.729500in}{1.574661in}}%
\pgfpathlineto{\pgfqpoint{3.745000in}{0.897021in}}%
\pgfpathlineto{\pgfqpoint{3.760500in}{1.331964in}}%
\pgfpathlineto{\pgfqpoint{3.776000in}{0.821658in}}%
\pgfpathlineto{\pgfqpoint{3.791500in}{1.877400in}}%
\pgfpathlineto{\pgfqpoint{3.807000in}{0.813091in}}%
\pgfpathlineto{\pgfqpoint{3.822500in}{1.089526in}}%
\pgfpathlineto{\pgfqpoint{3.838000in}{0.763051in}}%
\pgfpathlineto{\pgfqpoint{3.853500in}{1.317079in}}%
\pgfpathlineto{\pgfqpoint{3.869000in}{0.796124in}}%
\pgfpathlineto{\pgfqpoint{3.884500in}{1.500865in}}%
\pgfpathlineto{\pgfqpoint{3.900000in}{0.779715in}}%
\pgfpathlineto{\pgfqpoint{3.915500in}{1.122828in}}%
\pgfpathlineto{\pgfqpoint{3.931000in}{1.216145in}}%
\pgfpathlineto{\pgfqpoint{3.946500in}{1.173681in}}%
\pgfpathlineto{\pgfqpoint{3.962000in}{0.922205in}}%
\pgfpathlineto{\pgfqpoint{3.977500in}{1.372656in}}%
\pgfpathlineto{\pgfqpoint{3.993000in}{1.014446in}}%
\pgfpathlineto{\pgfqpoint{4.008500in}{0.889160in}}%
\pgfpathlineto{\pgfqpoint{4.024000in}{1.299005in}}%
\pgfpathlineto{\pgfqpoint{4.039500in}{0.946818in}}%
\pgfpathlineto{\pgfqpoint{4.055000in}{0.913412in}}%
\pgfpathlineto{\pgfqpoint{4.070500in}{0.939022in}}%
\pgfpathlineto{\pgfqpoint{4.086000in}{1.550271in}}%
\pgfpathlineto{\pgfqpoint{4.101500in}{1.074036in}}%
\pgfpathlineto{\pgfqpoint{4.117000in}{0.963668in}}%
\pgfpathlineto{\pgfqpoint{4.132500in}{1.557559in}}%
\pgfpathlineto{\pgfqpoint{4.148000in}{0.805203in}}%
\pgfpathlineto{\pgfqpoint{4.163500in}{1.038899in}}%
\pgfpathlineto{\pgfqpoint{4.179000in}{0.796579in}}%
\pgfpathlineto{\pgfqpoint{4.194500in}{1.398919in}}%
\pgfpathlineto{\pgfqpoint{4.210000in}{0.796693in}}%
\pgfpathlineto{\pgfqpoint{4.225500in}{1.030081in}}%
\pgfpathlineto{\pgfqpoint{4.241000in}{1.082176in}}%
\pgfpathlineto{\pgfqpoint{4.256500in}{1.113728in}}%
\pgfpathlineto{\pgfqpoint{4.272000in}{0.796862in}}%
\pgfpathlineto{\pgfqpoint{4.287500in}{1.657570in}}%
\pgfpathlineto{\pgfqpoint{4.303000in}{0.888640in}}%
\pgfpathlineto{\pgfqpoint{4.318500in}{1.039596in}}%
\pgfpathlineto{\pgfqpoint{4.334000in}{1.265255in}}%
\pgfpathlineto{\pgfqpoint{4.349500in}{1.568840in}}%
\pgfpathlineto{\pgfqpoint{4.365000in}{0.829513in}}%
\pgfpathlineto{\pgfqpoint{4.380500in}{1.039819in}}%
\pgfpathlineto{\pgfqpoint{4.396000in}{1.281998in}}%
\pgfpathlineto{\pgfqpoint{4.411500in}{1.023344in}}%
\pgfpathlineto{\pgfqpoint{4.427000in}{1.263795in}}%
\pgfpathlineto{\pgfqpoint{4.442500in}{1.291516in}}%
\pgfpathlineto{\pgfqpoint{4.458000in}{0.796494in}}%
\pgfpathlineto{\pgfqpoint{4.473500in}{0.956024in}}%
\pgfpathlineto{\pgfqpoint{4.489000in}{1.190334in}}%
\pgfpathlineto{\pgfqpoint{4.504500in}{1.139284in}}%
\pgfpathlineto{\pgfqpoint{4.520000in}{0.846950in}}%
\pgfpathlineto{\pgfqpoint{4.535500in}{1.055626in}}%
\pgfpathlineto{\pgfqpoint{4.551000in}{1.081066in}}%
\pgfpathlineto{\pgfqpoint{4.566500in}{1.022828in}}%
\pgfpathlineto{\pgfqpoint{4.582000in}{0.796352in}}%
\pgfpathlineto{\pgfqpoint{4.597500in}{0.955722in}}%
\pgfpathlineto{\pgfqpoint{4.613000in}{1.240016in}}%
\pgfpathlineto{\pgfqpoint{4.628500in}{1.005930in}}%
\pgfpathlineto{\pgfqpoint{4.644000in}{0.696000in}}%
\pgfpathlineto{\pgfqpoint{4.659500in}{1.264895in}}%
\pgfpathlineto{\pgfqpoint{4.675000in}{0.788116in}}%
\pgfpathlineto{\pgfqpoint{4.690500in}{1.266727in}}%
\pgfpathlineto{\pgfqpoint{4.706000in}{1.197626in}}%
\pgfpathlineto{\pgfqpoint{4.721500in}{1.065470in}}%
\pgfpathlineto{\pgfqpoint{4.737000in}{0.737803in}}%
\pgfpathlineto{\pgfqpoint{4.752500in}{1.290887in}}%
\pgfpathlineto{\pgfqpoint{4.768000in}{1.156530in}}%
\pgfpathlineto{\pgfqpoint{4.783500in}{1.482137in}}%
\pgfpathlineto{\pgfqpoint{4.799000in}{0.905954in}}%
\pgfpathlineto{\pgfqpoint{4.814500in}{0.913362in}}%
\pgfpathlineto{\pgfqpoint{4.830000in}{1.006135in}}%
\pgfpathlineto{\pgfqpoint{4.845500in}{1.843414in}}%
\pgfpathlineto{\pgfqpoint{4.861000in}{1.030831in}}%
\pgfpathlineto{\pgfqpoint{4.876500in}{1.139280in}}%
\pgfpathlineto{\pgfqpoint{4.892000in}{1.224652in}}%
\pgfpathlineto{\pgfqpoint{4.907500in}{1.650579in}}%
\pgfpathlineto{\pgfqpoint{4.923000in}{1.014229in}}%
\pgfpathlineto{\pgfqpoint{4.938500in}{1.165198in}}%
\pgfpathlineto{\pgfqpoint{4.954000in}{1.483325in}}%
\pgfpathlineto{\pgfqpoint{4.969500in}{1.030376in}}%
\pgfpathlineto{\pgfqpoint{4.985000in}{0.931086in}}%
\pgfpathlineto{\pgfqpoint{5.000500in}{0.997480in}}%
\pgfpathlineto{\pgfqpoint{5.016000in}{1.206207in}}%
\pgfpathlineto{\pgfqpoint{5.031500in}{1.240841in}}%
\pgfpathlineto{\pgfqpoint{5.047000in}{0.821692in}}%
\pgfpathlineto{\pgfqpoint{5.062500in}{1.390252in}}%
\pgfpathlineto{\pgfqpoint{5.078000in}{1.173578in}}%
\pgfpathlineto{\pgfqpoint{5.093500in}{1.157622in}}%
\pgfpathlineto{\pgfqpoint{5.109000in}{1.122178in}}%
\pgfpathlineto{\pgfqpoint{5.124500in}{1.157522in}}%
\pgfpathlineto{\pgfqpoint{5.140000in}{0.779765in}}%
\pgfpathlineto{\pgfqpoint{5.155500in}{1.064321in}}%
\pgfpathlineto{\pgfqpoint{5.171000in}{0.946758in}}%
\pgfpathlineto{\pgfqpoint{5.186500in}{1.443597in}}%
\pgfpathlineto{\pgfqpoint{5.202000in}{0.879861in}}%
\pgfpathlineto{\pgfqpoint{5.217500in}{1.064730in}}%
\pgfpathlineto{\pgfqpoint{5.233000in}{1.381726in}}%
\pgfpathlineto{\pgfqpoint{5.248500in}{4.056000in}}%
\pgfpathlineto{\pgfqpoint{5.264000in}{2.475572in}}%
\pgfpathlineto{\pgfqpoint{5.279500in}{2.470869in}}%
\pgfpathlineto{\pgfqpoint{5.295000in}{1.408178in}}%
\pgfpathlineto{\pgfqpoint{5.310500in}{2.388185in}}%
\pgfpathlineto{\pgfqpoint{5.326000in}{1.732583in}}%
\pgfpathlineto{\pgfqpoint{5.341500in}{1.500026in}}%
\pgfpathlineto{\pgfqpoint{5.357000in}{1.753920in}}%
\pgfpathlineto{\pgfqpoint{5.372500in}{1.289986in}}%
\pgfpathlineto{\pgfqpoint{5.388000in}{1.139081in}}%
\pgfpathlineto{\pgfqpoint{5.403500in}{1.056596in}}%
\pgfpathlineto{\pgfqpoint{5.419000in}{0.905498in}}%
\pgfpathlineto{\pgfqpoint{5.434500in}{1.173161in}}%
\pgfpathlineto{\pgfqpoint{5.450000in}{0.854894in}}%
\pgfpathlineto{\pgfqpoint{5.465500in}{1.173884in}}%
\pgfpathlineto{\pgfqpoint{5.481000in}{0.855457in}}%
\pgfpathlineto{\pgfqpoint{5.496500in}{1.223508in}}%
\pgfpathlineto{\pgfqpoint{5.512000in}{0.804721in}}%
\pgfpathlineto{\pgfqpoint{5.527500in}{1.500305in}}%
\pgfpathlineto{\pgfqpoint{5.543000in}{0.913651in}}%
\pgfpathlineto{\pgfqpoint{5.558500in}{2.386813in}}%
\pgfpathlineto{\pgfqpoint{5.574000in}{0.955800in}}%
\pgfpathlineto{\pgfqpoint{5.589500in}{1.216235in}}%
\pgfpathlineto{\pgfqpoint{5.605000in}{0.762790in}}%
\pgfpathlineto{\pgfqpoint{5.620500in}{0.947043in}}%
\pgfpathlineto{\pgfqpoint{5.636000in}{1.777179in}}%
\pgfpathlineto{\pgfqpoint{5.651500in}{1.122892in}}%
\pgfpathlineto{\pgfqpoint{5.667000in}{1.047842in}}%
\pgfpathlineto{\pgfqpoint{5.682500in}{1.499844in}}%
\pgfpathlineto{\pgfqpoint{5.698000in}{0.779677in}}%
\pgfpathlineto{\pgfqpoint{5.713500in}{0.922186in}}%
\pgfpathlineto{\pgfqpoint{5.729000in}{0.746385in}}%
\pgfpathlineto{\pgfqpoint{5.744500in}{1.064913in}}%
\pgfpathlineto{\pgfqpoint{5.760000in}{0.771369in}}%
\pgfpathlineto{\pgfqpoint{5.770000in}{0.938706in}}%
\pgfpathlineto{\pgfqpoint{5.770000in}{0.938706in}}%
\pgfusepath{stroke}%
\end{pgfscope}%
\begin{pgfscope}%
\pgfsetrectcap%
\pgfsetmiterjoin%
\pgfsetlinewidth{1.003750pt}%
\definecolor{currentstroke}{rgb}{1.000000,1.000000,1.000000}%
\pgfsetstrokecolor{currentstroke}%
\pgfsetdash{}{0pt}%
\pgfpathmoveto{\pgfqpoint{0.800000in}{0.528000in}}%
\pgfpathlineto{\pgfqpoint{0.800000in}{4.224000in}}%
\pgfusepath{stroke}%
\end{pgfscope}%
\begin{pgfscope}%
\pgfsetrectcap%
\pgfsetmiterjoin%
\pgfsetlinewidth{1.003750pt}%
\definecolor{currentstroke}{rgb}{1.000000,1.000000,1.000000}%
\pgfsetstrokecolor{currentstroke}%
\pgfsetdash{}{0pt}%
\pgfpathmoveto{\pgfqpoint{5.760000in}{0.528000in}}%
\pgfpathlineto{\pgfqpoint{5.760000in}{4.224000in}}%
\pgfusepath{stroke}%
\end{pgfscope}%
\begin{pgfscope}%
\pgfsetrectcap%
\pgfsetmiterjoin%
\pgfsetlinewidth{1.003750pt}%
\definecolor{currentstroke}{rgb}{1.000000,1.000000,1.000000}%
\pgfsetstrokecolor{currentstroke}%
\pgfsetdash{}{0pt}%
\pgfpathmoveto{\pgfqpoint{0.800000in}{0.528000in}}%
\pgfpathlineto{\pgfqpoint{5.760000in}{0.528000in}}%
\pgfusepath{stroke}%
\end{pgfscope}%
\begin{pgfscope}%
\pgfsetrectcap%
\pgfsetmiterjoin%
\pgfsetlinewidth{1.003750pt}%
\definecolor{currentstroke}{rgb}{1.000000,1.000000,1.000000}%
\pgfsetstrokecolor{currentstroke}%
\pgfsetdash{}{0pt}%
\pgfpathmoveto{\pgfqpoint{0.800000in}{4.224000in}}%
\pgfpathlineto{\pgfqpoint{5.760000in}{4.224000in}}%
\pgfusepath{stroke}%
\end{pgfscope}%
\begin{pgfscope}%
\pgfsetbuttcap%
\pgfsetmiterjoin%
\definecolor{currentfill}{rgb}{0.898039,0.898039,0.898039}%
\pgfsetfillcolor{currentfill}%
\pgfsetfillopacity{0.800000}%
\pgfsetlinewidth{0.501875pt}%
\definecolor{currentstroke}{rgb}{0.800000,0.800000,0.800000}%
\pgfsetstrokecolor{currentstroke}%
\pgfsetstrokeopacity{0.800000}%
\pgfsetdash{}{0pt}%
\pgfpathmoveto{\pgfqpoint{0.897222in}{3.908413in}}%
\pgfpathlineto{\pgfqpoint{3.616993in}{3.908413in}}%
\pgfpathquadraticcurveto{\pgfqpoint{3.644770in}{3.908413in}}{\pgfqpoint{3.644770in}{3.936191in}}%
\pgfpathlineto{\pgfqpoint{3.644770in}{4.126778in}}%
\pgfpathquadraticcurveto{\pgfqpoint{3.644770in}{4.154556in}}{\pgfqpoint{3.616993in}{4.154556in}}%
\pgfpathlineto{\pgfqpoint{0.897222in}{4.154556in}}%
\pgfpathquadraticcurveto{\pgfqpoint{0.869444in}{4.154556in}}{\pgfqpoint{0.869444in}{4.126778in}}%
\pgfpathlineto{\pgfqpoint{0.869444in}{3.936191in}}%
\pgfpathquadraticcurveto{\pgfqpoint{0.869444in}{3.908413in}}{\pgfqpoint{0.897222in}{3.908413in}}%
\pgfpathclose%
\pgfusepath{stroke,fill}%
\end{pgfscope}%
\begin{pgfscope}%
\pgfsetrectcap%
\pgfsetroundjoin%
\pgfsetlinewidth{1.505625pt}%
\definecolor{currentstroke}{rgb}{0.886275,0.290196,0.200000}%
\pgfsetstrokecolor{currentstroke}%
\pgfsetdash{}{0pt}%
\pgfpathmoveto{\pgfqpoint{0.925000in}{4.043444in}}%
\pgfpathlineto{\pgfqpoint{1.202778in}{4.043444in}}%
\pgfusepath{stroke}%
\end{pgfscope}%
\begin{pgfscope}%
\pgftext[x=1.313889in,y=3.994833in,left,base]{\sffamily\fontsize{10.000000}{12.000000}\selectfont Operaciones I/O en disco por segundo}%
\end{pgfscope}%
\end{pgfpicture}%
\makeatother%
\endgroup%

    \caption[I/O en disco durante simulación]{Lecturas y escrituras de disco por segundo durante una simulación con factor de demanda 100\%.}
    \label{fig:systemload:io}
\end{figure} 


\subsection{Simulación vehicular}

\begin{figure}
    \centering
    % This file was created by matplotlib2tikz v0.6.10.
\begin{tikzpicture}

\definecolor{color0}{rgb}{0.2,0.8,0.133333333333333}

\begin{axis}[
xlabel={Distancia Total $[m]$},
ylabel={Tiempo Total $[s]$},
xmin=-141.460057014178, xmax=2838.93005701418,
ymin=-47.6666843820862, ymax=944.166684382086,
width=\figurewidth,
height=\figureheight,
tick align=outside,
tick pos=left,
xticklabel style={rotate=45},
xmajorgrids,
x grid style={lightgray, opacity=0.7},
ymajorgrids,
y grid style={lightgray, opacity=0.7},
axis line style={black, opacity=0.0},
legend style={at={(0.97,0.03)}, anchor=south east, draw=white!80.0!black, fill=white!89.803921568627459!black},
legend cell align={left},
legend entries={{Con comunicaci\'on},{Sin comunicaci\'on}}
]
\addplot [only marks, draw=red, mark size=1.0, fill=red, opacity=0.75, colormap/viridis]
table{%
x                      y
+2.789250000000000e+02 +2.100000000000000e+01
+2.751700000000000e+02 +2.050000000000000e+01
+2.827610000000000e+02 +2.050000000000000e+01
+3.244580000000000e+02 +2.250000000000000e+01
+2.927260000000000e+02 +3.550000000000000e+01
+3.345040000000000e+02 +2.450000000000000e+01
+2.774470000000000e+02 +2.100000000000000e+01
+2.799990000000000e+02 +2.100000000000000e+01
+4.920510000000000e+02 +2.950000000000000e+01
+7.761550000000000e+02 +4.700000000000000e+01
+7.702950000000000e+02 +4.750000000000000e+01
+7.883260000000000e+02 +4.800000000000000e+01
+6.929910000000001e+02 +5.200000000000000e+01
+6.937250000000000e+02 +5.050000000000000e+01
+7.652639999999999e+02 +4.650000000000000e+01
+7.750030000000000e+02 +4.700000000000000e+01
+6.769030000000000e+02 +4.500000000000000e+01
+3.374630000000000e+02 +2.450000000000000e+01
+4.942410000000000e+02 +3.850000000000000e+01
+7.652170000000000e+02 +4.550000000000000e+01
+7.680560000000000e+02 +4.750000000000000e+01
+7.729119999999998e+02 +4.650000000000000e+01
+4.536680000000000e+02 +3.350000000000000e+01
+7.615050000000000e+02 +5.450000000000000e+01
+7.695260000000002e+02 +4.750000000000000e+01
+7.503439999999998e+02 +4.550000000000000e+01
+7.516039999999998e+02 +5.500000000000000e+01
+7.820750000000000e+02 +4.850000000000000e+01
+4.916920000000000e+02 +7.000000000000000e+01
+7.782580000000000e+02 +4.700000000000000e+01
+8.424169999999998e+02 +6.050000000000000e+01
+8.793720000000000e+02 +6.250000000000000e+01
+1.083850000000000e+02 +1.250000000000000e+01
+9.149010000000000e+02 +6.750000000000000e+01
+4.859460000000000e+02 +5.950000000000000e+01
+8.701080000000002e+02 +6.150000000000000e+01
+8.813600000000000e+02 +6.400000000000000e+01
+4.922770000000000e+02 +6.350000000000000e+01
+7.707869999999998e+02 +4.550000000000000e+01
+7.743650000000000e+02 +6.100000000000000e+01
+4.937380000000001e+02 +5.650000000000000e+01
+8.909920000000000e+02 +6.400000000000000e+01
+4.910120000000000e+02 +4.550000000000000e+01
+8.859299999999999e+02 +6.600000000000000e+01
+8.765510000000000e+02 +6.650000000000000e+01
+3.368600000000000e+02 +2.400000000000000e+01
+8.571039999999998e+02 +8.350000000000000e+01
+7.774040000000000e+02 +6.800000000000000e+01
+6.599019999999998e+02 +8.450000000000000e+01
+6.250269999999998e+02 +7.200000000000000e+01
+4.870110000000000e+02 +4.500000000000000e+01
+8.392420000000000e+02 +6.100000000000000e+01
+8.554490000000000e+02 +6.700000000000000e+01
+6.028950000000000e+02 +7.250000000000000e+01
+6.405459999999998e+02 +7.850000000000000e+01
+6.657650000000000e+02 +7.450000000000000e+01
+7.727520000000000e+02 +5.550000000000000e+01
+9.011180000000001e+02 +6.300000000000000e+01
+8.678980000000000e+02 +8.750000000000000e+01
+6.511390000000000e+02 +6.950000000000000e+01
+8.638570000000000e+02 +6.100000000000000e+01
+8.826810000000000e+02 +6.450000000000000e+01
+6.699600000000000e+02 +6.950000000000000e+01
+8.406660000000001e+02 +6.000000000000000e+01
+4.980300000000000e+02 +4.000000000000000e+01
+6.808989999999999e+02 +7.850000000000000e+01
+8.693700000000000e+02 +6.150000000000000e+01
+6.372060000000000e+02 +6.200000000000000e+01
+6.556920000000000e+02 +5.700000000000000e+01
+9.252900000000000e+02 +6.500000000000000e+01
+6.403869999999999e+02 +6.000000000000000e+01
+6.367410000000000e+02 +6.100000000000000e+01
+7.528860000000002e+02 +8.600000000000000e+01
+6.645080000000000e+02 +5.750000000000000e+01
+1.078750000000000e+02 +2.000000000000000e+01
+6.794160000000001e+02 +7.750000000000000e+01
+9.343110000000000e+02 +6.400000000000000e+01
+6.427090000000002e+02 +6.300000000000000e+01
+6.051230000000000e+02 +5.050000000000000e+01
+4.959750000000000e+02 +3.900000000000000e+01
+6.612380000000001e+02 +5.300000000000000e+01
+9.510910000000000e+02 +7.050000000000000e+01
+8.795350000000000e+02 +5.950000000000000e+01
+6.311300000000000e+02 +5.700000000000000e+01
+6.514410000000000e+02 +5.200000000000000e+01
+7.530640000000000e+02 +7.700000000000000e+01
+1.082450000000000e+03 +1.000000000000000e+02
+9.943860000000000e+02 +9.300000000000000e+01
+9.092920000000000e+02 +8.900000000000000e+01
+6.336230000000000e+02 +5.050000000000000e+01
+9.018980000000000e+02 +9.700000000000000e+01
+8.776870000000000e+02 +6.150000000000000e+01
+6.679220000000000e+02 +8.200000000000000e+01
+6.742680000000000e+02 +8.950000000000000e+01
+1.005390000000000e+03 +1.040000000000000e+02
+1.004130000000000e+03 +8.150000000000000e+01
+7.538180000000000e+02 +7.450000000000000e+01
+7.544160000000001e+02 +6.250000000000000e+01
+1.153810000000000e+02 +1.300000000000000e+01
+6.746410000000002e+02 +7.900000000000000e+01
+1.017310000000000e+03 +7.250000000000000e+01
+1.061130000000000e+03 +9.450000000000000e+01
+6.876990000000000e+02 +7.450000000000000e+01
+1.059400000000000e+03 +9.800000000000000e+01
+9.873360000000000e+02 +7.200000000000000e+01
+6.854330000000000e+02 +7.350000000000000e+01
+1.076600000000000e+03 +1.085000000000000e+02
+1.061180000000000e+03 +9.500000000000000e+01
+1.082910000000000e+03 +1.000000000000000e+02
+1.083180000000000e+03 +9.650000000000000e+01
+6.779220000000000e+02 +6.200000000000000e+01
+1.035650000000000e+03 +1.225000000000000e+02
+1.045950000000000e+03 +1.065000000000000e+02
+1.113520000000000e+03 +1.205000000000000e+02
+1.120120000000000e+03 +1.115000000000000e+02
+9.784960000000000e+02 +1.245000000000000e+02
+1.033020000000000e+03 +1.120000000000000e+02
+2.722100000000000e+02 +5.900000000000000e+01
+9.970080000000000e+02 +1.050000000000000e+02
+1.168360000000000e+02 +1.100000000000000e+01
+9.931680000000000e+02 +1.275000000000000e+02
+9.726490000000000e+02 +1.200000000000000e+02
+1.000600000000000e+03 +1.280000000000000e+02
+1.007110000000000e+03 +1.220000000000000e+02
+1.003110000000000e+03 +1.115000000000000e+02
+2.741560000000000e+02 +6.150000000000000e+01
+1.008090000000000e+03 +1.265000000000000e+02
+1.029620000000000e+03 +1.130000000000000e+02
+1.131450000000000e+03 +8.650000000000000e+01
+9.890240000000000e+02 +1.055000000000000e+02
+9.837650000000000e+02 +1.260000000000000e+02
+1.055960000000000e+03 +1.100000000000000e+02
+1.195210000000000e+03 +8.400000000000000e+01
+2.818490000000000e+02 +9.350000000000000e+01
+1.126080000000000e+03 +1.135000000000000e+02
+1.281020000000000e+03 +1.220000000000000e+02
+1.361980000000000e+03 +1.330000000000000e+02
+2.819530000000000e+02 +8.600000000000000e+01
+2.771970000000000e+02 +9.250000000000000e+01
+9.760050000000000e+02 +1.165000000000000e+02
+9.976070000000000e+02 +1.135000000000000e+02
+6.616760000000000e+02 +4.550000000000000e+01
+6.867030000000000e+02 +1.130000000000000e+02
+9.034400000000001e+02 +1.220000000000000e+02
+5.886840000000000e+02 +9.150000000000000e+01
+2.761120000000000e+02 +8.500000000000000e+01
+9.929140000000000e+02 +1.130000000000000e+02
+9.569390000000000e+02 +1.030000000000000e+02
+4.580520000000000e+02 +6.650000000000000e+01
+1.333760000000000e+03 +1.305000000000000e+02
+2.769890000000000e+02 +8.200000000000000e+01
+2.810300000000000e+02 +6.700000000000000e+01
+9.172310000000000e+02 +1.215000000000000e+02
+6.220290000000000e+02 +1.320000000000000e+02
+6.205850000000000e+02 +1.335000000000000e+02
+8.542030000000000e+02 +1.225000000000000e+02
+6.209950000000000e+02 +1.235000000000000e+02
+2.842980000000000e+02 +7.600000000000000e+01
+9.735540000000000e+02 +9.850000000000000e+01
+6.226500000000000e+02 +9.850000000000000e+01
+6.613760000000002e+02 +8.400000000000000e+01
+9.097670000000001e+02 +1.230000000000000e+02
+1.056970000000000e+03 +1.150000000000000e+02
+6.668960000000002e+02 +1.260000000000000e+02
+1.037340000000000e+03 +1.320000000000000e+02
+9.808030000000000e+02 +1.030000000000000e+02
+8.797869999999998e+02 +9.300000000000000e+01
+8.846790000000000e+02 +9.550000000000000e+01
+1.063840000000000e+03 +1.405000000000000e+02
+2.834900000000000e+02 +5.350000000000000e+01
+9.209600000000000e+02 +1.180000000000000e+02
+6.252730000000000e+02 +5.950000000000000e+01
+6.245790000000002e+02 +6.800000000000000e+01
+6.606139999999998e+02 +5.400000000000000e+01
+1.077230000000000e+03 +1.330000000000000e+02
+9.745210000000000e+02 +9.950000000000000e+01
+9.109400000000001e+02 +1.095000000000000e+02
+6.214370000000000e+02 +5.050000000000000e+01
+1.018190000000000e+03 +1.095000000000000e+02
+1.056790000000000e+03 +9.000000000000000e+01
+6.224390000000000e+02 +4.950000000000000e+01
+8.732840000000000e+02 +1.080000000000000e+02
+2.700270000000000e+02 +5.100000000000000e+01
+6.062370000000000e+02 +4.900000000000000e+01
+1.112550000000000e+03 +1.225000000000000e+02
+1.230000000000000e+03 +1.275000000000000e+02
+9.072340000000000e+02 +8.750000000000000e+01
+1.107910000000000e+03 +1.155000000000000e+02
+2.991810000000000e+02 +5.050000000000000e+01
+9.125350000000000e+02 +8.950000000000000e+01
+1.115000000000000e+03 +1.025000000000000e+02
+1.084100000000000e+03 +1.050000000000000e+02
+6.327730000000000e+02 +5.300000000000000e+01
+1.063320000000000e+03 +1.160000000000000e+02
+2.940270000000000e+02 +5.500000000000000e+01
+1.309740000000000e+03 +1.325000000000000e+02
+9.093339999999999e+02 +8.150000000000000e+01
+1.022410000000000e+03 +1.090000000000000e+02
+7.862030000000000e+02 +1.040000000000000e+02
+1.289490000000000e+03 +1.335000000000000e+02
+9.998480000000000e+02 +1.440000000000000e+02
+1.405180000000000e+03 +1.480000000000000e+02
+7.652769999999998e+02 +1.160000000000000e+02
+6.256669999999998e+02 +4.750000000000000e+01
+9.064010000000000e+02 +9.600000000000000e+01
+8.499360000000000e+02 +9.300000000000000e+01
+8.389560000000000e+02 +1.205000000000000e+02
+9.213240000000000e+02 +1.510000000000000e+02
+7.681710000000000e+02 +1.155000000000000e+02
+7.684710000000000e+02 +1.075000000000000e+02
+8.658339999999999e+02 +9.350000000000000e+01
+8.795910000000000e+02 +8.600000000000000e+01
+9.142070000000000e+02 +1.420000000000000e+02
+1.321400000000000e+03 +1.325000000000000e+02
+7.450100000000000e+02 +1.000000000000000e+02
+4.463610000000000e+02 +8.450000000000000e+01
+7.751060000000001e+02 +1.115000000000000e+02
+1.306300000000000e+03 +1.320000000000000e+02
+7.745960000000000e+02 +1.175000000000000e+02
+7.673850000000000e+02 +1.140000000000000e+02
+8.772430000000001e+02 +8.800000000000000e+01
+8.624530000000000e+02 +1.240000000000000e+02
+3.224100000000000e+02 +8.550000000000000e+01
+8.868789999999998e+02 +1.290000000000000e+02
+7.563610000000001e+02 +1.080000000000000e+02
+7.664480000000000e+02 +8.050000000000000e+01
+1.411260000000000e+03 +1.445000000000000e+02
+7.724630000000002e+02 +1.090000000000000e+02
+1.316050000000000e+03 +1.510000000000000e+02
+7.644370000000000e+02 +1.075000000000000e+02
+8.930740000000000e+02 +8.800000000000000e+01
+6.927239999999998e+02 +1.450000000000000e+02
+3.251410000000000e+02 +8.550000000000000e+01
+6.881130000000001e+02 +1.335000000000000e+02
+8.044800000000000e+02 +1.025000000000000e+02
+7.773270000000000e+02 +1.075000000000000e+02
+7.713680000000001e+02 +9.450000000000000e+01
+8.394560000000000e+02 +8.050000000000000e+01
+7.708620000000000e+02 +1.050000000000000e+02
+3.332760000000000e+02 +6.800000000000000e+01
+6.939380000000000e+02 +1.295000000000000e+02
+7.563689999999998e+02 +1.025000000000000e+02
+7.790549999999999e+02 +1.075000000000000e+02
+8.559760000000001e+02 +8.450000000000000e+01
+8.614190000000000e+02 +1.060000000000000e+02
+7.652080000000002e+02 +1.015000000000000e+02
+9.129770000000000e+02 +1.285000000000000e+02
+9.006480000000000e+02 +8.850000000000000e+01
+8.081730000000000e+02 +9.950000000000000e+01
+7.810250000000000e+02 +1.065000000000000e+02
+6.617710000000002e+02 +1.005000000000000e+02
+7.691210000000002e+02 +1.005000000000000e+02
+1.296930000000000e+03 +1.395000000000000e+02
+4.800760000000000e+02 +8.400000000000000e+01
+7.874839999999998e+02 +1.035000000000000e+02
+7.845210000000002e+02 +9.800000000000000e+01
+8.853380000000002e+02 +8.650000000000000e+01
+8.622980000000000e+02 +1.025000000000000e+02
+6.925230000000000e+02 +1.090000000000000e+02
+7.458620000000000e+02 +1.285000000000000e+02
+7.669240000000000e+02 +8.700000000000000e+01
+8.965960000000000e+02 +8.600000000000000e+01
+4.940560000000000e+02 +5.400000000000000e+01
+6.918860000000002e+02 +1.360000000000000e+02
+9.034990000000000e+02 +1.425000000000000e+02
+6.616050000000000e+02 +9.600000000000000e+01
+7.519169999999998e+02 +9.900000000000000e+01
+7.669720000000000e+02 +9.600000000000000e+01
+4.864470000000000e+02 +6.450000000000000e+01
+1.300960000000000e+03 +1.295000000000000e+02
+7.717700000000000e+02 +8.150000000000000e+01
+7.833930000000000e+02 +9.800000000000000e+01
+1.294660000000000e+03 +1.210000000000000e+02
+7.753550000000000e+02 +8.850000000000000e+01
+6.971100000000000e+02 +1.100000000000000e+02
+9.478620000000000e+02 +1.180000000000000e+02
+4.756930000000000e+02 +3.150000000000000e+01
+7.477020000000000e+02 +1.170000000000000e+02
+7.798380000000002e+02 +8.200000000000000e+01
+6.948639999999998e+02 +7.850000000000000e+01
+1.091750000000000e+03 +1.595000000000000e+02
+9.005039999999998e+02 +8.950000000000000e+01
+9.850130000000000e+02 +1.365000000000000e+02
+7.895139999999999e+02 +8.450000000000000e+01
+7.835410000000001e+02 +7.900000000000000e+01
+7.799080000000000e+02 +8.150000000000000e+01
+1.420550000000000e+03 +1.655000000000000e+02
+6.949750000000000e+02 +1.190000000000000e+02
+4.824170000000000e+02 +3.150000000000000e+01
+9.754500000000000e+02 +1.180000000000000e+02
+8.630560000000000e+02 +1.185000000000000e+02
+6.977500000000000e+02 +1.060000000000000e+02
+1.094000000000000e+03 +1.365000000000000e+02
+1.315300000000000e+03 +1.250000000000000e+02
+9.775520000000000e+02 +1.180000000000000e+02
+6.752310000000001e+02 +1.200000000000000e+02
+1.316370000000000e+03 +1.685000000000000e+02
+7.852170000000000e+02 +8.150000000000000e+01
+9.388960000000000e+02 +1.485000000000000e+02
+7.564780000000002e+02 +9.000000000000000e+01
+6.896230000000000e+02 +1.255000000000000e+02
+7.690700000000001e+02 +7.350000000000000e+01
+9.066620000000000e+02 +8.200000000000000e+01
+3.338320000000000e+02 +2.550000000000000e+01
+1.297830000000000e+03 +1.580000000000000e+02
+4.852120000000000e+02 +3.300000000000000e+01
+6.801230000000000e+02 +9.850000000000000e+01
+1.128730000000000e+03 +1.695000000000000e+02
+7.831380000000000e+02 +7.300000000000000e+01
+1.295870000000000e+03 +1.545000000000000e+02
+7.885830000000002e+02 +8.650000000000000e+01
+6.939789999999998e+02 +7.300000000000000e+01
+7.531920000000000e+02 +9.550000000000000e+01
+8.991519999999998e+02 +7.750000000000000e+01
+7.651389999999999e+02 +7.700000000000000e+01
+8.432230000000002e+02 +8.700000000000000e+01
+9.775950000000000e+02 +1.145000000000000e+02
+9.783480000000000e+02 +1.100000000000000e+02
+1.125690000000000e+03 +1.545000000000000e+02
+7.657520000000000e+02 +6.850000000000000e+01
+5.756210000000000e+02 +7.000000000000000e+01
+1.294110000000000e+03 +1.385000000000000e+02
+6.909390000000000e+02 +6.950000000000000e+01
+7.753450000000000e+02 +6.750000000000000e+01
+7.666170000000000e+02 +1.510000000000000e+02
+9.897880000000000e+02 +1.135000000000000e+02
+7.487020000000000e+02 +8.500000000000000e+01
+7.534100000000000e+02 +8.450000000000000e+01
+8.732050000000000e+02 +9.150000000000000e+01
+7.816569999999998e+02 +7.750000000000000e+01
+7.552239999999998e+02 +1.110000000000000e+02
+7.837439999999998e+02 +6.800000000000000e+01
+7.795350000000000e+02 +7.650000000000000e+01
+8.685360000000002e+02 +1.110000000000000e+02
+6.941239999999998e+02 +8.750000000000000e+01
+9.080270000000000e+02 +7.800000000000000e+01
+9.458560000000000e+02 +7.900000000000000e+01
+7.722239999999998e+02 +7.000000000000000e+01
+7.661880000000000e+02 +6.500000000000000e+01
+7.677070000000000e+02 +6.200000000000000e+01
+7.816020000000000e+02 +6.950000000000000e+01
+4.907540000000000e+02 +4.050000000000000e+01
+1.140290000000000e+03 +1.595000000000000e+02
+6.944530000000000e+02 +1.110000000000000e+02
+7.827160000000000e+02 +6.500000000000000e+01
+6.811880000000000e+02 +7.700000000000000e+01
+3.349040000000000e+02 +2.550000000000000e+01
+4.898910000000000e+02 +1.135000000000000e+02
+9.351220000000000e+02 +8.900000000000000e+01
+7.485770000000000e+02 +1.560000000000000e+02
+7.472550000000000e+02 +7.900000000000000e+01
+7.795490000000000e+02 +6.200000000000000e+01
+7.060740000000000e+02 +9.200000000000000e+01
+1.149660000000000e+03 +1.510000000000000e+02
+6.873520000000000e+02 +1.645000000000000e+02
+6.595980000000002e+02 +5.550000000000000e+01
+7.471100000000000e+02 +6.300000000000000e+01
+4.822420000000000e+02 +1.060000000000000e+02
+7.703670000000000e+02 +6.150000000000000e+01
+7.756389999999999e+02 +6.150000000000000e+01
+7.798680000000001e+02 +6.000000000000000e+01
+6.798819999999999e+02 +1.545000000000000e+02
+4.912730000000000e+02 +9.350000000000000e+01
+7.508360000000000e+02 +6.100000000000000e+01
+7.780069999999999e+02 +5.750000000000000e+01
+8.938839999999999e+02 +1.495000000000000e+02
+1.312460000000000e+03 +1.290000000000000e+02
+7.683090000000000e+02 +5.600000000000000e+01
+7.447130000000002e+02 +5.350000000000000e+01
+4.895660000000000e+02 +1.005000000000000e+02
+7.764430000000000e+02 +5.600000000000000e+01
+7.762719999999998e+02 +5.350000000000000e+01
+7.590330000000000e+02 +1.475000000000000e+02
+6.850150000000000e+02 +1.555000000000000e+02
+1.121130000000000e+02 +1.400000000000000e+01
+4.859700000000000e+02 +8.250000000000000e+01
+6.930480000000000e+02 +1.315000000000000e+02
+7.541100000000000e+02 +1.565000000000000e+02
+8.807900000000000e+02 +7.450000000000000e+01
+4.896330000000000e+02 +8.600000000000000e+01
+6.823560000000001e+02 +1.530000000000000e+02
+7.845030000000000e+02 +5.900000000000000e+01
+7.706220000000000e+02 +4.800000000000000e+01
+8.994750000000000e+02 +7.750000000000000e+01
+4.889190000000000e+02 +7.500000000000000e+01
+7.595950000000000e+02 +1.405000000000000e+02
+6.313009999999998e+02 +9.000000000000000e+01
+9.701990000000000e+02 +1.115000000000000e+02
+1.106650000000000e+03 +1.430000000000000e+02
+6.806880000000000e+02 +1.560000000000000e+02
+7.629580000000002e+02 +4.550000000000000e+01
+7.804639999999998e+02 +1.020000000000000e+02
+7.545500000000000e+02 +7.100000000000000e+01
+6.429530000000000e+02 +1.845000000000000e+02
+4.864290000000000e+02 +7.800000000000000e+01
+7.728770000000000e+02 +4.800000000000000e+01
+7.553610000000001e+02 +7.550000000000000e+01
+8.582970000000000e+02 +6.900000000000000e+01
+4.880160000000000e+02 +7.250000000000000e+01
+6.789600000000000e+02 +1.440000000000000e+02
+4.915810000000000e+02 +6.550000000000000e+01
+1.393260000000000e+03 +1.865000000000000e+02
+1.349740000000000e+03 +1.880000000000000e+02
+1.096110000000000e+03 +1.510000000000000e+02
+6.744980000000000e+02 +1.040000000000000e+02
+1.561150000000000e+03 +1.920000000000000e+02
+1.718540000000000e+03 +1.925000000000000e+02
+4.856900000000000e+02 +5.100000000000000e+01
+1.289110000000000e+03 +1.795000000000000e+02
+6.802900000000000e+02 +9.700000000000000e+01
+1.547940000000000e+03 +1.805000000000000e+02
+1.322320000000000e+03 +1.850000000000000e+02
+1.100150000000000e+03 +1.710000000000000e+02
+4.919410000000000e+02 +6.950000000000000e+01
+1.555860000000000e+03 +1.980000000000000e+02
+1.552270000000000e+03 +1.555000000000000e+02
+7.506139999999998e+02 +9.350000000000000e+01
+6.856400000000000e+02 +1.345000000000000e+02
+6.927280000000002e+02 +1.310000000000000e+02
+6.739550000000000e+02 +1.430000000000000e+02
+9.893690000000000e+02 +1.585000000000000e+02
+8.915089999999999e+02 +1.550000000000000e+02
+8.934250000000000e+02 +1.275000000000000e+02
+1.569290000000000e+03 +1.770000000000000e+02
+7.611139999999998e+02 +7.050000000000000e+01
+6.763070000000000e+02 +1.375000000000000e+02
+6.659150000000000e+02 +1.325000000000000e+02
+6.353040000000000e+02 +1.525000000000000e+02
+5.027070000000000e+02 +5.450000000000000e+01
+6.793869999999999e+02 +8.600000000000000e+01
+9.884400000000001e+02 +1.520000000000000e+02
+1.574100000000000e+03 +1.935000000000000e+02
+6.727919999999998e+02 +1.195000000000000e+02
+4.952050000000000e+02 +5.900000000000000e+01
+9.300359999999999e+02 +1.560000000000000e+02
+6.550770000000000e+02 +1.290000000000000e+02
+6.608510000000001e+02 +1.385000000000000e+02
+6.333490000000000e+02 +1.545000000000000e+02
+6.976860000000000e+02 +1.210000000000000e+02
+1.150560000000000e+03 +1.745000000000000e+02
+6.190930000000002e+02 +1.270000000000000e+02
+9.334200000000000e+02 +1.140000000000000e+02
+6.528960000000000e+02 +1.775000000000000e+02
+6.366450000000000e+02 +1.520000000000000e+02
+6.338330000000002e+02 +1.480000000000000e+02
+6.671830000000000e+02 +1.285000000000000e+02
+1.859400000000000e+03 +1.930000000000000e+02
+1.036370000000000e+03 +1.625000000000000e+02
+6.649370000000000e+02 +1.760000000000000e+02
+1.058910000000000e+03 +1.765000000000000e+02
+1.108120000000000e+03 +1.755000000000000e+02
+1.148060000000000e+03 +1.400000000000000e+02
+6.473910000000000e+02 +1.510000000000000e+02
+6.571550000000000e+02 +1.495000000000000e+02
+6.482530000000000e+02 +1.190000000000000e+02
+6.193060000000000e+02 +1.045000000000000e+02
+6.844290000000000e+02 +1.550000000000000e+02
+6.203460000000000e+02 +1.380000000000000e+02
+1.084310000000000e+02 +2.000000000000000e+01
+6.173540000000000e+02 +1.490000000000000e+02
+6.521060000000000e+02 +1.015000000000000e+02
+1.440040000000000e+03 +2.040000000000000e+02
+8.974910000000001e+02 +8.550000000000000e+01
+6.396130000000001e+02 +1.460000000000000e+02
+6.502170000000000e+02 +1.440000000000000e+02
+6.743570000000000e+02 +1.015000000000000e+02
+6.417900000000000e+02 +1.170000000000000e+02
+1.475220000000000e+03 +2.010000000000000e+02
+6.868360000000000e+02 +1.350000000000000e+02
+8.786260000000002e+02 +9.500000000000000e+01
+6.542210000000000e+02 +1.300000000000000e+02
+6.624560000000000e+02 +1.470000000000000e+02
+1.038500000000000e+03 +1.505000000000000e+02
+6.534620000000000e+02 +7.000000000000000e+01
+8.966810000000000e+02 +7.700000000000000e+01
+1.523430000000000e+03 +2.120000000000000e+02
+1.100330000000000e+02 +2.500000000000000e+01
+6.756139999999998e+02 +1.390000000000000e+02
+6.459910000000000e+02 +1.430000000000000e+02
+6.639230000000000e+02 +1.325000000000000e+02
+1.033870000000000e+03 +1.360000000000000e+02
+1.518630000000000e+03 +1.960000000000000e+02
+1.365120000000000e+03 +2.040000000000000e+02
+1.158710000000000e+03 +1.845000000000000e+02
+6.369340000000000e+02 +5.800000000000000e+01
+6.154080000000000e+02 +1.180000000000000e+02
+1.544140000000000e+03 +2.160000000000000e+02
+1.045370000000000e+03 +1.525000000000000e+02
+1.486080000000000e+03 +1.985000000000000e+02
+6.714340000000000e+02 +1.420000000000000e+02
+2.601390000000000e+02 +1.020000000000000e+02
+1.375450000000000e+03 +1.960000000000000e+02
+6.573819999999999e+02 +1.150000000000000e+02
+6.476970000000000e+02 +1.000000000000000e+02
+6.882130000000002e+02 +9.300000000000000e+01
+6.612880000000000e+02 +1.415000000000000e+02
+6.596780000000000e+02 +6.950000000000000e+01
+6.060920000000000e+02 +1.330000000000000e+02
+6.357480000000000e+02 +9.500000000000000e+01
+1.008110000000000e+03 +1.715000000000000e+02
+1.134130000000000e+03 +1.945000000000000e+02
+6.385090000000000e+02 +6.000000000000000e+01
+9.058440000000001e+02 +6.950000000000000e+01
+1.005690000000000e+03 +1.670000000000000e+02
+6.405400000000000e+02 +9.450000000000000e+01
+6.515710000000000e+02 +1.380000000000000e+02
+8.372339999999998e+02 +1.020000000000000e+02
+1.001170000000000e+03 +1.595000000000000e+02
+1.155440000000000e+03 +2.105000000000000e+02
+6.576519999999998e+02 +8.300000000000000e+01
+1.069470000000000e+03 +1.655000000000000e+02
+6.827530000000000e+02 +1.285000000000000e+02
+6.561770000000000e+02 +1.125000000000000e+02
+8.963389999999998e+02 +1.125000000000000e+02
+1.072620000000000e+03 +1.890000000000000e+02
+6.565730000000000e+02 +8.500000000000000e+01
+1.112390000000000e+03 +1.605000000000000e+02
+1.251470000000000e+03 +2.085000000000000e+02
+6.629850000000000e+02 +1.385000000000000e+02
+8.910110000000002e+02 +1.110000000000000e+02
+6.344240000000000e+02 +9.100000000000000e+01
+9.969480000000000e+02 +1.230000000000000e+02
+1.271830000000000e+03 +1.730000000000000e+02
+1.322910000000000e+03 +1.455000000000000e+02
+6.832600000000000e+02 +1.170000000000000e+02
+1.078230000000000e+03 +1.715000000000000e+02
+2.738260000000000e+02 +8.950000000000000e+01
+1.083360000000000e+03 +1.465000000000000e+02
+6.581930000000000e+02 +1.360000000000000e+02
+6.451080000000002e+02 +9.450000000000000e+01
+9.929980000000000e+02 +1.365000000000000e+02
+1.050970000000000e+03 +2.080000000000000e+02
+1.074230000000000e+03 +2.045000000000000e+02
+9.059829999999999e+02 +1.115000000000000e+02
+1.065400000000000e+03 +1.700000000000000e+02
+1.341560000000000e+03 +1.365000000000000e+02
+6.730880000000002e+02 +1.310000000000000e+02
+9.006189999999998e+02 +1.090000000000000e+02
+1.069040000000000e+03 +1.955000000000000e+02
+6.357940000000000e+02 +1.235000000000000e+02
+1.116720000000000e+03 +1.660000000000000e+02
+1.242070000000000e+03 +1.460000000000000e+02
+6.367150000000000e+02 +1.185000000000000e+02
+1.101900000000000e+03 +1.525000000000000e+02
+9.894230000000000e+02 +1.825000000000000e+02
+1.571790000000000e+03 +1.750000000000000e+02
+1.083660000000000e+03 +2.085000000000000e+02
+1.079160000000000e+03 +1.940000000000000e+02
+1.066100000000000e+03 +2.115000000000000e+02
+1.112200000000000e+03 +1.740000000000000e+02
+1.063510000000000e+03 +1.315000000000000e+02
+1.644840000000000e+03 +1.650000000000000e+02
+1.003520000000000e+03 +1.725000000000000e+02
+1.052090000000000e+03 +1.690000000000000e+02
+6.378640000000000e+02 +8.150000000000000e+01
+1.042870000000000e+03 +1.345000000000000e+02
+1.073070000000000e+03 +1.820000000000000e+02
+1.058260000000000e+03 +1.830000000000000e+02
+6.407130000000002e+02 +1.130000000000000e+02
+1.106910000000000e+03 +1.495000000000000e+02
+2.711850000000000e+02 +8.250000000000000e+01
+6.803190000000000e+02 +1.195000000000000e+02
+1.078600000000000e+03 +1.415000000000000e+02
+1.080370000000000e+03 +1.540000000000000e+02
+6.389150000000000e+02 +9.200000000000000e+01
+9.017670000000001e+02 +1.130000000000000e+02
+1.069270000000000e+03 +1.230000000000000e+02
+1.045150000000000e+03 +1.255000000000000e+02
+1.568560000000000e+03 +1.735000000000000e+02
+1.065480000000000e+03 +2.040000000000000e+02
+1.154520000000000e+03 +1.930000000000000e+02
+1.363140000000000e+03 +2.070000000000000e+02
+6.565230000000000e+02 +1.390000000000000e+02
+1.114770000000000e+03 +1.100000000000000e+02
+6.774040000000000e+02 +1.435000000000000e+02
+1.051410000000000e+03 +1.705000000000000e+02
+9.779750000000000e+02 +1.670000000000000e+02
+9.103060000000000e+02 +1.135000000000000e+02
+1.074870000000000e+03 +1.730000000000000e+02
+9.754450000000001e+02 +1.650000000000000e+02
+6.720069999999999e+02 +9.800000000000000e+01
+6.569730000000002e+02 +1.190000000000000e+02
+8.363240000000000e+02 +9.000000000000000e+01
+1.066510000000000e+03 +1.430000000000000e+02
+1.334640000000000e+03 +2.250000000000000e+02
+1.008140000000000e+03 +1.620000000000000e+02
+8.813420000000000e+02 +1.050000000000000e+02
+1.070020000000000e+03 +1.975000000000000e+02
+1.035320000000000e+03 +1.585000000000000e+02
+1.215990000000000e+03 +1.800000000000000e+02
+9.090460000000000e+02 +1.145000000000000e+02
+8.630369999999998e+02 +1.070000000000000e+02
+1.781850000000000e+03 +2.345000000000000e+02
+1.048800000000000e+03 +1.450000000000000e+02
+1.080970000000000e+03 +1.525000000000000e+02
+1.303840000000000e+03 +2.325000000000000e+02
+1.215670000000000e+03 +2.090000000000000e+02
+1.367530000000000e+03 +2.175000000000000e+02
+8.622930000000000e+02 +1.830000000000000e+02
+8.705910000000000e+02 +1.175000000000000e+02
+8.743049999999999e+02 +1.150000000000000e+02
+1.369960000000000e+03 +2.375000000000000e+02
+7.817389999999998e+02 +1.775000000000000e+02
+7.761680000000000e+02 +1.850000000000000e+02
+9.989560000000000e+02 +1.595000000000000e+02
+1.073050000000000e+03 +1.410000000000000e+02
+9.959030000000000e+02 +1.760000000000000e+02
+8.626820000000000e+02 +1.780000000000000e+02
+7.841319999999999e+02 +1.115000000000000e+02
+1.855480000000000e+03 +2.505000000000000e+02
+9.992940000000000e+02 +1.915000000000000e+02
+8.530930000000002e+02 +1.055000000000000e+02
+9.105990000000000e+02 +1.045000000000000e+02
+9.273869999999999e+02 +1.200000000000000e+02
+1.288470000000000e+03 +1.855000000000000e+02
+1.071580000000000e+03 +1.645000000000000e+02
+9.893940000000000e+02 +1.570000000000000e+02
+9.789890000000000e+02 +1.590000000000000e+02
+9.952020000000000e+02 +1.985000000000000e+02
+1.000240000000000e+03 +1.790000000000000e+02
+1.049500000000000e+03 +1.750000000000000e+02
+8.673830000000000e+02 +2.205000000000000e+02
+4.744770000000000e+02 +7.850000000000000e+01
+9.576130000000001e+02 +1.380000000000000e+02
+9.624520000000000e+02 +1.555000000000000e+02
+9.707790000000000e+02 +1.495000000000000e+02
+1.767740000000000e+03 +2.040000000000000e+02
+1.006500000000000e+03 +1.685000000000000e+02
+9.695410000000001e+02 +1.460000000000000e+02
+2.818990000000000e+02 +6.250000000000000e+01
+8.675650000000001e+02 +1.725000000000000e+02
+9.862000000000000e+02 +1.760000000000000e+02
+1.010930000000000e+03 +1.635000000000000e+02
+1.072290000000000e+03 +1.410000000000000e+02
+1.132600000000000e+03 +2.020000000000000e+02
+8.738630000000001e+02 +1.985000000000000e+02
+9.732569999999999e+02 +1.525000000000000e+02
+4.776820000000000e+02 +7.600000000000000e+01
+9.937260000000000e+02 +1.375000000000000e+02
+2.766260000000000e+02 +5.200000000000000e+01
+7.893950000000000e+02 +1.740000000000000e+02
+9.944160000000001e+02 +1.260000000000000e+02
+9.969510000000000e+02 +1.255000000000000e+02
+8.716720000000000e+02 +1.365000000000000e+02
+1.128970000000000e+03 +1.930000000000000e+02
+1.078040000000000e+03 +1.770000000000000e+02
+9.793140000000000e+02 +1.560000000000000e+02
+1.000680000000000e+03 +1.515000000000000e+02
+9.990839999999999e+02 +1.275000000000000e+02
+1.406640000000000e+03 +2.465000000000000e+02
+7.918130000000000e+02 +1.730000000000000e+02
+8.614180000000000e+02 +9.800000000000000e+01
+9.014410000000000e+02 +9.550000000000000e+01
+1.407000000000000e+03 +2.325000000000000e+02
+9.933530000000000e+02 +1.315000000000000e+02
+9.069570000000000e+02 +1.825000000000000e+02
+2.774420000000000e+02 +4.000000000000000e+01
+1.092470000000000e+03 +1.425000000000000e+02
+8.487940000000000e+02 +1.695000000000000e+02
+1.003640000000000e+03 +1.565000000000000e+02
+1.045520000000000e+02 +9.000000000000000e+00
+6.182840000000000e+02 +1.395000000000000e+02
+8.078560000000001e+02 +1.375000000000000e+02
+1.861630000000000e+03 +2.520000000000000e+02
+8.709660000000000e+02 +1.010000000000000e+02
+1.416610000000000e+03 +2.380000000000000e+02
+9.892740000000000e+02 +1.485000000000000e+02
+9.976410000000000e+02 +1.525000000000000e+02
+1.299640000000000e+03 +1.890000000000000e+02
+8.933450000000000e+02 +1.310000000000000e+02
+6.185010000000000e+02 +1.390000000000000e+02
+1.047620000000000e+03 +1.560000000000000e+02
+8.955870000000000e+02 +1.045000000000000e+02
+1.392550000000000e+03 +2.215000000000000e+02
+9.860470000000000e+02 +1.390000000000000e+02
+6.573560000000001e+02 +1.350000000000000e+02
+9.258730000000000e+02 +9.650000000000000e+01
+6.256630000000000e+02 +1.290000000000000e+02
+6.213270000000000e+02 +1.190000000000000e+02
+1.065530000000000e+03 +1.675000000000000e+02
+1.105600000000000e+03 +2.225000000000000e+02
+9.787480000000000e+02 +1.305000000000000e+02
+9.779580000000000e+02 +1.295000000000000e+02
+6.301110000000000e+02 +1.150000000000000e+02
+1.414820000000000e+03 +2.170000000000000e+02
+6.293930000000000e+02 +1.185000000000000e+02
+1.083240000000000e+03 +1.390000000000000e+02
+6.617180000000002e+02 +1.280000000000000e+02
+1.126530000000000e+03 +1.930000000000000e+02
+1.189700000000000e+03 +1.325000000000000e+02
+1.282820000000000e+03 +2.300000000000000e+02
+9.248040000000000e+02 +1.115000000000000e+02
+6.209760000000000e+02 +1.160000000000000e+02
+6.636419999999998e+02 +1.245000000000000e+02
+1.290160000000000e+03 +2.365000000000000e+02
+9.942520000000000e+02 +1.490000000000000e+02
+6.317940000000000e+02 +9.900000000000000e+01
+2.782850000000000e+02 +2.050000000000000e+01
+1.286990000000000e+03 +2.095000000000000e+02
+9.312170000000000e+02 +8.750000000000000e+01
+6.203140000000000e+02 +8.150000000000000e+01
+8.948670000000000e+02 +9.800000000000000e+01
+6.241400000000000e+02 +1.030000000000000e+02
+1.463780000000000e+03 +2.370000000000000e+02
+9.587390000000000e+02 +1.180000000000000e+02
+6.202950000000000e+02 +8.900000000000000e+01
+6.683660000000001e+02 +1.295000000000000e+02
+4.570590000000000e+02 +3.450000000000000e+01
+1.679330000000000e+03 +2.465000000000000e+02
+6.713650000000000e+02 +1.280000000000000e+02
+2.855090000000000e+02 +2.150000000000000e+01
+9.064000000000000e+02 +1.060000000000000e+02
+8.921980000000000e+02 +9.650000000000000e+01
+9.903170000000000e+02 +1.215000000000000e+02
+1.471300000000000e+03 +1.805000000000000e+02
+1.853750000000000e+03 +2.575000000000000e+02
+6.218070000000000e+02 +9.750000000000000e+01
+4.513120000000000e+02 +3.500000000000000e+01
+1.048940000000000e+03 +1.385000000000000e+02
+1.306340000000000e+03 +2.395000000000000e+02
+9.773390000000001e+02 +1.190000000000000e+02
+9.092460000000000e+02 +9.300000000000000e+01
+9.920250000000000e+02 +1.205000000000000e+02
+6.229109999999999e+02 +7.850000000000000e+01
+6.807880000000000e+02 +1.305000000000000e+02
+7.913819999999999e+02 +1.065000000000000e+02
+8.071160000000001e+02 +1.835000000000000e+02
+1.297890000000000e+03 +1.970000000000000e+02
+1.470230000000000e+03 +1.540000000000000e+02
+1.866930000000000e+03 +2.490000000000000e+02
+8.923500000000000e+02 +8.350000000000000e+01
+1.036800000000000e+03 +1.365000000000000e+02
+6.629200000000000e+02 +1.190000000000000e+02
+1.616920000000000e+03 +2.615000000000000e+02
+8.546600000000000e+02 +8.750000000000000e+01
+7.711860000000000e+02 +1.305000000000000e+02
+1.476300000000000e+03 +1.955000000000000e+02
+9.761200000000000e+02 +1.130000000000000e+02
+1.867420000000000e+03 +2.575000000000000e+02
+7.746720000000000e+02 +1.190000000000000e+02
+7.843370000000000e+02 +1.220000000000000e+02
+6.630790000000000e+02 +1.160000000000000e+02
+1.847880000000000e+03 +2.480000000000000e+02
+7.715770000000000e+02 +1.140000000000000e+02
+7.763930000000000e+02 +1.180000000000000e+02
+7.728850000000000e+02 +1.275000000000000e+02
+1.596540000000000e+03 +2.525000000000000e+02
+1.326600000000000e+03 +1.805000000000000e+02
+1.862580000000000e+03 +2.370000000000000e+02
+1.626420000000000e+03 +2.515000000000000e+02
+9.780170000000001e+02 +1.145000000000000e+02
+7.655939999999998e+02 +1.170000000000000e+02
+7.685119999999999e+02 +1.130000000000000e+02
+8.992970000000000e+02 +9.500000000000000e+01
+6.597739999999999e+02 +1.150000000000000e+02
+1.628500000000000e+03 +2.240000000000000e+02
+7.640610000000000e+02 +1.035000000000000e+02
+6.948070000000000e+02 +1.470000000000000e+02
+6.880690000000000e+02 +1.575000000000000e+02
+9.729930000000001e+02 +2.145000000000000e+02
+1.283130000000000e+03 +1.775000000000000e+02
+1.848620000000000e+03 +2.230000000000000e+02
+9.422560000000000e+02 +2.305000000000000e+02
+1.868570000000000e+03 +2.355000000000000e+02
+7.619220000000000e+02 +1.015000000000000e+02
+8.803080000000000e+02 +8.450000000000000e+01
+9.621390000000000e+02 +1.145000000000000e+02
+7.765860000000000e+02 +1.155000000000000e+02
+9.020960000000000e+02 +8.800000000000000e+01
+7.645230000000000e+02 +1.350000000000000e+02
+1.303490000000000e+03 +1.690000000000000e+02
+7.800830000000002e+02 +8.100000000000000e+01
+8.868580000000002e+02 +9.300000000000000e+01
+6.981580000000000e+02 +1.700000000000000e+02
+6.937189999999998e+02 +1.735000000000000e+02
+7.835930000000002e+02 +7.900000000000000e+01
+8.049750000000000e+02 +1.975000000000000e+02
+7.856960000000000e+02 +1.155000000000000e+02
+3.242440000000000e+02 +6.300000000000000e+01
+7.864220000000000e+02 +1.170000000000000e+02
+7.872210000000000e+02 +1.155000000000000e+02
+2.881940000000000e+02 +2.100000000000000e+01
+6.941280000000000e+02 +1.210000000000000e+02
+1.292100000000000e+03 +1.655000000000000e+02
+7.712110000000000e+02 +1.130000000000000e+02
+9.284299999999999e+02 +1.980000000000000e+02
+9.891130000000001e+02 +1.155000000000000e+02
+3.944520000000000e+02 +3.000000000000000e+01
+7.831039999999998e+02 +1.050000000000000e+02
+1.293640000000000e+03 +2.015000000000000e+02
+8.712680000000000e+02 +8.350000000000000e+01
+6.896480000000000e+02 +1.155000000000000e+02
+7.809680000000002e+02 +1.085000000000000e+02
+8.143750000000000e+02 +1.445000000000000e+02
+6.932639999999999e+02 +1.670000000000000e+02
+7.904889999999998e+02 +1.035000000000000e+02
+9.033560000000000e+02 +8.750000000000000e+01
+4.752800000000000e+02 +8.850000000000000e+01
+7.523230000000000e+02 +1.060000000000000e+02
+6.912950000000000e+02 +1.640000000000000e+02
+7.897760000000002e+02 +9.250000000000000e+01
+1.189600000000000e+03 +1.420000000000000e+02
+7.685670000000000e+02 +9.400000000000000e+01
+7.933339999999999e+02 +1.110000000000000e+02
+7.870700000000001e+02 +8.450000000000000e+01
+2.307360000000000e+03 +2.775000000000000e+02
+6.932810000000002e+02 +1.390000000000000e+02
+1.347140000000000e+03 +1.610000000000000e+02
+7.434900000000000e+02 +1.195000000000000e+02
+6.078520000000000e+02 +7.350000000000000e+01
+1.307610000000000e+03 +1.825000000000000e+02
+7.416180000000001e+02 +1.020000000000000e+02
+1.355970000000000e+03 +1.985000000000000e+02
+1.433300000000000e+03 +1.980000000000000e+02
+7.942320000000000e+02 +1.110000000000000e+02
+1.688000000000000e+03 +2.610000000000000e+02
+1.561180000000000e+03 +2.195000000000000e+02
+6.920780000000000e+02 +5.900000000000000e+01
+1.083040000000000e+02 +1.000000000000000e+01
+7.609610000000000e+02 +8.500000000000000e+01
+1.566360000000000e+03 +1.575000000000000e+02
+7.474069999999998e+02 +9.100000000000000e+01
+1.301220000000000e+03 +2.170000000000000e+02
+1.310550000000000e+03 +1.820000000000000e+02
+7.887210000000000e+02 +8.650000000000000e+01
+7.875560000000000e+02 +1.110000000000000e+02
+8.580080000000000e+02 +1.160000000000000e+02
+8.929030000000000e+02 +1.725000000000000e+02
+4.295350000000000e+02 +3.250000000000000e+01
+1.305180000000000e+03 +2.085000000000000e+02
+6.920889999999998e+02 +1.470000000000000e+02
+1.310780000000000e+03 +1.920000000000000e+02
+7.744910000000001e+02 +8.300000000000000e+01
+9.082310000000000e+02 +2.090000000000000e+02
+9.775800000000000e+02 +1.140000000000000e+02
+7.727980000000000e+02 +1.010000000000000e+02
+4.816240000000000e+02 +1.200000000000000e+02
+6.939860000000001e+02 +1.295000000000000e+02
+8.522990000000000e+02 +1.665000000000000e+02
+1.311360000000000e+03 +2.075000000000000e+02
+1.177910000000000e+02 +1.300000000000000e+01
+7.493670000000000e+02 +8.900000000000000e+01
+7.833589999999998e+02 +8.250000000000000e+01
+7.831080000000002e+02 +1.015000000000000e+02
+6.902560000000002e+02 +8.050000000000000e+01
+1.563110000000000e+03 +2.080000000000000e+02
+7.870030000000000e+02 +7.950000000000000e+01
+4.633770000000000e+02 +7.000000000000000e+01
+9.675490000000000e+02 +1.390000000000000e+02
+1.309010000000000e+03 +1.805000000000000e+02
+7.822110000000000e+02 +7.700000000000000e+01
+4.921460000000000e+02 +1.330000000000000e+02
+1.296350000000000e+03 +2.000000000000000e+02
+7.858589999999998e+02 +1.000000000000000e+02
+8.849800000000000e+02 +1.705000000000000e+02
+7.447510000000002e+02 +5.900000000000000e+01
+6.929560000000000e+02 +7.500000000000000e+01
+7.657630000000000e+02 +9.350000000000000e+01
+6.895030000000000e+02 +6.500000000000000e+01
+1.110340000000000e+03 +1.915000000000000e+02
+7.890160000000002e+02 +7.750000000000000e+01
+8.100430000000000e+02 +2.010000000000000e+02
+4.837160000000000e+02 +1.310000000000000e+02
+1.303720000000000e+03 +2.185000000000000e+02
+8.593739999999998e+02 +1.040000000000000e+02
+7.840760000000000e+02 +9.050000000000000e+01
+1.299410000000000e+03 +2.075000000000000e+02
+4.782460000000000e+02 +5.850000000000000e+01
+1.284610000000000e+03 +1.995000000000000e+02
+7.771860000000000e+02 +8.600000000000000e+01
+9.782569999999999e+02 +1.240000000000000e+02
+9.650069999999999e+02 +1.155000000000000e+02
+7.789870000000000e+02 +7.600000000000000e+01
+6.897320000000000e+02 +7.550000000000000e+01
+7.699280000000000e+02 +1.260000000000000e+02
+1.109260000000000e+03 +2.365000000000000e+02
+4.962450000000000e+02 +1.140000000000000e+02
+4.944750000000000e+02 +1.275000000000000e+02
+7.906250000000000e+02 +7.950000000000000e+01
+9.724580000000000e+02 +1.185000000000000e+02
+1.134440000000000e+03 +1.825000000000000e+02
+1.315130000000000e+03 +1.995000000000000e+02
+7.881319999999999e+02 +7.600000000000000e+01
+7.693889999999999e+02 +8.700000000000000e+01
+1.105790000000000e+03 +2.250000000000000e+02
+7.830380000000000e+02 +7.300000000000000e+01
+7.799349999999999e+02 +8.550000000000000e+01
+1.295770000000000e+03 +1.955000000000000e+02
+4.744870000000000e+02 +1.280000000000000e+02
+1.294050000000000e+03 +1.490000000000000e+02
+7.770430000000000e+02 +8.400000000000000e+01
+6.785060000000002e+02 +1.870000000000000e+02
+8.574140000000000e+02 +9.300000000000000e+01
+4.887740000000000e+02 +1.260000000000000e+02
+6.379250000000000e+02 +5.950000000000000e+01
+1.286280000000000e+03 +1.920000000000000e+02
+7.788160000000000e+02 +7.100000000000000e+01
+7.608539999999998e+02 +8.450000000000000e+01
+6.932890000000000e+02 +8.750000000000000e+01
+9.458200000000001e+02 +1.965000000000000e+02
+6.808339999999999e+02 +1.525000000000000e+02
+4.963380000000000e+02 +1.315000000000000e+02
+1.059500000000000e+03 +1.800000000000000e+02
+1.317410000000000e+03 +1.955000000000000e+02
+7.510580000000000e+02 +6.650000000000000e+01
+7.798280000000000e+02 +8.300000000000000e+01
+6.832880000000000e+02 +1.445000000000000e+02
+1.307120000000000e+03 +1.475000000000000e+02
+7.788819999999999e+02 +7.100000000000000e+01
+8.580200000000000e+02 +9.100000000000000e+01
+1.303220000000000e+03 +1.640000000000000e+02
+9.296330000000000e+02 +1.800000000000000e+02
+4.933930000000000e+02 +1.230000000000000e+02
+4.887770000000000e+02 +1.125000000000000e+02
+7.871070000000000e+02 +7.050000000000000e+01
+7.477210000000000e+02 +5.350000000000000e+01
+1.094750000000000e+03 +2.220000000000000e+02
+8.624130000000000e+02 +1.225000000000000e+02
+8.707719999999998e+02 +8.950000000000000e+01
+3.250210000000000e+02 +2.300000000000000e+01
+1.293860000000000e+03 +1.725000000000000e+02
+7.894299999999999e+02 +7.000000000000000e+01
+7.793560000000001e+02 +8.350000000000000e+01
+7.557680000000000e+02 +9.550000000000000e+01
+7.828539999999998e+02 +6.700000000000000e+01
+7.516270000000000e+02 +7.950000000000000e+01
+8.555280000000000e+02 +7.700000000000000e+01
+1.127450000000000e+03 +2.160000000000000e+02
+4.942210000000000e+02 +1.230000000000000e+02
+7.455470000000000e+02 +6.150000000000000e+01
+7.828170000000000e+02 +8.150000000000000e+01
+6.464190000000000e+02 +2.550000000000000e+02
+7.543969999999998e+02 +1.170000000000000e+02
+9.707030000000000e+02 +9.900000000000000e+01
+1.339850000000000e+03 +2.755000000000000e+02
+1.389660000000000e+03 +2.690000000000000e+02
+9.310340000000000e+02 +1.995000000000000e+02
+7.886890000000000e+02 +6.700000000000000e+01
+7.632410000000001e+02 +6.150000000000000e+01
+4.854800000000000e+02 +9.050000000000000e+01
+7.794900000000000e+02 +7.900000000000000e+01
+7.894140000000000e+02 +7.850000000000000e+01
+1.301740000000000e+03 +1.595000000000000e+02
+8.699299999999999e+02 +8.600000000000000e+01
+1.330150000000000e+03 +2.930000000000000e+02
+4.931880000000001e+02 +9.100000000000000e+01
+1.084620000000000e+03 +1.795000000000000e+02
+1.306590000000000e+03 +1.540000000000000e+02
+7.827560000000002e+02 +6.650000000000000e+01
+8.706700000000000e+02 +7.850000000000000e+01
+4.868200000000000e+02 +1.040000000000000e+02
+7.415250000000000e+02 +1.425000000000000e+02
+1.380860000000000e+03 +2.665000000000000e+02
+1.108860000000000e+03 +2.675000000000000e+02
+7.534580000000002e+02 +7.400000000000000e+01
+4.649990000000000e+02 +4.400000000000000e+01
+1.329440000000000e+03 +2.325000000000000e+02
+4.860720000000000e+02 +8.050000000000000e+01
+7.762580000000000e+02 +7.650000000000000e+01
+1.324990000000000e+03 +2.760000000000000e+02
+1.041750000000000e+03 +1.780000000000000e+02
+7.624950000000000e+02 +7.550000000000000e+01
+8.715440000000000e+02 +1.205000000000000e+02
+1.069780000000000e+03 +1.770000000000000e+02
+1.332970000000000e+03 +2.150000000000000e+02
+4.900820000000000e+02 +8.400000000000000e+01
+1.117740000000000e+03 +1.555000000000000e+02
+7.455900000000000e+02 +5.150000000000000e+01
+8.571319999999999e+02 +7.900000000000000e+01
+8.175700000000001e+02 +1.820000000000000e+02
+7.549160000000001e+02 +6.150000000000000e+01
+1.011560000000000e+03 +1.945000000000000e+02
+1.100260000000000e+03 +1.335000000000000e+02
+1.356210000000000e+03 +1.935000000000000e+02
+1.365020000000000e+03 +2.045000000000000e+02
+1.730490000000000e+03 +2.910000000000000e+02
+1.011340000000000e+03 +1.880000000000000e+02
+7.405889999999998e+02 +7.800000000000000e+01
+1.280900000000000e+03 +2.435000000000000e+02
+8.963320000000000e+02 +1.185000000000000e+02
+6.742439999999998e+02 +1.485000000000000e+02
+8.433140000000000e+02 +7.550000000000000e+01
+7.538110000000000e+02 +6.850000000000000e+01
+1.221440000000000e+03 +2.660000000000000e+02
+6.893400000000000e+02 +1.565000000000000e+02
+7.070330000000000e+02 +2.025000000000000e+02
+6.791660000000001e+02 +1.740000000000000e+02
+9.242510000000000e+02 +1.760000000000000e+02
+9.040570000000000e+02 +1.255000000000000e+02
+1.015920000000000e+03 +1.240000000000000e+02
+1.102980000000000e+03 +2.210000000000000e+02
+1.075640000000000e+03 +2.885000000000000e+02
+6.580419999999998e+02 +1.405000000000000e+02
+1.811120000000000e+03 +2.925000000000000e+02
+1.004430000000000e+03 +1.920000000000000e+02
+6.901200000000000e+02 +9.400000000000000e+01
+1.009750000000000e+03 +1.965000000000000e+02
+1.066830000000000e+03 +1.540000000000000e+02
+6.538300000000000e+02 +1.455000000000000e+02
+1.580330000000000e+03 +2.350000000000000e+02
+8.725490000000000e+02 +6.800000000000000e+01
+1.075460000000000e+03 +2.975000000000000e+02
+6.674780000000002e+02 +1.505000000000000e+02
+8.956270000000000e+02 +1.315000000000000e+02
+6.756820000000000e+02 +1.130000000000000e+02
+1.741130000000000e+03 +2.385000000000000e+02
+6.345290000000000e+02 +1.550000000000000e+02
+6.913250000000000e+02 +1.225000000000000e+02
+1.528950000000000e+03 +2.815000000000000e+02
+1.081580000000000e+03 +1.195000000000000e+02
+1.225660000000000e+03 +2.400000000000000e+02
+1.068350000000000e+03 +2.805000000000000e+02
+9.256849999999999e+02 +1.520000000000000e+02
+1.000740000000000e+03 +1.820000000000000e+02
+1.331260000000000e+03 +2.535000000000000e+02
+6.744789999999998e+02 +1.000000000000000e+02
+2.664430000000000e+03 +3.285000000000000e+02
+9.229520000000000e+02 +1.660000000000000e+02
+6.954330000000000e+02 +9.950000000000000e+01
+6.352650000000000e+02 +1.505000000000000e+02
+1.742680000000000e+03 +2.875000000000000e+02
+1.517340000000000e+03 +2.095000000000000e+02
+1.075780000000000e+03 +2.425000000000000e+02
+1.788050000000000e+03 +2.375000000000000e+02
+6.267330000000002e+02 +1.955000000000000e+02
+9.594180000000000e+02 +1.805000000000000e+02
+6.389380000000000e+02 +1.445000000000000e+02
+6.589090000000000e+02 +1.390000000000000e+02
+1.048300000000000e+03 +1.950000000000000e+02
+6.410599999999999e+02 +1.575000000000000e+02
+9.096770000000000e+02 +1.210000000000000e+02
+8.983330000000002e+02 +1.110000000000000e+02
+1.077510000000000e+02 +2.650000000000000e+01
+1.074910000000000e+03 +2.395000000000000e+02
+1.039350000000000e+03 +1.395000000000000e+02
+6.350700000000001e+02 +1.515000000000000e+02
+1.151150000000000e+03 +2.105000000000000e+02
+9.673440000000001e+02 +9.900000000000000e+01
+6.554600000000000e+02 +1.285000000000000e+02
+1.855720000000000e+03 +2.970000000000000e+02
+8.967600000000000e+02 +1.215000000000000e+02
+1.322290000000000e+03 +2.990000000000000e+02
+6.558260000000000e+02 +1.205000000000000e+02
+2.631300000000000e+03 +3.170000000000000e+02
+6.464930000000001e+02 +2.515000000000000e+02
+6.394109999999999e+02 +1.470000000000000e+02
+1.319680000000000e+03 +2.540000000000000e+02
+1.064620000000000e+03 +1.110000000000000e+02
+6.368410000000000e+02 +1.055000000000000e+02
+6.185110000000000e+02 +1.310000000000000e+02
+1.063070000000000e+03 +1.865000000000000e+02
+6.509770000000000e+02 +1.265000000000000e+02
+8.931039999999998e+02 +1.055000000000000e+02
+1.353730000000000e+03 +2.335000000000000e+02
+6.521970000000000e+02 +1.045000000000000e+02
+1.034360000000000e+03 +1.825000000000000e+02
+6.270119999999999e+02 +2.200000000000000e+02
+1.454410000000000e+03 +2.720000000000000e+02
+6.388270000000000e+02 +1.310000000000000e+02
+8.711080000000002e+02 +7.550000000000000e+01
+6.380210000000000e+02 +1.165000000000000e+02
+9.029500000000000e+02 +1.240000000000000e+02
+8.923830000000000e+02 +1.205000000000000e+02
+1.157470000000000e+03 +2.355000000000000e+02
+6.436510000000000e+02 +8.700000000000000e+01
+8.695880000000002e+02 +8.600000000000000e+01
+6.553530000000002e+02 +1.315000000000000e+02
+6.782430000000001e+02 +1.040000000000000e+02
+1.143580000000000e+02 +3.100000000000000e+01
+1.486170000000000e+03 +2.540000000000000e+02
+1.122000000000000e+03 +2.000000000000000e+02
+6.759589999999999e+02 +1.450000000000000e+02
+6.632539999999998e+02 +8.750000000000000e+01
+1.000580000000000e+03 +1.080000000000000e+02
+8.906160000000001e+02 +1.050000000000000e+02
+1.055390000000000e+03 +1.940000000000000e+02
+1.186760000000000e+02 +1.600000000000000e+01
+1.320070000000000e+03 +2.300000000000000e+02
+7.496430000000000e+02 +1.490000000000000e+02
+6.600939999999998e+02 +1.135000000000000e+02
+1.142470000000000e+03 +2.315000000000000e+02
+7.458980000000000e+02 +5.250000000000000e+01
+6.519710000000000e+02 +1.170000000000000e+02
+8.959310000000000e+02 +1.165000000000000e+02
+6.555570000000000e+02 +1.070000000000000e+02
+6.605860000000000e+02 +7.750000000000000e+01
+8.405130000000000e+02 +6.550000000000000e+01
+1.561370000000000e+03 +2.150000000000000e+02
+7.509939999999998e+02 +1.420000000000000e+02
+6.496860000000000e+02 +1.140000000000000e+02
+6.564130000000000e+02 +6.250000000000000e+01
+8.580260000000002e+02 +1.055000000000000e+02
+1.361740000000000e+03 +2.130000000000000e+02
+6.351730000000000e+02 +1.020000000000000e+02
+1.039340000000000e+03 +1.690000000000000e+02
+6.799490000000000e+02 +1.360000000000000e+02
+6.148910000000000e+02 +1.045000000000000e+02
+6.336849999999999e+02 +1.010000000000000e+02
+9.946559999999999e+02 +7.650000000000000e+01
+8.677769999999998e+02 +1.005000000000000e+02
+6.950069999999999e+02 +1.690000000000000e+02
+6.521590000000000e+02 +9.950000000000000e+01
+8.983300000000000e+02 +9.900000000000000e+01
+1.093930000000000e+02 +1.400000000000000e+01
+6.380430000000000e+02 +1.050000000000000e+02
+8.904510000000000e+02 +1.005000000000000e+02
+8.704169999999998e+02 +9.900000000000000e+01
+6.626830000000000e+02 +9.950000000000000e+01
+1.082120000000000e+03 +1.925000000000000e+02
+8.814019999999998e+02 +1.480000000000000e+02
+8.512910000000001e+02 +9.600000000000000e+01
+6.677360000000001e+02 +1.080000000000000e+02
+6.519540000000002e+02 +9.100000000000000e+01
+1.068640000000000e+03 +3.095000000000000e+02
+6.636100000000000e+02 +7.950000000000000e+01
+8.862890000000000e+02 +1.015000000000000e+02
+1.074890000000000e+03 +1.705000000000000e+02
+6.353740000000000e+02 +9.550000000000000e+01
+1.066780000000000e+03 +2.090000000000000e+02
+1.082490000000000e+03 +2.080000000000000e+02
+6.348720000000000e+02 +7.600000000000000e+01
+6.640360000000002e+02 +7.700000000000000e+01
+1.373290000000000e+03 +2.895000000000000e+02
+6.933750000000000e+02 +1.950000000000000e+02
+1.073580000000000e+03 +1.955000000000000e+02
+6.647150000000000e+02 +9.100000000000000e+01
+1.081620000000000e+03 +2.105000000000000e+02
+6.608860000000002e+02 +7.950000000000000e+01
+6.685450000000000e+02 +7.100000000000000e+01
+1.340690000000000e+03 +2.585000000000000e+02
+1.161210000000000e+03 +1.750000000000000e+02
+1.237930000000000e+03 +2.155000000000000e+02
+6.973049999999999e+02 +1.465000000000000e+02
+6.676430000000000e+02 +1.015000000000000e+02
+9.023830000000000e+02 +1.045000000000000e+02
+1.065430000000000e+03 +1.900000000000000e+02
+1.325290000000000e+03 +2.255000000000000e+02
+1.068890000000000e+03 +2.980000000000000e+02
+6.297940000000000e+02 +8.200000000000000e+01
+1.063560000000000e+03 +1.575000000000000e+02
+6.356100000000000e+02 +8.600000000000000e+01
+1.063240000000000e+03 +1.665000000000000e+02
+6.157840000000000e+02 +1.415000000000000e+02
+1.014130000000000e+03 +1.390000000000000e+02
+1.065290000000000e+03 +1.740000000000000e+02
+1.350040000000000e+03 +2.355000000000000e+02
+1.061650000000000e+03 +3.030000000000000e+02
+6.610160000000002e+02 +7.700000000000000e+01
+6.851330000000000e+02 +1.025000000000000e+02
+1.069510000000000e+03 +1.535000000000000e+02
+5.943890000000000e+02 +1.625000000000000e+02
+8.983270000000000e+02 +1.070000000000000e+02
+1.010260000000000e+03 +1.765000000000000e+02
+1.092240000000000e+03 +1.415000000000000e+02
+8.907850000000000e+02 +2.315000000000000e+02
+6.949140000000000e+02 +1.035000000000000e+02
+6.996039999999998e+02 +1.640000000000000e+02
+9.113160000000000e+02 +1.080000000000000e+02
+1.047300000000000e+03 +1.405000000000000e+02
+8.619400000000001e+02 +6.450000000000000e+01
+6.765889999999998e+02 +1.125000000000000e+02
+8.805710000000000e+02 +1.005000000000000e+02
+1.049270000000000e+03 +1.120000000000000e+02
+8.902470000000000e+02 +2.205000000000000e+02
+6.523060000000000e+02 +9.400000000000000e+01
+6.761039999999998e+02 +1.415000000000000e+02
+9.131470000000000e+02 +1.040000000000000e+02
+9.623800000000000e+02 +1.335000000000000e+02
+8.593210000000000e+02 +1.305000000000000e+02
+8.589630000000002e+02 +1.335000000000000e+02
+8.586950000000001e+02 +1.295000000000000e+02
+9.786120000000000e+02 +1.625000000000000e+02
+1.073770000000000e+03 +2.795000000000000e+02
+9.300260000000000e+02 +1.195000000000000e+02
+1.080960000000000e+03 +1.460000000000000e+02
+8.803410000000000e+02 +2.065000000000000e+02
+7.060239999999999e+02 +9.950000000000000e+01
+1.229980000000000e+03 +2.515000000000000e+02
+8.574920000000000e+02 +1.220000000000000e+02
+9.782390000000000e+02 +1.370000000000000e+02
+1.065530000000000e+03 +2.355000000000000e+02
+1.215190000000000e+03 +2.585000000000000e+02
+1.010890000000000e+03 +1.185000000000000e+02
+1.572440000000000e+03 +1.790000000000000e+02
+1.071360000000000e+03 +2.105000000000000e+02
+8.936050000000000e+02 +1.900000000000000e+02
+1.559630000000000e+03 +3.090000000000000e+02
+1.001700000000000e+03 +2.160000000000000e+02
+8.894380000000000e+02 +9.850000000000000e+01
+8.529970000000000e+02 +2.010000000000000e+02
+7.809110000000002e+02 +1.555000000000000e+02
+1.132570000000000e+03 +2.640000000000000e+02
+1.568390000000000e+03 +3.060000000000000e+02
+7.754220000000000e+02 +1.970000000000000e+02
+1.067240000000000e+03 +2.755000000000000e+02
+9.460069999999999e+02 +7.550000000000000e+01
+5.910660000000000e+02 +9.650000000000000e+01
+1.088780000000000e+03 +1.255000000000000e+02
+9.950460000000000e+02 +2.130000000000000e+02
+7.471790000000000e+02 +9.800000000000000e+01
+9.964790000000000e+02 +1.850000000000000e+02
+4.898100000000000e+02 +8.850000000000000e+01
+1.532860000000000e+03 +3.005000000000000e+02
+8.645230000000000e+02 +1.360000000000000e+02
+1.787660000000000e+03 +3.660000000000000e+02
+8.651330000000000e+02 +1.895000000000000e+02
+1.083840000000000e+03 +1.685000000000000e+02
+1.081080000000000e+03 +1.350000000000000e+02
+1.051820000000000e+03 +2.675000000000000e+02
+2.797730000000000e+02 +1.025000000000000e+02
+2.815610000000000e+02 +8.650000000000000e+01
+9.963260000000000e+02 +1.640000000000000e+02
+1.621000000000000e+03 +3.275000000000000e+02
+1.071050000000000e+03 +1.525000000000000e+02
+7.952430000000001e+02 +1.685000000000000e+02
+7.808180000000000e+02 +1.010000000000000e+02
+1.005040000000000e+03 +2.070000000000000e+02
+1.005560000000000e+03 +1.830000000000000e+02
+1.038010000000000e+03 +1.360000000000000e+02
+1.276110000000000e+03 +2.660000000000000e+02
+1.399120000000000e+03 +3.005000000000000e+02
+9.917329999999999e+02 +1.510000000000000e+02
+6.094109999999999e+02 +1.545000000000000e+02
+1.595160000000000e+03 +2.715000000000000e+02
+8.034400000000001e+02 +1.105000000000000e+02
+6.359580000000002e+02 +6.300000000000000e+01
+2.753600000000000e+02 +6.650000000000000e+01
+9.532600000000000e+02 +1.240000000000000e+02
+9.886060000000000e+02 +1.600000000000000e+02
+1.064730000000000e+03 +2.045000000000000e+02
+4.542900000000000e+02 +6.600000000000000e+01
+6.101180000000001e+02 +1.330000000000000e+02
+9.914390000000000e+02 +1.975000000000000e+02
+9.987840000000000e+02 +1.860000000000000e+02
+1.288290000000000e+03 +2.325000000000000e+02
+1.086570000000000e+03 +1.495000000000000e+02
+9.141960000000000e+02 +1.725000000000000e+02
+1.567790000000000e+03 +1.850000000000000e+02
+1.631730000000000e+03 +3.530000000000000e+02
+9.784430000000000e+02 +1.860000000000000e+02
+1.642310000000000e+03 +3.240000000000000e+02
+1.060780000000000e+03 +1.815000000000000e+02
+2.836060000000000e+02 +6.500000000000000e+01
+9.763090000000000e+02 +2.150000000000000e+02
+1.252950000000000e+03 +2.085000000000000e+02
+1.000350000000000e+03 +1.330000000000000e+02
+1.574440000000000e+03 +2.830000000000000e+02
+1.913790000000000e+03 +2.575000000000000e+02
+1.630860000000000e+03 +3.620000000000000e+02
+1.554070000000000e+03 +3.020000000000000e+02
+1.291620000000000e+03 +2.495000000000000e+02
+1.017500000000000e+03 +1.870000000000000e+02
+1.408410000000000e+03 +2.540000000000000e+02
+6.178120000000000e+02 +1.440000000000000e+02
+6.236619999999998e+02 +1.365000000000000e+02
+1.044120000000000e+03 +1.235000000000000e+02
+1.410560000000000e+03 +2.765000000000000e+02
+1.309950000000000e+03 +2.735000000000000e+02
+1.136090000000000e+03 +2.365000000000000e+02
+1.361400000000000e+03 +2.960000000000000e+02
+6.200040000000000e+02 +1.245000000000000e+02
+1.220480000000000e+02 +2.000000000000000e+01
+1.087160000000000e+03 +1.400000000000000e+02
+1.403540000000000e+03 +2.435000000000000e+02
+6.297869999999998e+02 +1.345000000000000e+02
+6.697230000000002e+02 +2.110000000000000e+02
+1.562640000000000e+03 +2.345000000000000e+02
+9.775130000000000e+02 +1.265000000000000e+02
+9.328090000000000e+02 +1.445000000000000e+02
+9.889140000000000e+02 +2.090000000000000e+02
+9.888190000000000e+02 +1.590000000000000e+02
+1.588280000000000e+03 +3.450000000000000e+02
+1.112240000000000e+03 +1.685000000000000e+02
+1.561360000000000e+03 +2.650000000000000e+02
+1.584210000000000e+03 +2.705000000000000e+02
+2.324560000000000e+03 +3.445000000000000e+02
+1.410980000000000e+03 +2.550000000000000e+02
+9.762880000000000e+02 +1.325000000000000e+02
+9.138320000000000e+02 +1.080000000000000e+02
+6.245530000000000e+02 +9.450000000000000e+01
+6.190180000000000e+02 +1.315000000000000e+02
+6.642380000000001e+02 +2.125000000000000e+02
+6.584910000000001e+02 +1.000000000000000e+02
+1.571690000000000e+03 +2.175000000000000e+02
+1.367620000000000e+03 +1.970000000000000e+02
+1.371060000000000e+03 +3.360000000000000e+02
+1.424250000000000e+03 +2.900000000000000e+02
+9.314030000000000e+02 +1.440000000000000e+02
+1.570920000000000e+03 +2.485000000000000e+02
+1.550710000000000e+03 +2.735000000000000e+02
+1.585200000000000e+03 +2.140000000000000e+02
+2.817700000000000e+02 +2.100000000000000e+01
+1.057900000000000e+02 +9.000000000000000e+00
+1.288540000000000e+03 +2.390000000000000e+02
+9.778180000000000e+02 +1.285000000000000e+02
+6.248210000000000e+02 +1.250000000000000e+02
+6.250509999999998e+02 +9.050000000000000e+01
+1.282440000000000e+03 +2.435000000000000e+02
+9.117960000000000e+02 +1.385000000000000e+02
+2.687510000000000e+03 +3.790000000000000e+02
+9.920560000000000e+02 +1.790000000000000e+02
+1.570030000000000e+03 +2.410000000000000e+02
+6.216659999999998e+02 +1.060000000000000e+02
+1.003530000000000e+03 +1.675000000000000e+02
+1.549580000000000e+03 +2.265000000000000e+02
+9.771520000000000e+02 +1.225000000000000e+02
+1.539390000000000e+03 +2.180000000000000e+02
+6.590230000000000e+02 +1.370000000000000e+02
+1.574800000000000e+03 +2.235000000000000e+02
+1.401580000000000e+03 +2.605000000000000e+02
+1.868550000000000e+03 +3.500000000000000e+02
+6.223110000000000e+02 +7.200000000000000e+01
+1.292450000000000e+03 +2.030000000000000e+02
+1.571220000000000e+03 +2.295000000000000e+02
+1.053700000000000e+03 +1.885000000000000e+02
+6.220190000000000e+02 +7.000000000000000e+01
+6.576530000000000e+02 +9.700000000000000e+01
+9.856650000000000e+02 +1.150000000000000e+02
+6.364250000000000e+02 +1.070000000000000e+02
+2.679850000000000e+03 +3.830000000000000e+02
+1.848540000000000e+03 +3.210000000000000e+02
+6.189400000000001e+02 +8.950000000000000e+01
+1.569480000000000e+03 +2.340000000000000e+02
+1.323950000000000e+03 +2.105000000000000e+02
+2.765570000000000e+02 +2.150000000000000e+01
+6.188600000000000e+02 +6.550000000000000e+01
+9.786980000000000e+02 +1.145000000000000e+02
+1.061060000000000e+03 +1.850000000000000e+02
+6.646120000000000e+02 +7.750000000000000e+01
+1.247330000000000e+03 +2.510000000000000e+02
+1.426450000000000e+03 +2.440000000000000e+02
+6.236690000000000e+02 +6.250000000000000e+01
+1.417780000000000e+03 +2.075000000000000e+02
+7.774700000000000e+02 +1.390000000000000e+02
+6.224560000000000e+02 +5.750000000000000e+01
+7.476350000000000e+02 +1.265000000000000e+02
+1.305390000000000e+03 +2.210000000000000e+02
+7.867510000000002e+02 +1.510000000000000e+02
+1.298800000000000e+03 +2.405000000000000e+02
+1.296560000000000e+03 +2.330000000000000e+02
+1.231470000000000e+03 +2.280000000000000e+02
+1.295110000000000e+03 +2.190000000000000e+02
+7.696760000000000e+02 +1.320000000000000e+02
+7.781330000000000e+02 +1.410000000000000e+02
+7.748049999999999e+02 +1.450000000000000e+02
+1.496370000000000e+03 +3.150000000000000e+02
+1.613760000000000e+03 +3.100000000000000e+02
+1.615560000000000e+03 +3.465000000000000e+02
+2.816110000000000e+02 +2.450000000000000e+01
+6.873580000000002e+02 +1.440000000000000e+02
+1.434350000000000e+03 +2.125000000000000e+02
+6.892660000000002e+02 +1.475000000000000e+02
+1.483140000000000e+03 +2.765000000000000e+02
+7.847070000000000e+02 +1.330000000000000e+02
+7.802239999999998e+02 +1.240000000000000e+02
+1.602540000000000e+03 +3.500000000000000e+02
+1.609440000000000e+03 +3.080000000000000e+02
+7.800030000000000e+02 +1.300000000000000e+02
+6.956319999999999e+02 +1.260000000000000e+02
+1.470270000000000e+03 +2.905000000000000e+02
+1.869350000000000e+03 +3.650000000000000e+02
+1.840090000000000e+03 +3.640000000000000e+02
+7.830100000000000e+02 +1.415000000000000e+02
+1.631610000000000e+03 +3.425000000000000e+02
+1.588410000000000e+03 +3.500000000000000e+02
+7.796360000000002e+02 +1.235000000000000e+02
+7.837710000000002e+02 +1.220000000000000e+02
+2.722370000000000e+02 +1.935000000000000e+02
+1.490860000000000e+03 +2.930000000000000e+02
+1.853040000000000e+03 +3.570000000000000e+02
+6.079520000000000e+02 +9.850000000000000e+01
+1.614570000000000e+03 +2.830000000000000e+02
+6.928560000000001e+02 +1.370000000000000e+02
+7.802210000000000e+02 +1.380000000000000e+02
+9.590750000000000e+02 +2.625000000000000e+02
+7.702630000000000e+02 +1.135000000000000e+02
+1.433810000000000e+03 +1.955000000000000e+02
+1.867350000000000e+03 +3.540000000000000e+02
+6.920670000000000e+02 +1.245000000000000e+02
+1.060980000000000e+02 +1.150000000000000e+01
+1.302890000000000e+03 +2.520000000000000e+02
+7.901680000000000e+02 +1.375000000000000e+02
+7.631950000000001e+02 +1.180000000000000e+02
+2.823050000000000e+02 +2.150000000000000e+01
+1.456170000000000e+03 +1.830000000000000e+02
+7.547040000000000e+02 +1.115000000000000e+02
+7.909930000000001e+02 +1.380000000000000e+02
+1.297800000000000e+03 +2.095000000000000e+02
+6.933710000000002e+02 +1.395000000000000e+02
+9.249500000000000e+02 +2.315000000000000e+02
+7.801940000000000e+02 +1.295000000000000e+02
+8.834130000000000e+02 +1.675000000000000e+02
+1.298730000000000e+03 +1.765000000000000e+02
+7.759910000000001e+02 +1.250000000000000e+02
+8.806519999999998e+02 +1.580000000000000e+02
+6.968750000000000e+02 +1.175000000000000e+02
+7.759240000000000e+02 +1.160000000000000e+02
+1.302440000000000e+03 +1.755000000000000e+02
+1.452760000000000e+03 +1.635000000000000e+02
+7.515490000000000e+02 +1.200000000000000e+02
+6.913510000000001e+02 +1.110000000000000e+02
+1.117180000000000e+03 +2.735000000000000e+02
+5.028980000000000e+02 +1.685000000000000e+02
+9.806870000000000e+02 +1.105000000000000e+02
+7.840060000000002e+02 +1.125000000000000e+02
+1.300960000000000e+03 +1.880000000000000e+02
+8.795139999999999e+02 +1.560000000000000e+02
+6.945290000000000e+02 +1.360000000000000e+02
+8.587420000000000e+02 +2.070000000000000e+02
+4.934060000000000e+02 +1.640000000000000e+02
+7.803110000000000e+02 +1.250000000000000e+02
+6.911710000000000e+02 +7.300000000000000e+01
+1.127300000000000e+03 +2.480000000000000e+02
+7.687719999999998e+02 +1.130000000000000e+02
+1.320000000000000e+03 +1.620000000000000e+02
+7.873800000000000e+02 +1.125000000000000e+02
+1.305480000000000e+03 +2.425000000000000e+02
+7.809030000000000e+02 +1.245000000000000e+02
+6.966150000000000e+02 +7.950000000000000e+01
+1.740760000000000e+03 +2.985000000000000e+02
+5.019030000000000e+02 +1.635000000000000e+02
+1.319260000000000e+03 +2.335000000000000e+02
+7.525670000000000e+02 +9.750000000000000e+01
+8.883290000000000e+02 +1.390000000000000e+02
+6.910830000000002e+02 +1.145000000000000e+02
+3.260670000000000e+02 +9.650000000000000e+01
+1.145340000000000e+03 +2.760000000000000e+02
+7.732830000000000e+02 +1.210000000000000e+02
+8.810700000000001e+02 +1.370000000000000e+02
+1.066040000000000e+03 +1.670000000000000e+02
+8.709019999999998e+02 +1.385000000000000e+02
+1.311690000000000e+03 +2.285000000000000e+02
+7.828339999999999e+02 +1.015000000000000e+02
+1.310600000000000e+03 +2.400000000000000e+02
+6.892270000000000e+02 +9.700000000000000e+01
+1.123510000000000e+03 +2.650000000000000e+02
+1.865110000000000e+03 +3.090000000000000e+02
+4.862930000000000e+02 +1.545000000000000e+02
+7.838969999999998e+02 +1.000000000000000e+02
+1.089610000000000e+03 +1.625000000000000e+02
+1.058130000000000e+03 +1.605000000000000e+02
+6.942960000000000e+02 +9.400000000000000e+01
+7.863689999999998e+02 +1.205000000000000e+02
+9.086070000000000e+02 +1.380000000000000e+02
+1.087410000000000e+02 +1.250000000000000e+01
+6.573580000000002e+02 +1.405000000000000e+02
+6.961310000000002e+02 +8.850000000000000e+01
+9.736760000000000e+02 +1.120000000000000e+02
+1.314740000000000e+03 +2.340000000000000e+02
+7.895410000000001e+02 +1.155000000000000e+02
+1.043750000000000e+03 +1.135000000000000e+02
+1.298640000000000e+03 +2.230000000000000e+02
+9.086830000000000e+02 +1.380000000000000e+02
+1.117310000000000e+03 +1.620000000000000e+02
+1.869350000000000e+03 +3.090000000000000e+02
+7.766100000000000e+02 +9.650000000000000e+01
+8.692689999999999e+02 +2.415000000000000e+02
+4.772770000000000e+02 +1.055000000000000e+02
+2.813530000000000e+02 +1.140000000000000e+02
+9.768620000000000e+02 +1.070000000000000e+02
+1.849970000000000e+03 +3.280000000000000e+02
+7.788150000000001e+02 +9.550000000000000e+01
+1.023080000000000e+03 +1.630000000000000e+02
+1.046790000000000e+03 +1.715000000000000e+02
+6.882669999999998e+02 +1.000000000000000e+02
+9.527990000000000e+02 +2.535000000000000e+02
+1.738420000000000e+03 +3.440000000000000e+02
+1.143130000000000e+03 +2.570000000000000e+02
+4.930510000000000e+02 +1.255000000000000e+02
+9.935050000000000e+02 +1.610000000000000e+02
+7.687100000000000e+02 +1.090000000000000e+02
+8.810460000000000e+02 +1.485000000000000e+02
+1.479690000000000e+03 +3.725000000000000e+02
+1.296690000000000e+03 +2.170000000000000e+02
+8.635300000000000e+02 +1.080000000000000e+02
+6.507680000000000e+02 +1.145000000000000e+02
+1.741680000000000e+03 +3.410000000000000e+02
+1.474000000000000e+03 +2.690000000000000e+02
+7.784440000000000e+02 +1.070000000000000e+02
+6.935110000000002e+02 +5.550000000000000e+01
+1.772890000000000e+03 +2.905000000000000e+02
+1.592950000000000e+03 +3.895000000000000e+02
+7.003230000000000e+02 +1.065000000000000e+02
+9.775040000000000e+02 +1.195000000000000e+02
+4.910370000000000e+02 +1.420000000000000e+02
+1.299320000000000e+03 +2.140000000000000e+02
+1.317820000000000e+03 +1.990000000000000e+02
+1.088910000000000e+03 +1.305000000000000e+02
+1.093870000000000e+03 +1.075000000000000e+02
+6.749720000000000e+02 +1.170000000000000e+02
+1.860850000000000e+03 +3.860000000000000e+02
+1.477290000000000e+03 +2.080000000000000e+02
+7.511200000000000e+02 +9.300000000000000e+01
+9.526070000000000e+02 +1.525000000000000e+02
+7.932650000000000e+02 +1.085000000000000e+02
+1.299200000000000e+03 +2.115000000000000e+02
+1.299810000000000e+03 +1.930000000000000e+02
+1.107530000000000e+03 +1.575000000000000e+02
+1.876380000000000e+03 +3.785000000000000e+02
+7.855810000000000e+02 +1.055000000000000e+02
+9.455790000000000e+02 +1.240000000000000e+02
+6.288020000000000e+02 +2.740000000000000e+02
+1.128610000000000e+03 +2.190000000000000e+02
+1.082660000000000e+03 +1.475000000000000e+02
+1.588120000000000e+03 +3.560000000000000e+02
+6.943980000000000e+02 +7.900000000000000e+01
+6.678670000000000e+02 +1.155000000000000e+02
+4.995450000000000e+02 +1.040000000000000e+02
+7.740030000000000e+02 +9.650000000000000e+01
+9.075590000000000e+02 +1.310000000000000e+02
+9.531369999999999e+02 +2.615000000000000e+02
+6.895290000000000e+02 +6.700000000000000e+01
+1.108310000000000e+03 +2.205000000000000e+02
+6.930069999999999e+02 +1.940000000000000e+02
+7.446799999999999e+02 +9.100000000000000e+01
+7.469330000000000e+02 +1.300000000000000e+02
+4.837150000000000e+02 +1.225000000000000e+02
+7.911280000000000e+02 +1.015000000000000e+02
+1.081920000000000e+03 +1.080000000000000e+02
+1.605010000000000e+03 +3.715000000000000e+02
+1.357230000000000e+03 +2.780000000000000e+02
+1.310430000000000e+03 +2.070000000000000e+02
+1.324830000000000e+03 +1.860000000000000e+02
+1.926020000000000e+03 +3.075000000000000e+02
+9.281880000000000e+02 +2.420000000000000e+02
+4.676780000000001e+02 +8.450000000000000e+01
+9.858960000000000e+02 +1.465000000000000e+02
+7.409370000000000e+02 +1.095000000000000e+02
+7.734470000000000e+02 +9.500000000000000e+01
+1.268240000000000e+03 +2.915000000000000e+02
+1.155820000000000e+03 +2.595000000000000e+02
+7.706740000000000e+02 +9.650000000000000e+01
+1.307420000000000e+03 +1.685000000000000e+02
+1.064450000000000e+03 +1.155000000000000e+02
+2.664320000000000e+03 +3.820000000000000e+02
+1.302430000000000e+03 +1.720000000000000e+02
+4.783400000000000e+02 +8.100000000000000e+01
+7.800850000000000e+02 +9.450000000000000e+01
+7.796500000000000e+02 +9.500000000000000e+01
+9.678890000000000e+02 +1.450000000000000e+02
+1.335700000000000e+03 +2.940000000000000e+02
+1.327980000000000e+03 +2.760000000000000e+02
+7.461160000000001e+02 +1.055000000000000e+02
+5.053580000000000e+02 +1.070000000000000e+02
+9.253600000000000e+02 +1.270000000000000e+02
+7.834850000000000e+02 +9.400000000000000e+01
+1.180290000000000e+03 +2.270000000000000e+02
+7.047420000000000e+02 +1.860000000000000e+02
+4.965020000000000e+02 +7.900000000000000e+01
+9.878590000000000e+02 +1.430000000000000e+02
+1.160730000000000e+03 +2.640000000000000e+02
+7.896220000000000e+02 +9.450000000000000e+01
+8.934160000000001e+02 +1.125000000000000e+02
+8.781270000000000e+02 +1.310000000000000e+02
+1.302380000000000e+03 +3.060000000000000e+02
+7.815069999999999e+02 +9.400000000000000e+01
+1.295760000000000e+03 +1.440000000000000e+02
+9.010490000000000e+02 +1.345000000000000e+02
+4.838840000000000e+02 +6.150000000000000e+01
+7.540970000000000e+02 +1.065000000000000e+02
+7.009450000000001e+02 +2.060000000000000e+02
+9.685060000000000e+02 +1.435000000000000e+02
+1.328300000000000e+03 +3.370000000000000e+02
+7.840280000000000e+02 +9.100000000000000e+01
+7.873190000000000e+02 +9.400000000000000e+01
+4.849410000000000e+02 +1.120000000000000e+02
+4.966870000000000e+02 +1.030000000000000e+02
+1.326630000000000e+03 +3.455000000000000e+02
+7.412410000000001e+02 +9.350000000000000e+01
+8.776339999999999e+02 +1.095000000000000e+02
+9.429680000000000e+02 +2.595000000000000e+02
+7.806750000000000e+02 +8.750000000000000e+01
+7.973110000000000e+02 +9.350000000000000e+01
+9.818720000000000e+02 +1.415000000000000e+02
+1.394480000000000e+03 +2.645000000000000e+02
+6.747070000000000e+02 +2.355000000000000e+02
+1.300440000000000e+03 +1.675000000000000e+02
+9.281079999999999e+02 +1.040000000000000e+02
+1.130400000000000e+03 +2.170000000000000e+02
+8.331880000000000e+02 +9.900000000000000e+01
+7.461260000000002e+02 +8.550000000000000e+01
+7.790700000000001e+02 +8.100000000000000e+01
+1.156140000000000e+03 +2.965000000000000e+02
+1.113450000000000e+03 +2.610000000000000e+02
+7.718720000000000e+02 +8.750000000000000e+01
+7.744080000000000e+02 +8.450000000000000e+01
+9.917950000000000e+02 +2.425000000000000e+02
+1.151070000000000e+03 +2.340000000000000e+02
+6.762050000000000e+02 +2.425000000000000e+02
+1.320380000000000e+03 +2.050000000000000e+02
+9.095440000000000e+02 +1.155000000000000e+02
+9.057220000000000e+02 +1.030000000000000e+02
+9.547410000000000e+02 +2.475000000000000e+02
+1.363900000000000e+03 +2.400000000000000e+02
+9.207040000000000e+02 +2.140000000000000e+02
+7.726139999999998e+02 +7.950000000000000e+01
+7.781080000000002e+02 +7.900000000000000e+01
+6.249190000000000e+02 +2.810000000000000e+02
+7.854440000000000e+02 +7.250000000000000e+01
+9.124960000000000e+02 +1.320000000000000e+02
+9.716130000000001e+02 +1.540000000000000e+02
+1.297200000000000e+03 +3.950000000000000e+02
+1.339670000000000e+03 +2.535000000000000e+02
+1.630170000000000e+03 +3.845000000000000e+02
+7.841500000000000e+02 +7.950000000000000e+01
+4.696520000000000e+02 +4.850000000000000e+01
+6.930750000000000e+02 +2.320000000000000e+02
+8.421970000000000e+02 +9.050000000000000e+01
+7.546339999999999e+02 +5.700000000000000e+01
+1.077720000000000e+03 +2.135000000000000e+02
+8.604510000000000e+02 +1.610000000000000e+02
+1.315430000000000e+03 +3.595000000000000e+02
+9.363180000000000e+02 +2.205000000000000e+02
+1.037380000000000e+03 +2.035000000000000e+02
+1.304180000000000e+03 +3.380000000000000e+02
+1.063050000000000e+03 +1.995000000000000e+02
+1.017230000000000e+03 +1.595000000000000e+02
+9.074660000000000e+02 +1.020000000000000e+02
+8.673550000000000e+02 +1.405000000000000e+02
+6.913060000000000e+02 +1.800000000000000e+02
+1.877730000000000e+03 +3.355000000000000e+02
+1.854860000000000e+03 +3.485000000000000e+02
+1.004200000000000e+03 +2.220000000000000e+02
+1.613220000000000e+03 +3.965000000000000e+02
+1.849990000000000e+03 +3.455000000000000e+02
+6.979730000000002e+02 +1.455000000000000e+02
+6.424360000000000e+02 +1.555000000000000e+02
+9.990690000000000e+02 +2.230000000000000e+02
+8.419820000000000e+02 +1.225000000000000e+02
+9.140510000000000e+02 +1.010000000000000e+02
+6.546350000000000e+02 +2.705000000000000e+02
+1.483600000000000e+03 +2.315000000000000e+02
+6.640089999999999e+02 +1.390000000000000e+02
+8.803150000000001e+02 +1.100000000000000e+02
+9.153869999999999e+02 +1.730000000000000e+02
+1.854800000000000e+03 +3.170000000000000e+02
+6.855800000000000e+02 +1.610000000000000e+02
+6.405970000000000e+02 +1.430000000000000e+02
+6.731000000000000e+02 +1.440000000000000e+02
+1.835600000000000e+03 +4.005000000000000e+02
+9.319990000000000e+02 +2.230000000000000e+02
+1.007510000000000e+03 +2.085000000000000e+02
+6.472160000000000e+02 +2.620000000000000e+02
+6.456200000000000e+02 +1.595000000000000e+02
+1.005830000000000e+03 +2.015000000000000e+02
+1.272240000000000e+03 +2.785000000000000e+02
+1.106860000000000e+03 +3.050000000000000e+02
+6.572170000000000e+02 +1.335000000000000e+02
+1.888230000000000e+03 +3.380000000000000e+02
+1.529770000000000e+03 +3.000000000000000e+02
+8.913919999999998e+02 +9.750000000000000e+01
+6.379410000000000e+02 +1.495000000000000e+02
+8.876950000000001e+02 +2.285000000000000e+02
+8.630169999999998e+02 +1.410000000000000e+02
+6.609989999999998e+02 +2.790000000000000e+02
+6.925280000000000e+02 +1.545000000000000e+02
+6.620889999999998e+02 +1.600000000000000e+02
+2.769680000000000e+02 +2.100000000000000e+01
+6.381920000000000e+02 +1.460000000000000e+02
+9.287180000000000e+02 +2.220000000000000e+02
+8.857719999999998e+02 +9.200000000000000e+01
+8.989019999999998e+02 +2.355000000000000e+02
+1.338400000000000e+03 +2.895000000000000e+02
+1.232720000000000e+03 +1.915000000000000e+02
+6.369040000000000e+02 +1.535000000000000e+02
+6.536910000000000e+02 +1.405000000000000e+02
+1.137900000000000e+03 +3.045000000000000e+02
+7.908589999999998e+02 +1.545000000000000e+02
+6.853570000000000e+02 +1.305000000000000e+02
+6.321090000000000e+02 +1.355000000000000e+02
+1.105400000000000e+03 +2.545000000000000e+02
+9.393350000000000e+02 +2.215000000000000e+02
+2.650270000000000e+02 +9.900000000000000e+01
+1.163130000000000e+03 +2.835000000000000e+02
+1.154490000000000e+03 +2.135000000000000e+02
+8.933670000000000e+02 +7.900000000000000e+01
+6.405190000000000e+02 +1.460000000000000e+02
+6.349510000000000e+02 +1.105000000000000e+02
+1.138260000000000e+03 +2.835000000000000e+02
+7.080630000000000e+02 +2.090000000000000e+02
+6.177800000000000e+02 +1.390000000000000e+02
+1.140440000000000e+03 +2.755000000000000e+02
+1.084160000000000e+03 +2.800000000000000e+02
+1.134130000000000e+03 +2.165000000000000e+02
+1.036450000000000e+03 +1.950000000000000e+02
+6.348450000000000e+02 +1.220000000000000e+02
+8.653869999999999e+02 +1.910000000000000e+02
+1.073740000000000e+03 +3.220000000000000e+02
+6.344550000000000e+02 +1.150000000000000e+02
+9.493579999999999e+02 +2.220000000000000e+02
+1.134500000000000e+03 +2.420000000000000e+02
+6.165790000000002e+02 +9.550000000000000e+01
+2.027940000000000e+03 +4.200000000000000e+02
+1.101620000000000e+03 +2.245000000000000e+02
+9.007420000000000e+02 +9.750000000000000e+01
+8.113600000000000e+02 +1.475000000000000e+02
+1.469470000000000e+03 +2.485000000000000e+02
+1.101530000000000e+03 +1.895000000000000e+02
+6.427660000000000e+02 +1.340000000000000e+02
+6.494310000000000e+02 +1.135000000000000e+02
+6.511920000000000e+02 +7.950000000000000e+01
+1.939750000000000e+03 +3.680000000000000e+02
+1.173610000000000e+03 +2.605000000000000e+02
+6.726469999999998e+02 +1.010000000000000e+02
+9.042060000000000e+02 +9.300000000000000e+01
+9.020580000000000e+02 +8.150000000000000e+01
+1.070140000000000e+03 +2.460000000000000e+02
+6.354510000000000e+02 +1.285000000000000e+02
+6.346669999999998e+02 +1.025000000000000e+02
+6.180820000000000e+02 +7.000000000000000e+01
+9.131930000000000e+02 +1.130000000000000e+02
+7.522189999999998e+02 +1.445000000000000e+02
+6.463730000000000e+02 +1.020000000000000e+02
+6.137600000000000e+02 +9.050000000000000e+01
+6.371580000000000e+02 +7.100000000000000e+01
+6.871239999999998e+02 +1.920000000000000e+02
+1.378100000000000e+03 +3.080000000000000e+02
+1.117270000000000e+03 +2.045000000000000e+02
+6.356730000000000e+02 +1.070000000000000e+02
+6.524770000000000e+02 +7.550000000000000e+01
+1.887100000000000e+03 +3.725000000000000e+02
+9.141620000000000e+02 +1.000000000000000e+02
+6.551000000000000e+02 +9.050000000000000e+01
+1.103560000000000e+03 +2.005000000000000e+02
+1.081330000000000e+03 +2.645000000000000e+02
+1.563610000000000e+03 +2.830000000000000e+02
+6.801289999999998e+02 +1.680000000000000e+02
+6.382400000000000e+02 +8.900000000000000e+01
+6.352769999999998e+02 +6.800000000000000e+01
+1.157350000000000e+03 +3.110000000000000e+02
+6.406880000000000e+02 +6.800000000000000e+01
+1.862040000000000e+03 +3.800000000000000e+02
+1.572440000000000e+03 +3.010000000000000e+02
+1.698550000000000e+03 +2.290000000000000e+02
+7.046950000000001e+02 +2.220000000000000e+02
+6.455459999999998e+02 +6.650000000000000e+01
+6.274760000000000e+02 +9.500000000000000e+01
+6.687200000000000e+02 +6.650000000000000e+01
+1.564350000000000e+03 +2.605000000000000e+02
+1.542500000000000e+03 +2.255000000000000e+02
+1.656430000000000e+03 +2.740000000000000e+02
+1.046760000000000e+03 +2.220000000000000e+02
+1.322300000000000e+03 +2.510000000000000e+02
+6.926210000000002e+02 +2.375000000000000e+02
+6.397250000000000e+02 +8.350000000000000e+01
+6.634190000000000e+02 +6.500000000000000e+01
+9.050839999999999e+02 +6.750000000000000e+01
+1.183240000000000e+02 +3.150000000000000e+01
+1.567630000000000e+03 +2.710000000000000e+02
+1.047870000000000e+03 +2.065000000000000e+02
+1.086600000000000e+02 +2.300000000000000e+01
+1.276790000000000e+03 +2.965000000000000e+02
+6.498480000000002e+02 +6.450000000000000e+01
+1.622100000000000e+03 +4.015000000000000e+02
+1.692850000000000e+03 +2.285000000000000e+02
+6.730020000000000e+02 +9.700000000000000e+01
+1.874510000000000e+03 +3.755000000000000e+02
+1.250430000000000e+03 +2.890000000000000e+02
+1.077740000000000e+02 +2.350000000000000e+01
+1.561130000000000e+03 +2.655000000000000e+02
+1.531120000000000e+03 +2.350000000000000e+02
+1.236450000000000e+03 +2.120000000000000e+02
+1.566060000000000e+03 +2.350000000000000e+02
+1.568280000000000e+03 +1.920000000000000e+02
+1.092960000000000e+03 +2.155000000000000e+02
+6.781330000000000e+02 +1.690000000000000e+02
+7.075760000000000e+02 +1.495000000000000e+02
+8.951050000000000e+02 +2.610000000000000e+02
+6.704960000000002e+02 +9.550000000000000e+01
+8.946519999999998e+02 +6.900000000000000e+01
+7.911100000000000e+02 +2.150000000000000e+02
+1.035960000000000e+03 +2.040000000000000e+02
+1.089900000000000e+03 +2.080000000000000e+02
+1.563580000000000e+03 +2.230000000000000e+02
+6.337050000000000e+02 +7.150000000000000e+01
+7.919190000000000e+02 +2.005000000000000e+02
+1.677170000000000e+03 +2.965000000000000e+02
+1.095660000000000e+03 +2.245000000000000e+02
+8.927050000000000e+02 +2.760000000000000e+02
+6.365020000000000e+02 +8.250000000000000e+01
+7.956080000000002e+02 +1.810000000000000e+02
+1.054980000000000e+03 +1.965000000000000e+02
+1.583980000000000e+03 +2.480000000000000e+02
+1.011400000000000e+03 +2.210000000000000e+02
+1.083730000000000e+03 +2.140000000000000e+02
+7.953960000000002e+02 +1.770000000000000e+02
+1.083580000000000e+03 +2.095000000000000e+02
+1.070730000000000e+03 +1.570000000000000e+02
+7.019980000000000e+02 +1.365000000000000e+02
+6.938900000000000e+02 +1.285000000000000e+02
+1.083220000000000e+03 +1.445000000000000e+02
+4.209520000000000e+02 +3.500000000000000e+01
+1.024020000000000e+03 +2.285000000000000e+02
+6.952189999999998e+02 +1.470000000000000e+02
+4.568890000000000e+02 +4.450000000000000e+01
+7.001799999999999e+02 +1.365000000000000e+02
+1.545710000000000e+03 +3.560000000000000e+02
+1.012590000000000e+03 +1.880000000000000e+02
+4.808260000000000e+02 +4.100000000000000e+01
+8.491849999999999e+02 +2.860000000000000e+02
+1.050930000000000e+03 +1.985000000000000e+02
+1.068210000000000e+03 +1.475000000000000e+02
+1.069460000000000e+03 +1.620000000000000e+02
+1.566090000000000e+03 +3.340000000000000e+02
+1.575840000000000e+03 +2.765000000000000e+02
+1.038970000000000e+03 +1.780000000000000e+02
+1.065480000000000e+03 +1.385000000000000e+02
+7.022530000000000e+02 +1.030000000000000e+02
+1.030780000000000e+03 +1.710000000000000e+02
+1.090310000000000e+03 +1.750000000000000e+02
+1.572950000000000e+03 +2.515000000000000e+02
+1.878480000000000e+03 +4.445000000000000e+02
+8.798460000000000e+02 +1.720000000000000e+02
+8.585660000000000e+02 +1.595000000000000e+02
+9.756100000000000e+02 +1.770000000000000e+02
+9.726830000000000e+02 +1.830000000000000e+02
+1.616440000000000e+03 +3.925000000000000e+02
+9.682200000000000e+02 +8.150000000000000e+01
+8.606870000000000e+02 +1.570000000000000e+02
+1.065900000000000e+03 +1.420000000000000e+02
+6.920790000000000e+02 +2.235000000000000e+02
+7.088780000000000e+02 +1.905000000000000e+02
+1.127380000000000e+03 +2.555000000000000e+02
+1.490320000000000e+03 +4.135000000000000e+02
+1.016800000000000e+03 +1.175000000000000e+02
+1.476330000000000e+03 +2.950000000000000e+02
+1.499540000000000e+03 +3.685000000000000e+02
+1.062450000000000e+03 +1.745000000000000e+02
+9.053700000000000e+02 +2.105000000000000e+02
+1.815440000000000e+03 +4.210000000000000e+02
+8.689960000000002e+02 +1.990000000000000e+02
+7.806760000000000e+02 +1.760000000000000e+02
+2.047940000000000e+03 +4.370000000000000e+02
+9.674220000000000e+02 +9.250000000000000e+01
+9.941210000000000e+02 +1.635000000000000e+02
+7.961500000000000e+02 +1.360000000000000e+02
+9.962809999999999e+02 +2.790000000000000e+02
+4.567290000000000e+02 +3.450000000000000e+01
+7.961380000000000e+02 +1.455000000000000e+02
+2.031180000000000e+03 +2.930000000000000e+02
+1.017330000000000e+03 +2.390000000000000e+02
+1.298220000000000e+03 +3.140000000000000e+02
+9.901849999999999e+02 +2.745000000000000e+02
+8.506430000000000e+02 +1.145000000000000e+02
+2.818490000000000e+02 +1.035000000000000e+02
+2.794370000000000e+02 +1.045000000000000e+02
+9.799000000000000e+02 +1.690000000000000e+02
+9.782530000000000e+02 +1.660000000000000e+02
+6.855760000000000e+02 +6.050000000000000e+01
+1.030710000000000e+03 +2.200000000000000e+02
+1.481770000000000e+03 +4.440000000000000e+02
+1.311080000000000e+03 +3.130000000000000e+02
+8.084169999999998e+02 +1.590000000000000e+02
+1.027680000000000e+03 +2.560000000000000e+02
+1.011740000000000e+03 +2.215000000000000e+02
+1.040660000000000e+03 +1.630000000000000e+02
+9.983869999999999e+02 +2.025000000000000e+02
+2.840520000000000e+02 +1.020000000000000e+02
+2.812750000000000e+02 +9.250000000000000e+01
+1.285250000000000e+03 +3.270000000000000e+02
+9.906240000000000e+02 +1.840000000000000e+02
+1.007930000000000e+03 +2.265000000000000e+02
+1.990490000000000e+03 +3.400000000000000e+02
+1.369720000000000e+03 +3.185000000000000e+02
+1.045370000000000e+03 +1.780000000000000e+02
+9.881390000000000e+02 +1.555000000000000e+02
+9.153080000000000e+02 +1.885000000000000e+02
+1.010830000000000e+03 +2.555000000000000e+02
+1.139550000000000e+03 +1.460000000000000e+02
+1.011540000000000e+03 +1.155000000000000e+02
+2.817420000000000e+02 +7.800000000000000e+01
+1.375090000000000e+03 +3.625000000000000e+02
+9.830230000000000e+02 +1.625000000000000e+02
+9.535510000000000e+02 +1.500000000000000e+02
+1.403960000000000e+03 +3.795000000000000e+02
+1.406010000000000e+03 +3.640000000000000e+02
+6.232959999999998e+02 +1.440000000000000e+02
+1.279970000000000e+03 +2.525000000000000e+02
+6.678780000000000e+02 +4.350000000000000e+01
+1.371980000000000e+03 +2.880000000000000e+02
+9.162610000000000e+02 +1.645000000000000e+02
+6.207010000000000e+02 +1.420000000000000e+02
+1.307410000000000e+03 +2.655000000000000e+02
+6.268740000000000e+02 +1.295000000000000e+02
+6.347530000000000e+02 +1.430000000000000e+02
+6.622960000000000e+02 +1.265000000000000e+02
+9.949180000000000e+02 +1.935000000000000e+02
+1.059760000000000e+03 +1.875000000000000e+02
+2.818570000000000e+02 +5.400000000000000e+01
+1.423080000000000e+03 +3.115000000000000e+02
+1.398540000000000e+03 +3.415000000000000e+02
+9.928080000000000e+02 +2.055000000000000e+02
+6.196380000000000e+02 +1.110000000000000e+02
+6.591369999999999e+02 +1.210000000000000e+02
+8.805710000000000e+02 +1.270000000000000e+02
+8.953339999999999e+02 +1.240000000000000e+02
+1.288980000000000e+03 +2.080000000000000e+02
+9.854000000000000e+02 +1.780000000000000e+02
+9.256750000000000e+02 +1.880000000000000e+02
+1.006670000000000e+03 +1.950000000000000e+02
+2.818420000000000e+02 +5.100000000000000e+01
+1.414440000000000e+03 +3.550000000000000e+02
+9.866470000000000e+02 +1.700000000000000e+02
+1.402850000000000e+03 +2.855000000000000e+02
+6.251160000000000e+02 +1.375000000000000e+02
+1.063560000000000e+03 +1.760000000000000e+02
+1.021990000000000e+03 +1.860000000000000e+02
+9.065110000000000e+02 +1.825000000000000e+02
+6.190060000000000e+02 +1.190000000000000e+02
+8.989870000000000e+02 +1.145000000000000e+02
+1.094400000000000e+03 +1.720000000000000e+02
+6.620239999999999e+02 +1.205000000000000e+02
+7.895980000000002e+02 +1.445000000000000e+02
+9.897250000000000e+02 +1.605000000000000e+02
+9.003020000000000e+02 +1.270000000000000e+02
+9.824800000000000e+02 +1.520000000000000e+02
+2.813320000000000e+02 +6.500000000000000e+01
+1.032050000000000e+03 +1.635000000000000e+02
+6.396390000000000e+02 +1.215000000000000e+02
+8.882760000000002e+02 +1.180000000000000e+02
+1.845520000000000e+03 +4.035000000000000e+02
+9.130590000000000e+02 +1.770000000000000e+02
+1.014450000000000e+03 +1.520000000000000e+02
+6.346799999999999e+02 +1.200000000000000e+02
+6.580650000000001e+02 +7.100000000000000e+01
+9.955490000000000e+02 +1.660000000000000e+02
+1.848280000000000e+03 +3.900000000000000e+02
+8.966980000000000e+02 +1.135000000000000e+02
+2.842320000000000e+02 +8.150000000000000e+01
+6.253780000000000e+02 +9.650000000000000e+01
+9.056240000000000e+02 +9.300000000000000e+01
+1.038670000000000e+03 +1.400000000000000e+02
+1.849460000000000e+03 +3.810000000000000e+02
+6.226990000000002e+02 +9.800000000000000e+01
+1.858830000000000e+03 +3.670000000000000e+02
+1.075460000000000e+03 +1.385000000000000e+02
+1.493320000000000e+03 +3.670000000000000e+02
+6.609000000000000e+02 +9.200000000000000e+01
+2.788510000000000e+02 +4.100000000000000e+01
+9.766050000000000e+02 +1.440000000000000e+02
+1.768110000000000e+03 +4.170000000000000e+02
+1.110570000000000e+03 +1.255000000000000e+02
+1.060060000000000e+03 +1.935000000000000e+02
+1.745590000000000e+03 +3.650000000000000e+02
+6.249910000000000e+02 +8.000000000000000e+01
+6.182780000000000e+02 +1.035000000000000e+02
+1.063090000000000e+02 +9.500000000000000e+00
+1.837250000000000e+03 +4.495000000000000e+02
+6.605470000000000e+02 +6.450000000000000e+01
+8.953120000000000e+02 +1.035000000000000e+02
+9.745119999999999e+02 +1.410000000000000e+02
+8.196690000000000e+02 +9.150000000000000e+01
+9.985380000000000e+02 +1.320000000000000e+02
+9.718220000000000e+02 +1.785000000000000e+02
+6.239670000000000e+02 +6.950000000000000e+01
+6.209680000000002e+02 +6.850000000000000e+01
+1.201700000000000e+03 +2.920000000000000e+02
+2.885880000000000e+02 +6.000000000000000e+01
+1.876810000000000e+03 +4.125000000000000e+02
+1.313360000000000e+03 +2.755000000000000e+02
+7.817270000000000e+02 +1.385000000000000e+02
+1.107760000000000e+03 +1.655000000000000e+02
+1.865340000000000e+03 +3.845000000000000e+02
+9.415140000000000e+02 +1.815000000000000e+02
+6.281010000000000e+02 +6.750000000000000e+01
+7.752370000000000e+02 +1.470000000000000e+02
+9.087350000000000e+02 +1.115000000000000e+02
+1.067460000000000e+03 +1.375000000000000e+02
+1.440450000000000e+03 +3.180000000000000e+02
+6.244900000000000e+02 +6.200000000000000e+01
+1.851990000000000e+03 +4.100000000000000e+02
+1.867790000000000e+03 +4.245000000000000e+02
+9.890940000000001e+02 +2.170000000000000e+02
+9.594310000000000e+02 +2.455000000000000e+02
+1.446790000000000e+03 +3.010000000000000e+02
+1.418460000000000e+03 +2.555000000000000e+02
+7.886990000000000e+02 +1.325000000000000e+02
+1.871100000000000e+03 +3.760000000000000e+02
+9.087809999999999e+02 +1.585000000000000e+02
+7.723360000000000e+02 +1.430000000000000e+02
+1.130740000000000e+03 +1.235000000000000e+02
+9.771070000000000e+02 +1.545000000000000e+02
+6.757260000000001e+02 +1.765000000000000e+02
+1.846280000000000e+03 +4.215000000000000e+02
+1.409370000000000e+03 +2.360000000000000e+02
+1.063760000000000e+03 +1.750000000000000e+02
+7.685620000000000e+02 +1.345000000000000e+02
+9.093180000000000e+02 +9.700000000000000e+01
+1.086500000000000e+02 +1.000000000000000e+01
+7.802130000000002e+02 +1.280000000000000e+02
+8.734310000000000e+02 +1.045000000000000e+02
+1.123880000000000e+03 +1.245000000000000e+02
+2.672500000000000e+02 +5.150000000000000e+01
+6.631010000000001e+02 +1.750000000000000e+02
+1.614740000000000e+03 +3.640000000000000e+02
+9.760960000000000e+02 +1.380000000000000e+02
+7.854800000000000e+02 +1.265000000000000e+02
+1.011070000000000e+03 +1.110000000000000e+02
+8.978099999999999e+02 +3.010000000000000e+02
+7.911319999999999e+02 +1.250000000000000e+02
+3.454450000000000e+02 +1.010000000000000e+02
+4.130720000000000e+02 +5.750000000000000e+01
+6.800920000000000e+02 +1.740000000000000e+02
+8.592500000000000e+02 +8.150000000000000e+01
+6.922689999999999e+02 +1.400000000000000e+02
+6.892189999999998e+02 +1.340000000000000e+02
+9.836150000000000e+02 +1.370000000000000e+02
+4.922920000000000e+02 +1.900000000000000e+02
+1.590920000000000e+03 +2.620000000000000e+02
+9.025230000000000e+02 +2.705000000000000e+02
+7.853850000000000e+02 +1.345000000000000e+02
+3.280780000000001e+02 +9.100000000000000e+01
+7.826870000000000e+02 +1.370000000000000e+02
+8.750549999999999e+02 +1.725000000000000e+02
+6.925730000000000e+02 +1.405000000000000e+02
+6.904040000000000e+02 +8.300000000000000e+01
+9.834170000000000e+02 +1.295000000000000e+02
+7.773240000000000e+02 +1.255000000000000e+02
+4.884930000000001e+02 +1.925000000000000e+02
+7.874760000000001e+02 +1.250000000000000e+02
+7.708270000000000e+02 +1.275000000000000e+02
+8.936419999999998e+02 +2.750000000000000e+02
+8.842919999999998e+02 +3.125000000000000e+02
+7.843830000000000e+02 +1.055000000000000e+02
+7.880350000000000e+02 +1.255000000000000e+02
+1.526630000000000e+03 +2.230000000000000e+02
+1.299230000000000e+03 +2.350000000000000e+02
+7.693389999999998e+02 +1.185000000000000e+02
+1.119600000000000e+03 +1.610000000000000e+02
+1.750860000000000e+03 +4.410000000000000e+02
+9.907050000000000e+02 +1.390000000000000e+02
+7.811730000000000e+02 +1.025000000000000e+02
+7.773969999999998e+02 +1.225000000000000e+02
+1.060610000000000e+03 +1.880000000000000e+02
+7.788290000000000e+02 +1.610000000000000e+02
+9.897470000000000e+02 +1.440000000000000e+02
+8.976760000000000e+02 +2.890000000000000e+02
+8.901669999999998e+02 +1.125000000000000e+02
+9.913660000000000e+02 +1.325000000000000e+02
+1.085680000000000e+03 +2.970000000000000e+02
+5.073800000000000e+02 +1.935000000000000e+02
+7.755820000000000e+02 +1.195000000000000e+02
+1.607480000000000e+03 +3.605000000000000e+02
+7.663700000000000e+02 +1.175000000000000e+02
+1.072270000000000e+02 +1.000000000000000e+01
+7.571760000000000e+02 +9.700000000000000e+01
+1.037240000000000e+03 +1.375000000000000e+02
+1.505810000000000e+03 +1.285000000000000e+02
+7.710510000000000e+02 +1.175000000000000e+02
+9.066390000000000e+02 +1.095000000000000e+02
+7.509670000000000e+02 +1.905000000000000e+02
+9.732340000000000e+02 +1.195000000000000e+02
+9.980839999999999e+02 +1.285000000000000e+02
+4.945080000000000e+02 +1.975000000000000e+02
+7.826350000000000e+02 +1.195000000000000e+02
+1.072110000000000e+03 +1.655000000000000e+02
+8.803539999999998e+02 +1.825000000000000e+02
+1.609130000000000e+03 +3.470000000000000e+02
+7.640720000000000e+02 +1.150000000000000e+02
+1.080420000000000e+03 +1.565000000000000e+02
+1.911300000000000e+03 +2.905000000000000e+02
+9.127450000000000e+02 +9.700000000000000e+01
+1.151240000000000e+03 +2.680000000000000e+02
+6.397310000000000e+02 +1.065000000000000e+02
+1.754020000000000e+03 +4.235000000000000e+02
+1.558910000000000e+03 +2.105000000000000e+02
+7.831460000000002e+02 +1.030000000000000e+02
+7.692810000000002e+02 +1.140000000000000e+02
+5.057190000000000e+02 +1.975000000000000e+02
+7.917700000000000e+02 +1.035000000000000e+02
+1.336500000000000e+03 +2.165000000000000e+02
+7.933660000000001e+02 +1.135000000000000e+02
+1.425960000000000e+03 +2.430000000000000e+02
+7.657060000000000e+02 +9.400000000000000e+01
+8.656640000000000e+02 +1.655000000000000e+02
+1.385240000000000e+03 +1.985000000000000e+02
+1.109990000000000e+03 +2.920000000000000e+02
+8.949180000000000e+02 +8.700000000000000e+01
+8.993090000000000e+02 +7.250000000000000e+01
+4.881460000000000e+02 +1.970000000000000e+02
+1.435320000000000e+03 +2.515000000000000e+02
+1.132710000000000e+03 +1.830000000000000e+02
+7.681310000000002e+02 +9.100000000000000e+01
+1.287190000000000e+03 +2.030000000000000e+02
+5.132320000000000e+02 +5.900000000000000e+01
+9.998890000000000e+02 +1.365000000000000e+02
+7.750610000000000e+02 +1.105000000000000e+02
+1.280690000000000e+03 +2.020000000000000e+02
+7.489180000000000e+02 +8.400000000000000e+01
+8.639780000000002e+02 +8.500000000000000e+01
+8.884480000000000e+02 +8.150000000000000e+01
+5.014090000000000e+02 +2.000000000000000e+02
+1.286150000000000e+03 +1.995000000000000e+02
+1.938060000000000e+03 +3.575000000000000e+02
+8.692719999999998e+02 +1.560000000000000e+02
+6.662600000000000e+02 +1.300000000000000e+02
+8.588730000000000e+02 +1.215000000000000e+02
+8.397200000000000e+02 +1.665000000000000e+02
+9.751090000000000e+02 +1.225000000000000e+02
+7.747550000000000e+02 +1.110000000000000e+02
+1.314690000000000e+03 +1.935000000000000e+02
+4.981690000000000e+02 +1.985000000000000e+02
+1.319540000000000e+03 +1.910000000000000e+02
+7.754030000000000e+02 +1.095000000000000e+02
+8.891000000000000e+02 +8.650000000000000e+01
+1.387980000000000e+03 +3.625000000000000e+02
+1.327840000000000e+03 +3.510000000000000e+02
+1.301530000000000e+03 +1.795000000000000e+02
+7.894050000000000e+02 +9.000000000000000e+01
+3.358610000000000e+02 +2.750000000000000e+01
+9.884450000000001e+02 +1.195000000000000e+02
+7.412530000000000e+02 +1.360000000000000e+02
+4.870380000000000e+02 +1.690000000000000e+02
+1.311890000000000e+03 +1.935000000000000e+02
+7.743980000000000e+02 +1.050000000000000e+02
+1.295980000000000e+03 +1.640000000000000e+02
+7.697050000000000e+02 +8.350000000000000e+01
+8.468630000000001e+02 +7.450000000000000e+01
+1.340240000000000e+03 +3.895000000000000e+02
+1.040700000000000e+03 +1.820000000000000e+02
+1.335790000000000e+03 +3.775000000000000e+02
+4.895490000000000e+02 +1.900000000000000e+02
+1.301820000000000e+03 +1.670000000000000e+02
+7.518400000000000e+02 +7.900000000000000e+01
+4.825490000000000e+02 +4.200000000000000e+01
+8.610250000000000e+02 +2.130000000000000e+02
+1.300750000000000e+03 +1.905000000000000e+02
+7.783570000000000e+02 +1.000000000000000e+02
+8.797980000000000e+02 +1.700000000000000e+02
+4.896410000000000e+02 +1.645000000000000e+02
+1.336120000000000e+03 +3.500000000000000e+02
+1.356340000000000e+03 +2.805000000000000e+02
+7.469119999999998e+02 +1.135000000000000e+02
+7.742910000000001e+02 +8.100000000000000e+01
+8.980660000000000e+02 +8.600000000000000e+01
+1.305760000000000e+03 +1.795000000000000e+02
+1.294020000000000e+03 +1.590000000000000e+02
+1.953250000000000e+03 +3.650000000000000e+02
+8.734470000000000e+02 +1.425000000000000e+02
+4.911120000000000e+02 +1.880000000000000e+02
+7.774160000000001e+02 +8.050000000000000e+01
+7.792130000000002e+02 +9.950000000000000e+01
+8.784700000000000e+02 +7.750000000000000e+01
+1.326910000000000e+03 +3.070000000000000e+02
+1.382900000000000e+03 +3.445000000000000e+02
+4.888930000000000e+02 +1.555000000000000e+02
+7.439390000000000e+02 +1.030000000000000e+02
+7.025910000000000e+02 +1.850000000000000e+02
+7.912710000000002e+02 +9.900000000000000e+01
+7.770680000000000e+02 +7.800000000000000e+01
+1.298160000000000e+03 +1.575000000000000e+02
+7.665970000000000e+02 +9.200000000000000e+01
+1.074150000000000e+03 +4.040000000000000e+02
+6.851060000000001e+02 +2.345000000000000e+02
+1.301660000000000e+03 +1.565000000000000e+02
+1.166040000000000e+03 +2.855000000000000e+02
+9.030180000000000e+02 +8.300000000000000e+01
+1.127710000000000e+03 +2.175000000000000e+02
+1.294960000000000e+03 +1.445000000000000e+02
+7.810280000000000e+02 +9.300000000000000e+01
+7.801650000000000e+02 +7.850000000000000e+01
+1.067790000000000e+03 +3.680000000000000e+02
+7.610419999999998e+02 +9.450000000000000e+01
+1.284280000000000e+03 +1.495000000000000e+02
+9.016650000000000e+02 +8.200000000000000e+01
+1.112970000000000e+02 +1.350000000000000e+01
+3.435480000000000e+02 +3.500000000000000e+01
+1.321040000000000e+03 +3.455000000000000e+02
+9.266190000000000e+02 +1.845000000000000e+02
+8.796920000000000e+02 +1.710000000000000e+02
+7.837700000000000e+02 +7.650000000000000e+01
+7.939720000000000e+02 +9.350000000000000e+01
+1.306270000000000e+03 +3.880000000000000e+02
+7.611180000000001e+02 +9.550000000000000e+01
+1.073210000000000e+03 +3.745000000000000e+02
+1.102880000000000e+03 +1.585000000000000e+02
+7.812270000000000e+02 +7.600000000000000e+01
+9.751900000000001e+02 +1.565000000000000e+02
+6.913610000000001e+02 +1.765000000000000e+02
+6.639430000000000e+02 +3.330000000000000e+02
+1.072530000000000e+03 +3.505000000000000e+02
+1.306800000000000e+03 +3.255000000000000e+02
+7.840369999999998e+02 +7.600000000000000e+01
+1.114430000000000e+03 +3.670000000000000e+02
+8.748739999999998e+02 +7.700000000000000e+01
+7.448630000000001e+02 +5.550000000000000e+01
+6.703260000000000e+02 +1.730000000000000e+02
+1.308300000000000e+03 +1.495000000000000e+02
+7.764920000000000e+02 +9.150000000000000e+01
+1.307640000000000e+03 +3.330000000000000e+02
+7.757040000000000e+02 +6.700000000000000e+01
+7.742030000000000e+02 +8.800000000000000e+01
+1.122050000000000e+03 +4.055000000000000e+02
+7.696460000000002e+02 +7.000000000000000e+01
+6.799000000000000e+02 +2.060000000000000e+02
+1.637890000000000e+03 +2.900000000000000e+02
+9.630230000000000e+02 +1.465000000000000e+02
+7.822970000000000e+02 +8.800000000000000e+01
+2.308980000000000e+03 +4.900000000000000e+02
+1.591350000000000e+03 +2.335000000000000e+02
+7.658630000000001e+02 +1.190000000000000e+02
+7.796550000000000e+02 +6.750000000000000e+01
+7.873260000000000e+02 +8.700000000000000e+01
+2.315500000000000e+03 +4.655000000000000e+02
+6.524670000000000e+02 +3.185000000000000e+02
+6.910010000000002e+02 +2.310000000000000e+02
+2.911920000000000e+02 +3.200000000000000e+01
+9.383850000000000e+02 +2.950000000000000e+02
+7.511770000000000e+02 +6.300000000000000e+01
+8.146810000000000e+02 +2.745000000000000e+02
+1.063950000000000e+03 +3.100000000000000e+02
+8.456910000000000e+02 +1.465000000000000e+02
+3.618580000000000e+02 +1.065000000000000e+02
+1.072540000000000e+03 +3.035000000000000e+02
+6.623300000000000e+02 +1.005000000000000e+02
+1.010150000000000e+03 +2.020000000000000e+02
+8.261250000000000e+02 +2.405000000000000e+02
+6.826100000000000e+02 +1.695000000000000e+02
+9.854660000000000e+02 +2.815000000000000e+02
+1.583060000000000e+03 +3.330000000000000e+02
+6.162710000000000e+02 +3.620000000000000e+02
+1.107940000000000e+03 +3.305000000000000e+02
+1.084470000000000e+03 +2.910000000000000e+02
+7.050650000000001e+02 +2.055000000000000e+02
+7.152760000000002e+02 +1.430000000000000e+02
+7.004770000000000e+02 +1.970000000000000e+02
+8.308689999999998e+02 +1.655000000000000e+02
+1.536290000000000e+03 +2.810000000000000e+02
+1.163520000000000e+03 +2.500000000000000e+02
+1.050090000000000e+03 +2.575000000000000e+02
+1.664660000000000e+03 +2.745000000000000e+02
+1.148160000000000e+03 +3.290000000000000e+02
+6.338180000000000e+02 +1.480000000000000e+02
+4.858720000000000e+02 +3.400000000000000e+01
+6.359870000000000e+02 +1.535000000000000e+02
+2.039270000000000e+03 +4.955000000000000e+02
+6.963160000000000e+02 +2.760000000000000e+02
+6.479190000000000e+02 +1.565000000000000e+02
+9.190309999999999e+02 +2.165000000000000e+02
+8.488430000000002e+02 +1.630000000000000e+02
+9.391710000000000e+02 +2.845000000000000e+02
+1.144970000000000e+03 +2.795000000000000e+02
+6.728150000000001e+02 +3.560000000000000e+02
+1.064680000000000e+03 +2.615000000000000e+02
+6.340190000000000e+02 +1.485000000000000e+02
+6.376770000000000e+02 +1.290000000000000e+02
+7.324739999999998e+02 +8.850000000000000e+01
+6.624100000000000e+02 +1.595000000000000e+02
+9.289190000000000e+02 +2.100000000000000e+02
+6.807750000000000e+02 +2.530000000000000e+02
+6.588030000000000e+02 +1.405000000000000e+02
+1.143340000000000e+03 +2.845000000000000e+02
+9.941460000000000e+02 +1.990000000000000e+02
+1.078020000000000e+03 +2.610000000000000e+02
+2.864180000000000e+02 +1.590000000000000e+02
+6.179380000000000e+02 +1.445000000000000e+02
+6.159420000000000e+02 +1.405000000000000e+02
+2.093700000000000e+03 +5.115000000000000e+02
+6.182260000000000e+02 +1.085000000000000e+02
+1.066560000000000e+03 +2.090000000000000e+02
+6.847719999999998e+02 +2.760000000000000e+02
+6.577940000000000e+02 +1.440000000000000e+02
+9.036670000000000e+02 +2.325000000000000e+02
+6.486030000000002e+02 +1.465000000000000e+02
+6.081120000000000e+02 +1.490000000000000e+02
+6.541210000000000e+02 +1.115000000000000e+02
+2.093050000000000e+03 +5.220000000000000e+02
+1.047860000000000e+03 +1.990000000000000e+02
+1.140060000000000e+03 +2.015000000000000e+02
+7.031930000000000e+02 +2.170000000000000e+02
+6.378510000000000e+02 +1.335000000000000e+02
+1.287160000000000e+03 +3.155000000000000e+02
+6.406440000000000e+02 +1.305000000000000e+02
+9.411000000000000e+02 +2.360000000000000e+02
+6.419950000000000e+02 +1.070000000000000e+02
+1.397010000000000e+03 +3.290000000000000e+02
+1.056290000000000e+03 +3.830000000000000e+02
+6.387520000000000e+02 +1.250000000000000e+02
+6.361120000000000e+02 +1.165000000000000e+02
+2.067940000000000e+03 +5.210000000000000e+02
+6.200660000000000e+02 +8.750000000000000e+01
+6.773480000000002e+02 +1.690000000000000e+02
+8.868240000000000e+02 +2.240000000000000e+02
+1.033020000000000e+03 +2.220000000000000e+02
+6.785910000000000e+02 +1.000000000000000e+02
+6.644540000000000e+02 +1.310000000000000e+02
+8.897880000000000e+02 +1.615000000000000e+02
+7.387410000000001e+02 +2.500000000000000e+02
+6.369450000000001e+02 +7.500000000000000e+01
+6.363930000000000e+02 +9.750000000000000e+01
+1.585550000000000e+03 +3.050000000000000e+02
+1.159480000000000e+03 +2.735000000000000e+02
+1.168430000000000e+03 +3.015000000000000e+02
+6.667160000000000e+02 +9.950000000000000e+01
+6.604889999999998e+02 +6.500000000000000e+01
+1.332940000000000e+03 +2.795000000000000e+02
+6.386369999999999e+02 +1.245000000000000e+02
+1.141600000000000e+03 +3.815000000000000e+02
+1.076360000000000e+03 +3.780000000000000e+02
+9.375930000000000e+02 +1.875000000000000e+02
+6.590620000000000e+02 +9.950000000000000e+01
+6.357809999999999e+02 +7.750000000000000e+01
+1.236860000000000e+03 +3.360000000000000e+02
+1.164240000000000e+03 +3.125000000000000e+02
+9.315460000000000e+02 +2.095000000000000e+02
+8.969360000000000e+02 +8.150000000000000e+01
+7.016220000000000e+02 +1.135000000000000e+02
+1.849060000000000e+03 +4.740000000000000e+02
+6.701160000000001e+02 +1.010000000000000e+02
+6.568780000000000e+02 +9.600000000000000e+01
+2.127850000000000e+03 +4.070000000000000e+02
+6.408780000000000e+02 +1.210000000000000e+02
+1.061360000000000e+03 +1.945000000000000e+02
+6.352640000000000e+02 +1.215000000000000e+02
+6.498130000000000e+02 +7.350000000000000e+01
+6.356190000000000e+02 +6.350000000000000e+01
+7.916690000000000e+02 +2.240000000000000e+02
+1.201330000000000e+03 +3.155000000000000e+02
+6.362930000000000e+02 +1.165000000000000e+02
+6.516780000000000e+02 +1.040000000000000e+02
+8.048800000000000e+02 +2.170000000000000e+02
+7.065520000000000e+02 +1.445000000000000e+02
+9.994000000000000e+02 +1.925000000000000e+02
+6.689210000000000e+02 +1.280000000000000e+02
+6.373000000000000e+02 +1.080000000000000e+02
+7.920230000000000e+02 +2.020000000000000e+02
+6.600450000000000e+02 +9.900000000000000e+01
+8.696230000000000e+02 +1.015000000000000e+02
+2.699940000000000e+03 +5.140000000000000e+02
+8.969550000000000e+02 +1.140000000000000e+02
+7.729710000000000e+02 +1.465000000000000e+02
+1.076230000000000e+03 +3.765000000000000e+02
+6.541369999999999e+02 +7.900000000000000e+01
+1.601440000000000e+03 +3.260000000000000e+02
+1.260850000000000e+03 +2.520000000000000e+02
+6.545180000000000e+02 +1.030000000000000e+02
+6.356210000000000e+02 +8.300000000000000e+01
+8.076160000000001e+02 +2.145000000000000e+02
+1.600380000000000e+03 +3.185000000000000e+02
+8.365410000000001e+02 +9.750000000000000e+01
+1.236160000000000e+03 +1.815000000000000e+02
+4.866730000000000e+02 +5.150000000000000e+01
+8.976189999999998e+02 +4.010000000000000e+02
+6.349090000000000e+02 +9.700000000000000e+01
+1.540490000000000e+03 +2.760000000000000e+02
+1.007290000000000e+03 +1.395000000000000e+02
+6.164299999999999e+02 +7.950000000000000e+01
+1.863700000000000e+03 +4.750000000000000e+02
+1.126350000000000e+03 +1.940000000000000e+02
+1.019640000000000e+03 +1.650000000000000e+02
+1.078720000000000e+03 +2.390000000000000e+02
+6.854670000000000e+02 +1.940000000000000e+02
+8.547389999999998e+02 +1.105000000000000e+02
+1.598230000000000e+03 +4.980000000000000e+02
+1.070750000000000e+03 +2.585000000000000e+02
+6.599980000000000e+02 +2.080000000000000e+02
+1.636130000000000e+03 +4.905000000000000e+02
+1.040930000000000e+03 +1.745000000000000e+02
+1.625760000000000e+03 +4.930000000000000e+02
+7.039960000000002e+02 +1.485000000000000e+02
+1.069300000000000e+03 +1.835000000000000e+02
+9.918300000000000e+02 +1.260000000000000e+02
+1.077000000000000e+03 +2.140000000000000e+02
+6.885210000000002e+02 +2.075000000000000e+02
+8.940110000000002e+02 +1.095000000000000e+02
+1.268930000000000e+03 +2.410000000000000e+02
+1.628280000000000e+03 +4.535000000000000e+02
+1.046930000000000e+03 +1.965000000000000e+02
+1.555710000000000e+03 +2.185000000000000e+02
+1.325530000000000e+03 +2.410000000000000e+02
+6.968930000000000e+02 +1.840000000000000e+02
+6.794850000000000e+02 +1.970000000000000e+02
+1.095030000000000e+03 +1.965000000000000e+02
+1.086460000000000e+03 +1.865000000000000e+02
+1.783560000000000e+03 +3.065000000000000e+02
+1.075570000000000e+03 +1.265000000000000e+02
+6.972650000000000e+02 +1.725000000000000e+02
+1.931700000000000e+03 +4.235000000000000e+02
+1.220200000000000e+03 +2.010000000000000e+02
+1.921010000000000e+03 +3.605000000000000e+02
+6.833539999999998e+02 +1.425000000000000e+02
+7.099440000000000e+02 +1.640000000000000e+02
+1.573440000000000e+03 +2.510000000000000e+02
+8.911700000000000e+02 +1.010000000000000e+02
+1.132360000000000e+03 +1.530000000000000e+02
+1.079370000000000e+03 +2.400000000000000e+02
+1.072110000000000e+03 +1.760000000000000e+02
+6.593250000000000e+02 +1.285000000000000e+02
+1.671870000000000e+03 +2.825000000000000e+02
+1.617620000000000e+03 +2.635000000000000e+02
+1.508630000000000e+03 +4.640000000000000e+02
+1.077040000000000e+03 +2.475000000000000e+02
+8.773550000000000e+02 +1.970000000000000e+02
+8.757289999999998e+02 +1.700000000000000e+02
+8.704010000000002e+02 +1.905000000000000e+02
+1.073050000000000e+03 +2.235000000000000e+02
+1.079090000000000e+03 +1.900000000000000e+02
+1.671450000000000e+03 +3.000000000000000e+02
+1.122310000000000e+03 +3.380000000000000e+02
+9.743240000000000e+02 +1.805000000000000e+02
+8.396220000000000e+02 +8.400000000000000e+01
+1.572960000000000e+03 +2.630000000000000e+02
+9.004190000000000e+02 +1.040000000000000e+02
+8.075350000000000e+02 +2.295000000000000e+02
+1.063800000000000e+03 +2.350000000000000e+02
+1.585410000000000e+03 +2.870000000000000e+02
+8.045139999999999e+02 +2.205000000000000e+02
+1.086450000000000e+03 +1.795000000000000e+02
+1.585490000000000e+03 +2.305000000000000e+02
+1.107180000000000e+03 +1.600000000000000e+02
+1.039640000000000e+03 +1.230000000000000e+02
+9.926140000000000e+02 +1.295000000000000e+02
+1.290030000000000e+03 +4.100000000000000e+02
+9.878460000000000e+02 +1.735000000000000e+02
+1.383040000000000e+03 +2.305000000000000e+02
+7.924420000000000e+02 +1.460000000000000e+02
+1.008710000000000e+03 +2.285000000000000e+02
+1.394770000000000e+03 +2.430000000000000e+02
+1.053060000000000e+03 +1.460000000000000e+02
+1.569380000000000e+03 +4.420000000000000e+02
+1.389640000000000e+03 +3.045000000000000e+02
+8.668650000000000e+02 +9.450000000000000e+01
+1.048030000000000e+03 +1.785000000000000e+02
+6.985870000000000e+02 +1.795000000000000e+02
+9.933440000000001e+02 +7.700000000000000e+01
+1.086490000000000e+03 +1.270000000000000e+02
+8.860930000000002e+02 +7.950000000000000e+01
+1.911830000000000e+03 +4.310000000000000e+02
+1.130100000000000e+03 +2.130000000000000e+02
+1.079450000000000e+03 +2.365000000000000e+02
+9.891140000000000e+02 +1.770000000000000e+02
+1.012410000000000e+03 +2.520000000000000e+02
+1.379240000000000e+03 +2.975000000000000e+02
+1.575970000000000e+03 +1.865000000000000e+02
+1.006150000000000e+03 +2.395000000000000e+02
+1.064940000000000e+03 +1.545000000000000e+02
+7.027539999999998e+02 +1.755000000000000e+02
+1.384680000000000e+03 +2.260000000000000e+02
+8.033900000000000e+02 +1.700000000000000e+02
+1.107800000000000e+03 +1.325000000000000e+02
+1.096760000000000e+03 +1.575000000000000e+02
+9.764670000000000e+02 +1.025000000000000e+02
+1.047550000000000e+03 +2.260000000000000e+02
+1.014880000000000e+03 +2.725000000000000e+02
+2.818490000000000e+02 +9.750000000000000e+01
+2.794470000000000e+02 +9.800000000000000e+01
+1.338140000000000e+03 +2.510000000000000e+02
+1.065510000000000e+03 +1.910000000000000e+02
+1.387010000000000e+03 +2.630000000000000e+02
+8.601260000000002e+02 +9.650000000000000e+01
+9.961200000000000e+02 +1.490000000000000e+02
+2.747480000000000e+02 +8.000000000000000e+01
+1.000690000000000e+03 +1.840000000000000e+02
+9.835700000000001e+02 +1.695000000000000e+02
+9.930510000000000e+02 +2.215000000000000e+02
+1.114660000000000e+03 +2.240000000000000e+02
+1.007980000000000e+03 +1.150000000000000e+02
+1.325970000000000e+03 +2.545000000000000e+02
+1.074460000000000e+03 +2.160000000000000e+02
+1.423690000000000e+03 +4.080000000000000e+02
+1.402510000000000e+03 +3.630000000000000e+02
+1.289320000000000e+03 +2.635000000000000e+02
+9.023150000000001e+02 +8.300000000000000e+01
+1.144130000000000e+03 +1.845000000000000e+02
+1.146110000000000e+03 +2.470000000000000e+02
+9.934950000000000e+02 +1.785000000000000e+02
+6.818120000000000e+02 +6.450000000000000e+01
+9.956720000000000e+02 +1.030000000000000e+02
+1.394380000000000e+03 +4.280000000000000e+02
+7.096519999999998e+02 +7.450000000000000e+01
+1.064460000000000e+03 +2.100000000000000e+02
+9.871559999999999e+02 +1.625000000000000e+02
+1.002060000000000e+03 +2.075000000000000e+02
+8.864939999999998e+02 +2.475000000000000e+02
+1.001300000000000e+03 +2.360000000000000e+02
+9.967600000000000e+02 +9.600000000000000e+01
+1.073290000000000e+03 +2.265000000000000e+02
+8.974470000000000e+02 +1.000000000000000e+02
+8.275419999999998e+02 +8.800000000000000e+01
+1.227070000000000e+03 +2.870000000000000e+02
+9.875850000000000e+02 +1.560000000000000e+02
+6.198300000000000e+02 +1.595000000000000e+02
+1.162960000000000e+03 +3.275000000000000e+02
+1.122320000000000e+03 +2.950000000000000e+02
+1.148620000000000e+03 +1.705000000000000e+02
+1.428820000000000e+03 +4.370000000000000e+02
+6.205850000000000e+02 +1.555000000000000e+02
+1.619850000000000e+03 +4.265000000000000e+02
+9.892340000000000e+02 +1.930000000000000e+02
+1.595400000000000e+03 +4.920000000000000e+02
+2.817890000000000e+02 +3.350000000000000e+01
+1.082490000000000e+03 +2.675000000000000e+02
+1.448210000000000e+03 +2.960000000000000e+02
+9.300119999999999e+02 +2.530000000000000e+02
+9.767340000000000e+02 +1.640000000000000e+02
+8.930939999999998e+02 +1.000000000000000e+02
+6.614100000000000e+02 +1.515000000000000e+02
+1.380870000000000e+03 +4.190000000000000e+02
+1.053000000000000e+03 +1.890000000000000e+02
+6.247840000000000e+02 +1.560000000000000e+02
+6.247110000000000e+02 +1.500000000000000e+02
+1.362360000000000e+03 +3.835000000000000e+02
+1.598730000000000e+03 +4.865000000000000e+02
+1.289700000000000e+03 +3.875000000000000e+02
+1.072770000000000e+03 +2.455000000000000e+02
+1.407040000000000e+03 +4.040000000000000e+02
+1.006260000000000e+03 +2.840000000000000e+02
+1.626190000000000e+03 +4.460000000000000e+02
+1.010740000000000e+03 +2.070000000000000e+02
+8.982160000000000e+02 +9.200000000000000e+01
+1.296840000000000e+03 +3.935000000000000e+02
+1.914850000000000e+03 +5.115000000000000e+02
+9.908800000000000e+02 +1.340000000000000e+02
+9.621609999999999e+02 +1.585000000000000e+02
+6.225810000000000e+02 +1.345000000000000e+02
+2.816630000000000e+02 +8.750000000000000e+01
+2.792490000000000e+02 +9.650000000000000e+01
+1.410610000000000e+03 +4.295000000000000e+02
+1.025900000000000e+03 +1.860000000000000e+02
+1.461200000000000e+03 +3.115000000000000e+02
+9.337880000000000e+02 +2.425000000000000e+02
+1.865770000000000e+03 +4.750000000000000e+02
+8.783260000000000e+02 +7.600000000000000e+01
+1.052040000000000e+03 +2.065000000000000e+02
+6.244400000000001e+02 +1.150000000000000e+02
+6.643110000000000e+02 +1.205000000000000e+02
+6.551750000000000e+02 +4.300000000000000e+01
+2.827820000000000e+02 +5.150000000000000e+01
+1.313730000000000e+03 +2.775000000000000e+02
+9.886120000000000e+02 +1.340000000000000e+02
+9.002510000000002e+02 +6.650000000000000e+01
+2.962260000000000e+02 +9.800000000000000e+01
+6.252130000000002e+02 +1.305000000000000e+02
+6.228600000000000e+02 +1.120000000000000e+02
+1.642170000000000e+03 +4.445000000000000e+02
+8.744000000000000e+02 +7.500000000000000e+01
+1.334910000000000e+03 +3.190000000000000e+02
+6.615089999999999e+02 +9.750000000000000e+01
+1.859530000000000e+03 +4.590000000000000e+02
+6.598910000000002e+02 +8.850000000000000e+01
+1.870670000000000e+03 +4.470000000000000e+02
+2.819980000000000e+02 +3.750000000000000e+01
+6.205860000000000e+02 +7.000000000000000e+01
+8.878400000000000e+02 +7.000000000000000e+01
+1.071710000000000e+03 +2.705000000000000e+02
+8.738950000000000e+02 +3.615000000000000e+02
+9.145510000000000e+02 +2.200000000000000e+02
+1.047100000000000e+03 +2.305000000000000e+02
+2.805150000000000e+02 +3.150000000000000e+01
+1.738790000000000e+03 +4.660000000000000e+02
+6.356430000000000e+02 +7.100000000000000e+01
+6.307200000000000e+02 +4.300000000000000e+01
+6.607550000000000e+02 +7.750000000000000e+01
+1.881820000000000e+03 +4.370000000000000e+02
+8.986120000000000e+02 +7.650000000000000e+01
+1.066010000000000e+03 +1.935000000000000e+02
+9.945710000000000e+02 +1.575000000000000e+02
+9.884780000000000e+02 +1.455000000000000e+02
+6.235700000000001e+02 +6.050000000000000e+01
+1.077980000000000e+03 +1.990000000000000e+02
+1.309790000000000e+03 +2.515000000000000e+02
+9.780620000000000e+02 +1.305000000000000e+02
+9.284070000000000e+02 +2.500000000000000e+02
+6.594320000000000e+02 +8.100000000000000e+01
+1.289330000000000e+03 +3.550000000000000e+02
+6.242470000000000e+02 +5.100000000000000e+01
+1.004090000000000e+03 +2.060000000000000e+02
+8.945530000000000e+02 +6.900000000000000e+01
+6.257909999999998e+02 +4.950000000000000e+01
+2.832220000000000e+02 +2.150000000000000e+01
+4.898500000000000e+02 +2.345000000000000e+02
+1.011820000000000e+03 +1.370000000000000e+02
+1.317590000000000e+03 +3.065000000000000e+02
+6.585050000000000e+02 +6.550000000000000e+01
+1.589260000000000e+03 +2.755000000000000e+02
+9.143970000000000e+02 +2.185000000000000e+02
+9.920140000000000e+02 +1.280000000000000e+02
+9.796300000000000e+02 +2.015000000000000e+02
+7.814820000000000e+02 +1.385000000000000e+02
+6.258150000000001e+02 +4.950000000000000e+01
+4.920060000000000e+02 +2.295000000000000e+02
+7.841680000000000e+02 +1.350000000000000e+02
+1.609210000000000e+03 +3.615000000000000e+02
+9.990300000000000e+02 +1.595000000000000e+02
+6.604220000000000e+02 +5.950000000000000e+01
+2.828090000000000e+02 +2.100000000000000e+01
+1.067050000000000e+03 +2.200000000000000e+02
+7.767830000000000e+02 +1.590000000000000e+02
+1.060760000000000e+03 +1.895000000000000e+02
+9.777130000000000e+02 +1.095000000000000e+02
+9.329850000000000e+02 +2.150000000000000e+02
+4.861300000000000e+02 +2.140000000000000e+02
+1.866440000000000e+03 +4.755000000000000e+02
+7.907530000000000e+02 +1.310000000000000e+02
+1.303570000000000e+03 +3.010000000000000e+02
+9.879520000000000e+02 +1.245000000000000e+02
+6.780419999999998e+02 +1.870000000000000e+02
+7.850860000000000e+02 +1.400000000000000e+02
+9.300970000000000e+02 +2.145000000000000e+02
+4.960720000000000e+02 +2.210000000000000e+02
+7.844839999999998e+02 +1.225000000000000e+02
+8.720020000000000e+02 +6.500000000000000e+01
+8.639860000000001e+02 +1.855000000000000e+02
+6.925260000000002e+02 +1.615000000000000e+02
+1.293520000000000e+03 +2.835000000000000e+02
+6.461770000000000e+02 +1.565000000000000e+02
+7.684299999999999e+02 +1.515000000000000e+02
+1.607340000000000e+03 +2.985000000000000e+02
+1.852070000000000e+03 +4.815000000000000e+02
+7.857450000000000e+02 +1.350000000000000e+02
+1.290040000000000e+03 +3.765000000000000e+02
+9.738380000000000e+02 +1.085000000000000e+02
+7.832050000000000e+02 +1.215000000000000e+02
+7.700850000000000e+02 +1.390000000000000e+02
+4.881460000000000e+02 +2.095000000000000e+02
+7.689010000000002e+02 +1.470000000000000e+02
+7.821590000000000e+02 +1.280000000000000e+02
+6.974180000000000e+02 +2.160000000000000e+02
+1.288340000000000e+03 +2.560000000000000e+02
+4.931890000000000e+02 +2.090000000000000e+02
+7.740130000000000e+02 +1.435000000000000e+02
+8.949210000000000e+02 +6.550000000000000e+01
+8.750419999999998e+02 +3.925000000000000e+02
+1.862830000000000e+03 +5.200000000000000e+02
+1.613670000000000e+03 +3.285000000000000e+02
+1.911140000000000e+03 +3.405000000000000e+02
+7.809280000000000e+02 +1.370000000000000e+02
+6.953720000000000e+02 +1.710000000000000e+02
+1.219460000000000e+03 +2.050000000000000e+02
+9.777880000000000e+02 +1.120000000000000e+02
+4.910380000000000e+02 +2.080000000000000e+02
+7.878960000000002e+02 +1.230000000000000e+02
+6.952569999999999e+02 +2.065000000000000e+02
+1.853700000000000e+03 +4.595000000000000e+02
+7.864900000000000e+02 +1.355000000000000e+02
+6.908450000000000e+02 +2.025000000000000e+02
+7.563770000000000e+02 +1.310000000000000e+02
+7.874200000000000e+02 +1.190000000000000e+02
+6.895219999999998e+02 +1.375000000000000e+02
+4.871660000000000e+02 +1.960000000000000e+02
+7.878040000000000e+02 +1.305000000000000e+02
+7.743190000000000e+02 +1.135000000000000e+02
+8.718420000000000e+02 +2.010000000000000e+02
+1.324350000000000e+03 +2.985000000000000e+02
+1.855070000000000e+03 +5.830000000000000e+02
+1.609240000000000e+03 +2.570000000000000e+02
+7.911970000000000e+02 +1.145000000000000e+02
+8.777220000000000e+02 +2.085000000000000e+02
+1.850420000000000e+03 +4.755000000000000e+02
+1.506450000000000e+03 +3.425000000000000e+02
+1.606570000000000e+03 +2.420000000000000e+02
+1.539360000000000e+03 +3.465000000000000e+02
+1.344250000000000e+03 +3.525000000000000e+02
+4.942910000000000e+02 +2.025000000000000e+02
+7.767150000000000e+02 +1.270000000000000e+02
+7.542080000000002e+02 +1.105000000000000e+02
+1.608990000000000e+03 +2.280000000000000e+02
+7.833939999999999e+02 +1.135000000000000e+02
+7.846110000000001e+02 +1.260000000000000e+02
+1.450240000000000e+03 +4.085000000000000e+02
+1.355970000000000e+03 +2.005000000000000e+02
+6.909050000000000e+02 +1.920000000000000e+02
+9.951150000000000e+02 +1.175000000000000e+02
+1.847890000000000e+03 +5.045000000000000e+02
+7.852139999999998e+02 +1.255000000000000e+02
+1.573630000000000e+03 +4.085000000000000e+02
+8.660690000000000e+02 +2.005000000000000e+02
+8.970000000000000e+02 +3.995000000000000e+02
+1.318380000000000e+03 +2.645000000000000e+02
+1.690220000000000e+03 +3.825000000000000e+02
+1.670890000000000e+03 +2.725000000000000e+02
+1.126930000000000e+03 +2.395000000000000e+02
+4.864550000000000e+02 +1.980000000000000e+02
+7.752050000000000e+02 +1.090000000000000e+02
+7.946189999999998e+02 +1.155000000000000e+02
+1.295470000000000e+03 +2.250000000000000e+02
+1.561840000000000e+03 +2.400000000000000e+02
+1.603770000000000e+03 +2.855000000000000e+02
+1.297120000000000e+03 +2.930000000000000e+02
+1.271640000000000e+03 +2.880000000000000e+02
+1.376030000000000e+03 +3.385000000000000e+02
+1.107930000000000e+03 +2.210000000000000e+02
+7.852719999999998e+02 +1.100000000000000e+02
+1.573170000000000e+03 +2.605000000000000e+02
+3.407530000000001e+02 +1.090000000000000e+02
+7.847780000000000e+02 +1.245000000000000e+02
+1.273610000000000e+03 +1.870000000000000e+02
+1.572160000000000e+03 +3.020000000000000e+02
+5.024430000000000e+02 +2.060000000000000e+02
+1.708390000000000e+03 +3.110000000000000e+02
+7.807320000000000e+02 +1.090000000000000e+02
+1.362210000000000e+03 +3.420000000000000e+02
+1.873030000000000e+03 +5.065000000000000e+02
+1.861680000000000e+03 +4.810000000000000e+02
+1.301520000000000e+03 +2.310000000000000e+02
+7.886640000000000e+02 +1.235000000000000e+02
+1.304600000000000e+03 +2.295000000000000e+02
+1.661180000000000e+03 +2.920000000000000e+02
+1.372250000000000e+03 +3.600000000000000e+02
+1.569910000000000e+03 +2.595000000000000e+02
+1.823480000000000e+03 +4.930000000000000e+02
+2.320400000000000e+03 +5.460000000000000e+02
+1.302160000000000e+03 +2.330000000000000e+02
+7.452400000000000e+02 +1.365000000000000e+02
+5.002260000000000e+02 +2.015000000000000e+02
+1.304800000000000e+03 +2.120000000000000e+02
+1.369180000000000e+03 +3.855000000000000e+02
+7.498630000000001e+02 +1.185000000000000e+02
+2.318940000000000e+03 +5.690000000000000e+02
+8.074250000000000e+02 +3.100000000000000e+02
+7.830430000000000e+02 +1.095000000000000e+02
+1.304300000000000e+03 +2.230000000000000e+02
+8.635169999999998e+02 +1.805000000000000e+02
+1.317250000000000e+03 +2.855000000000000e+02
+9.737150000000000e+02 +2.020000000000000e+02
+1.318210000000000e+03 +1.930000000000000e+02
+7.868900000000000e+02 +1.250000000000000e+02
+6.580820000000000e+02 +4.290000000000000e+02
+1.348920000000000e+03 +3.715000000000000e+02
+1.307480000000000e+03 +1.870000000000000e+02
+7.836189999999998e+02 +1.095000000000000e+02
+1.597200000000000e+03 +3.095000000000000e+02
+7.580930000000002e+02 +1.385000000000000e+02
+7.861790000000000e+02 +1.225000000000000e+02
+2.318510000000000e+03 +5.225000000000000e+02
+1.372930000000000e+03 +3.550000000000000e+02
+1.310850000000000e+03 +2.085000000000000e+02
+1.284550000000000e+03 +1.705000000000000e+02
+3.414110000000000e+02 +1.025000000000000e+02
+1.328240000000000e+03 +3.750000000000000e+02
+7.817310000000001e+02 +1.205000000000000e+02
+7.665119999999999e+02 +1.050000000000000e+02
+8.584939999999998e+02 +1.610000000000000e+02
+1.299590000000000e+03 +1.845000000000000e+02
+1.291530000000000e+03 +3.180000000000000e+02
+9.633810000000000e+02 +1.170000000000000e+02
+6.740889999999998e+02 +1.190000000000000e+02
+7.492650000000000e+02 +1.130000000000000e+02
+1.309440000000000e+03 +1.920000000000000e+02
+7.745030000000000e+02 +1.040000000000000e+02
+1.293380000000000e+03 +1.795000000000000e+02
+7.927310000000001e+02 +1.185000000000000e+02
+1.306090000000000e+03 +1.610000000000000e+02
+1.919280000000000e+03 +3.625000000000000e+02
+8.817200000000000e+02 +1.770000000000000e+02
+1.289260000000000e+03 +3.970000000000000e+02
+7.879310000000000e+02 +1.035000000000000e+02
+1.361980000000000e+03 +3.400000000000000e+02
+3.378219999999999e+02 +8.600000000000000e+01
+7.873680000000001e+02 +1.180000000000000e+02
+7.872660000000002e+02 +1.015000000000000e+02
+8.753989999999999e+02 +1.620000000000000e+02
+1.356130000000000e+03 +3.180000000000000e+02
+1.317250000000000e+03 +1.695000000000000e+02
+3.274480000000000e+02 +8.300000000000000e+01
+1.310060000000000e+03 +1.905000000000000e+02
+7.506250000000000e+02 +1.140000000000000e+02
+8.555740000000000e+02 +1.525000000000000e+02
+1.404920000000000e+03 +3.170000000000000e+02
+4.934730000000000e+02 +1.180000000000000e+02
+1.286320000000000e+03 +1.530000000000000e+02
+7.879320000000000e+02 +1.015000000000000e+02
+1.311040000000000e+03 +1.665000000000000e+02
+7.779710000000000e+02 +1.165000000000000e+02
+3.250290000000000e+02 +8.200000000000000e+01
+7.872040000000000e+02 +1.005000000000000e+02
+1.084260000000000e+03 +4.465000000000000e+02
+7.792730000000000e+02 +1.125000000000000e+02
+1.068570000000000e+03 +4.500000000000000e+02
+1.298800000000000e+03 +1.585000000000000e+02
+1.319590000000000e+03 +1.785000000000000e+02
+7.840939999999998e+02 +9.800000000000000e+01
+5.039160000000000e+02 +5.500000000000000e+01
+7.915390000000000e+02 +9.850000000000000e+01
+1.319870000000000e+03 +3.495000000000000e+02
+1.299540000000000e+03 +1.495000000000000e+02
+1.300670000000000e+03 +1.485000000000000e+02
+7.594470000000000e+02 +1.090000000000000e+02
+8.598080000000000e+02 +1.605000000000000e+02
+1.576970000000000e+03 +3.690000000000000e+02
+1.918290000000000e+03 +4.315000000000000e+02
+7.822439999999998e+02 +1.135000000000000e+02
+1.282450000000000e+03 +1.375000000000000e+02
+7.916270000000000e+02 +1.130000000000000e+02
+1.071260000000000e+03 +3.675000000000000e+02
+6.669989999999998e+02 +2.045000000000000e+02
+7.928869999999999e+02 +1.040000000000000e+02
+7.875580000000000e+02 +1.115000000000000e+02
+3.473150000000000e+02 +8.500000000000000e+01
+1.080260000000000e+03 +4.150000000000000e+02
+7.887080000000002e+02 +1.015000000000000e+02
+6.669900000000000e+02 +4.495000000000000e+02
+4.884410000000000e+02 +9.000000000000000e+01
+1.064600000000000e+03 +3.715000000000000e+02
+7.847120000000000e+02 +1.005000000000000e+02
+7.764349999999999e+02 +9.500000000000000e+01
+7.866770000000000e+02 +1.110000000000000e+02
+1.084390000000000e+03 +2.065000000000000e+02
+7.761150000000000e+02 +9.650000000000000e+01
+7.936669999999998e+02 +1.105000000000000e+02
+7.939670000000000e+02 +9.750000000000000e+01
+3.463510000000000e+02 +6.600000000000000e+01
+1.097430000000000e+03 +4.015000000000000e+02
+7.897970000000000e+02 +9.450000000000000e+01
+1.129110000000000e+03 +3.490000000000000e+02
+1.064900000000000e+03 +3.595000000000000e+02
+9.861890000000000e+02 +2.140000000000000e+02
+1.110160000000000e+03 +1.930000000000000e+02
+6.537800000000000e+02 +4.255000000000000e+02
+7.062910000000001e+02 +2.225000000000000e+02
+8.462890000000000e+02 +9.850000000000000e+01
+1.677950000000000e+03 +3.230000000000000e+02
+1.035100000000000e+03 +3.680000000000000e+02
+8.660470000000000e+02 +2.135000000000000e+02
+8.583210000000000e+02 +9.050000000000000e+01
+1.087520000000000e+03 +2.025000000000000e+02
+8.151300000000000e+02 +3.540000000000000e+02
+1.866120000000000e+03 +5.220000000000000e+02
+1.029600000000000e+03 +1.780000000000000e+02
+1.030360000000000e+03 +3.790000000000000e+02
+6.931580000000000e+02 +2.155000000000000e+02
+7.114910000000001e+02 +2.405000000000000e+02
+7.106480000000000e+02 +2.245000000000000e+02
+6.688620000000000e+02 +9.150000000000000e+01
+1.690660000000000e+03 +3.545000000000000e+02
+1.671390000000000e+03 +3.550000000000000e+02
+1.269340000000000e+03 +3.120000000000000e+02
+1.479730000000000e+03 +4.290000000000000e+02
+1.675590000000000e+03 +3.420000000000000e+02
+1.035150000000000e+03 +3.140000000000000e+02
+6.357460000000000e+02 +1.365000000000000e+02
+1.872240000000000e+03 +5.990000000000000e+02
+1.670470000000000e+03 +3.320000000000000e+02
+6.343850000000000e+02 +1.510000000000000e+02
+9.629120000000000e+02 +3.135000000000000e+02
+1.020090000000000e+03 +1.600000000000000e+02
+6.968330000000002e+02 +1.880000000000000e+02
+7.048400000000000e+02 +2.610000000000000e+02
+7.922680000000000e+02 +1.415000000000000e+02
+1.378350000000000e+03 +2.525000000000000e+02
+6.377700000000000e+02 +1.445000000000000e+02
+2.449090000000000e+03 +6.435000000000000e+02
+2.026650000000000e+03 +5.960000000000000e+02
+1.110080000000000e+03 +1.795000000000000e+02
+1.081470000000000e+03 +1.555000000000000e+02
+1.027480000000000e+03 +3.685000000000000e+02
+6.974520000000000e+02 +1.860000000000000e+02
+6.244430000000000e+02 +1.540000000000000e+02
+6.570219999999998e+02 +1.575000000000000e+02
+1.390480000000000e+03 +2.635000000000000e+02
+6.367040000000002e+02 +1.320000000000000e+02
+8.950910000000000e+02 +1.005000000000000e+02
+6.345680000000000e+02 +1.610000000000000e+02
+1.308820000000000e+03 +5.095000000000000e+02
+1.384590000000000e+03 +3.645000000000000e+02
+6.181980000000000e+02 +1.320000000000000e+02
+1.124280000000000e+03 +1.125000000000000e+02
+1.011440000000000e+03 +3.415000000000000e+02
+1.154010000000000e+03 +3.430000000000000e+02
+1.679280000000000e+03 +2.920000000000000e+02
+2.440740000000000e+03 +6.865000000000000e+02
+9.112220000000000e+02 +1.020000000000000e+02
+6.513250000000000e+02 +1.255000000000000e+02
+6.622530000000000e+02 +1.555000000000000e+02
+6.508170000000000e+02 +1.545000000000000e+02
+2.690030000000000e+03 +4.845000000000000e+02
+1.381170000000000e+03 +3.590000000000000e+02
+9.001450000000000e+02 +9.700000000000000e+01
+6.881560000000002e+02 +9.600000000000000e+01
+1.857100000000000e+03 +5.250000000000000e+02
+9.042190000000001e+02 +9.600000000000000e+01
+6.342660000000000e+02 +1.470000000000000e+02
+6.723290000000000e+02 +9.650000000000000e+01
+1.373920000000000e+03 +3.375000000000000e+02
+6.618919999999998e+02 +1.540000000000000e+02
+2.107570000000000e+03 +6.170000000000000e+02
+1.738860000000000e+03 +4.910000000000000e+02
+1.099050000000000e+03 +1.410000000000000e+02
+1.069460000000000e+03 +1.230000000000000e+02
+6.915139999999999e+02 +1.955000000000000e+02
+6.567669999999998e+02 +1.300000000000000e+02
+9.092760000000000e+02 +9.400000000000000e+01
+1.433970000000000e+03 +1.675000000000000e+02
+6.638700000000000e+02 +1.485000000000000e+02
+6.057430000000001e+02 +8.450000000000000e+01
+2.115860000000000e+03 +6.145000000000000e+02
+1.856630000000000e+03 +5.170000000000000e+02
+8.386489999999999e+02 +9.100000000000000e+01
+8.949839999999998e+02 +9.350000000000000e+01
+6.331990000000002e+02 +1.465000000000000e+02
+6.553989999999999e+02 +9.550000000000000e+01
+1.872480000000000e+03 +4.445000000000000e+02
+6.639280000000000e+02 +2.280000000000000e+02
+1.548790000000000e+03 +4.020000000000000e+02
+8.869169999999998e+02 +8.750000000000000e+01
+9.107270000000000e+02 +7.950000000000000e+01
+1.862420000000000e+03 +4.500000000000000e+02
+8.989910000000001e+02 +4.030000000000000e+02
+7.359739999999998e+02 +1.600000000000000e+02
+6.883570000000000e+02 +1.850000000000000e+02
+6.575910000000000e+02 +1.365000000000000e+02
+6.243940000000000e+02 +8.000000000000000e+01
+1.471030000000000e+03 +3.695000000000000e+02
+6.537569999999999e+02 +1.410000000000000e+02
+8.976250000000000e+02 +9.600000000000000e+01
+1.757980000000000e+03 +4.180000000000000e+02
+6.088560000000000e+02 +1.665000000000000e+02
+8.801230000000000e+02 +7.850000000000000e+01
+6.791750000000000e+02 +2.055000000000000e+02
+6.339030000000000e+02 +8.900000000000000e+01
+6.782110000000000e+02 +1.040000000000000e+02
+1.303790000000000e+03 +2.565000000000000e+02
+6.710850000000000e+02 +1.340000000000000e+02
+1.859000000000000e+03 +5.095000000000000e+02
+8.789460000000000e+02 +2.010000000000000e+02
+6.401930000000000e+02 +1.295000000000000e+02
+8.447200000000000e+02 +8.850000000000000e+01
+8.843420000000000e+02 +3.960000000000000e+02
+1.219210000000000e+03 +4.065000000000000e+02
+8.734670000000000e+02 +2.170000000000000e+02
+6.587320000000000e+02 +9.050000000000000e+01
+1.728840000000000e+03 +3.975000000000000e+02
+1.351880000000000e+03 +3.215000000000000e+02
+1.167350000000000e+03 +1.925000000000000e+02
+6.951289999999998e+02 +2.865000000000000e+02
+6.636450000000000e+02 +1.340000000000000e+02
+1.594750000000000e+03 +3.960000000000000e+02
+6.201469999999998e+02 +1.230000000000000e+02
+1.853130000000000e+03 +5.210000000000000e+02
+8.887210000000000e+02 +8.800000000000000e+01
+6.337809999999999e+02 +1.240000000000000e+02
+6.378800000000000e+02 +7.700000000000000e+01
+1.535320000000000e+03 +3.790000000000000e+02
+6.088150000000001e+02 +1.050000000000000e+02
+1.608200000000000e+03 +4.455000000000000e+02
+8.317430000000001e+02 +7.400000000000000e+01
+1.223100000000000e+03 +3.570000000000000e+02
+7.754380000000000e+02 +1.540000000000000e+02
+1.856060000000000e+03 +4.135000000000000e+02
+1.590520000000000e+03 +4.335000000000000e+02
+6.534019999999998e+02 +1.395000000000000e+02
+8.804920000000000e+02 +7.050000000000000e+01
+6.964050000000000e+02 +2.025000000000000e+02
+6.419820000000000e+02 +8.700000000000000e+01
+9.039970000000000e+02 +8.400000000000000e+01
+6.461950000000001e+02 +9.850000000000000e+01
+1.588710000000000e+03 +2.405000000000000e+02
+1.599520000000000e+03 +3.840000000000000e+02
+8.665970000000000e+02 +2.140000000000000e+02
+1.303150000000000e+03 +3.615000000000000e+02
+1.615150000000000e+03 +5.240000000000000e+02
+6.526410000000000e+02 +1.290000000000000e+02
+1.855410000000000e+03 +4.360000000000000e+02
+1.055940000000000e+03 +4.785000000000000e+02
+6.340780000000000e+02 +7.600000000000000e+01
+1.728250000000000e+03 +3.875000000000000e+02
+1.065360000000000e+03 +4.770000000000000e+02
+6.738439999999998e+02 +9.300000000000000e+01
+1.852860000000000e+03 +4.380000000000000e+02
+1.868840000000000e+03 +5.005000000000000e+02
+6.394400000000001e+02 +1.180000000000000e+02
+1.729530000000000e+03 +3.270000000000000e+02
+6.615760000000000e+02 +1.130000000000000e+02
+1.568970000000000e+03 +2.885000000000000e+02
+8.824110000000002e+02 +1.105000000000000e+02
+9.070680000000000e+02 +6.950000000000000e+01
+8.756710000000000e+02 +1.860000000000000e+02
+1.068600000000000e+03 +2.485000000000000e+02
+8.194989999999998e+02 +2.690000000000000e+02
+1.854070000000000e+03 +4.530000000000000e+02
+1.070520000000000e+03 +2.475000000000000e+02
+6.506870000000000e+02 +1.185000000000000e+02
+4.627030000000000e+02 +7.750000000000000e+01
+1.914900000000000e+03 +2.355000000000000e+02
+6.477959999999998e+02 +1.050000000000000e+02
+1.605860000000000e+03 +2.765000000000000e+02
+8.292900000000000e+02 +7.050000000000000e+01
+1.057020000000000e+03 +2.425000000000000e+02
+1.063150000000000e+03 +2.535000000000000e+02
+6.922430000000001e+02 +2.010000000000000e+02
+6.489870000000000e+02 +1.145000000000000e+02
+7.905280000000000e+02 +1.830000000000000e+02
+8.742170000000000e+02 +2.090000000000000e+02
+4.577800000000000e+02 +5.100000000000000e+01
+6.411270000000000e+02 +7.450000000000000e+01
+1.638590000000000e+03 +5.135000000000000e+02
+8.598170000000000e+02 +7.450000000000000e+01
+1.583570000000000e+03 +5.380000000000000e+02
+8.700419999999998e+02 +2.350000000000000e+02
+1.046300000000000e+03 +2.375000000000000e+02
+6.621489999999999e+02 +1.080000000000000e+02
+8.079630000000002e+02 +1.835000000000000e+02
+1.863420000000000e+03 +4.515000000000000e+02
+8.312689999999999e+02 +1.015000000000000e+02
+1.069380000000000e+03 +1.895000000000000e+02
+6.631210000000002e+02 +1.035000000000000e+02
+1.626280000000000e+03 +5.145000000000000e+02
+1.065420000000000e+03 +1.875000000000000e+02
+8.514349999999999e+02 +5.950000000000000e+01
+7.020510000000000e+02 +2.350000000000000e+02
+1.060520000000000e+03 +1.515000000000000e+02
+6.339019999999998e+02 +1.025000000000000e+02
+1.579120000000000e+03 +2.645000000000000e+02
+1.063150000000000e+03 +2.150000000000000e+02
+1.062520000000000e+03 +1.400000000000000e+02
+7.110610000000000e+02 +2.840000000000000e+02
+1.944210000000000e+03 +3.630000000000000e+02
+1.061760000000000e+03 +1.695000000000000e+02
+8.446070000000000e+02 +5.850000000000000e+01
+1.063500000000000e+03 +1.405000000000000e+02
+1.057820000000000e+03 +1.700000000000000e+02
+7.028670000000000e+02 +2.915000000000000e+02
+7.103450000000000e+02 +2.075000000000000e+02
+1.305510000000000e+03 +4.940000000000000e+02
+1.306120000000000e+03 +4.765000000000000e+02
+1.083880000000000e+03 +1.235000000000000e+02
+8.979340000000000e+02 +6.600000000000000e+01
+1.062300000000000e+03 +1.575000000000000e+02
+7.068280000000000e+02 +2.845000000000000e+02
+1.569000000000000e+03 +2.580000000000000e+02
+1.592710000000000e+03 +5.480000000000000e+02
+9.770480000000000e+02 +1.820000000000000e+02
+1.456140000000000e+03 +5.535000000000000e+02
+1.291970000000000e+03 +4.300000000000000e+02
+1.075790000000000e+03 +1.445000000000000e+02
+1.039530000000000e+03 +1.880000000000000e+02
+9.876630000000000e+02 +1.795000000000000e+02
+1.074390000000000e+03 +1.485000000000000e+02
+8.954520000000000e+02 +6.550000000000000e+01
+8.955219999999998e+02 +7.150000000000000e+01
+8.132030000000000e+02 +3.045000000000000e+02
+1.290200000000000e+03 +4.605000000000000e+02
+1.077670000000000e+03 +1.350000000000000e+02
+6.519299999999999e+02 +1.175000000000000e+02
+6.514950000000000e+02 +1.170000000000000e+02
+1.404950000000000e+03 +2.395000000000000e+02
+8.722730000000000e+02 +1.500000000000000e+02
+1.909780000000000e+03 +3.700000000000000e+02
+1.062870000000000e+03 +1.380000000000000e+02
+1.142840000000000e+03 +4.340000000000000e+02
+1.056180000000000e+03 +1.335000000000000e+02
+1.007290000000000e+03 +2.925000000000000e+02
+9.673880000000000e+02 +1.810000000000000e+02
+1.216280000000000e+03 +2.875000000000000e+02
+1.004980000000000e+03 +3.350000000000000e+02
+1.019070000000000e+03 +2.820000000000000e+02
+8.935269999999998e+02 +6.400000000000000e+01
+1.268880000000000e+03 +1.755000000000000e+02
+7.527189999999998e+02 +9.750000000000000e+01
+1.009930000000000e+03 +2.360000000000000e+02
+2.818490000000000e+02 +9.500000000000000e+01
+1.785840000000000e+03 +3.240000000000000e+02
+1.015000000000000e+03 +2.840000000000000e+02
+1.007430000000000e+03 +2.255000000000000e+02
+1.337510000000000e+03 +3.330000000000000e+02
+9.794990000000000e+02 +1.865000000000000e+02
+7.018110000000000e+02 +1.995000000000000e+02
+1.021880000000000e+03 +2.890000000000000e+02
+1.013800000000000e+03 +2.420000000000000e+02
+1.603720000000000e+03 +4.960000000000000e+02
+2.784860000000000e+02 +8.700000000000000e+01
+9.906020000000000e+02 +1.830000000000000e+02
+1.269140000000000e+03 +3.705000000000000e+02
+8.843950000000000e+02 +6.150000000000000e+01
+1.629690000000000e+03 +4.800000000000000e+02
+1.073230000000000e+03 +2.770000000000000e+02
+1.290910000000000e+03 +4.505000000000000e+02
+1.267880000000000e+03 +2.575000000000000e+02
+1.459060000000000e+03 +4.115000000000000e+02
+1.197680000000000e+03 +2.440000000000000e+02
+1.007760000000000e+03 +2.270000000000000e+02
+1.003720000000000e+03 +2.235000000000000e+02
+1.660070000000000e+03 +3.250000000000000e+02
+1.621650000000000e+03 +2.340000000000000e+02
+1.101930000000000e+03 +3.335000000000000e+02
+1.071180000000000e+03 +2.640000000000000e+02
+1.282500000000000e+03 +3.355000000000000e+02
+1.009530000000000e+03 +2.680000000000000e+02
+1.038030000000000e+03 +2.170000000000000e+02
+1.926340000000000e+03 +3.780000000000000e+02
+2.746240000000000e+02 +5.850000000000000e+01
+1.423480000000000e+03 +4.590000000000000e+02
+9.780210000000000e+02 +1.740000000000000e+02
+6.183470000000000e+02 +1.470000000000000e+02
+1.068470000000000e+03 +2.310000000000000e+02
+9.903010000000000e+02 +1.890000000000000e+02
+1.384280000000000e+03 +4.595000000000000e+02
+9.148120000000000e+02 +3.095000000000000e+02
+6.210250000000000e+02 +1.280000000000000e+02
+1.648380000000000e+03 +1.905000000000000e+02
+7.053320000000000e+02 +9.100000000000000e+01
+1.016440000000000e+03 +3.160000000000000e+02
+1.145010000000000e+03 +3.440000000000000e+02
+1.680920000000000e+03 +3.770000000000000e+02
+2.876790000000001e+02 +6.150000000000000e+01
+1.083960000000000e+03 +2.260000000000000e+02
+1.421300000000000e+03 +5.335000000000000e+02
+6.248110000000000e+02 +7.350000000000000e+01
+6.647370000000000e+02 +1.470000000000000e+02
+1.006260000000000e+03 +2.065000000000000e+02
+6.233910000000000e+02 +1.410000000000000e+02
+1.505170000000000e+03 +4.480000000000000e+02
+7.873639999999998e+02 +1.265000000000000e+02
+1.589310000000000e+03 +2.500000000000000e+02
+1.067250000000000e+03 +1.775000000000000e+02
+9.204990000000000e+02 +3.060000000000000e+02
+1.545670000000000e+03 +2.770000000000000e+02
+1.270650000000000e+03 +2.715000000000000e+02
+6.195760000000000e+02 +1.000000000000000e+02
+6.183940000000000e+02 +7.450000000000000e+01
+6.592130000000002e+02 +9.700000000000000e+01
+9.934010000000000e+02 +1.765000000000000e+02
+1.063150000000000e+03 +2.015000000000000e+02
+9.962660000000000e+02 +1.905000000000000e+02
+2.747700000000000e+02 +3.800000000000000e+01
+8.909160000000001e+02 +3.055000000000000e+02
+1.289560000000000e+03 +1.635000000000000e+02
+1.088080000000000e+03 +2.840000000000000e+02
+8.990880000000002e+02 +4.225000000000000e+02
+6.199150000000000e+02 +5.550000000000000e+01
+6.631010000000001e+02 +1.100000000000000e+02
+5.138310000000000e+02 +2.975000000000000e+02
+1.299770000000000e+03 +2.525000000000000e+02
+6.208280000000000e+02 +7.150000000000000e+01
+7.786120000000000e+02 +9.750000000000000e+01
+1.358380000000000e+03 +2.620000000000000e+02
+9.684750000000000e+02 +1.120000000000000e+02
+9.749070000000000e+02 +1.745000000000000e+02
+9.535770000000000e+02 +1.110000000000000e+02
+3.176000000000000e+02 +9.150000000000000e+01
+4.991090000000000e+02 +2.905000000000000e+02
+1.391890000000000e+03 +2.385000000000000e+02
+4.807600000000000e+02 +6.450000000000000e+01
+1.858670000000000e+03 +5.505000000000000e+02
+6.581669999999998e+02 +5.250000000000000e+01
+1.264860000000000e+03 +2.575000000000000e+02
+1.828720000000000e+03 +5.675000000000000e+02
+1.396440000000000e+03 +2.325000000000000e+02
+1.073180000000000e+03 +2.800000000000000e+02
+1.480670000000000e+03 +4.860000000000000e+02
+5.001600000000000e+02 +2.985000000000000e+02
+9.934170000000000e+02 +1.565000000000000e+02
+1.384000000000000e+03 +2.020000000000000e+02
+4.184730000000000e+02 +6.450000000000000e+01
+1.309220000000000e+03 +2.650000000000000e+02
+9.595990000000000e+02 +1.650000000000000e+02
+1.465390000000000e+03 +4.780000000000000e+02
+4.989940000000000e+02 +2.765000000000000e+02
+3.394240000000001e+02 +8.950000000000000e+01
+1.758530000000000e+03 +5.835000000000000e+02
+2.327120000000000e+03 +3.640000000000000e+02
+9.046110000000000e+02 +4.280000000000000e+02
+7.836239999999998e+02 +1.565000000000000e+02
+1.878320000000000e+03 +5.560000000000000e+02
+4.919940000000000e+02 +2.495000000000000e+02
+1.617150000000000e+03 +3.010000000000000e+02
+7.788939999999999e+02 +1.605000000000000e+02
+9.630000000000000e+02 +1.570000000000000e+02
+9.949840000000000e+02 +1.550000000000000e+02
+9.135300000000000e+02 +2.955000000000000e+02
+7.809610000000000e+02 +1.595000000000000e+02
+7.762370000000000e+02 +1.480000000000000e+02
+7.804320000000000e+02 +1.700000000000000e+02
+1.627050000000000e+03 +2.955000000000000e+02
+7.860910000000000e+02 +2.940000000000000e+02
+4.875800000000000e+02 +2.825000000000000e+02
+2.822160000000000e+02 +2.100000000000000e+01
+7.731619999999998e+02 +1.665000000000000e+02
+7.770549999999999e+02 +1.455000000000000e+02
+7.657320000000000e+02 +1.565000000000000e+02
+7.793210000000000e+02 +2.415000000000000e+02
+3.593769999999999e+02 +8.150000000000000e+01
+6.878789999999998e+02 +2.530000000000000e+02
+4.887830000000000e+02 +2.390000000000000e+02
+1.627590000000000e+03 +2.800000000000000e+02
+7.874169999999998e+02 +1.390000000000000e+02
+7.765200000000000e+02 +1.395000000000000e+02
+1.355810000000000e+03 +1.470000000000000e+02
+8.788020000000000e+02 +4.895000000000000e+02
+1.590510000000000e+03 +2.140000000000000e+02
+3.325250000000000e+02 +6.800000000000000e+01
+9.788410000000000e+02 +1.780000000000000e+02
+7.812500000000000e+02 +1.635000000000000e+02
+6.901810000000000e+02 +2.760000000000000e+02
+7.474119999999998e+02 +1.185000000000000e+02
+7.847310000000001e+02 +1.475000000000000e+02
+1.091890000000000e+03 +2.720000000000000e+02
+5.043640000000000e+02 +2.830000000000000e+02
+7.960160000000002e+02 +1.620000000000000e+02
+1.604150000000000e+03 +2.175000000000000e+02
+2.291900000000000e+03 +6.845000000000000e+02
+7.514620000000000e+02 +1.275000000000000e+02
+9.908780000000000e+02 +1.585000000000000e+02
+7.627170000000000e+02 +1.185000000000000e+02
+7.897239999999998e+02 +1.405000000000000e+02
+1.874020000000000e+03 +5.585000000000000e+02
+4.976250000000000e+02 +2.465000000000000e+02
+3.365760000000000e+02 +6.450000000000000e+01
+9.792340000000000e+02 +1.700000000000000e+02
+1.614700000000000e+03 +2.820000000000000e+02
+7.901469999999998e+02 +1.410000000000000e+02
+1.386760000000000e+03 +4.125000000000000e+02
+1.339700000000000e+03 +4.905000000000000e+02
+7.934270000000000e+02 +1.640000000000000e+02
+1.624350000000000e+03 +4.920000000000000e+02
+9.925700000000001e+02 +1.695000000000000e+02
+1.865050000000000e+03 +5.665000000000000e+02
+9.873000000000000e+02 +1.510000000000000e+02
+7.788430000000002e+02 +1.300000000000000e+02
+7.885989999999998e+02 +1.485000000000000e+02
+1.660270000000000e+03 +2.370000000000000e+02
+7.938330000000002e+02 +1.340000000000000e+02
+1.003010000000000e+03 +1.705000000000000e+02
+1.097770000000000e+03 +3.425000000000000e+02
+1.378320000000000e+03 +4.085000000000000e+02
+1.876120000000000e+03 +5.435000000000000e+02
+4.888150000000000e+02 +2.445000000000000e+02
+7.720700000000001e+02 +1.260000000000000e+02
+9.950200000000000e+02 +1.485000000000000e+02
+7.959780000000002e+02 +1.245000000000000e+02
+7.707280000000002e+02 +1.275000000000000e+02
+4.886710000000000e+02 +9.900000000000000e+01
+6.940599999999999e+02 +2.755000000000000e+02
+1.259590000000000e+03 +2.825000000000000e+02
+1.872320000000000e+03 +5.545000000000000e+02
+1.306720000000000e+03 +2.130000000000000e+02
+7.896960000000000e+02 +1.500000000000000e+02
+8.560219999999998e+02 +1.520000000000000e+02
+1.355530000000000e+03 +3.925000000000000e+02
+1.844070000000000e+03 +5.620000000000000e+02
+5.023130000000001e+02 +2.470000000000000e+02
+7.840039999999998e+02 +1.380000000000000e+02
+1.170460000000000e+03 +3.045000000000000e+02
+7.981350000000000e+02 +1.360000000000000e+02
+6.689310000000000e+02 +1.080000000000000e+02
+7.836039999999998e+02 +1.165000000000000e+02
+1.335480000000000e+03 +3.645000000000000e+02
+1.375960000000000e+03 +3.905000000000000e+02
+1.487860000000000e+03 +4.305000000000000e+02
+4.845500000000000e+02 +2.325000000000000e+02
+1.314730000000000e+03 +1.990000000000000e+02
+1.319770000000000e+03 +2.235000000000000e+02
+7.437410000000001e+02 +1.095000000000000e+02
+3.383819999999999e+02 +2.650000000000000e+01
+1.391420000000000e+03 +3.755000000000000e+02
+7.949710000000000e+02 +1.475000000000000e+02
+9.993430000000000e+02 +1.500000000000000e+02
+7.975410000000001e+02 +1.185000000000000e+02
+1.101300000000000e+03 +2.685000000000000e+02
+9.882850000000000e+02 +1.660000000000000e+02
+1.299180000000000e+03 +2.225000000000000e+02
+1.083380000000000e+03 +1.970000000000000e+02
+8.892489999999998e+02 +9.500000000000000e+01
+1.786740000000000e+03 +4.580000000000000e+02
+6.730720000000000e+02 +5.190000000000000e+02
+1.294630000000000e+03 +2.110000000000000e+02
+1.300890000000000e+03 +2.055000000000000e+02
+7.544390000000000e+02 +1.180000000000000e+02
+1.371440000000000e+03 +4.295000000000000e+02
+8.937400000000000e+02 +1.735000000000000e+02
+1.299700000000000e+03 +1.630000000000000e+02
+1.299160000000000e+03 +2.015000000000000e+02
+7.795580000000000e+02 +1.140000000000000e+02
+1.090690000000000e+03 +2.305000000000000e+02
+1.092100000000000e+03 +1.595000000000000e+02
+3.388640000000001e+02 +4.200000000000000e+01
+1.080530000000000e+03 +4.650000000000000e+02
+8.017320000000000e+02 +1.470000000000000e+02
+1.067960000000000e+03 +1.815000000000000e+02
+8.699710000000000e+02 +9.400000000000000e+01
+9.869790000000000e+02 +1.455000000000000e+02
+7.810770000000000e+02 +1.115000000000000e+02
+1.307680000000000e+03 +1.795000000000000e+02
+1.009020000000000e+03 +1.960000000000000e+02
+8.619080000000000e+02 +1.345000000000000e+02
+1.587560000000000e+03 +4.530000000000000e+02
+9.792140000000001e+02 +1.330000000000000e+02
+1.312590000000000e+03 +1.720000000000000e+02
+7.870599999999999e+02 +1.265000000000000e+02
+1.665810000000000e+03 +3.625000000000000e+02
+6.728930000000000e+02 +4.965000000000000e+02
+1.104540000000000e+03 +2.275000000000000e+02
+1.169710000000000e+03 +3.195000000000000e+02
+7.698120000000000e+02 +1.090000000000000e+02
+3.465760000000000e+02 +2.900000000000000e+01
+1.069490000000000e+03 +3.415000000000000e+02
+9.597290000000000e+02 +1.610000000000000e+02
+1.312630000000000e+03 +1.985000000000000e+02
+1.310820000000000e+03 +1.525000000000000e+02
+1.584250000000000e+03 +4.845000000000000e+02
+7.483380000000002e+02 +1.480000000000000e+02
+7.920560000000000e+02 +1.270000000000000e+02
+1.937140000000000e+03 +4.190000000000000e+02
+6.645269999999998e+02 +5.200000000000000e+02
+6.601360000000002e+02 +6.350000000000000e+01
+1.022300000000000e+03 +1.200000000000000e+02
+1.687150000000000e+03 +3.840000000000000e+02
+1.296940000000000e+03 +1.380000000000000e+02
+7.820400000000000e+02 +1.195000000000000e+02
+9.942180000000000e+02 +1.825000000000000e+02
+9.072870000000000e+02 +1.465000000000000e+02
+1.032410000000000e+03 +1.635000000000000e+02
+7.423910000000002e+02 +1.450000000000000e+02
+7.658880000000000e+02 +1.060000000000000e+02
+9.743150000000001e+02 +1.285000000000000e+02
+8.609570000000000e+02 +7.700000000000000e+01
+7.802980000000000e+02 +1.175000000000000e+02
+1.098510000000000e+03 +1.215000000000000e+02
+1.065110000000000e+03 +2.435000000000000e+02
+1.658610000000000e+03 +4.115000000000000e+02
+6.677430000000001e+02 +4.590000000000000e+02
+1.069370000000000e+03 +1.100000000000000e+02
+1.044720000000000e+03 +1.745000000000000e+02
+7.451660000000001e+02 +1.375000000000000e+02
+7.819889999999998e+02 +1.065000000000000e+02
+7.452020000000000e+02 +1.105000000000000e+02
+1.317470000000000e+03 +1.400000000000000e+02
+1.319750000000000e+03 +1.595000000000000e+02
+1.068790000000000e+03 +2.105000000000000e+02
+7.736310000000002e+02 +9.550000000000000e+01
+1.128200000000000e+03 +1.560000000000000e+02
+9.736330000000000e+02 +1.280000000000000e+02
+1.680080000000000e+03 +3.895000000000000e+02
+7.784660000000000e+02 +1.165000000000000e+02
+7.747669999999998e+02 +1.045000000000000e+02
+7.742130000000002e+02 +9.350000000000000e+01
+1.081280000000000e+03 +1.440000000000000e+02
+1.054730000000000e+03 +2.065000000000000e+02
+8.783290000000000e+02 +1.140000000000000e+02
+7.456860000000000e+02 +1.090000000000000e+02
+7.789100000000000e+02 +1.135000000000000e+02
+8.792970000000000e+02 +1.015000000000000e+02
+7.759060000000002e+02 +8.800000000000000e+01
+1.672250000000000e+03 +4.090000000000000e+02
+1.310080000000000e+03 +1.425000000000000e+02
+7.773630000000001e+02 +1.045000000000000e+02
+1.089600000000000e+03 +1.665000000000000e+02
+4.949320000000000e+02 +3.450000000000000e+01
+7.455770000000000e+02 +1.050000000000000e+02
+1.316740000000000e+03 +1.545000000000000e+02
+1.078380000000000e+03 +2.110000000000000e+02
+1.956160000000000e+03 +4.290000000000000e+02
+7.686180000000001e+02 +8.050000000000000e+01
+1.678030000000000e+03 +3.910000000000000e+02
+9.024870000000000e+02 +9.950000000000000e+01
+8.613099999999999e+02 +8.550000000000000e+01
+7.443110000000000e+02 +1.025000000000000e+02
+7.727950000000000e+02 +1.035000000000000e+02
+7.820350000000000e+02 +8.100000000000000e+01
+2.976640000000000e+02 +1.885000000000000e+02
+7.692739999999999e+02 +9.900000000000000e+01
+8.789220000000000e+02 +1.045000000000000e+02
+9.946600000000000e+02 +1.210000000000000e+02
+7.448439999999998e+02 +9.500000000000000e+01
+7.871410000000002e+02 +7.950000000000000e+01
+7.819480000000000e+02 +9.850000000000000e+01
+1.061000000000000e+03 +1.855000000000000e+02
+1.305290000000000e+03 +1.320000000000000e+02
+8.344010000000002e+02 +9.850000000000000e+01
+1.099860000000000e+03 +2.070000000000000e+02
+9.031670000000000e+02 +9.250000000000000e+01
+7.463270000000000e+02 +9.100000000000000e+01
+1.312290000000000e+03 +1.445000000000000e+02
+7.838880000000000e+02 +7.850000000000000e+01
+7.826210000000002e+02 +9.850000000000000e+01
+7.464260000000000e+02 +7.750000000000000e+01
+7.817780000000000e+02 +7.100000000000000e+01
+9.087770000000000e+02 +8.550000000000000e+01
+7.595630000000000e+02 +8.200000000000000e+01
+9.836480000000000e+02 +9.850000000000000e+01
+1.139230000000000e+03 +4.360000000000000e+02
+7.442139999999998e+02 +6.100000000000000e+01
+7.829050000000000e+02 +6.850000000000000e+01
+7.897370000000000e+02 +9.550000000000000e+01
+7.051330000000000e+02 +3.580000000000000e+02
+7.698860000000002e+02 +8.650000000000000e+01
+8.852360000000001e+02 +1.070000000000000e+02
+8.716089999999998e+02 +9.600000000000000e+01
+1.986940000000000e+03 +4.370000000000000e+02
+6.992010000000000e+02 +2.495000000000000e+02
+7.756569999999998e+02 +6.550000000000000e+01
+7.758489999999998e+02 +6.150000000000000e+01
+7.793489999999998e+02 +8.700000000000000e+01
+8.584900000000000e+02 +1.005000000000000e+02
+7.064240000000000e+02 +2.255000000000000e+02
+8.966080000000002e+02 +9.450000000000000e+01
+7.699930000000001e+02 +7.900000000000000e+01
+9.048300000000000e+02 +1.075000000000000e+02
+7.721950000000001e+02 +2.780000000000000e+02
+7.035440000000000e+02 +2.075000000000000e+02
+8.294900000000000e+02 +9.150000000000000e+01
+9.662089999999999e+02 +1.765000000000000e+02
+8.926700000000000e+02 +9.550000000000000e+01
+7.596160000000001e+02 +2.590000000000000e+02
+6.995410000000001e+02 +3.135000000000000e+02
+1.014400000000000e+03 +4.330000000000000e+02
+8.499950000000000e+02 +9.500000000000000e+01
+1.325740000000000e+03 +3.560000000000000e+02
+1.378190000000000e+03 +2.810000000000000e+02
+1.080970000000000e+03 +4.515000000000000e+02
+1.061800000000000e+03 +5.525000000000000e+02
+8.970380000000000e+02 +9.500000000000000e+01
+7.195750000000000e+02 +3.320000000000000e+02
+6.597339999999998e+02 +2.095000000000000e+02
+1.408150000000000e+03 +1.980000000000000e+02
+8.723070000000000e+02 +3.060000000000000e+02
+1.464490000000000e+03 +4.165000000000000e+02
+7.214069999999998e+02 +3.495000000000000e+02
+1.231020000000000e+03 +3.065000000000000e+02
+1.014160000000000e+03 +4.355000000000000e+02
+8.976530000000000e+02 +1.365000000000000e+02
+9.169340000000000e+02 +3.370000000000000e+02
+6.884299999999999e+02 +2.570000000000000e+02
+6.562250000000000e+02 +1.325000000000000e+02
+6.922110000000000e+02 +1.595000000000000e+02
+1.809030000000000e+03 +6.000000000000000e+02
+8.730430000000000e+02 +8.750000000000000e+01
+7.705050000000000e+02 +2.755000000000000e+02
+1.235600000000000e+03 +2.665000000000000e+02
+6.689560000000000e+02 +1.715000000000000e+02
+6.448110000000000e+02 +2.155000000000000e+02
+1.076760000000000e+03 +3.605000000000000e+02
+8.833099999999999e+02 +3.560000000000000e+02
+1.845720000000000e+03 +6.245000000000000e+02
+8.271669999999998e+02 +8.050000000000000e+01
+9.001020000000000e+02 +8.150000000000000e+01
+9.920240000000000e+02 +3.395000000000000e+02
+1.487300000000000e+03 +4.170000000000000e+02
+1.014580000000000e+03 +4.125000000000000e+02
+1.820640000000000e+03 +6.450000000000000e+02
+1.747480000000000e+03 +4.770000000000000e+02
+6.415630000000000e+02 +2.140000000000000e+02
+6.787060000000000e+02 +1.255000000000000e+02
+6.612389999999998e+02 +1.750000000000000e+02
+6.497690000000000e+02 +1.645000000000000e+02
+8.884510000000000e+02 +8.900000000000000e+01
+8.918960000000002e+02 +5.180000000000000e+02
+9.076930000000000e+02 +3.540000000000000e+02
+1.078290000000000e+03 +4.385000000000000e+02
+7.077150000000000e+02 +2.550000000000000e+02
+4.998880000000000e+02 +9.300000000000000e+01
+9.808420000000000e+02 +4.395000000000000e+02
+6.397100000000000e+02 +1.580000000000000e+02
+6.571650000000000e+02 +1.315000000000000e+02
+9.415990000000000e+02 +3.540000000000000e+02
+1.051630000000000e+03 +4.295000000000000e+02
+8.968330000000002e+02 +8.050000000000000e+01
+1.221580000000000e+03 +2.750000000000000e+02
+6.822950000000000e+02 +3.250000000000000e+02
+6.352980000000000e+02 +9.800000000000000e+01
+1.640340000000000e+03 +6.645000000000000e+02
+9.216110000000000e+02 +3.265000000000000e+02
+5.123430000000002e+02 +8.850000000000000e+01
+8.946750000000000e+02 +8.850000000000000e+01
+6.699800000000000e+02 +1.680000000000000e+02
+9.274700000000000e+02 +3.115000000000000e+02
+9.197190000000001e+02 +3.370000000000000e+02
+1.620320000000000e+03 +4.215000000000000e+02
+6.369030000000000e+02 +9.200000000000000e+01
+6.794989999999998e+02 +9.700000000000000e+01
+6.358640000000000e+02 +1.475000000000000e+02
+6.379400000000001e+02 +1.590000000000000e+02
+4.552430000000001e+02 +5.100000000000000e+01
+1.294930000000000e+03 +3.525000000000000e+02
+8.820050000000000e+02 +8.350000000000000e+01
+1.071700000000000e+03 +5.555000000000000e+02
+6.338000000000000e+02 +8.400000000000000e+01
+8.531500000000000e+02 +3.535000000000000e+02
+1.589060000000000e+03 +4.360000000000000e+02
+6.579970000000000e+02 +1.395000000000000e+02
+4.566190000000000e+02 +3.700000000000000e+01
+8.606110000000001e+02 +3.235000000000000e+02
+1.053370000000000e+03 +5.300000000000000e+02
+6.398910000000000e+02 +1.450000000000000e+02
+6.572350000000000e+02 +8.300000000000000e+01
+9.063830000000000e+02 +2.985000000000000e+02
+1.586980000000000e+03 +2.940000000000000e+02
+8.914360000000000e+02 +8.450000000000000e+01
+1.791190000000000e+03 +4.910000000000000e+02
+6.195440000000000e+02 +1.235000000000000e+02
+6.479600000000000e+02 +1.430000000000000e+02
+6.581230000000000e+02 +6.450000000000000e+01
+8.933560000000001e+02 +2.985000000000000e+02
+1.562510000000000e+03 +2.670000000000000e+02
+1.779060000000000e+03 +2.605000000000000e+02
+8.702200000000000e+02 +6.750000000000000e+01
+6.594820000000000e+02 +1.270000000000000e+02
+6.490169999999998e+02 +1.220000000000000e+02
+6.558819999999999e+02 +6.250000000000000e+01
+8.054910000000001e+02 +2.565000000000000e+02
+1.567930000000000e+03 +2.415000000000000e+02
+6.905610000000000e+02 +2.950000000000000e+02
+8.672660000000002e+02 +1.200000000000000e+02
+1.577380000000000e+03 +2.665000000000000e+02
+6.659630000000002e+02 +1.140000000000000e+02
+6.624140000000000e+02 +1.270000000000000e+02
+8.876120000000000e+02 +6.850000000000000e+01
+1.073790000000000e+03 +3.305000000000000e+02
+6.512580000000000e+02 +8.950000000000000e+01
+1.626330000000000e+03 +5.490000000000000e+02
+8.970230000000000e+02 +6.700000000000000e+01
+1.665500000000000e+03 +2.305000000000000e+02
+1.374150000000000e+03 +5.330000000000000e+02
+1.619480000000000e+03 +5.135000000000000e+02
+6.370780000000000e+02 +9.150000000000000e+01
+1.632690000000000e+03 +5.720000000000000e+02
+1.069730000000000e+03 +3.375000000000000e+02
+6.402370000000000e+02 +1.180000000000000e+02
+6.361860000000000e+02 +7.050000000000000e+01
+1.623080000000000e+03 +5.760000000000000e+02
+1.611610000000000e+03 +5.820000000000000e+02
+1.083000000000000e+03 +3.615000000000000e+02
+6.502859999999999e+02 +1.150000000000000e+02
+1.568760000000000e+03 +2.265000000000000e+02
+6.371690000000000e+02 +1.085000000000000e+02
+8.072460000000002e+02 +2.220000000000000e+02
+6.367700000000000e+02 +7.850000000000000e+01
+1.600310000000000e+03 +1.985000000000000e+02
+1.562020000000000e+03 +2.870000000000000e+02
+1.081370000000000e+03 +2.850000000000000e+02
+6.449200000000000e+02 +1.065000000000000e+02
+1.586250000000000e+03 +4.955000000000000e+02
+6.800089999999999e+02 +2.060000000000000e+02
+6.502040000000002e+02 +6.650000000000000e+01
+1.561590000000000e+03 +2.025000000000000e+02
+6.572230000000002e+02 +8.200000000000000e+01
+8.963310000000000e+02 +6.850000000000000e+01
+1.072710000000000e+03 +3.550000000000000e+02
+6.385069999999999e+02 +9.950000000000000e+01
+6.524780000000002e+02 +2.065000000000000e+02
+1.557180000000000e+03 +1.905000000000000e+02
+2.318550000000000e+03 +5.560000000000000e+02
+6.343030000000000e+02 +7.150000000000000e+01
+1.005140000000000e+03 +3.560000000000000e+02
+1.664100000000000e+03 +2.280000000000000e+02
+1.058060000000000e+03 +2.245000000000000e+02
+8.026160000000001e+02 +2.380000000000000e+02
+7.089789999999998e+02 +2.945000000000000e+02
+7.184900000000000e+02 +3.665000000000000e+02
+8.201130000000001e+02 +2.995000000000000e+02
+1.389770000000000e+03 +4.975000000000000e+02
+6.471200000000000e+02 +7.000000000000000e+01
+7.080860000000000e+02 +2.010000000000000e+02
+1.082550000000000e+03 +2.240000000000000e+02
+8.180160000000002e+02 +2.395000000000000e+02
+1.068790000000000e+03 +2.150000000000000e+02
+1.080050000000000e+03 +2.170000000000000e+02
+1.334000000000000e+03 +3.635000000000000e+02
+6.642539999999998e+02 +2.420000000000000e+02
+6.709839999999998e+02 +1.775000000000000e+02
+1.060480000000000e+03 +1.875000000000000e+02
+2.322630000000000e+03 +4.915000000000000e+02
+7.216360000000002e+02 +2.270000000000000e+02
+1.574680000000000e+03 +2.840000000000000e+02
+1.062380000000000e+03 +1.840000000000000e+02
+7.097110000000000e+02 +2.780000000000000e+02
+1.875830000000000e+03 +4.595000000000000e+02
+1.063610000000000e+03 +2.085000000000000e+02
+1.847210000000000e+03 +5.825000000000000e+02
+1.635190000000000e+03 +4.865000000000000e+02
+1.064990000000000e+03 +1.130000000000000e+02
+2.155180000000000e+03 +5.010000000000000e+02
+1.081460000000000e+03 +2.270000000000000e+02
+9.743840000000000e+02 +1.810000000000000e+02
+9.924220000000000e+02 +7.750000000000000e+01
+8.513980000000000e+02 +1.040000000000000e+02
+1.855520000000000e+03 +4.425000000000000e+02
+1.611140000000000e+03 +5.480000000000000e+02
+1.064850000000000e+03 +1.140000000000000e+02
+1.085130000000000e+03 +2.210000000000000e+02
+1.563060000000000e+03 +2.575000000000000e+02
+5.919200000000000e+02 +1.155000000000000e+02
+1.494010000000000e+03 +6.345000000000000e+02
+9.811910000000000e+02 +1.825000000000000e+02
+1.088460000000000e+03 +2.135000000000000e+02
+8.975700000000001e+02 +6.750000000000000e+01
+1.270440000000000e+03 +3.510000000000000e+02
+1.333040000000000e+03 +4.460000000000000e+02
+1.225610000000000e+03 +1.340000000000000e+02
+1.732080000000000e+03 +2.970000000000000e+02
+1.004130000000000e+03 +3.135000000000000e+02
+1.013640000000000e+03 +2.970000000000000e+02
+1.006060000000000e+03 +3.105000000000000e+02
+1.005820000000000e+03 +3.065000000000000e+02
+1.064250000000000e+03 +1.305000000000000e+02
+9.746570000000000e+02 +1.765000000000000e+02
+1.289020000000000e+03 +2.850000000000000e+02
+5.700990000000000e+02 +8.550000000000000e+01
+9.759130000000000e+02 +1.270000000000000e+02
+1.589410000000000e+03 +5.730000000000000e+02
+1.296200000000000e+03 +4.890000000000000e+02
+1.044920000000000e+03 +1.135000000000000e+02
+1.094820000000000e+03 +3.415000000000000e+02
+1.010620000000000e+03 +3.190000000000000e+02
+1.011580000000000e+03 +2.965000000000000e+02
+1.024030000000000e+03 +4.275000000000000e+02
+9.809670000000000e+02 +1.640000000000000e+02
+6.551820000000000e+02 +1.095000000000000e+02
+1.017820000000000e+03 +2.870000000000000e+02
+1.559960000000000e+03 +5.065000000000000e+02
+1.217010000000000e+03 +4.155000000000000e+02
+2.822470000000000e+02 +8.150000000000000e+01
+1.080270000000000e+03 +1.170000000000000e+02
+1.071180000000000e+03 +2.010000000000000e+02
+1.406790000000000e+03 +5.945000000000000e+02
+1.007330000000000e+03 +4.070000000000000e+02
+1.329890000000000e+03 +3.135000000000000e+02
+1.011760000000000e+03 +2.745000000000000e+02
+1.012270000000000e+03 +3.550000000000000e+02
+1.448820000000000e+03 +5.055000000000000e+02
+1.085590000000000e+03 +3.055000000000000e+02
+2.783490000000000e+02 +7.750000000000000e+01
+1.065970000000000e+03 +1.505000000000000e+02
+6.550369999999998e+02 +4.850000000000000e+01
+1.005790000000000e+03 +3.230000000000000e+02
+9.806260000000000e+02 +1.945000000000000e+02
+9.755450000000000e+02 +2.485000000000000e+02
+1.007830000000000e+03 +2.910000000000000e+02
+2.753950000000000e+02 +6.150000000000000e+01
+1.078410000000000e+03 +1.240000000000000e+02
+1.618640000000000e+03 +5.855000000000000e+02
+1.414760000000000e+03 +6.025000000000000e+02
+5.046700000000000e+02 +3.320000000000000e+02
+8.991020000000000e+02 +5.070000000000000e+02
+1.116810000000000e+03 +3.150000000000000e+02
+1.066730000000000e+03 +1.495000000000000e+02
+1.442600000000000e+03 +4.790000000000000e+02
+9.830010000000000e+02 +1.875000000000000e+02
+8.823200000000001e+02 +1.295000000000000e+02
+1.845440000000000e+03 +5.485000000000000e+02
+2.778890000000000e+02 +4.450000000000000e+01
+8.704290000000000e+02 +4.920000000000000e+02
+1.023050000000000e+03 +2.520000000000000e+02
+6.241070000000000e+02 +1.540000000000000e+02
+4.867180000000000e+02 +3.130000000000000e+02
+1.045770000000000e+03 +1.155000000000000e+02
+6.220150000000000e+02 +1.470000000000000e+02
+6.679839999999998e+02 +4.800000000000000e+01
+1.012410000000000e+03 +2.725000000000000e+02
+9.779820000000000e+02 +1.530000000000000e+02
+9.905380000000000e+02 +2.390000000000000e+02
+1.467120000000000e+03 +4.755000000000000e+02
+1.206740000000000e+03 +3.905000000000000e+02
+1.292160000000000e+03 +3.475000000000000e+02
+6.648150000000001e+02 +1.480000000000000e+02
+2.486760000000000e+03 +6.690000000000000e+02
+1.420100000000000e+03 +3.525000000000000e+02
+9.804870000000000e+02 +1.320000000000000e+02
+6.221849999999999e+02 +1.575000000000000e+02
+1.004280000000000e+03 +2.735000000000000e+02
+1.612420000000000e+03 +5.410000000000000e+02
+2.820590000000000e+02 +7.500000000000000e+01
+6.263790000000000e+02 +9.550000000000000e+01
+5.119470000000000e+02 +3.365000000000000e+02
+1.399710000000000e+03 +5.630000000000000e+02
+6.222380000000001e+02 +1.175000000000000e+02
+8.685020000000000e+02 +9.800000000000000e+01
+2.683440000000000e+03 +6.970000000000000e+02
+6.643950000000000e+02 +1.455000000000000e+02
+6.678330000000002e+02 +1.380000000000000e+02
+8.778900000000000e+02 +9.750000000000000e+01
+9.926340000000000e+02 +1.530000000000000e+02
+6.268480000000002e+02 +9.450000000000000e+01
+3.343980000000000e+02 +7.950000000000000e+01
+6.230680000000000e+02 +9.050000000000000e+01
+6.616050000000000e+02 +7.700000000000000e+01
+8.800530000000000e+02 +9.500000000000000e+01
+7.600210000000002e+02 +1.040000000000000e+02
+8.786080000000002e+02 +9.150000000000000e+01
+1.134930000000000e+03 +2.410000000000000e+02
+1.081890000000000e+03 +1.810000000000000e+02
+1.362780000000000e+03 +2.310000000000000e+02
+9.298880000000000e+02 +3.925000000000000e+02
+9.885220000000000e+02 +1.545000000000000e+02
+4.877800000000000e+02 +3.000000000000000e+02
+1.590520000000000e+03 +2.870000000000000e+02
+6.234270000000000e+02 +6.700000000000000e+01
+6.208480000000002e+02 +7.700000000000000e+01
+1.376980000000000e+03 +2.340000000000000e+02
+1.664880000000000e+03 +4.335000000000000e+02
+2.664660000000000e+03 +6.700000000000000e+02
+3.374390000000000e+02 +7.600000000000000e+01
+6.626870000000000e+02 +7.250000000000000e+01
+4.920720000000000e+02 +3.070000000000000e+02
+1.078330000000000e+03 +2.020000000000000e+02
+1.076430000000000e+03 +2.025000000000000e+02
+1.750400000000000e+03 +7.190000000000000e+02
+1.455560000000000e+03 +4.785000000000000e+02
+7.480520000000000e+02 +5.700000000000000e+01
+1.580630000000000e+03 +5.415000000000000e+02
+1.008620000000000e+03 +1.520000000000000e+02
+9.815690000000000e+02 +1.255000000000000e+02
+9.945480000000000e+02 +1.330000000000000e+02
+6.399970000000000e+02 +7.350000000000000e+01
+6.357710000000000e+02 +6.850000000000000e+01
+4.841650000000000e+02 +2.850000000000000e+02
+1.278520000000000e+03 +3.170000000000000e+02
+1.069670000000000e+03 +2.290000000000000e+02
+8.930119999999999e+02 +1.205000000000000e+02
+6.190670000000000e+02 +5.700000000000000e+01
+4.891080000000000e+02 +2.975000000000000e+02
+2.669520000000000e+03 +6.770000000000000e+02
+1.209940000000000e+03 +1.015000000000000e+02
+1.759230000000000e+03 +5.810000000000000e+02
+1.857380000000000e+03 +6.240000000000000e+02
+1.410570000000000e+03 +1.500000000000000e+02
+1.054600000000000e+03 +1.485000000000000e+02
+1.089760000000000e+03 +2.225000000000000e+02
+9.792150000000000e+02 +1.245000000000000e+02
+8.915820000000000e+02 +3.395000000000000e+02
+1.091970000000000e+03 +2.350000000000000e+02
+3.333490000000000e+02 +6.550000000000000e+01
+6.204209999999998e+02 +5.400000000000000e+01
+8.933720000000000e+02 +9.050000000000000e+01
+4.855590000000000e+02 +2.845000000000000e+02
+1.108280000000000e+03 +1.585000000000000e+02
+1.865840000000000e+03 +6.245000000000000e+02
+7.728600000000000e+02 +1.410000000000000e+02
+1.061110000000000e+03 +1.800000000000000e+02
+7.871500000000000e+02 +3.470000000000000e+02
+6.574430000000000e+02 +4.950000000000000e+01
+1.634420000000000e+03 +3.030000000000000e+02
+7.885290000000000e+02 +1.410000000000000e+02
+7.726489999999999e+02 +1.255000000000000e+02
+8.927589999999999e+02 +9.400000000000000e+01
+1.082510000000000e+03 +1.730000000000000e+02
+7.752530000000000e+02 +1.335000000000000e+02
+1.113670000000000e+03 +1.530000000000000e+02
+6.356170000000000e+02 +1.180000000000000e+02
+1.588440000000000e+03 +2.155000000000000e+02
+7.856260000000002e+02 +3.840000000000000e+02
+1.071270000000000e+03 +2.540000000000000e+02
+1.227160000000000e+03 +2.585000000000000e+02
+9.820700000000001e+02 +1.225000000000000e+02
+7.789370000000000e+02 +1.260000000000000e+02
+1.778940000000000e+03 +5.145000000000000e+02
+4.831750000000000e+02 +7.300000000000000e+01
+1.377250000000000e+03 +4.580000000000000e+02
+1.591620000000000e+03 +2.230000000000000e+02
+9.237230000000000e+02 +3.915000000000000e+02
+4.978180000000000e+02 +3.000000000000000e+02
+7.842300000000000e+02 +1.340000000000000e+02
+9.776540000000000e+02 +1.235000000000000e+02
+7.857669999999998e+02 +1.375000000000000e+02
+1.350620000000000e+03 +4.070000000000000e+02
+9.797340000000000e+02 +1.195000000000000e+02
+1.476300000000000e+03 +5.020000000000000e+02
+7.823839999999999e+02 +1.400000000000000e+02
+1.798820000000000e+03 +4.950000000000000e+02
+8.932289999999998e+02 +8.900000000000000e+01
+6.898470000000000e+02 +3.815000000000000e+02
+1.869740000000000e+03 +6.700000000000000e+02
+4.958330000000000e+02 +2.885000000000000e+02
+7.856849999999999e+02 +1.155000000000000e+02
+7.767719999999998e+02 +1.320000000000000e+02
+7.714349999999999e+02 +1.175000000000000e+02
+8.681810000000000e+02 +3.250000000000000e+02
+4.814680000000000e+02 +9.850000000000000e+01
+1.067490000000000e+03 +2.030000000000000e+02
+1.330640000000000e+03 +4.325000000000000e+02
+7.550440000000000e+02 +1.235000000000000e+02
+8.846480000000000e+02 +9.400000000000000e+01
+1.479970000000000e+03 +5.095000000000000e+02
+7.887600000000000e+02 +1.260000000000000e+02
+7.921000000000000e+02 +1.215000000000000e+02
+1.680780000000000e+03 +4.420000000000000e+02
+8.494860000000001e+02 +7.700000000000000e+01
+1.083580000000000e+03 +1.640000000000000e+02
+1.262920000000000e+03 +2.740000000000000e+02
+1.486280000000000e+03 +4.970000000000000e+02
+1.331130000000000e+03 +3.940000000000000e+02
+9.293240000000000e+02 +3.130000000000000e+02
+4.898690000000000e+02 +2.920000000000000e+02
+8.978480000000002e+02 +9.700000000000000e+01
+7.506510000000002e+02 +1.030000000000000e+02
+1.122080000000000e+03 +1.920000000000000e+02
+1.490830000000000e+03 +4.660000000000000e+02
+8.990810000000000e+02 +8.050000000000000e+01
+8.882350000000000e+02 +3.430000000000000e+02
+4.851450000000000e+02 +2.800000000000000e+02
+7.883810000000002e+02 +1.170000000000000e+02
+7.897669999999998e+02 +1.305000000000000e+02
+1.447860000000000e+03 +1.745000000000000e+02
+1.387410000000000e+03 +4.305000000000000e+02
+7.674110000000002e+02 +9.700000000000000e+01
+1.574970000000000e+03 +4.715000000000000e+02
+1.323760000000000e+03 +4.370000000000000e+02
+2.303800000000000e+03 +6.265000000000000e+02
+7.688270000000000e+02 +1.130000000000000e+02
+1.106890000000000e+03 +1.435000000000000e+02
+6.645069999999999e+02 +5.925000000000000e+02
+7.523550000000000e+02 +9.300000000000000e+01
+7.819510000000000e+02 +1.265000000000000e+02
+7.490310000000002e+02 +3.855000000000000e+02
+1.570120000000000e+03 +2.990000000000000e+02
+2.285540000000000e+03 +6.725000000000000e+02
+9.091660000000001e+02 +9.900000000000000e+01
+8.961500000000000e+02 +9.200000000000000e+01
+9.351220000000000e+02 +3.975000000000000e+02
+2.288310000000000e+03 +6.590000000000000e+02
+7.734989999999998e+02 +1.180000000000000e+02
+1.331520000000000e+03 +4.080000000000000e+02
+1.303500000000000e+03 +4.185000000000000e+02
+7.708600000000000e+02 +1.035000000000000e+02
+8.711430000000000e+02 +9.550000000000000e+01
+7.612539999999998e+02 +3.405000000000000e+02
+7.847160000000000e+02 +9.000000000000000e+01
+4.840740000000000e+02 +5.800000000000000e+01
+1.848500000000000e+03 +6.090000000000000e+02
+9.162560000000000e+02 +1.215000000000000e+02
+1.306430000000000e+03 +2.185000000000000e+02
+8.760230000000000e+02 +8.650000000000000e+01
+1.324270000000000e+03 +3.870000000000000e+02
+7.858980000000000e+02 +9.800000000000000e+01
+7.905730000000000e+02 +1.195000000000000e+02
+1.066090000000000e+03 +4.585000000000000e+02
+1.294980000000000e+03 +1.925000000000000e+02
+8.295260000000002e+02 +6.300000000000000e+01
+1.490780000000000e+03 +4.085000000000000e+02
+9.014610000000000e+02 +7.900000000000000e+01
+7.802000000000000e+02 +8.600000000000000e+01
+1.594770000000000e+03 +5.610000000000000e+02
+1.858860000000000e+03 +6.660000000000000e+02
+9.581900000000001e+02 +1.275000000000000e+02
+7.817189999999998e+02 +9.500000000000000e+01
+7.744240000000000e+02 +1.145000000000000e+02
+1.305670000000000e+03 +1.790000000000000e+02
+1.462350000000000e+03 +4.200000000000000e+02
+1.315860000000000e+03 +1.875000000000000e+02
+7.817780000000000e+02 +1.140000000000000e+02
+9.116750000000000e+02 +7.150000000000000e+01
+7.860480000000000e+02 +9.200000000000000e+01
+1.294830000000000e+03 +1.945000000000000e+02
+8.862960000000000e+02 +7.550000000000000e+01
+1.298210000000000e+03 +1.945000000000000e+02
+9.696079999999999e+02 +1.440000000000000e+02
+7.572710000000002e+02 +8.300000000000000e+01
+6.451870000000000e+02 +5.160000000000000e+02
+1.320640000000000e+03 +1.945000000000000e+02
+8.875950000000000e+02 +7.550000000000000e+01
+7.806450000000000e+02 +1.095000000000000e+02
+7.766790000000000e+02 +8.500000000000000e+01
+8.641189999999998e+02 +1.010000000000000e+02
+8.695549999999999e+02 +7.250000000000000e+01
+3.250050000000000e+02 +2.700000000000000e+01
+7.675219999999998e+02 +7.900000000000000e+01
+7.810360000000002e+02 +1.080000000000000e+02
+1.309880000000000e+03 +2.235000000000000e+02
+8.779540000000000e+02 +7.100000000000000e+01
+7.816180000000001e+02 +7.800000000000000e+01
+7.910930000000002e+02 +9.350000000000000e+01
+9.779950000000000e+02 +1.180000000000000e+02
+1.299220000000000e+03 +1.430000000000000e+02
+9.750280000000000e+02 +1.125000000000000e+02
+1.306830000000000e+03 +2.000000000000000e+02
+7.471630000000000e+02 +7.350000000000000e+01
+7.688489999999998e+02 +1.030000000000000e+02
+1.296910000000000e+03 +1.375000000000000e+02
+1.020450000000000e+03 +8.050000000000000e+02
+2.174620000000000e+03 +7.980000000000000e+02
+1.394640000000000e+03 +7.955000000000000e+02
+1.415470000000000e+03 +7.930000000000000e+02
+1.699310000000000e+03 +7.885000000000000e+02
+9.874700000000000e+02 +7.850000000000000e+02
+9.637840000000000e+02 +7.845000000000000e+02
+2.358990000000000e+03 +7.800000000000000e+02
+9.903240000000000e+02 +7.780000000000000e+02
+1.360800000000000e+03 +7.775000000000000e+02
+1.655260000000000e+03 +7.695000000000000e+02
+1.367260000000000e+03 +7.585000000000000e+02
+1.809830000000000e+03 +7.460000000000000e+02
+1.362030000000000e+03 +7.385000000000000e+02
+1.789210000000000e+03 +7.370000000000000e+02
+1.829130000000000e+02 +7.275000000000000e+02
+1.350870000000000e+03 +7.240000000000000e+02
+8.403130000000000e+02 +7.220000000000000e+02
+2.621520000000000e+03 +7.200000000000000e+02
+3.283630000000001e+02 +7.200000000000000e+02
+9.705810000000000e+02 +7.200000000000000e+02
+1.268130000000000e+03 +7.180000000000000e+02
+1.797950000000000e+03 +7.165000000000000e+02
+1.714840000000000e+02 +7.160000000000000e+02
+1.358360000000000e+03 +7.160000000000000e+02
+1.235910000000000e+03 +7.155000000000000e+02
+1.446990000000000e+03 +7.130000000000000e+02
+1.727790000000000e+03 +7.020000000000000e+02
+9.973840000000000e+02 +7.015000000000000e+02
+1.325460000000000e+03 +6.960000000000000e+02
+1.476700000000000e+03 +6.885000000000000e+02
+1.181600000000000e+03 +6.835000000000000e+02
+8.411560000000002e+02 +6.825000000000000e+02
+1.538320000000000e+03 +6.740000000000000e+02
+1.825580000000000e+03 +6.725000000000000e+02
+8.380599999999999e+02 +6.695000000000000e+02
+1.234830000000000e+03 +6.670000000000000e+02
+1.804490000000000e+03 +6.665000000000000e+02
+1.230570000000000e+03 +6.650000000000000e+02
+1.035990000000000e+03 +6.625000000000000e+02
+3.029180000000000e+02 +6.610000000000000e+02
+1.915630000000000e+03 +6.600000000000000e+02
+1.582980000000000e+03 +6.590000000000000e+02
+1.341550000000000e+03 +6.580000000000000e+02
+1.846260000000000e+03 +6.560000000000000e+02
+1.835020000000000e+03 +6.545000000000000e+02
+8.382160000000000e+02 +6.540000000000000e+02
+1.075590000000000e+03 +6.465000000000000e+02
+9.359710000000000e+02 +6.445000000000000e+02
+9.836330000000000e+02 +6.440000000000000e+02
+2.722150000000000e+02 +6.405000000000000e+02
+9.878070000000000e+02 +6.405000000000000e+02
+1.197420000000000e+03 +6.380000000000000e+02
+9.662440000000000e+02 +6.360000000000000e+02
+9.511480000000000e+02 +6.355000000000000e+02
+9.206860000000000e+02 +6.350000000000000e+02
+1.041490000000000e+03 +6.285000000000000e+02
+9.164100000000000e+02 +6.235000000000000e+02
+1.025440000000000e+03 +6.235000000000000e+02
+1.485660000000000e+03 +6.230000000000000e+02
+1.419000000000000e+03 +6.230000000000000e+02
+1.208660000000000e+03 +6.210000000000000e+02
+9.509860000000000e+02 +6.205000000000000e+02
+9.277120000000000e+02 +6.170000000000000e+02
+1.331510000000000e+03 +6.160000000000000e+02
+8.611910000000000e+02 +6.160000000000000e+02
+2.901320000000000e+02 +6.160000000000000e+02
+1.865600000000000e+03 +6.150000000000000e+02
+9.188060000000000e+02 +6.120000000000000e+02
+8.358180000000000e+02 +6.085000000000000e+02
+9.010800000000000e+02 +6.085000000000000e+02
+8.972930000000000e+02 +6.080000000000000e+02
+1.247560000000000e+03 +6.075000000000000e+02
+7.816780000000000e+02 +6.075000000000000e+02
+9.221470000000000e+02 +6.070000000000000e+02
+1.166420000000000e+03 +6.045000000000000e+02
+6.741630000000000e+02 +6.040000000000000e+02
+1.310110000000000e+03 +6.030000000000000e+02
+1.538580000000000e+03 +6.020000000000000e+02
+8.640530000000000e+02 +6.005000000000000e+02
+1.896150000000000e+03 +6.000000000000000e+02
+2.841880000000000e+02 +5.970000000000000e+02
+2.585660000000000e+02 +5.965000000000000e+02
+1.431180000000000e+03 +5.955000000000000e+02
+6.522430000000001e+02 +5.945000000000000e+02
+1.849430000000000e+03 +5.935000000000000e+02
+2.722010000000000e+02 +5.925000000000000e+02
+1.370630000000000e+03 +5.925000000000000e+02
+1.864220000000000e+03 +5.910000000000000e+02
+1.102730000000000e+03 +5.900000000000000e+02
+6.696350000000000e+02 +5.895000000000000e+02
+1.279040000000000e+03 +5.895000000000000e+02
+1.300090000000000e+03 +5.890000000000000e+02
+1.894520000000000e+03 +5.870000000000000e+02
+7.828420000000000e+02 +5.865000000000000e+02
+1.285720000000000e+03 +5.850000000000000e+02
+9.953500000000000e+02 +5.840000000000000e+02
+9.021920000000000e+02 +5.815000000000000e+02
+1.481420000000000e+03 +5.815000000000000e+02
+1.091760000000000e+03 +5.805000000000000e+02
+1.880570000000000e+03 +5.800000000000000e+02
+8.520169999999998e+02 +5.790000000000000e+02
+7.703370000000000e+02 +5.785000000000000e+02
+8.607830000000000e+02 +5.770000000000000e+02
+1.273030000000000e+03 +5.770000000000000e+02
+8.046730000000000e+02 +5.750000000000000e+02
+2.654310000000000e+02 +5.750000000000000e+02
+8.128700000000000e+02 +5.740000000000000e+02
+1.075850000000000e+03 +5.740000000000000e+02
+1.267240000000000e+03 +5.730000000000000e+02
+8.083489999999998e+02 +5.710000000000000e+02
+1.196520000000000e+03 +5.710000000000000e+02
+7.733339999999999e+02 +5.705000000000000e+02
+1.688430000000000e+03 +5.705000000000000e+02
+1.259170000000000e+03 +5.685000000000000e+02
+8.001790000000000e+02 +5.675000000000000e+02
+1.292170000000000e+03 +5.670000000000000e+02
+7.711389999999999e+02 +5.660000000000000e+02
+1.300890000000000e+03 +5.660000000000000e+02
+1.285210000000000e+03 +5.650000000000000e+02
+7.847370000000000e+02 +5.650000000000000e+02
+7.891310000000002e+02 +5.645000000000000e+02
+1.151220000000000e+03 +5.640000000000000e+02
+1.047530000000000e+03 +5.640000000000000e+02
+1.265880000000000e+03 +5.635000000000000e+02
+7.735790000000000e+02 +5.625000000000000e+02
+7.357410000000001e+02 +5.625000000000000e+02
+7.878620000000000e+02 +5.615000000000000e+02
+2.603160000000000e+02 +5.605000000000000e+02
+1.849670000000000e+03 +5.600000000000000e+02
+2.525920000000000e+02 +5.595000000000000e+02
+7.952470000000000e+02 +5.585000000000000e+02
+1.870290000000000e+03 +5.580000000000000e+02
+1.070460000000000e+03 +5.575000000000000e+02
+7.793850000000000e+02 +5.565000000000000e+02
+1.482070000000000e+03 +5.560000000000000e+02
+5.035680000000000e+02 +5.540000000000000e+02
+7.925760000000000e+01 +5.530000000000000e+02
+1.254900000000000e+03 +5.530000000000000e+02
+7.563320000000000e+02 +5.525000000000000e+02
+1.090080000000000e+03 +5.525000000000000e+02
+4.932260000000000e+02 +5.520000000000000e+02
+7.860080000000000e+02 +5.520000000000000e+02
+8.185110000000002e+02 +5.515000000000000e+02
+8.004970000000000e+02 +5.510000000000000e+02
+8.464480000000000e+02 +5.505000000000000e+02
+1.294390000000000e+03 +5.500000000000000e+02
+6.935139999999999e+02 +5.495000000000000e+02
+5.514840000000000e+02 +5.480000000000000e+02
+1.187400000000000e+03 +5.470000000000000e+02
+1.142480000000000e+03 +5.455000000000000e+02
+6.812360000000001e+02 +5.450000000000000e+02
+7.910910000000000e+02 +5.445000000000000e+02
+9.241350000000000e+02 +5.405000000000000e+02
+6.907589999999999e+02 +5.400000000000000e+02
+1.804930000000000e+03 +5.400000000000000e+02
+2.410080000000000e+02 +5.400000000000000e+02
+1.931210000000000e+03 +5.400000000000000e+02
+9.220740000000000e+02 +5.400000000000000e+02
+7.027760000000002e+02 +5.380000000000000e+02
+7.209130000000000e+02 +5.365000000000000e+02
+6.976720000000000e+02 +5.365000000000000e+02
+7.072539999999998e+02 +5.360000000000000e+02
+9.783440000000001e+02 +5.360000000000000e+02
+2.583200000000000e+02 +5.350000000000000e+02
+2.168880000000000e+02 +5.335000000000000e+02
+7.699090000000000e+02 +5.320000000000000e+02
+7.291089999999998e+02 +5.315000000000000e+02
+6.778480000000002e+02 +5.305000000000000e+02
+6.958889999999999e+02 +5.300000000000000e+02
+6.737010000000000e+02 +5.295000000000000e+02
+7.567869999999998e+02 +5.285000000000000e+02
+2.328130000000000e+02 +5.280000000000000e+02
+9.319349999999999e+02 +5.255000000000000e+02
+1.114290000000000e+03 +5.245000000000000e+02
+7.142739999999999e+02 +5.235000000000000e+02
+7.518170000000000e+02 +5.230000000000000e+02
+7.122350000000000e+02 +5.225000000000000e+02
+2.519970000000000e+02 +5.225000000000000e+02
+1.115800000000000e+03 +5.220000000000000e+02
+1.586350000000000e+03 +5.220000000000000e+02
+1.163740000000000e+03 +5.220000000000000e+02
+7.679630000000002e+02 +5.210000000000000e+02
+7.521669999999998e+02 +5.210000000000000e+02
+4.613470000000000e+02 +5.210000000000000e+02
+6.469119999999998e+02 +5.205000000000000e+02
+2.543990000000000e+02 +5.200000000000000e+02
+7.480419999999998e+02 +5.195000000000000e+02
+7.378680000000001e+02 +5.195000000000000e+02
+1.231270000000000e+03 +5.195000000000000e+02
+4.845190000000000e+02 +5.185000000000000e+02
+1.181750000000000e+03 +5.185000000000000e+02
+6.664980000000000e+02 +5.175000000000000e+02
+7.565760000000000e+02 +5.170000000000000e+02
+4.665940000000000e+02 +5.165000000000000e+02
+4.602380000000001e+02 +5.155000000000000e+02
+9.667180000000000e+02 +5.155000000000000e+02
+8.507930000000000e+02 +5.150000000000000e+02
+9.988400000000000e+02 +5.150000000000000e+02
+6.438910000000000e+02 +5.145000000000000e+02
+9.008360000000000e+02 +5.135000000000000e+02
+7.670750000000000e+02 +5.130000000000000e+02
+7.652160000000000e+02 +5.120000000000000e+02
+1.342270000000000e+03 +5.120000000000000e+02
+6.899430000000000e+02 +5.115000000000000e+02
+8.138160000000000e+02 +5.110000000000000e+02
+1.210190000000000e+03 +5.110000000000000e+02
+7.592619999999999e+02 +5.085000000000000e+02
+4.475740000000000e+02 +5.080000000000000e+02
+8.907470000000000e+02 +5.080000000000000e+02
+1.156600000000000e+03 +5.065000000000000e+02
+2.185480000000000e+02 +5.065000000000000e+02
+7.434340000000000e+02 +5.055000000000000e+02
+2.394020000000000e+02 +5.045000000000000e+02
+1.340860000000000e+03 +5.045000000000000e+02
+7.358580000000002e+02 +5.030000000000000e+02
+4.336640000000000e+02 +5.030000000000000e+02
+9.475560000000000e+02 +5.025000000000000e+02
+1.098050000000000e+03 +5.025000000000000e+02
+2.321540000000000e+02 +5.020000000000000e+02
+6.851750000000000e+02 +4.995000000000000e+02
+4.119740000000000e+02 +4.990000000000000e+02
+3.994740000000000e+02 +4.960000000000000e+02
+9.489860000000000e+02 +4.955000000000000e+02
+6.044930000000001e+02 +4.955000000000000e+02
+6.973020000000000e+02 +4.950000000000000e+02
+8.556020000000000e+02 +4.945000000000000e+02
+6.035860000000000e+02 +4.940000000000000e+02
+7.167850000000000e+02 +4.935000000000000e+02
+6.053750000000000e+02 +4.935000000000000e+02
+7.100119999999999e+02 +4.930000000000000e+02
+3.987690000000000e+02 +4.920000000000000e+02
+3.370390000000000e+02 +4.920000000000000e+02
+3.724370000000000e+02 +4.915000000000000e+02
+9.894290000000000e+02 +4.905000000000000e+02
+2.110120000000000e+02 +4.900000000000000e+02
+7.106239999999998e+02 +4.890000000000000e+02
+8.669270000000000e+02 +4.890000000000000e+02
+1.066710000000000e+03 +4.890000000000000e+02
+5.830050000000000e+02 +4.885000000000000e+02
+6.346040000000000e+02 +4.875000000000000e+02
+1.091920000000000e+03 +4.875000000000000e+02
+8.570269999999998e+02 +4.860000000000000e+02
+4.731080000000000e+02 +4.855000000000000e+02
+7.749810000000001e+02 +4.850000000000000e+02
+1.579520000000000e+03 +4.850000000000000e+02
+3.576110000000000e+02 +4.845000000000000e+02
+7.734530000000000e+02 +4.845000000000000e+02
+3.603360000000000e+02 +4.840000000000000e+02
+5.722070000000000e+02 +4.830000000000000e+02
+6.225190000000000e+02 +4.830000000000000e+02
+6.452790000000000e+02 +4.825000000000000e+02
+9.972770000000000e+02 +4.820000000000000e+02
+8.170530000000000e+02 +4.815000000000000e+02
+9.320430000000000e+02 +4.815000000000000e+02
+1.030080000000000e+03 +4.810000000000000e+02
+5.555710000000000e+02 +4.805000000000000e+02
+9.969720000000000e+02 +4.805000000000000e+02
+1.418140000000000e+03 +4.800000000000000e+02
+9.425100000000000e+02 +4.800000000000000e+02
+6.418090000000000e+02 +4.795000000000000e+02
+3.772340000000000e+02 +4.790000000000000e+02
+5.485419999999998e+02 +4.790000000000000e+02
+2.028690000000000e+02 +4.790000000000000e+02
+3.849690000000000e+02 +4.780000000000000e+02
+5.132840000000000e+02 +4.780000000000000e+02
+9.255250000000000e+02 +4.780000000000000e+02
+7.108410000000000e+02 +4.775000000000000e+02
+7.731369999999999e+02 +4.775000000000000e+02
+3.532680000000001e+02 +4.770000000000000e+02
+9.856470000000000e+02 +4.770000000000000e+02
+8.444750000000000e+02 +4.770000000000000e+02
+9.655820000000000e+02 +4.765000000000000e+02
+3.692870000000000e+02 +4.760000000000000e+02
+3.689780000000000e+02 +4.755000000000000e+02
+6.474500000000000e+02 +4.755000000000000e+02
+6.537180000000002e+02 +4.755000000000000e+02
+9.652540000000000e+02 +4.750000000000000e+02
+7.053210000000000e+02 +4.745000000000000e+02
+1.004270000000000e+03 +4.745000000000000e+02
+7.639330000000000e+02 +4.745000000000000e+02
+5.166940000000000e+02 +4.740000000000000e+02
+5.078040000000000e+02 +4.740000000000000e+02
+3.974870000000000e+02 +4.730000000000000e+02
+5.200040000000000e+02 +4.730000000000000e+02
+8.615210000000002e+02 +4.730000000000000e+02
+3.310740000000000e+02 +4.725000000000000e+02
+5.209150000000000e+02 +4.720000000000000e+02
+3.556260000000000e+02 +4.715000000000000e+02
+1.309510000000000e+03 +4.710000000000000e+02
+6.598720000000000e+02 +4.705000000000000e+02
+3.432680000000001e+02 +4.700000000000000e+02
+8.677660000000002e+02 +4.685000000000000e+02
+9.425270000000000e+02 +4.685000000000000e+02
+3.579010000000000e+02 +4.680000000000000e+02
+6.515180000000000e+02 +4.670000000000000e+02
+6.444780000000002e+02 +4.670000000000000e+02
+9.013900000000000e+02 +4.660000000000000e+02
+7.943620000000000e+02 +4.655000000000000e+02
+6.895239999999999e+02 +4.650000000000000e+02
+3.260870000000000e+02 +4.640000000000000e+02
+5.797909999999998e+02 +4.640000000000000e+02
+6.943310000000000e+02 +4.610000000000000e+02
+9.348760000000000e+02 +4.635000000000000e+02
+9.427120000000000e+02 +4.630000000000000e+02
+5.224790000000000e+02 +4.625000000000000e+02
+5.015290000000000e+02 +4.620000000000000e+02
+4.720980000000000e+02 +4.610000000000000e+02
+7.803300000000000e+02 +4.605000000000000e+02
+1.004510000000000e+03 +4.605000000000000e+02
+9.297240000000000e+02 +4.600000000000000e+02
+6.818090000000000e+02 +4.575000000000000e+02
+1.053190000000000e+03 +4.590000000000000e+02
+3.409060000000000e+02 +4.540000000000000e+02
+7.831840000000000e+02 +4.575000000000000e+02
+5.461600000000000e+02 +4.565000000000000e+02
+7.547639999999999e+02 +4.560000000000000e+02
+6.812830000000000e+02 +4.540000000000000e+02
+3.340600000000000e+02 +4.535000000000000e+02
+6.735839999999999e+02 +4.530000000000000e+02
+5.204670000000000e+02 +4.530000000000000e+02
+7.552730000000000e+02 +4.530000000000000e+02
+6.565830000000002e+02 +4.490000000000000e+02
+9.033560000000000e+02 +4.525000000000000e+02
+7.439950000000000e+02 +4.520000000000000e+02
+4.722760000000000e+02 +4.515000000000000e+02
+1.053110000000000e+03 +4.515000000000000e+02
+5.216990000000002e+02 +4.510000000000000e+02
+7.373700000000000e+02 +4.495000000000000e+02
+6.691920000000000e+02 +4.490000000000000e+02
+7.965369999999998e+02 +4.485000000000000e+02
+3.470790000000000e+02 +4.480000000000000e+02
+1.250350000000000e+03 +4.480000000000000e+02
+7.257250000000000e+02 +4.475000000000000e+02
+1.016720000000000e+03 +4.475000000000000e+02
+7.078700000000000e+02 +4.470000000000000e+02
+9.279640000000001e+02 +4.470000000000000e+02
+5.084970000000000e+02 +4.465000000000000e+02
+1.363510000000000e+03 +4.465000000000000e+02
+9.104240000000000e+02 +4.465000000000000e+02
+6.512250000000000e+02 +4.460000000000000e+02
+5.142270000000000e+02 +4.450000000000000e+02
+6.213460000000000e+02 +4.440000000000000e+02
+4.676730000000000e+02 +4.435000000000000e+02
+4.411210000000000e+02 +4.430000000000000e+02
+7.083520000000000e+02 +4.430000000000000e+02
+7.236860000000000e+02 +4.425000000000000e+02
+8.719739999999998e+02 +4.420000000000000e+02
+4.845780000000000e+02 +4.410000000000000e+02
+1.210740000000000e+03 +4.400000000000000e+02
+7.044639999999998e+02 +4.400000000000000e+02
+9.676470000000000e+02 +4.395000000000000e+02
+3.293630000000001e+02 +4.370000000000000e+02
+7.683220000000000e+02 +4.365000000000000e+02
+6.218610000000000e+02 +4.360000000000000e+02
+7.377639999999999e+02 +4.355000000000000e+02
+5.625549999999999e+02 +4.350000000000000e+02
+9.015560000000000e+02 +4.350000000000000e+02
+1.219510000000000e+03 +4.345000000000000e+02
+3.690850000000000e+02 +4.340000000000000e+02
+1.067090000000000e+03 +4.340000000000000e+02
+6.840860000000000e+02 +4.330000000000000e+02
+3.253750000000000e+02 +4.320000000000000e+02
+8.203500000000000e+02 +4.315000000000000e+02
+9.004580000000002e+02 +4.315000000000000e+02
+9.335660000000000e+02 +4.310000000000000e+02
+5.295090000000000e+02 +4.300000000000000e+02
+7.318049999999999e+02 +4.300000000000000e+02
+7.874260000000000e+02 +4.290000000000000e+02
+9.140800000000000e+02 +4.290000000000000e+02
+6.007240000000000e+02 +4.285000000000000e+02
+1.302880000000000e+03 +4.275000000000000e+02
+6.172890000000000e+02 +4.270000000000000e+02
+6.397480000000000e+02 +4.250000000000000e+02
+7.485780000000000e+02 +4.245000000000000e+02
+1.241700000000000e+02 +4.245000000000000e+02
+3.074270000000000e+02 +4.240000000000000e+02
+9.469650000000000e+02 +4.235000000000000e+02
+8.945690000000000e+02 +4.235000000000000e+02
+3.056180000000000e+02 +4.215000000000000e+02
+6.013400000000000e+02 +4.215000000000000e+02
+5.471080000000002e+02 +4.210000000000000e+02
+1.164380000000000e+03 +4.210000000000000e+02
+6.386709999999998e+02 +4.200000000000000e+02
+2.919350000000000e+02 +4.200000000000000e+02
+1.421390000000000e+03 +4.200000000000000e+02
+1.810920000000000e+03 +4.200000000000000e+02
+4.926250000000000e+02 +4.190000000000000e+02
+7.700330000000000e+02 +4.185000000000000e+02
+8.888720000000000e+02 +4.185000000000000e+02
+6.543520000000000e+02 +4.180000000000000e+02
+9.182700000000000e+02 +4.175000000000000e+02
+7.525939999999998e+02 +4.170000000000000e+02
+1.278490000000000e+03 +4.160000000000000e+02
+8.642189999999998e+02 +4.155000000000000e+02
+1.031740000000000e+03 +4.155000000000000e+02
+7.645230000000000e+02 +4.155000000000000e+02
+7.070180000000000e+02 +4.150000000000000e+02
+1.053600000000000e+03 +4.145000000000000e+02
+5.247390000000000e+02 +4.135000000000000e+02
+6.242480000000000e+02 +4.135000000000000e+02
+7.612730000000000e+02 +4.130000000000000e+02
+5.644410000000000e+02 +4.125000000000000e+02
+2.855340000000000e+02 +4.120000000000000e+02
+6.011910000000000e+02 +4.115000000000000e+02
+3.267320000000000e+02 +4.105000000000000e+02
+2.848990000000000e+02 +4.085000000000000e+02
+9.680990000000000e+02 +4.075000000000000e+02
+1.926670000000000e+02 +4.075000000000000e+02
+2.705740000000000e+02 +4.060000000000000e+02
+7.264080000000000e+02 +4.060000000000000e+02
+8.040490000000000e+01 +4.045000000000000e+02
+1.874710000000000e+02 +4.040000000000000e+02
+1.178580000000000e+02 +4.035000000000000e+02
+2.320450000000000e+02 +4.025000000000000e+02
+9.398430000000000e+02 +4.025000000000000e+02
+1.340610000000000e+02 +4.015000000000000e+02
+7.230920000000000e+02 +4.015000000000000e+02
+5.771210000000000e+02 +4.005000000000000e+02
+2.026470000000000e+02 +4.000000000000000e+02
+6.790150000000000e+02 +4.000000000000000e+02
+7.737239999999998e+02 +3.995000000000000e+02
+8.110419999999998e+02 +3.990000000000000e+02
+4.747360000000000e+02 +3.985000000000000e+02
+9.228690000000000e+02 +3.985000000000000e+02
+7.636920000000000e+02 +3.985000000000000e+02
+1.075200000000000e+02 +3.975000000000000e+02
+7.695150000000000e+02 +3.975000000000000e+02
+5.663500000000000e+01 +3.975000000000000e+02
+6.860280000000000e+02 +3.970000000000000e+02
+2.256130000000000e+02 +3.970000000000000e+02
+1.611750000000000e+02 +3.965000000000000e+02
+7.501980000000000e+02 +3.965000000000000e+02
+1.004220000000000e+02 +3.950000000000000e+02
+9.225970000000000e+02 +3.950000000000000e+02
+1.043190000000000e+03 +3.940000000000000e+02
+6.026100000000000e+02 +3.935000000000000e+02
+3.038180000000000e+01 +3.930000000000000e+02
+9.430839999999999e+02 +3.930000000000000e+02
+6.310610000000000e+02 +3.925000000000000e+02
+2.004960000000000e+01 +3.910000000000000e+02
+1.672030000000000e+02 +3.910000000000000e+02
+8.678860000000002e+02 +3.910000000000000e+02
+1.111410000000000e+03 +3.910000000000000e+02
+1.305700000000000e+02 +3.905000000000000e+02
+1.752030000000000e+02 +3.900000000000000e+02
+3.583270000000000e+02 +3.895000000000000e+02
+2.184000000000000e+02 +3.895000000000000e+02
+5.371080000000002e+02 +3.890000000000000e+02
+1.089940000000000e+03 +3.885000000000000e+02
+6.587680000000000e+02 +3.885000000000000e+02
+8.964789999999998e+02 +3.885000000000000e+02
+7.736900000000001e+02 +3.880000000000000e+02
+1.420170000000000e+01 +3.875000000000000e+02
+2.162840000000000e+02 +3.870000000000000e+02
+8.273080000000000e+02 +3.870000000000000e+02
+6.580380000000000e+02 +3.870000000000000e+02
+8.317960000000000e+02 +3.865000000000000e+02
+7.465660000000000e+02 +3.865000000000000e+02
+9.086290000000000e+02 +3.860000000000000e+02
+6.566830000000000e+02 +3.850000000000000e+02
+4.105240000000000e+01 +3.850000000000000e+02
+5.459280000000000e+01 +3.845000000000000e+02
+8.342230000000002e+02 +3.845000000000000e+02
+8.205330000000000e+02 +3.845000000000000e+02
+3.765330000000000e+01 +3.840000000000000e+02
+5.740069999999999e+02 +3.835000000000000e+02
+7.072320000000000e+02 +3.835000000000000e+02
+6.250520000000000e+02 +3.830000000000000e+02
+1.139760000000000e+03 +3.815000000000000e+02
+9.149680000000000e+02 +3.815000000000000e+02
+1.209950000000000e+01 +3.810000000000000e+02
+6.849130000000000e+02 +3.810000000000000e+02
+2.398020000000000e+01 +3.805000000000000e+02
+9.137590000000000e+02 +3.805000000000000e+02
+8.631150000000000e+02 +3.805000000000000e+02
+1.867090000000000e+01 +3.800000000000000e+02
+8.544589999999999e+02 +3.795000000000000e+02
+1.318860000000000e+03 +3.785000000000000e+02
+8.727450000000000e+02 +3.780000000000000e+02
+8.613500000000000e+02 +3.780000000000000e+02
+1.112030000000000e+03 +3.775000000000000e+02
+6.270040000000000e+02 +3.770000000000000e+02
+8.519430000000000e+02 +3.770000000000000e+02
+6.635030000000000e+02 +3.765000000000000e+02
+2.328670000000000e+02 +3.760000000000000e+02
+2.239760000000000e+02 +3.760000000000000e+02
+1.172010000000000e+03 +3.760000000000000e+02
+5.475180000000000e+02 +3.750000000000000e+02
+6.968230000000000e+02 +3.750000000000000e+02
+2.552480000000000e+01 +3.735000000000000e+02
+6.121280000000000e+02 +3.730000000000000e+02
+5.977890000000000e+02 +3.725000000000000e+02
+1.754970000000000e+02 +3.125000000000000e+02
+6.978969999999998e+02 +3.720000000000000e+02
+7.907760000000002e+02 +3.715000000000000e+02
+1.662710000000000e+02 +2.920000000000000e+02
+1.545500000000000e+02 +3.705000000000000e+02
+8.847940000000000e+02 +3.705000000000000e+02
+1.515170000000000e+02 +3.700000000000000e+02
+8.398639999999998e+02 +3.700000000000000e+02
+7.331619999999998e+02 +3.695000000000000e+02
+1.068030000000000e+02 +1.925000000000000e+02
+6.333560000000000e+02 +3.680000000000000e+02
+7.344250000000000e+02 +3.675000000000000e+02
+1.479710000000000e+02 +3.675000000000000e+02
+8.538660000000001e+02 +3.675000000000000e+02
+1.663920000000000e+02 +3.670000000000000e+02
+8.322389999999998e+02 +3.665000000000000e+02
+6.555350000000000e+02 +3.655000000000000e+02
+1.155210000000000e+01 +2.295000000000000e+02
+1.747460000000000e+02 +3.650000000000000e+02
+6.505440000000000e+02 +3.645000000000000e+02
+5.952580000000000e+02 +3.645000000000000e+02
+7.130419999999998e+02 +3.640000000000000e+02
+7.170760000000000e+02 +3.635000000000000e+02
+2.577220000000000e+02 +3.595000000000000e+02
+5.578500000000000e+02 +3.625000000000000e+02
+1.493040000000000e+02 +3.625000000000000e+02
+3.019980000000000e+01 +3.620000000000000e+02
+6.827120000000000e+02 +3.620000000000000e+02
+1.949910000000000e+01 +3.620000000000000e+02
+2.386600000000000e+02 +3.375000000000000e+02
+7.998989999999999e+02 +3.600000000000000e+02
+1.387760000000000e+03 +3.600000000000000e+02
+1.382070000000000e+02 +3.600000000000000e+02
+1.293960000000000e+03 +3.600000000000000e+02
+1.602460000000000e+03 +3.600000000000000e+02
+1.905020000000000e+03 +3.600000000000000e+02
+1.309860000000000e+03 +3.600000000000000e+02
+1.345880000000000e+03 +3.600000000000000e+02
+3.283090000000000e+02 +3.600000000000000e+02
+2.266570000000000e+01 +3.595000000000000e+02
+6.085549999999999e+02 +3.590000000000000e+02
+3.117710000000000e+01 +3.580000000000000e+02
+1.121370000000000e+03 +3.580000000000000e+02
+6.517540000000000e+02 +3.570000000000000e+02
+1.136940000000000e+03 +3.565000000000000e+02
+4.531090000000000e+02 +3.555000000000000e+02
+1.437320000000000e+01 +1.975000000000000e+02
+5.657260000000000e+02 +3.550000000000000e+02
+5.952950000000000e+02 +3.550000000000000e+02
+6.990030000000000e+02 +3.545000000000000e+02
+8.369950000000000e+02 +3.540000000000000e+02
+1.084310000000000e+03 +3.540000000000000e+02
+6.459970000000000e+02 +3.535000000000000e+02
+1.151980000000000e+03 +3.535000000000000e+02
+2.051950000000000e+02 +3.165000000000000e+02
+7.572630000000000e+02 +3.520000000000000e+02
+1.766300000000000e+02 +2.970000000000000e+02
+7.241130000000001e+02 +3.505000000000000e+02
+1.615090000000000e+01 +3.505000000000000e+02
+8.567800000000000e+02 +3.505000000000000e+02
+8.952539999999998e+02 +3.505000000000000e+02
+9.691220000000000e+02 +3.505000000000000e+02
+6.050300000000000e+02 +3.495000000000000e+02
+7.067050000000000e+02 +3.490000000000000e+02
+5.246200000000000e+02 +3.490000000000000e+02
+6.199460000000000e+02 +3.490000000000000e+02
+1.681160000000000e+02 +3.485000000000000e+02
+7.517650000000000e+02 +3.485000000000000e+02
+5.869230000000000e+02 +3.480000000000000e+02
+6.108490000000000e+02 +3.480000000000000e+02
+1.170730000000000e+01 +3.475000000000000e+02
+5.481810000000000e+02 +3.465000000000000e+02
+5.161310000000000e+02 +3.460000000000000e+02
+5.344200000000000e+02 +3.455000000000000e+02
+6.453510000000000e+02 +3.455000000000000e+02
+2.493730000000000e+02 +3.455000000000000e+02
+1.088070000000000e+03 +3.450000000000000e+02
+1.111620000000000e+02 +3.450000000000000e+02
+1.536320000000000e+01 +3.085000000000000e+02
+8.978429999999999e+00 +2.375000000000000e+02
+2.392920000000000e+01 +3.440000000000000e+02
+1.093490000000000e+03 +3.440000000000000e+02
+1.245310000000000e+02 +3.430000000000000e+02
+1.255120000000000e+02 +2.110000000000000e+02
+4.708770000000000e+02 +3.415000000000000e+02
+5.149209999999998e+02 +3.415000000000000e+02
+8.595180000000000e+02 +3.415000000000000e+02
+6.323020000000000e+02 +3.410000000000000e+02
+5.972170000000000e+02 +3.410000000000000e+02
+1.068380000000000e+03 +3.410000000000000e+02
+6.865630000000000e+01 +3.410000000000000e+02
+1.181130000000000e+02 +3.400000000000000e+02
+1.413720000000000e+01 +3.400000000000000e+02
+4.719500000000000e+02 +3.395000000000000e+02
+5.541740000000000e+02 +3.390000000000000e+02
+4.645580000000000e+02 +3.390000000000000e+02
+1.544940000000000e+01 +3.385000000000000e+02
+1.088470000000000e+02 +3.380000000000000e+02
+1.264580000000000e+01 +3.375000000000000e+02
+5.334410000000000e+02 +3.360000000000000e+02
+5.146540000000000e+02 +3.360000000000000e+02
+7.192930000000000e+02 +3.350000000000000e+02
+1.015690000000000e+02 +3.350000000000000e+02
+1.410020000000000e+02 +2.320000000000000e+02
+6.388490000000000e+02 +3.345000000000000e+02
+1.452300000000000e+03 +3.345000000000000e+02
+6.144500000000000e+02 +3.340000000000000e+02
+5.926540000000000e+02 +3.335000000000000e+02
+3.456100000000000e+02 +3.335000000000000e+02
+1.618440000000000e+02 +3.330000000000000e+02
+8.944060000000002e+02 +3.325000000000000e+02
+5.943670000000000e+02 +3.325000000000000e+02
+5.735610000000000e+02 +3.325000000000000e+02
+7.207110000000000e+02 +3.315000000000000e+02
+4.931610000000000e+02 +3.310000000000000e+02
+6.590640000000000e+01 +1.950000000000000e+02
+6.416110000000000e+02 +3.305000000000000e+02
+9.284930000000001e+02 +3.305000000000000e+02
+1.348620000000000e+02 +3.295000000000000e+02
+8.808930000000000e+02 +3.295000000000000e+02
+1.057310000000000e+02 +3.290000000000000e+02
+5.666830000000000e+02 +3.285000000000000e+02
+2.112340000000000e+02 +3.280000000000000e+02
+8.008380000000002e+02 +3.275000000000000e+02
+1.287990000000000e+02 +3.275000000000000e+02
+5.877100000000000e+02 +3.270000000000000e+02
+1.046200000000000e+02 +3.270000000000000e+02
+2.564270000000000e+02 +3.255000000000000e+02
+1.523280000000000e+02 +2.555000000000000e+02
+5.829330000000000e+02 +3.250000000000000e+02
+6.598030000000000e+02 +3.245000000000000e+02
+2.012010000000000e+01 +3.245000000000000e+02
+7.848099999999999e+02 +3.240000000000000e+02
+8.557180000000002e+02 +3.240000000000000e+02
+4.984740000000000e+02 +3.240000000000000e+02
+9.332100000000000e+01 +3.240000000000000e+02
+6.225100000000000e+02 +3.240000000000000e+02
+8.000490000000000e+02 +3.240000000000000e+02
+4.164150000000000e+02 +3.235000000000000e+02
+3.928830000000000e+02 +3.235000000000000e+02
+9.353420000000000e+01 +3.235000000000000e+02
+6.078260000000000e+02 +3.235000000000000e+02
+7.996339999999999e+02 +3.230000000000000e+02
+8.105690000000000e+02 +3.225000000000000e+02
+3.700970000000000e+02 +3.225000000000000e+02
+8.784980000000000e+02 +3.220000000000000e+02
+3.872470000000000e+02 +3.215000000000000e+02
+7.531380000000000e+02 +3.215000000000000e+02
+2.109980000000000e+02 +3.215000000000000e+02
+1.188590000000000e+02 +1.960000000000000e+02
+5.310630000000000e+02 +3.205000000000000e+02
+5.764990000000000e+02 +3.205000000000000e+02
+5.875380000000000e+02 +3.200000000000000e+02
+7.285200000000000e+02 +3.195000000000000e+02
+2.806230000000000e+02 +3.190000000000000e+02
+6.194290000000000e+02 +3.190000000000000e+02
+1.509010000000000e+02 +2.370000000000000e+02
+4.740890000000000e+02 +3.185000000000000e+02
+8.601030000000000e+01 +3.185000000000000e+02
+8.756260000000002e+02 +3.185000000000000e+02
+9.085359999999999e+02 +3.180000000000000e+02
+5.032530000000000e+02 +3.175000000000000e+02
+6.728620000000000e+02 +3.175000000000000e+02
+8.710630000000000e+01 +3.170000000000000e+02
+2.208770000000000e+02 +3.165000000000000e+02
+8.803220000000000e+01 +3.160000000000000e+02
+1.398430000000000e+01 +3.160000000000000e+02
+3.834820000000000e+02 +3.155000000000000e+02
+5.150720000000000e+02 +3.155000000000000e+02
+3.067360000000000e+02 +3.155000000000000e+02
+1.890140000000000e+02 +2.735000000000000e+02
+3.808890000000000e+02 +3.150000000000000e+02
+6.287120000000000e+02 +3.140000000000000e+02
+3.336900000000000e+01 +3.135000000000000e+02
+7.174440000000000e+02 +3.135000000000000e+02
+8.778789999999999e+01 +3.135000000000000e+02
+3.029580000000000e+02 +3.135000000000000e+02
+1.419900000000000e+03 +3.130000000000000e+02
+9.101880000000000e+01 +3.125000000000000e+02
+3.547020000000000e+02 +3.120000000000000e+02
+4.187490000000000e+02 +3.120000000000000e+02
+9.807089999999999e+01 +2.010000000000000e+02
+7.010770000000000e+02 +3.115000000000000e+02
+2.277820000000000e+02 +2.385000000000000e+02
+7.509450000000001e+02 +3.110000000000000e+02
+1.033480000000000e+03 +3.105000000000000e+02
+3.838650000000000e+02 +3.100000000000000e+02
+1.105660000000000e+03 +3.100000000000000e+02
+4.894600000000000e+02 +3.100000000000000e+02
+6.196569999999998e+02 +3.100000000000000e+02
+1.087830000000000e+03 +3.100000000000000e+02
+5.129280000000000e+02 +3.070000000000000e+02
+6.206669999999998e+02 +3.085000000000000e+02
+6.345050000000000e+02 +3.085000000000000e+02
+1.115830000000000e+02 +3.085000000000000e+02
+9.669040000000000e+01 +3.080000000000000e+02
+6.597970000000000e+02 +3.075000000000000e+02
+2.517190000000000e+02 +3.070000000000000e+02
+4.448560000000000e+02 +3.070000000000000e+02
+1.731530000000000e+02 +2.360000000000000e+02
+6.152960000000000e+00 +2.935000000000000e+02
+1.452320000000000e+03 +3.060000000000000e+02
+3.369510000000000e+02 +3.055000000000000e+02
+5.346170000000000e+02 +3.040000000000000e+02
+3.188010000000000e+02 +3.030000000000000e+02
+5.729610000000000e+02 +3.025000000000000e+02
+6.413049999999999e+02 +3.020000000000000e+02
+4.875960000000000e+02 +3.020000000000000e+02
+7.122070000000000e+02 +3.020000000000000e+02
+5.366319999999999e+02 +3.010000000000000e+02
+7.944720000000000e+01 +3.005000000000000e+02
+1.165390000000000e+03 +3.000000000000000e+02
+1.603220000000000e+03 +3.000000000000000e+02
+1.157820000000000e+03 +3.000000000000000e+02
+1.291380000000000e+03 +3.000000000000000e+02
+1.309970000000000e+03 +3.000000000000000e+02
+2.470190000000000e+02 +2.995000000000000e+02
+7.409100000000002e+01 +2.995000000000000e+02
+1.039750000000000e+03 +2.995000000000000e+02
+1.367300000000000e+02 +2.035000000000000e+02
+3.373400000000000e+02 +2.990000000000000e+02
+8.969020000000000e+01 +2.020000000000000e+02
+4.109930000000001e+02 +2.985000000000000e+02
+1.866630000000000e+01 +6.900000000000000e+01
+1.134160000000000e+01 +2.980000000000000e+02
+7.298950000000001e+01 +2.975000000000000e+02
+3.030800000000000e+02 +2.970000000000000e+02
+7.697660000000002e+02 +2.970000000000000e+02
+3.229320000000000e+02 +2.965000000000000e+02
+3.170850000000000e+02 +2.960000000000000e+02
+2.473100000000000e+02 +2.955000000000000e+02
+5.681350000000000e+02 +2.955000000000000e+02
+4.412540000000000e+02 +2.955000000000000e+02
+7.514130000000000e+02 +2.950000000000000e+02
+9.261870000000000e+00 +2.950000000000000e+02
+4.422970000000000e+02 +2.945000000000000e+02
+6.196830000000000e+02 +2.945000000000000e+02
+4.432640000000000e+01 +2.945000000000000e+02
+1.059030000000000e+03 +2.940000000000000e+02
+5.920240000000000e+02 +2.940000000000000e+02
+1.071620000000000e+03 +2.940000000000000e+02
+2.676140000000000e+02 +2.935000000000000e+02
+6.763920000000000e+01 +2.935000000000000e+02
+3.439420000000000e+02 +2.930000000000000e+02
+4.066970000000000e+02 +2.925000000000000e+02
+7.524050000000000e+01 +2.925000000000000e+02
+4.629170000000000e+02 +2.920000000000000e+02
+5.278969999999998e+02 +2.920000000000000e+02
+3.375260000000000e+00 +2.920000000000000e+02
+1.611810000000000e+02 +2.330000000000000e+02
+5.389059999999999e+02 +2.915000000000000e+02
+6.675600000000000e+01 +2.915000000000000e+02
+1.232590000000000e+03 +2.915000000000000e+02
+6.729789999999998e+02 +2.910000000000000e+02
+7.255329999999999e+01 +2.910000000000000e+02
+6.948560000000001e+01 +2.310000000000000e+02
+6.051530000000000e+02 +2.895000000000000e+02
+9.130050000000000e+02 +2.895000000000000e+02
+8.518320000000000e+02 +2.890000000000000e+02
+6.602139999999998e+02 +2.885000000000000e+02
+6.724450000000000e+01 +2.885000000000000e+02
+2.673230000000000e+02 +2.545000000000000e+02
+5.000650000000000e+02 +2.885000000000000e+02
+5.045800000000000e+02 +2.880000000000000e+02
+2.125030000000000e+02 +2.880000000000000e+02
+7.821930000000000e+02 +2.870000000000000e+02
+6.877460000000001e+01 +2.865000000000000e+02
+1.128690000000000e+03 +2.865000000000000e+02
+4.482290000000000e+02 +2.865000000000000e+02
+5.117870000000000e+02 +2.860000000000000e+02
+4.287780000000000e+02 +2.855000000000000e+02
+4.429820000000000e+02 +2.850000000000000e+02
+8.766170000000000e+02 +2.845000000000000e+02
+4.598470000000000e+02 +2.840000000000000e+02
+5.472930000000000e+01 +2.840000000000000e+02
+1.035620000000000e+03 +2.835000000000000e+02
+2.796840000000000e+01 +6.700000000000000e+01
+2.396510000000000e+02 +2.830000000000000e+02
+4.410980000000000e+02 +2.825000000000000e+02
+8.383560000000000e+01 +2.355000000000000e+02
+6.018120000000000e+01 +2.820000000000000e+02
+1.258190000000000e+01 +2.820000000000000e+02
+2.733700000000000e+02 +2.815000000000000e+02
+4.545200000000000e+02 +2.810000000000000e+02
+1.276910000000000e+03 +2.810000000000000e+02
+4.696420000000000e+02 +2.805000000000000e+02
+1.163870000000000e+02 +2.220000000000000e+02
+7.884540000000000e+02 +2.800000000000000e+02
+4.546450000000000e+02 +2.795000000000000e+02
+2.168330000000000e+02 +2.790000000000000e+02
+8.209180000000000e+02 +2.790000000000000e+02
+1.466410000000000e+03 +2.790000000000000e+02
+3.465820000000000e+02 +2.785000000000000e+02
+2.513680000000000e+02 +2.780000000000000e+02
+5.468090000000000e+02 +2.780000000000000e+02
+2.942960000000000e+02 +2.775000000000000e+02
+3.155970000000000e+02 +2.770000000000000e+02
+5.576490000000000e+01 +2.770000000000000e+02
+7.880060000000002e+02 +2.770000000000000e+02
+4.895070000000000e+02 +2.765000000000000e+02
+4.461190000000000e+01 +2.765000000000000e+02
+5.611830000000000e+02 +2.760000000000000e+02
+5.371880000000000e+01 +2.760000000000000e+02
+2.454670000000000e+02 +2.755000000000000e+02
+5.043630000000001e+02 +2.755000000000000e+02
+4.897560000000000e+02 +2.740000000000000e+02
+8.640300000000000e+02 +2.750000000000000e+02
+9.087750000000000e+02 +2.750000000000000e+02
+4.218160000000000e+02 +2.750000000000000e+02
+9.638410000000000e+02 +2.745000000000000e+02
+4.912970000000000e+01 +2.745000000000000e+02
+4.564580000000000e+02 +2.740000000000000e+02
+6.962810000000002e+02 +2.730000000000000e+02
+9.981140000000000e+02 +2.725000000000000e+02
+2.451530000000000e+02 +2.530000000000000e+02
+4.763080000000000e+02 +2.720000000000000e+02
+7.682719999999998e+02 +2.715000000000000e+02
+4.385240000000000e+02 +2.715000000000000e+02
+3.950660000000000e+02 +2.710000000000000e+02
+1.009270000000000e+03 +2.710000000000000e+02
+1.529340000000000e+02 +2.290000000000000e+02
+1.355050000000000e+02 +2.285000000000000e+02
+7.782510000000002e+02 +2.700000000000000e+02
+2.135040000000000e+02 +2.695000000000000e+02
+2.331730000000000e+01 +2.690000000000000e+02
+1.038300000000000e+03 +2.690000000000000e+02
+3.216110000000000e+02 +2.690000000000000e+02
+4.577730000000000e+02 +2.685000000000000e+02
+1.845050000000000e+01 +2.680000000000000e+02
+3.917380000000001e+02 +2.680000000000000e+02
+4.697700000000000e+02 +2.675000000000000e+02
+5.994560000000000e+02 +2.675000000000000e+02
+3.926770000000000e+02 +2.670000000000000e+02
+5.210720000000000e+02 +2.670000000000000e+02
+4.717590000000000e+01 +2.670000000000000e+02
+7.045280000000000e+02 +2.670000000000000e+02
+8.563180000000000e+02 +2.670000000000000e+02
+2.608480000000000e+02 +2.665000000000000e+02
+4.827570000000000e+01 +2.665000000000000e+02
+2.579320000000000e+02 +2.660000000000000e+02
+4.296170000000000e+01 +2.655000000000000e+02
+7.868140000000000e+02 +2.645000000000000e+02
+7.596880000000000e+02 +2.645000000000000e+02
+8.385530000000000e+02 +2.645000000000000e+02
+1.690440000000000e+02 +2.640000000000000e+02
+5.579660000000000e+02 +2.640000000000000e+02
+7.578070000000000e+02 +2.640000000000000e+02
+3.573200000000000e+01 +6.600000000000000e+01
+2.157550000000000e+02 +2.635000000000000e+02
+4.185630000000001e+02 +2.635000000000000e+02
+5.667809999999999e+02 +2.635000000000000e+02
+4.094030000000000e+00 +2.635000000000000e+02
+5.113080000000000e+02 +2.630000000000000e+02
+2.708910000000000e+02 +2.545000000000000e+02
+3.650540000000000e+01 +2.625000000000000e+02
+8.317669999999998e+02 +2.625000000000000e+02
+1.157330000000000e+02 +2.385000000000000e+02
+5.065670000000000e+01 +2.620000000000000e+02
+5.564450000000001e+02 +2.610000000000000e+02
+8.205430000000000e+02 +2.610000000000000e+02
+2.239790000000000e+01 +2.605000000000000e+02
+4.191700000000000e+01 +2.605000000000000e+02
+5.436910000000000e+02 +2.600000000000000e+02
+1.295800000000000e+01 +2.600000000000000e+02
+4.497930000000000e+01 +2.600000000000000e+02
+3.508350000000000e+02 +2.595000000000000e+02
+1.683690000000000e+03 +2.595000000000000e+02
+3.497800000000000e+02 +2.595000000000000e+02
+4.780890000000000e+02 +2.590000000000000e+02
+7.967750000000000e+00 +2.590000000000000e+02
+2.425410000000000e+02 +2.590000000000000e+02
+5.370950000000000e+02 +2.585000000000000e+02
+6.699950000000000e+01 +1.915000000000000e+02
+4.086200000000000e+02 +2.580000000000000e+02
+3.813720000000000e+01 +2.580000000000000e+02
+3.080720000000000e+02 +2.580000000000000e+02
+4.274890000000000e+02 +2.575000000000000e+02
+3.167650000000000e+02 +2.575000000000000e+02
+3.864240000000000e+01 +2.575000000000000e+02
+1.093010000000000e+03 +2.570000000000000e+02
+7.878610000000001e+02 +2.565000000000000e+02
+2.750230000000000e+02 +2.565000000000000e+02
+2.556500000000000e+01 +2.555000000000000e+02
+3.887400000000000e+01 +2.555000000000000e+02
+1.825100000000000e+02 +2.550000000000000e+02
+3.926270000000000e+02 +2.545000000000000e+02
+3.795200000000000e+02 +2.545000000000000e+02
+7.919670000000000e+02 +2.545000000000000e+02
+4.439010000000000e+02 +2.545000000000000e+02
+2.569690000000000e+02 +2.540000000000000e+02
+1.235690000000000e+02 +2.535000000000000e+02
+2.915940000000000e+02 +2.530000000000000e+02
+4.425990000000001e+00 +2.530000000000000e+02
+2.384540000000000e+02 +2.525000000000000e+02
+8.029620000000000e+02 +2.525000000000000e+02
+3.668200000000000e+02 +2.520000000000000e+02
+6.477809999999999e+02 +2.520000000000000e+02
+8.668140000000000e+01 +2.520000000000000e+02
+2.403600000000000e+01 +2.515000000000000e+02
+5.939680000000002e+02 +2.510000000000000e+02
+7.859780000000002e+02 +2.505000000000000e+02
+4.565540000000000e+00 +2.500000000000000e+02
+7.776270000000000e+02 +2.500000000000000e+02
+2.211860000000000e+02 +2.495000000000000e+02
+2.792160000000000e+01 +2.495000000000000e+02
+9.444760000000000e+02 +2.495000000000000e+02
+7.492060000000000e+02 +2.490000000000000e+02
+9.929200000000000e+00 +2.485000000000000e+02
+7.889500000000000e+01 +2.480000000000000e+02
+2.021900000000000e+00 +2.475000000000000e+02
+5.347290000000000e+02 +2.470000000000000e+02
+2.321930000000000e+02 +2.465000000000000e+02
+1.175130000000000e+02 +2.465000000000000e+02
+4.417880000000000e+02 +2.465000000000000e+02
+7.114570000000000e+02 +2.460000000000000e+02
+9.240670000000000e+02 +2.460000000000000e+02
+5.343580000000002e+02 +2.455000000000000e+02
+2.615660000000000e+02 +2.450000000000000e+02
+3.372050000000000e+02 +2.445000000000000e+02
+1.244660000000000e+03 +2.445000000000000e+02
+2.555290000000000e+02 +2.435000000000000e+02
+2.406180000000000e+02 +2.430000000000000e+02
+1.978170000000000e+01 +2.430000000000000e+02
+2.352910000000000e+02 +2.420000000000000e+02
+2.614870000000000e+01 +2.420000000000000e+02
+2.307510000000000e+02 +2.420000000000000e+02
+3.209900000000000e+02 +2.420000000000000e+02
+1.486080000000000e+01 +2.420000000000000e+02
+2.146990000000000e+02 +2.410000000000000e+02
+2.340420000000000e+02 +2.410000000000000e+02
+4.506640000000000e+02 +2.405000000000000e+02
+4.136320000000000e+02 +2.405000000000000e+02
+1.895930000000000e+01 +2.395000000000000e+02
+2.844300000000000e+01 +2.395000000000000e+02
+8.387130000000002e+02 +2.395000000000000e+02
+5.065430000000000e+01 +1.945000000000000e+02
+2.638670000000000e+01 +2.390000000000000e+02
+9.148510000000000e+01 +2.385000000000000e+02
+7.806730000000000e+02 +2.385000000000000e+02
+2.421750000000000e+01 +2.380000000000000e+02
+8.942439999999998e+02 +2.370000000000000e+02
+1.367510000000000e+03 +2.370000000000000e+02
+2.954400000000000e+02 +2.365000000000000e+02
+1.740760000000000e+02 +2.325000000000000e+02
+6.675450000000000e+02 +2.360000000000000e+02
+1.025440000000000e+03 +2.360000000000000e+02
+4.107920000000000e+02 +2.350000000000000e+02
+5.338030000000000e+01 +1.900000000000000e+02
+3.199400000000000e+00 +2.345000000000000e+02
+3.457560000000000e+02 +2.345000000000000e+02
+2.511920000000000e+02 +2.345000000000000e+02
+4.440820000000000e+02 +2.340000000000000e+02
+1.038170000000000e+03 +2.340000000000000e+02
+4.657770000000000e+02 +2.330000000000000e+02
+9.500610000000000e+01 +1.875000000000000e+02
+2.295120000000000e+02 +2.320000000000000e+02
+9.670280000000000e-01 +2.320000000000000e+02
+8.842910000000001e+02 +2.320000000000000e+02
+1.525730000000000e+01 +2.320000000000000e+02
+2.255050000000000e+02 +2.315000000000000e+02
+3.954590000000000e+02 +2.310000000000000e+02
+1.323940000000000e+02 +2.305000000000000e+02
+2.716720000000000e+02 +2.305000000000000e+02
+2.134110000000000e+02 +2.305000000000000e+02
+1.154070000000000e+01 +2.300000000000000e+02
+6.624589999999999e+02 +2.300000000000000e+02
+7.974020000000000e+01 +2.300000000000000e+02
+1.126380000000000e+02 +2.070000000000000e+02
+1.376640000000000e+01 +2.295000000000000e+02
+2.314830000000000e+02 +2.295000000000000e+02
+5.827640000000000e+02 +2.295000000000000e+02
+3.419560000000000e+02 +2.290000000000000e+02
+4.277630000000000e+02 +2.290000000000000e+02
+2.139950000000000e+02 +2.285000000000000e+02
+6.667280000000000e+00 +2.285000000000000e+02
+2.538410000000000e+02 +2.285000000000000e+02
+1.063060000000000e+02 +2.280000000000000e+02
+3.785750000000000e+02 +2.280000000000000e+02
+1.158220000000000e+01 +2.275000000000000e+02
+6.109620000000000e+02 +2.275000000000000e+02
+6.629160000000001e+00 +2.270000000000000e+02
+4.238300000000000e+02 +2.270000000000000e+02
+2.140130000000000e+02 +2.270000000000000e+02
+3.325470000000000e+02 +2.265000000000000e+02
+2.621260000000000e+02 +2.260000000000000e+02
+5.359349999999999e+02 +2.260000000000000e+02
+1.187550000000000e+01 +2.260000000000000e+02
+1.147320000000000e+01 +2.255000000000000e+02
+5.248650000000000e+02 +2.255000000000000e+02
+6.628880000000000e+02 +2.250000000000000e+02
+6.550580000000000e+02 +2.250000000000000e+02
+1.998510000000000e+01 +1.785000000000000e+02
+4.781020000000000e+02 +2.240000000000000e+02
+2.270760000000000e+02 +2.235000000000000e+02
+1.033040000000000e+02 +2.235000000000000e+02
+8.859889999999998e+02 +2.235000000000000e+02
+8.634620000000001e-01 +2.230000000000000e+02
+2.005710000000000e+02 +2.230000000000000e+02
+9.239530000000000e+00 +2.230000000000000e+02
+2.403050000000000e+01 +2.225000000000000e+02
+3.201690000000001e+02 +2.220000000000000e+02
+6.169000000000000e+02 +2.220000000000000e+02
+1.032930000000000e+02 +2.215000000000000e+02
+4.424840000000000e+02 +2.115000000000000e+02
+2.927480000000000e+02 +2.210000000000000e+02
+9.402780000000000e+02 +2.210000000000000e+02
+2.030770000000000e+02 +2.205000000000000e+02
+3.931260000000000e+02 +2.195000000000000e+02
+2.901210000000000e+02 +2.195000000000000e+02
+5.052350000000000e+00 +2.170000000000000e+02
+7.896510000000002e+02 +2.185000000000000e+02
+5.110720000000000e+02 +2.180000000000000e+02
+1.529940000000000e+02 +2.100000000000000e+02
+4.328200000000000e+02 +2.175000000000000e+02
+2.754400000000000e+02 +2.175000000000000e+02
+3.249340000000000e+00 +2.175000000000000e+02
+7.477510000000002e+02 +2.140000000000000e+02
+9.210950000000000e+02 +2.170000000000000e+02
+4.426970000000000e+02 +2.165000000000000e+02
+5.450630000000000e+01 +2.155000000000000e+02
+8.032150000000000e+00 +2.155000000000000e+02
+3.480220000000000e+02 +2.100000000000000e+02
+2.522900000000000e+02 +2.150000000000000e+02
+1.912530000000000e+02 +2.150000000000000e+02
+2.378880000000000e+01 +2.150000000000000e+02
+3.440340000000000e+02 +2.145000000000000e+02
+4.567950000000000e+02 +2.140000000000000e+02
+8.082830000000000e+02 +2.130000000000000e+02
+6.832370000000000e+02 +2.130000000000000e+02
+9.262820000000000e+02 +2.130000000000000e+02
+1.242500000000000e+02 +2.100000000000000e+02
+7.768980000000000e+02 +2.125000000000000e+02
+7.077930000000000e+02 +2.125000000000000e+02
+4.049480000000000e+02 +2.120000000000000e+02
+2.656550000000000e+02 +2.120000000000000e+02
+4.724570000000000e+02 +2.115000000000000e+02
+7.487510000000002e+02 +2.115000000000000e+02
+1.567140000000000e+02 +2.115000000000000e+02
+2.763200000000000e+02 +2.115000000000000e+02
+2.360500000000000e+00 +2.110000000000000e+02
+3.185760000000000e+02 +2.105000000000000e+02
+2.272640000000000e+02 +2.095000000000000e+02
+4.416730000000000e+02 +2.095000000000000e+02
+6.752910000000001e+02 +2.090000000000000e+02
+3.723540000000000e+02 +1.810000000000000e+02
+9.720880000000000e+02 +2.085000000000000e+02
+4.578430000000000e+02 +2.085000000000000e+02
+4.911420000000000e+02 +2.085000000000000e+02
+4.129640000000000e+02 +1.840000000000000e+02
+8.172180000000000e+01 +1.840000000000000e+02
+5.075470000000000e+02 +1.825000000000000e+02
+1.566840000000000e+01 +2.065000000000000e+02
+6.748260000000000e+02 +2.065000000000000e+02
+3.532710000000000e+02 +2.060000000000000e+02
+3.974510000000000e+02 +1.855000000000000e+02
+7.178639999999998e+02 +2.060000000000000e+02
+7.352240000000000e+01 +2.060000000000000e+02
+9.317569999999999e+02 +2.055000000000000e+02
+4.374830000000000e+02 +2.055000000000000e+02
+8.461559999999999e-01 +2.050000000000000e+02
+3.814150000000000e+02 +1.860000000000000e+02
+1.061090000000000e+03 +2.050000000000000e+02
+8.292500000000000e+02 +2.050000000000000e+02
+9.724840000000000e+02 +2.045000000000000e+02
+2.095560000000000e+02 +2.045000000000000e+02
+4.256290000000000e+02 +1.850000000000000e+02
+6.705269999999998e+02 +2.040000000000000e+02
+5.128099999999999e+02 +1.870000000000000e+02
+4.899730000000000e+02 +2.035000000000000e+02
+8.332350000000000e+02 +2.025000000000000e+02
+1.056200000000000e+01 +2.015000000000000e+02
+3.570810000000000e+02 +1.830000000000000e+02
+7.066240000000000e+00 +2.015000000000000e+02
+2.187930000000000e+02 +2.005000000000000e+02
+7.268320000000000e+02 +2.005000000000000e+02
+7.300080000000000e+02 +2.005000000000000e+02
+4.272130000000000e+02 +2.000000000000000e+02
+1.320490000000000e+03 +2.000000000000000e+02
+7.846000000000000e+02 +1.995000000000000e+02
+3.865430000000000e+00 +1.980000000000000e+02
+2.480550000000000e+02 +1.885000000000000e+02
+7.428240000000000e+02 +1.980000000000000e+02
+2.518240000000000e+02 +1.970000000000000e+02
+9.459510000000000e+02 +1.890000000000000e+02
+7.425780000000000e+02 +1.965000000000000e+02
+4.939650000000000e+02 +1.960000000000000e+02
+8.031790000000000e+02 +1.960000000000000e+02
+4.305760000000000e+02 +1.955000000000000e+02
+2.538680000000000e+02 +1.865000000000000e+02
+8.865730000000000e+02 +1.955000000000000e+02
+1.776559999999999e+02 +1.950000000000000e+02
+1.450720000000000e+03 +1.950000000000000e+02
+3.077580000000000e+01 +1.020000000000000e+02
+4.452270000000000e+02 +1.945000000000000e+02
+4.421140000000000e+02 +1.945000000000000e+02
+2.851740000000001e+02 +1.945000000000000e+02
+5.074850000000000e+00 +1.940000000000000e+02
+2.009170000000000e+02 +1.935000000000000e+02
+7.255140000000000e+01 +1.935000000000000e+02
+4.472200000000000e+02 +1.935000000000000e+02
+2.471790000000000e+02 +1.935000000000000e+02
+1.944580000000000e+00 +1.930000000000000e+02
+1.541750000000000e+02 +1.930000000000000e+02
+2.268620000000000e+02 +1.930000000000000e+02
+3.658960000000000e+02 +1.925000000000000e+02
+6.103640000000000e+02 +1.920000000000000e+02
+2.275130000000000e+02 +1.920000000000000e+02
+4.009380000000000e+00 +1.920000000000000e+02
+1.629370000000000e+03 +1.920000000000000e+02
+6.132400000000000e+00 +1.915000000000000e+02
+8.140400000000000e+02 +1.910000000000000e+02
+2.119600000000000e+02 +1.905000000000000e+02
+2.330070000000000e+02 +1.810000000000000e+02
+4.244630000000000e+02 +1.900000000000000e+02
+2.289420000000000e+02 +1.900000000000000e+02
+7.934580000000000e-01 +1.900000000000000e+02
+7.882040000000000e+02 +1.900000000000000e+02
+7.668770000000000e+02 +1.900000000000000e+02
+8.727780000000001e+00 +1.895000000000000e+02
+1.114970000000000e+00 +1.895000000000000e+02
+4.316070000000000e+02 +1.885000000000000e+02
+9.461540000000000e+02 +1.885000000000000e+02
+4.060190000000000e+02 +1.880000000000000e+02
+3.932010000000000e+02 +1.815000000000000e+02
+6.409940000000000e+02 +1.880000000000000e+02
+7.095450000000000e+02 +1.880000000000000e+02
+8.612680000000000e+02 +1.880000000000000e+02
+1.619860000000000e+02 +1.875000000000000e+02
+7.155100000000000e+02 +1.875000000000000e+02
+7.534349999999999e+02 +1.870000000000000e+02
+8.213420000000000e+02 +1.865000000000000e+02
+2.198490000000000e+02 +1.860000000000000e+02
+1.491820000000000e-01 +1.860000000000000e+02
+6.075740000000001e+00 +1.255000000000000e+02
+3.652880000000000e+02 +1.855000000000000e+02
+1.604000000000000e+02 +1.855000000000000e+02
+7.901020000000000e+02 +1.855000000000000e+02
+1.250370000000000e+03 +1.850000000000000e+02
+3.254940000000000e+01 +1.845000000000000e+02
+2.394200000000000e+02 +1.820000000000000e+02
+4.575920000000000e+02 +1.825000000000000e+02
+2.240350000000000e+02 +1.835000000000000e+02
+2.294780000000000e+02 +1.825000000000000e+02
+9.685790000000000e+02 +1.830000000000000e+02
+1.082980000000000e+03 +1.825000000000000e+02
+6.398210000000000e+02 +1.820000000000000e+02
+1.056390000000000e+00 +1.815000000000000e+02
+4.030600000000000e+02 +1.790000000000000e+02
+1.589540000000000e+02 +1.815000000000000e+02
+9.142370000000000e-01 +1.815000000000000e+02
+2.417470000000000e+02 +1.770000000000000e+02
+3.637690000000000e+02 +1.800000000000000e+02
+5.015130000000000e+02 +1.810000000000000e+02
+8.638190000000000e+02 +1.810000000000000e+02
+1.263010000000000e+03 +1.800000000000000e+02
+7.582200000000000e+02 +1.800000000000000e+02
+9.696020000000000e+02 +1.800000000000000e+02
+7.459900000000000e+02 +1.800000000000000e+02
+6.568880000000000e-01 +1.800000000000000e+02
+1.154980000000000e+03 +1.800000000000000e+02
+9.486050000000000e+02 +1.800000000000000e+02
+1.028400000000000e+01 +1.800000000000000e+02
+5.355910000000000e+02 +1.800000000000000e+02
+7.391980000000000e+02 +1.800000000000000e+02
+1.161480000000000e+03 +1.800000000000000e+02
+9.248860000000000e+02 +1.800000000000000e+02
+1.262830000000000e+02 +1.800000000000000e+02
+1.245510000000000e+03 +1.795000000000000e+02
+4.080070000000000e+02 +1.785000000000000e+02
+1.200890000000000e+03 +1.780000000000000e+02
+1.921390000000000e+02 +1.780000000000000e+02
+1.935260000000000e+02 +1.775000000000000e+02
+1.296040000000000e+02 +1.775000000000000e+02
+5.801460000000000e+01 +1.110000000000000e+02
+3.513860000000000e+02 +1.770000000000000e+02
+6.684230000000000e+02 +1.770000000000000e+02
+2.338960000000000e+02 +1.765000000000000e+02
+1.158710000000000e+03 +1.760000000000000e+02
+2.102990000000000e+02 +1.755000000000000e+02
+5.125630000000000e+02 +1.745000000000000e+02
+9.274540000000000e+02 +1.745000000000000e+02
+2.596470000000000e+01 +1.740000000000000e+02
+1.292380000000000e+02 +1.735000000000000e+02
+3.286780000000001e+02 +1.735000000000000e+02
+4.593460000000000e+02 +1.715000000000000e+02
+2.049300000000000e+02 +1.710000000000000e+02
+2.418600000000000e+01 +1.710000000000000e+02
+1.701150000000000e+02 +1.705000000000000e+02
+3.471319999999999e+02 +1.695000000000000e+02
+4.692790000000000e+02 +1.695000000000000e+02
+3.258540000000001e+02 +1.695000000000000e+02
+3.324940000000000e+02 +1.690000000000000e+02
+7.413960000000002e+02 +1.685000000000000e+02
+4.402780000000000e+02 +1.685000000000000e+02
+5.640040000000000e+00 +6.500000000000000e+00
+3.671520000000000e+02 +1.680000000000000e+02
+1.868360000000000e+02 +1.680000000000000e+02
+2.603090000000000e+00 +1.675000000000000e+02
+6.372450000000000e+02 +1.675000000000000e+02
+5.603350000000000e+02 +1.670000000000000e+02
+1.254470000000000e+03 +1.665000000000000e+02
+7.903300000000000e+02 +1.665000000000000e+02
+1.639500000000000e+02 +1.600000000000000e+02
+1.652490000000000e+01 +1.520000000000000e+02
+1.870300000000000e+02 +1.645000000000000e+02
+1.163660000000000e+02 +1.175000000000000e+02
+3.052050000000000e+02 +1.645000000000000e+02
+1.612300000000000e+00 +1.640000000000000e+02
+5.431580000000000e+02 +1.640000000000000e+02
+8.337040000000000e+02 +1.230000000000000e+02
+9.605440000000000e+01 +8.750000000000000e+01
+1.179170000000000e+03 +1.635000000000000e+02
+6.982619999999999e+02 +1.635000000000000e+02
+9.409080000000000e+01 +1.305000000000000e+02
+2.033710000000000e+02 +1.630000000000000e+02
+1.450900000000000e+02 +1.205000000000000e+02
+8.898860000000002e+02 +1.615000000000000e+02
+6.051830000000000e+02 +1.610000000000000e+02
+1.133390000000000e+02 +1.610000000000000e+02
+1.833420000000000e+01 +1.605000000000000e+02
+2.102660000000000e+02 +1.505000000000000e+02
+1.787310000000000e+02 +1.600000000000000e+02
+3.678500000000000e+02 +1.595000000000000e+02
+8.756480000000000e+02 +1.335000000000000e+02
+1.862290000000000e+02 +1.590000000000000e+02
+3.330340000000000e+02 +1.590000000000000e+02
+3.861120000000000e+01 +8.250000000000000e+01
+6.087840000000000e+02 +1.585000000000000e+02
+1.712550000000000e+02 +1.585000000000000e+02
+1.579120000000000e+02 +1.260000000000000e+02
+1.685050000000000e+02 +1.005000000000000e+02
+6.811260000000000e+01 +1.560000000000000e+02
+4.617120000000000e+02 +1.575000000000000e+02
+8.003960000000002e+02 +1.570000000000000e+02
+1.671640000000000e+02 +1.565000000000000e+02
+1.500950000000000e+00 +1.530000000000000e+02
+1.140470000000000e+02 +1.210000000000000e+02
+1.119150000000000e+03 +1.550000000000000e+02
+1.189660000000000e+02 +1.545000000000000e+02
+2.408570000000000e+02 +1.540000000000000e+02
+9.306900000000000e+01 +1.320000000000000e+02
+1.972720000000000e+02 +1.535000000000000e+02
+1.082300000000000e+02 +1.535000000000000e+02
+7.647950000000000e+02 +1.345000000000000e+02
+8.066070000000000e+02 +1.525000000000000e+02
+1.631330000000000e+02 +1.525000000000000e+02
+1.629340000000000e+01 +8.450000000000000e+01
+3.195250000000000e+02 +1.520000000000000e+02
+6.942410000000000e+01 +1.180000000000000e+02
+1.793720000000000e+02 +1.515000000000000e+02
+2.390460000000000e+02 +9.850000000000000e+01
+2.618260000000000e+02 +1.515000000000000e+02
+2.801250000000000e+02 +1.515000000000000e+02
+6.523450000000000e+01 +1.515000000000000e+02
+1.468790000000000e+02 +1.510000000000000e+02
+4.007440000000000e+02 +1.510000000000000e+02
+1.112740000000000e+02 +1.510000000000000e+02
+2.130900000000000e+02 +1.505000000000000e+02
+2.944330000000000e+02 +1.500000000000000e+02
+5.646580000000000e+02 +1.500000000000000e+02
+6.809390000000000e+02 +1.490000000000000e+02
+1.729550000000000e+02 +1.480000000000000e+02
+1.672350000000000e+02 +1.475000000000000e+02
+1.371270000000000e+02 +1.470000000000000e+02
+1.961620000000000e+02 +1.470000000000000e+02
+5.585160000000000e+02 +1.470000000000000e+02
+9.923050000000001e+01 +1.465000000000000e+02
+3.258180000000000e+02 +1.460000000000000e+02
+5.085400000000000e+01 +7.350000000000000e+01
+8.110730000000000e+02 +1.450000000000000e+02
+5.887200000000000e+02 +1.445000000000000e+02
+8.245139999999999e+02 +1.445000000000000e+02
+2.368580000000000e+02 +1.440000000000000e+02
+2.215060000000000e+00 +1.800000000000000e+01
+7.120360000000002e+02 +1.010000000000000e+02
+8.012869999999998e+02 +1.435000000000000e+02
+9.762500000000000e+01 +1.425000000000000e+02
+1.148610000000000e+02 +8.850000000000000e+01
+1.876710000000000e+02 +1.420000000000000e+02
+8.263920000000000e+01 +1.325000000000000e+02
+1.792340000000000e+02 +1.415000000000000e+02
+6.371290000000000e+02 +1.410000000000000e+02
+8.957769999999998e+02 +1.315000000000000e+02
+2.031250000000000e+02 +9.400000000000000e+01
+5.412280000000002e+02 +1.405000000000000e+02
+3.472510000000000e+02 +1.405000000000000e+02
+1.668410000000000e+02 +1.400000000000000e+02
+8.213470000000000e+02 +1.400000000000000e+02
+1.609550000000000e+02 +1.385000000000000e+02
+1.090690000000000e+02 +1.300000000000000e+02
+7.021289999999998e+02 +1.380000000000000e+02
+1.149030000000000e+03 +1.380000000000000e+02
+8.863860000000002e+02 +1.380000000000000e+02
+8.946650000000000e+01 +1.375000000000000e+02
+7.957230000000002e+02 +1.375000000000000e+02
+6.406800000000000e+00 +1.370000000000000e+02
+6.000910000000000e+02 +1.370000000000000e+02
+8.256880000000000e+02 +1.365000000000000e+02
+1.715180000000000e+02 +1.365000000000000e+02
+5.673870000000000e+00 +6.000000000000000e+00
+6.800960000000000e+02 +1.360000000000000e+02
+1.099520000000000e+01 +5.500000000000000e+00
+1.301630000000000e+02 +1.350000000000000e+02
+1.973540000000000e+02 +1.345000000000000e+02
+1.542180000000000e+02 +1.345000000000000e+02
+9.852910000000001e+02 +1.260000000000000e+02
+4.411010000000000e+01 +9.000000000000000e+01
+6.304400000000001e+02 +1.340000000000000e+02
+1.910340000000000e+02 +1.335000000000000e+02
+6.748980000000000e+01 +1.335000000000000e+02
+3.940520000000000e+02 +1.330000000000000e+02
+1.501160000000000e+02 +1.320000000000000e+02
+4.338470000000000e+01 +1.100000000000000e+02
+1.474470000000000e+02 +1.320000000000000e+02
+9.025920000000001e+01 +1.315000000000000e+02
+8.403100000000001e+01 +1.315000000000000e+02
+3.492300000000000e+00 +1.310000000000000e+02
+1.193920000000000e+01 +1.310000000000000e+02
+4.617080000000000e+02 +1.310000000000000e+02
+8.094789999999998e+02 +1.305000000000000e+02
+4.609610000000000e+02 +1.305000000000000e+02
+1.441070000000000e+02 +1.300000000000000e+02
+7.716450000000000e+01 +1.300000000000000e+02
+5.092550000000000e+02 +1.300000000000000e+02
+2.190580000000000e+02 +1.300000000000000e+02
+5.670000000000000e+01 +9.200000000000000e+01
+2.906680000000000e+02 +1.295000000000000e+02
+4.621150000000000e+00 +6.750000000000000e+01
+3.213710000000000e+01 +1.290000000000000e+02
+8.127400000000000e+01 +1.290000000000000e+02
+4.915660000000000e+02 +1.285000000000000e+02
+3.440340000000000e-01 +1.275000000000000e+02
+4.400990000000000e+02 +1.275000000000000e+02
+5.694040000000000e+01 +9.750000000000000e+01
+4.917340000000000e+02 +1.265000000000000e+02
+7.714700000000000e+02 +1.265000000000000e+02
+2.361090000000000e+00 +5.600000000000000e+01
+7.632210000000000e+02 +1.260000000000000e+02
+4.984370000000000e+02 +1.260000000000000e+02
+4.362400000000000e+02 +1.260000000000000e+02
+1.279660000000000e+02 +9.450000000000000e+01
+4.896850000000000e+02 +1.255000000000000e+02
+6.331850000000000e+01 +8.450000000000000e+01
+2.466880000000000e+01 +1.250000000000000e+02
+1.878290000000000e+02 +9.250000000000000e+01
+2.174600000000000e+02 +1.245000000000000e+02
+1.270430000000000e+02 +1.245000000000000e+02
+3.744950000000000e+02 +1.105000000000000e+02
+7.589560000000000e+02 +1.240000000000000e+02
+8.517340000000000e+01 +1.240000000000000e+02
+1.206350000000000e+02 +1.235000000000000e+02
+4.847480000000001e+02 +1.235000000000000e+02
+2.245070000000000e+01 +1.235000000000000e+02
+1.171640000000000e+03 +1.235000000000000e+02
+1.120940000000000e+03 +1.230000000000000e+02
+1.054190000000000e+02 +1.225000000000000e+02
+1.142690000000000e+03 +1.225000000000000e+02
+1.827120000000000e+01 +1.220000000000000e+02
+3.381869999999999e+02 +1.220000000000000e+02
+7.182140000000000e+01 +1.215000000000000e+02
+4.346340000000000e+02 +1.215000000000000e+02
+6.403810000000000e+01 +7.350000000000000e+01
+6.885260000000002e+02 +1.210000000000000e+02
+6.336559999999999e+02 +1.205000000000000e+02
+3.848240000000000e+01 +7.350000000000000e+01
+1.312620000000000e+01 +1.195000000000000e+02
+6.920340000000000e+01 +1.195000000000000e+02
+4.656430000000000e+02 +1.190000000000000e+02
+2.172070000000000e+02 +1.190000000000000e+02
+6.621530000000000e+02 +1.185000000000000e+02
+4.664290000000000e+02 +1.185000000000000e+02
+9.560010000000000e+01 +1.185000000000000e+02
+4.265850000000000e+01 +1.185000000000000e+02
+5.095990000000000e+02 +1.185000000000000e+02
+6.143250000000000e+02 +1.180000000000000e+02
+5.289320000000000e+02 +1.180000000000000e+02
+6.073240000000002e+02 +1.180000000000000e+02
+7.250590000000000e+01 +8.650000000000000e+01
+4.558580000000000e+02 +1.175000000000000e+02
+1.102120000000000e+03 +1.175000000000000e+02
+8.878810000000002e+02 +1.175000000000000e+02
+8.281480000000001e+01 +8.600000000000000e+01
+5.357850000000000e+02 +1.165000000000000e+02
+4.718820000000000e+01 +9.300000000000000e+01
+8.347380000000000e+01 +1.160000000000000e+02
+7.737030000000000e+02 +1.155000000000000e+02
+5.978190000000000e+01 +1.155000000000000e+02
+4.631470000000000e+02 +1.150000000000000e+02
+7.883220000000000e+01 +1.145000000000000e+02
+6.413470000000000e+02 +1.145000000000000e+02
+2.664310000000000e+01 +6.750000000000000e+01
+4.595050000000000e+02 +1.140000000000000e+02
+1.992530000000000e+01 +1.140000000000000e+02
+1.151150000000000e+02 +1.010000000000000e+02
+4.399230000000000e+02 +1.135000000000000e+02
+3.789140000000000e+02 +1.135000000000000e+02
+3.999920000000000e+01 +1.130000000000000e+02
+4.393540000000000e+02 +1.130000000000000e+02
+7.207910000000001e+02 +1.130000000000000e+02
+2.321870000000000e+02 +1.130000000000000e+02
+9.635240000000000e+01 +9.100000000000000e+01
+7.311870000000000e+01 +1.125000000000000e+02
+1.826420000000000e+02 +1.120000000000000e+02
+4.068300000000000e+02 +1.120000000000000e+02
+1.627580000000000e+02 +1.120000000000000e+02
+7.596120000000001e+01 +9.900000000000000e+01
+4.779340000000000e+02 +1.115000000000000e+02
+4.422850000000000e+02 +1.115000000000000e+02
+6.254270000000000e+02 +1.115000000000000e+02
+4.205700000000000e+02 +1.110000000000000e+02
+1.498230000000000e+02 +1.105000000000000e+02
+4.285520000000000e+02 +1.105000000000000e+02
+5.426369999999999e+02 +1.105000000000000e+02
+1.155380000000000e+00 +1.085000000000000e+02
+5.680330000000000e+01 +1.100000000000000e+02
+4.328290000000000e+02 +1.100000000000000e+02
+3.031070000000000e+02 +1.100000000000000e+02
+1.326520000000000e+02 +1.090000000000000e+02
+6.366310000000000e+02 +1.090000000000000e+02
+1.801170000000000e+02 +1.085000000000000e+02
+4.577820000000000e+02 +1.085000000000000e+02
+1.120080000000000e+01 +9.000000000000000e+01
+4.135260000000000e+01 +1.085000000000000e+02
+3.174890000000000e+01 +1.080000000000000e+02
+2.568060000000000e+02 +1.080000000000000e+02
+3.702010000000000e+01 +9.600000000000000e+01
+3.293530000000000e+02 +1.075000000000000e+02
+4.338340000000000e+02 +1.075000000000000e+02
+6.696650000000000e+02 +1.075000000000000e+02
+1.795650000000000e+01 +6.300000000000000e+01
+4.627280000000000e+02 +1.070000000000000e+02
+3.607190000000000e+02 +1.070000000000000e+02
+1.882210000000000e+02 +6.950000000000000e+01
+6.226600000000000e+02 +1.060000000000000e+02
+4.349070000000000e+02 +1.055000000000000e+02
+4.066470000000000e+00 +1.055000000000000e+02
+4.121090000000000e+01 +6.050000000000000e+01
+4.328390000000000e+02 +1.050000000000000e+02
+1.939510000000000e+01 +1.050000000000000e+02
+4.523950000000000e+02 +1.045000000000000e+02
+1.952110000000000e+02 +7.400000000000000e+01
+3.502870000000000e+02 +1.035000000000000e+02
+4.110740000000000e+02 +1.035000000000000e+02
+7.786810000000000e+02 +1.025000000000000e+02
+7.723430000000000e+01 +9.200000000000000e+01
+4.356040000000000e+02 +1.020000000000000e+02
+6.253960000000000e+02 +1.020000000000000e+02
+3.137420000000000e+02 +1.015000000000000e+02
+4.427580000000000e+02 +1.015000000000000e+02
+4.137220000000000e+02 +1.010000000000000e+02
+1.229970000000000e+02 +5.850000000000000e+01
+6.204720000000000e+02 +1.010000000000000e+02
+3.569250000000000e+01 +8.300000000000000e+01
+5.663650000000000e+02 +1.005000000000000e+02
+5.884950000000000e+02 +1.000000000000000e+02
+4.013890000000000e+02 +1.000000000000000e+02
+5.408270000000000e+02 +1.000000000000000e+02
+1.391680000000000e+02 +9.000000000000000e+01
+7.398730000000000e+02 +9.950000000000000e+01
+6.231220000000000e+02 +9.950000000000000e+01
+5.264780000000000e+00 +9.950000000000000e+01
+3.153850000000000e+02 +9.900000000000000e+01
+7.483989999999999e+02 +9.850000000000000e+01
+4.357240000000000e+02 +9.850000000000000e+01
+7.410069999999999e+02 +9.850000000000000e+01
+4.601070000000000e+02 +9.850000000000000e+01
+8.432180000000000e+01 +9.800000000000000e+01
+7.237669999999998e+02 +9.800000000000000e+01
+1.831840000000000e+02 +9.800000000000000e+01
+3.592120000000000e+02 +9.800000000000000e+01
+1.523400000000000e+02 +5.950000000000000e+01
+7.493360000000000e+01 +9.750000000000000e+01
+4.738200000000000e+02 +9.750000000000000e+01
+4.394450000000000e+02 +9.700000000000000e+01
+5.872430000000001e+02 +9.650000000000000e+01
+5.580080000000000e+02 +9.650000000000000e+01
+1.057510000000000e+02 +8.100000000000000e+01
+3.838490000000000e+01 +9.600000000000000e+01
+4.217870000000000e+02 +9.600000000000000e+01
+5.296830000000000e+02 +9.600000000000000e+01
+9.197190000000001e+02 +9.600000000000000e+01
+2.849620000000000e+02 +9.600000000000000e+01
+9.083910000000000e+01 +9.600000000000000e+01
+3.509700000000000e+01 +6.950000000000000e+01
+2.125900000000000e+02 +8.800000000000000e+01
+4.202920000000000e+02 +9.500000000000000e+01
+5.283030000000000e+02 +9.500000000000000e+01
+6.513890000000000e+00 +9.500000000000000e+01
+5.394240000000000e+00 +5.450000000000000e+01
+5.619460000000000e+02 +9.450000000000000e+01
+6.928260000000000e+01 +9.450000000000000e+01
+5.944770000000000e+02 +9.450000000000000e+01
+1.156160000000000e+02 +9.450000000000000e+01
+5.898180000000000e+01 +6.800000000000000e+01
+8.790219999999998e+02 +9.400000000000000e+01
+6.386600000000001e+01 +9.400000000000000e+01
+2.432660000000000e+02 +9.400000000000000e+01
+2.829140000000000e+02 +9.350000000000000e+01
+2.010460000000000e+02 +7.550000000000000e+01
+9.856479999999999e+00 +9.350000000000000e+01
+1.706920000000000e+02 +9.350000000000000e+01
+5.240900000000000e+02 +9.300000000000000e+01
+2.373120000000000e+01 +9.250000000000000e+01
+3.415719999999999e+02 +8.680000000000000e+02
+7.039140000000000e+02 +9.250000000000000e+01
+2.751380000000000e+02 +9.200000000000000e+01
+3.624250000000000e+01 +9.200000000000000e+01
+6.281700000000000e+02 +9.200000000000000e+01
+1.138990000000000e+02 +9.150000000000000e+01
+4.196700000000000e+02 +9.150000000000000e+01
+1.630350000000000e+02 +5.750000000000000e+01
+4.776970000000000e+02 +9.100000000000000e+01
+5.467490000000000e+02 +9.100000000000000e+01
+8.575150000000000e+02 +9.050000000000000e+01
+4.137080000000000e+02 +9.050000000000000e+01
+2.052510000000000e+01 +9.050000000000000e+01
+1.010030000000000e+02 +9.050000000000000e+01
+4.018310000000000e+02 +9.000000000000000e+01
+4.622510000000000e+02 +9.000000000000000e+01
+5.752160000000000e+02 +9.000000000000000e+01
+1.165240000000000e+02 +8.950000000000000e+01
+1.002400000000000e+02 +5.700000000000000e+01
+2.685070000000000e+01 +5.100000000000000e+01
+1.364490000000000e+02 +5.600000000000000e+01
+4.273450000000000e+02 +8.900000000000000e+01
+1.904080000000000e+01 +4.850000000000000e+01
+7.244290000000000e+02 +8.850000000000000e+01
+4.435900000000000e+02 +8.800000000000000e+01
+9.674450000000000e+01 +5.450000000000000e+01
+3.739770000000000e+02 +8.800000000000000e+01
+8.969360000000000e+01 +8.750000000000000e+01
+5.623440000000001e+02 +8.700000000000000e+01
+3.968960000000000e+02 +8.700000000000000e+01
+6.522810000000000e+00 +8.700000000000000e+01
+4.261230000000001e+02 +8.650000000000000e+01
+3.940350000000000e+02 +8.600000000000000e+01
+4.761710000000000e-01 +8.600000000000000e+01
+1.257160000000000e+02 +8.550000000000000e+01
+1.413820000000000e+02 +5.350000000000000e+01
+1.064520000000000e+02 +8.500000000000000e+01
+2.872290000000001e+02 +8.500000000000000e+01
+2.181800000000000e+01 +4.700000000000000e+01
+6.651000000000000e+02 +8.500000000000000e+01
+1.111390000000000e+02 +8.450000000000000e+01
+7.693300000000001e+01 +5.300000000000000e+01
+3.049470000000000e+02 +8.400000000000000e+01
+2.747060000000000e+02 +8.400000000000000e+01
+2.595270000000000e+01 +5.850000000000000e+01
+4.247720000000000e+02 +8.300000000000000e+01
+5.392110000000000e+02 +8.250000000000000e+01
+6.428770000000000e+02 +8.250000000000000e+01
+1.112600000000000e+03 +8.200000000000000e+01
+2.714890000000000e+01 +8.200000000000000e+01
+1.941680000000000e-02 +8.200000000000000e+01
+2.248490000000000e+02 +8.200000000000000e+01
+2.062050000000000e+01 +7.300000000000000e+01
+3.971940000000000e+02 +8.100000000000000e+01
+1.213240000000000e+02 +6.150000000000000e+01
+3.775990000000000e+02 +8.100000000000000e+01
+7.213970000000000e+01 +8.050000000000000e+01
+6.146770000000000e+02 +8.050000000000000e+01
+4.142220000000000e+02 +8.050000000000000e+01
+4.203710000000000e+02 +8.000000000000000e+01
+4.946020000000000e+01 +6.300000000000000e+01
+1.125310000000000e+02 +7.900000000000000e+01
+3.290380000000000e+02 +7.900000000000000e+01
+7.223530000000002e+02 +7.850000000000000e+01
+5.943800000000000e+02 +7.850000000000000e+01
+3.524469999999999e+02 +7.800000000000000e+01
+2.596380000000000e+01 +4.700000000000000e+01
+5.954730000000002e+02 +7.800000000000000e+01
+3.779660000000000e+00 +5.450000000000000e+01
+4.120160000000000e+02 +7.750000000000000e+01
+1.414940000000000e+02 +7.100000000000000e+01
+7.724600000000000e+02 +7.750000000000000e+01
+1.148350000000000e+02 +4.850000000000000e+01
+2.766880000000000e+01 +7.700000000000000e+01
+7.157520000000000e+02 +7.700000000000000e+01
+5.661659999999998e+02 +7.650000000000000e+01
+1.130270000000000e+00 +7.650000000000000e+01
+1.180560000000000e+02 +4.650000000000000e+01
+1.225830000000000e+01 +7.600000000000000e+01
+4.357800000000000e+02 +7.600000000000000e+01
+2.518330000000000e+02 +7.600000000000000e+01
+3.245500000000000e+02 +7.600000000000000e+01
+5.848840000000000e+02 +7.550000000000000e+01
+2.765920000000000e+02 +7.550000000000000e+01
+1.549680000000000e+02 +6.150000000000000e+01
+8.730490000000000e+02 +7.550000000000000e+01
+4.188200000000000e+02 +7.500000000000000e+01
+4.922050000000000e+01 +6.750000000000000e+01
+4.197470000000000e+02 +7.450000000000000e+01
+9.499290000000000e+00 +7.450000000000000e+01
+6.442480000000000e+02 +7.450000000000000e+01
+3.703000000000000e+02 +7.450000000000000e+01
+3.993480000000000e+02 +7.450000000000000e+01
+1.808930000000000e+02 +6.650000000000000e+01
+8.981560000000002e+02 +7.400000000000000e+01
+3.208030000000000e+02 +7.350000000000000e+01
+1.405230000000000e+00 +5.650000000000000e+01
+3.659610000000000e+02 +7.300000000000000e+01
+4.253110000000000e+02 +7.250000000000000e+01
+3.299110000000000e+02 +7.200000000000000e+01
+1.740040000000000e+02 +6.450000000000000e+01
+5.423680000000001e+02 +7.200000000000000e+01
+1.885270000000000e+01 +7.000000000000000e+00
+4.003800000000000e+02 +7.100000000000000e+01
+6.871310000000002e+02 +7.100000000000000e+01
+8.269570000000000e+02 +7.000000000000000e+01
+2.981560000000000e+02 +7.000000000000000e+01
+2.986460000000000e+02 +1.950000000000000e+01
+4.144340000000000e+02 +6.950000000000000e+01
+3.604090000000000e+02 +6.950000000000000e+01
+8.351500000000000e+02 +6.950000000000000e+01
+1.077170000000000e+02 +6.150000000000000e+01
+1.266580000000000e+02 +6.900000000000000e+01
+6.663589999999998e+02 +6.900000000000000e+01
+5.774650000000000e+01 +6.900000000000000e+01
+1.059450000000000e+02 +6.850000000000000e+01
+4.761190000000000e+02 +6.850000000000000e+01
+3.669970000000000e+01 +1.650000000000000e+01
+1.798740000000001e+02 +6.850000000000000e+01
+3.886120000000000e+02 +6.800000000000000e+01
+1.587510000000000e+02 +6.800000000000000e+01
+5.603800000000000e-01 +6.750000000000000e+01
+5.299300000000000e+00 +6.750000000000000e+01
+8.223830000000000e+02 +6.750000000000000e+01
+2.909480000000000e+02 +6.700000000000000e+01
+1.024840000000000e+00 +6.050000000000000e+01
+8.016870000000000e+02 +6.700000000000000e+01
+7.016580000000000e+02 +6.650000000000000e+01
+4.353190000000000e+02 +6.650000000000000e+01
+1.529620000000000e+02 +6.650000000000000e+01
+4.821530000000000e+02 +6.600000000000000e+01
+4.279060000000000e+02 +6.600000000000000e+01
+1.467300000000000e+01 +5.750000000000000e+01
+3.732300000000000e+02 +6.550000000000000e+01
+1.264990000000000e+02 +4.950000000000000e+01
+3.977930000000000e+02 +6.550000000000000e+01
+7.442850000000000e+02 +6.500000000000000e+01
+4.462720000000000e-01 +6.500000000000000e+01
+3.369000000000000e+01 +6.500000000000000e+01
+6.076220000000000e+02 +6.500000000000000e+01
+3.854590000000000e+02 +6.500000000000000e+01
+5.225570000000000e+02 +6.450000000000000e+01
+3.447360000000000e+02 +6.450000000000000e+01
+2.844020000000000e+02 +6.400000000000000e+01
+1.342240000000000e+01 +6.400000000000000e+01
+7.115369999999998e+02 +6.350000000000000e+01
+3.136690000000001e+02 +6.350000000000000e+01
+6.275200000000000e+02 +6.350000000000000e+01
+4.094200000000000e+00 +8.000000000000000e+00
+2.703030000000000e+02 +6.300000000000000e+01
+1.201570000000000e+02 +6.300000000000000e+01
+9.948699999999999e+01 +6.250000000000000e+01
+5.370010000000000e+02 +6.250000000000000e+01
+3.787280000000000e+02 +6.250000000000000e+01
+9.463190000000000e+01 +6.150000000000000e+01
+8.031680000000000e-01 +6.150000000000000e+01
+1.266420000000000e+02 +6.150000000000000e+01
+5.909600000000000e+02 +6.150000000000000e+01
+7.224140000000000e+02 +6.100000000000000e+01
+2.381030000000000e+02 +6.100000000000000e+01
+7.556180000000000e+00 +5.900000000000000e+01
+1.179010000000000e+00 +4.500000000000000e+01
+1.258710000000000e+02 +6.050000000000000e+01
+3.868640000000000e+02 +6.050000000000000e+01
+6.989440000000000e+02 +6.050000000000000e+01
+2.578260000000000e+02 +6.050000000000000e+01
+1.006420000000000e+02 +6.000000000000000e+01
+7.051260000000002e+02 +6.000000000000000e+01
+1.536300000000000e+02 +5.550000000000000e+01
+5.011560000000000e+02 +6.000000000000000e+01
+4.763490000000000e+02 +6.000000000000000e+01
+4.040420000000000e+02 +6.000000000000000e+01
+3.593980000000000e+02 +6.000000000000000e+01
+2.081790000000000e+02 +6.000000000000000e+01
+3.990120000000000e+00 +1.150000000000000e+01
+6.911980000000000e+02 +5.950000000000000e+01
+2.613880000000000e+02 +5.950000000000000e+01
+5.743480000000002e+02 +5.950000000000000e+01
+4.073880000000000e+02 +5.950000000000000e+01
+2.367210000000000e+02 +5.950000000000000e+01
+7.231950000000001e+02 +5.900000000000000e+01
+7.062869999999999e+01 +1.750000000000000e+01
+5.875040000000000e+01 +5.900000000000000e+01
+2.264630000000000e+02 +5.850000000000000e+01
+1.297700000000000e+02 +5.450000000000000e+01
+2.425720000000000e+02 +5.850000000000000e+01
+6.956750000000000e+02 +5.750000000000000e+01
+2.424870000000000e+02 +5.750000000000000e+01
+4.056000000000000e+02 +5.750000000000000e+01
+2.120400000000000e+01 +5.700000000000000e+01
+3.600780000000000e+00 +4.850000000000000e+01
+4.014510000000000e+02 +5.650000000000000e+01
+3.690230000000000e+02 +5.650000000000000e+01
+1.790280000000000e+02 +5.650000000000000e+01
+1.612500000000000e+02 +5.600000000000000e+01
+5.995470000000000e+02 +5.550000000000000e+01
+1.486710000000000e+02 +5.250000000000000e+01
+3.684590000000000e+02 +5.550000000000000e+01
+5.437370000000000e+01 +1.250000000000000e+01
+8.688799999999998e+01 +5.500000000000000e+01
+2.044490000000000e+01 +4.700000000000000e+01
+5.438380000000002e+02 +5.500000000000000e+01
+3.221180000000000e+02 +5.500000000000000e+01
+6.167480000000000e+02 +5.500000000000000e+01
+1.063460000000000e+01 +4.500000000000000e+01
+3.858000000000000e+02 +5.450000000000000e+01
+1.219810000000000e+02 +5.000000000000000e+01
+2.772370000000000e+01 +5.400000000000000e+01
+6.514720000000000e+02 +5.350000000000000e+01
+2.429960000000000e+02 +5.350000000000000e+01
+6.246420000000000e+00 +3.350000000000000e+01
+5.317490000000000e+02 +5.300000000000000e+01
+4.010040000000000e+02 +5.300000000000000e+01
+2.347100000000000e+02 +5.250000000000000e+01
+2.669950000000000e+02 +5.250000000000000e+01
+5.234450000000001e+02 +5.250000000000000e+01
+3.841580000000000e+02 +5.200000000000000e+01
+4.236710000000000e+02 +5.200000000000000e+01
+6.184460000000000e+02 +5.200000000000000e+01
+8.786399999999998e+01 +5.150000000000000e+01
+6.354109999999999e+02 +5.150000000000000e+01
+2.418760000000000e+02 +5.150000000000000e+01
+3.818800000000000e+02 +5.150000000000000e+01
+2.936290000000000e+02 +5.150000000000000e+01
+2.598250000000000e+02 +5.100000000000000e+01
+1.075010000000000e+02 +4.650000000000000e+01
+5.872340000000000e+02 +5.100000000000000e+01
+2.121450000000000e+02 +5.050000000000000e+01
+2.328620000000000e+02 +5.050000000000000e+01
+3.237240000000000e+02 +5.050000000000000e+01
+4.813380000000000e+02 +5.050000000000000e+01
+3.324610000000000e+02 +5.050000000000000e+01
+7.027450000000000e+01 +5.000000000000000e+01
+8.869799999999999e+01 +4.950000000000000e+01
+2.305580000000000e+02 +4.950000000000000e+01
+2.641090000000000e+00 +4.500000000000000e+00
+6.487950000000000e+02 +4.900000000000000e+01
+1.188860000000000e+02 +4.900000000000000e+01
+8.000850000000000e+01 +4.900000000000000e+01
+4.926110000000000e+02 +4.850000000000000e+01
+3.976070000000000e+02 +4.850000000000000e+01
+2.393200000000000e+01 +4.500000000000000e+00
+5.488470000000000e+01 +4.850000000000000e+01
+5.827380000000001e+02 +4.800000000000000e+01
+1.824710000000000e+02 +4.800000000000000e+01
+3.519280000000001e+02 +4.750000000000000e+01
+4.697630000000000e+02 +4.750000000000000e+01
+3.691800000000000e+01 +4.750000000000000e+01
+2.535420000000000e+02 +4.750000000000000e+01
+3.222990000000001e+02 +4.700000000000000e+01
+1.226620000000000e+02 +4.600000000000000e+01
+2.483640000000000e+01 +4.500000000000000e+01
+1.086220000000000e+02 +4.300000000000000e+01
+3.751290000000000e+02 +4.650000000000000e+01
+2.303440000000000e+02 +4.600000000000000e+01
+1.882150000000000e+02 +4.550000000000000e+01
+3.153500000000000e+02 +4.550000000000000e+01
+8.098720000000000e+01 +4.500000000000000e+01
+6.461510000000000e+01 +4.500000000000000e+01
+3.405000000000000e+02 +4.400000000000000e+01
+2.228250000000000e+02 +4.350000000000000e+01
+4.708140000000000e+02 +4.350000000000000e+01
+2.188260000000000e+02 +4.300000000000000e+01
+6.149330000000000e+01 +4.250000000000000e+01
+2.261420000000000e+02 +4.300000000000000e+01
+7.569810000000000e+01 +4.250000000000000e+01
+9.631870000000001e+01 +4.150000000000000e+01
+5.724720000000000e+02 +4.250000000000000e+01
+2.115610000000000e+02 +4.250000000000000e+01
+3.765900000000000e+01 +3.850000000000000e+01
+4.018780000000000e+02 +4.200000000000000e+01
+7.664480000000000e+00 +2.250000000000000e+01
+4.420570000000000e+02 +4.150000000000000e+01
+1.352200000000000e+00 +1.150000000000000e+01
+3.676800000000000e+02 +4.100000000000000e+01
+2.267530000000000e+01 +9.500000000000000e+00
+1.453820000000000e+02 +4.050000000000000e+01
+2.633210000000000e+02 +4.050000000000000e+01
+2.843950000000000e+02 +3.950000000000000e+01
+2.336900000000000e+02 +3.950000000000000e+01
+3.292980000000000e+02 +3.900000000000000e+01
+2.507630000000000e+02 +3.900000000000000e+01
+8.671570000000000e+01 +3.750000000000000e+01
+2.378510000000000e+02 +3.850000000000000e+01
+9.106670000000000e+01 +3.700000000000000e+01
+2.902530000000000e+02 +3.750000000000000e+01
+3.143940000000000e+02 +3.700000000000000e+01
+4.339780000000000e+02 +3.700000000000000e+01
+1.369030000000000e+02 +3.650000000000000e+01
+9.613310000000000e+01 +3.650000000000000e+01
+2.038470000000000e+02 +3.650000000000000e+01
+3.626900000000000e+02 +3.650000000000000e+01
+7.269050000000000e+01 +3.600000000000000e+01
+5.285050000000000e+02 +3.600000000000000e+01
+1.024570000000000e+02 +3.600000000000000e+01
+1.936150000000000e+02 +3.600000000000000e+01
+3.668730000000001e+02 +3.600000000000000e+01
+6.703570000000001e+01 +3.550000000000000e+01
+9.595530000000000e+00 +1.000000000000000e+01
+2.616100000000000e+02 +3.550000000000000e+01
+1.731500000000000e+02 +3.550000000000000e+01
+1.023420000000000e+02 +3.500000000000000e+01
+4.223830000000000e+02 +3.500000000000000e+01
+1.152360000000000e+01 +3.450000000000000e+01
+8.230549999999999e+00 +4.000000000000000e+00
+1.411380000000000e+02 +3.450000000000000e+01
+1.916220000000000e+02 +3.400000000000000e+01
+2.692610000000000e+02 +3.400000000000000e+01
+1.196780000000000e+02 +3.350000000000000e+01
+3.562790000000000e+02 +3.350000000000000e+01
+1.946400000000000e+02 +3.350000000000000e+01
+1.035420000000000e+02 +3.350000000000000e+01
+6.545460000000000e+00 +5.500000000000000e+00
+4.490750000000000e+02 +3.300000000000000e+01
+6.034870000000000e+00 +2.200000000000000e+01
+2.475200000000000e+02 +3.300000000000000e+01
+2.430310000000000e+02 +3.300000000000000e+01
+6.359580000000000e+00 +3.000000000000000e+00
+1.800740000000000e+02 +3.250000000000000e+01
+3.417940000000001e+02 +3.250000000000000e+01
+1.158350000000000e+02 +3.200000000000000e+01
+8.516450000000000e+01 +3.200000000000000e+01
+2.640840000000000e+02 +3.200000000000000e+01
+3.774280000000001e+02 +3.200000000000000e+01
+3.750860000000000e+02 +3.150000000000000e+01
+8.922250000000000e-01 +2.000000000000000e+00
+2.991950000000000e+01 +3.100000000000000e+01
+2.523480000000000e+02 +3.100000000000000e+01
+2.111350000000000e+02 +3.100000000000000e+01
+3.978660000000000e+00 +3.100000000000000e+01
+1.913380000000000e+02 +3.050000000000000e+01
+1.074550000000000e+02 +3.050000000000000e+01
+1.192380000000000e+02 +3.050000000000000e+01
+1.681120000000000e+02 +3.050000000000000e+01
+2.373340000000000e+02 +3.000000000000000e+01
+1.579440000000000e+02 +1.350000000000000e+01
+2.264740000000000e+02 +2.950000000000000e+01
+3.154800000000000e+02 +2.950000000000000e+01
+1.719430000000000e+01 +2.900000000000000e+01
+3.096590000000000e+01 +2.850000000000000e+01
+3.458360000000000e+02 +2.850000000000000e+01
+2.014100000000000e+02 +2.800000000000000e+01
+7.627090000000000e+01 +2.800000000000000e+01
+5.962080000000000e+01 +2.750000000000000e+01
+4.036100000000000e+02 +2.750000000000000e+01
+7.901260000000001e+01 +2.750000000000000e+01
+3.387569999999999e+02 +2.750000000000000e+01
+3.333850000000000e+01 +2.750000000000000e+01
+1.255960000000000e+02 +2.700000000000000e+01
+2.850270000000000e+02 +2.700000000000000e+01
+1.605700000000000e+02 +2.650000000000000e+01
+2.094100000000000e+02 +2.600000000000000e+01
+3.202870000000001e+02 +2.600000000000000e+01
+2.176250000000000e+02 +2.550000000000000e+01
+3.963080000000000e+01 +9.500000000000000e+00
+2.142030000000000e+02 +2.500000000000000e+01
+2.564970000000000e+02 +2.500000000000000e+01
+1.043340000000000e+02 +2.450000000000000e+01
+6.539630000000000e+01 +2.400000000000000e+01
+3.210530000000000e+02 +2.400000000000000e+01
+1.540690000000000e+02 +2.400000000000000e+01
+2.450250000000000e+01 +2.350000000000000e+01
+2.930170000000000e+00 +2.300000000000000e+01
+2.098080000000000e+02 +2.300000000000000e+01
+2.225990000000000e+02 +2.300000000000000e+01
+3.084060000000000e+02 +2.250000000000000e+01
+1.684160000000000e+02 +2.250000000000000e+01
+3.013230000000000e+00 +2.250000000000000e+01
+1.804090000000000e+02 +2.250000000000000e+01
+1.912900000000000e+02 +2.250000000000000e+01
+6.327500000000000e+00 +7.000000000000000e+00
+5.144750000000000e-01 +2.200000000000000e+01
+4.398600000000000e+01 +2.150000000000000e+01
+2.027290000000000e+02 +2.150000000000000e+01
+5.197480000000000e+01 +2.100000000000000e+01
+8.888790000000000e+01 +2.100000000000000e+01
+2.208140000000000e+02 +2.100000000000000e+01
+2.314420000000000e+02 +2.100000000000000e+01
+3.012490000000000e+02 +2.050000000000000e+01
+2.421250000000000e+02 +2.050000000000000e+01
+8.258499999999999e+01 +2.000000000000000e+01
+1.389370000000000e+02 +2.000000000000000e+01
+1.496960000000000e+02 +2.000000000000000e+01
+2.762730000000000e+02 +1.950000000000000e+01
+2.310180000000000e+00 +1.950000000000000e+01
+2.659820000000000e+02 +1.950000000000000e+01
+6.236010000000000e+01 +1.900000000000000e+01
+1.770580000000000e+02 +1.850000000000000e+01
+1.635550000000000e+02 +1.800000000000000e+01
+1.242290000000000e+02 +1.800000000000000e+01
+1.470630000000000e+02 +1.750000000000000e+01
+4.472060000000000e+01 +1.700000000000000e+01
+2.585100000000000e+02 +1.700000000000000e+01
+5.806350000000000e+01 +1.650000000000000e+01
+2.080500000000000e+02 +1.650000000000000e+01
+2.404970000000000e+02 +1.650000000000000e+01
+2.166710000000000e+02 +1.600000000000000e+01
+1.291000000000000e+02 +1.600000000000000e+01
+3.346520000000000e-01 +2.000000000000000e+00
+1.974340000000000e+02 +1.550000000000000e+01
+1.069830000000000e+02 +1.550000000000000e+01
+2.236380000000000e+02 +1.550000000000000e+01
+2.127860000000000e+02 +1.500000000000000e+01
+2.089320000000000e+02 +1.450000000000000e+01
+1.377000000000000e+02 +1.450000000000000e+01
+1.035810000000000e+02 +1.400000000000000e+01
+1.544480000000000e+02 +1.400000000000000e+01
+1.637480000000000e+02 +1.350000000000000e+01
+5.647210000000000e+01 +1.350000000000000e+01
+1.971780000000000e+02 +1.350000000000000e+01
+4.224290000000000e+01 +1.300000000000000e+01
+4.534940000000000e+01 +1.300000000000000e+01
+2.401010000000000e+01 +1.300000000000000e+01
+1.243900000000000e+02 +1.300000000000000e+01
+9.850650000000000e+01 +1.300000000000000e+01
+7.094459999999999e+01 +1.300000000000000e+01
+4.762130000000000e+01 +1.250000000000000e+01
+2.133360000000000e+01 +1.200000000000000e+01
+5.515400000000000e+01 +1.150000000000000e+01
+8.368770000000001e+01 +1.150000000000000e+01
+1.414360000000000e+02 +1.150000000000000e+01
+7.140200000000000e+01 +1.100000000000000e+01
+9.516990000000000e+01 +1.100000000000000e+01
+1.125180000000000e+02 +1.050000000000000e+01
+8.174079999999999e+01 +1.050000000000000e+01
+1.501470000000000e+02 +1.050000000000000e+01
+8.192540000000000e+01 +1.050000000000000e+01
+9.256900000000000e+01 +1.050000000000000e+01
+3.849120000000000e+01 +1.000000000000000e+01
+1.033690000000000e+02 +1.000000000000000e+01
+1.255790000000000e+02 +1.000000000000000e+01
+1.405960000000000e+01 +9.500000000000000e+00
+1.391200000000000e+02 +9.500000000000000e+00
+4.220250000000000e+01 +9.000000000000000e+00
+6.852900000000000e+01 +8.500000000000000e+00
+6.890630000000000e+01 +8.500000000000000e+00
+9.432920000000000e+01 +8.500000000000000e+00
+9.147230000000000e+01 +8.500000000000000e+00
+6.357150000000000e+01 +8.000000000000000e+00
+7.075109999999999e+01 +8.000000000000000e+00
+8.456170000000000e+01 +8.000000000000000e+00
+2.581800000000000e+01 +7.500000000000000e+00
+6.587350000000001e+01 +7.500000000000000e+00
+6.751620000000000e+01 +7.500000000000000e+00
+9.074240000000000e+01 +7.500000000000000e+00
+1.006960000000000e+02 +7.500000000000000e+00
+1.553180000000000e+01 +7.000000000000000e+00
+5.666550000000000e+01 +7.000000000000000e+00
+1.018180000000000e+01 +7.000000000000000e+00
+8.089210000000000e+01 +6.500000000000000e+00
+7.364319999999998e+00 +6.000000000000000e+00
+5.427690000000000e+01 +6.000000000000000e+00
+2.825420000000000e+01 +5.500000000000000e+00
+4.044720000000000e-01 +5.500000000000000e+00
+6.650620000000001e+01 +5.500000000000000e+00
+5.966140000000000e+01 +5.500000000000000e+00
+3.285420000000000e+01 +5.000000000000000e+00
+2.477780000000000e+01 +4.500000000000000e+00
+3.570510000000000e+01 +4.500000000000000e+00
+3.884440000000000e+00 +4.500000000000000e+00
+1.589760000000000e+01 +4.500000000000000e+00
+4.394050000000000e+01 +4.000000000000000e+00
+4.148580000000000e+01 +4.000000000000000e+00
+3.787390000000000e+01 +4.000000000000000e+00
+2.865460000000000e+01 +3.500000000000000e+00
+1.366360000000000e+01 +2.500000000000000e+00
+6.474750000000000e-01 +2.500000000000000e+00
+1.492360000000000e+01 +2.500000000000000e+00
+2.287520000000000e+01 +2.500000000000000e+00
+1.843750000000000e+01 +2.500000000000000e+00
+1.338260000000000e+01 +2.000000000000000e+00
+4.971600000000000e+00 +1.000000000000000e+00
+4.197890000000000e+00 +1.000000000000000e+00
+2.574990000000000e+00 +5.000000000000000e-01
+0.000000000000000e+00 +5.000000000000000e-01
+0.000000000000000e+00 +0.000000000000000e+00
+0.000000000000000e+00 +0.000000000000000e+00
+0.000000000000000e+00 +0.000000000000000e+00
+0.000000000000000e+00 +0.000000000000000e+00
+0.000000000000000e+00 +0.000000000000000e+00
+0.000000000000000e+00 +0.000000000000000e+00
+0.000000000000000e+00 +0.000000000000000e+00
+0.000000000000000e+00 +0.000000000000000e+00
+0.000000000000000e+00 +0.000000000000000e+00
+0.000000000000000e+00 +0.000000000000000e+00
+0.000000000000000e+00 +0.000000000000000e+00
+0.000000000000000e+00 +0.000000000000000e+00
+2.156010000000000e+03 +8.450000000000000e+02
+1.058870000000000e+03 +8.420000000000000e+02
+1.710230000000000e+03 +8.215000000000000e+02
};
\addplot [only marks, draw=color0, mark size=1.0, fill=color0, opacity=0.75, colormap/viridis]
table{%
x                      y
+1.855480000000000e+03 +2.505000000000000e+02
+6.929910000000001e+02 +5.200000000000000e+01
+1.358890000000000e+03 +1.365000000000000e+02
+1.082450000000000e+03 +1.000000000000000e+02
+1.000600000000000e+03 +1.280000000000000e+02
+2.789250000000000e+02 +2.100000000000000e+01
+1.005390000000000e+03 +1.040000000000000e+02
+9.784960000000000e+02 +1.245000000000000e+02
+6.599019999999998e+02 +8.450000000000000e+01
+1.035650000000000e+03 +1.225000000000000e+02
+9.931680000000000e+02 +1.275000000000000e+02
+1.063840000000000e+03 +1.405000000000000e+02
+8.571039999999998e+02 +8.350000000000000e+01
+4.916920000000000e+02 +7.000000000000000e+01
+9.213240000000000e+02 +1.510000000000000e+02
+6.937250000000000e+02 +5.050000000000000e+01
+8.678980000000000e+02 +8.750000000000000e+01
+2.927260000000000e+02 +3.550000000000000e+01
+7.761550000000000e+02 +4.700000000000000e+01
+1.009020000000000e+03 +1.285000000000000e+02
+9.732370000000000e+02 +8.965000000000000e+02
+7.702950000000000e+02 +4.750000000000000e+01
+2.751700000000000e+02 +2.050000000000000e+01
+7.883260000000000e+02 +4.800000000000000e+01
+1.405180000000000e+03 +1.480000000000000e+02
+1.314050000000000e+03 +1.655000000000000e+02
+1.128820000000000e+03 +1.695000000000000e+02
+6.205850000000000e+02 +1.335000000000000e+02
+1.113520000000000e+03 +1.205000000000000e+02
+9.923200000000001e+02 +1.265000000000000e+02
+2.688120000000000e+03 +3.845000000000000e+02
+1.861630000000000e+03 +2.520000000000000e+02
+1.787660000000000e+03 +3.660000000000000e+02
+1.419050000000000e+03 +1.625000000000000e+02
+2.827610000000000e+02 +2.050000000000000e+01
+2.673470000000000e+03 +3.800000000000000e+02
+3.244580000000000e+02 +2.250000000000000e+01
+9.149010000000000e+02 +6.750000000000000e+01
+6.220290000000000e+02 +1.320000000000000e+02
+7.652639999999999e+02 +4.650000000000000e+01
+9.018980000000000e+02 +9.700000000000000e+01
+1.007110000000000e+03 +1.220000000000000e+02
+1.316050000000000e+03 +1.510000000000000e+02
+1.333760000000000e+03 +1.305000000000000e+02
+2.332510000000000e+03 +4.645000000000000e+02
+9.943860000000000e+02 +9.300000000000000e+01
+1.555860000000000e+03 +1.980000000000000e+02
+7.528860000000002e+02 +8.600000000000000e+01
+7.750030000000000e+02 +4.700000000000000e+01
+9.961030000000000e+02 +1.435000000000000e+02
+1.076600000000000e+03 +1.085000000000000e+02
+9.726490000000000e+02 +1.200000000000000e+02
+1.407650000000000e+03 +2.450000000000000e+02
+6.405459999999998e+02 +7.850000000000000e+01
+2.311750000000000e+03 +2.770000000000000e+02
+1.037340000000000e+03 +1.320000000000000e+02
+1.544140000000000e+03 +2.160000000000000e+02
+8.793720000000000e+02 +6.250000000000000e+01
+6.769030000000000e+02 +4.500000000000000e+01
+1.090250000000000e+03 +1.595000000000000e+02
+1.077230000000000e+03 +1.330000000000000e+02
+1.615110000000000e+03 +2.615000000000000e+02
+1.853750000000000e+03 +2.575000000000000e+02
+1.281020000000000e+03 +1.220000000000000e+02
+7.615050000000000e+02 +5.450000000000000e+01
+1.561150000000000e+03 +1.920000000000000e+02
+1.718540000000000e+03 +1.925000000000000e+02
+1.521870000000000e+03 +2.120000000000000e+02
+6.814839999999998e+02 +7.850000000000000e+01
+1.367550000000000e+03 +3.030000000000000e+02
+8.424169999999998e+02 +6.050000000000000e+01
+9.092920000000000e+02 +8.900000000000000e+01
+9.142070000000000e+02 +1.420000000000000e+02
+1.411260000000000e+03 +1.445000000000000e+02
+1.783500000000000e+03 +2.295000000000000e+02
+8.813600000000000e+02 +6.400000000000000e+01
+6.657650000000000e+02 +7.450000000000000e+01
+9.034400000000001e+02 +1.220000000000000e+02
+4.922770000000000e+02 +6.350000000000000e+01
+6.927030000000000e+02 +1.450000000000000e+02
+7.652170000000000e+02 +4.550000000000000e+01
+1.120120000000000e+03 +1.115000000000000e+02
+6.450870000000000e+02 +1.820000000000000e+02
+6.250269999999998e+02 +7.200000000000000e+01
+1.033020000000000e+03 +1.120000000000000e+02
+8.701080000000002e+02 +6.150000000000000e+01
+4.859460000000000e+02 +5.950000000000000e+01
+6.028950000000000e+02 +7.250000000000000e+01
+6.742680000000000e+02 +8.950000000000000e+01
+3.369630000000000e+02 +8.845000000000000e+02
+7.680560000000000e+02 +4.750000000000000e+01
+1.349740000000000e+03 +1.880000000000000e+02
+6.668960000000002e+02 +1.260000000000000e+02
+1.574100000000000e+03 +1.935000000000000e+02
+1.303820000000000e+03 +2.305000000000000e+02
+1.864100000000000e+03 +2.575000000000000e+02
+1.446080000000000e+03 +2.025000000000000e+02
+7.516039999999998e+02 +5.500000000000000e+01
+1.297880000000000e+03 +1.540000000000000e+02
+1.059400000000000e+03 +9.800000000000000e+01
+1.029620000000000e+03 +1.130000000000000e+02
+8.765510000000000e+02 +6.650000000000000e+01
+6.209950000000000e+02 +1.235000000000000e+02
+7.729119999999998e+02 +4.650000000000000e+01
+1.392970000000000e+03 +1.865000000000000e+02
+8.909920000000000e+02 +6.400000000000000e+01
+8.542030000000000e+02 +1.225000000000000e+02
+6.794160000000001e+02 +7.750000000000000e+01
+8.859299999999999e+02 +6.600000000000000e+01
+9.172310000000000e+02 +1.215000000000000e+02
+1.004570000000000e+03 +1.115000000000000e+02
+1.309740000000000e+03 +1.325000000000000e+02
+9.760050000000000e+02 +1.165000000000000e+02
+7.774040000000000e+02 +6.800000000000000e+01
+1.082910000000000e+03 +1.000000000000000e+02
+7.743650000000000e+02 +6.100000000000000e+01
+9.081860000000000e+02 +1.230000000000000e+02
+1.045950000000000e+03 +1.065000000000000e+02
+1.289490000000000e+03 +1.335000000000000e+02
+3.345040000000000e+02 +2.450000000000000e+01
+1.230000000000000e+03 +1.275000000000000e+02
+1.061130000000000e+03 +9.450000000000000e+01
+1.127560000000000e+03 +1.160000000000000e+02
+6.511390000000000e+02 +6.950000000000000e+01
+9.662790000000000e+02 +8.795000000000000e+02
+4.920510000000000e+02 +2.950000000000000e+01
+1.412390000000000e+03 +2.360000000000000e+02
+8.554490000000000e+02 +6.700000000000000e+01
+6.699600000000000e+02 +6.950000000000000e+01
+1.322320000000000e+03 +1.850000000000000e+02
+7.695260000000002e+02 +4.750000000000000e+01
+1.302550000000000e+03 +1.510000000000000e+02
+1.475220000000000e+03 +2.010000000000000e+02
+1.364020000000000e+03 +2.035000000000000e+02
+4.942410000000000e+02 +3.850000000000000e+01
+9.976070000000000e+02 +1.135000000000000e+02
+1.755750000000000e+03 +1.865000000000000e+02
+1.335690000000000e+03 +2.270000000000000e+02
+6.873520000000000e+02 +1.645000000000000e+02
+1.055960000000000e+03 +1.100000000000000e+02
+1.679330000000000e+03 +2.465000000000000e+02
+6.679220000000000e+02 +8.200000000000000e+01
+1.125690000000000e+03 +1.545000000000000e+02
+1.321400000000000e+03 +1.325000000000000e+02
+1.151700000000000e+03 +2.105000000000000e+02
+6.867030000000000e+02 +1.130000000000000e+02
+1.868770000000000e+03 +2.510000000000000e+02
+1.061180000000000e+03 +9.500000000000000e+01
+3.276340000000000e+02 +8.770000000000000e+02
+7.820750000000000e+02 +4.850000000000000e+01
+1.593120000000000e+03 +2.525000000000000e+02
+4.937380000000001e+02 +5.650000000000000e+01
+7.530640000000000e+02 +7.700000000000000e+01
+9.970080000000000e+02 +1.050000000000000e+02
+1.083920000000000e+03 +9.650000000000000e+01
+1.138390000000000e+03 +1.600000000000000e+02
+1.306300000000000e+03 +1.320000000000000e+02
+7.503439999999998e+02 +4.550000000000000e+01
+1.112550000000000e+03 +1.225000000000000e+02
+1.409210000000000e+03 +2.340000000000000e+02
+9.030790000000000e+02 +1.425000000000000e+02
+9.929140000000000e+02 +1.130000000000000e+02
+1.624530000000000e+03 +2.515000000000000e+02
+1.548120000000000e+03 +1.805000000000000e+02
+1.330140000000000e+03 +2.930000000000000e+02
+2.774470000000000e+02 +2.100000000000000e+01
+9.401740000000000e+02 +1.390000000000000e+02
+1.289110000000000e+03 +1.795000000000000e+02
+7.782580000000000e+02 +4.700000000000000e+01
+1.296930000000000e+03 +1.395000000000000e+02
+8.826810000000000e+02 +6.450000000000000e+01
+9.219180000000000e+02 +1.175000000000000e+02
+1.004130000000000e+03 +8.150000000000000e+01
+8.392420000000000e+02 +6.100000000000000e+01
+9.895470000000000e+02 +1.075000000000000e+02
+1.056690000000000e+03 +1.145000000000000e+02
+6.888700000000000e+02 +1.340000000000000e+02
+9.011180000000001e+02 +6.300000000000000e+01
+1.251460000000000e+03 +2.085000000000000e+02
+1.854570000000000e+03 +2.485000000000000e+02
+9.510910000000000e+02 +7.050000000000000e+01
+7.621510000000002e+02 +1.495000000000000e+02
+1.687820000000000e+03 +2.610000000000000e+02
+9.252900000000000e+02 +6.500000000000000e+01
+1.068550000000000e+03 +2.115000000000000e+02
+1.285780000000000e+03 +2.360000000000000e+02
+2.799990000000000e+02 +2.100000000000000e+01
+8.868789999999998e+02 +1.290000000000000e+02
+1.488760000000000e+03 +1.990000000000000e+02
+1.506670000000000e+03 +1.915000000000000e+02
+7.533420000000000e+02 +1.540000000000000e+02
+8.638570000000000e+02 +6.100000000000000e+01
+1.050970000000000e+03 +2.080000000000000e+02
+6.372060000000000e+02 +6.200000000000000e+01
+8.693700000000000e+02 +6.150000000000000e+01
+1.463790000000000e+03 +2.370000000000000e+02
+6.746410000000002e+02 +7.900000000000000e+01
+8.406660000000001e+02 +6.000000000000000e+01
+1.306350000000000e+03 +1.930000000000000e+02
+6.915980000000002e+02 +1.350000000000000e+02
+1.555840000000000e+03 +1.740000000000000e+02
+8.665139999999999e+02 +2.200000000000000e+02
+1.079740000000000e+03 +2.045000000000000e+02
+1.107910000000000e+03 +1.155000000000000e+02
+9.343110000000000e+02 +6.400000000000000e+01
+4.536680000000000e+02 +3.350000000000000e+01
+1.375300000000000e+03 +1.960000000000000e+02
+1.371760000000000e+03 +2.175000000000000e+02
+9.590110000000000e+02 +8.680000000000000e+02
+8.587040000000000e+02 +1.235000000000000e+02
+3.163020000000000e+02 +8.680000000000000e+02
+3.415719999999999e+02 +8.680000000000000e+02
+6.427090000000002e+02 +6.300000000000000e+01
+8.410200000000000e+02 +1.200000000000000e+02
+6.929989999999998e+02 +1.310000000000000e+02
+7.707869999999998e+02 +4.550000000000000e+01
+1.074230000000000e+03 +2.045000000000000e+02
+6.367410000000000e+02 +6.100000000000000e+01
+3.374630000000000e+02 +2.450000000000000e+01
+7.727520000000000e+02 +5.550000000000000e+01
+7.538180000000000e+02 +7.450000000000000e+01
+1.063320000000000e+03 +1.160000000000000e+02
+1.289720000000000e+03 +2.285000000000000e+02
+6.403869999999999e+02 +6.000000000000000e+01
+1.728760000000000e+03 +2.910000000000000e+02
+6.798819999999999e+02 +1.545000000000000e+02
+6.556920000000000e+02 +5.700000000000000e+01
+1.871470000000000e+03 +3.615000000000000e+02
+9.569390000000000e+02 +1.030000000000000e+02
+9.852000000000000e+02 +1.365000000000000e+02
+1.143960000000000e+03 +1.525000000000000e+02
+1.095920000000000e+03 +1.690000000000000e+02
+9.154320000000000e+02 +1.345000000000000e+02
+1.840350000000000e+03 +3.620000000000000e+02
+1.018190000000000e+03 +1.095000000000000e+02
+1.717270000000000e+03 +2.120000000000000e+02
+9.112089999999999e+02 +1.095000000000000e+02
+1.860700000000000e+03 +3.595000000000000e+02
+4.910120000000000e+02 +4.550000000000000e+01
+6.522100000000000e+02 +1.720000000000000e+02
+6.853520000000000e+02 +1.555000000000000e+02
+1.101700000000000e+03 +1.370000000000000e+02
+9.594070000000000e+02 +8.635000000000000e+02
+7.663120000000000e+02 +1.160000000000000e+02
+1.134520000000000e+03 +1.945000000000000e+02
+6.645080000000000e+02 +5.750000000000000e+01
+7.520870000000000e+02 +1.495000000000000e+02
+1.363140000000000e+03 +2.070000000000000e+02
+1.867270000000000e+03 +3.600000000000000e+02
+1.146150000000000e+03 +1.730000000000000e+02
+7.458620000000000e+02 +1.285000000000000e+02
+7.681710000000000e+02 +1.155000000000000e+02
+1.300960000000000e+03 +1.295000000000000e+02
+8.795350000000000e+02 +5.950000000000000e+01
+8.732840000000000e+02 +1.080000000000000e+02
+2.326670000000000e+03 +3.460000000000000e+02
+1.320140000000000e+03 +2.895000000000000e+02
+7.746770000000000e+02 +1.175000000000000e+02
+1.865620000000000e+03 +2.375000000000000e+02
+1.060150000000000e+03 +1.830000000000000e+02
+1.295920000000000e+03 +1.360000000000000e+02
+1.932970000000000e+03 +3.175000000000000e+02
+1.753470000000000e+03 +2.130000000000000e+02
+1.388780000000000e+03 +2.215000000000000e+02
+1.094170000000000e+03 +2.190000000000000e+02
+6.664250000000000e+02 +1.705000000000000e+02
+2.818490000000000e+02 +9.350000000000000e+01
+1.017310000000000e+03 +7.250000000000000e+01
+3.224120000000001e+02 +8.605000000000000e+02
+9.808030000000000e+02 +1.030000000000000e+02
+1.215670000000000e+03 +2.090000000000000e+02
+1.064130000000000e+03 +1.985000000000000e+02
+6.876990000000000e+02 +7.450000000000000e+01
+1.340300000000000e+03 +2.755000000000000e+02
+8.938839999999999e+02 +1.495000000000000e+02
+1.108810000000000e+03 +1.590000000000000e+02
+6.311300000000000e+02 +5.700000000000000e+01
+1.156770000000000e+03 +2.500000000000000e+02
+1.874760000000000e+03 +3.635000000000000e+02
+6.782950000000000e+02 +1.520000000000000e+02
+1.017980000000000e+03 +1.715000000000000e+02
+4.870110000000000e+02 +4.500000000000000e+01
+2.771970000000000e+02 +9.250000000000000e+01
+7.673850000000000e+02 +1.140000000000000e+02
+1.868780000000000e+03 +2.360000000000000e+02
+6.868489999999998e+02 +1.535000000000000e+02
+6.226500000000000e+02 +9.850000000000000e+01
+9.735540000000000e+02 +9.850000000000000e+01
+8.776870000000000e+02 +6.150000000000000e+01
+1.074710000000000e+03 +1.955000000000000e+02
+6.854330000000000e+02 +7.350000000000000e+01
+9.873360000000000e+02 +7.200000000000000e+01
+1.084100000000000e+03 +1.050000000000000e+02
+1.066170000000000e+03 +2.000000000000000e+02
+6.612380000000001e+02 +5.300000000000000e+01
+1.853290000000000e+03 +3.550000000000000e+02
+7.751060000000001e+02 +1.115000000000000e+02
+1.885520000000000e+03 +3.490000000000000e+02
+7.634880000000001e+02 +1.440000000000000e+02
+5.886840000000000e+02 +9.150000000000000e+01
+6.051230000000000e+02 +5.050000000000000e+01
+1.301590000000000e+03 +2.845000000000000e+02
+1.740150000000000e+03 +2.835000000000000e+02
+1.072860000000000e+03 +1.890000000000000e+02
+3.017960000000000e+02 +8.545000000000000e+02
+9.745210000000000e+02 +9.950000000000000e+01
+1.326760000000000e+03 +2.765000000000000e+02
+1.131450000000000e+03 +8.650000000000000e+01
+1.079120000000000e+03 +1.940000000000000e+02
+7.544160000000001e+02 +6.250000000000000e+01
+2.667070000000000e+03 +2.850000000000000e+02
+1.115000000000000e+03 +1.025000000000000e+02
+1.412080000000000e+03 +2.125000000000000e+02
+7.684710000000000e+02 +1.075000000000000e+02
+1.385550000000000e+03 +2.030000000000000e+02
+2.340510000000000e+03 +3.210000000000000e+02
+8.846790000000000e+02 +9.550000000000000e+01
+1.478670000000000e+03 +3.640000000000000e+02
+1.371670000000000e+03 +2.150000000000000e+02
+7.862030000000000e+02 +1.040000000000000e+02
+9.374980000000000e+02 +1.610000000000000e+02
+1.294660000000000e+03 +1.210000000000000e+02
+6.514410000000000e+02 +5.200000000000000e+01
+1.314960000000000e+03 +1.295000000000000e+02
+1.736100000000000e+03 +2.075000000000000e+02
+2.819530000000000e+02 +8.600000000000000e+01
+1.868040000000000e+03 +3.500000000000000e+02
+1.195210000000000e+03 +8.400000000000000e+01
+6.896230000000000e+02 +1.255000000000000e+02
+1.767740000000000e+03 +2.040000000000000e+02
+9.855599999999999e+02 +1.550000000000000e+02
+9.644340000000000e+02 +8.510000000000000e+02
+3.122490000000000e+02 +8.510000000000000e+02
+7.724630000000002e+02 +1.090000000000000e+02
+1.432610000000000e+03 +2.015000000000000e+02
+1.485030000000000e+03 +3.195000000000000e+02
+7.563610000000001e+02 +1.080000000000000e+02
+8.797869999999998e+02 +9.300000000000000e+01
+1.070180000000000e+03 +1.975000000000000e+02
+4.980300000000000e+02 +4.000000000000000e+01
+9.409660000000000e+02 +8.495000000000000e+02
+2.910620000000000e+02 +8.495000000000000e+02
+1.553080000000000e+03 +1.580000000000000e+02
+9.848819999999999e+02 +8.490000000000000e+02
+7.644370000000000e+02 +1.075000000000000e+02
+1.154520000000000e+03 +1.930000000000000e+02
+2.761120000000000e+02 +8.500000000000000e+01
+6.336230000000000e+02 +5.050000000000000e+01
+1.642100000000000e+03 +3.240000000000000e+02
+9.473900000000000e+02 +1.225000000000000e+02
+1.099940000000000e+03 +1.625000000000000e+02
+1.137910000000000e+03 +2.055000000000000e+02
+1.040900000000000e+03 +1.615000000000000e+02
+7.773270000000000e+02 +1.075000000000000e+02
+1.095900000000000e+03 +1.535000000000000e+02
+9.908400000000000e+02 +1.965000000000000e+02
+1.383490000000000e+03 +2.705000000000000e+02
+1.621390000000000e+03 +2.240000000000000e+02
+8.854490000000000e+02 +1.565000000000000e+02
+5.109180000000000e+02 +8.470000000000000e+02
+7.477020000000000e+02 +1.170000000000000e+02
+7.006310000000002e+02 +1.235000000000000e+02
+1.994490000000000e+03 +3.200000000000000e+02
+6.752819999999998e+02 +1.200000000000000e+02
+1.853440000000000e+03 +2.235000000000000e+02
+8.630560000000000e+02 +1.185000000000000e+02
+1.100030000000000e+03 +1.390000000000000e+02
+7.790549999999999e+02 +1.075000000000000e+02
+9.754500000000000e+02 +1.180000000000000e+02
+9.395630000000000e+02 +1.585000000000000e+02
+1.062900000000000e+03 +9.150000000000000e+01
+7.450100000000000e+02 +1.000000000000000e+02
+2.346410000000000e+03 +4.015000000000000e+02
+8.733410000000000e+02 +1.980000000000000e+02
+9.422270000000000e+02 +8.445000000000000e+02
+2.793860000000000e+02 +8.445000000000000e+02
+9.775520000000000e+02 +1.180000000000000e+02
+6.353480000000002e+02 +1.525000000000000e+02
+8.616540000000000e+02 +1.065000000000000e+02
+7.708620000000000e+02 +1.050000000000000e+02
+1.286160000000000e+03 +2.090000000000000e+02
+2.769890000000000e+02 +8.200000000000000e+01
+9.895940000000001e+02 +1.790000000000000e+02
+7.536330000000000e+02 +1.335000000000000e+02
+9.842980000000000e+02 +1.485000000000000e+02
+6.613760000000002e+02 +8.400000000000000e+01
+6.880050000000000e+02 +1.090000000000000e+02
+9.079470000000000e+02 +9.800000000000000e+01
+4.959750000000000e+02 +3.900000000000000e+01
+8.044800000000000e+02 +1.025000000000000e+02
+6.333490000000000e+02 +1.545000000000000e+02
+3.053900000000000e+02 +8.425000000000000e+02
+1.957680000000000e+03 +3.115000000000000e+02
+1.061520000000000e+03 +2.830000000000000e+02
+6.787100000000000e+02 +1.440000000000000e+02
+9.757900000000000e+02 +8.420000000000000e+02
+7.810250000000000e+02 +1.065000000000000e+02
+1.059000000000000e+03 +1.830000000000000e+02
+1.008060000000000e+03 +1.670000000000000e+02
+9.093170000000000e+02 +9.000000000000000e+01
+9.908130000000000e+02 +1.915000000000000e+02
+1.071600000000000e+03 +1.815000000000000e+02
+7.005340000000000e+02 +1.325000000000000e+02
+7.563689999999998e+02 +1.025000000000000e+02
+9.069530000000000e+02 +8.700000000000000e+01
+8.522619999999999e+02 +9.100000000000000e+01
+6.421030000000002e+02 +1.605000000000000e+02
+3.368600000000000e+02 +2.400000000000000e+01
+1.559630000000000e+03 +3.090000000000000e+02
+1.071320000000000e+03 +1.895000000000000e+02
+1.217790000000000e+03 +2.740000000000000e+02
+1.083850000000000e+02 +1.250000000000000e+01
+1.312460000000000e+03 +1.290000000000000e+02
+8.718589999999998e+02 +9.250000000000000e+01
+6.895599999999999e+02 +1.355000000000000e+02
+7.652080000000002e+02 +1.015000000000000e+02
+1.995420000000000e+03 +3.095000000000000e+02
+1.857680000000000e+03 +3.395000000000000e+02
+6.366369999999999e+02 +1.520000000000000e+02
+7.874839999999998e+02 +1.035000000000000e+02
+1.834420000000000e+03 +3.830000000000000e+02
+6.779220000000000e+02 +6.200000000000000e+01
+9.775950000000000e+02 +1.145000000000000e+02
+1.118680000000000e+03 +1.905000000000000e+02
+9.545810000000000e+02 +1.450000000000000e+02
+8.599119999999998e+02 +1.035000000000000e+02
+1.269720000000000e+03 +1.720000000000000e+02
+6.837869999999998e+02 +1.480000000000000e+02
+6.739550000000000e+02 +1.430000000000000e+02
+1.568390000000000e+03 +3.060000000000000e+02
+6.617710000000002e+02 +1.005000000000000e+02
+1.001510000000000e+03 +1.725000000000000e+02
+1.571630000000000e+03 +1.750000000000000e+02
+7.621860000000000e+02 +9.550000000000000e+01
+1.074730000000000e+03 +1.680000000000000e+02
+8.081730000000000e+02 +9.950000000000000e+01
+2.841130000000000e+02 +7.600000000000000e+01
+1.107510000000000e+03 +2.240000000000000e+02
+9.897880000000000e+02 +1.135000000000000e+02
+7.647830000000000e+02 +1.460000000000000e+02
+6.474290000000000e+02 +1.510000000000000e+02
+7.767300000000000e+02 +1.000000000000000e+02
+1.109260000000000e+03 +1.725000000000000e+02
+1.284240000000000e+03 +1.830000000000000e+02
+6.338350000000000e+02 +1.480000000000000e+02
+1.372080000000000e+03 +2.895000000000000e+02
+8.695520000000000e+02 +1.830000000000000e+02
+6.570790000000000e+02 +1.495000000000000e+02
+6.935280000000000e+02 +1.075000000000000e+02
+9.783480000000000e+02 +1.100000000000000e+02
+7.813930000000000e+02 +9.750000000000000e+01
+1.068750000000000e+03 +1.675000000000000e+02
+1.215980000000000e+03 +1.800000000000000e+02
+1.065360000000000e+03 +1.860000000000000e+02
+1.010290000000000e+03 +1.630000000000000e+02
+2.316700000000000e+03 +4.110000000000000e+02
+7.519169999999998e+02 +9.900000000000000e+01
+8.780500000000000e+02 +8.900000000000000e+01
+1.849450000000000e+03 +3.210000000000000e+02
+2.741560000000000e+02 +6.150000000000000e+01
+9.541310000000000e+02 +8.315000000000000e+02
+8.836020000000000e+02 +8.500000000000000e+01
+2.894710000000000e+02 +8.315000000000000e+02
+2.722100000000000e+02 +5.900000000000000e+01
+7.520630000000000e+02 +1.070000000000000e+02
+9.085930000000000e+02 +8.150000000000000e+01
+1.447420000000000e+03 +2.755000000000000e+02
+1.571700000000000e+03 +2.195000000000000e+02
+1.300200000000000e+03 +1.975000000000000e+02
+8.685360000000002e+02 +1.110000000000000e+02
+6.173750000000000e+02 +1.490000000000000e+02
+1.568770000000000e+03 +1.735000000000000e+02
+7.833930000000000e+02 +9.800000000000000e+01
+8.927630000000000e+02 +8.850000000000000e+01
+1.124050000000000e+03 +1.885000000000000e+02
+4.898910000000000e+02 +1.135000000000000e+02
+6.616050000000000e+02 +9.600000000000000e+01
+6.763070000000000e+02 +1.375000000000000e+02
+4.463600000000000e+02 +8.450000000000000e+01
+7.750110000000002e+02 +9.150000000000000e+01
+1.537790000000000e+03 +3.005000000000000e+02
+4.580520000000000e+02 +6.650000000000000e+01
+2.655820000000000e+02 +8.285000000000000e+02
+3.224100000000000e+02 +8.550000000000000e+01
+1.052960000000000e+03 +1.695000000000000e+02
+2.810300000000000e+02 +6.700000000000000e+01
+6.608700000000000e+02 +1.385000000000000e+02
+6.856400000000000e+02 +1.345000000000000e+02
+1.001390000000000e+03 +1.795000000000000e+02
+1.038310000000000e+03 +1.440000000000000e+02
+1.119310000000000e+03 +1.645000000000000e+02
+7.711050000000000e+02 +1.775000000000000e+02
+8.625419999999998e+02 +1.780000000000000e+02
+1.074490000000000e+03 +1.725000000000000e+02
+1.000060000000000e+03 +1.600000000000000e+02
+6.918090000000000e+02 +9.850000000000000e+01
+1.929100000000000e+03 +2.995000000000000e+02
+1.302610000000000e+03 +2.135000000000000e+02
+1.401780000000000e+03 +3.045000000000000e+02
+1.057420000000000e+03 +1.535000000000000e+02
+3.251410000000000e+02 +8.550000000000000e+01
+1.109320000000000e+03 +1.625000000000000e+02
+1.491390000000000e+03 +4.180000000000000e+02
+1.004920000000000e+03 +1.765000000000000e+02
+1.043590000000000e+03 +1.695000000000000e+02
+1.737300000000000e+03 +4.345000000000000e+02
+9.064880000000001e+02 +1.825000000000000e+02
+6.397780000000000e+02 +1.460000000000000e+02
+1.306460000000000e+03 +1.505000000000000e+02
+1.078750000000000e+02 +2.000000000000000e+01
+6.927280000000002e+02 +1.310000000000000e+02
+6.243390000000001e+02 +6.800000000000000e+01
+1.554790000000000e+03 +3.020000000000000e+02
+9.001080000000002e+02 +8.950000000000000e+01
+1.643010000000000e+03 +1.645000000000000e+02
+6.570520000000000e+02 +1.485000000000000e+02
+6.659150000000000e+02 +1.325000000000000e+02
+1.147470000000000e+03 +1.390000000000000e+02
+6.801230000000000e+02 +9.850000000000000e+01
+2.289960000000000e+03 +4.880000000000000e+02
+9.885520000000000e+02 +1.780000000000000e+02
+1.048400000000000e+03 +1.810000000000000e+02
+6.509550000000000e+02 +1.440000000000000e+02
+7.665280000000000e+02 +8.050000000000000e+01
+8.564400000000001e+02 +8.500000000000000e+01
+1.473170000000000e+03 +1.935000000000000e+02
+1.325310000000000e+03 +2.545000000000000e+02
+9.779730000000000e+02 +1.670000000000000e+02
+1.076090000000000e+03 +1.660000000000000e+02
+2.697470000000000e+03 +5.210000000000000e+02
+2.353080000000000e+03 +4.100000000000000e+02
+8.828780000000000e+02 +8.750000000000000e+01
+1.100910000000000e+03 +2.185000000000000e+02
+7.645510000000000e+02 +8.950000000000000e+01
+1.747960000000000e+03 +4.320000000000000e+02
+1.302030000000000e+03 +2.135000000000000e+02
+2.013590000000000e+03 +2.965000000000000e+02
+8.437950000000000e+02 +8.100000000000000e+01
+8.679060000000002e+02 +1.725000000000000e+02
+7.894340000000000e+02 +1.740000000000000e+02
+6.202510000000000e+02 +1.380000000000000e+02
+7.513789999999998e+02 +1.005000000000000e+02
+7.745980000000002e+02 +8.350000000000000e+01
+9.000630000000000e+02 +8.350000000000000e+01
+8.934250000000000e+02 +1.275000000000000e+02
+4.822860000000000e+02 +1.060000000000000e+02
+9.755839999999999e+02 +1.650000000000000e+02
+4.800760000000000e+02 +8.400000000000000e+01
+6.452130000000002e+02 +1.410000000000000e+02
+6.550530000000000e+02 +1.290000000000000e+02
+9.047310000000000e+02 +8.850000000000000e+01
+1.281020000000000e+03 +2.485000000000000e+02
+1.426250000000000e+02 +8.180000000000000e+02
+1.294290000000000e+03 +1.990000000000000e+02
+1.317090000000000e+03 +2.500000000000000e+02
+7.918639999999998e+02 +1.730000000000000e+02
+8.051940000000000e+02 +1.865000000000000e+02
+1.305950000000000e+03 +2.050000000000000e+02
+1.362340000000000e+03 +2.910000000000000e+02
+9.026230000000000e+02 +2.015000000000000e+02
+1.745480000000000e+03 +3.040000000000000e+02
+1.006470000000000e+03 +1.685000000000000e+02
+7.564780000000002e+02 +9.000000000000000e+01
+9.337050000000000e+02 +1.945000000000000e+02
+6.186330000000000e+02 +5.900000000000000e+01
+2.008270000000000e+03 +3.645000000000000e+02
+1.010440000000000e+03 +1.540000000000000e+02
+6.715030000000000e+02 +1.400000000000000e+02
+1.321370000000000e+03 +2.460000000000000e+02
+6.829069999999998e+02 +1.350000000000000e+02
+6.200520000000000e+02 +1.270000000000000e+02
+1.308850000000000e+03 +2.035000000000000e+02
+9.701990000000000e+02 +1.115000000000000e+02
+6.625000000000000e+02 +1.280000000000000e+02
+6.452040000000002e+02 +2.330000000000000e+02
+1.562040000000000e+03 +1.615000000000000e+02
+7.685910000000000e+02 +8.200000000000000e+01
+1.111010000000000e+03 +1.485000000000000e+02
+1.074890000000000e+03 +2.565000000000000e+02
+6.816640000000000e+02 +1.435000000000000e+02
+1.065210000000000e+03 +1.650000000000000e+02
+6.612930000000000e+02 +1.395000000000000e+02
+2.326120000000000e+03 +4.740000000000000e+02
+7.869800000000000e+02 +9.050000000000000e+01
+1.615310000000000e+03 +3.125000000000000e+02
+8.762460000000002e+02 +8.850000000000000e+01
+1.319820000000000e+03 +2.320000000000000e+02
+8.487940000000000e+02 +1.695000000000000e+02
+1.081300000000000e+03 +1.545000000000000e+02
+8.049220000000000e+02 +1.750000000000000e+02
+1.032840000000000e+03 +1.320000000000000e+02
+1.325730000000000e+03 +2.470000000000000e+02
+7.848960000000002e+02 +8.500000000000000e+01
+1.768030000000000e+03 +2.740000000000000e+02
+1.471360000000000e+03 +1.805000000000000e+02
+1.037910000000000e+03 +1.585000000000000e+02
+7.474190000000000e+02 +7.850000000000000e+01
+1.776370000000000e+03 +2.740000000000000e+02
+1.010250000000000e+03 +1.635000000000000e+02
+2.833470000000000e+02 +5.350000000000000e+01
+8.427150000000000e+02 +8.950000000000000e+01
+6.666430000000000e+02 +4.600000000000000e+01
+1.853830000000000e+03 +3.070000000000000e+02
+1.095390000000000e+03 +2.050000000000000e+02
+1.487440000000000e+03 +2.535000000000000e+02
+6.606139999999998e+02 +5.400000000000000e+01
+7.552110000000000e+02 +8.000000000000000e+01
+1.081190000000000e+03 +1.485000000000000e+02
+4.895660000000000e+02 +1.005000000000000e+02
+1.362440000000000e+03 +1.980000000000000e+02
+1.010850000000000e+03 +1.575000000000000e+02
+6.727919999999998e+02 +1.195000000000000e+02
+1.877950000000000e+03 +3.120000000000000e+02
+2.830630000000000e+02 +8.095000000000000e+02
+1.321620000000000e+03 +1.465000000000000e+02
+1.433640000000000e+03 +1.945000000000000e+02
+1.206440000000000e+03 +2.460000000000000e+02
+6.973700000000000e+02 +1.210000000000000e+02
+1.618460000000000e+03 +3.110000000000000e+02
+1.742300000000000e+03 +2.385000000000000e+02
+6.657360000000001e+02 +1.345000000000000e+02
+7.857339999999998e+02 +8.300000000000000e+01
+1.297900000000000e+03 +2.025000000000000e+02
+1.117060000000000e+03 +1.475000000000000e+02
+9.117630000000000e+02 +7.850000000000000e+01
+6.891310000000002e+02 +7.600000000000000e+01
+7.747790000000000e+02 +7.700000000000000e+01
+9.789890000000000e+02 +1.590000000000000e+02
+1.942310000000000e+03 +3.620000000000000e+02
+6.545140000000000e+02 +1.300000000000000e+02
+1.241810000000000e+03 +1.455000000000000e+02
+1.763760000000000e+03 +3.835000000000000e+02
+7.488099999999999e+02 +8.500000000000000e+01
+2.319060000000000e+03 +3.885000000000000e+02
+6.902739999999999e+02 +9.250000000000000e+01
+3.332760000000000e+02 +6.800000000000000e+01
+1.782100000000000e+03 +2.705000000000000e+02
+1.064850000000000e+03 +1.535000000000000e+02
+9.627830000000000e+02 +8.060000000000000e+02
+1.573340000000000e+03 +1.635000000000000e+02
+7.031540000000000e+02 +9.850000000000000e+01
+1.569280000000000e+03 +2.835000000000000e+02
+1.967380000000000e+03 +2.715000000000000e+02
+1.426260000000000e+03 +2.895000000000000e+02
+7.562210000000000e+02 +8.650000000000000e+01
+6.400960000000000e+02 +1.345000000000000e+02
+6.230760000000000e+02 +5.050000000000000e+01
+9.876860000000000e+02 +1.570000000000000e+02
+1.061240000000000e+03 +2.575000000000000e+02
+2.887470000000000e+02 +5.200000000000000e+01
+9.199310000000000e+02 +8.050000000000000e+02
+1.860820000000000e+03 +3.770000000000000e+02
+4.912730000000000e+02 +9.350000000000000e+01
+2.654390000000000e+02 +5.100000000000000e+01
+1.321260000000000e+03 +1.805000000000000e+02
+2.476750000000000e+02 +8.045000000000000e+02
+1.782300000000000e+03 +3.195000000000000e+02
+9.179360000000000e+02 +8.500000000000000e+01
+1.081360000000000e+03 +1.500000000000000e+02
+6.314010000000000e+02 +5.250000000000000e+01
+1.148280000000000e+02 +1.250000000000000e+01
+1.297950000000000e+03 +1.905000000000000e+02
+1.331810000000000e+03 +3.335000000000000e+02
+1.855760000000000e+03 +2.945000000000000e+02
+6.523640000000000e+02 +1.345000000000000e+02
+7.710790000000000e+02 +9.550000000000000e+01
+6.229310000000000e+02 +4.950000000000000e+01
+9.762480000000000e+02 +1.555000000000000e+02
+6.662869999999998e+02 +1.280000000000000e+02
+8.259540000000000e+02 +1.695000000000000e+02
+6.785820000000000e+02 +1.190000000000000e+02
+6.062000000000000e+02 +4.900000000000000e+01
+2.946740000000001e+02 +5.200000000000000e+01
+1.875040000000000e+03 +3.020000000000000e+02
+1.126730000000000e+03 +2.100000000000000e+02
+8.970450000000000e+02 +7.350000000000000e+01
+1.331490000000000e+03 +3.595000000000000e+02
+6.761139999999998e+02 +1.080000000000000e+02
+9.514600000000000e+02 +8.005000000000000e+02
+7.684510000000000e+02 +7.850000000000000e+01
+9.783760000000000e+02 +1.560000000000000e+02
+9.476210000000000e+02 +2.180000000000000e+02
+1.321000000000000e+03 +1.735000000000000e+02
+1.308710000000000e+03 +1.925000000000000e+02
+9.690110000000000e+02 +1.415000000000000e+02
+1.862610000000000e+03 +3.745000000000000e+02
+7.690850000000000e+02 +7.200000000000000e+01
+1.089790000000000e+03 +1.400000000000000e+02
+1.846800000000000e+03 +2.915000000000000e+02
+9.983230000000000e+02 +1.550000000000000e+02
+6.490640000000000e+02 +1.320000000000000e+02
+6.726389999999999e+02 +1.440000000000000e+02
+1.283460000000000e+03 +1.945000000000000e+02
+9.796280000000000e+02 +1.530000000000000e+02
+1.340630000000000e+03 +1.390000000000000e+02
+7.791419999999998e+02 +7.900000000000000e+01
+9.294420000000000e+02 +9.400000000000000e+01
+2.335980000000000e+03 +3.875000000000000e+02
+7.763720000000000e+02 +7.100000000000000e+01
+1.590060000000000e+03 +2.715000000000000e+02
+1.408370000000000e+03 +2.745000000000000e+02
+9.707110000000000e+02 +1.495000000000000e+02
+1.054800000000000e+03 +1.425000000000000e+02
+2.605300000000000e+02 +7.970000000000000e+02
+4.864470000000000e+02 +6.450000000000000e+01
+9.096380000000000e+02 +7.450000000000000e+01
+1.846660000000000e+03 +2.995000000000000e+02
+1.065540000000000e+03 +1.465000000000000e+02
+1.051100000000000e+03 +1.475000000000000e+02
+1.311820000000000e+03 +1.935000000000000e+02
+1.003340000000000e+03 +1.495000000000000e+02
+7.821189999999998e+02 +7.800000000000000e+01
+6.536900000000001e+02 +1.300000000000000e+02
+2.359510000000000e+03 +4.845000000000000e+02
+1.057460000000000e+03 +2.550000000000000e+02
+6.513410000000000e+02 +1.350000000000000e+02
+1.852630000000000e+03 +3.675000000000000e+02
+2.722880000000000e+02 +7.950000000000000e+02
+1.861550000000000e+03 +2.995000000000000e+02
+6.614330000000000e+02 +2.010000000000000e+02
+1.308010000000000e+03 +2.285000000000000e+02
+6.251120000000000e+02 +4.750000000000000e+01
+1.311600000000000e+03 +1.825000000000000e+02
+1.004200000000000e+03 +2.245000000000000e+02
+7.458800000000000e+02 +7.900000000000000e+01
+1.130830000000000e+03 +2.640000000000000e+02
+1.308450000000000e+03 +2.695000000000000e+02
+1.006730000000000e+03 +1.485000000000000e+02
+1.083930000000000e+03 +1.900000000000000e+02
+6.880939999999998e+02 +7.050000000000000e+01
+1.037360000000000e+03 +1.315000000000000e+02
+1.104100000000000e+03 +2.240000000000000e+02
+6.782530000000000e+02 +1.315000000000000e+02
+2.513040000000000e+02 +7.930000000000000e+02
+2.366290000000000e+03 +4.800000000000000e+02
+6.313009999999998e+02 +9.000000000000000e+01
+6.882170000000000e+02 +6.750000000000000e+01
+1.337650000000000e+03 +2.335000000000000e+02
+1.159280000000000e+03 +2.355000000000000e+02
+5.727900000000000e+02 +7.050000000000000e+01
+1.868090000000000e+03 +2.875000000000000e+02
+6.536830000000000e+02 +1.190000000000000e+02
+1.484340000000000e+03 +2.320000000000000e+02
+1.346140000000000e+03 +2.120000000000000e+02
+1.751490000000000e+03 +2.840000000000000e+02
+1.277630000000000e+03 +2.655000000000000e+02
+4.896330000000000e+02 +8.600000000000000e+01
+6.875330000000000e+02 +1.045000000000000e+02
+2.312500000000000e+03 +4.510000000000000e+02
+2.523460000000000e+02 +7.915000000000000e+02
+7.623500000000000e+02 +6.650000000000000e+01
+1.063970000000000e+03 +1.325000000000000e+02
+1.469460000000000e+03 +2.875000000000000e+02
+1.308310000000000e+03 +1.825000000000000e+02
+1.857400000000000e+03 +2.890000000000000e+02
+1.068510000000000e+03 +1.465000000000000e+02
+1.298660000000000e+03 +1.915000000000000e+02
+4.859700000000000e+02 +8.250000000000000e+01
+7.008680000000001e+02 +6.950000000000000e+01
+9.885900000000000e+02 +1.505000000000000e+02
+1.219640000000000e+03 +2.535000000000000e+02
+1.313000000000000e+03 +1.600000000000000e+02
+7.762100000000000e+02 +7.450000000000000e+01
+1.557650000000000e+03 +2.710000000000000e+02
+2.778350000000000e+02 +7.895000000000000e+02
+9.492140000000001e+02 +1.845000000000000e+02
+1.570870000000000e+03 +2.695000000000000e+02
+2.353210000000000e+03 +4.745000000000000e+02
+7.508240000000000e+02 +7.100000000000000e+01
+4.913500000000000e+02 +7.880000000000000e+02
+1.069030000000000e+03 +1.370000000000000e+02
+1.010190000000000e+03 +1.370000000000000e+02
+7.631610000000002e+02 +6.450000000000000e+01
+1.774670000000000e+03 +2.540000000000000e+02
+4.940560000000000e+02 +5.400000000000000e+01
+6.295010000000000e+02 +1.155000000000000e+02
+7.527300000000000e+02 +8.200000000000000e+01
+6.340219999999998e+02 +1.045000000000000e+02
+7.840300000000000e+02 +7.300000000000000e+01
+1.300860000000000e+03 +1.855000000000000e+02
+1.020390000000000e+03 +1.260000000000000e+02
+1.737210000000000e+03 +2.645000000000000e+02
+1.158220000000000e+03 +2.315000000000000e+02
+1.309160000000000e+03 +1.835000000000000e+02
+1.088470000000000e+03 +1.445000000000000e+02
+6.372060000000000e+02 +1.215000000000000e+02
+2.631070000000000e+03 +4.895000000000000e+02
+1.319580000000000e+03 +2.310000000000000e+02
+9.557850000000000e+02 +7.845000000000000e+02
+1.309850000000000e+03 +1.880000000000000e+02
+1.294470000000000e+03 +1.560000000000000e+02
+9.941609999999999e+02 +1.415000000000000e+02
+9.360630000000000e+02 +1.985000000000000e+02
+1.227580000000000e+03 +2.510000000000000e+02
+9.948330000000000e+02 +1.380000000000000e+02
+1.570060000000000e+03 +2.675000000000000e+02
+7.658639999999998e+02 +6.200000000000000e+01
+6.900039999999998e+02 +6.800000000000000e+01
+1.037520000000000e+03 +1.225000000000000e+02
+1.847270000000000e+03 +2.910000000000000e+02
+1.470490000000000e+03 +1.540000000000000e+02
+9.008620000000000e+02 +1.190000000000000e+02
+1.612110000000000e+03 +2.855000000000000e+02
+8.671430000000000e+02 +1.370000000000000e+02
+6.454320000000000e+02 +1.015000000000000e+02
+6.182809999999999e+02 +1.395000000000000e+02
+1.135130000000000e+02 +1.150000000000000e+01
+8.936519999999998e+02 +1.720000000000000e+02
+6.804019999999998e+02 +1.140000000000000e+02
+7.864060000000002e+02 +6.750000000000000e+01
+8.824570000000000e+02 +1.135000000000000e+02
+1.402120000000000e+03 +2.590000000000000e+02
+6.817170000000000e+02 +9.950000000000000e+01
+1.864830000000000e+03 +2.925000000000000e+02
+6.885540000000000e+02 +6.050000000000000e+01
+1.064660000000000e+03 +1.235000000000000e+02
+1.924940000000000e+03 +3.025000000000000e+02
+1.913800000000000e+03 +2.575000000000000e+02
+2.066380000000000e+03 +7.800000000000000e+02
+6.788910000000002e+02 +1.040000000000000e+02
+8.093889999999999e+02 +1.375000000000000e+02
+6.365750000000000e+02 +1.135000000000000e+02
+1.132360000000000e+03 +1.855000000000000e+02
+6.185010000000000e+02 +1.390000000000000e+02
+1.364020000000000e+03 +2.045000000000000e+02
+1.502090000000000e+03 +3.670000000000000e+02
+4.889190000000000e+02 +7.500000000000000e+01
+1.862410000000000e+03 +3.645000000000000e+02
+7.476840000000000e+02 +5.900000000000000e+01
+6.667850000000000e+02 +2.060000000000000e+02
+4.864290000000000e+02 +7.800000000000000e+01
+6.406510000000000e+02 +1.165000000000000e+02
+1.478140000000000e+03 +2.770000000000000e+02
+9.808390000000001e+02 +1.430000000000000e+02
+8.970670000000000e+02 +1.135000000000000e+02
+2.651520000000000e+03 +4.725000000000000e+02
+6.867660000000002e+02 +1.560000000000000e+02
+6.793040000000000e+02 +1.190000000000000e+02
+2.336300000000000e+03 +4.715000000000000e+02
+6.788439999999998e+02 +9.250000000000000e+01
+7.629620000000000e+02 +5.850000000000000e+01
+1.421190000000000e+03 +1.245000000000000e+02
+7.447760000000002e+02 +6.250000000000000e+01
+8.980400000000000e+02 +1.225000000000000e+02
+1.080780000000000e+03 +1.365000000000000e+02
+7.073489999999998e+02 +1.190000000000000e+02
+1.401420000000000e+03 +2.570000000000000e+02
+1.001490000000000e+03 +1.270000000000000e+02
+7.533819999999999e+02 +7.500000000000000e+01
+2.726200000000000e+02 +1.165000000000000e+02
+1.146710000000000e+03 +1.870000000000000e+02
+6.640269999999998e+02 +1.355000000000000e+02
+7.714620000000000e+02 +6.500000000000000e+01
+1.352740000000000e+03 +2.010000000000000e+02
+1.589940000000000e+03 +3.560000000000000e+02
+8.841540000000000e+02 +1.745000000000000e+02
+1.303180000000000e+03 +3.015000000000000e+02
+8.713450000000000e+02 +1.185000000000000e+02
+8.778889999999999e+02 +1.685000000000000e+02
+7.650580000000000e+02 +5.800000000000000e+01
+1.370060000000000e+03 +1.615000000000000e+02
+6.725180000000000e+02 +1.190000000000000e+02
+1.402330000000000e+03 +2.580000000000000e+02
+8.729510000000000e+02 +9.450000000000000e+01
+9.443240000000000e+02 +7.725000000000000e+02
+7.527800000000000e+02 +7.100000000000000e+01
+1.429070000000000e+03 +2.595000000000000e+02
+1.867390000000000e+03 +2.790000000000000e+02
+9.995280000000000e+02 +1.255000000000000e+02
+6.409059999999999e+02 +1.035000000000000e+02
+7.789989999999998e+02 +5.800000000000000e+01
+9.407150000000000e+02 +1.310000000000000e+02
+9.775260000000000e+02 +1.260000000000000e+02
+6.370630000000000e+02 +1.115000000000000e+02
+7.589420000000000e+02 +6.550000000000000e+01
+1.291860000000000e+03 +2.450000000000000e+02
+1.865570000000000e+03 +2.815000000000000e+02
+4.880160000000000e+02 +7.250000000000000e+01
+9.922450000000000e+02 +1.280000000000000e+02
+1.079690000000000e+03 +1.720000000000000e+02
+8.994639999999998e+02 +1.165000000000000e+02
+1.010900000000000e+03 +1.945000000000000e+02
+9.031060000000000e+02 +1.165000000000000e+02
+6.947030000000000e+02 +1.470000000000000e+02
+1.747900000000000e+03 +2.945000000000000e+02
+8.523600000000000e+02 +1.020000000000000e+02
+2.359710000000000e+03 +4.625000000000000e+02
+1.054600000000000e+03 +1.320000000000000e+02
+9.056500000000000e+02 +1.195000000000000e+02
+1.178740000000000e+03 +1.260000000000000e+02
+1.864550000000000e+03 +2.815000000000000e+02
+6.634250000000000e+02 +5.750000000000000e+01
+9.259770000000000e+02 +1.695000000000000e+02
+8.585670000000000e+02 +6.650000000000000e+01
+9.746860000000000e+02 +1.305000000000000e+02
+1.333550000000000e+03 +1.910000000000000e+02
+9.093210000000000e+02 +1.180000000000000e+02
+8.900020000000000e+02 +1.740000000000000e+02
+1.012040000000000e+03 +1.945000000000000e+02
+6.233500000000000e+02 +1.290000000000000e+02
+7.712010000000000e+02 +5.950000000000000e+01
+1.498560000000000e+03 +2.145000000000000e+02
+8.518250000000000e+02 +1.165000000000000e+02
+1.073500000000000e+03 +2.320000000000000e+02
+1.567260000000000e+03 +1.580000000000000e+02
+9.758620000000000e+02 +1.315000000000000e+02
+9.199880000000001e+02 +1.145000000000000e+02
+1.372590000000000e+03 +2.265000000000000e+02
+7.826060000000001e+02 +5.900000000000000e+01
+1.023740000000000e+03 +1.100000000000000e+02
+9.213690000000000e+02 +7.655000000000000e+02
+9.104800000000000e+02 +1.185000000000000e+02
+1.416280000000000e+03 +2.515000000000000e+02
+1.334160000000000e+03 +2.465000000000000e+02
+6.345920000000000e+02 +9.500000000000000e+01
+1.118250000000000e+03 +1.830000000000000e+02
+7.646480000000000e+02 +5.100000000000000e+01
+9.010200000000000e+02 +8.550000000000000e+01
+6.846619999999998e+02 +1.750000000000000e+02
+6.580400000000000e+02 +1.315000000000000e+02
+1.579590000000000e+03 +2.465000000000000e+02
+1.325520000000000e+03 +2.230000000000000e+02
+1.328920000000000e+03 +2.120000000000000e+02
+1.005230000000000e+03 +1.935000000000000e+02
+1.035940000000000e+03 +1.285000000000000e+02
+2.308840000000000e+03 +4.235000000000000e+02
+1.075150000000000e+03 +2.250000000000000e+02
+4.885090000000000e+02 +6.800000000000000e+01
+7.768130000000000e+02 +5.350000000000000e+01
+1.395640000000000e+03 +2.915000000000000e+02
+1.010970000000000e+03 +1.880000000000000e+02
+1.152590000000000e+03 +1.830000000000000e+02
+7.866480000000000e+02 +5.350000000000000e+01
+1.415770000000000e+03 +2.465000000000000e+02
+4.915810000000000e+02 +6.550000000000000e+01
+6.404740000000000e+02 +9.450000000000000e+01
+7.603680000000001e+02 +6.350000000000000e+01
+7.778380000000002e+02 +5.600000000000000e+01
+6.718480000000002e+02 +1.950000000000000e+02
+6.693720000000000e+02 +1.335000000000000e+02
+1.083920000000000e+03 +1.650000000000000e+02
+1.047120000000000e+03 +1.295000000000000e+02
+4.759490000000000e+02 +3.150000000000000e+01
+9.579780000000000e+02 +1.730000000000000e+02
+7.876750000000000e+02 +2.280000000000000e+02
+1.850980000000000e+03 +2.755000000000000e+02
+8.123710000000002e+02 +1.310000000000000e+02
+6.633489999999998e+02 +1.225000000000000e+02
+1.301070000000000e+03 +1.645000000000000e+02
+6.701810000000000e+02 +1.235000000000000e+02
+1.159650000000000e+03 +2.965000000000000e+02
+8.557760000000002e+02 +1.445000000000000e+02
+6.794019999999998e+02 +1.340000000000000e+02
+4.712450000000000e+02 +3.100000000000000e+01
+1.240080000000000e+03 +3.180000000000000e+02
+1.502600000000000e+03 +3.415000000000000e+02
+2.331090000000000e+03 +3.490000000000000e+02
+1.285760000000000e+03 +2.440000000000000e+02
+8.177990000000000e+02 +1.310000000000000e+02
+1.856020000000000e+03 +3.380000000000000e+02
+6.213270000000000e+02 +1.190000000000000e+02
+4.783400000000000e+02 +3.200000000000000e+01
+8.894210000000000e+02 +1.060000000000000e+02
+1.489490000000000e+03 +2.580000000000000e+02
+6.418790000000000e+02 +7.900000000000000e+01
+1.240520000000000e+03 +2.155000000000000e+02
+4.904700000000000e+02 +4.050000000000000e+01
+7.502430000000001e+02 +6.050000000000000e+01
+1.128850000000000e+03 +2.375000000000000e+02
+1.300690000000000e+03 +3.510000000000000e+02
+1.100230000000000e+03 +2.690000000000000e+02
+2.487260000000000e+02 +7.560000000000000e+02
+1.044020000000000e+03 +1.765000000000000e+02
+1.857940000000000e+03 +3.755000000000000e+02
+6.350230000000000e+02 +9.100000000000000e+01
+7.633099999999999e+02 +6.050000000000000e+01
+6.249880000000001e+02 +1.200000000000000e+02
+8.931239999999998e+02 +2.185000000000000e+02
+1.189650000000000e+03 +1.420000000000000e+02
+6.926669999999998e+02 +1.365000000000000e+02
+1.578920000000000e+03 +2.425000000000000e+02
+1.290190000000000e+03 +2.340000000000000e+02
+1.115010000000000e+03 +1.770000000000000e+02
+1.114960000000000e+03 +3.040000000000000e+02
+8.539900000000000e+02 +1.055000000000000e+02
+1.069660000000000e+03 +2.105000000000000e+02
+1.563440000000000e+03 +2.395000000000000e+02
+1.750210000000000e+03 +2.625000000000000e+02
+1.078640000000000e+03 +2.090000000000000e+02
+2.557130000000000e+02 +7.535000000000000e+02
+1.300870000000000e+03 +1.595000000000000e+02
+8.997160000000000e+02 +1.080000000000000e+02
+8.966810000000000e+02 +7.700000000000000e+01
+1.836470000000000e+03 +3.715000000000000e+02
+2.737550000000000e+02 +9.150000000000000e+01
+1.082480000000000e+03 +2.080000000000000e+02
+6.236730000000000e+02 +1.150000000000000e+02
+7.711289999999998e+02 +4.950000000000000e+01
+6.743170000000000e+02 +1.785000000000000e+02
+7.573520000000000e+02 +1.575000000000000e+02
+9.889320000000000e+02 +1.195000000000000e+02
+1.436610000000000e+03 +2.440000000000000e+02
+1.848410000000000e+03 +3.400000000000000e+02
+6.194620000000000e+02 +1.140000000000000e+02
+1.751760000000000e+03 +3.585000000000000e+02
+1.423420000000000e+03 +3.485000000000000e+02
+1.003970000000000e+03 +2.210000000000000e+02
+1.127060000000000e+03 +2.610000000000000e+02
+6.962020000000000e+02 +1.475000000000000e+02
+3.314580000000000e+02 +2.450000000000000e+01
+1.348450000000000e+03 +2.800000000000000e+02
+6.595630000000000e+02 +8.700000000000000e+01
+1.585940000000000e+03 +3.340000000000000e+02
+6.620810000000000e+02 +1.860000000000000e+02
+6.541960000000000e+02 +8.400000000000000e+01
+9.765309999999999e+02 +1.215000000000000e+02
+1.542300000000000e+03 +2.225000000000000e+02
+1.045600000000000e+03 +1.825000000000000e+02
+9.941750000000000e+02 +1.165000000000000e+02
+1.140840000000000e+03 +1.975000000000000e+02
+1.428310000000000e+03 +1.775000000000000e+02
+4.926440000000000e+02 +5.550000000000000e+01
+6.404150000000000e+02 +9.100000000000000e+01
+1.315690000000000e+03 +2.545000000000000e+02
+1.119370000000000e+03 +2.995000000000000e+02
+9.111380000000000e+02 +1.725000000000000e+02
+1.468690000000000e+03 +2.510000000000000e+02
+7.686120000000000e+02 +4.800000000000000e+01
+9.856920000000000e+02 +1.205000000000000e+02
+7.816270000000000e+02 +1.220000000000000e+02
+1.329390000000000e+03 +2.815000000000000e+02
+7.624750000000000e+02 +4.550000000000000e+01
+6.791560000000002e+02 +1.740000000000000e+02
+9.167790000000000e+02 +1.080000000000000e+02
+4.857490000000000e+02 +5.400000000000000e+01
+1.008460000000000e+03 +2.160000000000000e+02
+6.537020000000000e+02 +7.000000000000000e+01
+1.057420000000000e+03 +1.875000000000000e+02
+1.331670000000000e+03 +2.775000000000000e+02
+9.319020000000000e+02 +2.810000000000000e+02
+5.067460000000000e+02 +5.600000000000000e+01
+6.605839999999999e+02 +1.145000000000000e+02
+2.010580000000000e+03 +3.305000000000000e+02
+1.617780000000000e+03 +3.625000000000000e+02
+7.746720000000000e+02 +1.190000000000000e+02
+1.301610000000000e+03 +2.395000000000000e+02
+6.810210000000002e+02 +1.535000000000000e+02
+8.990500000000000e+02 +1.155000000000000e+02
+2.642860000000000e+02 +7.445000000000000e+02
+9.767030000000000e+02 +1.400000000000000e+02
+1.558400000000000e+03 +2.285000000000000e+02
+8.857310000000001e+02 +1.080000000000000e+02
+1.847040000000000e+03 +3.655000000000000e+02
+1.374910000000000e+03 +3.085000000000000e+02
+7.783819999999999e+02 +1.180000000000000e+02
+1.292770000000000e+03 +1.465000000000000e+02
+8.768800000000000e+02 +1.085000000000000e+02
+1.306070000000000e+03 +1.500000000000000e+02
+1.168290000000000e+03 +2.740000000000000e+02
+9.909610000000000e+02 +2.150000000000000e+02
+9.060830000000000e+02 +1.150000000000000e+02
+6.718310000000000e+02 +2.745000000000000e+02
+1.032230000000000e+03 +1.845000000000000e+02
+6.574720000000000e+02 +1.140000000000000e+02
+1.575010000000000e+03 +2.270000000000000e+02
+1.066830000000000e+03 +1.950000000000000e+02
+3.454010000000000e+02 +2.750000000000000e+01
+2.758740000000000e+02 +8.500000000000000e+01
+6.342180000000002e+02 +8.400000000000000e+01
+1.078810000000000e+03 +1.900000000000000e+02
+7.871480000000000e+02 +1.175000000000000e+02
+6.565239999999999e+02 +7.700000000000000e+01
+1.548140000000000e+03 +2.230000000000000e+02
+1.870990000000000e+03 +4.040000000000000e+02
+2.383640000000000e+02 +7.395000000000000e+02
+9.072130000000000e+02 +1.060000000000000e+02
+1.400770000000000e+03 +3.355000000000000e+02
+8.460219999999998e+02 +2.045000000000000e+02
+6.873610000000001e+02 +1.570000000000000e+02
+1.079840000000000e+03 +1.955000000000000e+02
+7.687790000000000e+02 +1.140000000000000e+02
+9.855930000000000e+02 +1.150000000000000e+02
+1.869570000000000e+03 +4.685000000000000e+02
+9.863740000000000e+02 +2.140000000000000e+02
+1.142240000000000e+03 +2.470000000000000e+02
+2.343850000000000e+03 +6.785000000000000e+02
+6.592880000000000e+02 +1.150000000000000e+02
+9.718690000000000e+02 +1.120000000000000e+02
+1.739660000000000e+03 +3.500000000000000e+02
+9.058440000000001e+02 +6.950000000000000e+01
+1.307190000000000e+03 +1.465000000000000e+02
+4.921650000000000e+02 +1.330000000000000e+02
+1.129160000000000e+03 +2.490000000000000e+02
+1.295810000000000e+03 +2.440000000000000e+02
+7.801660000000001e+02 +1.155000000000000e+02
+1.985380000000000e+03 +3.175000000000000e+02
+9.126310000000000e+02 +1.010000000000000e+02
+1.872910000000000e+03 +3.615000000000000e+02
+6.894650000000000e+02 +1.465000000000000e+02
+7.704960000000002e+02 +1.130000000000000e+02
+1.567530000000000e+03 +2.250000000000000e+02
+9.501730000000000e+02 +2.410000000000000e+02
+1.304780000000000e+03 +2.715000000000000e+02
+6.198830000000000e+02 +1.020000000000000e+02
+7.865480000000000e+02 +1.170000000000000e+02
+9.948630000000001e+02 +1.140000000000000e+02
+1.327350000000000e+03 +2.955000000000000e+02
+1.302190000000000e+03 +1.525000000000000e+02
+1.331270000000000e+03 +3.065000000000000e+02
+9.298600000000000e+02 +1.010000000000000e+02
+7.910039999999998e+02 +1.070000000000000e+02
+9.558530000000000e+02 +2.360000000000000e+02
+7.787930000000000e+02 +1.150000000000000e+02
+1.720230000000000e+03 +3.050000000000000e+02
+6.911080000000002e+02 +1.700000000000000e+02
+2.678710000000000e+03 +4.440000000000000e+02
+7.869180000000000e+02 +1.165000000000000e+02
+1.581940000000000e+03 +2.180000000000000e+02
+1.002240000000000e+03 +2.110000000000000e+02
+1.858720000000000e+03 +3.415000000000000e+02
+6.308350000000000e+02 +9.850000000000000e+01
+1.002480000000000e+03 +1.155000000000000e+02
+9.197619999999999e+02 +1.670000000000000e+02
+1.609050000000000e+03 +2.380000000000000e+02
+4.836950000000000e+02 +1.310000000000000e+02
+6.374430000000000e+02 +6.150000000000000e+01
+1.158370000000000e+03 +2.550000000000000e+02
+1.572840000000000e+03 +3.010000000000000e+02
+1.109500000000000e+03 +1.620000000000000e+02
+1.295450000000000e+03 +3.235000000000000e+02
+8.714050000000000e+02 +7.320000000000000e+02
+1.754730000000000e+03 +4.635000000000000e+02
+1.223680000000000e+03 +2.010000000000000e+02
+1.064160000000000e+03 +1.900000000000000e+02
+7.428020000000000e+02 +1.195000000000000e+02
+1.145970000000000e+03 +2.490000000000000e+02
+6.899930000000001e+02 +1.520000000000000e+02
+7.739610000000000e+02 +1.130000000000000e+02
+8.920740000000000e+02 +1.085000000000000e+02
+1.247170000000000e+03 +2.090000000000000e+02
+8.589240000000000e+02 +2.035000000000000e+02
+1.546190000000000e+03 +2.200000000000000e+02
+6.691230000000000e+02 +2.110000000000000e+02
+1.308370000000000e+03 +2.270000000000000e+02
+1.055770000000000e+03 +1.720000000000000e+02
+1.550790000000000e+03 +2.170000000000000e+02
+1.855040000000000e+03 +3.655000000000000e+02
+6.848190000000000e+02 +1.505000000000000e+02
+6.490590000000000e+02 +2.125000000000000e+02
+1.394060000000000e+03 +2.630000000000000e+02
+6.342780000000000e+02 +6.150000000000000e+01
+1.853670000000000e+03 +3.405000000000000e+02
+9.962300000000000e+02 +2.175000000000000e+02
+1.282220000000000e+03 +2.765000000000000e+02
+6.250790000000002e+02 +9.550000000000000e+01
+1.863020000000000e+03 +4.550000000000000e+02
+1.156610000000000e+03 +2.520000000000000e+02
+1.767210000000000e+03 +2.545000000000000e+02
+8.882030000000000e+02 +1.095000000000000e+02
+6.733140000000000e+02 +2.645000000000000e+02
+4.821910000000000e+02 +1.185000000000000e+02
+7.727970000000000e+02 +1.965000000000000e+02
+8.755119999999999e+02 +7.265000000000000e+02
+7.662819999999998e+02 +1.085000000000000e+02
+1.302110000000000e+03 +1.370000000000000e+02
+1.140040000000000e+03 +2.815000000000000e+02
+4.744540000000000e+02 +7.850000000000000e+01
+1.460170000000000e+03 +5.520000000000000e+02
+1.304000000000000e+03 +2.410000000000000e+02
+4.856240000000000e+02 +1.300000000000000e+02
+1.072860000000000e+03 +1.505000000000000e+02
+9.473460000000000e+02 +2.650000000000000e+02
+8.580560000000000e+02 +1.160000000000000e+02
+6.417430000000001e+02 +1.575000000000000e+02
+7.641880000000000e+02 +1.320000000000000e+02
+1.529990000000000e+03 +3.125000000000000e+02
+7.834580000000002e+02 +1.120000000000000e+02
+7.870570000000000e+02 +1.140000000000000e+02
+1.054010000000000e+03 +2.650000000000000e+02
+7.823919999999998e+02 +7.240000000000000e+02
+2.343770000000000e+03 +4.345000000000000e+02
+7.698839999999999e+02 +1.020000000000000e+02
+1.337430000000000e+03 +3.650000000000000e+02
+1.341110000000000e+03 +3.195000000000000e+02
+5.044190000000000e+02 +1.245000000000000e+02
+9.784970000000000e+02 +1.240000000000000e+02
+6.378680000000001e+02 +1.590000000000000e+02
+7.351760000000000e+02 +1.380000000000000e+02
+1.294770000000000e+03 +2.230000000000000e+02
+1.290030000000000e+03 +3.645000000000000e+02
+6.540520000000000e+02 +2.640000000000000e+02
+4.921630000000000e+02 +1.315000000000000e+02
+1.089440000000000e+03 +2.595000000000000e+02
+9.118110000000000e+02 +9.050000000000000e+01
+9.904180000000000e+02 +2.015000000000000e+02
+6.740210000000002e+02 +1.485000000000000e+02
+2.300310000000000e+02 +7.220000000000000e+02
+7.488110000000000e+02 +1.065000000000000e+02
+1.756150000000000e+03 +3.370000000000000e+02
+4.776780000000001e+02 +7.600000000000000e+01
+1.298900000000000e+03 +3.205000000000000e+02
+1.066960000000000e+02 +1.700000000000000e+01
+7.743090000000000e+02 +1.045000000000000e+02
+9.836130000000001e+02 +1.150000000000000e+02
+9.430870000000000e+02 +9.050000000000000e+01
+6.588739999999998e+02 +1.580000000000000e+02
+8.845410000000001e+02 +7.210000000000000e+02
+1.871110000000000e+03 +3.355000000000000e+02
+6.200419999999998e+02 +8.900000000000000e+01
+1.316380000000000e+03 +2.185000000000000e+02
+9.492590000000000e+02 +2.585000000000000e+02
+9.046210000000000e+02 +1.950000000000000e+02
+7.843730000000000e+02 +1.055000000000000e+02
+1.829190000000000e+03 +3.485000000000000e+02
+1.591560000000000e+03 +4.380000000000000e+02
+7.926870000000000e+02 +1.025000000000000e+02
+1.810230000000000e+03 +7.200000000000000e+02
+2.176260000000000e+02 +7.200000000000000e+02
+1.939680000000000e+03 +3.630000000000000e+02
+1.911930000000000e+03 +3.115000000000000e+02
+1.322270000000000e+03 +2.075000000000000e+02
+1.042180000000000e+03 +1.765000000000000e+02
+1.951580000000000e+03 +3.710000000000000e+02
+1.924320000000000e+03 +3.490000000000000e+02
+2.129750000000000e+03 +4.185000000000000e+02
+9.344230000000000e+02 +7.200000000000000e+02
+2.535750000000000e+02 +7.195000000000000e+02
+1.322590000000000e+03 +4.310000000000000e+02
+4.913140000000000e+02 +1.260000000000000e+02
+1.673960000000000e+03 +3.070000000000000e+02
+6.962769999999998e+02 +1.430000000000000e+02
+1.853480000000000e+03 +3.265000000000000e+02
+1.148560000000000e+03 +2.730000000000000e+02
+7.423310000000000e+02 +1.070000000000000e+02
+8.913860000000002e+02 +9.900000000000000e+01
+6.418690000000000e+02 +1.470000000000000e+02
+2.278390000000000e+02 +7.180000000000000e+02
+1.314490000000000e+03 +2.300000000000000e+02
+1.380490000000000e+03 +4.425000000000000e+02
+1.951830000000000e+03 +5.325000000000000e+02
+8.568800000000000e+02 +1.915000000000000e+02
+1.858200000000000e+03 +3.580000000000000e+02
+1.607310000000000e+03 +3.445000000000000e+02
+1.166740000000000e+03 +2.745000000000000e+02
+1.205220000000000e+03 +2.685000000000000e+02
+6.537600000000000e+02 +1.455000000000000e+02
+1.952100000000000e+03 +2.925000000000000e+02
+1.060180000000000e+03 +1.740000000000000e+02
+6.403840000000000e+02 +1.500000000000000e+02
+1.794320000000000e+03 +4.195000000000000e+02
+2.312590000000000e+03 +4.065000000000000e+02
+9.737940000000000e+02 +1.200000000000000e+02
+1.578740000000000e+03 +2.010000000000000e+02
+1.768270000000000e+03 +2.995000000000000e+02
+1.284530000000000e+03 +3.160000000000000e+02
+1.428140000000000e+03 +2.095000000000000e+02
+6.249600000000000e+02 +8.450000000000000e+01
+1.307050000000000e+03 +1.300000000000000e+02
+1.256600000000000e+03 +3.010000000000000e+02
+2.853860000000000e+02 +7.150000000000000e+02
+1.422650000000000e+03 +1.340000000000000e+02
+8.590520000000000e+02 +7.145000000000000e+02
+1.856990000000000e+03 +3.985000000000000e+02
+9.643740000000000e+02 +1.160000000000000e+02
+8.985210000000002e+02 +1.000000000000000e+02
+9.892100000000000e+02 +1.895000000000000e+02
+1.380590000000000e+03 +2.005000000000000e+02
+7.409420000000000e+02 +1.020000000000000e+02
+7.764500000000000e+02 +7.135000000000000e+02
+1.368330000000000e+03 +2.860000000000000e+02
+8.969950000000000e+02 +1.385000000000000e+02
+1.065000000000000e+03 +1.785000000000000e+02
+9.325380000000000e+02 +7.130000000000000e+02
+2.348740000000000e+03 +5.155000000000000e+02
+6.956280000000000e+02 +1.405000000000000e+02
+1.114120000000000e+03 +3.670000000000000e+02
+6.473020000000000e+02 +2.515000000000000e+02
+4.950630000000001e+02 +1.235000000000000e+02
+1.433040000000000e+03 +3.100000000000000e+02
+4.887560000000000e+02 +1.245000000000000e+02
+1.323240000000000e+03 +2.465000000000000e+02
+9.000920000000000e+02 +1.335000000000000e+02
+9.897859999999999e+02 +1.830000000000000e+02
+6.411130000000001e+02 +1.445000000000000e+02
+8.636780000000000e+02 +1.280000000000000e+02
+1.004580000000000e+03 +1.840000000000000e+02
+1.570040000000000e+03 +1.830000000000000e+02
+9.840050000000000e+02 +1.860000000000000e+02
+2.818940000000000e+02 +6.250000000000000e+01
+1.428800000000000e+03 +4.295000000000000e+02
+1.003040000000000e+03 +1.890000000000000e+02
+6.350160000000000e+02 +1.470000000000000e+02
+9.492809999999999e+02 +2.225000000000000e+02
+1.552820000000000e+03 +1.795000000000000e+02
+1.116810000000000e+03 +1.405000000000000e+02
+9.256910000000000e+02 +1.355000000000000e+02
+4.831970000000000e+02 +1.230000000000000e+02
+8.258900000000000e+02 +7.085000000000000e+02
+1.878480000000000e+03 +4.100000000000000e+02
+7.711820000000000e+02 +1.000000000000000e+02
+7.806500000000000e+02 +1.005000000000000e+02
+2.027390000000000e+02 +7.075000000000000e+02
+6.250409999999998e+02 +7.850000000000000e+01
+7.643830000000000e+02 +9.350000000000000e+01
+1.068750000000000e+03 +1.740000000000000e+02
+2.691470000000000e+03 +5.930000000000000e+02
+7.890219999999998e+02 +1.025000000000000e+02
+1.296830000000000e+03 +2.235000000000000e+02
+1.390050000000000e+03 +3.460000000000000e+02
+8.981940000000000e+02 +1.305000000000000e+02
+1.417380000000000e+03 +3.060000000000000e+02
+1.472380000000000e+03 +4.185000000000000e+02
+9.911720000000000e+02 +1.910000000000000e+02
+1.057360000000000e+03 +1.795000000000000e+02
+1.869260000000000e+03 +3.930000000000000e+02
+2.206570000000000e+02 +7.055000000000000e+02
+8.625380000000000e+02 +1.055000000000000e+02
+6.566200000000000e+02 +1.445000000000000e+02
+1.292250000000000e+03 +2.135000000000000e+02
+1.052980000000000e+03 +2.785000000000000e+02
+6.827280000000002e+02 +2.415000000000000e+02
+9.887809999999999e+02 +1.345000000000000e+02
+1.659120000000000e+03 +2.705000000000000e+02
+7.571039999999998e+02 +1.480000000000000e+02
+7.891080000000002e+02 +1.015000000000000e+02
+8.486860000000000e+02 +2.860000000000000e+02
+1.859430000000000e+03 +3.700000000000000e+02
+7.663350000000000e+02 +7.035000000000000e+02
+8.696790000000000e+02 +7.035000000000000e+02
+1.876780000000000e+03 +3.425000000000000e+02
+4.824150000000000e+02 +8.550000000000000e+01
+1.861310000000000e+03 +4.325000000000000e+02
+9.144220000000000e+02 +1.300000000000000e+02
+1.961400000000000e+03 +3.820000000000000e+02
+9.310480000000000e+02 +1.435000000000000e+02
+7.670730000000000e+02 +1.185000000000000e+02
+7.666900000000001e+02 +8.500000000000000e+01
+1.364010000000000e+03 +3.410000000000000e+02
+1.946930000000000e+03 +5.190000000000000e+02
+2.031840000000000e+03 +3.645000000000000e+02
+4.892110000000000e+02 +1.110000000000000e+02
+6.751780000000000e+02 +1.280000000000000e+02
+1.081430000000000e+02 +2.550000000000000e+01
+6.892160000000000e+02 +1.570000000000000e+02
+7.437689999999999e+02 +9.100000000000000e+01
+1.058410000000000e+03 +1.735000000000000e+02
+1.097680000000000e+02 +2.250000000000000e+01
+1.289860000000000e+03 +4.085000000000000e+02
+5.872760000000000e+02 +8.800000000000000e+01
+6.704030000000000e+02 +1.500000000000000e+02
+1.154970000000000e+03 +2.585000000000000e+02
+8.823630000000001e+02 +1.270000000000000e+02
+1.388970000000000e+03 +4.145000000000000e+02
+8.461439999999999e+02 +6.990000000000000e+02
+7.762270000000000e+02 +8.900000000000000e+01
+8.930590000000000e+02 +3.900000000000000e+02
+1.573580000000000e+03 +2.720000000000000e+02
+1.382260000000000e+03 +3.430000000000000e+02
+9.190430000000000e+02 +1.355000000000000e+02
+1.129110000000000e+03 +1.690000000000000e+02
+1.065020000000000e+03 +1.600000000000000e+02
+9.315470000000000e+02 +1.325000000000000e+02
+1.024910000000000e+03 +2.665000000000000e+02
+2.765450000000000e+02 +5.200000000000000e+01
+7.883420000000000e+02 +8.950000000000000e+01
+1.299680000000000e+03 +2.110000000000000e+02
+1.044610000000000e+03 +1.305000000000000e+02
+1.873970000000000e+03 +3.270000000000000e+02
+8.547719999999998e+02 +6.965000000000000e+02
+1.893110000000000e+02 +6.965000000000000e+02
+7.726110000000001e+02 +9.500000000000000e+01
+9.787160000000000e+02 +1.625000000000000e+02
+9.155860000000000e+02 +1.725000000000000e+02
+8.673130000000000e+02 +2.190000000000000e+02
+6.142250000000000e+02 +2.425000000000000e+02
+1.079810000000000e+03 +1.580000000000000e+02
+2.229350000000000e+02 +6.960000000000000e+02
+1.188650000000000e+03 +2.640000000000000e+02
+1.342160000000000e+03 +6.960000000000000e+02
+8.913220000000000e+02 +1.500000000000000e+02
+7.495210000000002e+02 +8.900000000000000e+01
+7.918680000000001e+02 +9.350000000000000e+01
+7.883739999999998e+02 +1.655000000000000e+02
+7.538910000000002e+02 +1.425000000000000e+02
+8.763950000000000e+02 +6.950000000000000e+02
+1.948270000000000e+02 +6.950000000000000e+02
+7.469810000000001e+02 +8.500000000000000e+01
+1.352520000000000e+03 +3.415000000000000e+02
+1.077270000000000e+03 +3.770000000000000e+02
+1.137360000000000e+03 +3.360000000000000e+02
+1.846960000000000e+03 +4.280000000000000e+02
+1.010650000000000e+03 +1.715000000000000e+02
+1.744390000000000e+03 +3.745000000000000e+02
+1.426280000000000e+03 +1.865000000000000e+02
+9.915160000000000e+02 +1.515000000000000e+02
+1.380020000000000e+03 +2.950000000000000e+02
+1.594960000000000e+03 +2.765000000000000e+02
+1.421890000000000e+03 +4.155000000000000e+02
+2.729040000000000e+02 +1.935000000000000e+02
+6.532170000000000e+02 +1.280000000000000e+02
+1.298420000000000e+03 +2.145000000000000e+02
+9.071810000000000e+02 +1.335000000000000e+02
+1.872110000000000e+03 +5.395000000000000e+02
+6.908910000000002e+02 +1.225000000000000e+02
+6.969939999999998e+02 +1.380000000000000e+02
+1.071380000000000e+03 +1.535000000000000e+02
+7.812890000000000e+02 +8.700000000000000e+01
+1.556920000000000e+03 +2.620000000000000e+02
+2.512230000000000e+02 +6.915000000000000e+02
+6.424590000000002e+02 +1.325000000000000e+02
+7.899480000000000e+02 +9.150000000000000e+01
+9.963650000000000e+02 +1.640000000000000e+02
+1.554410000000000e+03 +2.585000000000000e+02
+6.560269999999998e+02 +1.260000000000000e+02
+2.666070000000000e+03 +6.560000000000000e+02
+1.376510000000000e+03 +2.840000000000000e+02
+7.756039999999998e+02 +6.890000000000000e+02
+7.581680000000000e+02 +8.250000000000000e+01
+1.298950000000000e+03 +2.120000000000000e+02
+1.064640000000000e+03 +1.275000000000000e+02
+6.920139999999999e+02 +2.145000000000000e+02
+1.983900000000000e+03 +2.740000000000000e+02
+1.010940000000000e+03 +2.775000000000000e+02
+6.530840000000002e+02 +1.315000000000000e+02
+1.866310000000000e+03 +3.050000000000000e+02
+1.564710000000000e+03 +2.540000000000000e+02
+2.287250000000000e+03 +4.345000000000000e+02
+1.868530000000000e+03 +3.190000000000000e+02
+6.881770000000000e+02 +1.170000000000000e+02
+1.130730000000000e+03 +2.750000000000000e+02
+1.489400000000000e+03 +6.460000000000000e+02
+6.935360000000002e+02 +7.250000000000000e+01
+8.976020000000000e+02 +1.215000000000000e+02
+6.724889999999998e+02 +3.370000000000000e+02
+1.479120000000000e+03 +1.925000000000000e+02
+8.606489999999999e+02 +9.150000000000000e+01
+6.542080000000002e+02 +1.265000000000000e+02
+7.899720000000000e+02 +1.555000000000000e+02
+8.975570000000000e+02 +1.990000000000000e+02
+1.463360000000000e+03 +2.960000000000000e+02
+7.790950000000000e+02 +8.250000000000000e+01
+9.883260000000000e+02 +1.600000000000000e+02
+6.078450000000000e+02 +7.350000000000000e+01
+7.879860000000001e+02 +8.300000000000000e+01
+8.990930000000002e+02 +1.280000000000000e+02
+4.902260000000000e+02 +1.000000000000000e+02
+8.252089999999999e+02 +6.845000000000000e+02
+7.823660000000001e+02 +8.750000000000000e+01
+1.483570000000000e+03 +1.890000000000000e+02
+9.753140000000000e+02 +1.000000000000000e+02
+1.755180000000000e+03 +2.975000000000000e+02
+1.075070000000000e+03 +3.600000000000000e+02
+6.602139999999998e+02 +1.265000000000000e+02
+6.755839999999999e+02 +1.130000000000000e+02
+1.141150000000000e+03 +2.400000000000000e+02
+2.778890000000000e+02 +3.950000000000000e+01
+1.780310000000000e+03 +2.700000000000000e+02
+1.405830000000000e+03 +4.095000000000000e+02
+8.838439999999998e+02 +3.690000000000000e+02
+1.000090000000000e+03 +2.730000000000000e+02
+7.574180000000000e+02 +9.300000000000000e+01
+3.261000000000000e+02 +6.000000000000000e+01
+8.963900000000000e+02 +1.145000000000000e+02
+1.127560000000000e+03 +2.075000000000000e+02
+1.942590000000000e+02 +6.815000000000000e+02
+7.832210000000000e+02 +8.250000000000000e+01
+9.002040000000000e+02 +1.270000000000000e+02
+1.410900000000000e+03 +5.180000000000000e+02
+6.175730000000000e+02 +7.150000000000000e+01
+1.006620000000000e+03 +1.650000000000000e+02
+8.599650000000000e+02 +8.800000000000000e+01
+7.773660000000001e+02 +8.550000000000000e+01
+6.056180000000001e+02 +1.545000000000000e+02
+1.311920000000000e+03 +2.080000000000000e+02
+1.334140000000000e+03 +3.305000000000000e+02
+6.911810000000000e+02 +6.600000000000000e+01
+7.879900000000000e+02 +8.300000000000000e+01
+1.213960000000000e+03 +2.580000000000000e+02
+6.618450000000000e+02 +3.360000000000000e+02
+1.072660000000000e+03 +1.440000000000000e+02
+1.002330000000000e+03 +1.425000000000000e+02
+1.913290000000000e+02 +6.790000000000000e+02
+1.450500000000000e+03 +1.850000000000000e+02
+1.078610000000000e+03 +1.470000000000000e+02
+1.291910000000000e+03 +1.905000000000000e+02
+1.001780000000000e+03 +1.550000000000000e+02
+1.093440000000000e+03 +1.415000000000000e+02
+6.844349999999999e+02 +6.200000000000000e+01
+7.841410000000002e+02 +8.450000000000000e+01
+8.965889999999998e+02 +1.250000000000000e+02
+1.872180000000000e+03 +3.185000000000000e+02
+1.307770000000000e+03 +2.180000000000000e+02
+7.850630000000000e+02 +8.450000000000000e+01
+1.313300000000000e+03 +1.985000000000000e+02
+9.467650000000000e+02 +2.215000000000000e+02
+1.007320000000000e+03 +2.685000000000000e+02
+8.772089999999999e+02 +8.950000000000000e+01
+7.442189999999998e+02 +7.700000000000000e+01
+9.921350000000000e+02 +1.510000000000000e+02
+8.962300000000000e+02 +1.095000000000000e+02
+1.859540000000000e+03 +5.250000000000000e+02
+1.001310000000000e+03 +2.230000000000000e+02
+9.018740000000000e+02 +1.165000000000000e+02
+1.048480000000000e+03 +1.490000000000000e+02
+6.746720000000000e+02 +1.380000000000000e+02
+9.371600000000000e+02 +2.210000000000000e+02
+4.662390000000000e+02 +8.450000000000000e+01
+7.925200000000000e+02 +8.500000000000000e+01
+6.611170000000000e+02 +3.345000000000000e+02
+4.853040000000000e+02 +9.450000000000000e+01
+6.373770000000000e+02 +1.195000000000000e+02
+1.885600000000000e+02 +6.740000000000000e+02
+7.608350000000000e+02 +8.150000000000000e+01
+1.952480000000000e+03 +2.520000000000000e+02
+9.156340000000000e+02 +1.115000000000000e+02
+6.853190000000000e+02 +1.450000000000000e+02
+1.086950000000000e+03 +1.430000000000000e+02
+2.281470000000000e+03 +4.190000000000000e+02
+4.934820000000000e+02 +8.900000000000000e+01
+7.832919999999998e+02 +8.150000000000000e+01
+9.545400000000000e+02 +6.725000000000000e+02
+9.893950000000000e+02 +6.720000000000000e+02
+1.066960000000000e+03 +1.010000000000000e+02
+6.544180000000000e+02 +3.305000000000000e+02
+6.539480000000000e+02 +1.175000000000000e+02
+1.306100000000000e+03 +2.100000000000000e+02
+1.294790000000000e+03 +5.060000000000000e+02
+9.045300000000000e+02 +1.205000000000000e+02
+7.505480000000000e+02 +6.705000000000000e+02
+1.308810000000000e+03 +1.715000000000000e+02
+7.805260000000002e+02 +8.350000000000000e+01
+1.300990000000000e+03 +1.930000000000000e+02
+1.580690000000000e+03 +2.445000000000000e+02
+1.252120000000000e+03 +3.845000000000000e+02
+9.254770000000000e+02 +2.195000000000000e+02
+8.672769999999998e+02 +9.200000000000000e+01
+6.894530000000000e+02 +6.050000000000000e+01
+9.621150000000000e+02 +1.370000000000000e+02
+6.394670000000000e+02 +1.165000000000000e+02
+7.882840000000000e+02 +8.000000000000000e+01
+6.921740000000000e+02 +1.370000000000000e+02
+1.013510000000000e+03 +1.325000000000000e+02
+6.731550000000000e+02 +1.000000000000000e+02
+1.304060000000000e+03 +1.705000000000000e+02
+1.301640000000000e+03 +4.975000000000000e+02
+8.614320000000000e+02 +1.335000000000000e+02
+1.580840000000000e+03 +2.450000000000000e+02
+1.143150000000000e+03 +3.770000000000000e+02
+1.061780000000000e+03 +4.745000000000000e+02
+6.940010000000002e+02 +9.750000000000000e+01
+6.908420000000000e+02 +6.800000000000000e+01
+1.319990000000000e+03 +2.035000000000000e+02
+2.410700000000000e+03 +5.740000000000000e+02
+1.287430000000000e+03 +2.710000000000000e+02
+6.527390000000000e+02 +1.145000000000000e+02
+9.250010000000000e+02 +6.665000000000000e+02
+7.891890000000000e+02 +8.200000000000000e+01
+4.506030000000000e+02 +3.450000000000000e+01
+6.955450000000000e+02 +2.300000000000000e+02
+6.363670000000000e+02 +1.035000000000000e+02
+7.461790000000000e+02 +7.500000000000000e+01
+1.586420000000000e+03 +3.600000000000000e+02
+9.386340000000000e+02 +2.190000000000000e+02
+1.304170000000000e+03 +2.585000000000000e+02
+6.950300000000000e+02 +9.800000000000000e+01
+6.245580000000000e+02 +1.445000000000000e+02
+4.635030000000000e+02 +3.550000000000000e+01
+1.051470000000000e+03 +3.400000000000000e+02
+7.658160000000000e+02 +6.650000000000000e+02
+8.996139999999998e+02 +1.150000000000000e+02
+1.277450000000000e+03 +6.650000000000000e+02
+8.586239999999998e+02 +1.325000000000000e+02
+1.107190000000000e+03 +4.455000000000000e+02
+8.700640000000000e+02 +1.360000000000000e+02
+8.590989999999998e+02 +1.315000000000000e+02
+6.498650000000000e+02 +1.040000000000000e+02
+8.576950000000001e+02 +8.200000000000000e+01
+7.769540000000000e+02 +7.750000000000000e+01
+9.328770000000000e+02 +1.445000000000000e+02
+8.937970000000000e+02 +2.115000000000000e+02
+1.000580000000000e+03 +1.080000000000000e+02
+1.368860000000000e+03 +3.800000000000000e+02
+7.757460000000002e+02 +6.630000000000000e+02
+9.832859999999999e+02 +1.075000000000000e+02
+1.013910000000000e+03 +2.520000000000000e+02
+1.077250000000000e+03 +1.335000000000000e+02
+7.447589999999999e+02 +5.900000000000000e+01
+1.037960000000000e+03 +1.940000000000000e+02
+7.810250000000000e+02 +7.850000000000000e+01
+1.311070000000000e+03 +1.865000000000000e+02
+1.208980000000000e+03 +1.290000000000000e+02
+7.530700000000001e+02 +7.650000000000000e+01
+8.835650000000001e+02 +1.675000000000000e+02
+8.316070000000000e+02 +6.615000000000000e+02
+1.331370000000000e+03 +3.050000000000000e+02
+4.932870000000000e+02 +7.900000000000000e+01
+6.576750000000000e+02 +1.000000000000000e+02
+1.805470000000000e+02 +6.610000000000000e+02
+7.772919999999998e+02 +7.850000000000000e+01
+1.522610000000000e+03 +2.345000000000000e+02
+1.155910000000000e+03 +3.225000000000000e+02
+4.855390000000000e+02 +1.710000000000000e+02
+8.116210000000002e+02 +6.605000000000000e+02
+1.563710000000000e+03 +2.330000000000000e+02
+9.072220000000000e+02 +1.135000000000000e+02
+4.960260000000000e+02 +1.770000000000000e+02
+1.699170000000000e+03 +2.315000000000000e+02
+1.455740000000000e+03 +3.430000000000000e+02
+8.761360000000002e+02 +8.800000000000000e+01
+9.342500000000000e+02 +1.450000000000000e+02
+1.918550000000000e+03 +4.290000000000000e+02
+1.910920000000000e+03 +2.970000000000000e+02
+1.809360000000000e+03 +6.600000000000000e+02
+6.767330000000002e+02 +1.055000000000000e+02
+1.004730000000000e+03 +2.550000000000000e+02
+1.078030000000000e+03 +2.345000000000000e+02
+1.556710000000000e+03 +2.355000000000000e+02
+6.555160000000002e+02 +3.120000000000000e+02
+1.016710000000000e+03 +2.045000000000000e+02
+1.140280000000000e+03 +2.940000000000000e+02
+7.868730000000000e+02 +7.800000000000000e+01
+1.316830000000000e+03 +1.920000000000000e+02
+1.778060000000000e+03 +2.420000000000000e+02
+4.800520000000000e+02 +5.900000000000000e+01
+6.237869999999998e+02 +1.365000000000000e+02
+8.702700000000000e+02 +1.105000000000000e+02
+2.409430000000000e+03 +5.700000000000000e+02
+7.749280000000000e+02 +6.575000000000000e+02
+1.157790000000000e+03 +2.135000000000000e+02
+7.895239999999999e+02 +1.515000000000000e+02
+6.128580000000002e+02 +5.150000000000000e+01
+1.748120000000000e+02 +6.570000000000000e+02
+7.977339999999998e+02 +1.525000000000000e+02
+7.819019999999998e+02 +7.150000000000000e+01
+9.380780000000000e+02 +6.560000000000000e+02
+7.949130000000000e+02 +7.550000000000000e+01
+9.936510000000000e+02 +1.330000000000000e+02
+1.448840000000000e+03 +1.650000000000000e+02
+1.687940000000000e+03 +4.045000000000000e+02
+6.455440000000000e+02 +1.075000000000000e+02
+1.073310000000000e+03 +1.995000000000000e+02
+8.583170000000000e+02 +8.650000000000000e+01
+1.555700000000000e+03 +2.255000000000000e+02
+7.061920000000000e+02 +1.855000000000000e+02
+6.672660000000002e+02 +1.050000000000000e+02
+1.083160000000000e+03 +1.255000000000000e+02
+2.782850000000000e+02 +2.050000000000000e+01
+9.237490000000000e+02 +6.545000000000000e+02
+7.602089999999999e+02 +7.050000000000000e+01
+8.780860000000000e+02 +1.200000000000000e+02
+6.598989999999999e+02 +1.005000000000000e+02
+7.911750000000000e+02 +6.540000000000000e+02
+6.243000000000000e+02 +1.345000000000000e+02
+1.102400000000000e+03 +1.730000000000000e+02
+7.903700000000000e+02 +6.540000000000000e+02
+4.945090000000000e+02 +1.625000000000000e+02
+1.314010000000000e+03 +3.735000000000000e+02
+6.350040000000000e+02 +1.035000000000000e+02
+7.910770000000000e+02 +1.500000000000000e+02
+1.046220000000000e+03 +1.750000000000000e+02
+1.851130000000000e+03 +3.960000000000000e+02
+7.589169999999998e+02 +6.650000000000000e+01
+9.067430000000001e+02 +1.355000000000000e+02
+8.738510000000001e+02 +3.455000000000000e+02
+6.999980000000000e+02 +2.110000000000000e+02
+9.000450000000000e+02 +1.140000000000000e+02
+2.853980000000000e+02 +2.150000000000000e+01
+7.369260000000000e+02 +7.150000000000000e+01
+8.939689999999998e+02 +2.740000000000000e+02
+6.899240000000000e+02 +1.795000000000000e+02
+6.861330000000000e+02 +6.100000000000000e+01
+7.690780000000000e+02 +6.520000000000000e+02
+1.864200000000000e+03 +4.010000000000000e+02
+9.808400000000000e+02 +1.950000000000000e+02
+6.680100000000000e+02 +1.035000000000000e+02
+7.919720000000000e+02 +1.450000000000000e+02
+1.215620000000000e+03 +2.620000000000000e+02
+5.018070000000000e+02 +1.705000000000000e+02
+8.833810000000002e+02 +1.630000000000000e+02
+1.041330000000000e+02 +1.150000000000000e+01
+8.825889999999998e+02 +1.090000000000000e+02
+9.195330000000000e+02 +6.505000000000000e+02
+1.316490000000000e+03 +1.685000000000000e+02
+1.308690000000000e+03 +4.815000000000000e+02
+9.969400000000001e+02 +2.090000000000000e+02
+6.586550000000000e+02 +1.370000000000000e+02
+9.755660000000000e+02 +1.325000000000000e+02
+8.939760000000001e+02 +1.795000000000000e+02
+8.996740000000000e+02 +2.710000000000000e+02
+9.547770000000000e+02 +1.220000000000000e+02
+1.740960000000000e+02 +6.495000000000000e+02
+9.056750000000000e+02 +1.860000000000000e+02
+6.688839999999999e+02 +1.030000000000000e+02
+8.490989999999998e+02 +7.400000000000000e+01
+1.055850000000000e+03 +3.120000000000000e+02
+1.863350000000000e+03 +3.360000000000000e+02
+6.190180000000000e+02 +1.315000000000000e+02
+2.340170000000000e+02 +6.485000000000000e+02
+7.756310000000002e+02 +6.900000000000000e+01
+9.390560000000000e+02 +1.225000000000000e+02
+1.405000000000000e+03 +4.870000000000000e+02
+6.515130000000000e+02 +9.950000000000000e+01
+7.607230000000002e+02 +6.480000000000000e+02
+7.931519999999998e+02 +1.450000000000000e+02
+1.066370000000000e+03 +1.605000000000000e+02
+8.859800000000000e+02 +1.085000000000000e+02
+6.746230000000000e+02 +1.105000000000000e+02
+8.931200000000000e+02 +1.095000000000000e+02
+1.007290000000000e+03 +1.970000000000000e+02
+1.092690000000000e+03 +2.245000000000000e+02
+7.862840000000000e+02 +1.435000000000000e+02
+1.054750000000000e+03 +5.470000000000000e+02
+1.080940000000000e+03 +1.115000000000000e+02
+9.093390000000001e+02 +6.465000000000000e+02
+3.936290000000000e+02 +3.050000000000000e+01
+6.907700000000000e+02 +1.440000000000000e+02
+1.574350000000000e+03 +3.165000000000000e+02
+9.166910000000000e+02 +1.590000000000000e+02
+6.295130000000000e+02 +4.215000000000000e+02
+8.824040000000000e+02 +1.560000000000000e+02
+1.055050000000000e+03 +1.595000000000000e+02
+1.416780000000000e+03 +4.860000000000000e+02
+7.781310000000002e+02 +1.390000000000000e+02
+1.697710000000000e+02 +6.455000000000000e+02
+7.899390000000000e+02 +1.430000000000000e+02
+9.775170000000001e+02 +1.265000000000000e+02
+1.108700000000000e+03 +1.615000000000000e+02
+1.773650000000000e+02 +6.450000000000000e+02
+1.006810000000000e+03 +1.590000000000000e+02
+9.007610000000000e+02 +1.125000000000000e+02
+1.058510000000000e+03 +1.200000000000000e+02
+9.186780000000000e+02 +6.445000000000000e+02
+1.874650000000000e+03 +3.335000000000000e+02
+6.197650000000000e+02 +1.245000000000000e+02
+7.610830000000002e+02 +1.375000000000000e+02
+6.478750000000000e+02 +9.950000000000000e+01
+9.377550000000000e+02 +6.440000000000000e+02
+9.765760000000000e+02 +1.285000000000000e+02
+9.109430000000000e+02 +1.700000000000000e+02
+6.538960000000000e+02 +9.700000000000000e+01
+1.295600000000000e+03 +1.705000000000000e+02
+9.257380000000001e+02 +1.565000000000000e+02
+6.708869999999999e+02 +8.550000000000000e+01
+1.006160000000000e+03 +2.280000000000000e+02
+2.297830000000000e+03 +5.135000000000000e+02
+8.697669999999998e+02 +8.600000000000000e+01
+6.550880000000002e+02 +2.970000000000000e+02
+9.988070000000000e+02 +1.625000000000000e+02
+7.882510000000002e+02 +1.415000000000000e+02
+1.463860000000000e+03 +2.675000000000000e+02
+7.477210000000000e+02 +5.350000000000000e+01
+7.525820000000000e+02 +6.300000000000000e+01
+1.404380000000000e+03 +4.760000000000000e+02
+6.488850000000000e+02 +4.145000000000000e+02
+1.584330000000000e+03 +2.615000000000000e+02
+6.598730000000000e+02 +9.850000000000000e+01
+6.844750000000000e+02 +1.045000000000000e+02
+4.821550000000000e+02 +1.010000000000000e+02
+1.081750000000000e+03 +2.170000000000000e+02
+1.083050000000000e+03 +1.865000000000000e+02
+6.248099999999999e+02 +1.250000000000000e+02
+1.660940000000001e+02 +6.405000000000000e+02
+1.311140000000000e+03 +1.695000000000000e+02
+1.404320000000000e+03 +5.965000000000000e+02
+4.941960000000000e+02 +1.535000000000000e+02
+1.396000000000000e+03 +1.560000000000000e+02
+6.949380000000000e+02 +1.050000000000000e+02
+1.531050000000000e+02 +6.400000000000000e+02
+8.022530000000000e+02 +1.310000000000000e+02
+2.824030000000000e+02 +2.100000000000000e+01
+1.066500000000000e+03 +1.955000000000000e+02
+8.717810000000002e+02 +7.350000000000000e+01
+1.281710000000000e+03 +3.165000000000000e+02
+1.847380000000000e+03 +2.535000000000000e+02
+9.130370000000000e+02 +1.615000000000000e+02
+4.595590000000000e+02 +6.385000000000000e+02
+1.284930000000000e+03 +3.605000000000000e+02
+6.628120000000000e+02 +9.750000000000000e+01
+1.960140000000000e+03 +3.355000000000000e+02
+1.063990000000000e+03 +5.425000000000000e+02
+1.217760000000000e+03 +1.915000000000000e+02
+1.045450000000000e+03 +2.000000000000000e+02
+7.547790000000000e+02 +6.050000000000000e+01
+1.588790000000000e+03 +3.350000000000000e+02
+1.247180000000000e+03 +2.175000000000000e+02
+1.485350000000000e+03 +3.685000000000000e+02
+1.115710000000000e+03 +2.705000000000000e+02
+1.338120000000000e+03 +3.185000000000000e+02
+1.184070000000000e+03 +2.900000000000000e+02
+7.862460000000002e+02 +1.400000000000000e+02
+7.921890000000000e+02 +1.100000000000000e+02
+6.893450000000000e+02 +2.230000000000000e+02
+8.418270000000000e+02 +6.360000000000000e+02
+6.808510000000001e+02 +1.705000000000000e+02
+1.081100000000000e+03 +2.155000000000000e+02
+8.566039999999998e+02 +6.355000000000000e+02
+1.552740000000000e+02 +6.355000000000000e+02
+7.816799999999999e+02 +1.365000000000000e+02
+9.761110000000000e+02 +1.225000000000000e+02
+8.778900000000000e+02 +1.500000000000000e+02
+8.206350000000000e+02 +6.350000000000000e+02
+7.965230000000000e+02 +1.380000000000000e+02
+1.103780000000000e+03 +1.515000000000000e+02
+9.483920000000001e+02 +2.945000000000000e+02
+8.722260000000001e+02 +3.245000000000000e+02
+6.355459999999998e+02 +9.100000000000000e+01
+6.901130000000001e+02 +1.390000000000000e+02
+7.848800000000000e+02 +1.390000000000000e+02
+1.323290000000000e+03 +3.190000000000000e+02
+1.025060000000000e+03 +3.045000000000000e+02
+7.651030000000002e+02 +1.350000000000000e+02
+7.902360000000001e+02 +6.335000000000000e+02
+8.711080000000002e+02 +7.550000000000000e+01
+1.695870000000000e+03 +3.845000000000000e+02
+7.924989999999998e+02 +1.370000000000000e+02
+8.822569999999999e+02 +1.860000000000000e+02
+1.644070000000000e+03 +3.850000000000000e+02
+1.863390000000000e+03 +3.340000000000000e+02
+1.298440000000000e+03 +2.460000000000000e+02
+1.607190000000000e+03 +3.280000000000000e+02
+6.982439999999998e+02 +9.900000000000000e+01
+1.013390000000000e+03 +2.245000000000000e+02
+1.040350000000000e+03 +2.055000000000000e+02
+6.478390000000001e+02 +9.200000000000000e+01
+8.796750000000000e+02 +3.190000000000000e+02
+1.122000000000000e+03 +2.255000000000000e+02
+7.875150000000000e+02 +1.320000000000000e+02
+1.292180000000000e+03 +5.790000000000000e+02
+1.082230000000000e+03 +2.090000000000000e+02
+9.901330000000000e+02 +1.570000000000000e+02
+7.955110000000002e+02 +6.300000000000000e+02
+2.797730000000000e+02 +1.025000000000000e+02
+1.364190000000000e+03 +3.835000000000000e+02
+8.731489999999999e+02 +1.520000000000000e+02
+1.082940000000000e+03 +1.590000000000000e+02
+7.770520000000000e+02 +6.290000000000000e+02
+8.613389999999998e+02 +9.950000000000000e+01
+7.016960000000000e+02 +1.750000000000000e+02
+1.211050000000000e+03 +2.010000000000000e+02
+9.899600000000000e+02 +2.210000000000000e+02
+8.792410000000001e+02 +6.280000000000000e+02
+7.455889999999998e+02 +5.150000000000000e+01
+6.806799999999999e+02 +1.730000000000000e+02
+1.089550000000000e+03 +2.080000000000000e+02
+1.613290000000000e+03 +3.645000000000000e+02
+7.805960000000000e+02 +9.900000000000000e+01
+1.473860000000000e+03 +3.560000000000000e+02
+1.019910000000000e+03 +1.970000000000000e+02
+1.171510000000000e+03 +3.785000000000000e+02
+1.344600000000000e+03 +4.905000000000000e+02
+7.555350000000000e+02 +1.025000000000000e+02
+9.956319999999999e+02 +7.650000000000000e+01
+6.413810000000000e+02 +8.150000000000000e+01
+1.541690000000000e+02 +6.265000000000000e+02
+9.030549999999999e+02 +1.675000000000000e+02
+6.632689999999999e+02 +8.450000000000000e+01
+1.856410000000000e+03 +3.230000000000000e+02
+9.851420000000001e+02 +1.150000000000000e+02
+5.915369999999998e+02 +9.350000000000000e+01
+6.883439999999998e+02 +1.270000000000000e+02
+7.861319999999999e+02 +1.290000000000000e+02
+9.104610000000000e+02 +1.120000000000000e+02
+8.794180000000000e+02 +1.410000000000000e+02
+7.118539999999998e+02 +6.250000000000000e+02
+8.043739999999998e+02 +6.250000000000000e+02
+1.855000000000000e+03 +3.455000000000000e+02
+1.598670000000000e+03 +4.335000000000000e+02
+6.549470000000000e+02 +2.005000000000000e+02
+7.477339999999998e+02 +6.245000000000000e+02
+7.812960000000000e+02 +1.305000000000000e+02
+1.592860000000000e+03 +3.235000000000000e+02
+1.288520000000000e+03 +3.540000000000000e+02
+4.722520000000000e+02 +4.450000000000000e+01
+9.079520000000000e+02 +1.720000000000000e+02
+7.947739999999999e+02 +2.050000000000000e+02
+9.493440000000001e+02 +6.235000000000000e+02
+8.086820000000000e+02 +6.235000000000000e+02
+1.478000000000000e+02 +6.235000000000000e+02
+1.160050000000000e+03 +9.600000000000000e+01
+9.907960000000000e+02 +1.145000000000000e+02
+1.078240000000000e+03 +1.410000000000000e+02
+1.847870000000000e+03 +4.870000000000000e+02
+6.210740000000002e+02 +8.300000000000000e+01
+7.836860000000000e+02 +1.285000000000000e+02
+1.995830000000000e+03 +3.305000000000000e+02
+9.099950000000000e+02 +1.515000000000000e+02
+6.201870000000000e+02 +6.650000000000000e+01
+2.314230000000000e+03 +4.610000000000000e+02
+1.846820000000000e+03 +4.865000000000000e+02
+6.575080000000000e+02 +8.350000000000000e+01
+7.587980000000000e+02 +6.215000000000000e+02
+6.933980000000000e+02 +1.210000000000000e+02
+7.944370000000000e+02 +1.270000000000000e+02
+9.169310000000000e+02 +1.515000000000000e+02
+1.099920000000000e+03 +4.085000000000000e+02
+1.386840000000000e+03 +5.720000000000000e+02
+1.953390000000000e+03 +3.190000000000000e+02
+1.421030000000000e+03 +5.810000000000000e+02
+8.333730000000000e+02 +1.715000000000000e+02
+4.875220000000000e+02 +1.415000000000000e+02
+6.672280000000002e+02 +7.450000000000000e+01
+8.137160000000000e+02 +6.205000000000000e+02
+1.080200000000000e+02 +1.350000000000000e+01
+6.701189999999998e+02 +3.905000000000000e+02
+1.411320000000000e+03 +2.345000000000000e+02
+8.599680000000002e+02 +1.565000000000000e+02
+6.224420000000000e+02 +1.060000000000000e+02
+1.173250000000000e+02 +1.500000000000000e+01
+1.143810000000000e+03 +2.950000000000000e+02
+8.405150000000000e+02 +6.550000000000000e+01
+9.822140000000001e+02 +1.500000000000000e+02
+1.057450000000000e+03 +1.970000000000000e+02
+7.783510000000001e+02 +1.270000000000000e+02
+6.302950000000000e+02 +1.010000000000000e+02
+1.551530000000000e+03 +1.890000000000000e+02
+6.446930000000000e+02 +1.000000000000000e+02
+1.753160000000000e+03 +4.225000000000000e+02
+4.452200000000000e+02 +8.200000000000000e+01
+1.430010000000000e+03 +2.465000000000000e+02
+1.862790000000000e+03 +4.820000000000000e+02
+1.008580000000000e+03 +2.115000000000000e+02
+6.600880000000002e+02 +7.900000000000000e+01
+1.055500000000000e+03 +2.200000000000000e+02
+7.283760000000002e+02 +6.160000000000000e+02
+2.333310000000000e+02 +6.160000000000000e+02
+9.587940000000000e+02 +1.485000000000000e+02
+1.009370000000000e+03 +1.575000000000000e+02
+1.752730000000000e+03 +2.995000000000000e+02
+6.604490000000000e+02 +6.300000000000000e+01
+1.858040000000000e+03 +3.080000000000000e+02
+7.807400000000000e+02 +1.225000000000000e+02
+1.339080000000000e+03 +3.725000000000000e+02
+9.524850000000000e+02 +2.780000000000000e+02
+7.581460000000002e+02 +6.145000000000000e+02
+7.807339999999998e+02 +1.250000000000000e+02
+1.536240000000000e+03 +2.815000000000000e+02
+1.290190000000000e+03 +5.700000000000000e+02
+6.610300000000000e+02 +6.950000000000000e+01
+7.296630000000000e+02 +6.140000000000000e+02
+1.347220000000000e+03 +3.000000000000000e+02
+2.815090000000000e+02 +8.650000000000000e+01
+6.867420000000000e+02 +1.600000000000000e+02
+7.893480000000002e+02 +1.250000000000000e+02
+1.141340000000000e+03 +2.970000000000000e+02
+7.939939999999998e+02 +6.120000000000000e+02
+7.530660000000000e+02 +1.195000000000000e+02
+1.296850000000000e+03 +1.445000000000000e+02
+9.783780000000000e+02 +1.485000000000000e+02
+6.223230000000000e+02 +9.650000000000000e+01
+7.739920000000000e+02 +1.205000000000000e+02
+1.296930000000000e+03 +2.305000000000000e+02
+9.691090000000000e+02 +1.460000000000000e+02
+1.363580000000000e+02 +6.105000000000000e+02
+3.250260000000000e+02 +2.300000000000000e+01
+1.047330000000000e+03 +5.175000000000000e+02
+1.423610000000000e+03 +2.395000000000000e+02
+1.751280000000000e+03 +3.540000000000000e+02
+6.456180000000001e+02 +1.575000000000000e+02
+1.142340000000000e+03 +2.795000000000000e+02
+8.920970000000000e+02 +1.610000000000000e+02
+6.138090000000000e+02 +5.850000000000000e+01
+7.810230000000000e+02 +1.185000000000000e+02
+1.097180000000000e+03 +1.275000000000000e+02
+1.137120000000000e+03 +3.285000000000000e+02
+6.620880000000002e+02 +4.760000000000000e+02
+7.025810000000000e+02 +6.085000000000000e+02
+6.629450000000001e+02 +9.700000000000000e+01
+1.759090000000000e+03 +4.690000000000000e+02
+9.992820000000000e+02 +1.990000000000000e+02
+6.608070000000000e+02 +1.620000000000000e+02
+7.951540000000000e+02 +6.080000000000000e+02
+7.798080000000000e+02 +1.215000000000000e+02
+1.834840000000000e+03 +4.750000000000000e+02
+7.877360000000001e+02 +1.180000000000000e+02
+1.722570000000000e+03 +3.040000000000000e+02
+1.385640000000000e+03 +4.100000000000000e+02
+4.939470000000000e+02 +1.305000000000000e+02
+7.003860000000002e+02 +1.580000000000000e+02
+1.641360000000000e+03 +3.515000000000000e+02
+8.234520000000000e+02 +6.070000000000000e+02
+1.057430000000000e+01 +6.070000000000000e+02
+7.458980000000000e+02 +5.250000000000000e+01
+8.827339999999998e+02 +3.490000000000000e+02
+9.836200000000000e+02 +1.455000000000000e+02
+1.014500000000000e+03 +2.130000000000000e+02
+1.309480000000000e+03 +2.600000000000000e+02
+1.324520000000000e+03 +3.615000000000000e+02
+9.130030000000000e+02 +1.560000000000000e+02
+1.344440000000000e+03 +2.780000000000000e+02
+6.316170000000000e+02 +9.300000000000000e+01
+7.912170000000000e+02 +1.165000000000000e+02
+1.307020000000000e+03 +3.360000000000000e+02
+1.856920000000000e+03 +3.265000000000000e+02
+9.450910000000000e+02 +7.900000000000000e+01
+1.869440000000000e+03 +4.830000000000000e+02
+6.300890000000001e+02 +4.735000000000000e+02
+8.970829999999999e+01 +6.040000000000000e+02
+7.938410000000000e+02 +1.845000000000000e+02
+7.467739999999999e+02 +1.215000000000000e+02
+9.951799999999999e+02 +1.840000000000000e+02
+7.870599999999999e+02 +1.215000000000000e+02
+1.863910000000000e+03 +4.730000000000000e+02
+1.070630000000000e+03 +3.125000000000000e+02
+7.142580000000000e+02 +1.890000000000000e+02
+6.911590000000000e+02 +1.550000000000000e+02
+1.906070000000000e+02 +6.025000000000000e+02
+1.296660000000000e+03 +4.355000000000000e+02
+8.455050000000000e+02 +7.300000000000000e+01
+6.950580000000000e+02 +2.595000000000000e+02
+1.002760000000000e+03 +2.485000000000000e+02
+1.855460000000000e+03 +3.250000000000000e+02
+7.552810000000002e+02 +1.185000000000000e+02
+7.446469999999998e+02 +6.015000000000000e+02
+1.834930000000000e+02 +6.015000000000000e+02
+9.833380000000000e+02 +1.105000000000000e+02
+1.189730000000000e+03 +2.110000000000000e+02
+8.966220000000000e+02 +3.340000000000000e+02
+7.808760000000002e+02 +1.150000000000000e+02
+1.525420000000000e+03 +2.860000000000000e+02
+8.387230000000002e+02 +6.850000000000000e+01
+1.407330000000000e+03 +5.655000000000000e+02
+1.322060000000000e+03 +2.130000000000000e+02
+1.877420000000000e+03 +3.255000000000000e+02
+1.532320000000000e+03 +3.485000000000000e+02
+8.673939999999999e+02 +6.500000000000000e+01
+6.796940000000000e+02 +1.560000000000000e+02
+1.921080000000000e+03 +3.580000000000000e+02
+1.907810000000000e+03 +3.405000000000000e+02
+6.370119999999999e+02 +1.520000000000000e+02
+7.920430000000000e+02 +1.565000000000000e+02
+6.188660000000000e+02 +8.750000000000000e+01
+7.892930000000000e+02 +1.130000000000000e+02
+8.805360000000002e+02 +1.850000000000000e+02
+6.969019999999998e+02 +1.455000000000000e+02
+1.522840000000000e+03 +1.440000000000000e+02
+8.118510000000001e+02 +1.815000000000000e+02
+6.370840000000002e+02 +1.535000000000000e+02
+1.327540000000000e+03 +2.525000000000000e+02
+9.775860000000000e+02 +1.190000000000000e+02
+1.031970000000000e+03 +1.215000000000000e+02
+5.825850000000000e+02 +1.005000000000000e+02
+9.726830000000000e+02 +1.830000000000000e+02
+8.492030000000000e+02 +7.000000000000000e+01
+1.204480000000000e+03 +1.835000000000000e+02
+6.207619999999999e+02 +5.965000000000000e+02
+1.486230000000000e+02 +5.965000000000000e+02
+1.322980000000000e+03 +2.765000000000000e+02
+4.874680000000000e+02 +1.205000000000000e+02
+2.921790000000001e+02 +1.110000000000000e+02
+1.496970000000000e+03 +3.340000000000000e+02
+9.757130000000000e+02 +1.150000000000000e+02
+1.376310000000000e+03 +3.475000000000000e+02
+7.007010000000000e+02 +1.440000000000000e+02
+1.364930000000000e+02 +5.955000000000000e+02
+2.384680000000000e+03 +4.060000000000000e+02
+6.775219999999998e+02 +1.200000000000000e+02
+1.365580000000000e+03 +5.475000000000000e+02
+1.364730000000000e+03 +2.620000000000000e+02
+6.356010000000000e+02 +1.410000000000000e+02
+1.040850000000000e+03 +2.120000000000000e+02
+6.839520000000000e+02 +1.500000000000000e+02
+6.770440000000000e+02 +1.170000000000000e+02
+1.942340000000000e+02 +5.940000000000000e+02
+6.393400000000000e+02 +1.510000000000000e+02
+1.852740000000000e+03 +3.895000000000000e+02
+1.668470000000000e+03 +3.025000000000000e+02
+1.285150000000000e+03 +3.230000000000000e+02
+1.008840000000000e+03 +1.855000000000000e+02
+7.137589999999999e+02 +5.925000000000000e+02
+1.467750000000000e+02 +5.925000000000000e+02
+1.922450000000000e+03 +2.925000000000000e+02
+8.633160000000000e+02 +1.220000000000000e+02
+9.132360000000000e+02 +1.885000000000000e+02
+9.849290000000000e+02 +1.830000000000000e+02
+2.753600000000000e+02 +6.650000000000000e+01
+1.125590000000000e+03 +2.640000000000000e+02
+1.816460000000000e+03 +2.860000000000000e+02
+7.848789999999998e+02 +5.915000000000000e+02
+1.092240000000000e+02 +3.150000000000000e+01
+8.761760000000000e+02 +5.025000000000000e+02
+6.706660000000001e+02 +1.180000000000000e+02
+6.733670000000000e+02 +1.415000000000000e+02
+1.861270000000000e+03 +3.870000000000000e+02
+1.705490000000000e+02 +5.910000000000000e+02
+9.756220000000000e+02 +1.770000000000000e+02
+4.542820000000000e+02 +6.350000000000000e+01
+1.364280000000000e+03 +5.375000000000000e+02
+7.801419999999998e+02 +5.900000000000000e+02
+1.375910000000000e+02 +5.900000000000000e+02
+7.718770000000000e+02 +1.100000000000000e+02
+1.602360000000000e+03 +2.900000000000000e+02
+6.651980000000000e+02 +4.680000000000000e+02
+1.087220000000000e+03 +1.755000000000000e+02
+6.405319999999998e+02 +1.400000000000000e+02
+8.819270000000000e+02 +1.355000000000000e+02
+1.333890000000000e+03 +5.895000000000000e+02
+1.834990000000000e+03 +4.585000000000000e+02
+9.985880000000000e+02 +1.820000000000000e+02
+7.742990000000000e+02 +5.890000000000000e+02
+9.784480000000000e+02 +1.090000000000000e+02
+6.122010000000000e+02 +6.000000000000000e+01
+2.816580000000000e+02 +6.500000000000000e+01
+1.286620000000000e+03 +3.400000000000000e+02
+1.616280000000000e+03 +3.230000000000000e+02
+7.767089999999999e+02 +1.040000000000000e+02
+4.699100000000000e+02 +1.100000000000000e+02
+8.417400000000000e+02 +1.065000000000000e+02
+1.460740000000000e+03 +4.225000000000000e+02
+8.115039999999998e+02 +1.475000000000000e+02
+1.066400000000000e+03 +1.080000000000000e+02
+1.628200000000000e+03 +3.180000000000000e+02
+1.873670000000000e+03 +3.155000000000000e+02
+1.158880000000000e+02 +6.300000000000000e+01
+1.372450000000000e+03 +3.425000000000000e+02
+6.815989999999998e+02 +1.465000000000000e+02
+1.406780000000000e+03 +2.565000000000000e+02
+6.485800000000000e+02 +1.460000000000000e+02
+7.397900000000000e+02 +5.865000000000000e+02
+1.163340000000000e+02 +5.865000000000000e+02
+7.738000000000000e+02 +1.075000000000000e+02
+8.090160000000002e+02 +1.760000000000000e+02
+8.384310000000000e+02 +1.720000000000000e+02
+1.075360000000000e+03 +1.645000000000000e+02
+6.645630000000000e+02 +7.750000000000000e+01
+7.874180000000000e+02 +1.090000000000000e+02
+1.057910000000000e+03 +1.060000000000000e+02
+1.797020000000000e+03 +4.660000000000000e+02
+7.854119999999998e+02 +1.030000000000000e+02
+1.864040000000000e+03 +4.605000000000000e+02
+1.070270000000000e+03 +1.690000000000000e+02
+1.235290000000000e+02 +5.850000000000000e+02
+7.458589999999998e+02 +1.000000000000000e+02
+9.247340000000000e+02 +1.890000000000000e+02
+6.233900000000000e+02 +7.500000000000000e+01
+1.164400000000000e+03 +1.695000000000000e+02
+7.253339999999999e+02 +5.835000000000000e+02
+7.391619999999998e+02 +5.835000000000000e+02
+3.258570000000000e+02 +1.010000000000000e+02
+1.394630000000000e+03 +2.155000000000000e+02
+6.441910000000000e+02 +1.355000000000000e+02
+1.159600000000000e+02 +5.830000000000000e+02
+1.159860000000000e+03 +2.505000000000000e+02
+1.010890000000000e+03 +1.725000000000000e+02
+7.943830000000000e+02 +1.065000000000000e+02
+6.899260000000000e+02 +2.410000000000000e+02
+7.457339999999998e+02 +1.015000000000000e+02
+6.127640000000000e+02 +1.335000000000000e+02
+1.073200000000000e+03 +1.705000000000000e+02
+7.811619999999998e+02 +5.815000000000000e+02
+7.094490000000000e+02 +5.815000000000000e+02
+6.248360000000000e+02 +7.000000000000000e+01
+1.269050000000000e+02 +5.815000000000000e+02
+1.111200000000000e+03 +1.845000000000000e+02
+7.520110000000002e+02 +1.445000000000000e+02
+9.025390000000000e+02 +1.320000000000000e+02
+1.154900000000000e+03 +5.805000000000000e+02
+6.358700000000000e+02 +1.365000000000000e+02
+1.553850000000000e+03 +2.775000000000000e+02
+1.857540000000000e+03 +3.040000000000000e+02
+4.878150000000000e+02 +1.080000000000000e+02
+7.434440000000000e+02 +5.795000000000000e+02
+1.315850000000000e+03 +3.450000000000000e+02
+7.753150000000001e+02 +1.000000000000000e+02
+7.153750000000000e+02 +5.790000000000000e+02
+7.175310000000002e+02 +5.790000000000000e+02
+1.092400000000000e+02 +5.790000000000000e+02
+1.497790000000000e+03 +2.075000000000000e+02
+1.171590000000000e+03 +5.785000000000000e+02
+4.818390000000000e+02 +1.055000000000000e+02
+1.069500000000000e+03 +1.600000000000000e+02
+9.059230000000000e+02 +3.800000000000000e+02
+9.110970000000000e+02 +1.800000000000000e+02
+1.073160000000000e+03 +1.590000000000000e+02
+7.070700000000001e+02 +5.775000000000000e+02
+7.913480000000002e+02 +5.775000000000000e+02
+7.885380000000000e+02 +1.035000000000000e+02
+1.673200000000000e+03 +2.845000000000000e+02
+8.969150000000000e+02 +1.095000000000000e+02
+6.582330000000002e+02 +4.585000000000000e+02
+4.987620000000000e+02 +1.075000000000000e+02
+1.373820000000000e+03 +3.380000000000000e+02
+1.713750000000000e+03 +3.305000000000000e+02
+2.386740000000000e+03 +5.135000000000000e+02
+1.302920000000000e+03 +2.260000000000000e+02
+1.445480000000000e+03 +2.980000000000000e+02
+1.119840000000000e+03 +2.950000000000000e+02
+9.006590000000000e+02 +4.900000000000000e+02
+9.817080000000000e+02 +1.690000000000000e+02
+7.456230000000000e+02 +9.700000000000000e+01
+7.222489999999998e+02 +5.760000000000000e+02
+1.722860000000000e+03 +3.820000000000000e+02
+6.944100000000000e+02 +1.575000000000000e+02
+8.451189999999998e+02 +1.145000000000000e+02
+9.211100000000000e+02 +1.115000000000000e+02
+1.459300000000000e+03 +2.985000000000000e+02
+1.603440000000000e+03 +3.235000000000000e+02
+7.196220000000000e+02 +5.750000000000000e+02
+6.249180000000000e+02 +6.600000000000000e+01
+9.955870000000000e+01 +5.750000000000000e+02
+7.816389999999999e+02 +1.005000000000000e+02
+9.856590000000000e+02 +1.975000000000000e+02
+7.039750000000000e+02 +1.510000000000000e+02
+7.466519999999998e+02 +9.700000000000000e+01
+7.313439999999998e+02 +5.740000000000000e+02
+1.306310000000000e+03 +1.885000000000000e+02
+1.991750000000000e+03 +2.795000000000000e+02
+1.557610000000000e+03 +2.700000000000000e+02
+8.612270000000000e+02 +2.020000000000000e+02
+6.512450000000000e+02 +1.305000000000000e+02
+7.093220000000000e+02 +5.735000000000000e+02
+1.869610000000000e+03 +5.470000000000000e+02
+9.811920000000000e+02 +1.660000000000000e+02
+9.891040000000000e+01 +5.725000000000000e+02
+9.874250000000000e+02 +1.845000000000000e+02
+1.070350000000000e+03 +1.775000000000000e+02
+6.916189999999998e+02 +2.210000000000000e+02
+1.524980000000000e+03 +2.615000000000000e+02
+1.316230000000000e+03 +2.275000000000000e+02
+7.832880000000000e+02 +1.005000000000000e+02
+1.393470000000000e+03 +3.645000000000000e+02
+6.990230000000000e+02 +5.710000000000000e+02
+1.188830000000000e+02 +5.710000000000000e+02
+7.459860000000001e+02 +9.500000000000000e+01
+1.701660000000000e+03 +2.705000000000000e+02
+1.168140000000000e+03 +5.705000000000000e+02
+8.587410000000001e+02 +2.015000000000000e+02
+8.899770000000000e+02 +5.705000000000000e+02
+1.789040000000000e+03 +4.885000000000000e+02
+4.829530000000000e+02 +1.025000000000000e+02
+6.213530000000002e+02 +6.300000000000000e+01
+9.156910000000000e+02 +1.760000000000000e+02
+4.949270000000000e+02 +1.950000000000000e+02
+8.587030000000000e+02 +1.950000000000000e+02
+8.075150000000000e+02 +2.185000000000000e+02
+5.006640000000000e+02 +8.300000000000000e+01
+1.034200000000000e+03 +1.635000000000000e+02
+9.068160000000000e+02 +1.065000000000000e+02
+7.145920000000000e+02 +5.685000000000000e+02
+6.800770000000000e+02 +5.685000000000000e+02
+7.708580000000002e+02 +9.700000000000000e+01
+1.658100000000000e+03 +2.865000000000000e+02
+1.224100000000000e+03 +2.800000000000000e+02
+1.761340000000000e+03 +5.685000000000000e+02
+5.005240000000000e+02 +1.885000000000000e+02
+1.675300000000000e+03 +2.790000000000000e+02
+9.394650000000000e+02 +1.845000000000000e+02
+6.599290000000000e+02 +5.675000000000000e+02
+7.741920000000000e+02 +9.900000000000000e+01
+1.302580000000000e+03 +2.000000000000000e+02
+1.097670000000000e+03 +1.715000000000000e+02
+1.869230000000000e+03 +5.440000000000000e+02
+5.036340000000000e+02 +1.985000000000000e+02
+9.197960000000000e+01 +5.670000000000000e+02
+1.466680000000000e+03 +2.630000000000000e+02
+1.130210000000000e+03 +2.405000000000000e+02
+1.743070000000000e+03 +3.675000000000000e+02
+6.383670000000000e+02 +1.220000000000000e+02
+9.956460000000000e+02 +1.650000000000000e+02
+1.148010000000000e+03 +5.660000000000000e+02
+6.986450000000000e+02 +2.605000000000000e+02
+7.830510000000000e+02 +9.550000000000000e+01
+9.943660000000000e+02 +1.825000000000000e+02
+1.973340000000000e+03 +3.725000000000000e+02
+5.056770000000000e+02 +2.020000000000000e+02
+7.998560000000001e+02 +1.570000000000000e+02
+7.807200000000000e+02 +9.850000000000000e+01
+1.292280000000000e+03 +1.995000000000000e+02
+1.869790000000000e+03 +5.400000000000000e+02
+6.503240000000002e+02 +5.650000000000000e+02
+7.592600000000000e+02 +9.550000000000000e+01
+9.152320000000000e+02 +1.065000000000000e+02
+9.095890000000001e+02 +1.745000000000000e+02
+1.306710000000000e+03 +2.915000000000000e+02
+1.073940000000000e+03 +2.625000000000000e+02
+8.529920000000000e+02 +5.645000000000000e+02
+1.856280000000000e+03 +3.645000000000000e+02
+9.864780000000000e+02 +1.545000000000000e+02
+1.313720000000000e+03 +1.940000000000000e+02
+1.085030000000000e+03 +2.610000000000000e+02
+6.374230000000000e+02 +1.265000000000000e+02
+6.228500000000000e+02 +5.550000000000000e+01
+1.390050000000000e+03 +3.280000000000000e+02
+1.745180000000000e+03 +2.675000000000000e+02
+1.874400000000000e+03 +5.425000000000000e+02
+7.843530000000002e+02 +9.800000000000000e+01
+1.381660000000000e+03 +3.590000000000000e+02
+8.785590000000000e+02 +1.065000000000000e+02
+5.011920000000000e+02 +1.920000000000000e+02
+1.035190000000000e+03 +2.135000000000000e+02
+9.349930000000001e+01 +5.630000000000000e+02
+7.950250000000000e+02 +9.900000000000000e+01
+6.508040000000000e+02 +5.625000000000000e+02
+1.872860000000000e+03 +3.970000000000000e+02
+1.308570000000000e+03 +2.050000000000000e+02
+1.002620000000000e+03 +2.840000000000000e+02
+4.753100000000000e+02 +1.850000000000000e+02
+6.310620000000000e+02 +5.620000000000000e+02
+8.140009999999999e+01 +5.620000000000000e+02
+7.784530000000000e+02 +9.450000000000000e+01
+1.183730000000000e+03 +5.615000000000000e+02
+6.165500000000000e+02 +5.615000000000000e+02
+6.032310000000000e+02 +5.615000000000000e+02
+1.074110000000000e+03 +2.550000000000000e+02
+7.951330000000000e+02 +9.950000000000000e+01
+1.609390000000000e+03 +3.010000000000000e+02
+1.493480000000000e+03 +2.945000000000000e+02
+1.206430000000000e+03 +2.530000000000000e+02
+4.960080000000000e+02 +1.935000000000000e+02
+1.013790000000000e+03 +1.630000000000000e+02
+6.709639999999998e+02 +1.800000000000000e+02
+1.864860000000000e+03 +3.640000000000000e+02
+7.436510000000002e+02 +8.450000000000000e+01
+9.991840000000001e+01 +5.595000000000000e+02
+9.891900000000001e+02 +1.615000000000000e+02
+7.964650000000000e+02 +9.400000000000000e+01
+8.943200000000001e+02 +1.190000000000000e+02
+8.318439999999998e+02 +5.585000000000000e+02
+7.040710000000000e+02 +5.585000000000000e+02
+1.632420000000000e+03 +2.745000000000000e+02
+1.704760000000000e+03 +3.100000000000000e+02
+1.839300000000000e+03 +3.580000000000000e+02
+7.798400000000000e+02 +9.150000000000000e+01
+9.923370000000000e+02 +1.540000000000000e+02
+9.021180000000001e+02 +1.120000000000000e+02
+1.277850000000000e+03 +2.535000000000000e+02
+8.895360000000002e+02 +2.280000000000000e+02
+1.042460000000000e+03 +1.665000000000000e+02
+2.003500000000000e+03 +3.730000000000000e+02
+6.644000000000000e+02 +1.785000000000000e+02
+7.025000000000000e+02 +1.360000000000000e+02
+4.678090000000000e+02 +8.900000000000000e+01
+1.297370000000000e+03 +3.165000000000000e+02
+6.363220000000000e+02 +1.100000000000000e+02
+1.850560000000000e+03 +5.565000000000000e+02
+1.872500000000000e+03 +3.535000000000000e+02
+1.491480000000000e+03 +3.420000000000000e+02
+9.900340000000000e+02 +1.545000000000000e+02
+1.563970000000000e+03 +2.240000000000000e+02
+1.316870000000000e+03 +1.910000000000000e+02
+7.029230000000000e+02 +1.365000000000000e+02
+1.366050000000000e+03 +4.230000000000000e+02
+1.068020000000000e+03 +2.570000000000000e+02
+8.706110000000000e+01 +5.545000000000000e+02
+1.576650000000000e+03 +3.115000000000000e+02
+6.789050000000000e+02 +5.540000000000000e+02
+6.781880000000000e+02 +1.775000000000000e+02
+6.365690000000000e+02 +1.125000000000000e+02
+7.901369999999999e+02 +1.455000000000000e+02
+1.073690000000000e+03 +2.530000000000000e+02
+8.116380000000000e+02 +1.435000000000000e+02
+6.996260000000002e+02 +5.535000000000000e+02
+1.309990000000000e+03 +2.860000000000000e+02
+1.294270000000000e+03 +1.860000000000000e+02
+1.353700000000000e+03 +4.170000000000000e+02
+9.027320000000000e+02 +1.100000000000000e+02
+1.296870000000000e+03 +2.775000000000000e+02
+6.496110000000000e+02 +1.150000000000000e+02
+1.463310000000000e+00 +5.530000000000000e+02
+7.787589999999999e+02 +9.250000000000000e+01
+1.725930000000000e+03 +5.530000000000000e+02
+1.874920000000000e+03 +5.335000000000000e+02
+8.813610000000001e+02 +1.935000000000000e+02
+1.315410000000000e+03 +1.840000000000000e+02
+1.102070000000000e+03 +1.590000000000000e+02
+1.738030000000000e+03 +2.540000000000000e+02
+6.390180000000000e+02 +5.385000000000000e+02
+4.772200000000000e+02 +7.050000000000000e+01
+7.092060000000000e+02 +5.520000000000000e+02
+9.545510000000000e+02 +1.480000000000000e+02
+1.058420000000000e+03 +3.535000000000000e+02
+1.444150000000000e+03 +5.075000000000000e+02
+1.600620000000000e+03 +4.080000000000000e+02
+6.070970000000000e+02 +5.510000000000000e+02
+8.744210000000000e+02 +1.005000000000000e+02
+1.382600000000000e+03 +4.165000000000000e+02
+1.348450000000000e+03 +2.395000000000000e+02
+2.319180000000000e+03 +4.805000000000000e+02
+7.794299999999999e+02 +8.850000000000000e+01
+1.755460000000000e+03 +4.130000000000000e+02
+8.740419999999998e+02 +8.950000000000000e+01
+9.006780000000000e+02 +5.070000000000000e+02
+1.570940000000000e+03 +3.040000000000000e+02
+1.998620000000000e+03 +3.600000000000000e+02
+8.412320000000000e+02 +8.150000000000000e+01
+1.383570000000000e+03 +2.615000000000000e+02
+7.417580000000000e+02 +7.650000000000000e+01
+6.774480000000000e+02 +5.495000000000000e+02
+4.520880000000000e+01 +5.495000000000000e+02
+1.136890000000000e+03 +2.585000000000000e+02
+6.835130000000000e+02 +2.030000000000000e+02
+6.921280000000000e+02 +1.300000000000000e+02
+1.394660000000000e+03 +4.535000000000000e+02
+1.299980000000000e+03 +2.590000000000000e+02
+1.234610000000000e+03 +3.605000000000000e+02
+1.872360000000000e+03 +5.295000000000000e+02
+4.823500000000000e+02 +1.880000000000000e+02
+1.082600000000000e+03 +2.525000000000000e+02
+6.923530000000002e+02 +5.480000000000000e+02
+1.303590000000000e+03 +1.880000000000000e+02
+1.137690000000000e+03 +2.430000000000000e+02
+7.762410000000001e+02 +1.620000000000000e+02
+4.886330000000000e+02 +8.150000000000000e+01
+8.910050000000000e+02 +2.170000000000000e+02
+1.581020000000000e+03 +2.150000000000000e+02
+2.011490000000000e+03 +3.625000000000000e+02
+9.032140000000001e+02 +1.090000000000000e+02
+9.795020000000000e+02 +1.505000000000000e+02
+1.288210000000000e+03 +2.755000000000000e+02
+7.418589999999998e+02 +5.460000000000000e+02
+1.863360000000000e+03 +5.235000000000000e+02
+6.230169999999998e+02 +1.440000000000000e+02
+1.073790000000000e+03 +4.540000000000000e+02
+9.663500000000000e+02 +1.430000000000000e+02
+7.906350000000000e+02 +1.345000000000000e+02
+5.945300000000000e+02 +5.445000000000000e+02
+7.720570000000000e+02 +8.150000000000000e+01
+9.235580000000000e+02 +2.185000000000000e+02
+1.665360000000000e+03 +3.265000000000000e+02
+1.085660000000000e+03 +2.475000000000000e+02
+4.923280000000000e+02 +1.860000000000000e+02
+2.986780000000000e+02 +1.065000000000000e+02
+1.037030000000000e+03 +2.495000000000000e+02
+6.208830000000000e+02 +1.420000000000000e+02
+1.382990000000000e+03 +4.105000000000000e+02
+9.127630000000000e+02 +1.605000000000000e+02
+1.593830000000000e+03 +2.715000000000000e+02
+1.079060000000000e+03 +2.500000000000000e+02
+8.288480000000002e+02 +5.420000000000000e+02
+1.080590000000000e+03 +9.850000000000000e+01
+6.251350000000000e+02 +1.450000000000000e+02
+1.308050000000000e+03 +1.580000000000000e+02
+1.015460000000000e+03 +2.690000000000000e+02
+6.399610000000000e+02 +1.075000000000000e+02
+6.873180000000000e+02 +5.410000000000000e+02
+7.485180000000000e+01 +5.410000000000000e+02
+7.627320000000000e+02 +1.565000000000000e+02
+6.888339999999999e+02 +1.925000000000000e+02
+6.381410000000000e+02 +1.040000000000000e+02
+6.713260000000000e+02 +5.405000000000000e+02
+7.785100000000000e+02 +8.000000000000000e+01
+1.375540000000000e+03 +3.375000000000000e+02
+1.139690000000000e+03 +4.350000000000000e+02
+1.299480000000000e+03 +1.775000000000000e+02
+1.339630000000000e+03 +2.340000000000000e+02
+1.379760000000000e+03 +2.305000000000000e+02
+1.817900000000000e+03 +5.400000000000000e+02
+7.620780000000001e+01 +5.400000000000000e+02
+1.936860000000000e+03 +4.190000000000000e+02
+1.911930000000000e+03 +3.555000000000000e+02
+1.957040000000000e+03 +4.285000000000000e+02
+1.928000000000000e+03 +3.705000000000000e+02
+1.322320000000000e+03 +2.545000000000000e+02
+1.039620000000000e+03 +1.850000000000000e+02
+1.863730000000000e+03 +5.400000000000000e+02
+6.747180000000002e+02 +1.030000000000000e+02
+1.988100000000000e+03 +4.360000000000000e+02
+1.946130000000000e+03 +3.630000000000000e+02
+8.642919999999998e+02 +5.400000000000000e+02
+8.144000000000000e+02 +1.385000000000000e+02
+6.942610000000001e+01 +5.395000000000000e+02
+7.799349999999999e+02 +1.530000000000000e+02
+8.971750000000000e+02 +1.060000000000000e+02
+9.059410000000000e+02 +2.105000000000000e+02
+1.015400000000000e+03 +1.480000000000000e+02
+7.076130000000001e+02 +2.055000000000000e+02
+1.681040000000000e+03 +3.230000000000000e+02
+1.606720000000000e+03 +4.115000000000000e+02
+6.556089999999998e+02 +2.055000000000000e+02
+1.657790000000000e+03 +2.940000000000000e+02
+6.520650000000001e+02 +1.025000000000000e+02
+1.852910000000000e+03 +3.720000000000000e+02
+9.609299999999999e+02 +1.455000000000000e+02
+9.083990000000000e+02 +1.030000000000000e+02
+1.343500000000000e+03 +2.285000000000000e+02
+1.693170000000000e+03 +3.795000000000000e+02
+1.295980000000000e+03 +3.690000000000000e+02
+1.088280000000000e+03 +1.430000000000000e+02
+2.745690000000000e+02 +2.050000000000000e+01
+6.185010000000000e+02 +9.750000000000000e+01
+5.430800000000000e+02 +5.365000000000000e+02
+6.774589999999999e+02 +5.365000000000000e+02
+1.269500000000000e+03 +2.895000000000000e+02
+8.253370000000000e+02 +5.360000000000000e+02
+6.674960000000002e+02 +5.360000000000000e+02
+1.410890000000000e+03 +4.980000000000000e+02
+1.236880000000000e+03 +2.290000000000000e+02
+1.065740000000000e+03 +1.460000000000000e+02
+1.300630000000000e+03 +1.740000000000000e+02
+7.946840000000000e+02 +1.605000000000000e+02
+8.036810000000000e+02 +1.365000000000000e+02
+1.088860000000000e+03 +2.420000000000000e+02
+6.839780000000000e+01 +5.350000000000000e+02
+1.584410000000000e+03 +2.825000000000000e+02
+8.918610000000000e+01 +5.345000000000000e+02
+8.787739999999999e+02 +4.495000000000000e+02
+9.378790000000000e+02 +1.650000000000000e+02
+9.799550000000000e+02 +1.425000000000000e+02
+9.231910000000000e+02 +1.370000000000000e+02
+1.395340000000000e+03 +2.430000000000000e+02
+6.249980000000000e+02 +1.350000000000000e+02
+5.396290000000000e+01 +5.335000000000000e+02
+7.652150000000000e+02 +1.505000000000000e+02
+9.199550000000000e+02 +2.505000000000000e+02
+7.816610000000002e+02 +1.475000000000000e+02
+1.134170000000000e+03 +2.440000000000000e+02
+5.877800000000000e+02 +5.320000000000000e+02
+6.031630000000000e+02 +5.320000000000000e+02
+9.190490000000000e+02 +1.345000000000000e+02
+1.029250000000000e+03 +1.515000000000000e+02
+1.869130000000000e+03 +4.440000000000000e+02
+2.817010000000000e+02 +2.150000000000000e+01
+7.015390000000000e+02 +1.970000000000000e+02
+6.577869999999998e+02 +5.305000000000000e+02
+9.259990000000000e+02 +2.505000000000000e+02
+1.478630000000000e+03 +3.770000000000000e+02
+1.844360000000000e+03 +3.330000000000000e+02
+1.673650000000000e+03 +4.075000000000000e+02
+6.827869999999998e+02 +5.295000000000000e+02
+5.581150000000000e+02 +5.295000000000000e+02
+1.079690000000000e+03 +2.240000000000000e+02
+4.951210000000000e+02 +2.620000000000000e+02
+4.958500000000000e+02 +5.200000000000000e+01
+8.385880000000002e+02 +8.200000000000000e+01
+9.750700000000001e+02 +1.375000000000000e+02
+1.017230000000000e+03 +1.080000000000000e+02
+5.710910000000000e+02 +5.285000000000000e+02
+1.331260000000000e+03 +2.685000000000000e+02
+6.227220000000000e+02 +1.295000000000000e+02
+1.012970000000000e+03 +2.415000000000000e+02
+2.127360000000000e+02 +5.280000000000000e+02
+6.796890000000000e+02 +5.275000000000000e+02
+8.886920000000000e+02 +1.490000000000000e+02
+1.050850000000000e+03 +2.380000000000000e+02
+2.812460000000000e+02 +2.200000000000000e+01
+7.734360000000000e+02 +1.460000000000000e+02
+1.032840000000000e+03 +1.935000000000000e+02
+9.757280000000000e+02 +3.370000000000000e+02
+7.958750000000000e+02 +2.365000000000000e+02
+6.157390000000000e+02 +9.200000000000000e+01
+1.090860000000000e+03 +1.370000000000000e+02
+7.158110000000000e+02 +5.260000000000000e+02
+6.311970000000000e+02 +5.260000000000000e+02
+1.081010000000000e+03 +2.380000000000000e+02
+7.793510000000001e+02 +5.255000000000000e+02
+6.609989999999998e+02 +1.265000000000000e+02
+7.852780000000000e+02 +1.420000000000000e+02
+6.212640000000000e+02 +9.500000000000000e+01
+1.295100000000000e+03 +1.610000000000000e+02
+1.140840000000000e+03 +1.985000000000000e+02
+9.305170000000001e+02 +2.070000000000000e+02
+1.475450000000000e+03 +2.595000000000000e+02
+1.262600000000000e+03 +3.065000000000000e+02
+1.326640000000000e+03 +2.865000000000000e+02
+5.633070000000000e+01 +5.245000000000000e+02
+1.367060000000000e+03 +3.945000000000000e+02
+9.105860000000000e+02 +1.295000000000000e+02
+1.070000000000000e+03 +2.395000000000000e+02
+6.591870000000000e+02 +9.200000000000000e+01
+9.466020000000000e+02 +2.075000000000000e+02
+6.702950000000000e+02 +9.300000000000000e+01
+8.253700000000000e+02 +5.235000000000000e+02
+1.863510000000000e+03 +4.315000000000000e+02
+1.292930000000000e+03 +1.620000000000000e+02
+1.024290000000000e+03 +2.180000000000000e+02
+1.354500000000000e+03 +4.645000000000000e+02
+6.916439999999999e+02 +1.695000000000000e+02
+1.632540000000000e+03 +3.575000000000000e+02
+8.915720000000000e+02 +1.300000000000000e+02
+6.430830000000002e+02 +3.745000000000000e+02
+1.315840000000000e+03 +2.030000000000000e+02
+8.273630000000000e+01 +5.225000000000000e+02
+1.873890000000000e+03 +5.095000000000000e+02
+8.565720000000000e+01 +5.220000000000000e+02
+1.952480000000000e+03 +3.345000000000000e+02
+9.069890000000000e+02 +1.330000000000000e+02
+8.678260000000000e+02 +1.115000000000000e+02
+4.868270000000000e+02 +1.670000000000000e+02
+1.307620000000000e+03 +2.565000000000000e+02
+8.199750000000000e+02 +1.160000000000000e+02
+1.571050000000000e+03 +2.225000000000000e+02
+1.671220000000000e+03 +3.990000000000000e+02
+9.061150000000000e+02 +1.295000000000000e+02
+1.012540000000000e+03 +5.215000000000000e+02
+1.120600000000000e+03 +5.210000000000000e+02
+1.404070000000000e+03 +2.305000000000000e+02
+6.404720000000000e+02 +8.950000000000000e+01
+5.368960000000000e+02 +5.210000000000000e+02
+1.166440000000000e+03 +2.070000000000000e+02
+7.947200000000000e+02 +2.310000000000000e+02
+5.473230000000000e+02 +5.205000000000000e+02
+1.371240000000000e+03 +4.965000000000000e+02
+9.082850000000000e+02 +1.290000000000000e+02
+6.695939999999998e+02 +9.150000000000000e+01
+8.803670000000000e+02 +1.435000000000000e+02
+6.204000000000000e+01 +5.200000000000000e+02
+1.140600000000000e+03 +2.215000000000000e+02
+9.263720000000000e+02 +2.470000000000000e+02
+1.177090000000000e+03 +5.195000000000000e+02
+5.360050000000000e+02 +5.195000000000000e+02
+7.928040000000000e+01 +5.195000000000000e+02
+2.391990000000000e+01 +5.195000000000000e+02
+9.745760000000000e+02 +1.345000000000000e+02
+1.006750000000000e+03 +2.305000000000000e+02
+6.498860000000000e+02 +7.800000000000000e+01
+9.382060000000000e+02 +2.425000000000000e+02
+7.719390000000000e+02 +5.185000000000000e+02
+1.000620000000000e+03 +3.480000000000000e+02
+8.707930000000000e+02 +1.635000000000000e+02
+1.016310000000000e+03 +1.120000000000000e+02
+1.882760000000000e+03 +4.430000000000000e+02
+6.591139999999998e+02 +1.210000000000000e+02
+4.924690000000000e+01 +5.185000000000000e+02
+9.943500000000000e+02 +1.455000000000000e+02
+6.595790000000000e+02 +1.660000000000000e+02
+6.584870000000000e+02 +1.860000000000000e+02
+1.590200000000000e+03 +2.215000000000000e+02
+1.598830000000000e+03 +3.280000000000000e+02
+2.781780000000000e+02 +2.100000000000000e+01
+5.196490000000000e+02 +5.175000000000000e+02
+7.693550000000000e+02 +1.350000000000000e+02
+1.165260000000000e+03 +5.170000000000000e+02
+1.492050000000000e+03 +3.665000000000000e+02
+1.061870000000000e+03 +2.310000000000000e+02
+7.684760000000001e+02 +1.355000000000000e+02
+6.714240000000000e+02 +5.165000000000000e+02
+1.278810000000000e+03 +2.495000000000000e+02
+9.789660000000000e+02 +1.350000000000000e+02
+1.603920000000000e+03 +4.060000000000000e+02
+7.039780000000002e+02 +5.160000000000000e+02
+1.873010000000000e+03 +3.535000000000000e+02
+7.672930000000000e+02 +5.155000000000000e+02
+7.508819999999999e+02 +5.600000000000000e+01
+8.558049999999999e+02 +5.155000000000000e+02
+6.619880000000001e+02 +1.205000000000000e+02
+7.818480000000002e+02 +1.375000000000000e+02
+1.133490000000000e+03 +2.165000000000000e+02
+1.467760000000000e+03 +4.810000000000000e+02
+1.306570000000000e+03 +2.265000000000000e+02
+6.398940000000000e+02 +5.150000000000000e+02
+1.314020000000000e+03 +2.000000000000000e+02
+6.367560000000000e+02 +1.230000000000000e+02
+1.119100000000000e+03 +4.215000000000000e+02
+1.075940000000000e+03 +2.345000000000000e+02
+5.804660000000000e+02 +5.145000000000000e+02
+6.152390000000000e+02 +1.195000000000000e+02
+1.303020000000000e+03 +1.560000000000000e+02
+1.372340000000000e+03 +2.260000000000000e+02
+9.905290000000000e+02 +1.405000000000000e+02
+1.724410000000000e+03 +3.500000000000000e+02
+1.064890000000000e+03 +2.295000000000000e+02
+6.336910000000000e+01 +5.135000000000000e+02
+7.903560000000001e+02 +1.350000000000000e+02
+1.948250000000000e+03 +4.370000000000000e+02
+1.144840000000000e+03 +5.130000000000000e+02
+6.424540000000002e+02 +1.230000000000000e+02
+8.902020000000000e+02 +3.620000000000000e+02
+8.069390000000000e+02 +2.065000000000000e+02
+6.336120000000000e+02 +8.350000000000000e+01
+1.861130000000000e+03 +4.195000000000000e+02
+2.331010000000000e+03 +3.485000000000000e+02
+2.794370000000000e+02 +1.045000000000000e+02
+1.862890000000000e+03 +4.360000000000000e+02
+8.877220000000000e+02 +1.260000000000000e+02
+1.608140000000000e+03 +2.530000000000000e+02
+1.481410000000000e+03 +2.460000000000000e+02
+6.396380000000000e+02 +5.110000000000000e+02
+1.867870000000000e+03 +4.350000000000000e+02
+2.818490000000000e+02 +1.035000000000000e+02
+6.761560000000002e+02 +1.790000000000000e+02
+7.713280000000000e+02 +1.305000000000000e+02
+7.913650000000000e+02 +2.065000000000000e+02
+1.301330000000000e+03 +1.535000000000000e+02
+9.004430000000000e+02 +1.210000000000000e+02
+1.163610000000000e+03 +2.300000000000000e+02
+4.946500000000000e+02 +2.475000000000000e+02
+1.485300000000000e+03 +2.465000000000000e+02
+6.554000000000000e+02 +8.700000000000000e+01
+1.053620000000000e+03 +2.115000000000000e+02
+1.551610000000000e+03 +2.625000000000000e+02
+8.891350000000000e+02 +4.255000000000000e+02
+1.065430000000000e+03 +2.140000000000000e+02
+6.317930000000000e+02 +1.135000000000000e+02
+5.819730000000000e+01 +5.085000000000000e+02
+1.303600000000000e+03 +1.545000000000000e+02
+1.173270000000000e+03 +5.080000000000000e+02
+6.147010000000000e+02 +8.100000000000000e+01
+7.042580000000000e+02 +1.780000000000000e+02
+7.783049999999999e+02 +1.310000000000000e+02
+4.858830000000000e+02 +5.250000000000000e+01
+2.005210000000000e+03 +3.205000000000000e+02
+1.007330000000000e+03 +2.240000000000000e+02
+9.844960000000000e+02 +1.310000000000000e+02
+1.085030000000000e+03 +1.205000000000000e+02
+1.494680000000000e+03 +2.475000000000000e+02
+2.790550000000000e+02 +1.015000000000000e+02
+6.343650000000000e+02 +7.250000000000000e+01
+6.796530000000000e+02 +5.070000000000000e+02
+1.035970000000000e+03 +1.195000000000000e+02
+1.306620000000000e+03 +2.720000000000000e+02
+4.284390000000000e+01 +5.065000000000000e+02
+6.688180000000000e+02 +5.060000000000000e+02
+7.899310000000000e+02 +1.305000000000000e+02
+9.956190000000000e+02 +1.340000000000000e+02
+8.936480000000000e+02 +4.595000000000000e+02
+4.902590000000000e+01 +5.055000000000000e+02
+1.549410000000000e+03 +2.570000000000000e+02
+8.997689999999999e+02 +1.150000000000000e+02
+6.371110000000000e+02 +8.050000000000000e+01
+7.753610000000001e+02 +1.275000000000000e+02
+1.651680000000000e+03 +3.060000000000000e+02
+6.776980000000000e+02 +1.775000000000000e+02
+5.047880000000000e+01 +5.045000000000000e+02
+1.992970000000000e+03 +3.230000000000000e+02
+1.158310000000000e+03 +2.275000000000000e+02
+1.673260000000000e+03 +3.790000000000000e+02
+1.368470000000000e+03 +4.795000000000000e+02
+9.142640000000000e+02 +1.900000000000000e+02
+1.011690000000000e+03 +2.255000000000000e+02
+9.710160000000000e+02 +2.160000000000000e+02
+1.043830000000000e+03 +2.160000000000000e+02
+7.807730000000000e+02 +1.275000000000000e+02
+1.131210000000000e+03 +5.030000000000000e+02
+1.210530000000000e+03 +1.985000000000000e+02
+4.846010000000000e+02 +2.370000000000000e+02
+6.735189999999999e+02 +5.025000000000000e+02
+8.023090000000000e+02 +5.025000000000000e+02
+7.884980000000000e+02 +1.280000000000000e+02
+1.763900000000000e+03 +3.240000000000000e+02
+1.021330000000000e+03 +9.400000000000000e+01
+4.257450000000000e+01 +5.020000000000000e+02
+7.514130000000000e+02 +1.240000000000000e+02
+1.682510000000000e+03 +3.030000000000000e+02
+1.162190000000000e+03 +4.595000000000000e+02
+1.082540000000000e+03 +2.115000000000000e+02
+1.097680000000000e+03 +2.010000000000000e+02
+9.939850000000000e+02 +1.360000000000000e+02
+6.143290000000002e+02 +6.050000000000000e+01
+1.526170000000000e+03 +2.450000000000000e+02
+6.371240000000000e+02 +7.250000000000000e+01
+1.855040000000000e+03 +4.165000000000000e+02
+1.315710000000000e+03 +1.450000000000000e+02
+9.900630000000000e+02 +1.325000000000000e+02
+1.344140000000000e+03 +4.590000000000000e+02
+8.784030000000000e+02 +1.630000000000000e+02
+1.189840000000000e+03 +1.910000000000000e+02
+1.646820000000000e+03 +2.975000000000000e+02
+1.221590000000000e+03 +1.940000000000000e+02
+1.110700000000000e+03 +1.910000000000000e+02
+9.218370000000000e+02 +1.125000000000000e+02
+8.688240000000000e+02 +2.035000000000000e+02
+2.868760000000000e+02 +1.035000000000000e+02
+6.405910000000000e+02 +4.975000000000000e+02
+7.810460000000000e+02 +1.245000000000000e+02
+6.514740000000000e+02 +6.500000000000000e+01
+6.769190000000000e+02 +1.650000000000000e+02
+3.640710000000000e+01 +4.970000000000000e+02
+1.099480000000000e+03 +2.175000000000000e+02
+6.712430000000001e+02 +6.650000000000000e+01
+1.852100000000000e+03 +4.060000000000000e+02
+1.337090000000000e+03 +4.475000000000000e+02
+8.839370000000000e+02 +4.135000000000000e+02
+1.050750000000000e+03 +2.040000000000000e+02
+7.503150000000001e+02 +1.385000000000000e+02
+6.742160000000000e+02 +1.660000000000000e+02
+9.056300000000000e+02 +7.000000000000000e+01
+4.954180000000000e+02 +2.395000000000000e+02
+6.841760000000000e+02 +4.955000000000000e+02
+4.134860000000000e+02 +4.955000000000000e+02
+1.308430000000000e+03 +1.475000000000000e+02
+1.605990000000000e+03 +2.390000000000000e+02
+8.741960000000000e+02 +4.540000000000000e+02
+1.025400000000000e+03 +9.100000000000000e+01
+1.085850000000000e+03 +1.925000000000000e+02
+6.333819999999999e+02 +6.900000000000000e+01
+7.765110000000002e+02 +1.205000000000000e+02
+1.318440000000000e+03 +1.450000000000000e+02
+1.329770000000000e+03 +4.030000000000000e+02
+1.234770000000000e+03 +4.515000000000000e+02
+4.164240000000000e+02 +4.940000000000000e+02
+9.074840000000000e+02 +1.635000000000000e+02
+1.119910000000000e+03 +4.935000000000000e+02
+1.074050000000000e+03 +2.120000000000000e+02
+6.637040000000000e+02 +6.500000000000000e+01
+4.098810000000000e+02 +4.935000000000000e+02
+8.941760000000000e+02 +6.950000000000000e+01
+1.855810000000000e+03 +4.050000000000000e+02
+7.798339999999999e+02 +1.215000000000000e+02
+9.213460000000000e+02 +2.240000000000000e+02
+6.668789999999998e+02 +1.365000000000000e+02
+9.660510000000000e+02 +1.410000000000000e+02
+6.195190000000000e+02 +1.045000000000000e+02
+6.559470000000000e+02 +4.920000000000000e+02
+9.325520000000000e+02 +1.200000000000000e+02
+1.424770000000000e+03 +2.005000000000000e+02
+7.672370000000000e+02 +4.915000000000000e+02
+6.468370000000000e+02 +7.000000000000000e+01
+9.997960000000000e+02 +2.160000000000000e+02
+8.742850000000000e+02 +1.990000000000000e+02
+1.787650000000000e+03 +3.305000000000000e+02
+7.775740000000000e+02 +1.185000000000000e+02
+9.846470000000000e+02 +1.270000000000000e+02
+1.302120000000000e+03 +4.905000000000000e+02
+3.611970000000000e+01 +4.900000000000000e+02
+7.932470000000000e+02 +1.210000000000000e+02
+6.785520000000000e+02 +1.445000000000000e+02
+7.728000000000000e+02 +4.895000000000000e+02
+1.552700000000000e+03 +2.460000000000000e+02
+1.140800000000000e+03 +4.740000000000000e+02
+1.673490000000000e+03 +3.255000000000000e+02
+6.491130000000001e+02 +1.645000000000000e+02
+6.827089999999999e+02 +4.890000000000000e+02
+6.228310000000000e+02 +9.600000000000000e+01
+4.392180000000000e+01 +4.890000000000000e+02
+7.778739999999998e+02 +1.195000000000000e+02
+9.718440000000001e+02 +2.025000000000000e+02
+3.966480000000000e+02 +4.885000000000000e+02
+1.575230000000000e+03 +3.550000000000000e+02
+6.213020000000000e+02 +9.100000000000000e+01
+6.452950000000000e+02 +1.595000000000000e+02
+5.158880000000000e+02 +4.875000000000000e+02
+1.338780000000000e+03 +4.650000000000000e+02
+9.019890000000000e+02 +1.085000000000000e+02
+1.237290000000000e+03 +1.815000000000000e+02
+2.814160000000000e+02 +1.085000000000000e+02
+8.992610000000002e+02 +1.050000000000000e+02
+6.466720000000000e+02 +4.860000000000000e+02
+1.669650000000000e+03 +4.525000000000000e+02
+5.054780000000000e+02 +4.855000000000000e+02
+1.004700000000000e+03 +2.135000000000000e+02
+9.137350000000000e+02 +2.195000000000000e+02
+1.072440000000000e+03 +2.125000000000000e+02
+6.367010000000000e+02 +1.570000000000000e+02
+4.259630000000000e+02 +4.850000000000000e+02
+1.865520000000000e+03 +3.985000000000000e+02
+7.740939999999998e+02 +1.175000000000000e+02
+1.071550000000000e+03 +1.925000000000000e+02
+6.488060000000000e+02 +4.845000000000000e+02
+1.488930000000000e+03 +2.725000000000000e+02
+8.687960000000000e+02 +1.905000000000000e+02
+6.004940000000000e+02 +4.845000000000000e+02
+8.654000000000000e+02 +1.255000000000000e+02
+9.736900000000001e+02 +1.915000000000000e+02
+9.019810000000000e+02 +9.900000000000000e+01
+5.018080000000000e+02 +4.840000000000000e+02
+4.935080000000000e+02 +2.330000000000000e+02
+1.092800000000000e+03 +2.085000000000000e+02
+8.770980000000002e+02 +1.200000000000000e+02
+1.088270000000000e+03 +2.045000000000000e+02
+7.001220000000000e+02 +2.555000000000000e+02
+3.485000000000000e+02 +4.830000000000000e+02
+5.240850000000000e+02 +4.830000000000000e+02
+1.056330000000000e+03 +1.965000000000000e+02
+1.707620000000000e+03 +4.825000000000000e+02
+6.636910000000000e+02 +9.000000000000000e+01
+7.727550000000000e+02 +1.155000000000000e+02
+7.814639999999998e+02 +1.155000000000000e+02
+1.375460000000000e+03 +4.635000000000000e+02
+1.453520000000000e+03 +4.500000000000000e+02
+3.345260000000000e+02 +9.950000000000000e+01
+1.771020000000000e+03 +4.065000000000000e+02
+1.098330000000000e+03 +3.905000000000000e+02
+8.926469999999998e+02 +1.695000000000000e+02
+9.048140000000000e+02 +1.200000000000000e+02
+1.023370000000000e+03 +4.810000000000000e+02
+6.951089999999998e+02 +2.680000000000000e+02
+2.778570000000000e+02 +8.600000000000000e+01
+5.207700000000000e+02 +4.805000000000000e+02
+7.878320000000000e+02 +1.145000000000000e+02
+1.012180000000000e+03 +4.805000000000000e+02
+6.361260000000000e+02 +1.500000000000000e+02
+9.023490000000000e+02 +9.650000000000000e+01
+1.610970000000000e+03 +2.240000000000000e+02
+1.759900000000000e+03 +4.620000000000000e+02
+6.354820000000000e+02 +4.800000000000000e+02
+5.606190000000000e+02 +4.795000000000000e+02
+9.911630000000000e+02 +1.870000000000000e+02
+9.094030000000000e+02 +1.095000000000000e+02
+9.367530000000000e+02 +2.170000000000000e+02
+7.580020000000000e+02 +4.790000000000000e+02
+3.454110000000000e+02 +4.790000000000000e+02
+2.994060000000000e+01 +4.790000000000000e+02
+4.862930000000000e+02 +2.195000000000000e+02
+6.406890000000000e+02 +1.525000000000000e+02
+7.402180000000002e+02 +4.780000000000000e+02
+5.000160000000000e+02 +4.780000000000000e+02
+8.874800000000000e+02 +4.780000000000000e+02
+1.077100000000000e+03 +1.925000000000000e+02
+4.816670000000000e+02 +4.775000000000000e+02
+5.584159999999998e+02 +4.775000000000000e+02
+1.288130000000000e+03 +2.195000000000000e+02
+7.726130000000001e+02 +1.120000000000000e+02
+1.292420000000000e+03 +2.260000000000000e+02
+9.286700000000000e+02 +2.110000000000000e+02
+4.905370000000000e+02 +4.770000000000000e+02
+8.058330000000002e+02 +4.770000000000000e+02
+1.043900000000000e+03 +1.760000000000000e+02
+1.227890000000000e+03 +4.285000000000000e+02
+1.008490000000000e+03 +3.115000000000000e+02
+7.117189999999998e+02 +1.430000000000000e+02
+2.743620000000000e+01 +4.765000000000000e+02
+9.600450000000000e+02 +4.765000000000000e+02
+1.174400000000000e+03 +3.465000000000000e+02
+8.957689999999999e+02 +4.430000000000000e+02
+1.018640000000000e+03 +3.155000000000000e+02
+1.304460000000000e+03 +2.290000000000000e+02
+5.291430000000000e+02 +4.755000000000000e+02
+4.283100000000000e+02 +4.755000000000000e+02
+5.297400000000000e+02 +4.755000000000000e+02
+1.002670000000000e+03 +3.065000000000000e+02
+8.110230000000000e+02 +2.080000000000000e+02
+1.572390000000000e+03 +1.870000000000000e+02
+1.556190000000000e+03 +3.945000000000000e+02
+1.411110000000000e+03 +4.750000000000000e+02
+6.437880000000000e+02 +1.070000000000000e+02
+4.814730000000000e+02 +4.745000000000000e+02
+7.690080000000000e+02 +1.115000000000000e+02
+1.304420000000000e+03 +2.290000000000000e+02
+9.748150000000001e+02 +4.745000000000000e+02
+7.708350000000000e+02 +4.745000000000000e+02
+9.933330000000000e+02 +1.875000000000000e+02
+3.538970000000000e+02 +4.740000000000000e+02
+4.825150000000000e+02 +4.740000000000000e+02
+3.468260000000000e+00 +4.735000000000000e+02
+9.859690000000001e+02 +1.845000000000000e+02
+5.507750000000000e+02 +4.730000000000000e+02
+2.820110000000000e+02 +8.100000000000000e+01
+5.235269999999998e+02 +4.730000000000000e+02
+7.737239999999998e+02 +1.120000000000000e+02
+1.778190000000000e+03 +2.935000000000000e+02
+1.382410000000000e+03 +2.550000000000000e+02
+4.257780000000000e+02 +4.725000000000000e+02
+6.439430000000000e+02 +1.495000000000000e+02
+7.035010000000002e+02 +2.590000000000000e+02
+1.716700000000000e+03 +4.720000000000000e+02
+5.176090000000000e+02 +4.715000000000000e+02
+6.579030000000000e+02 +1.535000000000000e+02
+7.899530000000000e+02 +1.125000000000000e+02
+1.066290000000000e+03 +1.965000000000000e+02
+1.880930000000000e+03 +3.920000000000000e+02
+7.455520000000000e+02 +1.075000000000000e+02
+4.872180000000000e+02 +2.175000000000000e+02
+1.079760000000000e+03 +1.925000000000000e+02
+4.373230000000000e+02 +4.705000000000000e+02
+4.230570000000000e+01 +4.705000000000000e+02
+7.777260000000001e+02 +1.090000000000000e+02
+9.133860000000000e+02 +2.060000000000000e+02
+1.116460000000000e+03 +4.700000000000000e+02
+9.941380000000000e+02 +1.860000000000000e+02
+7.467320000000000e+02 +1.140000000000000e+02
+6.570680000000000e+02 +1.435000000000000e+02
+3.154460000000000e+02 +4.700000000000000e+02
+4.961670000000000e+02 +2.240000000000000e+02
+7.454410000000000e+02 +4.695000000000000e+02
+6.967210000000000e+02 +1.730000000000000e+02
+7.805169999999998e+02 +1.095000000000000e+02
+3.358990000000000e+02 +9.150000000000000e+01
+1.145270000000000e+03 +1.830000000000000e+02
+7.006770000000000e+02 +2.450000000000000e+02
+6.669420000000000e+02 +1.410000000000000e+02
+5.840910000000000e+02 +4.690000000000000e+02
+6.190490000000000e+02 +7.850000000000000e+01
+6.259990000000000e+02 +4.685000000000000e+02
+1.775880000000000e+03 +3.965000000000000e+02
+9.891980000000000e+02 +4.685000000000000e+02
+2.607350000000000e+02 +2.050000000000000e+01
+9.033530000000000e+02 +4.380000000000000e+02
+1.054890000000000e+03 +1.800000000000000e+02
+6.247650000000000e+02 +1.480000000000000e+02
+6.591230000000000e+02 +4.680000000000000e+02
+6.305830000000002e+02 +6.050000000000000e+01
+4.693670000000000e+02 +4.670000000000000e+02
+1.863250000000000e+03 +3.930000000000000e+02
+1.563950000000000e+03 +3.065000000000000e+02
+4.899990000000000e+02 +2.255000000000000e+02
+7.041400000000000e+02 +2.555000000000000e+02
+1.300500000000000e+03 +2.205000000000000e+02
+9.181400000000000e+02 +9.900000000000000e+01
+9.975130000000000e+02 +1.795000000000000e+02
+1.985750000000000e+03 +3.915000000000000e+02
+6.088200000000001e+02 +4.655000000000000e+02
+4.566630000000000e+02 +4.650000000000000e+02
+1.072240000000000e+03 +1.815000000000000e+02
+1.502220000000000e+01 +4.645000000000000e+02
+7.728989999999999e+02 +1.045000000000000e+02
+9.110850000000000e+02 +3.030000000000000e+02
+7.447060000000000e+02 +4.640000000000000e+02
+8.740250000000000e+02 +2.120000000000000e+02
+3.424550000000000e+02 +4.640000000000000e+02
+9.894380000000000e+02 +1.825000000000000e+02
+1.228890000000000e+03 +1.820000000000000e+02
+7.471310000000002e+02 +4.635000000000000e+02
+7.032500000000000e+02 +2.425000000000000e+02
+6.636610000000002e+02 +7.350000000000000e+01
+2.978260000000000e+01 +4.635000000000000e+02
+9.947820000000000e+02 +1.785000000000000e+02
+1.015540000000000e+03 +4.635000000000000e+02
+4.566750000000000e+02 +4.800000000000000e+01
+1.009900000000000e+03 +1.805000000000000e+02
+1.376930000000000e+03 +2.460000000000000e+02
+1.028250000000000e+03 +4.630000000000000e+02
+8.819040000000000e+02 +2.085000000000000e+02
+4.936950000000000e+02 +4.625000000000000e+02
+1.683470000000000e+03 +3.565000000000000e+02
+1.230340000000000e+03 +2.975000000000000e+02
+7.087050000000000e+02 +3.570000000000000e+02
+1.754350000000000e+03 +4.620000000000000e+02
+1.019970000000000e+03 +2.930000000000000e+02
+2.905990000000000e+02 +4.615000000000000e+02
+8.927160000000000e+02 +8.500000000000000e+01
+9.367130000000000e+02 +3.040000000000000e+02
+2.730020000000000e+02 +4.610000000000000e+02
+2.126800000000000e+01 +4.610000000000000e+02
+7.711910000000000e+02 +1.040000000000000e+02
+1.594110000000000e+03 +2.670000000000000e+02
+6.208550000000000e+02 +4.605000000000000e+02
+7.683600000000000e+02 +1.190000000000000e+02
+1.015310000000000e+03 +4.605000000000000e+02
+7.045060000000002e+02 +1.600000000000000e+02
+2.577980000000000e+01 +4.600000000000000e+02
+3.101790000000000e+01 +4.595000000000000e+02
+4.703480000000000e+02 +4.590000000000000e+02
+2.822410000000000e+02 +7.150000000000000e+01
+6.443060000000000e+02 +1.325000000000000e+02
+3.233170000000000e+01 +4.590000000000000e+02
+1.184370000000000e+03 +3.540000000000000e+02
+4.794230000000000e+02 +4.500000000000000e+01
+7.145780000000000e+02 +2.445000000000000e+02
+8.068830000000000e+02 +1.700000000000000e+02
+4.237350000000000e+02 +4.575000000000000e+02
+1.859740000000000e+03 +3.795000000000000e+02
+7.781239999999998e+02 +1.025000000000000e+02
+6.388009999999998e+02 +1.330000000000000e+02
+8.754930000000001e+02 +1.180000000000000e+02
+6.249760000000000e+02 +6.850000000000000e+01
+7.460790000000000e+02 +1.050000000000000e+02
+3.803940000000000e+02 +4.565000000000000e+02
+1.300880000000000e+03 +2.140000000000000e+02
+5.461430000000000e+02 +4.560000000000000e+02
+6.232470000000000e+02 +6.950000000000000e+01
+1.223850000000000e+03 +2.900000000000000e+02
+4.034470000000000e+02 +4.555000000000000e+02
+4.899090000000000e+02 +3.010000000000000e+02
+7.660560000000000e+02 +9.750000000000000e+01
+9.859700000000000e+02 +1.800000000000000e+02
+1.022600000000000e+03 +2.890000000000000e+02
+7.118780000000000e+02 +4.545000000000000e+02
+8.233720000000000e+02 +1.530000000000000e+02
+5.648180000000000e+02 +4.545000000000000e+02
+7.817250000000000e+02 +1.020000000000000e+02
+4.228530000000000e+02 +4.150000000000000e+01
+6.959700000000000e+02 +4.540000000000000e+02
+1.009730000000000e+03 +2.870000000000000e+02
+6.571439999999999e+02 +2.405000000000000e+02
+6.284290000000000e+02 +6.850000000000000e+01
+1.337130000000000e+03 +4.345000000000000e+02
+5.035820000000000e+02 +3.025000000000000e+02
+6.380369999999998e+02 +1.375000000000000e+02
+6.951330000000000e+02 +1.510000000000000e+02
+1.321730000000000e+03 +4.310000000000000e+02
+5.106490000000000e+02 +3.035000000000000e+02
+7.823610000000001e+02 +1.485000000000000e+02
+4.964130000000000e+02 +4.530000000000000e+02
+6.595020000000000e+02 +6.450000000000000e+01
+8.015670000000000e+02 +4.530000000000000e+02
+5.508250000000000e+02 +4.525000000000000e+02
+1.499680000000000e+03 +3.205000000000000e+02
+2.783400000000000e+02 +5.100000000000000e+01
+4.763000000000000e+02 +4.525000000000000e+02
+7.849130000000000e+02 +1.010000000000000e+02
+8.998930000000000e+02 +8.750000000000000e+01
+1.004290000000000e+03 +4.525000000000000e+02
+4.824550000000000e+02 +2.040000000000000e+02
+1.353140000000000e+03 +2.930000000000000e+02
+1.338920000000000e+03 +4.340000000000000e+02
+8.294930000000001e+02 +4.520000000000000e+02
+6.189660000000000e+02 +4.515000000000000e+02
+4.718240000000000e+02 +4.515000000000000e+02
+1.574490000000000e+03 +2.955000000000000e+02
+1.150730000000000e+03 +2.495000000000000e+02
+1.014340000000000e+03 +2.820000000000000e+02
+8.621960000000000e+02 +2.030000000000000e+02
+6.043860000000000e+02 +1.285000000000000e+02
+3.817430000000001e+02 +4.510000000000000e+02
+1.106350000000000e+03 +1.510000000000000e+02
+9.891570000000000e+02 +1.665000000000000e+02
+6.730119999999999e+02 +1.880000000000000e+02
+7.838339999999999e+02 +9.500000000000000e+01
+7.020300000000000e+02 +1.490000000000000e+02
+1.535310000000000e+01 +4.500000000000000e+02
+5.051720000000000e+02 +3.055000000000000e+02
+1.577790000000000e+01 +4.495000000000000e+02
+1.299790000000000e+03 +2.110000000000000e+02
+1.779290000000000e+03 +3.830000000000000e+02
+4.187500000000000e+02 +4.490000000000000e+02
+5.232730000000000e+02 +4.485000000000000e+02
+8.586900000000001e+02 +1.075000000000000e+02
+1.093370000000000e+03 +1.520000000000000e+02
+8.476990000000000e+02 +8.950000000000000e+01
+5.435330000000000e+02 +4.480000000000000e+02
+4.861990000000000e+02 +2.055000000000000e+02
+1.270080000000000e+03 +4.480000000000000e+02
+1.310110000000000e+03 +4.320000000000000e+02
+2.818770000000000e+02 +4.800000000000000e+01
+4.527910000000000e+02 +4.475000000000000e+02
+6.186369999999999e+02 +6.050000000000000e+01
+7.782990000000000e+02 +9.300000000000000e+01
+1.382050000000000e+03 +4.340000000000000e+02
+8.893360000000000e+02 +9.050000000000000e+01
+1.611540000000000e+03 +3.035000000000000e+02
+2.948230000000000e+02 +5.450000000000000e+01
+1.309030000000000e+03 +3.185000000000000e+02
+1.015790000000000e+03 +4.470000000000000e+02
+6.516319999999999e+02 +1.260000000000000e+02
+4.777130000000000e+02 +4.465000000000000e+02
+1.177320000000000e+01 +4.465000000000000e+02
+9.460090000000000e+02 +4.465000000000000e+02
+8.965690000000000e+02 +2.360000000000000e+02
+4.992450000000000e+02 +2.900000000000000e+02
+1.332600000000000e+03 +3.200000000000000e+02
+9.792690000000000e+02 +1.780000000000000e+02
+1.081120000000000e+03 +1.550000000000000e+02
+9.768030000000000e+02 +2.050000000000000e+02
+3.740970000000000e+02 +4.450000000000000e+02
+8.948919999999998e+02 +8.200000000000000e+01
+9.985640000000000e+02 +1.665000000000000e+02
+7.573550000000000e+02 +9.550000000000000e+01
+7.705730000000000e+02 +9.450000000000000e+01
+6.430540000000000e+02 +4.300000000000000e+01
+4.997500000000000e+02 +2.910000000000000e+02
+4.229730000000000e+02 +4.440000000000000e+02
+7.806089999999998e+02 +9.250000000000000e+01
+1.109140000000000e+03 +1.530000000000000e+02
+7.095730000000000e+02 +3.375000000000000e+02
+7.639620000000000e+02 +9.700000000000000e+01
+2.336120000000000e+02 +4.435000000000000e+02
+3.036640000000000e+02 +6.150000000000000e+01
+4.386270000000000e+02 +4.430000000000000e+02
+1.563550000000000e+03 +2.305000000000000e+02
+2.443910000000000e+01 +4.430000000000000e+02
+7.581260000000002e+02 +9.000000000000000e+01
+1.748250000000000e+03 +4.150000000000000e+02
+4.234020000000000e+02 +3.350000000000000e+01
+1.068210000000000e+03 +2.740000000000000e+02
+2.662660000000000e+02 +4.425000000000000e+02
+1.328530000000000e+03 +3.165000000000000e+02
+8.635920000000000e+02 +8.650000000000000e+01
+1.752070000000000e+01 +4.420000000000000e+02
+7.842130000000002e+02 +9.400000000000000e+01
+1.957050000000000e+03 +3.580000000000000e+02
+9.040870000000000e+02 +2.340000000000000e+02
+1.063500000000000e+03 +2.560000000000000e+02
+6.495830000000002e+02 +1.240000000000000e+02
+9.187790000000000e+02 +2.910000000000000e+02
+3.887540000000000e+02 +4.410000000000000e+02
+6.237040000000002e+02 +5.600000000000000e+01
+8.698850000000000e+02 +2.290000000000000e+02
+1.626570000000000e+03 +2.950000000000000e+02
+1.661640000000000e+03 +4.025000000000000e+02
+1.773740000000000e+03 +3.705000000000000e+02
+1.066460000000000e+03 +1.970000000000000e+02
+6.132520000000000e+02 +4.395000000000000e+02
+7.007639999999999e+02 +1.520000000000000e+02
+7.067800000000000e+02 +2.060000000000000e+02
+7.580540000000000e+02 +9.700000000000000e+01
+6.414150000000000e+02 +1.260000000000000e+02
+7.092960000000000e+02 +1.340000000000000e+02
+8.999100000000000e+02 +7.200000000000000e+01
+1.069040000000000e+03 +2.540000000000000e+02
+4.175550000000000e+02 +5.550000000000000e+01
+7.744400000000001e+02 +8.950000000000000e+01
+9.885020000000000e+02 +1.565000000000000e+02
+4.437130000000000e+02 +4.370000000000000e+02
+7.817500000000000e+02 +8.900000000000000e+01
+6.205850000000000e+02 +1.555000000000000e+02
+7.741660000000001e+02 +9.200000000000000e+01
+9.627990000000000e+02 +1.590000000000000e+02
+1.064200000000000e+03 +1.330000000000000e+02
+1.781960000000000e+03 +2.545000000000000e+02
+8.239290000000000e+02 +1.505000000000000e+02
+6.541540000000000e+02 +1.135000000000000e+02
+5.873440000000001e+02 +4.360000000000000e+02
+1.866720000000000e+03 +3.640000000000000e+02
+1.317010000000000e+03 +1.985000000000000e+02
+4.858900000000000e+02 +4.355000000000000e+02
+6.226870000000000e+02 +1.540000000000000e+02
+7.451369999999999e+02 +8.600000000000000e+01
+1.563690000000000e+03 +2.805000000000000e+02
+1.097460000000000e+03 +1.390000000000000e+02
+8.508049999999999e+02 +3.535000000000000e+02
+6.364540000000002e+02 +1.105000000000000e+02
+3.351170000000000e+02 +4.350000000000000e+02
+4.117570000000000e+02 +4.350000000000000e+02
+1.316270000000000e+03 +1.930000000000000e+02
+9.233170000000000e+02 +4.350000000000000e+02
+6.628170000000000e+02 +1.410000000000000e+02
+1.177190000000000e+03 +4.345000000000000e+02
+9.096180000000001e+02 +9.000000000000000e+01
+9.806060000000000e+02 +2.670000000000000e+02
+4.746810000000000e+02 +4.340000000000000e+02
+1.064200000000000e+03 +2.500000000000000e+02
+6.397550000000000e+02 +1.185000000000000e+02
+1.172060000000000e+03 +3.250000000000000e+02
+6.071060000000000e+02 +4.335000000000000e+02
+8.501760000000000e+02 +7.600000000000000e+01
+2.957360000000000e+02 +5.150000000000000e+01
+6.623950000000000e+02 +1.415000000000000e+02
+6.552040000000002e+02 +1.225000000000000e+02
+4.915100000000000e+02 +4.330000000000000e+02
+1.076570000000000e+03 +2.510000000000000e+02
+1.313740000000000e+03 +3.110000000000000e+02
+8.775920000000000e+02 +7.900000000000000e+01
+6.687239999999998e+02 +2.130000000000000e+02
+4.504030000000000e+02 +4.320000000000000e+02
+1.271780000000000e+03 +2.800000000000000e+02
+7.773150000000001e+02 +8.850000000000000e+01
+8.983220000000000e+02 +7.750000000000000e+01
+1.026810000000000e+03 +2.625000000000000e+02
+5.733099999999999e+02 +4.315000000000000e+02
+1.558810000000000e+03 +2.585000000000000e+02
+6.622780000000000e+02 +1.515000000000000e+02
+9.294330000000000e+02 +4.315000000000000e+02
+1.066980000000000e+03 +2.485000000000000e+02
+2.777000000000000e+02 +4.000000000000000e+01
+2.209510000000000e+02 +4.310000000000000e+02
+4.339080000000000e+02 +4.310000000000000e+02
+7.741289999999998e+02 +8.600000000000000e+01
+1.300390000000000e+03 +1.985000000000000e+02
+1.610970000000000e+03 +3.510000000000000e+02
+2.927940000000001e+02 +4.300000000000000e+02
+1.858940000000000e+03 +3.545000000000000e+02
+7.605060000000001e+00 +4.300000000000000e+02
+7.789360000000000e+02 +8.800000000000000e+01
+1.133460000000000e+03 +3.860000000000000e+02
+6.407950000000000e+02 +1.090000000000000e+02
+6.224259999999998e+02 +1.500000000000000e+02
+1.137240000000000e+03 +4.295000000000000e+02
+1.037040000000000e+03 +3.260000000000000e+02
+1.086540000000000e+03 +2.825000000000000e+02
+7.183530000000002e+02 +2.195000000000000e+02
+7.809860000000001e+02 +8.650000000000000e+01
+1.301830000000000e+03 +1.890000000000000e+02
+9.892320000000000e+02 +4.290000000000000e+02
+8.656750000000000e+02 +1.405000000000000e+02
+9.053610000000000e+02 +2.230000000000000e+02
+5.141990000000002e+02 +4.285000000000000e+02
+9.942210000000000e+02 +1.675000000000000e+02
+1.623050000000000e+03 +3.210000000000000e+02
+7.834380000000000e+02 +8.700000000000000e+01
+1.610040000000000e+03 +4.275000000000000e+02
+6.882610000000002e+02 +1.180000000000000e+02
+4.832080000000000e+02 +2.800000000000000e+02
+1.063520000000000e+03 +2.455000000000000e+02
+5.288510000000000e+02 +4.270000000000000e+02
+1.055760000000000e+03 +1.395000000000000e+02
+9.261310000000000e+02 +3.920000000000000e+02
+1.073250000000000e+03 +2.565000000000000e+02
+6.346250000000000e+02 +1.175000000000000e+02
+5.009630000000000e+02 +2.880000000000000e+02
+2.660650000000000e+02 +4.255000000000000e+02
+1.310120000000000e+03 +1.925000000000000e+02
+7.778750000000000e+02 +4.255000000000000e+02
+1.324910000000000e+03 +3.005000000000000e+02
+7.789010000000002e+02 +8.350000000000000e+01
+9.980650000000001e+02 +1.540000000000000e+02
+7.802530000000000e+02 +4.245000000000000e+02
+2.006610000000000e+02 +4.245000000000000e+02
+3.933390000000000e+02 +4.240000000000000e+02
+1.306350000000000e+01 +4.240000000000000e+02
+7.747970000000000e+02 +1.590000000000000e+02
+1.297350000000000e+03 +1.850000000000000e+02
+1.027900000000000e+03 +1.350000000000000e+02
+6.973070000000000e+02 +1.890000000000000e+02
+8.636760000000000e+02 +1.845000000000000e+02
+7.859620000000000e+02 +8.350000000000000e+01
+9.747850000000000e+02 +4.235000000000000e+02
+1.778470000000000e+03 +2.465000000000000e+02
+4.976240000000000e+02 +2.830000000000000e+02
+4.926270000000000e+02 +4.500000000000000e+01
+6.002180000000002e+02 +4.230000000000000e+02
+9.124990000000000e+02 +1.150000000000000e+02
+1.628190000000000e+03 +2.835000000000000e+02
+5.489970000000000e+02 +4.220000000000000e+02
+7.839830000000002e+02 +1.580000000000000e+02
+1.113670000000000e+03 +4.220000000000000e+02
+4.185930000000000e+02 +4.215000000000000e+02
+1.273540000000000e+03 +2.740000000000000e+02
+1.860420000000000e+03 +3.515000000000000e+02
+2.676790000000000e+02 +4.210000000000000e+02
+1.684160000000000e+03 +4.005000000000000e+02
+9.038720000000000e+02 +1.140000000000000e+02
+9.101680000000000e+02 +2.110000000000000e+02
+9.944970000000000e+02 +1.305000000000000e+02
+1.135040000000000e+03 +1.310000000000000e+02
+1.486250000000000e+03 +2.855000000000000e+02
+4.208060000000000e+02 +4.200000000000000e+02
+8.345219999999999e+00 +4.200000000000000e+02
+1.402710000000000e+03 +4.200000000000000e+02
+1.911740000000000e+03 +2.730000000000000e+02
+1.818440000000000e+03 +4.200000000000000e+02
+6.794730000000002e+02 +9.800000000000000e+01
+1.626990000000000e+03 +4.005000000000000e+02
+2.577850000000000e+02 +4.190000000000000e+02
+6.401250000000000e+02 +9.450000000000000e+01
+1.072430000000000e+03 +2.400000000000000e+02
+5.942530000000000e+02 +4.185000000000000e+02
+8.793260000000000e+02 +4.185000000000000e+02
+7.510230000000000e+02 +1.280000000000000e+02
+3.917540000000000e+02 +4.180000000000000e+02
+1.303900000000000e+03 +1.825000000000000e+02
+1.114160000000000e+03 +1.230000000000000e+02
+9.349380000000000e+02 +4.175000000000000e+02
+6.453880000000000e+02 +1.050000000000000e+02
+8.651790000000000e+02 +9.250000000000000e+01
+1.750980000000000e+03 +3.880000000000000e+02
+1.363620000000000e+03 +2.635000000000000e+02
+6.629030000000000e+02 +1.055000000000000e+02
+1.754900000000000e+01 +4.165000000000000e+02
+9.828470000000000e+02 +1.565000000000000e+02
+1.089090000000000e+03 +1.345000000000000e+02
+6.351690000000000e+02 +1.090000000000000e+02
+1.680800000000000e+03 +3.915000000000000e+02
+5.057490000000000e+02 +4.155000000000000e+02
+1.135580000000000e+03 +4.155000000000000e+02
+8.077769999999998e+02 +4.155000000000000e+02
+6.108830000000000e+02 +1.010000000000000e+02
+5.663030000000000e+02 +4.150000000000000e+02
+1.189910000000000e+03 +2.805000000000000e+02
+9.384070000000000e+02 +3.860000000000000e+02
+8.817890000000000e+02 +1.820000000000000e+02
+6.556820000000000e+02 +1.050000000000000e+02
+3.024490000000000e+02 +4.145000000000000e+02
+1.666610000000000e+03 +3.880000000000000e+02
+1.302520000000000e+03 +2.645000000000000e+02
+7.671180000000001e+02 +1.530000000000000e+02
+9.943930000000000e+02 +1.210000000000000e+02
+6.510150000000000e+02 +1.080000000000000e+02
+2.456280000000000e+02 +4.135000000000000e+02
+1.606370000000000e+03 +4.135000000000000e+02
+7.688049999999999e+02 +1.490000000000000e+02
+1.116700000000000e+03 +4.135000000000000e+02
+9.066369999999999e+02 +1.350000000000000e+02
+5.687750000000000e+02 +4.130000000000000e+02
+2.326400000000000e+02 +4.125000000000000e+02
+6.210010000000000e+02 +1.355000000000000e+02
+9.352230000000000e+02 +3.870000000000000e+02
+4.255320000000000e+02 +4.120000000000000e+02
+1.091530000000000e+03 +2.530000000000000e+02
+8.628889999999999e+02 +8.500000000000000e+01
+9.898160000000000e+02 +1.365000000000000e+02
+9.256079999999999e+02 +2.080000000000000e+02
+1.611860000000000e+03 +4.115000000000000e+02
+7.775610000000000e+02 +1.485000000000000e+02
+5.298190000000000e+02 +4.110000000000000e+02
+1.300420000000000e+03 +1.750000000000000e+02
+3.725490000000000e+02 +4.105000000000000e+02
+6.562730000000000e+02 +2.255000000000000e+02
+2.128540000000000e+02 +4.105000000000000e+02
+1.095940000000000e+03 +1.155000000000000e+02
+1.370250000000000e+03 +1.960000000000000e+02
+1.320660000000000e+03 +1.800000000000000e+02
+9.001600000000000e+02 +1.235000000000000e+02
+1.263360000000000e+03 +1.455000000000000e+02
+6.528220000000000e+02 +9.250000000000000e+01
+9.909760000000000e+02 +1.435000000000000e+02
+6.507260000000000e+02 +1.055000000000000e+02
+7.886360000000002e+02 +1.495000000000000e+02
+1.540190000000000e+03 +2.515000000000000e+02
+9.056690000000000e+02 +1.240000000000000e+02
+3.979520000000000e+02 +4.085000000000000e+02
+6.176460000000000e+02 +8.950000000000000e+01
+4.053740000000000e+02 +4.085000000000000e+02
+1.006600000000000e+03 +3.575000000000000e+02
+8.021770000000000e+00 +4.080000000000000e+02
+7.675139999999999e+02 +1.485000000000000e+02
+1.255750000000000e+03 +2.395000000000000e+02
+1.288190000000000e+03 +2.530000000000000e+02
+4.025250000000000e+02 +4.075000000000000e+02
+1.320460000000000e+01 +4.075000000000000e+02
+6.636010000000001e+02 +1.050000000000000e+02
+3.707200000000000e+02 +4.060000000000000e+02
+1.733810000000000e+03 +3.795000000000000e+02
+5.698120000000000e+02 +4.055000000000000e+02
+6.235150000000000e+02 +1.285000000000000e+02
+5.849400000000001e+02 +4.050000000000000e+02
+7.785060000000002e+02 +1.465000000000000e+02
+1.313310000000000e+03 +1.710000000000000e+02
+2.776000000000000e+02 +4.045000000000000e+02
+4.103670000000000e+02 +4.040000000000000e+02
+1.270850000000000e+01 +4.040000000000000e+02
+7.790590000000000e+02 +1.425000000000000e+02
+9.191770000000000e+02 +1.960000000000000e+02
+1.016280000000000e+03 +2.370000000000000e+02
+2.789650000000000e+00 +4.035000000000000e+02
+7.844889999999998e+02 +1.450000000000000e+02
+8.588220000000000e+02 +1.235000000000000e+02
+1.208340000000000e+03 +2.115000000000000e+02
+7.576480000000000e+02 +1.400000000000000e+02
+3.892970000000000e+02 +4.025000000000000e+02
+1.235950000000000e+03 +4.025000000000000e+02
+6.803670000000000e+02 +1.770000000000000e+02
+1.219320000000000e+03 +2.040000000000000e+02
+3.330470000000000e+02 +4.015000000000000e+02
+1.008970000000000e+03 +2.370000000000000e+02
+6.381200000000000e+02 +9.600000000000000e+01
+1.767390000000000e+03 +2.260000000000000e+02
+5.915940000000001e+02 +4.010000000000000e+02
+3.488570000000000e+02 +4.010000000000000e+02
+7.874660000000000e+02 +1.440000000000000e+02
+8.617800000000000e+02 +1.625000000000000e+02
+7.498650000000000e+02 +5.550000000000000e+01
+4.884960000000000e+02 +4.005000000000000e+02
+7.796630000000000e+02 +1.400000000000000e+02
+9.845710000000000e+02 +1.380000000000000e+02
+3.165290000000000e+02 +4.000000000000000e+02
+1.304850000000000e+03 +1.695000000000000e+02
+1.775780000000000e+03 +2.280000000000000e+02
+4.844880000000001e+02 +3.850000000000000e+01
+5.489830000000002e+02 +3.995000000000000e+02
+4.124000000000000e+02 +3.995000000000000e+02
+1.108370000000000e+03 +3.995000000000000e+02
+7.035889999999998e+02 +2.975000000000000e+02
+5.229590000000002e+02 +3.990000000000000e+02
+1.282230000000000e+03 +1.655000000000000e+02
+1.232630000000000e+03 +3.060000000000000e+02
+5.295419999999998e+02 +3.985000000000000e+02
+3.812870000000000e+02 +3.985000000000000e+02
+1.868980000000000e+03 +3.170000000000000e+02
+1.308340000000000e+01 +3.980000000000000e+02
+3.161210000000000e+02 +3.975000000000000e+02
+7.034260000000000e+02 +1.825000000000000e+02
+8.720060000000002e+02 +1.665000000000000e+02
+5.057890000000000e+02 +3.975000000000000e+02
+1.630090000000000e+03 +3.535000000000000e+02
+3.663840000000000e+02 +3.975000000000000e+02
+6.955490000000000e+02 +3.970000000000000e+02
+4.202620000000000e+02 +3.970000000000000e+02
+3.640370000000000e-01 +3.970000000000000e+02
+7.864639999999998e+02 +1.415000000000000e+02
+8.466400000000000e+02 +1.220000000000000e+02
+3.001880000000000e+02 +3.965000000000000e+02
+4.867590000000000e+02 +2.610000000000000e+02
+5.614310000000000e+02 +3.965000000000000e+02
+6.581070000000000e+02 +1.240000000000000e+02
+7.852700000000000e+02 +1.380000000000000e+02
+9.042150000000000e+02 +1.210000000000000e+02
+8.712650000000000e+02 +3.960000000000000e+02
+9.260570000000000e+02 +1.940000000000000e+02
+5.049020000000000e+00 +3.955000000000000e+02
+8.998400000000000e+02 +8.650000000000000e+01
+2.496320000000000e+02 +3.950000000000000e+02
+7.011510000000002e+02 +1.840000000000000e+02
+1.000640000000000e+03 +1.090000000000000e+02
+7.898260000000000e+02 +1.415000000000000e+02
+9.228280000000000e+02 +3.950000000000000e+02
+6.355430000000000e+02 +9.050000000000000e+01
+2.753880000000000e+00 +3.940000000000000e+02
+3.535860000000000e+02 +3.935000000000000e+02
+9.380150000000000e-01 +3.935000000000000e+02
+1.937250000000000e+02 +3.930000000000000e+02
+6.661750000000000e+02 +1.750000000000000e+02
+1.063280000000000e+03 +2.165000000000000e+02
+6.345309999999999e+02 +8.350000000000000e+01
+7.924450000000001e+02 +1.370000000000000e+02
+1.369320000000000e+03 +3.930000000000000e+02
+1.011270000000000e+03 +2.275000000000000e+02
+1.067540000000000e+03 +2.255000000000000e+02
+4.138400000000000e+02 +3.925000000000000e+02
+3.845800000000000e+02 +3.925000000000000e+02
+1.039800000000000e+03 +1.745000000000000e+02
+1.429310000000000e+03 +1.705000000000000e+02
+1.403950000000000e+03 +3.575000000000000e+02
+4.410880000000000e+01 +3.910000000000000e+02
+1.005690000000000e+03 +3.435000000000000e+02
+6.248070000000000e+02 +1.130000000000000e+02
+1.239560000000000e+03 +3.910000000000000e+02
+1.093630000000000e+03 +3.910000000000000e+02
+1.101090000000000e+02 +3.905000000000000e+02
+5.400530000000000e+02 +3.905000000000000e+02
+1.294780000000000e+03 +1.650000000000000e+02
+1.164340000000000e+03 +3.415000000000000e+02
+8.638020000000000e+02 +2.155000000000000e+02
+3.336980000000001e+02 +3.895000000000000e+02
+1.108270000000000e+03 +1.795000000000000e+02
+6.405160000000000e+02 +8.750000000000000e+01
+1.862580000000000e+03 +3.400000000000000e+02
+7.852569999999999e+02 +1.350000000000000e+02
+1.027950000000000e+03 +3.885000000000000e+02
+1.727630000000000e+03 +3.565000000000000e+02
+9.191260000000000e+02 +3.885000000000000e+02
+5.526530000000000e+02 +3.880000000000000e+02
+5.339360000000000e+02 +3.880000000000000e+02
+7.625740000000000e+02 +1.310000000000000e+02
+9.796520000000000e+02 +1.325000000000000e+02
+2.844360000000000e+01 +3.875000000000000e+02
+1.001110000000000e+03 +2.245000000000000e+02
+3.360290000000000e+02 +2.850000000000000e+01
+2.153660000000000e+01 +3.870000000000000e+02
+1.299840000000000e+03 +2.415000000000000e+02
+5.231440000000000e+02 +3.870000000000000e+02
+6.213590000000000e+02 +1.115000000000000e+02
+1.289620000000000e+03 +1.585000000000000e+02
+1.766750000000000e+03 +3.160000000000000e+02
+8.626120000000000e+02 +1.590000000000000e+02
+5.412120000000000e+02 +3.865000000000000e+02
+7.934580000000002e+02 +1.320000000000000e+02
+1.700420000000000e+03 +3.590000000000000e+02
+1.155850000000000e+03 +1.915000000000000e+02
+1.086710000000000e+03 +2.255000000000000e+02
+1.002850000000000e+03 +3.860000000000000e+02
+6.501700000000000e+02 +7.000000000000000e+01
+7.911319999999999e+02 +1.330000000000000e+02
+2.794470000000000e+02 +9.800000000000000e+01
+6.645710000000000e+02 +7.650000000000000e+01
+1.318250000000000e+03 +3.850000000000000e+02
+5.005700000000000e+01 +3.850000000000000e+02
+3.963490000000000e+01 +3.845000000000000e+02
+4.833970000000000e+02 +2.540000000000000e+02
+6.890960000000000e+02 +1.670000000000000e+02
+2.818490000000000e+02 +9.750000000000000e+01
+5.551890000000000e+02 +3.845000000000000e+02
+3.893450000000000e+02 +3.845000000000000e+02
+3.461070000000000e+02 +3.050000000000000e+01
+9.096319999999999e+02 +3.845000000000000e+02
+8.725630000000000e+02 +9.850000000000000e+01
+2.130320000000000e+01 +3.840000000000000e+02
+1.395060000000000e+03 +2.265000000000000e+02
+6.228990000000000e+02 +7.800000000000000e+01
+5.307290000000000e+02 +3.835000000000000e+02
+1.865250000000000e+03 +3.340000000000000e+02
+6.988520000000000e+02 +2.045000000000000e+02
+3.540060000000000e+02 +3.830000000000000e+02
+7.965470000000000e+02 +1.340000000000000e+02
+8.919510000000000e+02 +1.085000000000000e+02
+6.580080000000000e+02 +6.900000000000000e+01
+2.650280000000000e+02 +3.820000000000000e+02
+8.897850000000000e+02 +1.090000000000000e+02
+4.927900000000000e+02 +2.485000000000000e+02
+7.998020000000000e+02 +1.330000000000000e+02
+1.374610000000000e+03 +3.665000000000000e+02
+1.340370000000000e+03 +3.310000000000000e+02
+1.350740000000000e+01 +3.810000000000000e+02
+7.440720000000000e+02 +1.355000000000000e+02
+1.319370000000000e+03 +2.580000000000000e+02
+1.595470000000000e+01 +3.805000000000000e+02
+9.990180000000000e+02 +3.805000000000000e+02
+9.009800000000000e+02 +1.095000000000000e+02
+3.761790000000000e+02 +3.805000000000000e+02
+7.526890000000000e+02 +1.380000000000000e+02
+7.923000000000000e+02 +1.310000000000000e+02
+1.563600000000000e+01 +3.795000000000000e+02
+1.029560000000000e+03 +2.200000000000000e+02
+1.296780000000000e+03 +1.535000000000000e+02
+8.791000000000000e+02 +3.795000000000000e+02
+4.988460000000000e+02 +3.355000000000000e+02
+4.861030000000000e+02 +3.380000000000000e+02
+5.450200000000000e+02 +3.785000000000000e+02
+1.880120000000000e+03 +3.315000000000000e+02
+1.302930000000000e+03 +1.500000000000000e+02
+8.911600000000000e+02 +1.770000000000000e+02
+7.645510000000000e+02 +1.255000000000000e+02
+8.766950000000001e+02 +3.780000000000000e+02
+8.909390000000000e+02 +3.780000000000000e+02
+4.234310000000000e+01 +3.775000000000000e+02
+1.374570000000000e+03 +2.210000000000000e+02
+3.773160000000000e+02 +3.770000000000000e+02
+1.362230000000000e+03 +3.770000000000000e+02
+8.948150000000001e+02 +9.650000000000000e+01
+8.907780000000000e+02 +1.710000000000000e+02
+5.233330000000002e+02 +3.765000000000000e+02
+3.361230000000001e+02 +3.765000000000000e+02
+7.931720000000000e+02 +1.285000000000000e+02
+3.778450000000000e+02 +3.760000000000000e+02
+1.879880000000000e+02 +3.760000000000000e+02
+1.027740000000000e+03 +3.760000000000000e+02
+4.975350000000000e+02 +3.440000000000000e+02
+7.967680000000000e+02 +1.280000000000000e+02
+4.792600000000000e+02 +3.365000000000000e+02
+1.808570000000000e+02 +3.750000000000000e+02
+1.321850000000000e+03 +3.750000000000000e+02
+6.351050000000000e+02 +6.750000000000000e+01
+9.756900000000001e+02 +1.165000000000000e+02
+1.438430000000000e+03 +3.740000000000000e+02
+2.774880000000000e+02 +9.650000000000000e+01
+3.315250000000000e+02 +3.735000000000000e+02
+8.511050000000000e+02 +3.735000000000000e+02
+2.939050000000000e+02 +8.750000000000000e+01
+1.184230000000000e+03 +3.730000000000000e+02
+3.907430000000001e+02 +3.725000000000000e+02
+3.984790000000000e+02 +3.725000000000000e+02
+7.882919999999998e+02 +1.265000000000000e+02
+1.888530000000000e+01 +3.720000000000000e+02
+4.892000000000000e+02 +3.720000000000000e+02
+6.619220000000000e+02 +1.030000000000000e+02
+1.664540000000000e+03 +2.350000000000000e+02
+3.167150000000000e+02 +2.250000000000000e+01
+5.387630000000000e+02 +3.715000000000000e+02
+1.032060000000000e+03 +1.605000000000000e+02
+7.057650000000000e+02 +2.735000000000000e+02
+9.783380000000000e+02 +1.180000000000000e+02
+1.555800000000000e+03 +2.165000000000000e+02
+8.429160000000001e+02 +1.010000000000000e+02
+1.063110000000000e+03 +1.915000000000000e+02
+7.925520000000000e+02 +1.275000000000000e+02
+9.940010000000000e+02 +1.185000000000000e+02
+8.992460000000002e+02 +1.040000000000000e+02
+8.624530000000000e+02 +3.700000000000000e+02
+5.410660000000000e+02 +3.695000000000000e+02
+7.907550000000000e+02 +1.230000000000000e+02
+5.297360000000000e+02 +3.690000000000000e+02
+7.666419999999998e+02 +1.210000000000000e+02
+8.603570000000000e+02 +1.000000000000000e+02
+6.400260000000000e+02 +1.595000000000000e+02
+9.844730000000000e+02 +9.600000000000000e+01
+9.077800000000000e+02 +9.600000000000000e+01
+1.029350000000000e+03 +2.730000000000000e+02
+4.458070000000000e+02 +3.680000000000000e+02
+7.898380000000002e+02 +1.265000000000000e+02
+1.320360000000000e+02 +3.680000000000000e+02
+1.000930000000000e+03 +2.030000000000000e+02
+1.072040000000000e+03 +1.925000000000000e+02
+3.107680000000000e+02 +3.675000000000000e+02
+1.148780000000000e+03 +2.605000000000000e+02
+3.702080000000000e+02 +3.665000000000000e+02
+7.903660000000001e+02 +1.710000000000000e+02
+6.490060000000000e+02 +1.550000000000000e+02
+3.562150000000000e+02 +3.665000000000000e+02
+9.080820000000000e+02 +3.665000000000000e+02
+9.897300000000000e+02 +7.350000000000000e+01
+1.016490000000000e+03 +1.990000000000000e+02
+2.779060000000000e+02 +8.200000000000000e+01
+7.110730000000000e+02 +2.570000000000000e+02
+2.946970000000000e+02 +3.655000000000000e+02
+7.900100000000000e+02 +1.205000000000000e+02
+4.790970000000000e+02 +3.000000000000000e+01
+2.790430000000000e+02 +9.000000000000000e+01
+7.925889999999998e+02 +1.265000000000000e+02
+1.322990000000000e+03 +1.375000000000000e+02
+1.084370000000000e+03 +1.515000000000000e+02
+8.154989999999998e+02 +1.710000000000000e+02
+2.989290000000001e+02 +3.645000000000000e+02
+2.261130000000000e+02 +3.645000000000000e+02
+1.015250000000000e+03 +2.010000000000000e+02
+1.318050000000000e+03 +3.520000000000000e+02
+7.901550000000000e+02 +1.265000000000000e+02
+8.171089999999998e+02 +1.835000000000000e+02
+3.031890000000000e+02 +3.635000000000000e+02
+7.914660000000000e+02 +1.200000000000000e+02
+5.820359999999999e+02 +1.615000000000000e+02
+6.155810000000000e+02 +1.520000000000000e+02
+4.847740000000000e+02 +3.265000000000000e+02
+6.357390000000000e+02 +1.550000000000000e+02
+3.910100000000000e+02 +3.625000000000000e+02
+6.602719999999998e+02 +8.450000000000000e+01
+7.468960000000002e+02 +1.185000000000000e+02
+1.908530000000000e+02 +3.620000000000000e+02
+1.118040000000000e+03 +1.615000000000000e+02
+6.999000000000000e+02 +3.620000000000000e+02
+2.779710000000000e+01 +3.620000000000000e+02
+1.000130000000000e+03 +2.715000000000000e+02
+8.984510000000000e+02 +3.340000000000000e+02
+1.217620000000000e+03 +1.985000000000000e+02
+6.593000000000000e+02 +1.555000000000000e+02
+7.882860000000002e+02 +1.200000000000000e+02
+1.324910000000000e+03 +3.600000000000000e+02
+1.273110000000000e+03 +3.600000000000000e+02
+0.000000000000000e+00 +3.600000000000000e+02
+1.454640000000000e+03 +3.600000000000000e+02
+3.603480000000000e+02 +3.600000000000000e+02
+1.321090000000000e+03 +3.130000000000000e+02
+1.039700000000000e+03 +1.840000000000000e+02
+1.907970000000000e+03 +3.600000000000000e+02
+3.526770000000000e+02 +3.600000000000000e+02
+1.347210000000000e+03 +3.600000000000000e+02
+2.720850000000000e+02 +3.600000000000000e+02
+6.526250000000000e+02 +1.525000000000000e+02
+1.845420000000000e+02 +3.595000000000000e+02
+7.110260000000002e+02 +2.820000000000000e+02
+7.804620000000000e+02 +1.170000000000000e+02
+1.171210000000000e+03 +3.590000000000000e+02
+1.069920000000000e+03 +1.825000000000000e+02
+7.941310000000002e+02 +1.195000000000000e+02
+7.127970000000000e+02 +7.600000000000000e+01
+6.780230000000000e+02 +1.220000000000000e+02
+7.790450000000000e+02 +1.615000000000000e+02
+5.221550000000000e+00 +3.580000000000000e+02
+1.147920000000000e+03 +3.580000000000000e+02
+1.126650000000000e+03 +3.115000000000000e+02
+1.549590000000000e+03 +2.010000000000000e+02
+9.029990000000000e+02 +3.575000000000000e+02
+1.608710000000000e+03 +2.150000000000000e+02
+1.005070000000000e+03 +1.950000000000000e+02
+6.603090000000000e+02 +1.540000000000000e+02
+3.101310000000000e+02 +3.570000000000000e+02
+3.519610000000000e+02 +1.000000000000000e+02
+7.947830000000000e+02 +1.220000000000000e+02
+1.135260000000000e+03 +3.565000000000000e+02
+4.836450000000000e+02 +5.600000000000000e+01
+3.311000000000000e+02 +3.560000000000000e+02
+7.728290000000000e+02 +1.200000000000000e+02
+9.652700000000000e+02 +1.175000000000000e+02
+1.008620000000000e+03 +1.530000000000000e+02
+3.019430000000000e+02 +3.555000000000000e+02
+1.589230000000000e+03 +3.340000000000000e+02
+4.887780000000000e+02 +3.550000000000000e+02
+3.726030000000000e+02 +3.550000000000000e+02
+9.742020000000000e+02 +1.820000000000000e+02
+1.164630000000000e+03 +3.550000000000000e+02
+4.881780000000001e+02 +3.210000000000000e+02
+6.365900000000000e+02 +1.530000000000000e+02
+4.957170000000000e+02 +3.545000000000000e+02
+7.954600000000000e+02 +1.215000000000000e+02
+5.272740000000000e+02 +3.540000000000000e+02
+3.142620000000000e+02 +3.540000000000000e+02
+8.586230000000000e+02 +3.540000000000000e+02
+6.989310000000000e+02 +3.535000000000000e+02
+1.144220000000000e+03 +3.535000000000000e+02
+7.911569999999998e+02 +1.180000000000000e+02
+1.319050000000000e+03 +2.260000000000000e+02
+1.412410000000000e+03 +1.320000000000000e+02
+9.759950000000000e+02 +1.845000000000000e+02
+9.126210000000000e+02 +9.350000000000000e+01
+8.021799999999999e+02 +1.400000000000000e+02
+7.125670000000000e+02 +2.415000000000000e+02
+7.482030000000000e+02 +1.120000000000000e+02
+2.622250000000000e+02 +3.520000000000000e+02
+4.944570000000000e+02 +1.205000000000000e+02
+1.653800000000000e+03 +2.790000000000000e+02
+7.923739999999998e+02 +1.180000000000000e+02
+9.928860000000000e+02 +1.840000000000000e+02
+1.123130000000000e+03 +3.050000000000000e+02
+6.596280000000000e+02 +7.950000000000000e+01
+1.288680000000000e+03 +3.170000000000000e+02
+6.337569999999999e+02 +1.465000000000000e+02
+1.638110000000000e+02 +3.505000000000000e+02
+2.996100000000000e+02 +3.505000000000000e+02
+9.916960000000000e+02 +1.860000000000000e+02
+1.026900000000000e+03 +1.385000000000000e+02
+8.890820000000000e+02 +3.505000000000000e+02
+1.219470000000000e+03 +1.865000000000000e+02
+6.608339999999999e+02 +1.420000000000000e+02
+7.869830000000002e+02 +1.180000000000000e+02
+1.289520000000000e+03 +2.160000000000000e+02
+8.601400000000000e+02 +8.750000000000000e+01
+3.718320000000000e+02 +3.495000000000000e+02
+9.634860000000000e+02 +1.790000000000000e+02
+6.311380000000000e+02 +6.300000000000000e+01
+2.413060000000000e+02 +3.490000000000000e+02
+3.104190000000001e+02 +3.490000000000000e+02
+6.637310000000001e+02 +7.400000000000000e+01
+1.132320000000000e+03 +3.490000000000000e+02
+6.839850000000000e+02 +2.445000000000000e+02
+6.447500000000000e+02 +1.385000000000000e+02
+7.914310000000000e+02 +1.190000000000000e+02
+9.940599999999999e+02 +1.860000000000000e+02
+1.296440000000000e+03 +3.485000000000000e+02
+1.279810000000000e+03 +3.165000000000000e+02
+5.290860000000000e+02 +3.480000000000000e+02
+3.403190000000000e+02 +3.480000000000000e+02
+8.402700000000000e+02 +8.400000000000000e+01
+4.145830000000000e+02 +3.475000000000000e+02
+1.319950000000000e+02 +3.475000000000000e+02
+9.940250000000000e+02 +1.905000000000000e+02
+1.733230000000000e+03 +2.940000000000000e+02
+2.733630000000000e+02 +3.470000000000000e+02
+1.066350000000000e+03 +1.850000000000000e+02
+6.100050000000000e+02 +1.445000000000000e+02
+3.618370000000000e+02 +3.465000000000000e+02
+3.357890000000000e+02 +3.460000000000000e+02
+7.050820000000000e+02 +2.345000000000000e+02
+1.066970000000000e+03 +1.705000000000000e+02
+7.701270000000000e+02 +1.560000000000000e+02
+2.793050000000000e+02 +3.455000000000000e+02
+3.454250000000000e+00 +3.455000000000000e+02
+1.309020000000000e+03 +2.205000000000000e+02
+1.601410000000000e+03 +2.670000000000000e+02
+4.002280000000000e+02 +3.450000000000000e+02
+7.745340000000000e+02 +1.130000000000000e+02
+1.074680000000000e+03 +3.450000000000000e+02
+1.177710000000000e+03 +3.150000000000000e+02
+7.808670000000000e+02 +1.160000000000000e+02
+9.470630000000000e+01 +3.440000000000000e+02
+1.078980000000000e+03 +1.470000000000000e+02
+1.713330000000000e+03 +3.280000000000000e+02
+6.191820000000000e+02 +7.050000000000000e+01
+9.011440000000000e+02 +1.260000000000000e+02
+6.390620000000000e+02 +6.800000000000000e+01
+7.865740000000000e+02 +1.125000000000000e+02
+1.256910000000000e+03 +1.745000000000000e+02
+2.038860000000000e+02 +3.425000000000000e+02
+3.219900000000000e+02 +3.425000000000000e+02
+3.398300000000000e+02 +8.900000000000000e+01
+1.435060000000000e+03 +3.425000000000000e+02
+7.845419999999998e+02 +1.150000000000000e+02
+9.184610000000000e+02 +8.500000000000000e+01
+2.504420000000000e+02 +3.415000000000000e+02
+2.105390000000000e+02 +3.415000000000000e+02
+8.970910000000000e+02 +3.415000000000000e+02
+2.695940000000000e+02 +3.410000000000000e+02
+2.917650000000000e+02 +3.410000000000000e+02
+7.884550000000000e+02 +1.115000000000000e+02
+1.100920000000000e+03 +3.410000000000000e+02
+1.671820000000000e+01 +3.410000000000000e+02
+6.227710000000000e+02 +1.340000000000000e+02
+4.974550000000000e+02 +3.105000000000000e+02
+7.841790000000000e+02 +1.155000000000000e+02
+1.329560000000000e+01 +3.400000000000000e+02
+1.656170000000000e+02 +3.395000000000000e+02
+2.855940000000000e+02 +3.395000000000000e+02
+7.005950000000000e+02 +2.635000000000000e+02
+6.222540000000000e+02 +1.390000000000000e+02
+3.072270000000001e+02 +3.390000000000000e+02
+6.478950000000000e+02 +1.390000000000000e+02
+7.869390000000000e+02 +1.165000000000000e+02
+1.816110000000000e+03 +2.605000000000000e+02
+2.455990000000000e+01 +3.385000000000000e+02
+1.212890000000000e+03 +2.755000000000000e+02
+7.668580000000002e+02 +1.120000000000000e+02
+1.294770000000000e+03 +2.090000000000000e+02
+8.997030000000000e+02 +1.260000000000000e+02
+2.240680000000000e+00 +3.375000000000000e+02
+7.882270000000000e+02 +1.095000000000000e+02
+1.071810000000000e+03 +1.795000000000000e+02
+2.851690000000001e+02 +3.360000000000000e+02
+3.035760000000000e+02 +3.360000000000000e+02
+9.815720000000000e+02 +1.790000000000000e+02
+1.014460000000000e+03 +2.855000000000000e+02
+7.892170000000000e+02 +1.095000000000000e+02
+2.747590000000000e+02 +3.350000000000000e+02
+1.021770000000000e+03 +2.840000000000000e+02
+6.457240000000000e+02 +1.385000000000000e+02
+1.737380000000000e+02 +3.345000000000000e+02
+6.807139999999998e+02 +3.345000000000000e+02
+1.594020000000000e+03 +2.575000000000000e+02
+1.272680000000000e+03 +2.920000000000000e+02
+4.877120000000000e+02 +3.090000000000000e+02
+2.448330000000000e+02 +3.340000000000000e+02
+7.888370000000000e+02 +1.150000000000000e+02
+9.919730000000000e+02 +1.750000000000000e+02
+7.892510000000002e+02 +1.085000000000000e+02
+1.380430000000000e+03 +2.105000000000000e+02
+9.182960000000000e+02 +3.335000000000000e+02
+8.072619999999999e+02 +1.605000000000000e+02
+6.581289999999998e+02 +6.550000000000000e+01
+1.480550000000000e+03 +3.130000000000000e+02
+2.375030000000000e+02 +3.325000000000000e+02
+3.007620000000000e+02 +3.325000000000000e+02
+7.906020000000000e+02 +1.855000000000000e+02
+1.880420000000000e+01 +3.325000000000000e+02
+6.641110000000001e+02 +1.330000000000000e+02
+1.295360000000000e+03 +2.040000000000000e+02
+2.800930000000000e+02 +3.315000000000000e+02
+6.192619999999999e+02 +5.750000000000000e+01
+1.064350000000000e+03 +1.575000000000000e+02
+6.455309999999999e+02 +1.260000000000000e+02
+3.191470000000000e+02 +3.310000000000000e+02
+1.069070000000000e+03 +1.685000000000000e+02
+6.568530000000002e+02 +1.325000000000000e+02
+7.928099999999999e+02 +1.870000000000000e+02
+6.832020000000000e+02 +3.305000000000000e+02
+1.950290000000000e+03 +2.725000000000000e+02
+6.720010000000002e+02 +1.270000000000000e+02
+7.808020000000000e+02 +1.055000000000000e+02
+2.667780000000000e+02 +3.295000000000000e+02
+7.945790000000000e+02 +1.065000000000000e+02
+1.311090000000000e+03 +2.025000000000000e+02
+1.635180000000000e+03 +3.030000000000000e+02
+2.152440000000000e+02 +3.285000000000000e+02
+9.733690000000000e+02 +1.740000000000000e+02
+6.608900000000000e+02 +1.355000000000000e+02
+9.005280000000000e+02 +1.125000000000000e+02
+2.347740000000000e+02 +3.280000000000000e+02
+2.718720000000000e+02 +3.275000000000000e+02
+4.900850000000000e+02 +3.035000000000000e+02
+2.781070000000000e+02 +5.000000000000000e+01
+3.036030000000000e+02 +3.270000000000000e+02
+1.295090000000000e+03 +2.030000000000000e+02
+1.207420000000000e+03 +2.620000000000000e+02
+7.951260000000002e+02 +1.850000000000000e+02
+1.069920000000000e+03 +1.315000000000000e+02
+6.629280000000000e+02 +1.390000000000000e+02
+8.486460000000002e+02 +1.120000000000000e+02
+2.668420000000000e+02 +3.255000000000000e+02
+1.069370000000000e+03 +1.170000000000000e+02
+1.166600000000000e+01 +3.255000000000000e+02
+6.356770000000000e+02 +1.310000000000000e+02
+1.785950000000000e+02 +3.250000000000000e+02
+6.594720000000000e+02 +5.950000000000000e+01
+2.533330000000000e+02 +3.245000000000000e+02
+7.886980000000000e+02 +1.030000000000000e+02
+1.084730000000000e+03 +1.175000000000000e+02
+3.231950000000000e+02 +3.245000000000000e+02
+2.250950000000000e+02 +3.240000000000000e+02
+2.890830000000000e+02 +3.240000000000000e+02
+2.631420000000000e+02 +3.240000000000000e+02
+6.758960000000002e+02 +3.240000000000000e+02
+1.018660000000000e+03 +3.240000000000000e+02
+3.111660000000000e+02 +3.235000000000000e+02
+5.006770000000000e+02 +3.030000000000000e+02
+9.772260000000000e+02 +1.605000000000000e+02
+2.524400000000000e+02 +3.235000000000000e+02
+1.673390000000000e+03 +3.235000000000000e+02
+8.780590000000000e+02 +3.230000000000000e+02
+1.313660000000000e+03 +2.880000000000000e+02
+3.252370000000000e+02 +7.550000000000000e+01
+1.079120000000000e+03 +1.335000000000000e+02
+9.164630000000000e+02 +3.225000000000000e+02
+3.198080000000000e+02 +3.220000000000000e+02
+1.473410000000000e+03 +3.005000000000000e+02
+2.747980000000000e+02 +3.215000000000000e+02
+1.176320000000000e+03 +3.215000000000000e+02
+9.310250000000000e+02 +2.880000000000000e+02
+9.851650000000000e+02 +2.300000000000000e+02
+2.046500000000000e+02 +3.205000000000000e+02
+6.312300000000001e+01 +3.205000000000000e+02
+1.309730000000000e+03 +1.955000000000000e+02
+1.057060000000000e+03 +1.925000000000000e+02
+1.672320000000000e+03 +2.780000000000000e+02
+8.624570000000000e+02 +1.110000000000000e+02
+1.077640000000000e+03 +1.570000000000000e+02
+3.392060000000000e+01 +3.200000000000000e+02
+6.229910000000000e+02 +5.000000000000000e+01
+1.009230000000000e+03 +1.080000000000000e+02
+7.840139999999999e+02 +1.720000000000000e+02
+1.124840000000000e+03 +1.150000000000000e+02
+1.758370000000000e+03 +2.610000000000000e+02
+1.288810000000000e+03 +2.235000000000000e+02
+1.081810000000000e+02 +3.190000000000000e+02
+2.535050000000000e+02 +3.190000000000000e+02
+9.771270000000000e+02 +1.695000000000000e+02
+6.212660000000000e+02 +4.400000000000000e+01
+2.771110000000000e+02 +3.185000000000000e+02
+7.802760000000002e+02 +1.780000000000000e+02
+8.561280000000000e+02 +3.185000000000000e+02
+6.532980000000000e+02 +1.290000000000000e+02
+2.646640000000000e+02 +3.180000000000000e+02
+6.187970000000000e+02 +4.700000000000000e+01
+7.886830000000000e+02 +1.775000000000000e+02
+3.258130000000000e+02 +6.650000000000000e+01
+1.461640000000000e+03 +3.180000000000000e+02
+1.063990000000000e+03 +1.455000000000000e+02
+2.689130000000000e+01 +3.175000000000000e+02
+8.885470000000000e+02 +2.905000000000000e+02
+1.079990000000000e+03 +1.445000000000000e+02
+6.356519999999998e+02 +1.260000000000000e+02
+3.064530000000000e+02 +3.165000000000000e+02
+4.834480000000000e+02 +2.940000000000000e+02
+1.063130000000000e+03 +1.425000000000000e+02
+6.186360000000000e+02 +1.215000000000000e+02
+5.707730000000000e+01 +3.165000000000000e+02
+2.307310000000000e+02 +3.160000000000000e+02
+9.784890000000000e+02 +2.230000000000000e+02
+2.768550000000000e+02 +3.155000000000000e+02
+6.254870000000000e+02 +4.650000000000000e+01
+1.387380000000000e+03 +2.485000000000000e+02
+2.425850000000000e+02 +3.150000000000000e+02
+7.806900000000001e+02 +1.715000000000000e+02
+3.254290000000001e+02 +7.000000000000000e+01
+8.928000000000000e+02 +1.055000000000000e+02
+6.140500000000000e+02 +4.300000000000000e+01
+1.619900000000000e+03 +2.995000000000000e+02
+5.007760000000000e+02 +7.800000000000000e+01
+1.093310000000000e+03 +1.890000000000000e+02
+2.757980000000000e+02 +3.500000000000000e+01
+9.595820000000000e+02 +1.645000000000000e+02
+1.682550000000000e+03 +3.140000000000000e+02
+2.133230000000000e+02 +3.135000000000000e+02
+2.438670000000000e+02 +3.135000000000000e+02
+2.447210000000000e+02 +3.135000000000000e+02
+7.866510000000002e+02 +1.715000000000000e+02
+9.250810000000000e+02 +3.135000000000000e+02
+1.273830000000000e+03 +3.130000000000000e+02
+9.986240000000000e+02 +2.675000000000000e+02
+2.966200000000000e+02 +3.120000000000000e+02
+3.148140000000000e+02 +3.120000000000000e+02
+2.513500000000000e+02 +3.120000000000000e+02
+3.815930000000000e+02 +3.120000000000000e+02
+1.715650000000000e+02 +3.120000000000000e+02
+2.494350000000000e+02 +3.115000000000000e+02
+1.722690000000000e+03 +2.555000000000000e+02
+2.767510000000000e+02 +3.850000000000000e+01
+1.839520000000000e+01 +3.110000000000000e+02
+5.066700000000000e+02 +2.910000000000000e+02
+1.903870000000000e+02 +3.105000000000000e+02
+2.423520000000000e+02 +3.105000000000000e+02
+3.369090000000000e+02 +6.900000000000000e+01
+4.264680000000000e+02 +3.100000000000000e+02
+2.006910000000000e+02 +3.100000000000000e+02
+2.533020000000000e+02 +3.100000000000000e+02
+7.899299999999999e+02 +1.795000000000000e+02
+6.770450000000000e+02 +3.100000000000000e+02
+1.058340000000000e+03 +3.100000000000000e+02
+6.183470000000000e+02 +1.470000000000000e+02
+6.348170000000000e+02 +3.085000000000000e+02
+1.848720000000000e+02 +3.085000000000000e+02
+2.506600000000000e+02 +3.085000000000000e+02
+9.276480000000000e+02 +1.270000000000000e+02
+7.156780000000000e+02 +3.085000000000000e+02
+9.411990000000000e+02 +1.160000000000000e+02
+1.207200000000000e+03 +2.475000000000000e+02
+1.065640000000000e+03 +2.435000000000000e+02
+2.212900000000000e+02 +3.075000000000000e+02
+9.798970000000000e+02 +1.630000000000000e+02
+2.414380000000000e+02 +3.070000000000000e+02
+2.285220000000000e+02 +3.070000000000000e+02
+7.869900000000000e+02 +1.655000000000000e+02
+9.951980000000000e+02 +1.645000000000000e+02
+4.143570000000000e+01 +3.065000000000000e+02
+6.653460000000000e+02 +1.470000000000000e+02
+1.023520000000000e+03 +1.040000000000000e+02
+4.838170000000000e+02 +3.060000000000000e+02
+6.588750000000000e+02 +1.920000000000000e+02
+1.080260000000000e+03 +1.445000000000000e+02
+2.210210000000000e+02 +3.055000000000000e+02
+9.940900000000000e+02 +1.720000000000000e+02
+8.755690000000000e+02 +1.210000000000000e+02
+1.660730000000000e+03 +2.310000000000000e+02
+6.356780000000000e+02 +1.180000000000000e+02
+7.953090000000000e+02 +1.640000000000000e+02
+1.000830000000000e+03 +1.615000000000000e+02
+7.014160000000001e+02 +9.400000000000000e+01
+7.878550000000000e+02 +1.815000000000000e+02
+1.743430000000000e+02 +3.040000000000000e+02
+1.272890000000000e+02 +3.040000000000000e+02
+7.971230000000000e+02 +1.880000000000000e+02
+2.768370000000000e+02 +2.800000000000000e+01
+6.457030000000000e+02 +1.100000000000000e+02
+1.291250000000000e+02 +3.035000000000000e+02
+1.300150000000000e+03 +1.710000000000000e+02
+7.996130000000001e+02 +2.355000000000000e+02
+1.473150000000000e+02 +3.030000000000000e+02
+1.955780000000000e+02 +3.030000000000000e+02
+9.949770000000000e+02 +1.515000000000000e+02
+8.148230000000000e+02 +2.370000000000000e+02
+1.084610000000000e+03 +1.320000000000000e+02
+1.341430000000000e+01 +3.025000000000000e+02
+1.000630000000000e+03 +1.690000000000000e+02
+1.723440000000001e+02 +3.020000000000000e+02
+2.097180000000000e+02 +3.020000000000000e+02
+8.002550000000000e+02 +1.875000000000000e+02
+1.123640000000000e+03 +3.020000000000000e+02
+3.173810000000000e+02 +3.015000000000000e+02
+9.193810000000000e+02 +1.080000000000000e+02
+1.486170000000000e+02 +3.010000000000000e+02
+9.689070000000000e+02 +1.600000000000000e+02
+7.876650000000000e+02 +1.835000000000000e+02
+6.486270000000000e+02 +9.900000000000000e+01
+6.249370000000000e+02 +1.395000000000000e+02
+8.002460000000002e+02 +1.715000000000000e+02
+9.027700000000000e+02 +1.095000000000000e+02
+1.238700000000000e+03 +3.000000000000000e+02
+3.186260000000000e+02 +3.000000000000000e+02
+1.162720000000000e+03 +3.000000000000000e+02
+6.791300000000000e+02 +1.030000000000000e+02
+7.020169999999998e+02 +9.500000000000000e+01
+1.400650000000000e+03 +3.000000000000000e+02
+3.161850000000000e+02 +3.000000000000000e+02
+1.160790000000000e+03 +2.065000000000000e+02
+2.522670000000000e+02 +2.995000000000000e+02
+6.619060000000002e+02 +1.145000000000000e+02
+1.124640000000000e+03 +2.995000000000000e+02
+2.264690000000000e+02 +2.990000000000000e+02
+9.182809999999999e+02 +1.730000000000000e+02
+2.949970000000000e+02 +2.985000000000000e+02
+7.052810000000002e+02 +2.330000000000000e+02
+2.408200000000000e+02 +2.980000000000000e+02
+2.852690000000000e+02 +1.430000000000000e+02
+6.522460000000000e+02 +1.120000000000000e+02
+1.572820000000000e+03 +2.295000000000000e+02
+1.625850000000000e+02 +2.980000000000000e+02
+1.081860000000000e+03 +2.350000000000000e+02
+6.595820000000000e+02 +1.015000000000000e+02
+1.981380000000000e+02 +2.970000000000000e+02
+9.618390000000001e+02 +1.515000000000000e+02
+8.913639999999998e+02 +1.080000000000000e+02
+1.101230000000000e+03 +2.970000000000000e+02
+9.097970000000000e+00 +2.965000000000000e+02
+1.013380000000000e+03 +2.480000000000000e+02
+8.961700000000000e+02 +1.085000000000000e+02
+2.483690000000000e+02 +2.955000000000000e+02
+6.661849999999999e+02 +1.120000000000000e+02
+2.251850000000000e+02 +2.955000000000000e+02
+2.171610000000000e+02 +2.955000000000000e+02
+1.305260000000000e+03 +1.740000000000000e+02
+7.449260000000000e+02 +2.950000000000000e+02
+1.417220000000000e+02 +2.950000000000000e+02
+1.966590000000000e+02 +2.945000000000000e+02
+1.566830000000000e+03 +2.250000000000000e+02
+9.100520000000000e+02 +9.700000000000000e+01
+1.602590000000000e+03 +2.675000000000000e+02
+3.647230000000000e+02 +2.940000000000000e+02
+1.516170000000000e+02 +2.940000000000000e+02
+6.711860000000000e+02 +2.940000000000000e+02
+1.247590000000000e+03 +2.940000000000000e+02
+1.000670000000000e+03 +2.940000000000000e+02
+2.528300000000000e+02 +2.935000000000000e+02
+1.670600000000000e+03 +2.220000000000000e+02
+9.805750000000000e+02 +1.650000000000000e+02
+1.045150000000000e+03 +1.745000000000000e+02
+8.986260000000002e+02 +9.600000000000000e+01
+1.005220000000000e+03 +2.425000000000000e+02
+7.960610000000000e+02 +1.800000000000000e+02
+3.502300000000000e+02 +2.930000000000000e+02
+8.071650000000000e+02 +1.175000000000000e+02
+1.023990000000000e+02 +2.925000000000000e+02
+5.055100000000000e+02 +2.920000000000000e+02
+2.177910000000000e+02 +2.920000000000000e+02
+1.084480000000000e+03 +1.705000000000000e+02
+9.069410000000000e+02 +1.065000000000000e+02
+1.583300000000000e+00 +2.920000000000000e+02
+1.038530000000000e+01 +2.915000000000000e+02
+1.394910000000000e+03 +1.390000000000000e+02
+2.361460000000000e+02 +2.915000000000000e+02
+1.535580000000000e+03 +2.915000000000000e+02
+2.386020000000000e+02 +2.910000000000000e+02
+1.081810000000000e+03 +2.275000000000000e+02
+2.788420000000000e+02 +2.150000000000000e+01
+6.444460000000000e+02 +1.080000000000000e+02
+6.631450000000000e+02 +2.895000000000000e+02
+5.639119999999998e+02 +2.895000000000000e+02
+6.179640000000001e+02 +9.600000000000000e+01
+6.228620000000000e+02 +1.300000000000000e+02
+7.902220000000000e+02 +1.600000000000000e+02
+9.764630000000000e+02 +2.890000000000000e+02
+1.370500000000000e+03 +2.515000000000000e+02
+6.390010000000000e+02 +9.700000000000000e+01
+2.332070000000000e+02 +2.885000000000000e+02
+1.299970000000000e+03 +1.665000000000000e+02
+4.698890000000000e+02 +2.885000000000000e+02
+2.059900000000000e+02 +2.880000000000000e+02
+7.803880000000000e+02 +1.570000000000000e+02
+2.980660000000000e+02 +2.880000000000000e+02
+6.545150000000000e+02 +1.175000000000000e+02
+6.316669999999998e+02 +1.170000000000000e+02
+1.480850000000000e+03 +2.755000000000000e+02
+1.058740000000000e+03 +2.270000000000000e+02
+9.825950000000000e+02 +1.470000000000000e+02
+1.147650000000000e+03 +2.865000000000000e+02
+8.996010000000001e+02 +1.025000000000000e+02
+8.983520000000000e+02 +1.940000000000000e+02
+2.815560000000000e+02 +2.050000000000000e+01
+6.524019999999998e+02 +1.065000000000000e+02
+1.979230000000000e+02 +2.860000000000000e+02
+3.443830000000001e+02 +5.100000000000000e+01
+4.762000000000000e+02 +2.855000000000000e+02
+4.932410000000000e+02 +4.250000000000000e+01
+6.362809999999999e+02 +9.600000000000000e+01
+1.190330000000000e+02 +2.855000000000000e+02
+7.473310000000000e+02 +1.150000000000000e+02
+1.997060000000000e+02 +2.850000000000000e+02
+1.618810000000000e+02 +2.850000000000000e+02
+6.580580000000000e+02 +9.350000000000000e+01
+9.945720000000000e+02 +1.595000000000000e+02
+8.944420000000000e+02 +1.945000000000000e+02
+1.760390000000000e+01 +2.845000000000000e+02
+7.653589999999998e+02 +2.845000000000000e+02
+8.744280000000000e+02 +1.555000000000000e+02
+6.685839999999999e+02 +9.700000000000000e+01
+8.060150000000000e+01 +2.840000000000000e+02
+7.897970000000000e+02 +1.560000000000000e+02
+9.102320000000000e+02 +1.025000000000000e+02
+1.208600000000000e+02 +2.835000000000000e+02
+1.083640000000000e+03 +1.600000000000000e+02
+3.994490000000000e+02 +2.830000000000000e+02
+1.199770000000000e+03 +2.620000000000000e+02
+9.376960000000000e+02 +1.605000000000000e+02
+2.224680000000000e+02 +2.825000000000000e+02
+7.964830000000002e+02 +1.565000000000000e+02
+8.507480000000000e+02 +1.025000000000000e+02
+1.886900000000000e-01 +2.820000000000000e+02
+1.073910000000000e+03 +2.245000000000000e+02
+6.562089999999999e+02 +1.040000000000000e+02
+5.948010000000000e+01 +2.815000000000000e+02
+7.639220000000000e+02 +1.455000000000000e+02
+9.894400000000001e+02 +1.515000000000000e+02
+8.872930000000000e+02 +1.030000000000000e+02
+6.101020000000000e+02 +1.005000000000000e+02
+1.608430000000000e+02 +2.810000000000000e+02
+7.522750000000000e+02 +1.360000000000000e+02
+1.301850000000000e+03 +2.810000000000000e+02
+2.072070000000000e+02 +2.805000000000000e+02
+1.582190000000000e+02 +2.800000000000000e+02
+9.996910000000000e+02 +2.325000000000000e+02
+1.256940000000000e+02 +2.795000000000000e+02
+4.230780000000000e+02 +2.790000000000000e+02
+1.075900000000000e+03 +2.235000000000000e+02
+7.692400000000000e+02 +1.445000000000000e+02
+1.182340000000000e+03 +2.790000000000000e+02
+1.642090000000000e+03 +2.395000000000000e+02
+1.055700000000000e+03 +2.145000000000000e+02
+1.071790000000000e+03 +1.640000000000000e+02
+8.894680000000002e+02 +8.300000000000000e+01
+8.732370000000000e+02 +2.785000000000000e+02
+2.343650000000000e+02 +2.780000000000000e+02
+6.553180000000000e+02 +9.550000000000000e+01
+1.239090000000000e+02 +2.780000000000000e+02
+9.917270000000000e+02 +1.455000000000000e+02
+7.588310000000000e+02 +1.425000000000000e+02
+1.077030000000000e+03 +2.365000000000000e+02
+6.567760000000002e+02 +8.800000000000000e+01
+1.923770000000000e+02 +2.775000000000000e+02
+2.079550000000000e+02 +2.770000000000000e+02
+1.145540000000000e+03 +2.770000000000000e+02
+1.511430000000000e+02 +2.765000000000000e+02
+2.867510000000000e+02 +2.765000000000000e+02
+1.768110000000000e+02 +2.760000000000000e+02
+1.125520000000000e+03 +1.580000000000000e+02
+2.355230000000000e+02 +2.755000000000000e+02
+5.155409999999998e+02 +2.755000000000000e+02
+1.318000000000000e+03 +1.565000000000000e+02
+8.917439999999998e+02 +9.050000000000000e+01
+3.910080000000000e+02 +2.750000000000000e+02
+8.368360000000000e+02 +2.750000000000000e+02
+8.181569999999998e+02 +2.750000000000000e+02
+9.056430000000000e+02 +1.855000000000000e+02
+1.385760000000000e+03 +2.340000000000000e+02
+1.311490000000000e+03 +1.575000000000000e+02
+9.005160000000002e+02 +8.600000000000000e+01
+5.474850000000000e+02 +2.740000000000000e+02
+1.066730000000000e+03 +2.175000000000000e+02
+2.000990000000000e+02 +2.740000000000000e+02
+8.357489999999998e+02 +2.740000000000000e+02
+1.605260000000000e+02 +2.735000000000000e+02
+1.154680000000000e+03 +2.730000000000000e+02
+9.432410000000000e+02 +2.725000000000000e+02
+1.972060000000000e+02 +2.720000000000000e+02
+2.852910000000000e+02 +2.715000000000000e+02
+1.452430000000000e+03 +2.715000000000000e+02
+1.752010000000000e+02 +2.715000000000000e+02
+1.577370000000000e+03 +2.030000000000000e+02
+1.019640000000000e+03 +1.510000000000000e+02
+2.261470000000000e+02 +2.710000000000000e+02
+9.959160000000001e+02 +2.710000000000000e+02
+2.581460000000000e+02 +2.705000000000000e+02
+9.825830000000000e+02 +2.700000000000000e+02
+5.005780000000000e+02 +2.695000000000000e+02
+7.430839999999999e+02 +1.495000000000000e+02
+6.400670000000000e+02 +9.350000000000000e+01
+9.894180000000000e+02 +1.445000000000000e+02
+2.166150000000000e+01 +2.690000000000000e+02
+6.542809999999999e+02 +2.690000000000000e+02
+1.111390000000000e+03 +2.690000000000000e+02
+1.800660000000000e+02 +2.690000000000000e+02
+7.570939999999998e+02 +1.295000000000000e+02
+1.704170000000000e+02 +2.685000000000000e+02
+7.437330000000002e+02 +1.245000000000000e+02
+7.355249999999999e+01 +2.680000000000000e+02
+9.044730000000000e+02 +1.820000000000000e+02
+2.347520000000000e+02 +2.675000000000000e+02
+4.666040000000000e+02 +2.675000000000000e+02
+1.370560000000000e+03 +2.300000000000000e+02
+2.168380000000000e+02 +2.675000000000000e+02
+1.010020000000000e+03 +2.250000000000000e+02
+1.191700000000000e+02 +2.670000000000000e+02
+1.143360000000000e+03 +2.670000000000000e+02
+8.667990000000000e+02 +2.670000000000000e+02
+1.095260000000000e+01 +2.665000000000000e+02
+1.004320000000000e+03 +1.275000000000000e+02
+8.756840000000000e+02 +1.460000000000000e+02
+1.945610000000000e+01 +2.660000000000000e+02
+1.297770000000000e+03 +1.505000000000000e+02
+9.069270000000000e+02 +8.900000000000000e+01
+7.475810000000000e+02 +1.295000000000000e+02
+7.626550000000000e+02 +1.295000000000000e+02
+1.369400000000000e+03 +2.350000000000000e+02
+6.631150000000000e+02 +1.100000000000000e+02
+1.087950000000000e+03 +2.100000000000000e+02
+7.576500000000000e+02 +1.000000000000000e+02
+8.854800000000000e+02 +1.110000000000000e+02
+7.905130000000000e+02 +2.645000000000000e+02
+1.066590000000000e+03 +2.125000000000000e+02
+7.439460000000000e+02 +1.465000000000000e+02
+6.491130000000001e+02 +7.950000000000000e+01
+1.285210000000000e+02 +2.645000000000000e+02
+1.151820000000000e+03 +2.645000000000000e+02
+6.856660000000001e+02 +2.645000000000000e+02
+4.269690000000000e+02 +2.640000000000000e+02
+6.480090000000000e+02 +7.650000000000000e+01
+1.616230000000000e+02 +2.640000000000000e+02
+7.972719999999998e+02 +2.640000000000000e+02
+1.494850000000000e+01 +2.635000000000000e+02
+1.794170000000000e+02 +2.635000000000000e+02
+6.470060000000000e+02 +2.635000000000000e+02
+1.066970000000000e+03 +1.460000000000000e+02
+1.592300000000000e+03 +2.450000000000000e+02
+5.267830000000000e+02 +2.630000000000000e+02
+6.132010000000000e+02 +7.900000000000000e+01
+7.532980000000000e+02 +2.625000000000000e+02
+4.543700000000000e+02 +6.850000000000000e+01
+7.554150000000000e+02 +1.230000000000000e+02
+2.818490000000000e+02 +9.500000000000000e+01
+8.799200000000000e+02 +8.150000000000000e+01
+2.845230000000000e+02 +2.610000000000000e+02
+3.142330000000000e+02 +2.610000000000000e+02
+7.227290000000001e+01 +2.610000000000000e+02
+8.635350000000000e+02 +2.610000000000000e+02
+5.684720000000000e-01 +2.605000000000000e+02
+5.101120000000000e+02 +2.600000000000000e+02
+2.291570000000000e+02 +2.600000000000000e+02
+1.104240000000000e+03 +1.445000000000000e+02
+7.687960000000000e+02 +1.305000000000000e+02
+7.748960000000002e+02 +1.380000000000000e+02
+3.730000000000000e+02 +2.595000000000000e+02
+1.093590000000000e+03 +2.595000000000000e+02
+8.835620000000000e+02 +1.720000000000000e+02
+1.278280000000000e+02 +2.590000000000000e+02
+5.403770000000000e+01 +2.590000000000000e+02
+2.773960000000000e+02 +2.590000000000000e+02
+1.194680000000000e+01 +2.585000000000000e+02
+1.293410000000000e+03 +1.385000000000000e+02
+8.569410000000000e+02 +1.405000000000000e+02
+1.726710000000000e+02 +2.580000000000000e+02
+6.265950000000000e+02 +1.005000000000000e+02
+1.109900000000000e+03 +1.415000000000000e+02
+3.025180000000000e+02 +2.580000000000000e+02
+7.533400000000000e+02 +1.235000000000000e+02
+2.185750000000000e+02 +2.575000000000000e+02
+1.658750000000000e+02 +2.575000000000000e+02
+1.579030000000000e+03 +1.930000000000000e+02
+6.387530000000000e+02 +7.850000000000000e+01
+9.010920000000000e+02 +8.250000000000000e+01
+7.750160000000002e+02 +2.570000000000000e+02
+9.737920000000000e+02 +1.315000000000000e+02
+8.727230000000002e+02 +2.565000000000000e+02
+2.746160000000000e+02 +2.565000000000000e+02
+1.312690000000000e+03 +1.430000000000000e+02
+9.097809999999999e+02 +8.200000000000000e+01
+1.330280000000000e+01 +2.555000000000000e+02
+3.568450000000000e+01 +2.550000000000000e+02
+6.592089999999999e+02 +9.700000000000000e+01
+8.480720000000000e+02 +7.600000000000000e+01
+7.488470000000000e+02 +1.170000000000000e+02
+3.743240000000000e+02 +2.545000000000000e+02
+7.482700000000000e+02 +1.385000000000000e+02
+1.799220000000000e+02 +2.545000000000000e+02
+1.311100000000000e+03 +1.430000000000000e+02
+1.068340000000000e+03 +2.545000000000000e+02
+4.727840000000000e+02 +2.545000000000000e+02
+2.124140000000000e+01 +2.540000000000000e+02
+3.209800000000000e+01 +2.535000000000000e+02
+6.486050000000000e+02 +1.605000000000000e+02
+7.483969999999998e+02 +1.210000000000000e+02
+7.443720000000000e+02 +1.150000000000000e+02
+6.631840000000000e+02 +1.595000000000000e+02
+5.335000000000000e+01 +2.530000000000000e+02
+1.612480000000000e+00 +2.530000000000000e+02
+2.276850000000000e+02 +2.525000000000000e+02
+2.784860000000000e+02 +8.700000000000000e+01
+7.845860000000000e+02 +2.525000000000000e+02
+7.753989999999999e+02 +9.250000000000000e+01
+7.997900000000000e+02 +1.405000000000000e+02
+1.785410000000000e+02 +2.520000000000000e+02
+1.124460000000000e+03 +2.520000000000000e+02
+1.781550000000000e+02 +2.520000000000000e+02
+1.305820000000000e+03 +1.390000000000000e+02
+1.100510000000000e+03 +1.395000000000000e+02
+6.907289999999998e+02 +1.740000000000000e+02
+1.101210000000000e+03 +2.510000000000000e+02
+6.493450000000000e+02 +1.660000000000000e+02
+7.519410000000000e+02 +2.505000000000000e+02
+8.532520000000000e+00 +2.500000000000000e+02
+7.583510000000001e+02 +1.230000000000000e+02
+1.121480000000000e+03 +2.500000000000000e+02
+2.289880000000000e+02 +2.495000000000000e+02
+1.223830000000000e+03 +1.970000000000000e+02
+2.699860000000000e+02 +2.490000000000000e+02
+8.875580000000000e+02 +2.490000000000000e+02
+1.506740000000000e+01 +2.485000000000000e+02
+7.456940000000000e+02 +1.155000000000000e+02
+4.032410000000000e+02 +2.480000000000000e+02
+1.068140000000000e+03 +1.875000000000000e+02
+1.066920000000000e+03 +1.850000000000000e+02
+9.331120000000000e+00 +2.475000000000000e+02
+9.737010000000000e+02 +1.270000000000000e+02
+7.657480000000000e+02 +1.150000000000000e+02
+6.169490000000002e+02 +2.470000000000000e+02
+1.586770000000000e+03 +2.250000000000000e+02
+2.204020000000000e+02 +2.465000000000000e+02
+2.814890000000001e+02 +2.095000000000000e+02
+5.223540000000000e+01 +2.465000000000000e+02
+7.460180000000000e+02 +1.200000000000000e+02
+1.581460000000000e+02 +2.460000000000000e+02
+1.328190000000000e+03 +2.460000000000000e+02
+6.351020000000000e+02 +2.455000000000000e+02
+6.351830000000000e+02 +1.515000000000000e+02
+1.525620000000000e+02 +2.450000000000000e+02
+3.482480000000001e+02 +2.550000000000000e+01
+1.651610000000000e+02 +2.445000000000000e+02
+7.619710000000000e+02 +1.260000000000000e+02
+1.150910000000000e+03 +2.445000000000000e+02
+1.595700000000000e+02 +2.435000000000000e+02
+1.361030000000000e+02 +2.430000000000000e+02
+9.752190000000001e+02 +1.330000000000000e+02
+9.010560000000000e+02 +7.500000000000000e+01
+2.133510000000000e+02 +2.420000000000000e+02
+3.841950000000000e+02 +2.420000000000000e+02
+6.394320000000000e+02 +1.510000000000000e+02
+1.339000000000000e+00 +2.420000000000000e+02
+1.605940000000000e+02 +2.420000000000000e+02
+2.221790000000000e+02 +2.410000000000000e+02
+1.064010000000000e+03 +1.885000000000000e+02
+3.174750000000000e+01 +2.410000000000000e+02
+1.082230000000000e+03 +1.270000000000000e+02
+1.065570000000000e+02 +2.405000000000000e+02
+3.624300000000000e+01 +2.405000000000000e+02
+1.120290000000000e+03 +1.255000000000000e+02
+1.584220000000000e+03 +2.155000000000000e+02
+7.498860000000002e+02 +1.195000000000000e+02
+9.028270000000000e+02 +1.125000000000000e+02
+2.621170000000000e+02 +2.395000000000000e+02
+1.356510000000000e+01 +2.395000000000000e+02
+2.813010000000000e+02 +1.885000000000000e+02
+5.375570000000000e-01 +2.395000000000000e+02
+7.961840000000000e+02 +1.160000000000000e+02
+1.298950000000000e+03 +1.320000000000000e+02
+8.708080000000000e+02 +6.800000000000000e+01
+8.132139999999998e+02 +2.395000000000000e+02
+2.351580000000000e+02 +2.390000000000000e+02
+6.744750000000000e+02 +1.145000000000000e+02
+1.798820000000000e+01 +2.390000000000000e+02
+7.506100000000000e+02 +1.085000000000000e+02
+9.910970000000000e+02 +1.880000000000000e+02
+4.269020000000000e+01 +2.385000000000000e+02
+5.374260000000001e-01 +2.385000000000000e+02
+9.233560000000000e+02 +2.385000000000000e+02
+3.128570000000000e+02 +2.350000000000000e+01
+9.446510000000000e+02 +2.370000000000000e+02
+9.051520000000000e+02 +1.135000000000000e+02
+1.381140000000000e+03 +2.370000000000000e+02
+1.406480000000000e+02 +2.365000000000000e+02
+7.940930000000002e+02 +1.265000000000000e+02
+6.348360000000000e+02 +2.360000000000000e+02
+6.536460000000000e+02 +2.360000000000000e+02
+7.900089999999999e+02 +1.130000000000000e+02
+9.801840000000000e+02 +1.165000000000000e+02
+8.598160000000000e+02 +7.050000000000000e+01
+1.081370000000000e+03 +1.845000000000000e+02
+6.709630000000002e+00 +2.350000000000000e+02
+1.117790000000000e+03 +1.270000000000000e+02
+2.539370000000000e+00 +2.345000000000000e+02
+1.507270000000000e+02 +2.345000000000000e+02
+9.776630000000000e+02 +1.805000000000000e+02
+2.507720000000000e+02 +2.345000000000000e+02
+2.391250000000000e+02 +2.340000000000000e+02
+2.418410000000000e+02 +2.340000000000000e+02
+1.093540000000000e+02 +2.340000000000000e+02
+9.214340000000000e+02 +2.340000000000000e+02
+4.520890000000000e+02 +4.200000000000000e+01
+1.603190000000000e+02 +2.335000000000000e+02
+1.192820000000000e+02 +2.330000000000000e+02
+6.185240000000000e+02 +7.350000000000000e+01
+6.254950000000000e+02 +7.550000000000000e+01
+8.985780000000000e+02 +1.195000000000000e+02
+2.059570000000000e+02 +2.320000000000000e+02
+1.253950000000000e+01 +2.320000000000000e+02
+8.971319999999999e+02 +2.320000000000000e+02
+4.716360000000000e+02 +6.900000000000000e+01
+1.333850000000000e+02 +2.315000000000000e+02
+7.916870000000000e+02 +1.110000000000000e+02
+1.304520000000000e+03 +2.200000000000000e+02
+1.003360000000000e+00 +2.310000000000000e+02
+6.353819999999999e+02 +1.390000000000000e+02
+2.007500000000000e+00 +2.305000000000000e+02
+1.402510000000000e+02 +2.305000000000000e+02
+7.936530000000000e+02 +1.255000000000000e+02
+9.574740000000000e+02 +2.305000000000000e+02
+2.221860000000000e+02 +2.300000000000000e+02
+7.690460000000000e+02 +2.300000000000000e+02
+9.785080000000000e+02 +1.830000000000000e+02
+6.547840000000000e+02 +2.300000000000000e+02
+1.689330000000000e+02 +2.300000000000000e+02
+1.648880000000000e+02 +2.295000000000000e+02
+4.447010000000000e+00 +2.295000000000000e+02
+9.957380000000001e+02 +1.260000000000000e+02
+1.386770000000000e+02 +2.295000000000000e+02
+1.181680000000000e+03 +2.295000000000000e+02
+2.128470000000000e+02 +2.290000000000000e+02
+6.222240000000000e+00 +2.290000000000000e+02
+4.408420000000000e+02 +2.290000000000000e+02
+8.901050000000000e+02 +1.435000000000000e+02
+1.781340000000000e+02 +2.285000000000000e+02
+2.204830000000000e+02 +2.285000000000000e+02
+5.890430000000000e+00 +2.285000000000000e+02
+2.245940000000000e+01 +2.285000000000000e+02
+9.773479999999999e+00 +2.280000000000000e+02
+1.336560000000000e+02 +2.280000000000000e+02
+8.692150000000000e+02 +1.190000000000000e+02
+1.308470000000000e+03 +2.200000000000000e+02
+1.068100000000000e+03 +1.155000000000000e+02
+8.528850000000000e+02 +2.275000000000000e+02
+1.947170000000000e+02 +2.270000000000000e+02
+5.152290000000000e+00 +2.270000000000000e+02
+3.506290000000000e+00 +2.270000000000000e+02
+1.344240000000000e+02 +2.270000000000000e+02
+8.739460000000000e+02 +9.950000000000000e+01
+7.444720000000000e+02 +1.115000000000000e+02
+1.329490000000000e+02 +2.265000000000000e+02
+6.198340000000002e+02 +7.200000000000000e+01
+7.914160000000001e+02 +1.130000000000000e+02
+9.621860000000000e+02 +1.765000000000000e+02
+4.028620000000000e-02 +2.260000000000000e+02
+1.151870000000000e+02 +2.260000000000000e+02
+7.819400000000001e+02 +1.080000000000000e+02
+5.942859999999999e+02 +2.255000000000000e+02
+7.101799999999999e+02 +2.250000000000000e+02
+6.367690000000000e+02 +1.360000000000000e+02
+1.218110000000000e+03 +1.725000000000000e+02
+1.068440000000000e+03 +1.785000000000000e+02
+3.382820000000000e+02 +7.600000000000000e+01
+1.883600000000000e+02 +2.240000000000000e+02
+2.146890000000000e+00 +2.240000000000000e+02
+6.564889999999998e+02 +1.325000000000000e+02
+2.082120000000000e+02 +2.235000000000000e+02
+1.293380000000000e+03 +2.235000000000000e+02
+1.949470000000000e+02 +2.230000000000000e+02
+1.758420000000000e+02 +2.230000000000000e+02
+1.210950000000000e+02 +2.230000000000000e+02
+7.869340000000000e+02 +1.085000000000000e+02
+9.058010000000000e+02 +1.105000000000000e+02
+7.415160000000002e+02 +1.100000000000000e+02
+1.192170000000000e+02 +2.220000000000000e+02
+8.917910000000001e+02 +1.415000000000000e+02
+2.869520000000000e+02 +6.150000000000000e+01
+6.538520000000000e+02 +1.345000000000000e+02
+9.809070000000000e+01 +2.215000000000000e+02
+1.156440000000000e+02 +2.210000000000000e+02
+1.464060000000000e+03 +2.210000000000000e+02
+1.660750000000000e+02 +2.205000000000000e+02
+2.644780000000000e+02 +5.850000000000000e+01
+6.614820000000000e+02 +1.315000000000000e+02
+3.726520000000000e-01 +2.205000000000000e+02
+1.757470000000000e+02 +2.200000000000000e+02
+3.817300000000000e+00 +2.195000000000000e+02
+1.212950000000000e+02 +2.195000000000000e+02
+4.802330000000000e-01 +2.190000000000000e+02
+6.642790000000000e+02 +1.335000000000000e+02
+7.800039999999998e+02 +1.115000000000000e+02
+1.278010000000000e+03 +2.185000000000000e+02
+8.992210000000000e+02 +1.090000000000000e+02
+5.788160000000000e+02 +2.180000000000000e+02
+4.795870000000000e+02 +6.100000000000000e+01
+7.746520000000000e-01 +2.175000000000000e+02
+8.646169999999999e+01 +2.175000000000000e+02
+2.299450000000000e-01 +2.175000000000000e+02
+8.534200000000000e+02 +2.170000000000000e+02
+7.421660000000001e+02 +1.060000000000000e+02
+7.869800000000000e+02 +1.130000000000000e+02
+4.564940000000000e+02 +2.165000000000000e+02
+1.547230000000000e+02 +2.155000000000000e+02
+1.073540000000000e+03 +1.565000000000000e+02
+7.122830000000000e-01 +2.155000000000000e+02
+1.277260000000000e+02 +2.150000000000000e+02
+1.215420000000000e+00 +2.150000000000000e+02
+1.024580000000000e+02 +2.150000000000000e+02
+6.736039999999998e+02 +1.335000000000000e+02
+8.932400000000000e+02 +1.090000000000000e+02
+8.893500000000000e+02 +1.295000000000000e+02
+8.723030000000000e+01 +2.140000000000000e+02
+4.201530000000000e+02 +6.450000000000000e+01
+8.488310000000000e+02 +1.025000000000000e+02
+6.647480000000000e+02 +1.300000000000000e+02
+7.673800000000000e+02 +1.015000000000000e+02
+1.030760000000000e+03 +2.130000000000000e+02
+6.756160000000001e+02 +2.130000000000000e+02
+5.801120000000000e+02 +2.130000000000000e+02
+1.449510000000000e+02 +2.125000000000000e+02
+7.418170000000000e+02 +1.035000000000000e+02
+1.591220000000000e+03 +1.915000000000000e+02
+6.779980000000000e+02 +2.125000000000000e+02
+7.336450000000000e+02 +2.120000000000000e+02
+2.365970000000000e+02 +2.120000000000000e+02
+9.364400000000001e+01 +2.120000000000000e+02
+9.781470000000000e+02 +1.665000000000000e+02
+8.123020000000000e+01 +2.115000000000000e+02
+7.124410000000000e+02 +2.115000000000000e+02
+1.077940000000000e+02 +2.115000000000000e+02
+9.199030000000000e+02 +1.305000000000000e+02
+6.192919999999998e+02 +5.550000000000000e+01
+7.805180000000000e+02 +1.005000000000000e+02
+1.120170000000000e+03 +1.755000000000000e+02
+1.782460000000000e-01 +2.105000000000000e+02
+8.339900000000000e+02 +1.070000000000000e+02
+9.038869999999999e+02 +1.025000000000000e+02
+1.121660000000000e+03 +1.605000000000000e+02
+4.331260000000000e+02 +2.090000000000000e+02
+7.056630000000000e+02 +2.090000000000000e+02
+1.699030000000000e-02 +2.090000000000000e+02
+3.340490000000001e+02 +6.300000000000000e+01
+9.885060000000000e+02 +1.635000000000000e+02
+1.389140000000000e+03 +1.690000000000000e+02
+9.011820000000000e+01 +2.085000000000000e+02
+7.751560000000002e+02 +1.060000000000000e+02
+1.152510000000000e+03 +2.085000000000000e+02
+4.272970000000000e+02 +2.075000000000000e+02
+8.250959999999999e-01 +2.075000000000000e+02
+7.794680000000002e+02 +9.850000000000000e+01
+1.098520000000000e+03 +1.710000000000000e+02
+2.346770000000000e+02 +2.065000000000000e+02
+7.809570000000000e+02 +1.040000000000000e+02
+8.911239999999998e+02 +1.070000000000000e+02
+6.674570000000000e+02 +2.065000000000000e+02
+4.033340000000000e+02 +2.060000000000000e+02
+9.881710000000000e+02 +1.630000000000000e+02
+1.875300000000000e+02 +2.060000000000000e+02
+1.579650000000000e+03 +1.835000000000000e+02
+9.032350000000000e+02 +1.075000000000000e+02
+1.614280000000000e+02 +2.060000000000000e+02
+1.266300000000000e+03 +2.055000000000000e+02
+6.786680000000000e+01 +2.055000000000000e+02
+6.612120000000000e+02 +5.250000000000000e+01
+2.258760000000000e+01 +2.050000000000000e+02
+7.803839999999999e+02 +9.550000000000000e+01
+1.054280000000000e+03 +2.050000000000000e+02
+2.347060000000000e+00 +2.050000000000000e+02
+1.571290000000000e+02 +2.045000000000000e+02
+6.600470000000000e+02 +1.175000000000000e+02
+1.300720000000000e+03 +1.945000000000000e+02
+7.040980000000002e+02 +2.045000000000000e+02
+1.271820000000000e+00 +2.045000000000000e+02
+1.289570000000000e+02 +2.040000000000000e+02
+3.479330000000000e+01 +2.040000000000000e+02
+9.736609999999999e+02 +2.040000000000000e+02
+7.482220000000000e+02 +9.750000000000000e+01
+1.300140000000000e+03 +1.980000000000000e+02
+4.855970000000000e+02 +2.035000000000000e+02
+9.852089999999999e+02 +1.555000000000000e+02
+6.147680000000000e+02 +1.150000000000000e+02
+1.300780000000000e+03 +1.905000000000000e+02
+7.667980000000000e+02 +2.025000000000000e+02
+6.607189999999998e+02 +1.165000000000000e+02
+3.369800000000000e+02 +5.750000000000000e+01
+8.515650000000001e+02 +1.005000000000000e+02
+4.738400000000000e+02 +2.015000000000000e+02
+1.309790000000000e+03 +1.940000000000000e+02
+1.321030000000000e+03 +1.905000000000000e+02
+1.980450000000000e+02 +2.005000000000000e+02
+7.782750000000000e+02 +9.400000000000000e+01
+8.631710000000000e+02 +2.005000000000000e+02
+5.374939999999999e-01 +2.005000000000000e+02
+2.084510000000000e+02 +2.000000000000000e+02
+7.897689999999999e+02 +1.000000000000000e+02
+1.315670000000000e+03 +1.950000000000000e+02
+9.017400000000000e+02 +9.450000000000000e+01
+8.495239999999999e+02 +1.995000000000000e+02
+8.917180000000002e+02 +9.950000000000000e+01
+6.632869999999998e+02 +1.155000000000000e+02
+1.111400000000000e+03 +1.650000000000000e+02
+1.798950000000000e+02 +1.980000000000000e+02
+7.481130000000001e+02 +9.350000000000000e+01
+6.554260000000000e+02 +1.980000000000000e+02
+1.304120000000000e+03 +1.930000000000000e+02
+8.795610000000001e+01 +1.970000000000000e+02
+7.292589999999999e+02 +1.965000000000000e+02
+6.219030000000000e+02 +1.575000000000000e+02
+1.069340000000000e+03 +1.480000000000000e+02
+5.658840000000000e+02 +1.960000000000000e+02
+7.404050000000000e+02 +1.960000000000000e+02
+9.531530000000000e+02 +7.650000000000000e+01
+8.662569999999999e+02 +1.185000000000000e+02
+8.618339999999999e+02 +1.415000000000000e+02
+5.965040000000000e+01 +1.955000000000000e+02
+6.241080000000002e+02 +1.540000000000000e+02
+8.881849999999999e+02 +1.955000000000000e+02
+9.350539999999999e+01 +1.950000000000000e+02
+1.446630000000000e+03 +1.950000000000000e+02
+2.817120000000000e+02 +3.850000000000000e+01
+6.482989999999999e+01 +1.945000000000000e+02
+8.901840000000000e+02 +9.800000000000000e+01
+3.132250000000000e+02 +1.945000000000000e+02
+6.243250000000000e+02 +1.945000000000000e+02
+5.375100000000000e-01 +1.940000000000000e+02
+2.008560000000000e+02 +1.935000000000000e+02
+2.772760000000000e+02 +1.485000000000000e+02
+6.489610000000000e+01 +1.935000000000000e+02
+9.312380000000000e+01 +1.935000000000000e+02
+9.776310000000000e+02 +1.535000000000000e+02
+1.069740000000000e+02 +1.930000000000000e+02
+7.476010000000001e+02 +1.930000000000000e+02
+6.981770000000000e+01 +1.930000000000000e+02
+9.070410000000001e+02 +1.005000000000000e+02
+7.989409999999999e+01 +1.930000000000000e+02
+6.745959999999999e+01 +1.925000000000000e+02
+1.457980000000000e+03 +1.775000000000000e+02
+1.060270000000000e+03 +1.345000000000000e+02
+6.839689999999998e+02 +1.920000000000000e+02
+6.789470000000000e+01 +1.920000000000000e+02
+1.405250000000000e+03 +1.920000000000000e+02
+4.747260000000000e+02 +1.915000000000000e+02
+7.656170000000000e+02 +8.500000000000000e+01
+8.368180000000000e+02 +9.000000000000000e+01
+6.359740000000000e+02 +1.070000000000000e+02
+7.866030000000002e+02 +8.650000000000000e+01
+8.477819999999998e+02 +1.910000000000000e+02
+9.097080000000000e+02 +9.750000000000000e+01
+1.914570000000000e+02 +1.905000000000000e+02
+1.484200000000000e+02 +1.900000000000000e+02
+8.315140000000000e+01 +1.900000000000000e+02
+7.722130000000002e+02 +1.625000000000000e+02
+8.357060000000000e+02 +1.900000000000000e+02
+8.447810000000002e+02 +1.900000000000000e+02
+2.604900000000000e+00 +1.895000000000000e+02
+6.624730000000002e+02 +1.125000000000000e+02
+1.295810000000000e+03 +1.830000000000000e+02
+9.114220000000000e+02 +9.950000000000000e+01
+9.921710000000000e+02 +1.470000000000000e+02
+8.634910000000001e+02 +9.450000000000000e+01
+6.013310000000000e+01 +1.885000000000000e+02
+7.856990000000000e+02 +1.640000000000000e+02
+9.769210000000000e+02 +1.505000000000000e+02
+8.540089999999999e+02 +1.885000000000000e+02
+4.619960000000000e+02 +1.880000000000000e+02
+6.222030000000000e+02 +1.470000000000000e+02
+1.135050000000000e+03 +1.565000000000000e+02
+6.659230000000000e+02 +1.880000000000000e+02
+8.418880000000000e+02 +1.880000000000000e+02
+8.113270000000000e+01 +1.875000000000000e+02
+6.625570000000000e+02 +1.495000000000000e+02
+7.847600000000000e+02 +8.500000000000000e+01
+1.445630000000000e+03 +1.875000000000000e+02
+7.531710000000000e+02 +7.450000000000000e+01
+8.594090000000000e+02 +1.870000000000000e+02
+5.493950000000000e+01 +1.865000000000000e+02
+2.907750000000000e-01 +1.865000000000000e+02
+7.312470000000000e+01 +1.860000000000000e+02
+1.048160000000000e+03 +1.565000000000000e+02
+2.926440000000000e+02 +1.855000000000000e+02
+8.236940000000000e+01 +1.855000000000000e+02
+7.860680000000000e+02 +8.400000000000000e+01
+7.924530000000000e+02 +1.855000000000000e+02
+1.310050000000000e+03 +1.810000000000000e+02
+1.615980000000000e+02 +1.845000000000000e+02
+6.640760000000000e+02 +6.850000000000000e+01
+1.078740000000000e+03 +1.795000000000000e+02
+1.674760000000000e-01 +1.845000000000000e+02
+1.104880000000000e+03 +1.525000000000000e+02
+1.385880000000000e+02 +1.835000000000000e+02
+8.131540000000000e+01 +1.835000000000000e+02
+3.976350000000000e+01 +1.835000000000000e+02
+7.976049999999999e+01 +1.835000000000000e+02
+7.500570000000000e+02 +8.000000000000000e+01
+8.374600000000000e+02 +1.830000000000000e+02
+9.369550000000000e+02 +1.825000000000000e+02
+1.177650000000000e+03 +1.825000000000000e+02
+5.727420000000000e+02 +1.825000000000000e+02
+9.480950000000000e+02 +1.820000000000000e+02
+7.155070000000001e+01 +1.815000000000000e+02
+5.182150000000000e+02 +1.815000000000000e+02
+5.498360000000000e-01 +1.815000000000000e+02
+7.509320000000000e+01 +1.815000000000000e+02
+1.848850000000000e+02 +1.810000000000000e+02
+6.108990000000000e+02 +1.810000000000000e+02
+2.014980000000000e+02 +1.810000000000000e+02
+1.066230000000000e+03 +1.380000000000000e+02
+4.206050000000000e+02 +9.050000000000000e+01
+1.318070000000000e+03 +1.785000000000000e+02
+7.582200000000000e+02 +1.800000000000000e+02
+2.474070000000000e+02 +1.800000000000000e+02
+7.490039999999998e+02 +1.800000000000000e+02
+1.163710000000000e+03 +1.800000000000000e+02
+2.483240000000000e+02 +1.800000000000000e+02
+9.054080000000001e+00 +1.800000000000000e+02
+1.005280000000000e+03 +1.800000000000000e+02
+7.427739999999999e+02 +1.800000000000000e+02
+6.765390000000000e+02 +9.700000000000000e+01
+1.169650000000000e+03 +1.800000000000000e+02
+2.444960000000000e+02 +1.800000000000000e+02
+4.271340000000000e+00 +1.800000000000000e+02
+6.408320000000000e+02 +9.850000000000000e+01
+1.295560000000000e+03 +1.790000000000000e+02
+9.352120000000000e+02 +1.790000000000000e+02
+9.072960000000000e+02 +9.400000000000000e+01
+3.605810000000000e+01 +1.785000000000000e+02
+7.747089999999999e+02 +1.560000000000000e+02
+1.109040000000000e+03 +1.490000000000000e+02
+1.156650000000000e+03 +1.780000000000000e+02
+8.994760000000001e+02 +9.500000000000000e+01
+3.484880000000000e-01 +1.780000000000000e+02
+1.841490000000000e+02 +1.775000000000000e+02
+5.988290000000000e+01 +1.775000000000000e+02
+1.083790000000000e+03 +1.540000000000000e+02
+8.661610000000000e+01 +1.770000000000000e+02
+1.074650000000000e+03 +1.770000000000000e+02
+7.812810000000002e+02 +1.510000000000000e+02
+9.730380000000000e+02 +1.285000000000000e+02
+6.473950000000000e+02 +8.650000000000000e+01
+5.680140000000000e+02 +1.760000000000000e+02
+6.244710000000000e+02 +9.600000000000000e+01
+5.874780000000000e+01 +1.755000000000000e+02
+6.618240000000000e+02 +1.360000000000000e+02
+8.630030000000000e+02 +1.745000000000000e+02
+4.414420000000000e+02 +1.740000000000000e+02
+8.830790000000000e+02 +9.650000000000000e+01
+5.012080000000000e+02 +1.000000000000000e+02
+7.392440000000001e+01 +1.735000000000000e+02
+3.130090000000000e+02 +1.735000000000000e+02
+7.820800000000000e+02 +1.515000000000000e+02
+1.646710000000000e+01 +1.725000000000000e+02
+8.972040000000000e+02 +1.350000000000000e+02
+8.596060000000001e+02 +1.305000000000000e+02
+4.074930000000000e+01 +1.715000000000000e+02
+1.740430000000000e+01 +1.710000000000000e+02
+1.803420000000000e+02 +1.710000000000000e+02
+9.787600000000000e+02 +1.335000000000000e+02
+8.792210000000000e+02 +9.550000000000000e+01
+2.506550000000000e+00 +1.710000000000000e+02
+6.419100000000000e+02 +1.705000000000000e+02
+1.270880000000000e+02 +1.695000000000000e+02
+5.313170000000000e+02 +1.695000000000000e+02
+3.257150000000000e+02 +1.695000000000000e+02
+3.320530000000001e+02 +1.690000000000000e+02
+7.107010000000000e+02 +1.685000000000000e+02
+1.066680000000000e+03 +1.305000000000000e+02
+1.082400000000000e+03 +1.215000000000000e+02
+5.915050000000000e+01 +1.680000000000000e+02
+7.847120000000000e+02 +1.485000000000000e+02
+9.844640000000000e-01 +1.680000000000000e+02
+5.878990000000000e+02 +1.170000000000000e+02
+8.290010000000001e+00 +1.675000000000000e+02
+1.056750000000000e+03 +1.310000000000000e+02
+6.741469999999998e+02 +1.675000000000000e+02
+7.797850000000000e+02 +1.450000000000000e+02
+6.378810000000000e+02 +8.850000000000000e+01
+7.841330000000000e+02 +1.455000000000000e+02
+3.352020000000000e+02 +3.150000000000000e+01
+1.079640000000000e+03 +1.400000000000000e+02
+8.883930000000000e+02 +1.315000000000000e+02
+1.284200000000000e+03 +1.645000000000000e+02
+1.971890000000000e+02 +1.665000000000000e+02
+6.594090000000000e+02 +9.050000000000000e+01
+8.948950000000000e+02 +1.285000000000000e+02
+1.323730000000000e+02 +1.655000000000000e+02
+1.129020000000000e+02 +1.655000000000000e+02
+7.426610000000002e+02 +6.400000000000000e+01
+3.840870000000000e+01 +1.650000000000000e+02
+1.315250000000000e-01 +1.650000000000000e+02
+5.132830000000000e+01 +1.645000000000000e+02
+2.793690000000000e+02 +2.100000000000000e+01
+4.627680000000000e+01 +1.645000000000000e+02
+9.753780000000000e+02 +1.295000000000000e+02
+9.038520000000000e+02 +1.295000000000000e+02
+3.263350000000000e+02 +1.645000000000000e+02
+1.070210000000000e+03 +1.245000000000000e+02
+3.573260000000000e+02 +1.640000000000000e+02
+6.055840000000002e+02 +1.640000000000000e+02
+3.135530000000000e+02 +1.640000000000000e+02
+3.349470000000000e+00 +1.635000000000000e+02
+1.125860000000000e+03 +1.635000000000000e+02
+8.577310000000001e+02 +1.635000000000000e+02
+5.200660000000000e+02 +1.630000000000000e+02
+7.876790000000000e+02 +1.435000000000000e+02
+7.831369999999999e+02 +1.445000000000000e+02
+1.077590000000000e+03 +1.115000000000000e+02
+1.423970000000000e+02 +1.615000000000000e+02
+5.562950000000000e+02 +1.615000000000000e+02
+6.236330000000000e+02 +1.610000000000000e+02
+4.819400000000000e+01 +1.610000000000000e+02
+7.852830000000000e+02 +1.445000000000000e+02
+1.604550000000000e+02 +1.600000000000000e+02
+2.884230000000000e+01 +1.600000000000000e+02
+1.122600000000000e+03 +1.350000000000000e+02
+5.675980000000002e+02 +1.595000000000000e+02
+1.932750000000000e+02 +1.590000000000000e+02
+7.846020000000000e+02 +1.405000000000000e+02
+6.199109999999999e+02 +1.585000000000000e+02
+2.764090000000000e+01 +1.580000000000000e+02
+8.789500000000000e+02 +1.055000000000000e+02
+1.197360000000000e-01 +1.575000000000000e+02
+7.867619999999999e+02 +1.405000000000000e+02
+5.194180000000000e+02 +1.575000000000000e+02
+4.219140000000000e+02 +1.575000000000000e+02
+9.791400000000000e+02 +1.240000000000000e+02
+8.242310000000001e+02 +1.570000000000000e+02
+1.065860000000000e+02 +1.565000000000000e+02
+1.077900000000000e+03 +1.120000000000000e+02
+9.539560000000000e+01 +1.555000000000000e+02
+6.189360000000000e+02 +1.175000000000000e+02
+3.339070000000000e+02 +2.550000000000000e+01
+6.513099999999999e+02 +1.550000000000000e+02
+1.701220000000000e+01 +1.545000000000000e+02
+7.881310000000002e+02 +1.385000000000000e+02
+6.993890000000000e+00 +1.535000000000000e+02
+2.889200000000000e+01 +1.535000000000000e+02
+9.753190000000000e+02 +1.225000000000000e+02
+8.152589999999999e+02 +1.525000000000000e+02
+3.284790000000000e+01 +1.525000000000000e+02
+8.522430000000000e+00 +1.520000000000000e+02
+1.073610000000000e+02 +1.520000000000000e+02
+3.871840000000000e+01 +1.520000000000000e+02
+1.859290000000000e+02 +1.515000000000000e+02
+9.706440000000000e+02 +1.465000000000000e+02
+7.819520000000000e+02 +1.255000000000000e+02
+2.787460000000000e+02 +1.515000000000000e+02
+1.390720000000000e+02 +1.515000000000000e+02
+1.902150000000000e+02 +1.510000000000000e+02
+4.456290000000000e+01 +1.510000000000000e+02
+3.508100000000000e+01 +1.510000000000000e+02
+3.357590000000000e+02 +2.600000000000000e+01
+8.917000000000000e+02 +1.175000000000000e+02
+1.371220000000000e+00 +1.505000000000000e+02
+6.183320000000000e+01 +1.505000000000000e+02
+7.864730000000002e+02 +1.275000000000000e+02
+3.536050000000000e+01 +1.500000000000000e+02
+8.210319999999998e+02 +1.500000000000000e+02
+1.140380000000000e+02 +1.495000000000000e+02
+5.135080000000000e+02 +1.490000000000000e+02
+8.617000000000000e+02 +1.490000000000000e+02
+7.883270000000000e+02 +1.245000000000000e+02
+1.332860000000000e+02 +1.470000000000000e+02
+9.743270000000000e+02 +1.160000000000000e+02
+5.915990000000000e+02 +1.470000000000000e+02
+3.039230000000000e+01 +1.465000000000000e+02
+8.008270000000000e+02 +1.330000000000000e+02
+9.767400000000000e+02 +1.200000000000000e+02
+2.840670000000000e+01 +1.460000000000000e+02
+4.950460000000000e+02 +3.100000000000000e+01
+9.045560000000000e+02 +1.165000000000000e+02
+8.011510000000002e+02 +1.450000000000000e+02
+1.278260000000000e+03 +1.450000000000000e+02
+6.192360000000000e+02 +1.445000000000000e+02
+7.828250000000000e+02 +1.310000000000000e+02
+8.311810000000000e+02 +1.445000000000000e+02
+8.108869999999999e+02 +1.435000000000000e+02
+3.880470000000000e+01 +1.435000000000000e+02
+9.258150000000001e+02 +1.155000000000000e+02
+1.551730000000000e+02 +1.430000000000000e+02
+6.411410000000000e+02 +1.210000000000000e+02
+5.344890000000000e+02 +1.430000000000000e+02
+3.005870000000000e+01 +1.425000000000000e+02
+7.941189999999998e+02 +1.320000000000000e+02
+7.834360000000000e+02 +1.295000000000000e+02
+9.837809999999999e+02 +1.170000000000000e+02
+1.701600000000000e+02 +1.415000000000000e+02
+9.106210000000000e+02 +1.140000000000000e+02
+6.453270000000000e+02 +1.410000000000000e+02
+1.040270000000000e+03 +1.185000000000000e+02
+5.993840000000000e+02 +9.400000000000000e+01
+3.317570000000000e+01 +1.400000000000000e+02
+8.715650000000001e+02 +1.400000000000000e+02
+5.137919999999998e+02 +1.395000000000000e+02
+7.905410000000001e+02 +1.250000000000000e+02
+3.685750000000000e+02 +1.390000000000000e+02
+1.255500000000000e+03 +1.385000000000000e+02
+9.585980000000000e+02 +1.385000000000000e+02
+8.877439999999998e+02 +1.380000000000000e+02
+7.488450000000000e+02 +1.300000000000000e+02
+1.093460000000000e+03 +1.380000000000000e+02
+8.964360000000000e+02 +1.225000000000000e+02
+8.969620000000000e+02 +1.380000000000000e+02
+5.227370000000000e+02 +1.375000000000000e+02
+2.243580000000000e+01 +1.375000000000000e+02
+7.894040000000000e+02 +1.215000000000000e+02
+7.419400000000001e+02 +1.375000000000000e+02
+6.469650000000000e+02 +1.370000000000000e+02
+9.273250000000000e+02 +7.150000000000000e+01
+7.917330000000002e+02 +1.365000000000000e+02
+7.488860000000002e+02 +1.280000000000000e+02
+6.868519999999999e+01 +1.355000000000000e+02
+8.975590000000000e+02 +1.145000000000000e+02
+7.878980000000000e+02 +1.180000000000000e+02
+3.639280000000001e+02 +1.350000000000000e+02
+1.317550000000000e+02 +1.345000000000000e+02
+2.314520000000000e+01 +1.345000000000000e+02
+8.725500000000000e+02 +1.120000000000000e+02
+9.614299999999999e+02 +1.345000000000000e+02
+9.367890000000000e+02 +1.215000000000000e+02
+6.248470000000000e+02 +9.450000000000000e+01
+6.827420000000000e+02 +1.340000000000000e+02
+4.518720000000000e+02 +6.100000000000000e+01
+1.297180000000000e+02 +1.335000000000000e+02
+6.105250000000000e+01 +1.335000000000000e+02
+1.086200000000000e+02 +1.335000000000000e+02
+9.604000000000000e+02 +1.330000000000000e+02
+3.869330000000000e+01 +1.330000000000000e+02
+7.866439999999999e+02 +1.250000000000000e+02
+2.213950000000000e+01 +1.320000000000000e+02
+7.936630000000000e+02 +1.265000000000000e+02
+6.450960000000000e+02 +1.320000000000000e+02
+9.995599999999999e+02 +7.800000000000000e+01
+2.062390000000000e+00 +1.315000000000000e+02
+1.834280000000000e+01 +1.315000000000000e+02
+5.131360000000000e+02 +1.310000000000000e+02
+9.931760000000000e+01 +1.305000000000000e+02
+8.007719999999998e+02 +1.305000000000000e+02
+6.245540000000000e+02 +9.800000000000000e+01
+1.037430000000000e+03 +1.305000000000000e+02
+8.971020000000000e+02 +1.080000000000000e+02
+1.363750000000000e+02 +1.300000000000000e+02
+1.939850000000000e+01 +1.300000000000000e+02
+1.152620000000000e+02 +1.300000000000000e+02
+1.091870000000000e+03 +1.050000000000000e+02
+6.158170000000000e+02 +1.300000000000000e+02
+7.718539999999998e+02 +1.115000000000000e+02
+2.677950000000000e+02 +1.295000000000000e+02
+1.751660000000000e+01 +1.290000000000000e+02
+2.814530000000000e+02 +8.150000000000000e+01
+5.076670000000000e+02 +1.285000000000000e+02
+1.101270000000000e+03 +1.275000000000000e+02
+2.085190000000000e+01 +1.265000000000000e+02
+8.927780000000000e+02 +1.205000000000000e+02
+4.797780000000000e+01 +1.260000000000000e+02
+3.393090000000000e+00 +1.260000000000000e+02
+7.792189999999998e+02 +1.260000000000000e+02
+6.602209999999999e+01 +1.260000000000000e+02
+6.211300000000000e+02 +9.050000000000000e+01
+5.062940000000000e+02 +1.260000000000000e+02
+1.403950000000000e+02 +1.255000000000000e+02
+8.591039999999998e+02 +1.255000000000000e+02
+4.711390000000000e+02 +1.255000000000000e+02
+7.673200000000001e+02 +1.145000000000000e+02
+1.174800000000000e+01 +1.245000000000000e+02
+7.691590000000000e+00 +1.245000000000000e+02
+7.794100000000000e+02 +1.240000000000000e+02
+1.225710000000000e+02 +1.235000000000000e+02
+4.984950000000000e+02 +1.235000000000000e+02
+7.755230000000000e+02 +1.095000000000000e+02
+1.095020000000000e+03 +1.235000000000000e+02
+2.772820000000000e+02 +7.750000000000000e+01
+1.086090000000000e+03 +1.230000000000000e+02
+8.802800000000000e+00 +1.225000000000000e+02
+1.113270000000000e+03 +1.225000000000000e+02
+1.009610000000000e+02 +1.220000000000000e+02
+7.705319999999998e+02 +1.180000000000000e+02
+3.172640000000000e+02 +1.220000000000000e+02
+1.129870000000000e+01 +1.215000000000000e+02
+5.545570000000000e+02 +1.215000000000000e+02
+7.072100000000000e+02 +1.210000000000000e+02
+6.384590000000002e+02 +1.205000000000000e+02
+1.151420000000000e+02 +1.200000000000000e+02
+4.872980000000000e+02 +6.850000000000000e+01
+9.872300000000000e+02 +1.155000000000000e+02
+8.878030000000000e+02 +1.185000000000000e+02
+6.469620000000000e+00 +1.195000000000000e+02
+4.146380000000000e+02 +1.190000000000000e+02
+8.984480000000000e+02 +7.100000000000000e+01
+7.467050000000000e+02 +1.115000000000000e+02
+4.491590000000000e+02 +1.185000000000000e+02
+5.720210000000000e+00 +1.185000000000000e+02
+8.352509999999999e+01 +1.185000000000000e+02
+4.908670000000000e+02 +1.185000000000000e+02
+8.472139999999998e+02 +9.900000000000000e+01
+8.559770000000000e+01 +1.180000000000000e+02
+9.054170000000001e+01 +1.180000000000000e+02
+5.912490000000000e+02 +1.180000000000000e+02
+6.841000000000000e+02 +1.180000000000000e+02
+6.246230000000000e+02 +1.180000000000000e+02
+4.565240000000000e+02 +4.850000000000000e+01
+1.769520000000000e+01 +1.175000000000000e+02
+7.682700000000000e+02 +1.085000000000000e+02
+4.484270000000000e+02 +1.175000000000000e+02
+9.203350000000000e+02 +1.175000000000000e+02
+6.153030000000000e+02 +1.165000000000000e+02
+9.070660000000000e+02 +9.950000000000000e+01
+2.586830000000000e+00 +1.160000000000000e+02
+7.761400000000000e+02 +1.155000000000000e+02
+3.740220000000000e+00 +1.155000000000000e+02
+7.772970000000000e+02 +1.090000000000000e+02
+1.404280000000000e+01 +1.150000000000000e+02
+3.340410000000000e+02 +7.900000000000000e+01
+8.815280000000000e+02 +1.005000000000000e+02
+2.349260000000000e+00 +1.145000000000000e+02
+5.132520000000000e+02 +1.145000000000000e+02
+9.083620000000000e+02 +9.600000000000000e+01
+2.010000000000000e+01 +1.140000000000000e+02
+7.729639999999998e+02 +1.120000000000000e+02
+9.760230000000000e+02 +1.140000000000000e+02
+2.755640000000000e+02 +6.950000000000000e+01
+4.348900000000000e+02 +1.135000000000000e+02
+3.868060000000000e+02 +1.135000000000000e+02
+9.785860000000000e+02 +1.130000000000000e+02
+8.499830000000002e+02 +1.130000000000000e+02
+2.540950000000000e+02 +1.130000000000000e+02
+6.584110000000002e+02 +7.900000000000000e+01
+7.827200000000000e+02 +1.075000000000000e+02
+1.253040000000000e+02 +1.120000000000000e+02
+4.407890000000000e+02 +1.120000000000000e+02
+6.168840000000000e-01 +1.120000000000000e+02
+3.424850000000000e+00 +1.115000000000000e+02
+1.384700000000000e+01 +1.115000000000000e+02
+4.398880000000000e+02 +1.115000000000000e+02
+6.181830000000000e+02 +1.115000000000000e+02
+9.028049999999999e+02 +1.065000000000000e+02
+4.370440000000000e+02 +1.110000000000000e+02
+6.181600000000000e+02 +7.200000000000000e+01
+7.828010000000000e+02 +1.095000000000000e+02
+4.525960000000000e+02 +1.105000000000000e+02
+5.442619999999999e+02 +1.105000000000000e+02
+7.457850000000001e+01 +1.100000000000000e+02
+6.716880000000000e+02 +1.100000000000000e+02
+3.342720000000000e+02 +1.100000000000000e+02
+9.265060000000000e+02 +1.065000000000000e+02
+1.162170000000000e+02 +1.090000000000000e+02
+7.694010000000002e+02 +1.050000000000000e+02
+3.339870000000000e+02 +7.550000000000000e+01
+6.362700000000000e+02 +1.090000000000000e+02
+1.245290000000000e+02 +1.085000000000000e+02
+5.867500000000000e+01 +1.085000000000000e+02
+2.433980000000000e+02 +1.080000000000000e+02
+2.510230000000000e+01 +1.075000000000000e+02
+4.976210000000000e+02 +1.075000000000000e+02
+8.444370000000000e+02 +9.650000000000000e+01
+6.716760000000000e+02 +1.075000000000000e+02
+2.893740000000000e+01 +1.070000000000000e+02
+6.624360000000000e+02 +1.070000000000000e+02
+8.614589999999999e+02 +1.035000000000000e+02
+6.026820000000000e+02 +1.060000000000000e+02
+2.772920000000000e+02 +6.350000000000000e+01
+6.630580000000000e+02 +7.000000000000000e+01
+7.717220000000000e+02 +1.025000000000000e+02
+4.334980000000001e+02 +1.055000000000000e+02
+8.296050000000000e+02 +1.055000000000000e+02
+1.990320000000000e+01 +1.050000000000000e+02
+6.231680000000000e+02 +7.000000000000000e+01
+4.787060000000000e+01 +1.045000000000000e+02
+7.530250000000000e+02 +1.040000000000000e+02
+5.374750000000000e-01 +1.035000000000000e+02
+7.814490000000000e+02 +1.020000000000000e+02
+4.230920000000000e+02 +1.035000000000000e+02
+7.136180000000001e+02 +1.025000000000000e+02
+9.315260000000001e+01 +1.020000000000000e+02
+4.154450000000000e+02 +1.020000000000000e+02
+6.171849999999999e+02 +1.020000000000000e+02
+4.977590000000000e+01 +1.015000000000000e+02
+5.475980000000002e+02 +1.015000000000000e+02
+1.934610000000000e+02 +1.010000000000000e+02
+4.593690000000000e+01 +1.010000000000000e+02
+7.870660000000000e+02 +1.010000000000000e+02
+5.965910000000000e+02 +1.010000000000000e+02
+6.220190000000000e+02 +6.950000000000000e+01
+5.876300000000000e+02 +1.005000000000000e+02
+1.093250000000000e+03 +1.000000000000000e+02
+6.369750000000000e+02 +6.650000000000000e+01
+3.973210000000000e+02 +1.000000000000000e+02
+5.779290000000000e+02 +1.000000000000000e+02
+7.418250000000000e+02 +9.950000000000000e+01
+7.420210000000002e+02 +9.450000000000000e+01
+8.875910000000000e+02 +9.700000000000000e+01
+8.528370000000000e+02 +9.900000000000000e+01
+3.178400000000000e+02 +9.900000000000000e+01
+7.358539999999998e+02 +9.850000000000000e+01
+3.760460000000000e+01 +9.850000000000000e+01
+7.191230000000000e+02 +9.850000000000000e+01
+5.471559999999999e+02 +9.850000000000000e+01
+1.263960000000000e+02 +9.800000000000000e+01
+6.982840000000000e+02 +9.800000000000000e+01
+3.572550000000000e+02 +9.800000000000000e+01
+9.243450000000000e+01 +9.750000000000000e+01
+6.934120000000000e+01 +9.750000000000000e+01
+5.187230000000002e+02 +9.750000000000000e+01
+8.449970000000000e+02 +8.500000000000000e+01
+5.631220000000000e+01 +9.700000000000000e+01
+7.135120000000001e+01 +9.700000000000000e+01
+3.278820000000000e+01 +9.700000000000000e+01
+7.714730000000002e+02 +9.700000000000000e+01
+4.872820000000000e+02 +7.500000000000000e+01
+1.120190000000000e+03 +9.650000000000000e+01
+5.783800000000000e+02 +9.650000000000000e+01
+1.120810000000000e+01 +9.600000000000000e+01
+1.209440000000000e+02 +9.600000000000000e+01
+6.190820000000000e+01 +9.600000000000000e+01
+7.439040000000000e+02 +9.600000000000000e+01
+4.223750000000000e+02 +9.600000000000000e+01
+5.802859999999999e+02 +9.600000000000000e+01
+9.054480000000000e+02 +9.600000000000000e+01
+3.648210000000000e+02 +9.600000000000000e+01
+4.245830000000000e+02 +9.600000000000000e+01
+4.118250000000000e+02 +9.500000000000000e+01
+5.740180000000000e+02 +9.500000000000000e+01
+3.278810000000000e+01 +9.500000000000000e+01
+1.074350000000000e+02 +9.450000000000000e+01
+7.438420000000000e+02 +9.250000000000000e+01
+9.542820000000000e+02 +9.450000000000000e+01
+4.313740000000000e+02 +9.450000000000000e+01
+1.594580000000000e+02 +9.400000000000000e+01
+8.665770000000000e+02 +9.400000000000000e+01
+6.476580000000000e+02 +4.900000000000000e+01
+2.764920000000000e+02 +9.400000000000000e+01
+6.468290000000001e+00 +9.350000000000000e+01
+3.362770000000000e+02 +6.650000000000000e+01
+2.543070000000000e+02 +9.350000000000000e+01
+9.551600000000001e+01 +9.300000000000000e+01
+6.172790000000000e+01 +9.300000000000000e+01
+5.669069999999998e+02 +9.300000000000000e+01
+1.402890000000000e+02 +9.250000000000000e+01
+6.500480000000000e+02 +9.250000000000000e+01
+2.053000000000000e+01 +9.200000000000000e+01
+6.349960000000000e+02 +9.200000000000000e+01
+6.772830000000000e+01 +9.150000000000000e+01
+1.086700000000000e+02 +9.150000000000000e+01
+1.094820000000000e+02 +9.150000000000000e+01
+1.839990000000000e+01 +9.150000000000000e+01
+9.513650000000000e+01 +9.100000000000000e+01
+1.244530000000000e+02 +9.100000000000000e+01
+7.127030000000000e+02 +9.100000000000000e+01
+5.265090000000000e+02 +9.100000000000000e+01
+5.965880000000002e+02 +9.100000000000000e+01
+8.058869999999999e+02 +9.050000000000000e+01
+2.989360000000000e+00 +9.050000000000000e+01
+6.434580000000002e+02 +9.050000000000000e+01
+8.045930000000002e+02 +9.050000000000000e+01
+3.992860000000000e+02 +9.050000000000000e+01
+7.618460000000000e+01 +9.000000000000000e+01
+1.132400000000000e+02 +9.000000000000000e+01
+4.222170000000000e+02 +9.000000000000000e+01
+5.032200000000000e+02 +9.000000000000000e+01
+5.953750000000000e+02 +9.000000000000000e+01
+1.211960000000000e+02 +8.950000000000000e+01
+7.523950000000000e+02 +5.650000000000000e+01
+4.142730000000000e+02 +8.900000000000000e+01
+6.530219999999998e+02 +4.450000000000000e+01
+2.104250000000000e+01 +8.850000000000000e+01
+6.755480000000000e+01 +8.850000000000000e+01
+3.307410000000000e+01 +8.850000000000000e+01
+9.342779999999999e+00 +8.800000000000000e+01
+4.764010000000000e+02 +8.800000000000000e+01
+9.301410000000000e+01 +8.750000000000000e+01
+2.114770000000000e+02 +8.750000000000000e+01
+5.972030000000000e+01 +8.700000000000000e+01
+6.071040000000000e+02 +8.700000000000000e+01
+7.139439999999999e-01 +8.700000000000000e+01
+6.257590000000000e+02 +5.700000000000000e+01
+6.767919999999999e+01 +8.650000000000000e+01
+7.727870000000000e+01 +8.650000000000000e+01
+1.647680000000000e+01 +8.650000000000000e+01
+2.814910000000000e+02 +4.750000000000000e+01
+4.897200000000000e+02 +8.650000000000000e+01
+4.640230000000000e+02 +8.600000000000000e+01
+7.370910000000000e+02 +8.600000000000000e+01
+1.030290000000000e+02 +8.550000000000000e+01
+3.225490000000000e+01 +8.500000000000000e+01
+2.769150000000000e+01 +8.500000000000000e+01
+1.125670000000000e+02 +8.500000000000000e+01
+5.498569999999999e-01 +8.500000000000000e+01
+6.183740000000000e+02 +8.500000000000000e+01
+7.694850000000000e+01 +8.450000000000000e+01
+9.630980000000000e+01 +8.450000000000000e+01
+8.166900000000001e+02 +8.450000000000000e+01
+5.088520000000001e-01 +8.400000000000000e+01
+2.885800000000000e+02 +8.400000000000000e+01
+6.950409999999999e+01 +8.350000000000000e+01
+6.891130000000001e+02 +8.350000000000000e+01
+3.178430000000000e+00 +8.300000000000000e+01
+4.359060000000000e+02 +8.300000000000000e+01
+6.221760000000000e+01 +8.250000000000000e+01
+6.220020000000000e+01 +8.250000000000000e+01
+7.429800000000000e+02 +8.150000000000000e+01
+5.921120000000000e+02 +8.250000000000000e+01
+6.491540000000000e+02 +8.200000000000000e+01
+6.233410000000000e+02 +5.350000000000000e+01
+7.922480000000000e+02 +8.200000000000000e+01
+7.271590000000000e+02 +8.200000000000000e+01
+2.715850000000000e+02 +8.200000000000000e+01
+4.535400000000000e+01 +8.100000000000000e+01
+6.338980000000000e+01 +8.100000000000000e+01
+4.415870000000000e+00 +8.100000000000000e+01
+6.072160000000000e+02 +8.100000000000000e+01
+8.290000000000000e+02 +8.100000000000000e+01
+4.053240000000000e+01 +8.050000000000000e+01
+5.761780000000000e+02 +8.050000000000000e+01
+3.959010000000000e+02 +8.050000000000000e+01
+9.107780000000000e+02 +7.000000000000000e+01
+4.318930000000000e+01 +8.000000000000000e+01
+3.895060000000000e+02 +8.000000000000000e+01
+5.451420000000000e+01 +7.950000000000000e+01
+7.357420000000000e+02 +7.950000000000000e+01
+9.841310000000000e+01 +7.900000000000000e+01
+2.129540000000000e+00 +7.900000000000000e+01
+2.800780000000000e+01 +7.850000000000000e+01
+7.248470000000000e+02 +7.850000000000000e+01
+6.555130000000000e+02 +7.850000000000000e+01
+7.113869999999999e+02 +7.850000000000000e+01
+1.581600000000000e+00 +7.800000000000000e+01
+6.215050000000000e+02 +7.800000000000000e+01
+2.199700000000000e+00 +7.750000000000000e+01
+5.904320000000000e+02 +7.750000000000000e+01
+6.814190000000000e+02 +7.750000000000000e+01
+5.976860000000000e+02 +7.700000000000000e+01
+5.171390000000000e+02 +7.650000000000000e+01
+3.068920000000000e+01 +7.600000000000000e+01
+6.623830000000000e+02 +4.850000000000000e+01
+5.113490000000000e+02 +7.600000000000000e+01
+6.161120000000000e+02 +4.100000000000000e+01
+3.245580000000000e+02 +7.600000000000000e+01
+2.870990000000000e+01 +7.550000000000000e+01
+1.069640000000000e+03 +7.550000000000000e+01
+3.509120000000000e-01 +7.550000000000000e+01
+4.840770000000000e+02 +7.550000000000000e+01
+5.508480000000002e+02 +7.550000000000000e+01
+1.202650000000000e-01 +7.500000000000000e+01
+5.785450000000000e+02 +7.500000000000000e+01
+7.262089999999999e+02 +7.500000000000000e+01
+7.252350000000000e+02 +7.500000000000000e+01
+2.675600000000000e-01 +7.450000000000000e+01
+5.046700000000000e+02 +7.450000000000000e+01
+4.470980000000000e+02 +7.450000000000000e+01
+4.828710000000000e+02 +7.450000000000000e+01
+7.365060000000002e+02 +7.450000000000000e+01
+5.023490000000000e+01 +7.400000000000000e+01
+4.244990000000000e+01 +7.400000000000000e+01
+8.363810000000002e+02 +7.400000000000000e+01
+4.202510000000000e+01 +7.350000000000000e+01
+5.853140000000000e+00 +7.350000000000000e+01
+2.087730000000000e+00 +7.350000000000000e+01
+3.780920000000000e+02 +7.300000000000000e+01
+4.769920000000000e+02 +7.250000000000000e+01
+6.632209999999999e-01 +7.200000000000000e+01
+4.479970000000000e+02 +7.200000000000000e+01
+9.332500000000000e+01 +7.200000000000000e+01
+4.852370000000000e+02 +3.800000000000000e+01
+5.520840000000002e+02 +7.100000000000000e+01
+7.855870000000000e+02 +7.000000000000000e+01
+4.653050000000000e+02 +7.000000000000000e+01
+4.463600000000000e+01 +6.950000000000000e+01
+4.142800000000000e+02 +6.950000000000000e+01
+3.847710000000000e+02 +6.950000000000000e+01
+4.713590000000000e+02 +6.950000000000000e+01
+1.261420000000000e+02 +6.900000000000000e+01
+2.511090000000000e+02 +6.900000000000000e+01
+2.870650000000000e+02 +6.900000000000000e+01
+9.467500000000000e+01 +6.850000000000000e+01
+4.352740000000000e+02 +6.850000000000000e+01
+1.134350000000000e+02 +6.850000000000000e+01
+1.942380000000000e+02 +6.850000000000000e+01
+1.699830000000000e+02 +6.800000000000000e+01
+2.855830000000000e+01 +6.750000000000000e+01
+4.890730000000000e+02 +6.750000000000000e+01
+3.788840000000000e+00 +6.700000000000000e+01
+4.423750000000000e+02 +6.700000000000000e+01
+7.041710000000000e+02 +6.650000000000000e+01
+2.321090000000000e+02 +6.650000000000000e+01
+1.857590000000000e+02 +6.650000000000000e+01
+7.956319999999999e+01 +6.600000000000000e+01
+2.445210000000000e+02 +6.600000000000000e+01
+4.095270000000000e+02 +6.600000000000000e+01
+3.791050000000000e+02 +6.550000000000000e+01
+6.793730000000000e+02 +6.500000000000000e+01
+4.657100000000001e+01 +6.500000000000000e+01
+5.060950000000000e+02 +6.500000000000000e+01
+4.561320000000000e+02 +6.500000000000000e+01
+7.354820000000000e+02 +6.450000000000000e+01
+4.919950000000000e+02 +6.450000000000000e+01
+6.946780000000000e+02 +6.350000000000000e+01
+2.649730000000000e+02 +6.350000000000000e+01
+3.690110000000000e+01 +6.300000000000000e+01
+4.488700000000000e+02 +6.300000000000000e+01
+1.142570000000000e+02 +6.300000000000000e+01
+8.649070000000000e+01 +6.250000000000000e+01
+5.903660000000000e+02 +6.250000000000000e+01
+3.776490000000000e+02 +6.250000000000000e+01
+9.031540000000000e+01 +6.150000000000000e+01
+1.267320000000000e+02 +6.150000000000000e+01
+4.835580000000000e+02 +6.150000000000000e+01
+7.411900000000001e+02 +6.100000000000000e+01
+1.260530000000000e+02 +6.050000000000000e+01
+3.998450000000000e+02 +6.050000000000000e+01
+2.363270000000000e+02 +6.050000000000000e+01
+2.598980000000000e+02 +6.050000000000000e+01
+8.706900000000000e+01 +6.000000000000000e+01
+7.060810000000000e+02 +6.000000000000000e+01
+4.899180000000000e+02 +6.000000000000000e+01
+1.936170000000000e+02 +6.000000000000000e+01
+4.025810000000000e+02 +6.000000000000000e+01
+3.592410000000000e+02 +6.000000000000000e+01
+2.085920000000000e+02 +6.000000000000000e+01
+6.189800000000000e+02 +5.950000000000000e+01
+4.575130000000000e+02 +5.950000000000000e+01
+4.443620000000000e+02 +5.950000000000000e+01
+3.116520000000000e+02 +5.950000000000000e+01
+2.488000000000000e+02 +5.900000000000000e+01
+2.569890000000000e+00 +5.900000000000000e+01
+1.239950000000000e+02 +5.900000000000000e+01
+2.303650000000000e+02 +5.850000000000000e+01
+6.438360000000000e+02 +5.750000000000000e+01
+3.927380000000001e+02 +5.750000000000000e+01
+3.594720000000000e+02 +5.650000000000000e+01
+4.252040000000000e+02 +5.650000000000000e+01
+1.235740000000000e+02 +5.650000000000000e+01
+6.725930000000002e+02 +5.550000000000000e+01
+4.030200000000000e+02 +5.550000000000000e+01
+1.842310000000000e+00 +5.550000000000000e+01
+8.385550000000001e+01 +5.500000000000000e+01
+2.529440000000000e+02 +5.500000000000000e+01
+3.315640000000000e+02 +5.500000000000000e+01
+6.029390000000000e+02 +5.500000000000000e+01
+3.784490000000000e+02 +5.450000000000000e+01
+9.559170000000000e+00 +5.400000000000000e+01
+3.540380000000000e+01 +5.400000000000000e+01
+5.983710000000000e+02 +5.350000000000000e+01
+2.526220000000000e+02 +5.300000000000000e+01
+3.883530000000000e+02 +5.300000000000000e+01
+1.966230000000000e+02 +5.250000000000000e+01
+2.183780000000000e+02 +5.250000000000000e+01
+4.105160000000000e+02 +5.200000000000000e+01
+3.364760000000000e+02 +5.200000000000000e+01
+2.434550000000000e+02 +5.200000000000000e+01
+2.875660000000000e+00 +5.150000000000000e+01
+7.526909999999999e+01 +5.150000000000000e+01
+6.220350000000000e+02 +5.150000000000000e+01
+3.531890000000000e+02 +5.150000000000000e+01
+2.940040000000000e+02 +5.150000000000000e+01
+5.531569999999998e+02 +5.100000000000000e+01
+2.164730000000000e+02 +5.050000000000000e+01
+3.321430000000001e+02 +5.050000000000000e+01
+2.024730000000000e+02 +5.050000000000000e+01
+3.251510000000000e+02 +5.050000000000000e+01
+3.756420000000000e+02 +5.000000000000000e+01
+8.662290000000000e+01 +4.950000000000000e+01
+2.293550000000000e+02 +4.950000000000000e+01
+6.339010000000000e+02 +4.900000000000000e+01
+1.188780000000000e+02 +4.900000000000000e+01
+7.595610000000001e+01 +4.900000000000000e+01
+2.075820000000000e+02 +4.850000000000000e+01
+3.880080000000000e+02 +4.850000000000000e+01
+2.904310000000000e-01 +4.850000000000000e+01
+1.175370000000000e+02 +4.850000000000000e+01
+6.526469999999998e+02 +4.800000000000000e+01
+3.860220000000000e+02 +4.750000000000000e+01
+1.658050000000000e+02 +4.750000000000000e+01
+7.708150000000001e+01 +4.750000000000000e+01
+3.056080000000000e+02 +4.750000000000000e+01
+3.945580000000000e+02 +4.650000000000000e+01
+3.843780000000000e+00 +4.600000000000000e+01
+1.870590000000000e+02 +4.600000000000000e+01
+3.202920000000001e+02 +4.550000000000000e+01
+7.443639999999998e+00 +4.500000000000000e+01
+5.500899999999999e+01 +4.500000000000000e+01
+7.151360000000000e+01 +4.500000000000000e+01
+1.342730000000000e+02 +4.500000000000000e+01
+2.930870000000000e+02 +4.400000000000000e+01
+2.252590000000000e+01 +4.350000000000000e+01
+8.598610000000001e+00 +4.350000000000000e+01
+2.388740000000000e+02 +4.350000000000000e+01
+1.888070000000000e+02 +4.350000000000000e+01
+3.520440000000000e+00 +4.300000000000000e+01
+2.114490000000000e+02 +4.300000000000000e+01
+2.549160000000000e+02 +4.300000000000000e+01
+7.984360000000000e+01 +4.250000000000000e+01
+1.883130000000000e+02 +4.250000000000000e+01
+2.528900000000000e+02 +4.250000000000000e+01
+3.424140000000000e+02 +4.200000000000000e+01
+2.255770000000000e+02 +4.150000000000000e+01
+3.036080000000000e+02 +4.100000000000000e+01
+5.086579999999999e-01 +4.050000000000000e+01
+2.613250000000000e+02 +4.050000000000000e+01
+2.636640000000000e+02 +3.950000000000000e+01
+3.810060000000000e+02 +3.900000000000000e+01
+2.203400000000000e+02 +3.900000000000000e+01
+2.610330000000000e+02 +3.850000000000000e+01
+2.574830000000000e+02 +3.750000000000000e+01
+2.460910000000000e+02 +3.700000000000000e+01
+1.877900000000000e+02 +3.700000000000000e+01
+1.641660000000000e+02 +3.650000000000000e+01
+3.275600000000000e+02 +3.650000000000000e+01
+5.981650000000000e+01 +3.600000000000000e+01
+5.368370000000000e+02 +3.600000000000000e+01
+1.385900000000000e+02 +3.600000000000000e+01
+3.564870000000000e+02 +3.600000000000000e+01
+6.408450000000001e+01 +3.550000000000000e+01
+2.101020000000000e+02 +3.550000000000000e+01
+1.850030000000000e+02 +3.550000000000000e+01
+1.687509999999999e+02 +3.500000000000000e+01
+2.315280000000000e+01 +3.450000000000000e+01
+1.497940000000000e+02 +3.450000000000000e+01
+1.285790000000000e+02 +3.400000000000000e+01
+2.394000000000000e+02 +3.400000000000000e+01
+1.196780000000000e+02 +3.350000000000000e+01
+3.611300000000000e+02 +3.350000000000000e+01
+1.645370000000000e+02 +3.350000000000000e+01
+3.190230000000000e+02 +3.350000000000000e+01
+4.455130000000000e+02 +3.300000000000000e+01
+1.594520000000000e+02 +3.300000000000000e+01
+2.174920000000000e+02 +3.300000000000000e+01
+1.176250000000000e+02 +3.250000000000000e+01
+3.512870000000000e+02 +3.250000000000000e+01
+2.269150000000000e+02 +3.200000000000000e+01
+2.326420000000000e+02 +3.200000000000000e+01
+3.407490000000000e+02 +2.650000000000000e+01
+1.680350000000000e+02 +3.150000000000000e+01
+1.390350000000000e+02 +3.100000000000000e+01
+2.221220000000000e+02 +3.100000000000000e+01
+3.217740000000000e+01 +3.100000000000000e+01
+1.908560000000000e+02 +3.050000000000000e+01
+1.192380000000000e+02 +3.050000000000000e+01
+2.340440000000000e+02 +3.050000000000000e+01
+1.092770000000000e+02 +3.000000000000000e+01
+2.651500000000000e+02 +3.000000000000000e+01
+2.762120000000000e+02 +2.950000000000000e+01
+1.573920000000000e+02 +2.850000000000000e+01
+5.743170000000000e+01 +2.750000000000000e+01
+3.947170000000000e+02 +2.750000000000000e+01
+1.509550000000000e+02 +2.750000000000000e+01
+5.618110000000000e+01 +2.750000000000000e+01
+2.713780000000000e+02 +2.700000000000000e+01
+1.577390000000000e+00 +2.650000000000000e+01
+9.795880000000000e+01 +2.650000000000000e+01
+2.094100000000000e+02 +2.600000000000000e+01
+3.299450000000000e+02 +2.600000000000000e+01
+3.391310000000000e+00 +2.550000000000000e+01
+1.684870000000000e+02 +2.550000000000000e+01
+1.152970000000000e+02 +2.500000000000000e+01
+1.346240000000000e+02 +2.500000000000000e+01
+5.287370000000000e+01 +2.400000000000000e+01
+3.207330000000000e+02 +2.400000000000000e+01
+1.864830000000000e+02 +2.400000000000000e+01
+1.363730000000000e+01 +2.350000000000000e+01
+1.502730000000000e+02 +2.300000000000000e+01
+2.465240000000000e+02 +2.300000000000000e+01
+3.092360000000000e+02 +2.250000000000000e+01
+9.199850000000001e+01 +2.250000000000000e+01
+1.871050000000000e+02 +2.250000000000000e+01
+1.001080000000000e+02 +2.250000000000000e+01
+8.161730000000000e+01 +2.150000000000000e+01
+5.044660000000000e+01 +2.100000000000000e+01
+1.180270000000000e+02 +2.100000000000000e+01
+2.296230000000000e+02 +2.100000000000000e+01
+2.303830000000000e+02 +2.100000000000000e+01
+1.642160000000000e+01 +2.050000000000000e+01
+2.444720000000000e+02 +2.050000000000000e+01
+8.939010000000000e+01 +2.050000000000000e+01
+6.739600000000000e+01 +2.000000000000000e+01
+7.240360000000000e+01 +2.000000000000000e+01
+1.775150000000000e+02 +2.000000000000000e+01
+1.917630000000000e+02 +1.950000000000000e+01
+2.892760000000000e+02 +1.950000000000000e+01
+2.546500000000000e+01 +1.850000000000000e+01
+1.607770000000000e+02 +1.850000000000000e+01
+6.955070000000001e+01 +1.800000000000000e+01
+1.339170000000000e+02 +1.800000000000000e+01
+1.273920000000000e+02 +1.750000000000000e+01
+4.308620000000000e+01 +1.700000000000000e+01
+2.339410000000000e+02 +1.700000000000000e+01
+4.558400000000000e+01 +1.650000000000000e+01
+2.080500000000000e+02 +1.650000000000000e+01
+2.385610000000000e+02 +1.650000000000000e+01
+7.449110000000001e+00 +1.600000000000000e+01
+2.202950000000000e+02 +1.600000000000000e+01
+1.214050000000000e+02 +1.600000000000000e+01
+1.461290000000000e+02 +1.550000000000000e+01
+1.069830000000000e+02 +1.550000000000000e+01
+2.274160000000000e+02 +1.550000000000000e+01
+2.163830000000000e+02 +1.500000000000000e+01
+1.856420000000000e+02 +1.450000000000000e+01
+1.519180000000000e+02 +1.450000000000000e+01
+9.307570000000000e+01 +1.400000000000000e+01
+1.540080000000000e+02 +1.400000000000000e+01
+1.637480000000000e+02 +1.350000000000000e+01
+1.973400000000000e+02 +1.350000000000000e+01
+3.991740000000000e+01 +1.300000000000000e+01
+5.887060000000000e+01 +1.300000000000000e+01
+9.794130000000000e+01 +1.300000000000000e+01
+1.725550000000000e+02 +1.300000000000000e+01
+3.977610000000000e+01 +1.150000000000000e+01
+1.037170000000000e+02 +1.150000000000000e+01
+7.866180000000000e+01 +1.150000000000000e+01
+1.309940000000000e+02 +1.100000000000000e+01
+1.239150000000000e+02 +1.050000000000000e+01
+4.609090000000000e+01 +1.050000000000000e+01
+1.518670000000000e+02 +1.050000000000000e+01
+8.580860000000000e+01 +1.050000000000000e+01
+2.048710000000000e+01 +1.000000000000000e+01
+3.822630000000000e+01 +1.000000000000000e+01
+1.225250000000000e+02 +1.000000000000000e+01
+1.202580000000000e+02 +1.000000000000000e+01
+1.410660000000000e+02 +9.500000000000000e+00
+4.534470000000000e+01 +9.500000000000000e+00
+7.416800000000001e+01 +8.500000000000000e+00
+8.159470000000000e+01 +8.500000000000000e+00
+9.488750000000000e+01 +8.500000000000000e+00
+5.083770000000000e+01 +8.000000000000000e+00
+3.611700000000000e+01 +8.000000000000000e+00
+7.747780000000000e+01 +8.000000000000000e+00
+2.501030000000000e+01 +7.500000000000000e+00
+6.103360000000000e+01 +7.500000000000000e+00
+6.751620000000000e+01 +7.500000000000000e+00
+4.574360000000000e+01 +7.500000000000000e+00
+9.407859999999999e+01 +7.500000000000000e+00
+2.150570000000000e+01 +7.000000000000000e+00
+6.823620000000000e+01 +7.000000000000000e+00
+8.732890000000000e+01 +6.500000000000000e+00
+1.103930000000000e+01 +6.500000000000000e+00
+1.701720000000000e+01 +6.000000000000000e+00
+1.433880000000000e+01 +6.000000000000000e+00
+5.468140000000000e+01 +6.000000000000000e+00
+3.434620000000000e+00 +5.500000000000000e+00
+1.038430000000000e+01 +5.500000000000000e+00
+1.371090000000000e+01 +5.500000000000000e+00
+1.391870000000000e+01 +5.500000000000000e+00
+6.948660000000000e+01 +5.500000000000000e+00
+6.481820000000000e+01 +5.500000000000000e+00
+3.285420000000000e+01 +5.000000000000000e+00
+8.749599999999999e+00 +5.000000000000000e+00
+8.277690000000000e+00 +4.500000000000000e+00
+4.507150000000000e+01 +4.500000000000000e+00
+3.737760000000000e+00 +4.500000000000000e+00
+6.750910000000000e+00 +4.000000000000000e+00
+4.450420000000000e+01 +4.000000000000000e+00
+4.231930000000000e+01 +4.000000000000000e+00
+3.325210000000000e+01 +4.000000000000000e+00
+1.549400000000000e+01 +3.500000000000000e+00
+2.710910000000000e+00 +2.500000000000000e+00
+1.631730000000000e+01 +2.500000000000000e+00
+2.287520000000000e+01 +2.500000000000000e+00
+1.843750000000000e+01 +2.500000000000000e+00
+1.146520000000000e+00 +2.000000000000000e+00
+5.021090000000000e-01 +2.000000000000000e+00
+1.630000000000000e+01 +2.000000000000000e+00
+1.314640000000000e-02 +1.500000000000000e+00
+4.957180000000000e+00 +1.000000000000000e+00
+8.277580000000000e-01 +5.000000000000000e-01
+3.969170000000000e-02 +5.000000000000000e-01
+1.551810000000000e+00 +5.000000000000000e-01
+1.930340000000000e+00 +5.000000000000000e-01
+3.137500000000000e+00 +5.000000000000000e-01
+0.000000000000000e+00 +0.000000000000000e+00
+0.000000000000000e+00 +0.000000000000000e+00
+0.000000000000000e+00 +0.000000000000000e+00
+0.000000000000000e+00 +0.000000000000000e+00
+0.000000000000000e+00 +0.000000000000000e+00
+0.000000000000000e+00 +0.000000000000000e+00
+0.000000000000000e+00 +0.000000000000000e+00
+0.000000000000000e+00 +0.000000000000000e+00
+0.000000000000000e+00 +0.000000000000000e+00
+0.000000000000000e+00 +0.000000000000000e+00
+0.000000000000000e+00 +0.000000000000000e+00
+0.000000000000000e+00 +0.000000000000000e+00
+0.000000000000000e+00 +0.000000000000000e+00
};
\end{axis}

\end{tikzpicture}
    \caption{PER 0.0 vs PER 1.0}
    \label{fig:per00per10timedistance}
\end{figure}