\section{Resultados}\label{sec:results}

\subsection{Análisis}

Los resultados obtenidos de cada simulación fuero exportados desde OMNeT++ a archivos CSV, los cuales luego se analizaron utilizando Python 3.6.
Se utilizaron las librerías Pandas \autocite{pandas}, para el manejo de los datos de manera eficiente, Numpy \autocite{numpy}, para cálculos, y Matplotlib \autocite{matplotlib} para la generación de gráficos que permitiesen analizar los datos de manera más efectiva e intuitiva.

\subsection{Eficiencia}

La figura \ref{fig:vehiclesvstime} presenta los resultados obtenidos del tiempo de duración de la simulación versus la cantidad de vehículos promedio por instante de tiempo, mientras que la tabla \ref{table:vehiclesvstime} presenta estos mismos datos de manera resumida. De estos resultados se puede concluir que el \emph{framework}, como era de esperarse dada la naturaleza de la simulación, presenta una relación levemente exponencial entre la cantidad promedio de nodos en la red vehicular con el tiempo que efectivamente demora una simulación en completar su ejecución.

Se utilizó numpy para calcular un ajuste polinomial a los datos obtenidos, obteniendo la siguiente relación entre cantidad de vehículos promedio en la simulación $N_{v}$ y tiempo de ejecución de ésta en segundos, $T(N_{v})$:

\[ T(N_{v}) = (5.829 \times 10^{-4})N_{v}^{2} + (2.586 \times 10^{-1})N_{v} + 6.317 \]

Esto indica que a pesar de ser exponencial, la relación presenta una curva bastante suavizada. Es factible entonces simular escenarios de escala aún mayor que la presentada en esta memoria (la cual, cabe notar, no es menor), sin mayores dificultades.

\begin{figure}[htpb]
    \centering
    \begin{tabular}{@{}rrr@{}}
        \textbf{Factor de Demanda} & \textbf{Nro. Prom. Vehículos} & \textbf{Tiempo Promedio [s]} \\ \midrule
        100\%           & 1379.9          & 1471.5              \\ %\midrule
        75\%            & 868.75          & 683.5               \\ %\midrule
        50\%            & 514.5825        & 275.75              \\ %\midrule
        25\%            & 246.5675        & 113.25              \\ \bottomrule
    \end{tabular}
    \captionof{table}[Cantidad de vehículos vs. tiempo real de simulación]{Promedio cantidad de vehículos en simulación (por instante de tiempo) vs. tiempo promedio de simulación, 15 minutos de tiempo simulado.}
    \label{table:vehiclesvstime}
    
    
    \input{figuras/n_vhcs_vs_time.pgf}
    \captionof{figure}[Cantidad de vehículos vs. tiempo real de simulación]{Gráfico de dispersión del promedio de vehículos en simulación por instante de tiempo vs. tiempo total de simulación, para una simulación de 15 minutos de tiempo simulado.}
    \label{fig:vehiclesvstime}
\end{figure}

La figura \ref{fig:timevsvehicles_evolution} presenta además la evolución de la red vehicular en términos de cantidad de vehículos para un \emph{run} con factor de demanda de 100\%, tanto en tiempo real como en tiempo simulado. 

\begin{figure}[h]
    \centering
    \input{figuras/timevsvehicles_evolution.pgf}
    \caption[Evolución temporal de la cantidad de vehículos en la simulación.]{Evolución de la cantidad de vehículos en una simulación con factor de demanda 100\%, para tiempo real y simulado.}
    \label{fig:timevsvehicles_evolution}
\end{figure}

Por otro lado, en términos de carga sobre el entorno de simulación, se pueden observar los resultados obtenidos en las figuras \ref{fig:systemload:cpuram} y \ref{fig:systemload:io}. 

La figura \ref{fig:systemload:cpuram} ilustra la carga sobre el sistema en términos porcentuales. En específico, se puede observar como el uso promedio del procesador aumenta en aproximadamente un 20\% durante la simulación, situación fácilmente manejable para cualquier procesador moderno. Además, el uso de memoria aumenta en menos de un 5\% -- en términos numéricos, el sistema utiliza menos de 600 MB para simular un escenario con un promedio de 1400 nodos presentes en cualquier instante, lo cual es un valor muy razonable si se considera que el estándar de memoria RAM para computadores personales hoy en día es por lo menos 4 GB \autocite{steamhwsurvey, unityhardwaresurvey}.

\begin{figure}[h]
    \centering
    \input{figuras/system_performance.pgf}
    \caption{Carga sobre el sistema durante una simulación con factor de demanda 100\%.}
    \label{fig:systemload:cpuram}
\end{figure}
\begin{figure}[h]
    \centering
    \input{figuras/system_io.pgf}
    \caption[I/O en disco durante simulación]{Lecturas y escrituras de disco por segundo durante una simulación con factor de demanda 100\%.}
    \label{fig:systemload:io}
\end{figure} 


\subsection{Simulación vehicular}

\begin{figure}
    \centering
    \input{figuras/per00per10_timedistance.pgf}
    \caption{PER 0.0 vs PER 1.0}
    \label{fig:per00per10timedistance}
\end{figure}