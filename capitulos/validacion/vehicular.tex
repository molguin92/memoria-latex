\section{Modelo Vehicular}\label{sec:results:vehicular}

\subsection{Mediciones Realizadas}

Esta categoría de experimentos y mediciones pretende verificar el correcto funcionamiento del \emph{framework} para la simulación de Sistemas Inteligentes de Transporte, utilizando el escenario de simulación descrito al principio del presente capítulo, en la sección \ref{sec:experiments}. Cabe destacar que en ningún caso se pretende argumentar que el escenario en cuestión es óptimo, ni que lo parámetros del sistema de transporte son los correctos para este escenario, sino simplemente que PVEINS permite de manera precisa y confiable comparar y medir las ventajas que otorga un sistema de transporte dotado de comunicación intervehicular.

Para este fin se realizaron 6 \emph{runs} del escenario en cuestión, la mitad sin comunicación alguna entre vehículos y la otra mitad con comunicación perfecta\footnote{Es decir, con un factor de pérdida de paquetes en el medio de transmisión de un 0\%}, para tres factores de demanda distintos. Estos \emph{runs} se contrastaron principalmente en términos de la cantidad de vehículos que alcanzaron su destino dentro de los 15 minutos de simulación, factor que de manera intuitiva permite evaluar el desempeño del sistema. Para \emph{runs} con el mismo factor de demanda, una cantidad menor de vehículos que alcanzan su destino en un lapso de tiempo dado indica una menor eficiencia del sistema y un mayor retardo en los viajes realizados dentro del escenario. De esta manera, se pretende mostrar que el ``accidente'' modelado causa una cierta congestión en el sistema de transporte, y que, utilizando la comunicación intervehicular, es posible disminuir dicho impacto sobre la red. 

Además, se realizaron dos análisis un poco más avanzados para los casos con factor de demanda 100\%; uno contrastando distancia y tiempos de recorrido en ambas simulaciones (con y sin comunicación), y un segundo comparando la emisión de dióxido de carbono en ambos casos. 

Todos los valores analizados en estos experimentos fueron obtenidos desde OMNeT++ -- el simulador de redes de comunicación a su vez los obtiene desde el simulador de transporte o los calcula en base a datos proporcionados por este último. 

\subsection{Resultados}

Los resultados del conjunto de 6 \emph{runs} con distintos factores de demanda puede observarse en la figura \ref{fig:arrivedcomp}, representados en un diagrama de barras. Puede observarse que aquellos \emph{runs} dotados de capacidades de comunicación intervehicular constantemente presentan una mayor cantidad de vehículos que alcanzan su destino final en el escenario dentro del tiempo de simulación. Si bien esta mejora en términos porcentuales es de apenas un 2.33\% en promedio, en la realidad se traduce a una considerable cantidad de vehículos que en el escenario \emph{sin comunicación} se ven atascados en una congestión que no les permite alcanzar su destino, pero que que en el escenario \emph{con comunicación} logran evitar esta situación.

\begin{figure}[tpb]
    \centering
    % This file was created by matplotlib2tikz v0.6.10.
\begin{tikzpicture}

\begin{axis}[
xlabel={Factor de Demanda},
xmin=-0.22, xmax=2.42,
ymin=0, ymax=5000,
width=\figurewidth,
height=\figureheight,
xtick={0.1,1.1,2.1},
xticklabels={20\%,50\%,100\%},
tick align=outside,
tick pos=left,
xmajorgrids,
x grid style={lightgray, opacity=0.7},
ymajorgrids,
y grid style={lightgray, opacity=0.7},
axis line style={black, opacity=0.0},
legend cell align={left},
legend style={at={(0.03,0.97)}, anchor=north west, draw=white!80.0!black, fill=white!89.803921568627459!black},
legend entries={{Sin comunicaci\'on},{Con comunicaci\'on}}
]
\addlegendimage{ybar,ybar legend,fill=red,draw opacity=0};
\draw[fill=red,draw opacity=0] (axis cs:-0.1,0) rectangle (axis cs:0.1,985);
\draw[fill=red,draw opacity=0] (axis cs:0.9,0) rectangle (axis cs:1.1,2441);
\draw[fill=red,draw opacity=0] (axis cs:1.9,0) rectangle (axis cs:2.1,3801);
\addlegendimage{ybar,ybar legend,fill=blue,draw opacity=0};
\draw[fill=blue,draw opacity=0] (axis cs:0.1,0) rectangle (axis cs:0.3,1009);
\draw[fill=blue,draw opacity=0] (axis cs:1.1,0) rectangle (axis cs:1.3,2512);
\draw[fill=blue,draw opacity=0] (axis cs:2.1,0) rectangle (axis cs:2.3,3858);
\node at (axis cs:-0.05,1025)[
scale=1.0,
anchor=west,
text=black,
rotate=45.0
]{ 985};
\node at (axis cs:0.95,2481)[
scale=1.0,
anchor=west,
text=black,
rotate=45.0
]{ 2441};
\node at (axis cs:1.95,3841)[
scale=1.0,
anchor=west,
text=black,
rotate=45.0
]{ 3801};
\node at (axis cs:0.15,1049)[
scale=1.0,
anchor=west,
text=black,
rotate=45.0
]{ 1009};
\node at (axis cs:1.15,2552)[
scale=1.0,
anchor=west,
text=black,
rotate=45.0
]{ 2512};
\node at (axis cs:2.15,3898)[
scale=1.0,
anchor=west,
text=black,
rotate=45.0
]{ 3858};
\end{axis}

\end{tikzpicture}
    \caption[Comparación cantidad de vehículos que alcanzaron su destino]{Comparación cantidad de vehículos que alcanzaron su destino en 15 minutos de tiempo simulado con tres factores de demanda distintos.}
    \label{fig:arrivedcomp}
\end{figure}

Con el fin de ilustrar de mejor manera el impacto de la comunicación en la cantidad de vehículos que logran evitar congestión y trasladarse exitosamente a su destino, se realizaron dos simulaciones adicionales al conjunto mencionado anteriormente. Esta simulaciones fueron de una duración extendida de dos horas de tiempo simulado, y los resultados de éstas se pueden visualizar en la tabla \ref{table:2hrssimulation}, comparados con los resultados obtenidos de la simulación de 15 minutos con factor de demanda 100\%. Se puede evidenciar que la diferencia estudiada se acentúa con el tiempo -- mientras que para el escenario de 15 minutos la diferencia entre la cantidad de vehículos que alcanzan su destino para ambas configuraciones es de ``apenas'' 57, cuando el escenario se extiende a dos horas, esta diferencia alcanza casi los 500 vehículos. Esto indica que la comunicación tiene un efecto no-despreciable en el flujo vehicular.

\begin{table}[tpb]
    \centering
    \begin{tabular}{@{}l|rrrr@{}}
        \multicolumn{1}{c|}{\textbf{Duración}} & \multicolumn{1}{l}{\begin{tabular}[l]{@{}l@{}}\textbf{Alcanzaron destino}\\ \textbf{con comunicación}\end{tabular}} & \multicolumn{1}{l}{\begin{tabular}[l]{@{}l@{}}\textbf{Alcanzaron destino}\\ \textbf{sin comunicación}\end{tabular}} & \multicolumn{1}{c}{\textbf{$\triangle$}} & \multicolumn{1}{c}{\textbf{\% de mejora}} \\ \midrule
        15 min & 3858 & 3801 & 57 & 1.5\% \\
        120 min & 11488 & 11014 & 474 & 4.3\% \\ \bottomrule
    \end{tabular}
    \caption[Comparación simulaciones de 15 y 120 minutos de duración]{Comparación cantidad de vehículos que alcanzaron su destino con y sin comunicación intervehicular, para simulaciones de 15 minutos y 2 horas, con factor de demanda 100\%.}
    \label{table:2hrssimulation}
\end{table}

Por otro lado, los análisis de eficiencia del sistema de transporte en términos de distancia total y emisión total de dióxido de carbono pueden estudiarse a continuación, en las figuras \ref{fig:distvstime} y \ref{fig:distvsco2}. Estas figuras corresponden a gráficos de dispersión para la comparación de variables claves para los escenarios \emph{con} y \emph{sin} comunicación, y cada punto en el gráfico representa un vehículo en la simulación.

El primero de éstos, el gráfico \ref{fig:distvstime}, ilustra la relación entre distancia total recorrida y tiempo total de viaje para cada uno de los vehículos presentes en las simulaciones de 15 minutos de duración con un factor de demanda del 100\%. Puede notarse que si bien ambos \emph{runs} por lo general presentan un comportamiento similar, en el \emph{run} sin comunicación existen dos grupos con tendencia a presentar un mayor tiempo total en la simulación pero con distancias totales recorridas más bajas de lo esperado. En particular, es muy clara la presencia de un grupo de vehículos que están presentes en prácticamente la duración total de la simulación pero que sin embargo recorren distancias menores a 300 metros. Por otro lado, si bien el escenario con comunicación perfecta presenta una mayor dispersión de sus tiempos totales, no presenta mayores tendencias a largos tiempos de viaje asociados a distancias cortas. 

La interpretación de estos resultados es directa; aquellas ``ramas'' del escenario sin comunicación que tienden hacia tiempos mayores para distancias comparativamente más cortas, corresponden casi seguramente a aquellos vehículos que se ven atascados en la congestión de tráfico generada por el ``accidente'', y que por ende pasan mucho tiempo detenidos o con velocidades muy bajas. Por otro lado, la mayor dispersión de los puntos asociados al escenario con comunicación perfecta se asocia al hecho de que se ``redirige'' a todo vehículo que se estima potencialmente pudiese pasar por la calle afectada -- esto aumenta la distancia y tiempo de viaje para mucho de ellos. Estos es un \emph{tradeoff} para asegurar un mayor flujo vehicular.

\begin{figure}[tpb]
    \centering
    % This file was created by matplotlib2tikz v0.6.10.
\begin{tikzpicture}

\definecolor{color0}{rgb}{0.2,0.8,0.133333333333333}

\begin{axis}[
xlabel={Distancia Total $[m]$},
ylabel={Tiempo Total $[s]$},
xmin=-141.460057014178, xmax=2838.93005701418,
ymin=-47.6666843820862, ymax=944.166684382086,
width=\figurewidth,
height=\figureheight,
tick align=outside,
tick pos=left,
xticklabel style={rotate=45},
xmajorgrids,
x grid style={lightgray, opacity=0.7},
ymajorgrids,
y grid style={lightgray, opacity=0.7},
axis line style={black, opacity=0.0},
legend style={at={(0.97,0.03)}, anchor=south east, draw=white!80.0!black, fill=white!89.803921568627459!black},
legend cell align={left},
legend entries={{Con comunicaci\'on},{Sin comunicaci\'on}}
]
\addplot [only marks, draw=red, mark size=1.0, fill=red, opacity=0.75, colormap/viridis]
table{%
x                      y
+2.789250000000000e+02 +2.100000000000000e+01
+2.751700000000000e+02 +2.050000000000000e+01
+2.827610000000000e+02 +2.050000000000000e+01
+3.244580000000000e+02 +2.250000000000000e+01
+2.927260000000000e+02 +3.550000000000000e+01
+3.345040000000000e+02 +2.450000000000000e+01
+2.774470000000000e+02 +2.100000000000000e+01
+2.799990000000000e+02 +2.100000000000000e+01
+4.920510000000000e+02 +2.950000000000000e+01
+7.761550000000000e+02 +4.700000000000000e+01
+7.702950000000000e+02 +4.750000000000000e+01
+7.883260000000000e+02 +4.800000000000000e+01
+6.929910000000001e+02 +5.200000000000000e+01
+6.937250000000000e+02 +5.050000000000000e+01
+7.652639999999999e+02 +4.650000000000000e+01
+7.750030000000000e+02 +4.700000000000000e+01
+6.769030000000000e+02 +4.500000000000000e+01
+3.374630000000000e+02 +2.450000000000000e+01
+4.942410000000000e+02 +3.850000000000000e+01
+7.652170000000000e+02 +4.550000000000000e+01
+7.680560000000000e+02 +4.750000000000000e+01
+7.729119999999998e+02 +4.650000000000000e+01
+4.536680000000000e+02 +3.350000000000000e+01
+7.615050000000000e+02 +5.450000000000000e+01
+7.695260000000002e+02 +4.750000000000000e+01
+7.503439999999998e+02 +4.550000000000000e+01
+7.516039999999998e+02 +5.500000000000000e+01
+7.820750000000000e+02 +4.850000000000000e+01
+4.916920000000000e+02 +7.000000000000000e+01
+7.782580000000000e+02 +4.700000000000000e+01
+8.424169999999998e+02 +6.050000000000000e+01
+8.793720000000000e+02 +6.250000000000000e+01
+1.083850000000000e+02 +1.250000000000000e+01
+9.149010000000000e+02 +6.750000000000000e+01
+4.859460000000000e+02 +5.950000000000000e+01
+8.701080000000002e+02 +6.150000000000000e+01
+8.813600000000000e+02 +6.400000000000000e+01
+4.922770000000000e+02 +6.350000000000000e+01
+7.707869999999998e+02 +4.550000000000000e+01
+7.743650000000000e+02 +6.100000000000000e+01
+4.937380000000001e+02 +5.650000000000000e+01
+8.909920000000000e+02 +6.400000000000000e+01
+4.910120000000000e+02 +4.550000000000000e+01
+8.859299999999999e+02 +6.600000000000000e+01
+8.765510000000000e+02 +6.650000000000000e+01
+3.368600000000000e+02 +2.400000000000000e+01
+8.571039999999998e+02 +8.350000000000000e+01
+7.774040000000000e+02 +6.800000000000000e+01
+6.599019999999998e+02 +8.450000000000000e+01
+6.250269999999998e+02 +7.200000000000000e+01
+4.870110000000000e+02 +4.500000000000000e+01
+8.392420000000000e+02 +6.100000000000000e+01
+8.554490000000000e+02 +6.700000000000000e+01
+6.028950000000000e+02 +7.250000000000000e+01
+6.405459999999998e+02 +7.850000000000000e+01
+6.657650000000000e+02 +7.450000000000000e+01
+7.727520000000000e+02 +5.550000000000000e+01
+9.011180000000001e+02 +6.300000000000000e+01
+8.678980000000000e+02 +8.750000000000000e+01
+6.511390000000000e+02 +6.950000000000000e+01
+8.638570000000000e+02 +6.100000000000000e+01
+8.826810000000000e+02 +6.450000000000000e+01
+6.699600000000000e+02 +6.950000000000000e+01
+8.406660000000001e+02 +6.000000000000000e+01
+4.980300000000000e+02 +4.000000000000000e+01
+6.808989999999999e+02 +7.850000000000000e+01
+8.693700000000000e+02 +6.150000000000000e+01
+6.372060000000000e+02 +6.200000000000000e+01
+6.556920000000000e+02 +5.700000000000000e+01
+9.252900000000000e+02 +6.500000000000000e+01
+6.403869999999999e+02 +6.000000000000000e+01
+6.367410000000000e+02 +6.100000000000000e+01
+7.528860000000002e+02 +8.600000000000000e+01
+6.645080000000000e+02 +5.750000000000000e+01
+1.078750000000000e+02 +2.000000000000000e+01
+6.794160000000001e+02 +7.750000000000000e+01
+9.343110000000000e+02 +6.400000000000000e+01
+6.427090000000002e+02 +6.300000000000000e+01
+6.051230000000000e+02 +5.050000000000000e+01
+4.959750000000000e+02 +3.900000000000000e+01
+6.612380000000001e+02 +5.300000000000000e+01
+9.510910000000000e+02 +7.050000000000000e+01
+8.795350000000000e+02 +5.950000000000000e+01
+6.311300000000000e+02 +5.700000000000000e+01
+6.514410000000000e+02 +5.200000000000000e+01
+7.530640000000000e+02 +7.700000000000000e+01
+1.082450000000000e+03 +1.000000000000000e+02
+9.943860000000000e+02 +9.300000000000000e+01
+9.092920000000000e+02 +8.900000000000000e+01
+6.336230000000000e+02 +5.050000000000000e+01
+9.018980000000000e+02 +9.700000000000000e+01
+8.776870000000000e+02 +6.150000000000000e+01
+6.679220000000000e+02 +8.200000000000000e+01
+6.742680000000000e+02 +8.950000000000000e+01
+1.005390000000000e+03 +1.040000000000000e+02
+1.004130000000000e+03 +8.150000000000000e+01
+7.538180000000000e+02 +7.450000000000000e+01
+7.544160000000001e+02 +6.250000000000000e+01
+1.153810000000000e+02 +1.300000000000000e+01
+6.746410000000002e+02 +7.900000000000000e+01
+1.017310000000000e+03 +7.250000000000000e+01
+1.061130000000000e+03 +9.450000000000000e+01
+6.876990000000000e+02 +7.450000000000000e+01
+1.059400000000000e+03 +9.800000000000000e+01
+9.873360000000000e+02 +7.200000000000000e+01
+6.854330000000000e+02 +7.350000000000000e+01
+1.076600000000000e+03 +1.085000000000000e+02
+1.061180000000000e+03 +9.500000000000000e+01
+1.082910000000000e+03 +1.000000000000000e+02
+1.083180000000000e+03 +9.650000000000000e+01
+6.779220000000000e+02 +6.200000000000000e+01
+1.035650000000000e+03 +1.225000000000000e+02
+1.045950000000000e+03 +1.065000000000000e+02
+1.113520000000000e+03 +1.205000000000000e+02
+1.120120000000000e+03 +1.115000000000000e+02
+9.784960000000000e+02 +1.245000000000000e+02
+1.033020000000000e+03 +1.120000000000000e+02
+2.722100000000000e+02 +5.900000000000000e+01
+9.970080000000000e+02 +1.050000000000000e+02
+1.168360000000000e+02 +1.100000000000000e+01
+9.931680000000000e+02 +1.275000000000000e+02
+9.726490000000000e+02 +1.200000000000000e+02
+1.000600000000000e+03 +1.280000000000000e+02
+1.007110000000000e+03 +1.220000000000000e+02
+1.003110000000000e+03 +1.115000000000000e+02
+2.741560000000000e+02 +6.150000000000000e+01
+1.008090000000000e+03 +1.265000000000000e+02
+1.029620000000000e+03 +1.130000000000000e+02
+1.131450000000000e+03 +8.650000000000000e+01
+9.890240000000000e+02 +1.055000000000000e+02
+9.837650000000000e+02 +1.260000000000000e+02
+1.055960000000000e+03 +1.100000000000000e+02
+1.195210000000000e+03 +8.400000000000000e+01
+2.818490000000000e+02 +9.350000000000000e+01
+1.126080000000000e+03 +1.135000000000000e+02
+1.281020000000000e+03 +1.220000000000000e+02
+1.361980000000000e+03 +1.330000000000000e+02
+2.819530000000000e+02 +8.600000000000000e+01
+2.771970000000000e+02 +9.250000000000000e+01
+9.760050000000000e+02 +1.165000000000000e+02
+9.976070000000000e+02 +1.135000000000000e+02
+6.616760000000000e+02 +4.550000000000000e+01
+6.867030000000000e+02 +1.130000000000000e+02
+9.034400000000001e+02 +1.220000000000000e+02
+5.886840000000000e+02 +9.150000000000000e+01
+2.761120000000000e+02 +8.500000000000000e+01
+9.929140000000000e+02 +1.130000000000000e+02
+9.569390000000000e+02 +1.030000000000000e+02
+4.580520000000000e+02 +6.650000000000000e+01
+1.333760000000000e+03 +1.305000000000000e+02
+2.769890000000000e+02 +8.200000000000000e+01
+2.810300000000000e+02 +6.700000000000000e+01
+9.172310000000000e+02 +1.215000000000000e+02
+6.220290000000000e+02 +1.320000000000000e+02
+6.205850000000000e+02 +1.335000000000000e+02
+8.542030000000000e+02 +1.225000000000000e+02
+6.209950000000000e+02 +1.235000000000000e+02
+2.842980000000000e+02 +7.600000000000000e+01
+9.735540000000000e+02 +9.850000000000000e+01
+6.226500000000000e+02 +9.850000000000000e+01
+6.613760000000002e+02 +8.400000000000000e+01
+9.097670000000001e+02 +1.230000000000000e+02
+1.056970000000000e+03 +1.150000000000000e+02
+6.668960000000002e+02 +1.260000000000000e+02
+1.037340000000000e+03 +1.320000000000000e+02
+9.808030000000000e+02 +1.030000000000000e+02
+8.797869999999998e+02 +9.300000000000000e+01
+8.846790000000000e+02 +9.550000000000000e+01
+1.063840000000000e+03 +1.405000000000000e+02
+2.834900000000000e+02 +5.350000000000000e+01
+9.209600000000000e+02 +1.180000000000000e+02
+6.252730000000000e+02 +5.950000000000000e+01
+6.245790000000002e+02 +6.800000000000000e+01
+6.606139999999998e+02 +5.400000000000000e+01
+1.077230000000000e+03 +1.330000000000000e+02
+9.745210000000000e+02 +9.950000000000000e+01
+9.109400000000001e+02 +1.095000000000000e+02
+6.214370000000000e+02 +5.050000000000000e+01
+1.018190000000000e+03 +1.095000000000000e+02
+1.056790000000000e+03 +9.000000000000000e+01
+6.224390000000000e+02 +4.950000000000000e+01
+8.732840000000000e+02 +1.080000000000000e+02
+2.700270000000000e+02 +5.100000000000000e+01
+6.062370000000000e+02 +4.900000000000000e+01
+1.112550000000000e+03 +1.225000000000000e+02
+1.230000000000000e+03 +1.275000000000000e+02
+9.072340000000000e+02 +8.750000000000000e+01
+1.107910000000000e+03 +1.155000000000000e+02
+2.991810000000000e+02 +5.050000000000000e+01
+9.125350000000000e+02 +8.950000000000000e+01
+1.115000000000000e+03 +1.025000000000000e+02
+1.084100000000000e+03 +1.050000000000000e+02
+6.327730000000000e+02 +5.300000000000000e+01
+1.063320000000000e+03 +1.160000000000000e+02
+2.940270000000000e+02 +5.500000000000000e+01
+1.309740000000000e+03 +1.325000000000000e+02
+9.093339999999999e+02 +8.150000000000000e+01
+1.022410000000000e+03 +1.090000000000000e+02
+7.862030000000000e+02 +1.040000000000000e+02
+1.289490000000000e+03 +1.335000000000000e+02
+9.998480000000000e+02 +1.440000000000000e+02
+1.405180000000000e+03 +1.480000000000000e+02
+7.652769999999998e+02 +1.160000000000000e+02
+6.256669999999998e+02 +4.750000000000000e+01
+9.064010000000000e+02 +9.600000000000000e+01
+8.499360000000000e+02 +9.300000000000000e+01
+8.389560000000000e+02 +1.205000000000000e+02
+9.213240000000000e+02 +1.510000000000000e+02
+7.681710000000000e+02 +1.155000000000000e+02
+7.684710000000000e+02 +1.075000000000000e+02
+8.658339999999999e+02 +9.350000000000000e+01
+8.795910000000000e+02 +8.600000000000000e+01
+9.142070000000000e+02 +1.420000000000000e+02
+1.321400000000000e+03 +1.325000000000000e+02
+7.450100000000000e+02 +1.000000000000000e+02
+4.463610000000000e+02 +8.450000000000000e+01
+7.751060000000001e+02 +1.115000000000000e+02
+1.306300000000000e+03 +1.320000000000000e+02
+7.745960000000000e+02 +1.175000000000000e+02
+7.673850000000000e+02 +1.140000000000000e+02
+8.772430000000001e+02 +8.800000000000000e+01
+8.624530000000000e+02 +1.240000000000000e+02
+3.224100000000000e+02 +8.550000000000000e+01
+8.868789999999998e+02 +1.290000000000000e+02
+7.563610000000001e+02 +1.080000000000000e+02
+7.664480000000000e+02 +8.050000000000000e+01
+1.411260000000000e+03 +1.445000000000000e+02
+7.724630000000002e+02 +1.090000000000000e+02
+1.316050000000000e+03 +1.510000000000000e+02
+7.644370000000000e+02 +1.075000000000000e+02
+8.930740000000000e+02 +8.800000000000000e+01
+6.927239999999998e+02 +1.450000000000000e+02
+3.251410000000000e+02 +8.550000000000000e+01
+6.881130000000001e+02 +1.335000000000000e+02
+8.044800000000000e+02 +1.025000000000000e+02
+7.773270000000000e+02 +1.075000000000000e+02
+7.713680000000001e+02 +9.450000000000000e+01
+8.394560000000000e+02 +8.050000000000000e+01
+7.708620000000000e+02 +1.050000000000000e+02
+3.332760000000000e+02 +6.800000000000000e+01
+6.939380000000000e+02 +1.295000000000000e+02
+7.563689999999998e+02 +1.025000000000000e+02
+7.790549999999999e+02 +1.075000000000000e+02
+8.559760000000001e+02 +8.450000000000000e+01
+8.614190000000000e+02 +1.060000000000000e+02
+7.652080000000002e+02 +1.015000000000000e+02
+9.129770000000000e+02 +1.285000000000000e+02
+9.006480000000000e+02 +8.850000000000000e+01
+8.081730000000000e+02 +9.950000000000000e+01
+7.810250000000000e+02 +1.065000000000000e+02
+6.617710000000002e+02 +1.005000000000000e+02
+7.691210000000002e+02 +1.005000000000000e+02
+1.296930000000000e+03 +1.395000000000000e+02
+4.800760000000000e+02 +8.400000000000000e+01
+7.874839999999998e+02 +1.035000000000000e+02
+7.845210000000002e+02 +9.800000000000000e+01
+8.853380000000002e+02 +8.650000000000000e+01
+8.622980000000000e+02 +1.025000000000000e+02
+6.925230000000000e+02 +1.090000000000000e+02
+7.458620000000000e+02 +1.285000000000000e+02
+7.669240000000000e+02 +8.700000000000000e+01
+8.965960000000000e+02 +8.600000000000000e+01
+4.940560000000000e+02 +5.400000000000000e+01
+6.918860000000002e+02 +1.360000000000000e+02
+9.034990000000000e+02 +1.425000000000000e+02
+6.616050000000000e+02 +9.600000000000000e+01
+7.519169999999998e+02 +9.900000000000000e+01
+7.669720000000000e+02 +9.600000000000000e+01
+4.864470000000000e+02 +6.450000000000000e+01
+1.300960000000000e+03 +1.295000000000000e+02
+7.717700000000000e+02 +8.150000000000000e+01
+7.833930000000000e+02 +9.800000000000000e+01
+1.294660000000000e+03 +1.210000000000000e+02
+7.753550000000000e+02 +8.850000000000000e+01
+6.971100000000000e+02 +1.100000000000000e+02
+9.478620000000000e+02 +1.180000000000000e+02
+4.756930000000000e+02 +3.150000000000000e+01
+7.477020000000000e+02 +1.170000000000000e+02
+7.798380000000002e+02 +8.200000000000000e+01
+6.948639999999998e+02 +7.850000000000000e+01
+1.091750000000000e+03 +1.595000000000000e+02
+9.005039999999998e+02 +8.950000000000000e+01
+9.850130000000000e+02 +1.365000000000000e+02
+7.895139999999999e+02 +8.450000000000000e+01
+7.835410000000001e+02 +7.900000000000000e+01
+7.799080000000000e+02 +8.150000000000000e+01
+1.420550000000000e+03 +1.655000000000000e+02
+6.949750000000000e+02 +1.190000000000000e+02
+4.824170000000000e+02 +3.150000000000000e+01
+9.754500000000000e+02 +1.180000000000000e+02
+8.630560000000000e+02 +1.185000000000000e+02
+6.977500000000000e+02 +1.060000000000000e+02
+1.094000000000000e+03 +1.365000000000000e+02
+1.315300000000000e+03 +1.250000000000000e+02
+9.775520000000000e+02 +1.180000000000000e+02
+6.752310000000001e+02 +1.200000000000000e+02
+1.316370000000000e+03 +1.685000000000000e+02
+7.852170000000000e+02 +8.150000000000000e+01
+9.388960000000000e+02 +1.485000000000000e+02
+7.564780000000002e+02 +9.000000000000000e+01
+6.896230000000000e+02 +1.255000000000000e+02
+7.690700000000001e+02 +7.350000000000000e+01
+9.066620000000000e+02 +8.200000000000000e+01
+3.338320000000000e+02 +2.550000000000000e+01
+1.297830000000000e+03 +1.580000000000000e+02
+4.852120000000000e+02 +3.300000000000000e+01
+6.801230000000000e+02 +9.850000000000000e+01
+1.128730000000000e+03 +1.695000000000000e+02
+7.831380000000000e+02 +7.300000000000000e+01
+1.295870000000000e+03 +1.545000000000000e+02
+7.885830000000002e+02 +8.650000000000000e+01
+6.939789999999998e+02 +7.300000000000000e+01
+7.531920000000000e+02 +9.550000000000000e+01
+8.991519999999998e+02 +7.750000000000000e+01
+7.651389999999999e+02 +7.700000000000000e+01
+8.432230000000002e+02 +8.700000000000000e+01
+9.775950000000000e+02 +1.145000000000000e+02
+9.783480000000000e+02 +1.100000000000000e+02
+1.125690000000000e+03 +1.545000000000000e+02
+7.657520000000000e+02 +6.850000000000000e+01
+5.756210000000000e+02 +7.000000000000000e+01
+1.294110000000000e+03 +1.385000000000000e+02
+6.909390000000000e+02 +6.950000000000000e+01
+7.753450000000000e+02 +6.750000000000000e+01
+7.666170000000000e+02 +1.510000000000000e+02
+9.897880000000000e+02 +1.135000000000000e+02
+7.487020000000000e+02 +8.500000000000000e+01
+7.534100000000000e+02 +8.450000000000000e+01
+8.732050000000000e+02 +9.150000000000000e+01
+7.816569999999998e+02 +7.750000000000000e+01
+7.552239999999998e+02 +1.110000000000000e+02
+7.837439999999998e+02 +6.800000000000000e+01
+7.795350000000000e+02 +7.650000000000000e+01
+8.685360000000002e+02 +1.110000000000000e+02
+6.941239999999998e+02 +8.750000000000000e+01
+9.080270000000000e+02 +7.800000000000000e+01
+9.458560000000000e+02 +7.900000000000000e+01
+7.722239999999998e+02 +7.000000000000000e+01
+7.661880000000000e+02 +6.500000000000000e+01
+7.677070000000000e+02 +6.200000000000000e+01
+7.816020000000000e+02 +6.950000000000000e+01
+4.907540000000000e+02 +4.050000000000000e+01
+1.140290000000000e+03 +1.595000000000000e+02
+6.944530000000000e+02 +1.110000000000000e+02
+7.827160000000000e+02 +6.500000000000000e+01
+6.811880000000000e+02 +7.700000000000000e+01
+3.349040000000000e+02 +2.550000000000000e+01
+4.898910000000000e+02 +1.135000000000000e+02
+9.351220000000000e+02 +8.900000000000000e+01
+7.485770000000000e+02 +1.560000000000000e+02
+7.472550000000000e+02 +7.900000000000000e+01
+7.795490000000000e+02 +6.200000000000000e+01
+7.060740000000000e+02 +9.200000000000000e+01
+1.149660000000000e+03 +1.510000000000000e+02
+6.873520000000000e+02 +1.645000000000000e+02
+6.595980000000002e+02 +5.550000000000000e+01
+7.471100000000000e+02 +6.300000000000000e+01
+4.822420000000000e+02 +1.060000000000000e+02
+7.703670000000000e+02 +6.150000000000000e+01
+7.756389999999999e+02 +6.150000000000000e+01
+7.798680000000001e+02 +6.000000000000000e+01
+6.798819999999999e+02 +1.545000000000000e+02
+4.912730000000000e+02 +9.350000000000000e+01
+7.508360000000000e+02 +6.100000000000000e+01
+7.780069999999999e+02 +5.750000000000000e+01
+8.938839999999999e+02 +1.495000000000000e+02
+1.312460000000000e+03 +1.290000000000000e+02
+7.683090000000000e+02 +5.600000000000000e+01
+7.447130000000002e+02 +5.350000000000000e+01
+4.895660000000000e+02 +1.005000000000000e+02
+7.764430000000000e+02 +5.600000000000000e+01
+7.762719999999998e+02 +5.350000000000000e+01
+7.590330000000000e+02 +1.475000000000000e+02
+6.850150000000000e+02 +1.555000000000000e+02
+1.121130000000000e+02 +1.400000000000000e+01
+4.859700000000000e+02 +8.250000000000000e+01
+6.930480000000000e+02 +1.315000000000000e+02
+7.541100000000000e+02 +1.565000000000000e+02
+8.807900000000000e+02 +7.450000000000000e+01
+4.896330000000000e+02 +8.600000000000000e+01
+6.823560000000001e+02 +1.530000000000000e+02
+7.845030000000000e+02 +5.900000000000000e+01
+7.706220000000000e+02 +4.800000000000000e+01
+8.994750000000000e+02 +7.750000000000000e+01
+4.889190000000000e+02 +7.500000000000000e+01
+7.595950000000000e+02 +1.405000000000000e+02
+6.313009999999998e+02 +9.000000000000000e+01
+9.701990000000000e+02 +1.115000000000000e+02
+1.106650000000000e+03 +1.430000000000000e+02
+6.806880000000000e+02 +1.560000000000000e+02
+7.629580000000002e+02 +4.550000000000000e+01
+7.804639999999998e+02 +1.020000000000000e+02
+7.545500000000000e+02 +7.100000000000000e+01
+6.429530000000000e+02 +1.845000000000000e+02
+4.864290000000000e+02 +7.800000000000000e+01
+7.728770000000000e+02 +4.800000000000000e+01
+7.553610000000001e+02 +7.550000000000000e+01
+8.582970000000000e+02 +6.900000000000000e+01
+4.880160000000000e+02 +7.250000000000000e+01
+6.789600000000000e+02 +1.440000000000000e+02
+4.915810000000000e+02 +6.550000000000000e+01
+1.393260000000000e+03 +1.865000000000000e+02
+1.349740000000000e+03 +1.880000000000000e+02
+1.096110000000000e+03 +1.510000000000000e+02
+6.744980000000000e+02 +1.040000000000000e+02
+1.561150000000000e+03 +1.920000000000000e+02
+1.718540000000000e+03 +1.925000000000000e+02
+4.856900000000000e+02 +5.100000000000000e+01
+1.289110000000000e+03 +1.795000000000000e+02
+6.802900000000000e+02 +9.700000000000000e+01
+1.547940000000000e+03 +1.805000000000000e+02
+1.322320000000000e+03 +1.850000000000000e+02
+1.100150000000000e+03 +1.710000000000000e+02
+4.919410000000000e+02 +6.950000000000000e+01
+1.555860000000000e+03 +1.980000000000000e+02
+1.552270000000000e+03 +1.555000000000000e+02
+7.506139999999998e+02 +9.350000000000000e+01
+6.856400000000000e+02 +1.345000000000000e+02
+6.927280000000002e+02 +1.310000000000000e+02
+6.739550000000000e+02 +1.430000000000000e+02
+9.893690000000000e+02 +1.585000000000000e+02
+8.915089999999999e+02 +1.550000000000000e+02
+8.934250000000000e+02 +1.275000000000000e+02
+1.569290000000000e+03 +1.770000000000000e+02
+7.611139999999998e+02 +7.050000000000000e+01
+6.763070000000000e+02 +1.375000000000000e+02
+6.659150000000000e+02 +1.325000000000000e+02
+6.353040000000000e+02 +1.525000000000000e+02
+5.027070000000000e+02 +5.450000000000000e+01
+6.793869999999999e+02 +8.600000000000000e+01
+9.884400000000001e+02 +1.520000000000000e+02
+1.574100000000000e+03 +1.935000000000000e+02
+6.727919999999998e+02 +1.195000000000000e+02
+4.952050000000000e+02 +5.900000000000000e+01
+9.300359999999999e+02 +1.560000000000000e+02
+6.550770000000000e+02 +1.290000000000000e+02
+6.608510000000001e+02 +1.385000000000000e+02
+6.333490000000000e+02 +1.545000000000000e+02
+6.976860000000000e+02 +1.210000000000000e+02
+1.150560000000000e+03 +1.745000000000000e+02
+6.190930000000002e+02 +1.270000000000000e+02
+9.334200000000000e+02 +1.140000000000000e+02
+6.528960000000000e+02 +1.775000000000000e+02
+6.366450000000000e+02 +1.520000000000000e+02
+6.338330000000002e+02 +1.480000000000000e+02
+6.671830000000000e+02 +1.285000000000000e+02
+1.859400000000000e+03 +1.930000000000000e+02
+1.036370000000000e+03 +1.625000000000000e+02
+6.649370000000000e+02 +1.760000000000000e+02
+1.058910000000000e+03 +1.765000000000000e+02
+1.108120000000000e+03 +1.755000000000000e+02
+1.148060000000000e+03 +1.400000000000000e+02
+6.473910000000000e+02 +1.510000000000000e+02
+6.571550000000000e+02 +1.495000000000000e+02
+6.482530000000000e+02 +1.190000000000000e+02
+6.193060000000000e+02 +1.045000000000000e+02
+6.844290000000000e+02 +1.550000000000000e+02
+6.203460000000000e+02 +1.380000000000000e+02
+1.084310000000000e+02 +2.000000000000000e+01
+6.173540000000000e+02 +1.490000000000000e+02
+6.521060000000000e+02 +1.015000000000000e+02
+1.440040000000000e+03 +2.040000000000000e+02
+8.974910000000001e+02 +8.550000000000000e+01
+6.396130000000001e+02 +1.460000000000000e+02
+6.502170000000000e+02 +1.440000000000000e+02
+6.743570000000000e+02 +1.015000000000000e+02
+6.417900000000000e+02 +1.170000000000000e+02
+1.475220000000000e+03 +2.010000000000000e+02
+6.868360000000000e+02 +1.350000000000000e+02
+8.786260000000002e+02 +9.500000000000000e+01
+6.542210000000000e+02 +1.300000000000000e+02
+6.624560000000000e+02 +1.470000000000000e+02
+1.038500000000000e+03 +1.505000000000000e+02
+6.534620000000000e+02 +7.000000000000000e+01
+8.966810000000000e+02 +7.700000000000000e+01
+1.523430000000000e+03 +2.120000000000000e+02
+1.100330000000000e+02 +2.500000000000000e+01
+6.756139999999998e+02 +1.390000000000000e+02
+6.459910000000000e+02 +1.430000000000000e+02
+6.639230000000000e+02 +1.325000000000000e+02
+1.033870000000000e+03 +1.360000000000000e+02
+1.518630000000000e+03 +1.960000000000000e+02
+1.365120000000000e+03 +2.040000000000000e+02
+1.158710000000000e+03 +1.845000000000000e+02
+6.369340000000000e+02 +5.800000000000000e+01
+6.154080000000000e+02 +1.180000000000000e+02
+1.544140000000000e+03 +2.160000000000000e+02
+1.045370000000000e+03 +1.525000000000000e+02
+1.486080000000000e+03 +1.985000000000000e+02
+6.714340000000000e+02 +1.420000000000000e+02
+2.601390000000000e+02 +1.020000000000000e+02
+1.375450000000000e+03 +1.960000000000000e+02
+6.573819999999999e+02 +1.150000000000000e+02
+6.476970000000000e+02 +1.000000000000000e+02
+6.882130000000002e+02 +9.300000000000000e+01
+6.612880000000000e+02 +1.415000000000000e+02
+6.596780000000000e+02 +6.950000000000000e+01
+6.060920000000000e+02 +1.330000000000000e+02
+6.357480000000000e+02 +9.500000000000000e+01
+1.008110000000000e+03 +1.715000000000000e+02
+1.134130000000000e+03 +1.945000000000000e+02
+6.385090000000000e+02 +6.000000000000000e+01
+9.058440000000001e+02 +6.950000000000000e+01
+1.005690000000000e+03 +1.670000000000000e+02
+6.405400000000000e+02 +9.450000000000000e+01
+6.515710000000000e+02 +1.380000000000000e+02
+8.372339999999998e+02 +1.020000000000000e+02
+1.001170000000000e+03 +1.595000000000000e+02
+1.155440000000000e+03 +2.105000000000000e+02
+6.576519999999998e+02 +8.300000000000000e+01
+1.069470000000000e+03 +1.655000000000000e+02
+6.827530000000000e+02 +1.285000000000000e+02
+6.561770000000000e+02 +1.125000000000000e+02
+8.963389999999998e+02 +1.125000000000000e+02
+1.072620000000000e+03 +1.890000000000000e+02
+6.565730000000000e+02 +8.500000000000000e+01
+1.112390000000000e+03 +1.605000000000000e+02
+1.251470000000000e+03 +2.085000000000000e+02
+6.629850000000000e+02 +1.385000000000000e+02
+8.910110000000002e+02 +1.110000000000000e+02
+6.344240000000000e+02 +9.100000000000000e+01
+9.969480000000000e+02 +1.230000000000000e+02
+1.271830000000000e+03 +1.730000000000000e+02
+1.322910000000000e+03 +1.455000000000000e+02
+6.832600000000000e+02 +1.170000000000000e+02
+1.078230000000000e+03 +1.715000000000000e+02
+2.738260000000000e+02 +8.950000000000000e+01
+1.083360000000000e+03 +1.465000000000000e+02
+6.581930000000000e+02 +1.360000000000000e+02
+6.451080000000002e+02 +9.450000000000000e+01
+9.929980000000000e+02 +1.365000000000000e+02
+1.050970000000000e+03 +2.080000000000000e+02
+1.074230000000000e+03 +2.045000000000000e+02
+9.059829999999999e+02 +1.115000000000000e+02
+1.065400000000000e+03 +1.700000000000000e+02
+1.341560000000000e+03 +1.365000000000000e+02
+6.730880000000002e+02 +1.310000000000000e+02
+9.006189999999998e+02 +1.090000000000000e+02
+1.069040000000000e+03 +1.955000000000000e+02
+6.357940000000000e+02 +1.235000000000000e+02
+1.116720000000000e+03 +1.660000000000000e+02
+1.242070000000000e+03 +1.460000000000000e+02
+6.367150000000000e+02 +1.185000000000000e+02
+1.101900000000000e+03 +1.525000000000000e+02
+9.894230000000000e+02 +1.825000000000000e+02
+1.571790000000000e+03 +1.750000000000000e+02
+1.083660000000000e+03 +2.085000000000000e+02
+1.079160000000000e+03 +1.940000000000000e+02
+1.066100000000000e+03 +2.115000000000000e+02
+1.112200000000000e+03 +1.740000000000000e+02
+1.063510000000000e+03 +1.315000000000000e+02
+1.644840000000000e+03 +1.650000000000000e+02
+1.003520000000000e+03 +1.725000000000000e+02
+1.052090000000000e+03 +1.690000000000000e+02
+6.378640000000000e+02 +8.150000000000000e+01
+1.042870000000000e+03 +1.345000000000000e+02
+1.073070000000000e+03 +1.820000000000000e+02
+1.058260000000000e+03 +1.830000000000000e+02
+6.407130000000002e+02 +1.130000000000000e+02
+1.106910000000000e+03 +1.495000000000000e+02
+2.711850000000000e+02 +8.250000000000000e+01
+6.803190000000000e+02 +1.195000000000000e+02
+1.078600000000000e+03 +1.415000000000000e+02
+1.080370000000000e+03 +1.540000000000000e+02
+6.389150000000000e+02 +9.200000000000000e+01
+9.017670000000001e+02 +1.130000000000000e+02
+1.069270000000000e+03 +1.230000000000000e+02
+1.045150000000000e+03 +1.255000000000000e+02
+1.568560000000000e+03 +1.735000000000000e+02
+1.065480000000000e+03 +2.040000000000000e+02
+1.154520000000000e+03 +1.930000000000000e+02
+1.363140000000000e+03 +2.070000000000000e+02
+6.565230000000000e+02 +1.390000000000000e+02
+1.114770000000000e+03 +1.100000000000000e+02
+6.774040000000000e+02 +1.435000000000000e+02
+1.051410000000000e+03 +1.705000000000000e+02
+9.779750000000000e+02 +1.670000000000000e+02
+9.103060000000000e+02 +1.135000000000000e+02
+1.074870000000000e+03 +1.730000000000000e+02
+9.754450000000001e+02 +1.650000000000000e+02
+6.720069999999999e+02 +9.800000000000000e+01
+6.569730000000002e+02 +1.190000000000000e+02
+8.363240000000000e+02 +9.000000000000000e+01
+1.066510000000000e+03 +1.430000000000000e+02
+1.334640000000000e+03 +2.250000000000000e+02
+1.008140000000000e+03 +1.620000000000000e+02
+8.813420000000000e+02 +1.050000000000000e+02
+1.070020000000000e+03 +1.975000000000000e+02
+1.035320000000000e+03 +1.585000000000000e+02
+1.215990000000000e+03 +1.800000000000000e+02
+9.090460000000000e+02 +1.145000000000000e+02
+8.630369999999998e+02 +1.070000000000000e+02
+1.781850000000000e+03 +2.345000000000000e+02
+1.048800000000000e+03 +1.450000000000000e+02
+1.080970000000000e+03 +1.525000000000000e+02
+1.303840000000000e+03 +2.325000000000000e+02
+1.215670000000000e+03 +2.090000000000000e+02
+1.367530000000000e+03 +2.175000000000000e+02
+8.622930000000000e+02 +1.830000000000000e+02
+8.705910000000000e+02 +1.175000000000000e+02
+8.743049999999999e+02 +1.150000000000000e+02
+1.369960000000000e+03 +2.375000000000000e+02
+7.817389999999998e+02 +1.775000000000000e+02
+7.761680000000000e+02 +1.850000000000000e+02
+9.989560000000000e+02 +1.595000000000000e+02
+1.073050000000000e+03 +1.410000000000000e+02
+9.959030000000000e+02 +1.760000000000000e+02
+8.626820000000000e+02 +1.780000000000000e+02
+7.841319999999999e+02 +1.115000000000000e+02
+1.855480000000000e+03 +2.505000000000000e+02
+9.992940000000000e+02 +1.915000000000000e+02
+8.530930000000002e+02 +1.055000000000000e+02
+9.105990000000000e+02 +1.045000000000000e+02
+9.273869999999999e+02 +1.200000000000000e+02
+1.288470000000000e+03 +1.855000000000000e+02
+1.071580000000000e+03 +1.645000000000000e+02
+9.893940000000000e+02 +1.570000000000000e+02
+9.789890000000000e+02 +1.590000000000000e+02
+9.952020000000000e+02 +1.985000000000000e+02
+1.000240000000000e+03 +1.790000000000000e+02
+1.049500000000000e+03 +1.750000000000000e+02
+8.673830000000000e+02 +2.205000000000000e+02
+4.744770000000000e+02 +7.850000000000000e+01
+9.576130000000001e+02 +1.380000000000000e+02
+9.624520000000000e+02 +1.555000000000000e+02
+9.707790000000000e+02 +1.495000000000000e+02
+1.767740000000000e+03 +2.040000000000000e+02
+1.006500000000000e+03 +1.685000000000000e+02
+9.695410000000001e+02 +1.460000000000000e+02
+2.818990000000000e+02 +6.250000000000000e+01
+8.675650000000001e+02 +1.725000000000000e+02
+9.862000000000000e+02 +1.760000000000000e+02
+1.010930000000000e+03 +1.635000000000000e+02
+1.072290000000000e+03 +1.410000000000000e+02
+1.132600000000000e+03 +2.020000000000000e+02
+8.738630000000001e+02 +1.985000000000000e+02
+9.732569999999999e+02 +1.525000000000000e+02
+4.776820000000000e+02 +7.600000000000000e+01
+9.937260000000000e+02 +1.375000000000000e+02
+2.766260000000000e+02 +5.200000000000000e+01
+7.893950000000000e+02 +1.740000000000000e+02
+9.944160000000001e+02 +1.260000000000000e+02
+9.969510000000000e+02 +1.255000000000000e+02
+8.716720000000000e+02 +1.365000000000000e+02
+1.128970000000000e+03 +1.930000000000000e+02
+1.078040000000000e+03 +1.770000000000000e+02
+9.793140000000000e+02 +1.560000000000000e+02
+1.000680000000000e+03 +1.515000000000000e+02
+9.990839999999999e+02 +1.275000000000000e+02
+1.406640000000000e+03 +2.465000000000000e+02
+7.918130000000000e+02 +1.730000000000000e+02
+8.614180000000000e+02 +9.800000000000000e+01
+9.014410000000000e+02 +9.550000000000000e+01
+1.407000000000000e+03 +2.325000000000000e+02
+9.933530000000000e+02 +1.315000000000000e+02
+9.069570000000000e+02 +1.825000000000000e+02
+2.774420000000000e+02 +4.000000000000000e+01
+1.092470000000000e+03 +1.425000000000000e+02
+8.487940000000000e+02 +1.695000000000000e+02
+1.003640000000000e+03 +1.565000000000000e+02
+1.045520000000000e+02 +9.000000000000000e+00
+6.182840000000000e+02 +1.395000000000000e+02
+8.078560000000001e+02 +1.375000000000000e+02
+1.861630000000000e+03 +2.520000000000000e+02
+8.709660000000000e+02 +1.010000000000000e+02
+1.416610000000000e+03 +2.380000000000000e+02
+9.892740000000000e+02 +1.485000000000000e+02
+9.976410000000000e+02 +1.525000000000000e+02
+1.299640000000000e+03 +1.890000000000000e+02
+8.933450000000000e+02 +1.310000000000000e+02
+6.185010000000000e+02 +1.390000000000000e+02
+1.047620000000000e+03 +1.560000000000000e+02
+8.955870000000000e+02 +1.045000000000000e+02
+1.392550000000000e+03 +2.215000000000000e+02
+9.860470000000000e+02 +1.390000000000000e+02
+6.573560000000001e+02 +1.350000000000000e+02
+9.258730000000000e+02 +9.650000000000000e+01
+6.256630000000000e+02 +1.290000000000000e+02
+6.213270000000000e+02 +1.190000000000000e+02
+1.065530000000000e+03 +1.675000000000000e+02
+1.105600000000000e+03 +2.225000000000000e+02
+9.787480000000000e+02 +1.305000000000000e+02
+9.779580000000000e+02 +1.295000000000000e+02
+6.301110000000000e+02 +1.150000000000000e+02
+1.414820000000000e+03 +2.170000000000000e+02
+6.293930000000000e+02 +1.185000000000000e+02
+1.083240000000000e+03 +1.390000000000000e+02
+6.617180000000002e+02 +1.280000000000000e+02
+1.126530000000000e+03 +1.930000000000000e+02
+1.189700000000000e+03 +1.325000000000000e+02
+1.282820000000000e+03 +2.300000000000000e+02
+9.248040000000000e+02 +1.115000000000000e+02
+6.209760000000000e+02 +1.160000000000000e+02
+6.636419999999998e+02 +1.245000000000000e+02
+1.290160000000000e+03 +2.365000000000000e+02
+9.942520000000000e+02 +1.490000000000000e+02
+6.317940000000000e+02 +9.900000000000000e+01
+2.782850000000000e+02 +2.050000000000000e+01
+1.286990000000000e+03 +2.095000000000000e+02
+9.312170000000000e+02 +8.750000000000000e+01
+6.203140000000000e+02 +8.150000000000000e+01
+8.948670000000000e+02 +9.800000000000000e+01
+6.241400000000000e+02 +1.030000000000000e+02
+1.463780000000000e+03 +2.370000000000000e+02
+9.587390000000000e+02 +1.180000000000000e+02
+6.202950000000000e+02 +8.900000000000000e+01
+6.683660000000001e+02 +1.295000000000000e+02
+4.570590000000000e+02 +3.450000000000000e+01
+1.679330000000000e+03 +2.465000000000000e+02
+6.713650000000000e+02 +1.280000000000000e+02
+2.855090000000000e+02 +2.150000000000000e+01
+9.064000000000000e+02 +1.060000000000000e+02
+8.921980000000000e+02 +9.650000000000000e+01
+9.903170000000000e+02 +1.215000000000000e+02
+1.471300000000000e+03 +1.805000000000000e+02
+1.853750000000000e+03 +2.575000000000000e+02
+6.218070000000000e+02 +9.750000000000000e+01
+4.513120000000000e+02 +3.500000000000000e+01
+1.048940000000000e+03 +1.385000000000000e+02
+1.306340000000000e+03 +2.395000000000000e+02
+9.773390000000001e+02 +1.190000000000000e+02
+9.092460000000000e+02 +9.300000000000000e+01
+9.920250000000000e+02 +1.205000000000000e+02
+6.229109999999999e+02 +7.850000000000000e+01
+6.807880000000000e+02 +1.305000000000000e+02
+7.913819999999999e+02 +1.065000000000000e+02
+8.071160000000001e+02 +1.835000000000000e+02
+1.297890000000000e+03 +1.970000000000000e+02
+1.470230000000000e+03 +1.540000000000000e+02
+1.866930000000000e+03 +2.490000000000000e+02
+8.923500000000000e+02 +8.350000000000000e+01
+1.036800000000000e+03 +1.365000000000000e+02
+6.629200000000000e+02 +1.190000000000000e+02
+1.616920000000000e+03 +2.615000000000000e+02
+8.546600000000000e+02 +8.750000000000000e+01
+7.711860000000000e+02 +1.305000000000000e+02
+1.476300000000000e+03 +1.955000000000000e+02
+9.761200000000000e+02 +1.130000000000000e+02
+1.867420000000000e+03 +2.575000000000000e+02
+7.746720000000000e+02 +1.190000000000000e+02
+7.843370000000000e+02 +1.220000000000000e+02
+6.630790000000000e+02 +1.160000000000000e+02
+1.847880000000000e+03 +2.480000000000000e+02
+7.715770000000000e+02 +1.140000000000000e+02
+7.763930000000000e+02 +1.180000000000000e+02
+7.728850000000000e+02 +1.275000000000000e+02
+1.596540000000000e+03 +2.525000000000000e+02
+1.326600000000000e+03 +1.805000000000000e+02
+1.862580000000000e+03 +2.370000000000000e+02
+1.626420000000000e+03 +2.515000000000000e+02
+9.780170000000001e+02 +1.145000000000000e+02
+7.655939999999998e+02 +1.170000000000000e+02
+7.685119999999999e+02 +1.130000000000000e+02
+8.992970000000000e+02 +9.500000000000000e+01
+6.597739999999999e+02 +1.150000000000000e+02
+1.628500000000000e+03 +2.240000000000000e+02
+7.640610000000000e+02 +1.035000000000000e+02
+6.948070000000000e+02 +1.470000000000000e+02
+6.880690000000000e+02 +1.575000000000000e+02
+9.729930000000001e+02 +2.145000000000000e+02
+1.283130000000000e+03 +1.775000000000000e+02
+1.848620000000000e+03 +2.230000000000000e+02
+9.422560000000000e+02 +2.305000000000000e+02
+1.868570000000000e+03 +2.355000000000000e+02
+7.619220000000000e+02 +1.015000000000000e+02
+8.803080000000000e+02 +8.450000000000000e+01
+9.621390000000000e+02 +1.145000000000000e+02
+7.765860000000000e+02 +1.155000000000000e+02
+9.020960000000000e+02 +8.800000000000000e+01
+7.645230000000000e+02 +1.350000000000000e+02
+1.303490000000000e+03 +1.690000000000000e+02
+7.800830000000002e+02 +8.100000000000000e+01
+8.868580000000002e+02 +9.300000000000000e+01
+6.981580000000000e+02 +1.700000000000000e+02
+6.937189999999998e+02 +1.735000000000000e+02
+7.835930000000002e+02 +7.900000000000000e+01
+8.049750000000000e+02 +1.975000000000000e+02
+7.856960000000000e+02 +1.155000000000000e+02
+3.242440000000000e+02 +6.300000000000000e+01
+7.864220000000000e+02 +1.170000000000000e+02
+7.872210000000000e+02 +1.155000000000000e+02
+2.881940000000000e+02 +2.100000000000000e+01
+6.941280000000000e+02 +1.210000000000000e+02
+1.292100000000000e+03 +1.655000000000000e+02
+7.712110000000000e+02 +1.130000000000000e+02
+9.284299999999999e+02 +1.980000000000000e+02
+9.891130000000001e+02 +1.155000000000000e+02
+3.944520000000000e+02 +3.000000000000000e+01
+7.831039999999998e+02 +1.050000000000000e+02
+1.293640000000000e+03 +2.015000000000000e+02
+8.712680000000000e+02 +8.350000000000000e+01
+6.896480000000000e+02 +1.155000000000000e+02
+7.809680000000002e+02 +1.085000000000000e+02
+8.143750000000000e+02 +1.445000000000000e+02
+6.932639999999999e+02 +1.670000000000000e+02
+7.904889999999998e+02 +1.035000000000000e+02
+9.033560000000000e+02 +8.750000000000000e+01
+4.752800000000000e+02 +8.850000000000000e+01
+7.523230000000000e+02 +1.060000000000000e+02
+6.912950000000000e+02 +1.640000000000000e+02
+7.897760000000002e+02 +9.250000000000000e+01
+1.189600000000000e+03 +1.420000000000000e+02
+7.685670000000000e+02 +9.400000000000000e+01
+7.933339999999999e+02 +1.110000000000000e+02
+7.870700000000001e+02 +8.450000000000000e+01
+2.307360000000000e+03 +2.775000000000000e+02
+6.932810000000002e+02 +1.390000000000000e+02
+1.347140000000000e+03 +1.610000000000000e+02
+7.434900000000000e+02 +1.195000000000000e+02
+6.078520000000000e+02 +7.350000000000000e+01
+1.307610000000000e+03 +1.825000000000000e+02
+7.416180000000001e+02 +1.020000000000000e+02
+1.355970000000000e+03 +1.985000000000000e+02
+1.433300000000000e+03 +1.980000000000000e+02
+7.942320000000000e+02 +1.110000000000000e+02
+1.688000000000000e+03 +2.610000000000000e+02
+1.561180000000000e+03 +2.195000000000000e+02
+6.920780000000000e+02 +5.900000000000000e+01
+1.083040000000000e+02 +1.000000000000000e+01
+7.609610000000000e+02 +8.500000000000000e+01
+1.566360000000000e+03 +1.575000000000000e+02
+7.474069999999998e+02 +9.100000000000000e+01
+1.301220000000000e+03 +2.170000000000000e+02
+1.310550000000000e+03 +1.820000000000000e+02
+7.887210000000000e+02 +8.650000000000000e+01
+7.875560000000000e+02 +1.110000000000000e+02
+8.580080000000000e+02 +1.160000000000000e+02
+8.929030000000000e+02 +1.725000000000000e+02
+4.295350000000000e+02 +3.250000000000000e+01
+1.305180000000000e+03 +2.085000000000000e+02
+6.920889999999998e+02 +1.470000000000000e+02
+1.310780000000000e+03 +1.920000000000000e+02
+7.744910000000001e+02 +8.300000000000000e+01
+9.082310000000000e+02 +2.090000000000000e+02
+9.775800000000000e+02 +1.140000000000000e+02
+7.727980000000000e+02 +1.010000000000000e+02
+4.816240000000000e+02 +1.200000000000000e+02
+6.939860000000001e+02 +1.295000000000000e+02
+8.522990000000000e+02 +1.665000000000000e+02
+1.311360000000000e+03 +2.075000000000000e+02
+1.177910000000000e+02 +1.300000000000000e+01
+7.493670000000000e+02 +8.900000000000000e+01
+7.833589999999998e+02 +8.250000000000000e+01
+7.831080000000002e+02 +1.015000000000000e+02
+6.902560000000002e+02 +8.050000000000000e+01
+1.563110000000000e+03 +2.080000000000000e+02
+7.870030000000000e+02 +7.950000000000000e+01
+4.633770000000000e+02 +7.000000000000000e+01
+9.675490000000000e+02 +1.390000000000000e+02
+1.309010000000000e+03 +1.805000000000000e+02
+7.822110000000000e+02 +7.700000000000000e+01
+4.921460000000000e+02 +1.330000000000000e+02
+1.296350000000000e+03 +2.000000000000000e+02
+7.858589999999998e+02 +1.000000000000000e+02
+8.849800000000000e+02 +1.705000000000000e+02
+7.447510000000002e+02 +5.900000000000000e+01
+6.929560000000000e+02 +7.500000000000000e+01
+7.657630000000000e+02 +9.350000000000000e+01
+6.895030000000000e+02 +6.500000000000000e+01
+1.110340000000000e+03 +1.915000000000000e+02
+7.890160000000002e+02 +7.750000000000000e+01
+8.100430000000000e+02 +2.010000000000000e+02
+4.837160000000000e+02 +1.310000000000000e+02
+1.303720000000000e+03 +2.185000000000000e+02
+8.593739999999998e+02 +1.040000000000000e+02
+7.840760000000000e+02 +9.050000000000000e+01
+1.299410000000000e+03 +2.075000000000000e+02
+4.782460000000000e+02 +5.850000000000000e+01
+1.284610000000000e+03 +1.995000000000000e+02
+7.771860000000000e+02 +8.600000000000000e+01
+9.782569999999999e+02 +1.240000000000000e+02
+9.650069999999999e+02 +1.155000000000000e+02
+7.789870000000000e+02 +7.600000000000000e+01
+6.897320000000000e+02 +7.550000000000000e+01
+7.699280000000000e+02 +1.260000000000000e+02
+1.109260000000000e+03 +2.365000000000000e+02
+4.962450000000000e+02 +1.140000000000000e+02
+4.944750000000000e+02 +1.275000000000000e+02
+7.906250000000000e+02 +7.950000000000000e+01
+9.724580000000000e+02 +1.185000000000000e+02
+1.134440000000000e+03 +1.825000000000000e+02
+1.315130000000000e+03 +1.995000000000000e+02
+7.881319999999999e+02 +7.600000000000000e+01
+7.693889999999999e+02 +8.700000000000000e+01
+1.105790000000000e+03 +2.250000000000000e+02
+7.830380000000000e+02 +7.300000000000000e+01
+7.799349999999999e+02 +8.550000000000000e+01
+1.295770000000000e+03 +1.955000000000000e+02
+4.744870000000000e+02 +1.280000000000000e+02
+1.294050000000000e+03 +1.490000000000000e+02
+7.770430000000000e+02 +8.400000000000000e+01
+6.785060000000002e+02 +1.870000000000000e+02
+8.574140000000000e+02 +9.300000000000000e+01
+4.887740000000000e+02 +1.260000000000000e+02
+6.379250000000000e+02 +5.950000000000000e+01
+1.286280000000000e+03 +1.920000000000000e+02
+7.788160000000000e+02 +7.100000000000000e+01
+7.608539999999998e+02 +8.450000000000000e+01
+6.932890000000000e+02 +8.750000000000000e+01
+9.458200000000001e+02 +1.965000000000000e+02
+6.808339999999999e+02 +1.525000000000000e+02
+4.963380000000000e+02 +1.315000000000000e+02
+1.059500000000000e+03 +1.800000000000000e+02
+1.317410000000000e+03 +1.955000000000000e+02
+7.510580000000000e+02 +6.650000000000000e+01
+7.798280000000000e+02 +8.300000000000000e+01
+6.832880000000000e+02 +1.445000000000000e+02
+1.307120000000000e+03 +1.475000000000000e+02
+7.788819999999999e+02 +7.100000000000000e+01
+8.580200000000000e+02 +9.100000000000000e+01
+1.303220000000000e+03 +1.640000000000000e+02
+9.296330000000000e+02 +1.800000000000000e+02
+4.933930000000000e+02 +1.230000000000000e+02
+4.887770000000000e+02 +1.125000000000000e+02
+7.871070000000000e+02 +7.050000000000000e+01
+7.477210000000000e+02 +5.350000000000000e+01
+1.094750000000000e+03 +2.220000000000000e+02
+8.624130000000000e+02 +1.225000000000000e+02
+8.707719999999998e+02 +8.950000000000000e+01
+3.250210000000000e+02 +2.300000000000000e+01
+1.293860000000000e+03 +1.725000000000000e+02
+7.894299999999999e+02 +7.000000000000000e+01
+7.793560000000001e+02 +8.350000000000000e+01
+7.557680000000000e+02 +9.550000000000000e+01
+7.828539999999998e+02 +6.700000000000000e+01
+7.516270000000000e+02 +7.950000000000000e+01
+8.555280000000000e+02 +7.700000000000000e+01
+1.127450000000000e+03 +2.160000000000000e+02
+4.942210000000000e+02 +1.230000000000000e+02
+7.455470000000000e+02 +6.150000000000000e+01
+7.828170000000000e+02 +8.150000000000000e+01
+6.464190000000000e+02 +2.550000000000000e+02
+7.543969999999998e+02 +1.170000000000000e+02
+9.707030000000000e+02 +9.900000000000000e+01
+1.339850000000000e+03 +2.755000000000000e+02
+1.389660000000000e+03 +2.690000000000000e+02
+9.310340000000000e+02 +1.995000000000000e+02
+7.886890000000000e+02 +6.700000000000000e+01
+7.632410000000001e+02 +6.150000000000000e+01
+4.854800000000000e+02 +9.050000000000000e+01
+7.794900000000000e+02 +7.900000000000000e+01
+7.894140000000000e+02 +7.850000000000000e+01
+1.301740000000000e+03 +1.595000000000000e+02
+8.699299999999999e+02 +8.600000000000000e+01
+1.330150000000000e+03 +2.930000000000000e+02
+4.931880000000001e+02 +9.100000000000000e+01
+1.084620000000000e+03 +1.795000000000000e+02
+1.306590000000000e+03 +1.540000000000000e+02
+7.827560000000002e+02 +6.650000000000000e+01
+8.706700000000000e+02 +7.850000000000000e+01
+4.868200000000000e+02 +1.040000000000000e+02
+7.415250000000000e+02 +1.425000000000000e+02
+1.380860000000000e+03 +2.665000000000000e+02
+1.108860000000000e+03 +2.675000000000000e+02
+7.534580000000002e+02 +7.400000000000000e+01
+4.649990000000000e+02 +4.400000000000000e+01
+1.329440000000000e+03 +2.325000000000000e+02
+4.860720000000000e+02 +8.050000000000000e+01
+7.762580000000000e+02 +7.650000000000000e+01
+1.324990000000000e+03 +2.760000000000000e+02
+1.041750000000000e+03 +1.780000000000000e+02
+7.624950000000000e+02 +7.550000000000000e+01
+8.715440000000000e+02 +1.205000000000000e+02
+1.069780000000000e+03 +1.770000000000000e+02
+1.332970000000000e+03 +2.150000000000000e+02
+4.900820000000000e+02 +8.400000000000000e+01
+1.117740000000000e+03 +1.555000000000000e+02
+7.455900000000000e+02 +5.150000000000000e+01
+8.571319999999999e+02 +7.900000000000000e+01
+8.175700000000001e+02 +1.820000000000000e+02
+7.549160000000001e+02 +6.150000000000000e+01
+1.011560000000000e+03 +1.945000000000000e+02
+1.100260000000000e+03 +1.335000000000000e+02
+1.356210000000000e+03 +1.935000000000000e+02
+1.365020000000000e+03 +2.045000000000000e+02
+1.730490000000000e+03 +2.910000000000000e+02
+1.011340000000000e+03 +1.880000000000000e+02
+7.405889999999998e+02 +7.800000000000000e+01
+1.280900000000000e+03 +2.435000000000000e+02
+8.963320000000000e+02 +1.185000000000000e+02
+6.742439999999998e+02 +1.485000000000000e+02
+8.433140000000000e+02 +7.550000000000000e+01
+7.538110000000000e+02 +6.850000000000000e+01
+1.221440000000000e+03 +2.660000000000000e+02
+6.893400000000000e+02 +1.565000000000000e+02
+7.070330000000000e+02 +2.025000000000000e+02
+6.791660000000001e+02 +1.740000000000000e+02
+9.242510000000000e+02 +1.760000000000000e+02
+9.040570000000000e+02 +1.255000000000000e+02
+1.015920000000000e+03 +1.240000000000000e+02
+1.102980000000000e+03 +2.210000000000000e+02
+1.075640000000000e+03 +2.885000000000000e+02
+6.580419999999998e+02 +1.405000000000000e+02
+1.811120000000000e+03 +2.925000000000000e+02
+1.004430000000000e+03 +1.920000000000000e+02
+6.901200000000000e+02 +9.400000000000000e+01
+1.009750000000000e+03 +1.965000000000000e+02
+1.066830000000000e+03 +1.540000000000000e+02
+6.538300000000000e+02 +1.455000000000000e+02
+1.580330000000000e+03 +2.350000000000000e+02
+8.725490000000000e+02 +6.800000000000000e+01
+1.075460000000000e+03 +2.975000000000000e+02
+6.674780000000002e+02 +1.505000000000000e+02
+8.956270000000000e+02 +1.315000000000000e+02
+6.756820000000000e+02 +1.130000000000000e+02
+1.741130000000000e+03 +2.385000000000000e+02
+6.345290000000000e+02 +1.550000000000000e+02
+6.913250000000000e+02 +1.225000000000000e+02
+1.528950000000000e+03 +2.815000000000000e+02
+1.081580000000000e+03 +1.195000000000000e+02
+1.225660000000000e+03 +2.400000000000000e+02
+1.068350000000000e+03 +2.805000000000000e+02
+9.256849999999999e+02 +1.520000000000000e+02
+1.000740000000000e+03 +1.820000000000000e+02
+1.331260000000000e+03 +2.535000000000000e+02
+6.744789999999998e+02 +1.000000000000000e+02
+2.664430000000000e+03 +3.285000000000000e+02
+9.229520000000000e+02 +1.660000000000000e+02
+6.954330000000000e+02 +9.950000000000000e+01
+6.352650000000000e+02 +1.505000000000000e+02
+1.742680000000000e+03 +2.875000000000000e+02
+1.517340000000000e+03 +2.095000000000000e+02
+1.075780000000000e+03 +2.425000000000000e+02
+1.788050000000000e+03 +2.375000000000000e+02
+6.267330000000002e+02 +1.955000000000000e+02
+9.594180000000000e+02 +1.805000000000000e+02
+6.389380000000000e+02 +1.445000000000000e+02
+6.589090000000000e+02 +1.390000000000000e+02
+1.048300000000000e+03 +1.950000000000000e+02
+6.410599999999999e+02 +1.575000000000000e+02
+9.096770000000000e+02 +1.210000000000000e+02
+8.983330000000002e+02 +1.110000000000000e+02
+1.077510000000000e+02 +2.650000000000000e+01
+1.074910000000000e+03 +2.395000000000000e+02
+1.039350000000000e+03 +1.395000000000000e+02
+6.350700000000001e+02 +1.515000000000000e+02
+1.151150000000000e+03 +2.105000000000000e+02
+9.673440000000001e+02 +9.900000000000000e+01
+6.554600000000000e+02 +1.285000000000000e+02
+1.855720000000000e+03 +2.970000000000000e+02
+8.967600000000000e+02 +1.215000000000000e+02
+1.322290000000000e+03 +2.990000000000000e+02
+6.558260000000000e+02 +1.205000000000000e+02
+2.631300000000000e+03 +3.170000000000000e+02
+6.464930000000001e+02 +2.515000000000000e+02
+6.394109999999999e+02 +1.470000000000000e+02
+1.319680000000000e+03 +2.540000000000000e+02
+1.064620000000000e+03 +1.110000000000000e+02
+6.368410000000000e+02 +1.055000000000000e+02
+6.185110000000000e+02 +1.310000000000000e+02
+1.063070000000000e+03 +1.865000000000000e+02
+6.509770000000000e+02 +1.265000000000000e+02
+8.931039999999998e+02 +1.055000000000000e+02
+1.353730000000000e+03 +2.335000000000000e+02
+6.521970000000000e+02 +1.045000000000000e+02
+1.034360000000000e+03 +1.825000000000000e+02
+6.270119999999999e+02 +2.200000000000000e+02
+1.454410000000000e+03 +2.720000000000000e+02
+6.388270000000000e+02 +1.310000000000000e+02
+8.711080000000002e+02 +7.550000000000000e+01
+6.380210000000000e+02 +1.165000000000000e+02
+9.029500000000000e+02 +1.240000000000000e+02
+8.923830000000000e+02 +1.205000000000000e+02
+1.157470000000000e+03 +2.355000000000000e+02
+6.436510000000000e+02 +8.700000000000000e+01
+8.695880000000002e+02 +8.600000000000000e+01
+6.553530000000002e+02 +1.315000000000000e+02
+6.782430000000001e+02 +1.040000000000000e+02
+1.143580000000000e+02 +3.100000000000000e+01
+1.486170000000000e+03 +2.540000000000000e+02
+1.122000000000000e+03 +2.000000000000000e+02
+6.759589999999999e+02 +1.450000000000000e+02
+6.632539999999998e+02 +8.750000000000000e+01
+1.000580000000000e+03 +1.080000000000000e+02
+8.906160000000001e+02 +1.050000000000000e+02
+1.055390000000000e+03 +1.940000000000000e+02
+1.186760000000000e+02 +1.600000000000000e+01
+1.320070000000000e+03 +2.300000000000000e+02
+7.496430000000000e+02 +1.490000000000000e+02
+6.600939999999998e+02 +1.135000000000000e+02
+1.142470000000000e+03 +2.315000000000000e+02
+7.458980000000000e+02 +5.250000000000000e+01
+6.519710000000000e+02 +1.170000000000000e+02
+8.959310000000000e+02 +1.165000000000000e+02
+6.555570000000000e+02 +1.070000000000000e+02
+6.605860000000000e+02 +7.750000000000000e+01
+8.405130000000000e+02 +6.550000000000000e+01
+1.561370000000000e+03 +2.150000000000000e+02
+7.509939999999998e+02 +1.420000000000000e+02
+6.496860000000000e+02 +1.140000000000000e+02
+6.564130000000000e+02 +6.250000000000000e+01
+8.580260000000002e+02 +1.055000000000000e+02
+1.361740000000000e+03 +2.130000000000000e+02
+6.351730000000000e+02 +1.020000000000000e+02
+1.039340000000000e+03 +1.690000000000000e+02
+6.799490000000000e+02 +1.360000000000000e+02
+6.148910000000000e+02 +1.045000000000000e+02
+6.336849999999999e+02 +1.010000000000000e+02
+9.946559999999999e+02 +7.650000000000000e+01
+8.677769999999998e+02 +1.005000000000000e+02
+6.950069999999999e+02 +1.690000000000000e+02
+6.521590000000000e+02 +9.950000000000000e+01
+8.983300000000000e+02 +9.900000000000000e+01
+1.093930000000000e+02 +1.400000000000000e+01
+6.380430000000000e+02 +1.050000000000000e+02
+8.904510000000000e+02 +1.005000000000000e+02
+8.704169999999998e+02 +9.900000000000000e+01
+6.626830000000000e+02 +9.950000000000000e+01
+1.082120000000000e+03 +1.925000000000000e+02
+8.814019999999998e+02 +1.480000000000000e+02
+8.512910000000001e+02 +9.600000000000000e+01
+6.677360000000001e+02 +1.080000000000000e+02
+6.519540000000002e+02 +9.100000000000000e+01
+1.068640000000000e+03 +3.095000000000000e+02
+6.636100000000000e+02 +7.950000000000000e+01
+8.862890000000000e+02 +1.015000000000000e+02
+1.074890000000000e+03 +1.705000000000000e+02
+6.353740000000000e+02 +9.550000000000000e+01
+1.066780000000000e+03 +2.090000000000000e+02
+1.082490000000000e+03 +2.080000000000000e+02
+6.348720000000000e+02 +7.600000000000000e+01
+6.640360000000002e+02 +7.700000000000000e+01
+1.373290000000000e+03 +2.895000000000000e+02
+6.933750000000000e+02 +1.950000000000000e+02
+1.073580000000000e+03 +1.955000000000000e+02
+6.647150000000000e+02 +9.100000000000000e+01
+1.081620000000000e+03 +2.105000000000000e+02
+6.608860000000002e+02 +7.950000000000000e+01
+6.685450000000000e+02 +7.100000000000000e+01
+1.340690000000000e+03 +2.585000000000000e+02
+1.161210000000000e+03 +1.750000000000000e+02
+1.237930000000000e+03 +2.155000000000000e+02
+6.973049999999999e+02 +1.465000000000000e+02
+6.676430000000000e+02 +1.015000000000000e+02
+9.023830000000000e+02 +1.045000000000000e+02
+1.065430000000000e+03 +1.900000000000000e+02
+1.325290000000000e+03 +2.255000000000000e+02
+1.068890000000000e+03 +2.980000000000000e+02
+6.297940000000000e+02 +8.200000000000000e+01
+1.063560000000000e+03 +1.575000000000000e+02
+6.356100000000000e+02 +8.600000000000000e+01
+1.063240000000000e+03 +1.665000000000000e+02
+6.157840000000000e+02 +1.415000000000000e+02
+1.014130000000000e+03 +1.390000000000000e+02
+1.065290000000000e+03 +1.740000000000000e+02
+1.350040000000000e+03 +2.355000000000000e+02
+1.061650000000000e+03 +3.030000000000000e+02
+6.610160000000002e+02 +7.700000000000000e+01
+6.851330000000000e+02 +1.025000000000000e+02
+1.069510000000000e+03 +1.535000000000000e+02
+5.943890000000000e+02 +1.625000000000000e+02
+8.983270000000000e+02 +1.070000000000000e+02
+1.010260000000000e+03 +1.765000000000000e+02
+1.092240000000000e+03 +1.415000000000000e+02
+8.907850000000000e+02 +2.315000000000000e+02
+6.949140000000000e+02 +1.035000000000000e+02
+6.996039999999998e+02 +1.640000000000000e+02
+9.113160000000000e+02 +1.080000000000000e+02
+1.047300000000000e+03 +1.405000000000000e+02
+8.619400000000001e+02 +6.450000000000000e+01
+6.765889999999998e+02 +1.125000000000000e+02
+8.805710000000000e+02 +1.005000000000000e+02
+1.049270000000000e+03 +1.120000000000000e+02
+8.902470000000000e+02 +2.205000000000000e+02
+6.523060000000000e+02 +9.400000000000000e+01
+6.761039999999998e+02 +1.415000000000000e+02
+9.131470000000000e+02 +1.040000000000000e+02
+9.623800000000000e+02 +1.335000000000000e+02
+8.593210000000000e+02 +1.305000000000000e+02
+8.589630000000002e+02 +1.335000000000000e+02
+8.586950000000001e+02 +1.295000000000000e+02
+9.786120000000000e+02 +1.625000000000000e+02
+1.073770000000000e+03 +2.795000000000000e+02
+9.300260000000000e+02 +1.195000000000000e+02
+1.080960000000000e+03 +1.460000000000000e+02
+8.803410000000000e+02 +2.065000000000000e+02
+7.060239999999999e+02 +9.950000000000000e+01
+1.229980000000000e+03 +2.515000000000000e+02
+8.574920000000000e+02 +1.220000000000000e+02
+9.782390000000000e+02 +1.370000000000000e+02
+1.065530000000000e+03 +2.355000000000000e+02
+1.215190000000000e+03 +2.585000000000000e+02
+1.010890000000000e+03 +1.185000000000000e+02
+1.572440000000000e+03 +1.790000000000000e+02
+1.071360000000000e+03 +2.105000000000000e+02
+8.936050000000000e+02 +1.900000000000000e+02
+1.559630000000000e+03 +3.090000000000000e+02
+1.001700000000000e+03 +2.160000000000000e+02
+8.894380000000000e+02 +9.850000000000000e+01
+8.529970000000000e+02 +2.010000000000000e+02
+7.809110000000002e+02 +1.555000000000000e+02
+1.132570000000000e+03 +2.640000000000000e+02
+1.568390000000000e+03 +3.060000000000000e+02
+7.754220000000000e+02 +1.970000000000000e+02
+1.067240000000000e+03 +2.755000000000000e+02
+9.460069999999999e+02 +7.550000000000000e+01
+5.910660000000000e+02 +9.650000000000000e+01
+1.088780000000000e+03 +1.255000000000000e+02
+9.950460000000000e+02 +2.130000000000000e+02
+7.471790000000000e+02 +9.800000000000000e+01
+9.964790000000000e+02 +1.850000000000000e+02
+4.898100000000000e+02 +8.850000000000000e+01
+1.532860000000000e+03 +3.005000000000000e+02
+8.645230000000000e+02 +1.360000000000000e+02
+1.787660000000000e+03 +3.660000000000000e+02
+8.651330000000000e+02 +1.895000000000000e+02
+1.083840000000000e+03 +1.685000000000000e+02
+1.081080000000000e+03 +1.350000000000000e+02
+1.051820000000000e+03 +2.675000000000000e+02
+2.797730000000000e+02 +1.025000000000000e+02
+2.815610000000000e+02 +8.650000000000000e+01
+9.963260000000000e+02 +1.640000000000000e+02
+1.621000000000000e+03 +3.275000000000000e+02
+1.071050000000000e+03 +1.525000000000000e+02
+7.952430000000001e+02 +1.685000000000000e+02
+7.808180000000000e+02 +1.010000000000000e+02
+1.005040000000000e+03 +2.070000000000000e+02
+1.005560000000000e+03 +1.830000000000000e+02
+1.038010000000000e+03 +1.360000000000000e+02
+1.276110000000000e+03 +2.660000000000000e+02
+1.399120000000000e+03 +3.005000000000000e+02
+9.917329999999999e+02 +1.510000000000000e+02
+6.094109999999999e+02 +1.545000000000000e+02
+1.595160000000000e+03 +2.715000000000000e+02
+8.034400000000001e+02 +1.105000000000000e+02
+6.359580000000002e+02 +6.300000000000000e+01
+2.753600000000000e+02 +6.650000000000000e+01
+9.532600000000000e+02 +1.240000000000000e+02
+9.886060000000000e+02 +1.600000000000000e+02
+1.064730000000000e+03 +2.045000000000000e+02
+4.542900000000000e+02 +6.600000000000000e+01
+6.101180000000001e+02 +1.330000000000000e+02
+9.914390000000000e+02 +1.975000000000000e+02
+9.987840000000000e+02 +1.860000000000000e+02
+1.288290000000000e+03 +2.325000000000000e+02
+1.086570000000000e+03 +1.495000000000000e+02
+9.141960000000000e+02 +1.725000000000000e+02
+1.567790000000000e+03 +1.850000000000000e+02
+1.631730000000000e+03 +3.530000000000000e+02
+9.784430000000000e+02 +1.860000000000000e+02
+1.642310000000000e+03 +3.240000000000000e+02
+1.060780000000000e+03 +1.815000000000000e+02
+2.836060000000000e+02 +6.500000000000000e+01
+9.763090000000000e+02 +2.150000000000000e+02
+1.252950000000000e+03 +2.085000000000000e+02
+1.000350000000000e+03 +1.330000000000000e+02
+1.574440000000000e+03 +2.830000000000000e+02
+1.913790000000000e+03 +2.575000000000000e+02
+1.630860000000000e+03 +3.620000000000000e+02
+1.554070000000000e+03 +3.020000000000000e+02
+1.291620000000000e+03 +2.495000000000000e+02
+1.017500000000000e+03 +1.870000000000000e+02
+1.408410000000000e+03 +2.540000000000000e+02
+6.178120000000000e+02 +1.440000000000000e+02
+6.236619999999998e+02 +1.365000000000000e+02
+1.044120000000000e+03 +1.235000000000000e+02
+1.410560000000000e+03 +2.765000000000000e+02
+1.309950000000000e+03 +2.735000000000000e+02
+1.136090000000000e+03 +2.365000000000000e+02
+1.361400000000000e+03 +2.960000000000000e+02
+6.200040000000000e+02 +1.245000000000000e+02
+1.220480000000000e+02 +2.000000000000000e+01
+1.087160000000000e+03 +1.400000000000000e+02
+1.403540000000000e+03 +2.435000000000000e+02
+6.297869999999998e+02 +1.345000000000000e+02
+6.697230000000002e+02 +2.110000000000000e+02
+1.562640000000000e+03 +2.345000000000000e+02
+9.775130000000000e+02 +1.265000000000000e+02
+9.328090000000000e+02 +1.445000000000000e+02
+9.889140000000000e+02 +2.090000000000000e+02
+9.888190000000000e+02 +1.590000000000000e+02
+1.588280000000000e+03 +3.450000000000000e+02
+1.112240000000000e+03 +1.685000000000000e+02
+1.561360000000000e+03 +2.650000000000000e+02
+1.584210000000000e+03 +2.705000000000000e+02
+2.324560000000000e+03 +3.445000000000000e+02
+1.410980000000000e+03 +2.550000000000000e+02
+9.762880000000000e+02 +1.325000000000000e+02
+9.138320000000000e+02 +1.080000000000000e+02
+6.245530000000000e+02 +9.450000000000000e+01
+6.190180000000000e+02 +1.315000000000000e+02
+6.642380000000001e+02 +2.125000000000000e+02
+6.584910000000001e+02 +1.000000000000000e+02
+1.571690000000000e+03 +2.175000000000000e+02
+1.367620000000000e+03 +1.970000000000000e+02
+1.371060000000000e+03 +3.360000000000000e+02
+1.424250000000000e+03 +2.900000000000000e+02
+9.314030000000000e+02 +1.440000000000000e+02
+1.570920000000000e+03 +2.485000000000000e+02
+1.550710000000000e+03 +2.735000000000000e+02
+1.585200000000000e+03 +2.140000000000000e+02
+2.817700000000000e+02 +2.100000000000000e+01
+1.057900000000000e+02 +9.000000000000000e+00
+1.288540000000000e+03 +2.390000000000000e+02
+9.778180000000000e+02 +1.285000000000000e+02
+6.248210000000000e+02 +1.250000000000000e+02
+6.250509999999998e+02 +9.050000000000000e+01
+1.282440000000000e+03 +2.435000000000000e+02
+9.117960000000000e+02 +1.385000000000000e+02
+2.687510000000000e+03 +3.790000000000000e+02
+9.920560000000000e+02 +1.790000000000000e+02
+1.570030000000000e+03 +2.410000000000000e+02
+6.216659999999998e+02 +1.060000000000000e+02
+1.003530000000000e+03 +1.675000000000000e+02
+1.549580000000000e+03 +2.265000000000000e+02
+9.771520000000000e+02 +1.225000000000000e+02
+1.539390000000000e+03 +2.180000000000000e+02
+6.590230000000000e+02 +1.370000000000000e+02
+1.574800000000000e+03 +2.235000000000000e+02
+1.401580000000000e+03 +2.605000000000000e+02
+1.868550000000000e+03 +3.500000000000000e+02
+6.223110000000000e+02 +7.200000000000000e+01
+1.292450000000000e+03 +2.030000000000000e+02
+1.571220000000000e+03 +2.295000000000000e+02
+1.053700000000000e+03 +1.885000000000000e+02
+6.220190000000000e+02 +7.000000000000000e+01
+6.576530000000000e+02 +9.700000000000000e+01
+9.856650000000000e+02 +1.150000000000000e+02
+6.364250000000000e+02 +1.070000000000000e+02
+2.679850000000000e+03 +3.830000000000000e+02
+1.848540000000000e+03 +3.210000000000000e+02
+6.189400000000001e+02 +8.950000000000000e+01
+1.569480000000000e+03 +2.340000000000000e+02
+1.323950000000000e+03 +2.105000000000000e+02
+2.765570000000000e+02 +2.150000000000000e+01
+6.188600000000000e+02 +6.550000000000000e+01
+9.786980000000000e+02 +1.145000000000000e+02
+1.061060000000000e+03 +1.850000000000000e+02
+6.646120000000000e+02 +7.750000000000000e+01
+1.247330000000000e+03 +2.510000000000000e+02
+1.426450000000000e+03 +2.440000000000000e+02
+6.236690000000000e+02 +6.250000000000000e+01
+1.417780000000000e+03 +2.075000000000000e+02
+7.774700000000000e+02 +1.390000000000000e+02
+6.224560000000000e+02 +5.750000000000000e+01
+7.476350000000000e+02 +1.265000000000000e+02
+1.305390000000000e+03 +2.210000000000000e+02
+7.867510000000002e+02 +1.510000000000000e+02
+1.298800000000000e+03 +2.405000000000000e+02
+1.296560000000000e+03 +2.330000000000000e+02
+1.231470000000000e+03 +2.280000000000000e+02
+1.295110000000000e+03 +2.190000000000000e+02
+7.696760000000000e+02 +1.320000000000000e+02
+7.781330000000000e+02 +1.410000000000000e+02
+7.748049999999999e+02 +1.450000000000000e+02
+1.496370000000000e+03 +3.150000000000000e+02
+1.613760000000000e+03 +3.100000000000000e+02
+1.615560000000000e+03 +3.465000000000000e+02
+2.816110000000000e+02 +2.450000000000000e+01
+6.873580000000002e+02 +1.440000000000000e+02
+1.434350000000000e+03 +2.125000000000000e+02
+6.892660000000002e+02 +1.475000000000000e+02
+1.483140000000000e+03 +2.765000000000000e+02
+7.847070000000000e+02 +1.330000000000000e+02
+7.802239999999998e+02 +1.240000000000000e+02
+1.602540000000000e+03 +3.500000000000000e+02
+1.609440000000000e+03 +3.080000000000000e+02
+7.800030000000000e+02 +1.300000000000000e+02
+6.956319999999999e+02 +1.260000000000000e+02
+1.470270000000000e+03 +2.905000000000000e+02
+1.869350000000000e+03 +3.650000000000000e+02
+1.840090000000000e+03 +3.640000000000000e+02
+7.830100000000000e+02 +1.415000000000000e+02
+1.631610000000000e+03 +3.425000000000000e+02
+1.588410000000000e+03 +3.500000000000000e+02
+7.796360000000002e+02 +1.235000000000000e+02
+7.837710000000002e+02 +1.220000000000000e+02
+2.722370000000000e+02 +1.935000000000000e+02
+1.490860000000000e+03 +2.930000000000000e+02
+1.853040000000000e+03 +3.570000000000000e+02
+6.079520000000000e+02 +9.850000000000000e+01
+1.614570000000000e+03 +2.830000000000000e+02
+6.928560000000001e+02 +1.370000000000000e+02
+7.802210000000000e+02 +1.380000000000000e+02
+9.590750000000000e+02 +2.625000000000000e+02
+7.702630000000000e+02 +1.135000000000000e+02
+1.433810000000000e+03 +1.955000000000000e+02
+1.867350000000000e+03 +3.540000000000000e+02
+6.920670000000000e+02 +1.245000000000000e+02
+1.060980000000000e+02 +1.150000000000000e+01
+1.302890000000000e+03 +2.520000000000000e+02
+7.901680000000000e+02 +1.375000000000000e+02
+7.631950000000001e+02 +1.180000000000000e+02
+2.823050000000000e+02 +2.150000000000000e+01
+1.456170000000000e+03 +1.830000000000000e+02
+7.547040000000000e+02 +1.115000000000000e+02
+7.909930000000001e+02 +1.380000000000000e+02
+1.297800000000000e+03 +2.095000000000000e+02
+6.933710000000002e+02 +1.395000000000000e+02
+9.249500000000000e+02 +2.315000000000000e+02
+7.801940000000000e+02 +1.295000000000000e+02
+8.834130000000000e+02 +1.675000000000000e+02
+1.298730000000000e+03 +1.765000000000000e+02
+7.759910000000001e+02 +1.250000000000000e+02
+8.806519999999998e+02 +1.580000000000000e+02
+6.968750000000000e+02 +1.175000000000000e+02
+7.759240000000000e+02 +1.160000000000000e+02
+1.302440000000000e+03 +1.755000000000000e+02
+1.452760000000000e+03 +1.635000000000000e+02
+7.515490000000000e+02 +1.200000000000000e+02
+6.913510000000001e+02 +1.110000000000000e+02
+1.117180000000000e+03 +2.735000000000000e+02
+5.028980000000000e+02 +1.685000000000000e+02
+9.806870000000000e+02 +1.105000000000000e+02
+7.840060000000002e+02 +1.125000000000000e+02
+1.300960000000000e+03 +1.880000000000000e+02
+8.795139999999999e+02 +1.560000000000000e+02
+6.945290000000000e+02 +1.360000000000000e+02
+8.587420000000000e+02 +2.070000000000000e+02
+4.934060000000000e+02 +1.640000000000000e+02
+7.803110000000000e+02 +1.250000000000000e+02
+6.911710000000000e+02 +7.300000000000000e+01
+1.127300000000000e+03 +2.480000000000000e+02
+7.687719999999998e+02 +1.130000000000000e+02
+1.320000000000000e+03 +1.620000000000000e+02
+7.873800000000000e+02 +1.125000000000000e+02
+1.305480000000000e+03 +2.425000000000000e+02
+7.809030000000000e+02 +1.245000000000000e+02
+6.966150000000000e+02 +7.950000000000000e+01
+1.740760000000000e+03 +2.985000000000000e+02
+5.019030000000000e+02 +1.635000000000000e+02
+1.319260000000000e+03 +2.335000000000000e+02
+7.525670000000000e+02 +9.750000000000000e+01
+8.883290000000000e+02 +1.390000000000000e+02
+6.910830000000002e+02 +1.145000000000000e+02
+3.260670000000000e+02 +9.650000000000000e+01
+1.145340000000000e+03 +2.760000000000000e+02
+7.732830000000000e+02 +1.210000000000000e+02
+8.810700000000001e+02 +1.370000000000000e+02
+1.066040000000000e+03 +1.670000000000000e+02
+8.709019999999998e+02 +1.385000000000000e+02
+1.311690000000000e+03 +2.285000000000000e+02
+7.828339999999999e+02 +1.015000000000000e+02
+1.310600000000000e+03 +2.400000000000000e+02
+6.892270000000000e+02 +9.700000000000000e+01
+1.123510000000000e+03 +2.650000000000000e+02
+1.865110000000000e+03 +3.090000000000000e+02
+4.862930000000000e+02 +1.545000000000000e+02
+7.838969999999998e+02 +1.000000000000000e+02
+1.089610000000000e+03 +1.625000000000000e+02
+1.058130000000000e+03 +1.605000000000000e+02
+6.942960000000000e+02 +9.400000000000000e+01
+7.863689999999998e+02 +1.205000000000000e+02
+9.086070000000000e+02 +1.380000000000000e+02
+1.087410000000000e+02 +1.250000000000000e+01
+6.573580000000002e+02 +1.405000000000000e+02
+6.961310000000002e+02 +8.850000000000000e+01
+9.736760000000000e+02 +1.120000000000000e+02
+1.314740000000000e+03 +2.340000000000000e+02
+7.895410000000001e+02 +1.155000000000000e+02
+1.043750000000000e+03 +1.135000000000000e+02
+1.298640000000000e+03 +2.230000000000000e+02
+9.086830000000000e+02 +1.380000000000000e+02
+1.117310000000000e+03 +1.620000000000000e+02
+1.869350000000000e+03 +3.090000000000000e+02
+7.766100000000000e+02 +9.650000000000000e+01
+8.692689999999999e+02 +2.415000000000000e+02
+4.772770000000000e+02 +1.055000000000000e+02
+2.813530000000000e+02 +1.140000000000000e+02
+9.768620000000000e+02 +1.070000000000000e+02
+1.849970000000000e+03 +3.280000000000000e+02
+7.788150000000001e+02 +9.550000000000000e+01
+1.023080000000000e+03 +1.630000000000000e+02
+1.046790000000000e+03 +1.715000000000000e+02
+6.882669999999998e+02 +1.000000000000000e+02
+9.527990000000000e+02 +2.535000000000000e+02
+1.738420000000000e+03 +3.440000000000000e+02
+1.143130000000000e+03 +2.570000000000000e+02
+4.930510000000000e+02 +1.255000000000000e+02
+9.935050000000000e+02 +1.610000000000000e+02
+7.687100000000000e+02 +1.090000000000000e+02
+8.810460000000000e+02 +1.485000000000000e+02
+1.479690000000000e+03 +3.725000000000000e+02
+1.296690000000000e+03 +2.170000000000000e+02
+8.635300000000000e+02 +1.080000000000000e+02
+6.507680000000000e+02 +1.145000000000000e+02
+1.741680000000000e+03 +3.410000000000000e+02
+1.474000000000000e+03 +2.690000000000000e+02
+7.784440000000000e+02 +1.070000000000000e+02
+6.935110000000002e+02 +5.550000000000000e+01
+1.772890000000000e+03 +2.905000000000000e+02
+1.592950000000000e+03 +3.895000000000000e+02
+7.003230000000000e+02 +1.065000000000000e+02
+9.775040000000000e+02 +1.195000000000000e+02
+4.910370000000000e+02 +1.420000000000000e+02
+1.299320000000000e+03 +2.140000000000000e+02
+1.317820000000000e+03 +1.990000000000000e+02
+1.088910000000000e+03 +1.305000000000000e+02
+1.093870000000000e+03 +1.075000000000000e+02
+6.749720000000000e+02 +1.170000000000000e+02
+1.860850000000000e+03 +3.860000000000000e+02
+1.477290000000000e+03 +2.080000000000000e+02
+7.511200000000000e+02 +9.300000000000000e+01
+9.526070000000000e+02 +1.525000000000000e+02
+7.932650000000000e+02 +1.085000000000000e+02
+1.299200000000000e+03 +2.115000000000000e+02
+1.299810000000000e+03 +1.930000000000000e+02
+1.107530000000000e+03 +1.575000000000000e+02
+1.876380000000000e+03 +3.785000000000000e+02
+7.855810000000000e+02 +1.055000000000000e+02
+9.455790000000000e+02 +1.240000000000000e+02
+6.288020000000000e+02 +2.740000000000000e+02
+1.128610000000000e+03 +2.190000000000000e+02
+1.082660000000000e+03 +1.475000000000000e+02
+1.588120000000000e+03 +3.560000000000000e+02
+6.943980000000000e+02 +7.900000000000000e+01
+6.678670000000000e+02 +1.155000000000000e+02
+4.995450000000000e+02 +1.040000000000000e+02
+7.740030000000000e+02 +9.650000000000000e+01
+9.075590000000000e+02 +1.310000000000000e+02
+9.531369999999999e+02 +2.615000000000000e+02
+6.895290000000000e+02 +6.700000000000000e+01
+1.108310000000000e+03 +2.205000000000000e+02
+6.930069999999999e+02 +1.940000000000000e+02
+7.446799999999999e+02 +9.100000000000000e+01
+7.469330000000000e+02 +1.300000000000000e+02
+4.837150000000000e+02 +1.225000000000000e+02
+7.911280000000000e+02 +1.015000000000000e+02
+1.081920000000000e+03 +1.080000000000000e+02
+1.605010000000000e+03 +3.715000000000000e+02
+1.357230000000000e+03 +2.780000000000000e+02
+1.310430000000000e+03 +2.070000000000000e+02
+1.324830000000000e+03 +1.860000000000000e+02
+1.926020000000000e+03 +3.075000000000000e+02
+9.281880000000000e+02 +2.420000000000000e+02
+4.676780000000001e+02 +8.450000000000000e+01
+9.858960000000000e+02 +1.465000000000000e+02
+7.409370000000000e+02 +1.095000000000000e+02
+7.734470000000000e+02 +9.500000000000000e+01
+1.268240000000000e+03 +2.915000000000000e+02
+1.155820000000000e+03 +2.595000000000000e+02
+7.706740000000000e+02 +9.650000000000000e+01
+1.307420000000000e+03 +1.685000000000000e+02
+1.064450000000000e+03 +1.155000000000000e+02
+2.664320000000000e+03 +3.820000000000000e+02
+1.302430000000000e+03 +1.720000000000000e+02
+4.783400000000000e+02 +8.100000000000000e+01
+7.800850000000000e+02 +9.450000000000000e+01
+7.796500000000000e+02 +9.500000000000000e+01
+9.678890000000000e+02 +1.450000000000000e+02
+1.335700000000000e+03 +2.940000000000000e+02
+1.327980000000000e+03 +2.760000000000000e+02
+7.461160000000001e+02 +1.055000000000000e+02
+5.053580000000000e+02 +1.070000000000000e+02
+9.253600000000000e+02 +1.270000000000000e+02
+7.834850000000000e+02 +9.400000000000000e+01
+1.180290000000000e+03 +2.270000000000000e+02
+7.047420000000000e+02 +1.860000000000000e+02
+4.965020000000000e+02 +7.900000000000000e+01
+9.878590000000000e+02 +1.430000000000000e+02
+1.160730000000000e+03 +2.640000000000000e+02
+7.896220000000000e+02 +9.450000000000000e+01
+8.934160000000001e+02 +1.125000000000000e+02
+8.781270000000000e+02 +1.310000000000000e+02
+1.302380000000000e+03 +3.060000000000000e+02
+7.815069999999999e+02 +9.400000000000000e+01
+1.295760000000000e+03 +1.440000000000000e+02
+9.010490000000000e+02 +1.345000000000000e+02
+4.838840000000000e+02 +6.150000000000000e+01
+7.540970000000000e+02 +1.065000000000000e+02
+7.009450000000001e+02 +2.060000000000000e+02
+9.685060000000000e+02 +1.435000000000000e+02
+1.328300000000000e+03 +3.370000000000000e+02
+7.840280000000000e+02 +9.100000000000000e+01
+7.873190000000000e+02 +9.400000000000000e+01
+4.849410000000000e+02 +1.120000000000000e+02
+4.966870000000000e+02 +1.030000000000000e+02
+1.326630000000000e+03 +3.455000000000000e+02
+7.412410000000001e+02 +9.350000000000000e+01
+8.776339999999999e+02 +1.095000000000000e+02
+9.429680000000000e+02 +2.595000000000000e+02
+7.806750000000000e+02 +8.750000000000000e+01
+7.973110000000000e+02 +9.350000000000000e+01
+9.818720000000000e+02 +1.415000000000000e+02
+1.394480000000000e+03 +2.645000000000000e+02
+6.747070000000000e+02 +2.355000000000000e+02
+1.300440000000000e+03 +1.675000000000000e+02
+9.281079999999999e+02 +1.040000000000000e+02
+1.130400000000000e+03 +2.170000000000000e+02
+8.331880000000000e+02 +9.900000000000000e+01
+7.461260000000002e+02 +8.550000000000000e+01
+7.790700000000001e+02 +8.100000000000000e+01
+1.156140000000000e+03 +2.965000000000000e+02
+1.113450000000000e+03 +2.610000000000000e+02
+7.718720000000000e+02 +8.750000000000000e+01
+7.744080000000000e+02 +8.450000000000000e+01
+9.917950000000000e+02 +2.425000000000000e+02
+1.151070000000000e+03 +2.340000000000000e+02
+6.762050000000000e+02 +2.425000000000000e+02
+1.320380000000000e+03 +2.050000000000000e+02
+9.095440000000000e+02 +1.155000000000000e+02
+9.057220000000000e+02 +1.030000000000000e+02
+9.547410000000000e+02 +2.475000000000000e+02
+1.363900000000000e+03 +2.400000000000000e+02
+9.207040000000000e+02 +2.140000000000000e+02
+7.726139999999998e+02 +7.950000000000000e+01
+7.781080000000002e+02 +7.900000000000000e+01
+6.249190000000000e+02 +2.810000000000000e+02
+7.854440000000000e+02 +7.250000000000000e+01
+9.124960000000000e+02 +1.320000000000000e+02
+9.716130000000001e+02 +1.540000000000000e+02
+1.297200000000000e+03 +3.950000000000000e+02
+1.339670000000000e+03 +2.535000000000000e+02
+1.630170000000000e+03 +3.845000000000000e+02
+7.841500000000000e+02 +7.950000000000000e+01
+4.696520000000000e+02 +4.850000000000000e+01
+6.930750000000000e+02 +2.320000000000000e+02
+8.421970000000000e+02 +9.050000000000000e+01
+7.546339999999999e+02 +5.700000000000000e+01
+1.077720000000000e+03 +2.135000000000000e+02
+8.604510000000000e+02 +1.610000000000000e+02
+1.315430000000000e+03 +3.595000000000000e+02
+9.363180000000000e+02 +2.205000000000000e+02
+1.037380000000000e+03 +2.035000000000000e+02
+1.304180000000000e+03 +3.380000000000000e+02
+1.063050000000000e+03 +1.995000000000000e+02
+1.017230000000000e+03 +1.595000000000000e+02
+9.074660000000000e+02 +1.020000000000000e+02
+8.673550000000000e+02 +1.405000000000000e+02
+6.913060000000000e+02 +1.800000000000000e+02
+1.877730000000000e+03 +3.355000000000000e+02
+1.854860000000000e+03 +3.485000000000000e+02
+1.004200000000000e+03 +2.220000000000000e+02
+1.613220000000000e+03 +3.965000000000000e+02
+1.849990000000000e+03 +3.455000000000000e+02
+6.979730000000002e+02 +1.455000000000000e+02
+6.424360000000000e+02 +1.555000000000000e+02
+9.990690000000000e+02 +2.230000000000000e+02
+8.419820000000000e+02 +1.225000000000000e+02
+9.140510000000000e+02 +1.010000000000000e+02
+6.546350000000000e+02 +2.705000000000000e+02
+1.483600000000000e+03 +2.315000000000000e+02
+6.640089999999999e+02 +1.390000000000000e+02
+8.803150000000001e+02 +1.100000000000000e+02
+9.153869999999999e+02 +1.730000000000000e+02
+1.854800000000000e+03 +3.170000000000000e+02
+6.855800000000000e+02 +1.610000000000000e+02
+6.405970000000000e+02 +1.430000000000000e+02
+6.731000000000000e+02 +1.440000000000000e+02
+1.835600000000000e+03 +4.005000000000000e+02
+9.319990000000000e+02 +2.230000000000000e+02
+1.007510000000000e+03 +2.085000000000000e+02
+6.472160000000000e+02 +2.620000000000000e+02
+6.456200000000000e+02 +1.595000000000000e+02
+1.005830000000000e+03 +2.015000000000000e+02
+1.272240000000000e+03 +2.785000000000000e+02
+1.106860000000000e+03 +3.050000000000000e+02
+6.572170000000000e+02 +1.335000000000000e+02
+1.888230000000000e+03 +3.380000000000000e+02
+1.529770000000000e+03 +3.000000000000000e+02
+8.913919999999998e+02 +9.750000000000000e+01
+6.379410000000000e+02 +1.495000000000000e+02
+8.876950000000001e+02 +2.285000000000000e+02
+8.630169999999998e+02 +1.410000000000000e+02
+6.609989999999998e+02 +2.790000000000000e+02
+6.925280000000000e+02 +1.545000000000000e+02
+6.620889999999998e+02 +1.600000000000000e+02
+2.769680000000000e+02 +2.100000000000000e+01
+6.381920000000000e+02 +1.460000000000000e+02
+9.287180000000000e+02 +2.220000000000000e+02
+8.857719999999998e+02 +9.200000000000000e+01
+8.989019999999998e+02 +2.355000000000000e+02
+1.338400000000000e+03 +2.895000000000000e+02
+1.232720000000000e+03 +1.915000000000000e+02
+6.369040000000000e+02 +1.535000000000000e+02
+6.536910000000000e+02 +1.405000000000000e+02
+1.137900000000000e+03 +3.045000000000000e+02
+7.908589999999998e+02 +1.545000000000000e+02
+6.853570000000000e+02 +1.305000000000000e+02
+6.321090000000000e+02 +1.355000000000000e+02
+1.105400000000000e+03 +2.545000000000000e+02
+9.393350000000000e+02 +2.215000000000000e+02
+2.650270000000000e+02 +9.900000000000000e+01
+1.163130000000000e+03 +2.835000000000000e+02
+1.154490000000000e+03 +2.135000000000000e+02
+8.933670000000000e+02 +7.900000000000000e+01
+6.405190000000000e+02 +1.460000000000000e+02
+6.349510000000000e+02 +1.105000000000000e+02
+1.138260000000000e+03 +2.835000000000000e+02
+7.080630000000000e+02 +2.090000000000000e+02
+6.177800000000000e+02 +1.390000000000000e+02
+1.140440000000000e+03 +2.755000000000000e+02
+1.084160000000000e+03 +2.800000000000000e+02
+1.134130000000000e+03 +2.165000000000000e+02
+1.036450000000000e+03 +1.950000000000000e+02
+6.348450000000000e+02 +1.220000000000000e+02
+8.653869999999999e+02 +1.910000000000000e+02
+1.073740000000000e+03 +3.220000000000000e+02
+6.344550000000000e+02 +1.150000000000000e+02
+9.493579999999999e+02 +2.220000000000000e+02
+1.134500000000000e+03 +2.420000000000000e+02
+6.165790000000002e+02 +9.550000000000000e+01
+2.027940000000000e+03 +4.200000000000000e+02
+1.101620000000000e+03 +2.245000000000000e+02
+9.007420000000000e+02 +9.750000000000000e+01
+8.113600000000000e+02 +1.475000000000000e+02
+1.469470000000000e+03 +2.485000000000000e+02
+1.101530000000000e+03 +1.895000000000000e+02
+6.427660000000000e+02 +1.340000000000000e+02
+6.494310000000000e+02 +1.135000000000000e+02
+6.511920000000000e+02 +7.950000000000000e+01
+1.939750000000000e+03 +3.680000000000000e+02
+1.173610000000000e+03 +2.605000000000000e+02
+6.726469999999998e+02 +1.010000000000000e+02
+9.042060000000000e+02 +9.300000000000000e+01
+9.020580000000000e+02 +8.150000000000000e+01
+1.070140000000000e+03 +2.460000000000000e+02
+6.354510000000000e+02 +1.285000000000000e+02
+6.346669999999998e+02 +1.025000000000000e+02
+6.180820000000000e+02 +7.000000000000000e+01
+9.131930000000000e+02 +1.130000000000000e+02
+7.522189999999998e+02 +1.445000000000000e+02
+6.463730000000000e+02 +1.020000000000000e+02
+6.137600000000000e+02 +9.050000000000000e+01
+6.371580000000000e+02 +7.100000000000000e+01
+6.871239999999998e+02 +1.920000000000000e+02
+1.378100000000000e+03 +3.080000000000000e+02
+1.117270000000000e+03 +2.045000000000000e+02
+6.356730000000000e+02 +1.070000000000000e+02
+6.524770000000000e+02 +7.550000000000000e+01
+1.887100000000000e+03 +3.725000000000000e+02
+9.141620000000000e+02 +1.000000000000000e+02
+6.551000000000000e+02 +9.050000000000000e+01
+1.103560000000000e+03 +2.005000000000000e+02
+1.081330000000000e+03 +2.645000000000000e+02
+1.563610000000000e+03 +2.830000000000000e+02
+6.801289999999998e+02 +1.680000000000000e+02
+6.382400000000000e+02 +8.900000000000000e+01
+6.352769999999998e+02 +6.800000000000000e+01
+1.157350000000000e+03 +3.110000000000000e+02
+6.406880000000000e+02 +6.800000000000000e+01
+1.862040000000000e+03 +3.800000000000000e+02
+1.572440000000000e+03 +3.010000000000000e+02
+1.698550000000000e+03 +2.290000000000000e+02
+7.046950000000001e+02 +2.220000000000000e+02
+6.455459999999998e+02 +6.650000000000000e+01
+6.274760000000000e+02 +9.500000000000000e+01
+6.687200000000000e+02 +6.650000000000000e+01
+1.564350000000000e+03 +2.605000000000000e+02
+1.542500000000000e+03 +2.255000000000000e+02
+1.656430000000000e+03 +2.740000000000000e+02
+1.046760000000000e+03 +2.220000000000000e+02
+1.322300000000000e+03 +2.510000000000000e+02
+6.926210000000002e+02 +2.375000000000000e+02
+6.397250000000000e+02 +8.350000000000000e+01
+6.634190000000000e+02 +6.500000000000000e+01
+9.050839999999999e+02 +6.750000000000000e+01
+1.183240000000000e+02 +3.150000000000000e+01
+1.567630000000000e+03 +2.710000000000000e+02
+1.047870000000000e+03 +2.065000000000000e+02
+1.086600000000000e+02 +2.300000000000000e+01
+1.276790000000000e+03 +2.965000000000000e+02
+6.498480000000002e+02 +6.450000000000000e+01
+1.622100000000000e+03 +4.015000000000000e+02
+1.692850000000000e+03 +2.285000000000000e+02
+6.730020000000000e+02 +9.700000000000000e+01
+1.874510000000000e+03 +3.755000000000000e+02
+1.250430000000000e+03 +2.890000000000000e+02
+1.077740000000000e+02 +2.350000000000000e+01
+1.561130000000000e+03 +2.655000000000000e+02
+1.531120000000000e+03 +2.350000000000000e+02
+1.236450000000000e+03 +2.120000000000000e+02
+1.566060000000000e+03 +2.350000000000000e+02
+1.568280000000000e+03 +1.920000000000000e+02
+1.092960000000000e+03 +2.155000000000000e+02
+6.781330000000000e+02 +1.690000000000000e+02
+7.075760000000000e+02 +1.495000000000000e+02
+8.951050000000000e+02 +2.610000000000000e+02
+6.704960000000002e+02 +9.550000000000000e+01
+8.946519999999998e+02 +6.900000000000000e+01
+7.911100000000000e+02 +2.150000000000000e+02
+1.035960000000000e+03 +2.040000000000000e+02
+1.089900000000000e+03 +2.080000000000000e+02
+1.563580000000000e+03 +2.230000000000000e+02
+6.337050000000000e+02 +7.150000000000000e+01
+7.919190000000000e+02 +2.005000000000000e+02
+1.677170000000000e+03 +2.965000000000000e+02
+1.095660000000000e+03 +2.245000000000000e+02
+8.927050000000000e+02 +2.760000000000000e+02
+6.365020000000000e+02 +8.250000000000000e+01
+7.956080000000002e+02 +1.810000000000000e+02
+1.054980000000000e+03 +1.965000000000000e+02
+1.583980000000000e+03 +2.480000000000000e+02
+1.011400000000000e+03 +2.210000000000000e+02
+1.083730000000000e+03 +2.140000000000000e+02
+7.953960000000002e+02 +1.770000000000000e+02
+1.083580000000000e+03 +2.095000000000000e+02
+1.070730000000000e+03 +1.570000000000000e+02
+7.019980000000000e+02 +1.365000000000000e+02
+6.938900000000000e+02 +1.285000000000000e+02
+1.083220000000000e+03 +1.445000000000000e+02
+4.209520000000000e+02 +3.500000000000000e+01
+1.024020000000000e+03 +2.285000000000000e+02
+6.952189999999998e+02 +1.470000000000000e+02
+4.568890000000000e+02 +4.450000000000000e+01
+7.001799999999999e+02 +1.365000000000000e+02
+1.545710000000000e+03 +3.560000000000000e+02
+1.012590000000000e+03 +1.880000000000000e+02
+4.808260000000000e+02 +4.100000000000000e+01
+8.491849999999999e+02 +2.860000000000000e+02
+1.050930000000000e+03 +1.985000000000000e+02
+1.068210000000000e+03 +1.475000000000000e+02
+1.069460000000000e+03 +1.620000000000000e+02
+1.566090000000000e+03 +3.340000000000000e+02
+1.575840000000000e+03 +2.765000000000000e+02
+1.038970000000000e+03 +1.780000000000000e+02
+1.065480000000000e+03 +1.385000000000000e+02
+7.022530000000000e+02 +1.030000000000000e+02
+1.030780000000000e+03 +1.710000000000000e+02
+1.090310000000000e+03 +1.750000000000000e+02
+1.572950000000000e+03 +2.515000000000000e+02
+1.878480000000000e+03 +4.445000000000000e+02
+8.798460000000000e+02 +1.720000000000000e+02
+8.585660000000000e+02 +1.595000000000000e+02
+9.756100000000000e+02 +1.770000000000000e+02
+9.726830000000000e+02 +1.830000000000000e+02
+1.616440000000000e+03 +3.925000000000000e+02
+9.682200000000000e+02 +8.150000000000000e+01
+8.606870000000000e+02 +1.570000000000000e+02
+1.065900000000000e+03 +1.420000000000000e+02
+6.920790000000000e+02 +2.235000000000000e+02
+7.088780000000000e+02 +1.905000000000000e+02
+1.127380000000000e+03 +2.555000000000000e+02
+1.490320000000000e+03 +4.135000000000000e+02
+1.016800000000000e+03 +1.175000000000000e+02
+1.476330000000000e+03 +2.950000000000000e+02
+1.499540000000000e+03 +3.685000000000000e+02
+1.062450000000000e+03 +1.745000000000000e+02
+9.053700000000000e+02 +2.105000000000000e+02
+1.815440000000000e+03 +4.210000000000000e+02
+8.689960000000002e+02 +1.990000000000000e+02
+7.806760000000000e+02 +1.760000000000000e+02
+2.047940000000000e+03 +4.370000000000000e+02
+9.674220000000000e+02 +9.250000000000000e+01
+9.941210000000000e+02 +1.635000000000000e+02
+7.961500000000000e+02 +1.360000000000000e+02
+9.962809999999999e+02 +2.790000000000000e+02
+4.567290000000000e+02 +3.450000000000000e+01
+7.961380000000000e+02 +1.455000000000000e+02
+2.031180000000000e+03 +2.930000000000000e+02
+1.017330000000000e+03 +2.390000000000000e+02
+1.298220000000000e+03 +3.140000000000000e+02
+9.901849999999999e+02 +2.745000000000000e+02
+8.506430000000000e+02 +1.145000000000000e+02
+2.818490000000000e+02 +1.035000000000000e+02
+2.794370000000000e+02 +1.045000000000000e+02
+9.799000000000000e+02 +1.690000000000000e+02
+9.782530000000000e+02 +1.660000000000000e+02
+6.855760000000000e+02 +6.050000000000000e+01
+1.030710000000000e+03 +2.200000000000000e+02
+1.481770000000000e+03 +4.440000000000000e+02
+1.311080000000000e+03 +3.130000000000000e+02
+8.084169999999998e+02 +1.590000000000000e+02
+1.027680000000000e+03 +2.560000000000000e+02
+1.011740000000000e+03 +2.215000000000000e+02
+1.040660000000000e+03 +1.630000000000000e+02
+9.983869999999999e+02 +2.025000000000000e+02
+2.840520000000000e+02 +1.020000000000000e+02
+2.812750000000000e+02 +9.250000000000000e+01
+1.285250000000000e+03 +3.270000000000000e+02
+9.906240000000000e+02 +1.840000000000000e+02
+1.007930000000000e+03 +2.265000000000000e+02
+1.990490000000000e+03 +3.400000000000000e+02
+1.369720000000000e+03 +3.185000000000000e+02
+1.045370000000000e+03 +1.780000000000000e+02
+9.881390000000000e+02 +1.555000000000000e+02
+9.153080000000000e+02 +1.885000000000000e+02
+1.010830000000000e+03 +2.555000000000000e+02
+1.139550000000000e+03 +1.460000000000000e+02
+1.011540000000000e+03 +1.155000000000000e+02
+2.817420000000000e+02 +7.800000000000000e+01
+1.375090000000000e+03 +3.625000000000000e+02
+9.830230000000000e+02 +1.625000000000000e+02
+9.535510000000000e+02 +1.500000000000000e+02
+1.403960000000000e+03 +3.795000000000000e+02
+1.406010000000000e+03 +3.640000000000000e+02
+6.232959999999998e+02 +1.440000000000000e+02
+1.279970000000000e+03 +2.525000000000000e+02
+6.678780000000000e+02 +4.350000000000000e+01
+1.371980000000000e+03 +2.880000000000000e+02
+9.162610000000000e+02 +1.645000000000000e+02
+6.207010000000000e+02 +1.420000000000000e+02
+1.307410000000000e+03 +2.655000000000000e+02
+6.268740000000000e+02 +1.295000000000000e+02
+6.347530000000000e+02 +1.430000000000000e+02
+6.622960000000000e+02 +1.265000000000000e+02
+9.949180000000000e+02 +1.935000000000000e+02
+1.059760000000000e+03 +1.875000000000000e+02
+2.818570000000000e+02 +5.400000000000000e+01
+1.423080000000000e+03 +3.115000000000000e+02
+1.398540000000000e+03 +3.415000000000000e+02
+9.928080000000000e+02 +2.055000000000000e+02
+6.196380000000000e+02 +1.110000000000000e+02
+6.591369999999999e+02 +1.210000000000000e+02
+8.805710000000000e+02 +1.270000000000000e+02
+8.953339999999999e+02 +1.240000000000000e+02
+1.288980000000000e+03 +2.080000000000000e+02
+9.854000000000000e+02 +1.780000000000000e+02
+9.256750000000000e+02 +1.880000000000000e+02
+1.006670000000000e+03 +1.950000000000000e+02
+2.818420000000000e+02 +5.100000000000000e+01
+1.414440000000000e+03 +3.550000000000000e+02
+9.866470000000000e+02 +1.700000000000000e+02
+1.402850000000000e+03 +2.855000000000000e+02
+6.251160000000000e+02 +1.375000000000000e+02
+1.063560000000000e+03 +1.760000000000000e+02
+1.021990000000000e+03 +1.860000000000000e+02
+9.065110000000000e+02 +1.825000000000000e+02
+6.190060000000000e+02 +1.190000000000000e+02
+8.989870000000000e+02 +1.145000000000000e+02
+1.094400000000000e+03 +1.720000000000000e+02
+6.620239999999999e+02 +1.205000000000000e+02
+7.895980000000002e+02 +1.445000000000000e+02
+9.897250000000000e+02 +1.605000000000000e+02
+9.003020000000000e+02 +1.270000000000000e+02
+9.824800000000000e+02 +1.520000000000000e+02
+2.813320000000000e+02 +6.500000000000000e+01
+1.032050000000000e+03 +1.635000000000000e+02
+6.396390000000000e+02 +1.215000000000000e+02
+8.882760000000002e+02 +1.180000000000000e+02
+1.845520000000000e+03 +4.035000000000000e+02
+9.130590000000000e+02 +1.770000000000000e+02
+1.014450000000000e+03 +1.520000000000000e+02
+6.346799999999999e+02 +1.200000000000000e+02
+6.580650000000001e+02 +7.100000000000000e+01
+9.955490000000000e+02 +1.660000000000000e+02
+1.848280000000000e+03 +3.900000000000000e+02
+8.966980000000000e+02 +1.135000000000000e+02
+2.842320000000000e+02 +8.150000000000000e+01
+6.253780000000000e+02 +9.650000000000000e+01
+9.056240000000000e+02 +9.300000000000000e+01
+1.038670000000000e+03 +1.400000000000000e+02
+1.849460000000000e+03 +3.810000000000000e+02
+6.226990000000002e+02 +9.800000000000000e+01
+1.858830000000000e+03 +3.670000000000000e+02
+1.075460000000000e+03 +1.385000000000000e+02
+1.493320000000000e+03 +3.670000000000000e+02
+6.609000000000000e+02 +9.200000000000000e+01
+2.788510000000000e+02 +4.100000000000000e+01
+9.766050000000000e+02 +1.440000000000000e+02
+1.768110000000000e+03 +4.170000000000000e+02
+1.110570000000000e+03 +1.255000000000000e+02
+1.060060000000000e+03 +1.935000000000000e+02
+1.745590000000000e+03 +3.650000000000000e+02
+6.249910000000000e+02 +8.000000000000000e+01
+6.182780000000000e+02 +1.035000000000000e+02
+1.063090000000000e+02 +9.500000000000000e+00
+1.837250000000000e+03 +4.495000000000000e+02
+6.605470000000000e+02 +6.450000000000000e+01
+8.953120000000000e+02 +1.035000000000000e+02
+9.745119999999999e+02 +1.410000000000000e+02
+8.196690000000000e+02 +9.150000000000000e+01
+9.985380000000000e+02 +1.320000000000000e+02
+9.718220000000000e+02 +1.785000000000000e+02
+6.239670000000000e+02 +6.950000000000000e+01
+6.209680000000002e+02 +6.850000000000000e+01
+1.201700000000000e+03 +2.920000000000000e+02
+2.885880000000000e+02 +6.000000000000000e+01
+1.876810000000000e+03 +4.125000000000000e+02
+1.313360000000000e+03 +2.755000000000000e+02
+7.817270000000000e+02 +1.385000000000000e+02
+1.107760000000000e+03 +1.655000000000000e+02
+1.865340000000000e+03 +3.845000000000000e+02
+9.415140000000000e+02 +1.815000000000000e+02
+6.281010000000000e+02 +6.750000000000000e+01
+7.752370000000000e+02 +1.470000000000000e+02
+9.087350000000000e+02 +1.115000000000000e+02
+1.067460000000000e+03 +1.375000000000000e+02
+1.440450000000000e+03 +3.180000000000000e+02
+6.244900000000000e+02 +6.200000000000000e+01
+1.851990000000000e+03 +4.100000000000000e+02
+1.867790000000000e+03 +4.245000000000000e+02
+9.890940000000001e+02 +2.170000000000000e+02
+9.594310000000000e+02 +2.455000000000000e+02
+1.446790000000000e+03 +3.010000000000000e+02
+1.418460000000000e+03 +2.555000000000000e+02
+7.886990000000000e+02 +1.325000000000000e+02
+1.871100000000000e+03 +3.760000000000000e+02
+9.087809999999999e+02 +1.585000000000000e+02
+7.723360000000000e+02 +1.430000000000000e+02
+1.130740000000000e+03 +1.235000000000000e+02
+9.771070000000000e+02 +1.545000000000000e+02
+6.757260000000001e+02 +1.765000000000000e+02
+1.846280000000000e+03 +4.215000000000000e+02
+1.409370000000000e+03 +2.360000000000000e+02
+1.063760000000000e+03 +1.750000000000000e+02
+7.685620000000000e+02 +1.345000000000000e+02
+9.093180000000000e+02 +9.700000000000000e+01
+1.086500000000000e+02 +1.000000000000000e+01
+7.802130000000002e+02 +1.280000000000000e+02
+8.734310000000000e+02 +1.045000000000000e+02
+1.123880000000000e+03 +1.245000000000000e+02
+2.672500000000000e+02 +5.150000000000000e+01
+6.631010000000001e+02 +1.750000000000000e+02
+1.614740000000000e+03 +3.640000000000000e+02
+9.760960000000000e+02 +1.380000000000000e+02
+7.854800000000000e+02 +1.265000000000000e+02
+1.011070000000000e+03 +1.110000000000000e+02
+8.978099999999999e+02 +3.010000000000000e+02
+7.911319999999999e+02 +1.250000000000000e+02
+3.454450000000000e+02 +1.010000000000000e+02
+4.130720000000000e+02 +5.750000000000000e+01
+6.800920000000000e+02 +1.740000000000000e+02
+8.592500000000000e+02 +8.150000000000000e+01
+6.922689999999999e+02 +1.400000000000000e+02
+6.892189999999998e+02 +1.340000000000000e+02
+9.836150000000000e+02 +1.370000000000000e+02
+4.922920000000000e+02 +1.900000000000000e+02
+1.590920000000000e+03 +2.620000000000000e+02
+9.025230000000000e+02 +2.705000000000000e+02
+7.853850000000000e+02 +1.345000000000000e+02
+3.280780000000001e+02 +9.100000000000000e+01
+7.826870000000000e+02 +1.370000000000000e+02
+8.750549999999999e+02 +1.725000000000000e+02
+6.925730000000000e+02 +1.405000000000000e+02
+6.904040000000000e+02 +8.300000000000000e+01
+9.834170000000000e+02 +1.295000000000000e+02
+7.773240000000000e+02 +1.255000000000000e+02
+4.884930000000001e+02 +1.925000000000000e+02
+7.874760000000001e+02 +1.250000000000000e+02
+7.708270000000000e+02 +1.275000000000000e+02
+8.936419999999998e+02 +2.750000000000000e+02
+8.842919999999998e+02 +3.125000000000000e+02
+7.843830000000000e+02 +1.055000000000000e+02
+7.880350000000000e+02 +1.255000000000000e+02
+1.526630000000000e+03 +2.230000000000000e+02
+1.299230000000000e+03 +2.350000000000000e+02
+7.693389999999998e+02 +1.185000000000000e+02
+1.119600000000000e+03 +1.610000000000000e+02
+1.750860000000000e+03 +4.410000000000000e+02
+9.907050000000000e+02 +1.390000000000000e+02
+7.811730000000000e+02 +1.025000000000000e+02
+7.773969999999998e+02 +1.225000000000000e+02
+1.060610000000000e+03 +1.880000000000000e+02
+7.788290000000000e+02 +1.610000000000000e+02
+9.897470000000000e+02 +1.440000000000000e+02
+8.976760000000000e+02 +2.890000000000000e+02
+8.901669999999998e+02 +1.125000000000000e+02
+9.913660000000000e+02 +1.325000000000000e+02
+1.085680000000000e+03 +2.970000000000000e+02
+5.073800000000000e+02 +1.935000000000000e+02
+7.755820000000000e+02 +1.195000000000000e+02
+1.607480000000000e+03 +3.605000000000000e+02
+7.663700000000000e+02 +1.175000000000000e+02
+1.072270000000000e+02 +1.000000000000000e+01
+7.571760000000000e+02 +9.700000000000000e+01
+1.037240000000000e+03 +1.375000000000000e+02
+1.505810000000000e+03 +1.285000000000000e+02
+7.710510000000000e+02 +1.175000000000000e+02
+9.066390000000000e+02 +1.095000000000000e+02
+7.509670000000000e+02 +1.905000000000000e+02
+9.732340000000000e+02 +1.195000000000000e+02
+9.980839999999999e+02 +1.285000000000000e+02
+4.945080000000000e+02 +1.975000000000000e+02
+7.826350000000000e+02 +1.195000000000000e+02
+1.072110000000000e+03 +1.655000000000000e+02
+8.803539999999998e+02 +1.825000000000000e+02
+1.609130000000000e+03 +3.470000000000000e+02
+7.640720000000000e+02 +1.150000000000000e+02
+1.080420000000000e+03 +1.565000000000000e+02
+1.911300000000000e+03 +2.905000000000000e+02
+9.127450000000000e+02 +9.700000000000000e+01
+1.151240000000000e+03 +2.680000000000000e+02
+6.397310000000000e+02 +1.065000000000000e+02
+1.754020000000000e+03 +4.235000000000000e+02
+1.558910000000000e+03 +2.105000000000000e+02
+7.831460000000002e+02 +1.030000000000000e+02
+7.692810000000002e+02 +1.140000000000000e+02
+5.057190000000000e+02 +1.975000000000000e+02
+7.917700000000000e+02 +1.035000000000000e+02
+1.336500000000000e+03 +2.165000000000000e+02
+7.933660000000001e+02 +1.135000000000000e+02
+1.425960000000000e+03 +2.430000000000000e+02
+7.657060000000000e+02 +9.400000000000000e+01
+8.656640000000000e+02 +1.655000000000000e+02
+1.385240000000000e+03 +1.985000000000000e+02
+1.109990000000000e+03 +2.920000000000000e+02
+8.949180000000000e+02 +8.700000000000000e+01
+8.993090000000000e+02 +7.250000000000000e+01
+4.881460000000000e+02 +1.970000000000000e+02
+1.435320000000000e+03 +2.515000000000000e+02
+1.132710000000000e+03 +1.830000000000000e+02
+7.681310000000002e+02 +9.100000000000000e+01
+1.287190000000000e+03 +2.030000000000000e+02
+5.132320000000000e+02 +5.900000000000000e+01
+9.998890000000000e+02 +1.365000000000000e+02
+7.750610000000000e+02 +1.105000000000000e+02
+1.280690000000000e+03 +2.020000000000000e+02
+7.489180000000000e+02 +8.400000000000000e+01
+8.639780000000002e+02 +8.500000000000000e+01
+8.884480000000000e+02 +8.150000000000000e+01
+5.014090000000000e+02 +2.000000000000000e+02
+1.286150000000000e+03 +1.995000000000000e+02
+1.938060000000000e+03 +3.575000000000000e+02
+8.692719999999998e+02 +1.560000000000000e+02
+6.662600000000000e+02 +1.300000000000000e+02
+8.588730000000000e+02 +1.215000000000000e+02
+8.397200000000000e+02 +1.665000000000000e+02
+9.751090000000000e+02 +1.225000000000000e+02
+7.747550000000000e+02 +1.110000000000000e+02
+1.314690000000000e+03 +1.935000000000000e+02
+4.981690000000000e+02 +1.985000000000000e+02
+1.319540000000000e+03 +1.910000000000000e+02
+7.754030000000000e+02 +1.095000000000000e+02
+8.891000000000000e+02 +8.650000000000000e+01
+1.387980000000000e+03 +3.625000000000000e+02
+1.327840000000000e+03 +3.510000000000000e+02
+1.301530000000000e+03 +1.795000000000000e+02
+7.894050000000000e+02 +9.000000000000000e+01
+3.358610000000000e+02 +2.750000000000000e+01
+9.884450000000001e+02 +1.195000000000000e+02
+7.412530000000000e+02 +1.360000000000000e+02
+4.870380000000000e+02 +1.690000000000000e+02
+1.311890000000000e+03 +1.935000000000000e+02
+7.743980000000000e+02 +1.050000000000000e+02
+1.295980000000000e+03 +1.640000000000000e+02
+7.697050000000000e+02 +8.350000000000000e+01
+8.468630000000001e+02 +7.450000000000000e+01
+1.340240000000000e+03 +3.895000000000000e+02
+1.040700000000000e+03 +1.820000000000000e+02
+1.335790000000000e+03 +3.775000000000000e+02
+4.895490000000000e+02 +1.900000000000000e+02
+1.301820000000000e+03 +1.670000000000000e+02
+7.518400000000000e+02 +7.900000000000000e+01
+4.825490000000000e+02 +4.200000000000000e+01
+8.610250000000000e+02 +2.130000000000000e+02
+1.300750000000000e+03 +1.905000000000000e+02
+7.783570000000000e+02 +1.000000000000000e+02
+8.797980000000000e+02 +1.700000000000000e+02
+4.896410000000000e+02 +1.645000000000000e+02
+1.336120000000000e+03 +3.500000000000000e+02
+1.356340000000000e+03 +2.805000000000000e+02
+7.469119999999998e+02 +1.135000000000000e+02
+7.742910000000001e+02 +8.100000000000000e+01
+8.980660000000000e+02 +8.600000000000000e+01
+1.305760000000000e+03 +1.795000000000000e+02
+1.294020000000000e+03 +1.590000000000000e+02
+1.953250000000000e+03 +3.650000000000000e+02
+8.734470000000000e+02 +1.425000000000000e+02
+4.911120000000000e+02 +1.880000000000000e+02
+7.774160000000001e+02 +8.050000000000000e+01
+7.792130000000002e+02 +9.950000000000000e+01
+8.784700000000000e+02 +7.750000000000000e+01
+1.326910000000000e+03 +3.070000000000000e+02
+1.382900000000000e+03 +3.445000000000000e+02
+4.888930000000000e+02 +1.555000000000000e+02
+7.439390000000000e+02 +1.030000000000000e+02
+7.025910000000000e+02 +1.850000000000000e+02
+7.912710000000002e+02 +9.900000000000000e+01
+7.770680000000000e+02 +7.800000000000000e+01
+1.298160000000000e+03 +1.575000000000000e+02
+7.665970000000000e+02 +9.200000000000000e+01
+1.074150000000000e+03 +4.040000000000000e+02
+6.851060000000001e+02 +2.345000000000000e+02
+1.301660000000000e+03 +1.565000000000000e+02
+1.166040000000000e+03 +2.855000000000000e+02
+9.030180000000000e+02 +8.300000000000000e+01
+1.127710000000000e+03 +2.175000000000000e+02
+1.294960000000000e+03 +1.445000000000000e+02
+7.810280000000000e+02 +9.300000000000000e+01
+7.801650000000000e+02 +7.850000000000000e+01
+1.067790000000000e+03 +3.680000000000000e+02
+7.610419999999998e+02 +9.450000000000000e+01
+1.284280000000000e+03 +1.495000000000000e+02
+9.016650000000000e+02 +8.200000000000000e+01
+1.112970000000000e+02 +1.350000000000000e+01
+3.435480000000000e+02 +3.500000000000000e+01
+1.321040000000000e+03 +3.455000000000000e+02
+9.266190000000000e+02 +1.845000000000000e+02
+8.796920000000000e+02 +1.710000000000000e+02
+7.837700000000000e+02 +7.650000000000000e+01
+7.939720000000000e+02 +9.350000000000000e+01
+1.306270000000000e+03 +3.880000000000000e+02
+7.611180000000001e+02 +9.550000000000000e+01
+1.073210000000000e+03 +3.745000000000000e+02
+1.102880000000000e+03 +1.585000000000000e+02
+7.812270000000000e+02 +7.600000000000000e+01
+9.751900000000001e+02 +1.565000000000000e+02
+6.913610000000001e+02 +1.765000000000000e+02
+6.639430000000000e+02 +3.330000000000000e+02
+1.072530000000000e+03 +3.505000000000000e+02
+1.306800000000000e+03 +3.255000000000000e+02
+7.840369999999998e+02 +7.600000000000000e+01
+1.114430000000000e+03 +3.670000000000000e+02
+8.748739999999998e+02 +7.700000000000000e+01
+7.448630000000001e+02 +5.550000000000000e+01
+6.703260000000000e+02 +1.730000000000000e+02
+1.308300000000000e+03 +1.495000000000000e+02
+7.764920000000000e+02 +9.150000000000000e+01
+1.307640000000000e+03 +3.330000000000000e+02
+7.757040000000000e+02 +6.700000000000000e+01
+7.742030000000000e+02 +8.800000000000000e+01
+1.122050000000000e+03 +4.055000000000000e+02
+7.696460000000002e+02 +7.000000000000000e+01
+6.799000000000000e+02 +2.060000000000000e+02
+1.637890000000000e+03 +2.900000000000000e+02
+9.630230000000000e+02 +1.465000000000000e+02
+7.822970000000000e+02 +8.800000000000000e+01
+2.308980000000000e+03 +4.900000000000000e+02
+1.591350000000000e+03 +2.335000000000000e+02
+7.658630000000001e+02 +1.190000000000000e+02
+7.796550000000000e+02 +6.750000000000000e+01
+7.873260000000000e+02 +8.700000000000000e+01
+2.315500000000000e+03 +4.655000000000000e+02
+6.524670000000000e+02 +3.185000000000000e+02
+6.910010000000002e+02 +2.310000000000000e+02
+2.911920000000000e+02 +3.200000000000000e+01
+9.383850000000000e+02 +2.950000000000000e+02
+7.511770000000000e+02 +6.300000000000000e+01
+8.146810000000000e+02 +2.745000000000000e+02
+1.063950000000000e+03 +3.100000000000000e+02
+8.456910000000000e+02 +1.465000000000000e+02
+3.618580000000000e+02 +1.065000000000000e+02
+1.072540000000000e+03 +3.035000000000000e+02
+6.623300000000000e+02 +1.005000000000000e+02
+1.010150000000000e+03 +2.020000000000000e+02
+8.261250000000000e+02 +2.405000000000000e+02
+6.826100000000000e+02 +1.695000000000000e+02
+9.854660000000000e+02 +2.815000000000000e+02
+1.583060000000000e+03 +3.330000000000000e+02
+6.162710000000000e+02 +3.620000000000000e+02
+1.107940000000000e+03 +3.305000000000000e+02
+1.084470000000000e+03 +2.910000000000000e+02
+7.050650000000001e+02 +2.055000000000000e+02
+7.152760000000002e+02 +1.430000000000000e+02
+7.004770000000000e+02 +1.970000000000000e+02
+8.308689999999998e+02 +1.655000000000000e+02
+1.536290000000000e+03 +2.810000000000000e+02
+1.163520000000000e+03 +2.500000000000000e+02
+1.050090000000000e+03 +2.575000000000000e+02
+1.664660000000000e+03 +2.745000000000000e+02
+1.148160000000000e+03 +3.290000000000000e+02
+6.338180000000000e+02 +1.480000000000000e+02
+4.858720000000000e+02 +3.400000000000000e+01
+6.359870000000000e+02 +1.535000000000000e+02
+2.039270000000000e+03 +4.955000000000000e+02
+6.963160000000000e+02 +2.760000000000000e+02
+6.479190000000000e+02 +1.565000000000000e+02
+9.190309999999999e+02 +2.165000000000000e+02
+8.488430000000002e+02 +1.630000000000000e+02
+9.391710000000000e+02 +2.845000000000000e+02
+1.144970000000000e+03 +2.795000000000000e+02
+6.728150000000001e+02 +3.560000000000000e+02
+1.064680000000000e+03 +2.615000000000000e+02
+6.340190000000000e+02 +1.485000000000000e+02
+6.376770000000000e+02 +1.290000000000000e+02
+7.324739999999998e+02 +8.850000000000000e+01
+6.624100000000000e+02 +1.595000000000000e+02
+9.289190000000000e+02 +2.100000000000000e+02
+6.807750000000000e+02 +2.530000000000000e+02
+6.588030000000000e+02 +1.405000000000000e+02
+1.143340000000000e+03 +2.845000000000000e+02
+9.941460000000000e+02 +1.990000000000000e+02
+1.078020000000000e+03 +2.610000000000000e+02
+2.864180000000000e+02 +1.590000000000000e+02
+6.179380000000000e+02 +1.445000000000000e+02
+6.159420000000000e+02 +1.405000000000000e+02
+2.093700000000000e+03 +5.115000000000000e+02
+6.182260000000000e+02 +1.085000000000000e+02
+1.066560000000000e+03 +2.090000000000000e+02
+6.847719999999998e+02 +2.760000000000000e+02
+6.577940000000000e+02 +1.440000000000000e+02
+9.036670000000000e+02 +2.325000000000000e+02
+6.486030000000002e+02 +1.465000000000000e+02
+6.081120000000000e+02 +1.490000000000000e+02
+6.541210000000000e+02 +1.115000000000000e+02
+2.093050000000000e+03 +5.220000000000000e+02
+1.047860000000000e+03 +1.990000000000000e+02
+1.140060000000000e+03 +2.015000000000000e+02
+7.031930000000000e+02 +2.170000000000000e+02
+6.378510000000000e+02 +1.335000000000000e+02
+1.287160000000000e+03 +3.155000000000000e+02
+6.406440000000000e+02 +1.305000000000000e+02
+9.411000000000000e+02 +2.360000000000000e+02
+6.419950000000000e+02 +1.070000000000000e+02
+1.397010000000000e+03 +3.290000000000000e+02
+1.056290000000000e+03 +3.830000000000000e+02
+6.387520000000000e+02 +1.250000000000000e+02
+6.361120000000000e+02 +1.165000000000000e+02
+2.067940000000000e+03 +5.210000000000000e+02
+6.200660000000000e+02 +8.750000000000000e+01
+6.773480000000002e+02 +1.690000000000000e+02
+8.868240000000000e+02 +2.240000000000000e+02
+1.033020000000000e+03 +2.220000000000000e+02
+6.785910000000000e+02 +1.000000000000000e+02
+6.644540000000000e+02 +1.310000000000000e+02
+8.897880000000000e+02 +1.615000000000000e+02
+7.387410000000001e+02 +2.500000000000000e+02
+6.369450000000001e+02 +7.500000000000000e+01
+6.363930000000000e+02 +9.750000000000000e+01
+1.585550000000000e+03 +3.050000000000000e+02
+1.159480000000000e+03 +2.735000000000000e+02
+1.168430000000000e+03 +3.015000000000000e+02
+6.667160000000000e+02 +9.950000000000000e+01
+6.604889999999998e+02 +6.500000000000000e+01
+1.332940000000000e+03 +2.795000000000000e+02
+6.386369999999999e+02 +1.245000000000000e+02
+1.141600000000000e+03 +3.815000000000000e+02
+1.076360000000000e+03 +3.780000000000000e+02
+9.375930000000000e+02 +1.875000000000000e+02
+6.590620000000000e+02 +9.950000000000000e+01
+6.357809999999999e+02 +7.750000000000000e+01
+1.236860000000000e+03 +3.360000000000000e+02
+1.164240000000000e+03 +3.125000000000000e+02
+9.315460000000000e+02 +2.095000000000000e+02
+8.969360000000000e+02 +8.150000000000000e+01
+7.016220000000000e+02 +1.135000000000000e+02
+1.849060000000000e+03 +4.740000000000000e+02
+6.701160000000001e+02 +1.010000000000000e+02
+6.568780000000000e+02 +9.600000000000000e+01
+2.127850000000000e+03 +4.070000000000000e+02
+6.408780000000000e+02 +1.210000000000000e+02
+1.061360000000000e+03 +1.945000000000000e+02
+6.352640000000000e+02 +1.215000000000000e+02
+6.498130000000000e+02 +7.350000000000000e+01
+6.356190000000000e+02 +6.350000000000000e+01
+7.916690000000000e+02 +2.240000000000000e+02
+1.201330000000000e+03 +3.155000000000000e+02
+6.362930000000000e+02 +1.165000000000000e+02
+6.516780000000000e+02 +1.040000000000000e+02
+8.048800000000000e+02 +2.170000000000000e+02
+7.065520000000000e+02 +1.445000000000000e+02
+9.994000000000000e+02 +1.925000000000000e+02
+6.689210000000000e+02 +1.280000000000000e+02
+6.373000000000000e+02 +1.080000000000000e+02
+7.920230000000000e+02 +2.020000000000000e+02
+6.600450000000000e+02 +9.900000000000000e+01
+8.696230000000000e+02 +1.015000000000000e+02
+2.699940000000000e+03 +5.140000000000000e+02
+8.969550000000000e+02 +1.140000000000000e+02
+7.729710000000000e+02 +1.465000000000000e+02
+1.076230000000000e+03 +3.765000000000000e+02
+6.541369999999999e+02 +7.900000000000000e+01
+1.601440000000000e+03 +3.260000000000000e+02
+1.260850000000000e+03 +2.520000000000000e+02
+6.545180000000000e+02 +1.030000000000000e+02
+6.356210000000000e+02 +8.300000000000000e+01
+8.076160000000001e+02 +2.145000000000000e+02
+1.600380000000000e+03 +3.185000000000000e+02
+8.365410000000001e+02 +9.750000000000000e+01
+1.236160000000000e+03 +1.815000000000000e+02
+4.866730000000000e+02 +5.150000000000000e+01
+8.976189999999998e+02 +4.010000000000000e+02
+6.349090000000000e+02 +9.700000000000000e+01
+1.540490000000000e+03 +2.760000000000000e+02
+1.007290000000000e+03 +1.395000000000000e+02
+6.164299999999999e+02 +7.950000000000000e+01
+1.863700000000000e+03 +4.750000000000000e+02
+1.126350000000000e+03 +1.940000000000000e+02
+1.019640000000000e+03 +1.650000000000000e+02
+1.078720000000000e+03 +2.390000000000000e+02
+6.854670000000000e+02 +1.940000000000000e+02
+8.547389999999998e+02 +1.105000000000000e+02
+1.598230000000000e+03 +4.980000000000000e+02
+1.070750000000000e+03 +2.585000000000000e+02
+6.599980000000000e+02 +2.080000000000000e+02
+1.636130000000000e+03 +4.905000000000000e+02
+1.040930000000000e+03 +1.745000000000000e+02
+1.625760000000000e+03 +4.930000000000000e+02
+7.039960000000002e+02 +1.485000000000000e+02
+1.069300000000000e+03 +1.835000000000000e+02
+9.918300000000000e+02 +1.260000000000000e+02
+1.077000000000000e+03 +2.140000000000000e+02
+6.885210000000002e+02 +2.075000000000000e+02
+8.940110000000002e+02 +1.095000000000000e+02
+1.268930000000000e+03 +2.410000000000000e+02
+1.628280000000000e+03 +4.535000000000000e+02
+1.046930000000000e+03 +1.965000000000000e+02
+1.555710000000000e+03 +2.185000000000000e+02
+1.325530000000000e+03 +2.410000000000000e+02
+6.968930000000000e+02 +1.840000000000000e+02
+6.794850000000000e+02 +1.970000000000000e+02
+1.095030000000000e+03 +1.965000000000000e+02
+1.086460000000000e+03 +1.865000000000000e+02
+1.783560000000000e+03 +3.065000000000000e+02
+1.075570000000000e+03 +1.265000000000000e+02
+6.972650000000000e+02 +1.725000000000000e+02
+1.931700000000000e+03 +4.235000000000000e+02
+1.220200000000000e+03 +2.010000000000000e+02
+1.921010000000000e+03 +3.605000000000000e+02
+6.833539999999998e+02 +1.425000000000000e+02
+7.099440000000000e+02 +1.640000000000000e+02
+1.573440000000000e+03 +2.510000000000000e+02
+8.911700000000000e+02 +1.010000000000000e+02
+1.132360000000000e+03 +1.530000000000000e+02
+1.079370000000000e+03 +2.400000000000000e+02
+1.072110000000000e+03 +1.760000000000000e+02
+6.593250000000000e+02 +1.285000000000000e+02
+1.671870000000000e+03 +2.825000000000000e+02
+1.617620000000000e+03 +2.635000000000000e+02
+1.508630000000000e+03 +4.640000000000000e+02
+1.077040000000000e+03 +2.475000000000000e+02
+8.773550000000000e+02 +1.970000000000000e+02
+8.757289999999998e+02 +1.700000000000000e+02
+8.704010000000002e+02 +1.905000000000000e+02
+1.073050000000000e+03 +2.235000000000000e+02
+1.079090000000000e+03 +1.900000000000000e+02
+1.671450000000000e+03 +3.000000000000000e+02
+1.122310000000000e+03 +3.380000000000000e+02
+9.743240000000000e+02 +1.805000000000000e+02
+8.396220000000000e+02 +8.400000000000000e+01
+1.572960000000000e+03 +2.630000000000000e+02
+9.004190000000000e+02 +1.040000000000000e+02
+8.075350000000000e+02 +2.295000000000000e+02
+1.063800000000000e+03 +2.350000000000000e+02
+1.585410000000000e+03 +2.870000000000000e+02
+8.045139999999999e+02 +2.205000000000000e+02
+1.086450000000000e+03 +1.795000000000000e+02
+1.585490000000000e+03 +2.305000000000000e+02
+1.107180000000000e+03 +1.600000000000000e+02
+1.039640000000000e+03 +1.230000000000000e+02
+9.926140000000000e+02 +1.295000000000000e+02
+1.290030000000000e+03 +4.100000000000000e+02
+9.878460000000000e+02 +1.735000000000000e+02
+1.383040000000000e+03 +2.305000000000000e+02
+7.924420000000000e+02 +1.460000000000000e+02
+1.008710000000000e+03 +2.285000000000000e+02
+1.394770000000000e+03 +2.430000000000000e+02
+1.053060000000000e+03 +1.460000000000000e+02
+1.569380000000000e+03 +4.420000000000000e+02
+1.389640000000000e+03 +3.045000000000000e+02
+8.668650000000000e+02 +9.450000000000000e+01
+1.048030000000000e+03 +1.785000000000000e+02
+6.985870000000000e+02 +1.795000000000000e+02
+9.933440000000001e+02 +7.700000000000000e+01
+1.086490000000000e+03 +1.270000000000000e+02
+8.860930000000002e+02 +7.950000000000000e+01
+1.911830000000000e+03 +4.310000000000000e+02
+1.130100000000000e+03 +2.130000000000000e+02
+1.079450000000000e+03 +2.365000000000000e+02
+9.891140000000000e+02 +1.770000000000000e+02
+1.012410000000000e+03 +2.520000000000000e+02
+1.379240000000000e+03 +2.975000000000000e+02
+1.575970000000000e+03 +1.865000000000000e+02
+1.006150000000000e+03 +2.395000000000000e+02
+1.064940000000000e+03 +1.545000000000000e+02
+7.027539999999998e+02 +1.755000000000000e+02
+1.384680000000000e+03 +2.260000000000000e+02
+8.033900000000000e+02 +1.700000000000000e+02
+1.107800000000000e+03 +1.325000000000000e+02
+1.096760000000000e+03 +1.575000000000000e+02
+9.764670000000000e+02 +1.025000000000000e+02
+1.047550000000000e+03 +2.260000000000000e+02
+1.014880000000000e+03 +2.725000000000000e+02
+2.818490000000000e+02 +9.750000000000000e+01
+2.794470000000000e+02 +9.800000000000000e+01
+1.338140000000000e+03 +2.510000000000000e+02
+1.065510000000000e+03 +1.910000000000000e+02
+1.387010000000000e+03 +2.630000000000000e+02
+8.601260000000002e+02 +9.650000000000000e+01
+9.961200000000000e+02 +1.490000000000000e+02
+2.747480000000000e+02 +8.000000000000000e+01
+1.000690000000000e+03 +1.840000000000000e+02
+9.835700000000001e+02 +1.695000000000000e+02
+9.930510000000000e+02 +2.215000000000000e+02
+1.114660000000000e+03 +2.240000000000000e+02
+1.007980000000000e+03 +1.150000000000000e+02
+1.325970000000000e+03 +2.545000000000000e+02
+1.074460000000000e+03 +2.160000000000000e+02
+1.423690000000000e+03 +4.080000000000000e+02
+1.402510000000000e+03 +3.630000000000000e+02
+1.289320000000000e+03 +2.635000000000000e+02
+9.023150000000001e+02 +8.300000000000000e+01
+1.144130000000000e+03 +1.845000000000000e+02
+1.146110000000000e+03 +2.470000000000000e+02
+9.934950000000000e+02 +1.785000000000000e+02
+6.818120000000000e+02 +6.450000000000000e+01
+9.956720000000000e+02 +1.030000000000000e+02
+1.394380000000000e+03 +4.280000000000000e+02
+7.096519999999998e+02 +7.450000000000000e+01
+1.064460000000000e+03 +2.100000000000000e+02
+9.871559999999999e+02 +1.625000000000000e+02
+1.002060000000000e+03 +2.075000000000000e+02
+8.864939999999998e+02 +2.475000000000000e+02
+1.001300000000000e+03 +2.360000000000000e+02
+9.967600000000000e+02 +9.600000000000000e+01
+1.073290000000000e+03 +2.265000000000000e+02
+8.974470000000000e+02 +1.000000000000000e+02
+8.275419999999998e+02 +8.800000000000000e+01
+1.227070000000000e+03 +2.870000000000000e+02
+9.875850000000000e+02 +1.560000000000000e+02
+6.198300000000000e+02 +1.595000000000000e+02
+1.162960000000000e+03 +3.275000000000000e+02
+1.122320000000000e+03 +2.950000000000000e+02
+1.148620000000000e+03 +1.705000000000000e+02
+1.428820000000000e+03 +4.370000000000000e+02
+6.205850000000000e+02 +1.555000000000000e+02
+1.619850000000000e+03 +4.265000000000000e+02
+9.892340000000000e+02 +1.930000000000000e+02
+1.595400000000000e+03 +4.920000000000000e+02
+2.817890000000000e+02 +3.350000000000000e+01
+1.082490000000000e+03 +2.675000000000000e+02
+1.448210000000000e+03 +2.960000000000000e+02
+9.300119999999999e+02 +2.530000000000000e+02
+9.767340000000000e+02 +1.640000000000000e+02
+8.930939999999998e+02 +1.000000000000000e+02
+6.614100000000000e+02 +1.515000000000000e+02
+1.380870000000000e+03 +4.190000000000000e+02
+1.053000000000000e+03 +1.890000000000000e+02
+6.247840000000000e+02 +1.560000000000000e+02
+6.247110000000000e+02 +1.500000000000000e+02
+1.362360000000000e+03 +3.835000000000000e+02
+1.598730000000000e+03 +4.865000000000000e+02
+1.289700000000000e+03 +3.875000000000000e+02
+1.072770000000000e+03 +2.455000000000000e+02
+1.407040000000000e+03 +4.040000000000000e+02
+1.006260000000000e+03 +2.840000000000000e+02
+1.626190000000000e+03 +4.460000000000000e+02
+1.010740000000000e+03 +2.070000000000000e+02
+8.982160000000000e+02 +9.200000000000000e+01
+1.296840000000000e+03 +3.935000000000000e+02
+1.914850000000000e+03 +5.115000000000000e+02
+9.908800000000000e+02 +1.340000000000000e+02
+9.621609999999999e+02 +1.585000000000000e+02
+6.225810000000000e+02 +1.345000000000000e+02
+2.816630000000000e+02 +8.750000000000000e+01
+2.792490000000000e+02 +9.650000000000000e+01
+1.410610000000000e+03 +4.295000000000000e+02
+1.025900000000000e+03 +1.860000000000000e+02
+1.461200000000000e+03 +3.115000000000000e+02
+9.337880000000000e+02 +2.425000000000000e+02
+1.865770000000000e+03 +4.750000000000000e+02
+8.783260000000000e+02 +7.600000000000000e+01
+1.052040000000000e+03 +2.065000000000000e+02
+6.244400000000001e+02 +1.150000000000000e+02
+6.643110000000000e+02 +1.205000000000000e+02
+6.551750000000000e+02 +4.300000000000000e+01
+2.827820000000000e+02 +5.150000000000000e+01
+1.313730000000000e+03 +2.775000000000000e+02
+9.886120000000000e+02 +1.340000000000000e+02
+9.002510000000002e+02 +6.650000000000000e+01
+2.962260000000000e+02 +9.800000000000000e+01
+6.252130000000002e+02 +1.305000000000000e+02
+6.228600000000000e+02 +1.120000000000000e+02
+1.642170000000000e+03 +4.445000000000000e+02
+8.744000000000000e+02 +7.500000000000000e+01
+1.334910000000000e+03 +3.190000000000000e+02
+6.615089999999999e+02 +9.750000000000000e+01
+1.859530000000000e+03 +4.590000000000000e+02
+6.598910000000002e+02 +8.850000000000000e+01
+1.870670000000000e+03 +4.470000000000000e+02
+2.819980000000000e+02 +3.750000000000000e+01
+6.205860000000000e+02 +7.000000000000000e+01
+8.878400000000000e+02 +7.000000000000000e+01
+1.071710000000000e+03 +2.705000000000000e+02
+8.738950000000000e+02 +3.615000000000000e+02
+9.145510000000000e+02 +2.200000000000000e+02
+1.047100000000000e+03 +2.305000000000000e+02
+2.805150000000000e+02 +3.150000000000000e+01
+1.738790000000000e+03 +4.660000000000000e+02
+6.356430000000000e+02 +7.100000000000000e+01
+6.307200000000000e+02 +4.300000000000000e+01
+6.607550000000000e+02 +7.750000000000000e+01
+1.881820000000000e+03 +4.370000000000000e+02
+8.986120000000000e+02 +7.650000000000000e+01
+1.066010000000000e+03 +1.935000000000000e+02
+9.945710000000000e+02 +1.575000000000000e+02
+9.884780000000000e+02 +1.455000000000000e+02
+6.235700000000001e+02 +6.050000000000000e+01
+1.077980000000000e+03 +1.990000000000000e+02
+1.309790000000000e+03 +2.515000000000000e+02
+9.780620000000000e+02 +1.305000000000000e+02
+9.284070000000000e+02 +2.500000000000000e+02
+6.594320000000000e+02 +8.100000000000000e+01
+1.289330000000000e+03 +3.550000000000000e+02
+6.242470000000000e+02 +5.100000000000000e+01
+1.004090000000000e+03 +2.060000000000000e+02
+8.945530000000000e+02 +6.900000000000000e+01
+6.257909999999998e+02 +4.950000000000000e+01
+2.832220000000000e+02 +2.150000000000000e+01
+4.898500000000000e+02 +2.345000000000000e+02
+1.011820000000000e+03 +1.370000000000000e+02
+1.317590000000000e+03 +3.065000000000000e+02
+6.585050000000000e+02 +6.550000000000000e+01
+1.589260000000000e+03 +2.755000000000000e+02
+9.143970000000000e+02 +2.185000000000000e+02
+9.920140000000000e+02 +1.280000000000000e+02
+9.796300000000000e+02 +2.015000000000000e+02
+7.814820000000000e+02 +1.385000000000000e+02
+6.258150000000001e+02 +4.950000000000000e+01
+4.920060000000000e+02 +2.295000000000000e+02
+7.841680000000000e+02 +1.350000000000000e+02
+1.609210000000000e+03 +3.615000000000000e+02
+9.990300000000000e+02 +1.595000000000000e+02
+6.604220000000000e+02 +5.950000000000000e+01
+2.828090000000000e+02 +2.100000000000000e+01
+1.067050000000000e+03 +2.200000000000000e+02
+7.767830000000000e+02 +1.590000000000000e+02
+1.060760000000000e+03 +1.895000000000000e+02
+9.777130000000000e+02 +1.095000000000000e+02
+9.329850000000000e+02 +2.150000000000000e+02
+4.861300000000000e+02 +2.140000000000000e+02
+1.866440000000000e+03 +4.755000000000000e+02
+7.907530000000000e+02 +1.310000000000000e+02
+1.303570000000000e+03 +3.010000000000000e+02
+9.879520000000000e+02 +1.245000000000000e+02
+6.780419999999998e+02 +1.870000000000000e+02
+7.850860000000000e+02 +1.400000000000000e+02
+9.300970000000000e+02 +2.145000000000000e+02
+4.960720000000000e+02 +2.210000000000000e+02
+7.844839999999998e+02 +1.225000000000000e+02
+8.720020000000000e+02 +6.500000000000000e+01
+8.639860000000001e+02 +1.855000000000000e+02
+6.925260000000002e+02 +1.615000000000000e+02
+1.293520000000000e+03 +2.835000000000000e+02
+6.461770000000000e+02 +1.565000000000000e+02
+7.684299999999999e+02 +1.515000000000000e+02
+1.607340000000000e+03 +2.985000000000000e+02
+1.852070000000000e+03 +4.815000000000000e+02
+7.857450000000000e+02 +1.350000000000000e+02
+1.290040000000000e+03 +3.765000000000000e+02
+9.738380000000000e+02 +1.085000000000000e+02
+7.832050000000000e+02 +1.215000000000000e+02
+7.700850000000000e+02 +1.390000000000000e+02
+4.881460000000000e+02 +2.095000000000000e+02
+7.689010000000002e+02 +1.470000000000000e+02
+7.821590000000000e+02 +1.280000000000000e+02
+6.974180000000000e+02 +2.160000000000000e+02
+1.288340000000000e+03 +2.560000000000000e+02
+4.931890000000000e+02 +2.090000000000000e+02
+7.740130000000000e+02 +1.435000000000000e+02
+8.949210000000000e+02 +6.550000000000000e+01
+8.750419999999998e+02 +3.925000000000000e+02
+1.862830000000000e+03 +5.200000000000000e+02
+1.613670000000000e+03 +3.285000000000000e+02
+1.911140000000000e+03 +3.405000000000000e+02
+7.809280000000000e+02 +1.370000000000000e+02
+6.953720000000000e+02 +1.710000000000000e+02
+1.219460000000000e+03 +2.050000000000000e+02
+9.777880000000000e+02 +1.120000000000000e+02
+4.910380000000000e+02 +2.080000000000000e+02
+7.878960000000002e+02 +1.230000000000000e+02
+6.952569999999999e+02 +2.065000000000000e+02
+1.853700000000000e+03 +4.595000000000000e+02
+7.864900000000000e+02 +1.355000000000000e+02
+6.908450000000000e+02 +2.025000000000000e+02
+7.563770000000000e+02 +1.310000000000000e+02
+7.874200000000000e+02 +1.190000000000000e+02
+6.895219999999998e+02 +1.375000000000000e+02
+4.871660000000000e+02 +1.960000000000000e+02
+7.878040000000000e+02 +1.305000000000000e+02
+7.743190000000000e+02 +1.135000000000000e+02
+8.718420000000000e+02 +2.010000000000000e+02
+1.324350000000000e+03 +2.985000000000000e+02
+1.855070000000000e+03 +5.830000000000000e+02
+1.609240000000000e+03 +2.570000000000000e+02
+7.911970000000000e+02 +1.145000000000000e+02
+8.777220000000000e+02 +2.085000000000000e+02
+1.850420000000000e+03 +4.755000000000000e+02
+1.506450000000000e+03 +3.425000000000000e+02
+1.606570000000000e+03 +2.420000000000000e+02
+1.539360000000000e+03 +3.465000000000000e+02
+1.344250000000000e+03 +3.525000000000000e+02
+4.942910000000000e+02 +2.025000000000000e+02
+7.767150000000000e+02 +1.270000000000000e+02
+7.542080000000002e+02 +1.105000000000000e+02
+1.608990000000000e+03 +2.280000000000000e+02
+7.833939999999999e+02 +1.135000000000000e+02
+7.846110000000001e+02 +1.260000000000000e+02
+1.450240000000000e+03 +4.085000000000000e+02
+1.355970000000000e+03 +2.005000000000000e+02
+6.909050000000000e+02 +1.920000000000000e+02
+9.951150000000000e+02 +1.175000000000000e+02
+1.847890000000000e+03 +5.045000000000000e+02
+7.852139999999998e+02 +1.255000000000000e+02
+1.573630000000000e+03 +4.085000000000000e+02
+8.660690000000000e+02 +2.005000000000000e+02
+8.970000000000000e+02 +3.995000000000000e+02
+1.318380000000000e+03 +2.645000000000000e+02
+1.690220000000000e+03 +3.825000000000000e+02
+1.670890000000000e+03 +2.725000000000000e+02
+1.126930000000000e+03 +2.395000000000000e+02
+4.864550000000000e+02 +1.980000000000000e+02
+7.752050000000000e+02 +1.090000000000000e+02
+7.946189999999998e+02 +1.155000000000000e+02
+1.295470000000000e+03 +2.250000000000000e+02
+1.561840000000000e+03 +2.400000000000000e+02
+1.603770000000000e+03 +2.855000000000000e+02
+1.297120000000000e+03 +2.930000000000000e+02
+1.271640000000000e+03 +2.880000000000000e+02
+1.376030000000000e+03 +3.385000000000000e+02
+1.107930000000000e+03 +2.210000000000000e+02
+7.852719999999998e+02 +1.100000000000000e+02
+1.573170000000000e+03 +2.605000000000000e+02
+3.407530000000001e+02 +1.090000000000000e+02
+7.847780000000000e+02 +1.245000000000000e+02
+1.273610000000000e+03 +1.870000000000000e+02
+1.572160000000000e+03 +3.020000000000000e+02
+5.024430000000000e+02 +2.060000000000000e+02
+1.708390000000000e+03 +3.110000000000000e+02
+7.807320000000000e+02 +1.090000000000000e+02
+1.362210000000000e+03 +3.420000000000000e+02
+1.873030000000000e+03 +5.065000000000000e+02
+1.861680000000000e+03 +4.810000000000000e+02
+1.301520000000000e+03 +2.310000000000000e+02
+7.886640000000000e+02 +1.235000000000000e+02
+1.304600000000000e+03 +2.295000000000000e+02
+1.661180000000000e+03 +2.920000000000000e+02
+1.372250000000000e+03 +3.600000000000000e+02
+1.569910000000000e+03 +2.595000000000000e+02
+1.823480000000000e+03 +4.930000000000000e+02
+2.320400000000000e+03 +5.460000000000000e+02
+1.302160000000000e+03 +2.330000000000000e+02
+7.452400000000000e+02 +1.365000000000000e+02
+5.002260000000000e+02 +2.015000000000000e+02
+1.304800000000000e+03 +2.120000000000000e+02
+1.369180000000000e+03 +3.855000000000000e+02
+7.498630000000001e+02 +1.185000000000000e+02
+2.318940000000000e+03 +5.690000000000000e+02
+8.074250000000000e+02 +3.100000000000000e+02
+7.830430000000000e+02 +1.095000000000000e+02
+1.304300000000000e+03 +2.230000000000000e+02
+8.635169999999998e+02 +1.805000000000000e+02
+1.317250000000000e+03 +2.855000000000000e+02
+9.737150000000000e+02 +2.020000000000000e+02
+1.318210000000000e+03 +1.930000000000000e+02
+7.868900000000000e+02 +1.250000000000000e+02
+6.580820000000000e+02 +4.290000000000000e+02
+1.348920000000000e+03 +3.715000000000000e+02
+1.307480000000000e+03 +1.870000000000000e+02
+7.836189999999998e+02 +1.095000000000000e+02
+1.597200000000000e+03 +3.095000000000000e+02
+7.580930000000002e+02 +1.385000000000000e+02
+7.861790000000000e+02 +1.225000000000000e+02
+2.318510000000000e+03 +5.225000000000000e+02
+1.372930000000000e+03 +3.550000000000000e+02
+1.310850000000000e+03 +2.085000000000000e+02
+1.284550000000000e+03 +1.705000000000000e+02
+3.414110000000000e+02 +1.025000000000000e+02
+1.328240000000000e+03 +3.750000000000000e+02
+7.817310000000001e+02 +1.205000000000000e+02
+7.665119999999999e+02 +1.050000000000000e+02
+8.584939999999998e+02 +1.610000000000000e+02
+1.299590000000000e+03 +1.845000000000000e+02
+1.291530000000000e+03 +3.180000000000000e+02
+9.633810000000000e+02 +1.170000000000000e+02
+6.740889999999998e+02 +1.190000000000000e+02
+7.492650000000000e+02 +1.130000000000000e+02
+1.309440000000000e+03 +1.920000000000000e+02
+7.745030000000000e+02 +1.040000000000000e+02
+1.293380000000000e+03 +1.795000000000000e+02
+7.927310000000001e+02 +1.185000000000000e+02
+1.306090000000000e+03 +1.610000000000000e+02
+1.919280000000000e+03 +3.625000000000000e+02
+8.817200000000000e+02 +1.770000000000000e+02
+1.289260000000000e+03 +3.970000000000000e+02
+7.879310000000000e+02 +1.035000000000000e+02
+1.361980000000000e+03 +3.400000000000000e+02
+3.378219999999999e+02 +8.600000000000000e+01
+7.873680000000001e+02 +1.180000000000000e+02
+7.872660000000002e+02 +1.015000000000000e+02
+8.753989999999999e+02 +1.620000000000000e+02
+1.356130000000000e+03 +3.180000000000000e+02
+1.317250000000000e+03 +1.695000000000000e+02
+3.274480000000000e+02 +8.300000000000000e+01
+1.310060000000000e+03 +1.905000000000000e+02
+7.506250000000000e+02 +1.140000000000000e+02
+8.555740000000000e+02 +1.525000000000000e+02
+1.404920000000000e+03 +3.170000000000000e+02
+4.934730000000000e+02 +1.180000000000000e+02
+1.286320000000000e+03 +1.530000000000000e+02
+7.879320000000000e+02 +1.015000000000000e+02
+1.311040000000000e+03 +1.665000000000000e+02
+7.779710000000000e+02 +1.165000000000000e+02
+3.250290000000000e+02 +8.200000000000000e+01
+7.872040000000000e+02 +1.005000000000000e+02
+1.084260000000000e+03 +4.465000000000000e+02
+7.792730000000000e+02 +1.125000000000000e+02
+1.068570000000000e+03 +4.500000000000000e+02
+1.298800000000000e+03 +1.585000000000000e+02
+1.319590000000000e+03 +1.785000000000000e+02
+7.840939999999998e+02 +9.800000000000000e+01
+5.039160000000000e+02 +5.500000000000000e+01
+7.915390000000000e+02 +9.850000000000000e+01
+1.319870000000000e+03 +3.495000000000000e+02
+1.299540000000000e+03 +1.495000000000000e+02
+1.300670000000000e+03 +1.485000000000000e+02
+7.594470000000000e+02 +1.090000000000000e+02
+8.598080000000000e+02 +1.605000000000000e+02
+1.576970000000000e+03 +3.690000000000000e+02
+1.918290000000000e+03 +4.315000000000000e+02
+7.822439999999998e+02 +1.135000000000000e+02
+1.282450000000000e+03 +1.375000000000000e+02
+7.916270000000000e+02 +1.130000000000000e+02
+1.071260000000000e+03 +3.675000000000000e+02
+6.669989999999998e+02 +2.045000000000000e+02
+7.928869999999999e+02 +1.040000000000000e+02
+7.875580000000000e+02 +1.115000000000000e+02
+3.473150000000000e+02 +8.500000000000000e+01
+1.080260000000000e+03 +4.150000000000000e+02
+7.887080000000002e+02 +1.015000000000000e+02
+6.669900000000000e+02 +4.495000000000000e+02
+4.884410000000000e+02 +9.000000000000000e+01
+1.064600000000000e+03 +3.715000000000000e+02
+7.847120000000000e+02 +1.005000000000000e+02
+7.764349999999999e+02 +9.500000000000000e+01
+7.866770000000000e+02 +1.110000000000000e+02
+1.084390000000000e+03 +2.065000000000000e+02
+7.761150000000000e+02 +9.650000000000000e+01
+7.936669999999998e+02 +1.105000000000000e+02
+7.939670000000000e+02 +9.750000000000000e+01
+3.463510000000000e+02 +6.600000000000000e+01
+1.097430000000000e+03 +4.015000000000000e+02
+7.897970000000000e+02 +9.450000000000000e+01
+1.129110000000000e+03 +3.490000000000000e+02
+1.064900000000000e+03 +3.595000000000000e+02
+9.861890000000000e+02 +2.140000000000000e+02
+1.110160000000000e+03 +1.930000000000000e+02
+6.537800000000000e+02 +4.255000000000000e+02
+7.062910000000001e+02 +2.225000000000000e+02
+8.462890000000000e+02 +9.850000000000000e+01
+1.677950000000000e+03 +3.230000000000000e+02
+1.035100000000000e+03 +3.680000000000000e+02
+8.660470000000000e+02 +2.135000000000000e+02
+8.583210000000000e+02 +9.050000000000000e+01
+1.087520000000000e+03 +2.025000000000000e+02
+8.151300000000000e+02 +3.540000000000000e+02
+1.866120000000000e+03 +5.220000000000000e+02
+1.029600000000000e+03 +1.780000000000000e+02
+1.030360000000000e+03 +3.790000000000000e+02
+6.931580000000000e+02 +2.155000000000000e+02
+7.114910000000001e+02 +2.405000000000000e+02
+7.106480000000000e+02 +2.245000000000000e+02
+6.688620000000000e+02 +9.150000000000000e+01
+1.690660000000000e+03 +3.545000000000000e+02
+1.671390000000000e+03 +3.550000000000000e+02
+1.269340000000000e+03 +3.120000000000000e+02
+1.479730000000000e+03 +4.290000000000000e+02
+1.675590000000000e+03 +3.420000000000000e+02
+1.035150000000000e+03 +3.140000000000000e+02
+6.357460000000000e+02 +1.365000000000000e+02
+1.872240000000000e+03 +5.990000000000000e+02
+1.670470000000000e+03 +3.320000000000000e+02
+6.343850000000000e+02 +1.510000000000000e+02
+9.629120000000000e+02 +3.135000000000000e+02
+1.020090000000000e+03 +1.600000000000000e+02
+6.968330000000002e+02 +1.880000000000000e+02
+7.048400000000000e+02 +2.610000000000000e+02
+7.922680000000000e+02 +1.415000000000000e+02
+1.378350000000000e+03 +2.525000000000000e+02
+6.377700000000000e+02 +1.445000000000000e+02
+2.449090000000000e+03 +6.435000000000000e+02
+2.026650000000000e+03 +5.960000000000000e+02
+1.110080000000000e+03 +1.795000000000000e+02
+1.081470000000000e+03 +1.555000000000000e+02
+1.027480000000000e+03 +3.685000000000000e+02
+6.974520000000000e+02 +1.860000000000000e+02
+6.244430000000000e+02 +1.540000000000000e+02
+6.570219999999998e+02 +1.575000000000000e+02
+1.390480000000000e+03 +2.635000000000000e+02
+6.367040000000002e+02 +1.320000000000000e+02
+8.950910000000000e+02 +1.005000000000000e+02
+6.345680000000000e+02 +1.610000000000000e+02
+1.308820000000000e+03 +5.095000000000000e+02
+1.384590000000000e+03 +3.645000000000000e+02
+6.181980000000000e+02 +1.320000000000000e+02
+1.124280000000000e+03 +1.125000000000000e+02
+1.011440000000000e+03 +3.415000000000000e+02
+1.154010000000000e+03 +3.430000000000000e+02
+1.679280000000000e+03 +2.920000000000000e+02
+2.440740000000000e+03 +6.865000000000000e+02
+9.112220000000000e+02 +1.020000000000000e+02
+6.513250000000000e+02 +1.255000000000000e+02
+6.622530000000000e+02 +1.555000000000000e+02
+6.508170000000000e+02 +1.545000000000000e+02
+2.690030000000000e+03 +4.845000000000000e+02
+1.381170000000000e+03 +3.590000000000000e+02
+9.001450000000000e+02 +9.700000000000000e+01
+6.881560000000002e+02 +9.600000000000000e+01
+1.857100000000000e+03 +5.250000000000000e+02
+9.042190000000001e+02 +9.600000000000000e+01
+6.342660000000000e+02 +1.470000000000000e+02
+6.723290000000000e+02 +9.650000000000000e+01
+1.373920000000000e+03 +3.375000000000000e+02
+6.618919999999998e+02 +1.540000000000000e+02
+2.107570000000000e+03 +6.170000000000000e+02
+1.738860000000000e+03 +4.910000000000000e+02
+1.099050000000000e+03 +1.410000000000000e+02
+1.069460000000000e+03 +1.230000000000000e+02
+6.915139999999999e+02 +1.955000000000000e+02
+6.567669999999998e+02 +1.300000000000000e+02
+9.092760000000000e+02 +9.400000000000000e+01
+1.433970000000000e+03 +1.675000000000000e+02
+6.638700000000000e+02 +1.485000000000000e+02
+6.057430000000001e+02 +8.450000000000000e+01
+2.115860000000000e+03 +6.145000000000000e+02
+1.856630000000000e+03 +5.170000000000000e+02
+8.386489999999999e+02 +9.100000000000000e+01
+8.949839999999998e+02 +9.350000000000000e+01
+6.331990000000002e+02 +1.465000000000000e+02
+6.553989999999999e+02 +9.550000000000000e+01
+1.872480000000000e+03 +4.445000000000000e+02
+6.639280000000000e+02 +2.280000000000000e+02
+1.548790000000000e+03 +4.020000000000000e+02
+8.869169999999998e+02 +8.750000000000000e+01
+9.107270000000000e+02 +7.950000000000000e+01
+1.862420000000000e+03 +4.500000000000000e+02
+8.989910000000001e+02 +4.030000000000000e+02
+7.359739999999998e+02 +1.600000000000000e+02
+6.883570000000000e+02 +1.850000000000000e+02
+6.575910000000000e+02 +1.365000000000000e+02
+6.243940000000000e+02 +8.000000000000000e+01
+1.471030000000000e+03 +3.695000000000000e+02
+6.537569999999999e+02 +1.410000000000000e+02
+8.976250000000000e+02 +9.600000000000000e+01
+1.757980000000000e+03 +4.180000000000000e+02
+6.088560000000000e+02 +1.665000000000000e+02
+8.801230000000000e+02 +7.850000000000000e+01
+6.791750000000000e+02 +2.055000000000000e+02
+6.339030000000000e+02 +8.900000000000000e+01
+6.782110000000000e+02 +1.040000000000000e+02
+1.303790000000000e+03 +2.565000000000000e+02
+6.710850000000000e+02 +1.340000000000000e+02
+1.859000000000000e+03 +5.095000000000000e+02
+8.789460000000000e+02 +2.010000000000000e+02
+6.401930000000000e+02 +1.295000000000000e+02
+8.447200000000000e+02 +8.850000000000000e+01
+8.843420000000000e+02 +3.960000000000000e+02
+1.219210000000000e+03 +4.065000000000000e+02
+8.734670000000000e+02 +2.170000000000000e+02
+6.587320000000000e+02 +9.050000000000000e+01
+1.728840000000000e+03 +3.975000000000000e+02
+1.351880000000000e+03 +3.215000000000000e+02
+1.167350000000000e+03 +1.925000000000000e+02
+6.951289999999998e+02 +2.865000000000000e+02
+6.636450000000000e+02 +1.340000000000000e+02
+1.594750000000000e+03 +3.960000000000000e+02
+6.201469999999998e+02 +1.230000000000000e+02
+1.853130000000000e+03 +5.210000000000000e+02
+8.887210000000000e+02 +8.800000000000000e+01
+6.337809999999999e+02 +1.240000000000000e+02
+6.378800000000000e+02 +7.700000000000000e+01
+1.535320000000000e+03 +3.790000000000000e+02
+6.088150000000001e+02 +1.050000000000000e+02
+1.608200000000000e+03 +4.455000000000000e+02
+8.317430000000001e+02 +7.400000000000000e+01
+1.223100000000000e+03 +3.570000000000000e+02
+7.754380000000000e+02 +1.540000000000000e+02
+1.856060000000000e+03 +4.135000000000000e+02
+1.590520000000000e+03 +4.335000000000000e+02
+6.534019999999998e+02 +1.395000000000000e+02
+8.804920000000000e+02 +7.050000000000000e+01
+6.964050000000000e+02 +2.025000000000000e+02
+6.419820000000000e+02 +8.700000000000000e+01
+9.039970000000000e+02 +8.400000000000000e+01
+6.461950000000001e+02 +9.850000000000000e+01
+1.588710000000000e+03 +2.405000000000000e+02
+1.599520000000000e+03 +3.840000000000000e+02
+8.665970000000000e+02 +2.140000000000000e+02
+1.303150000000000e+03 +3.615000000000000e+02
+1.615150000000000e+03 +5.240000000000000e+02
+6.526410000000000e+02 +1.290000000000000e+02
+1.855410000000000e+03 +4.360000000000000e+02
+1.055940000000000e+03 +4.785000000000000e+02
+6.340780000000000e+02 +7.600000000000000e+01
+1.728250000000000e+03 +3.875000000000000e+02
+1.065360000000000e+03 +4.770000000000000e+02
+6.738439999999998e+02 +9.300000000000000e+01
+1.852860000000000e+03 +4.380000000000000e+02
+1.868840000000000e+03 +5.005000000000000e+02
+6.394400000000001e+02 +1.180000000000000e+02
+1.729530000000000e+03 +3.270000000000000e+02
+6.615760000000000e+02 +1.130000000000000e+02
+1.568970000000000e+03 +2.885000000000000e+02
+8.824110000000002e+02 +1.105000000000000e+02
+9.070680000000000e+02 +6.950000000000000e+01
+8.756710000000000e+02 +1.860000000000000e+02
+1.068600000000000e+03 +2.485000000000000e+02
+8.194989999999998e+02 +2.690000000000000e+02
+1.854070000000000e+03 +4.530000000000000e+02
+1.070520000000000e+03 +2.475000000000000e+02
+6.506870000000000e+02 +1.185000000000000e+02
+4.627030000000000e+02 +7.750000000000000e+01
+1.914900000000000e+03 +2.355000000000000e+02
+6.477959999999998e+02 +1.050000000000000e+02
+1.605860000000000e+03 +2.765000000000000e+02
+8.292900000000000e+02 +7.050000000000000e+01
+1.057020000000000e+03 +2.425000000000000e+02
+1.063150000000000e+03 +2.535000000000000e+02
+6.922430000000001e+02 +2.010000000000000e+02
+6.489870000000000e+02 +1.145000000000000e+02
+7.905280000000000e+02 +1.830000000000000e+02
+8.742170000000000e+02 +2.090000000000000e+02
+4.577800000000000e+02 +5.100000000000000e+01
+6.411270000000000e+02 +7.450000000000000e+01
+1.638590000000000e+03 +5.135000000000000e+02
+8.598170000000000e+02 +7.450000000000000e+01
+1.583570000000000e+03 +5.380000000000000e+02
+8.700419999999998e+02 +2.350000000000000e+02
+1.046300000000000e+03 +2.375000000000000e+02
+6.621489999999999e+02 +1.080000000000000e+02
+8.079630000000002e+02 +1.835000000000000e+02
+1.863420000000000e+03 +4.515000000000000e+02
+8.312689999999999e+02 +1.015000000000000e+02
+1.069380000000000e+03 +1.895000000000000e+02
+6.631210000000002e+02 +1.035000000000000e+02
+1.626280000000000e+03 +5.145000000000000e+02
+1.065420000000000e+03 +1.875000000000000e+02
+8.514349999999999e+02 +5.950000000000000e+01
+7.020510000000000e+02 +2.350000000000000e+02
+1.060520000000000e+03 +1.515000000000000e+02
+6.339019999999998e+02 +1.025000000000000e+02
+1.579120000000000e+03 +2.645000000000000e+02
+1.063150000000000e+03 +2.150000000000000e+02
+1.062520000000000e+03 +1.400000000000000e+02
+7.110610000000000e+02 +2.840000000000000e+02
+1.944210000000000e+03 +3.630000000000000e+02
+1.061760000000000e+03 +1.695000000000000e+02
+8.446070000000000e+02 +5.850000000000000e+01
+1.063500000000000e+03 +1.405000000000000e+02
+1.057820000000000e+03 +1.700000000000000e+02
+7.028670000000000e+02 +2.915000000000000e+02
+7.103450000000000e+02 +2.075000000000000e+02
+1.305510000000000e+03 +4.940000000000000e+02
+1.306120000000000e+03 +4.765000000000000e+02
+1.083880000000000e+03 +1.235000000000000e+02
+8.979340000000000e+02 +6.600000000000000e+01
+1.062300000000000e+03 +1.575000000000000e+02
+7.068280000000000e+02 +2.845000000000000e+02
+1.569000000000000e+03 +2.580000000000000e+02
+1.592710000000000e+03 +5.480000000000000e+02
+9.770480000000000e+02 +1.820000000000000e+02
+1.456140000000000e+03 +5.535000000000000e+02
+1.291970000000000e+03 +4.300000000000000e+02
+1.075790000000000e+03 +1.445000000000000e+02
+1.039530000000000e+03 +1.880000000000000e+02
+9.876630000000000e+02 +1.795000000000000e+02
+1.074390000000000e+03 +1.485000000000000e+02
+8.954520000000000e+02 +6.550000000000000e+01
+8.955219999999998e+02 +7.150000000000000e+01
+8.132030000000000e+02 +3.045000000000000e+02
+1.290200000000000e+03 +4.605000000000000e+02
+1.077670000000000e+03 +1.350000000000000e+02
+6.519299999999999e+02 +1.175000000000000e+02
+6.514950000000000e+02 +1.170000000000000e+02
+1.404950000000000e+03 +2.395000000000000e+02
+8.722730000000000e+02 +1.500000000000000e+02
+1.909780000000000e+03 +3.700000000000000e+02
+1.062870000000000e+03 +1.380000000000000e+02
+1.142840000000000e+03 +4.340000000000000e+02
+1.056180000000000e+03 +1.335000000000000e+02
+1.007290000000000e+03 +2.925000000000000e+02
+9.673880000000000e+02 +1.810000000000000e+02
+1.216280000000000e+03 +2.875000000000000e+02
+1.004980000000000e+03 +3.350000000000000e+02
+1.019070000000000e+03 +2.820000000000000e+02
+8.935269999999998e+02 +6.400000000000000e+01
+1.268880000000000e+03 +1.755000000000000e+02
+7.527189999999998e+02 +9.750000000000000e+01
+1.009930000000000e+03 +2.360000000000000e+02
+2.818490000000000e+02 +9.500000000000000e+01
+1.785840000000000e+03 +3.240000000000000e+02
+1.015000000000000e+03 +2.840000000000000e+02
+1.007430000000000e+03 +2.255000000000000e+02
+1.337510000000000e+03 +3.330000000000000e+02
+9.794990000000000e+02 +1.865000000000000e+02
+7.018110000000000e+02 +1.995000000000000e+02
+1.021880000000000e+03 +2.890000000000000e+02
+1.013800000000000e+03 +2.420000000000000e+02
+1.603720000000000e+03 +4.960000000000000e+02
+2.784860000000000e+02 +8.700000000000000e+01
+9.906020000000000e+02 +1.830000000000000e+02
+1.269140000000000e+03 +3.705000000000000e+02
+8.843950000000000e+02 +6.150000000000000e+01
+1.629690000000000e+03 +4.800000000000000e+02
+1.073230000000000e+03 +2.770000000000000e+02
+1.290910000000000e+03 +4.505000000000000e+02
+1.267880000000000e+03 +2.575000000000000e+02
+1.459060000000000e+03 +4.115000000000000e+02
+1.197680000000000e+03 +2.440000000000000e+02
+1.007760000000000e+03 +2.270000000000000e+02
+1.003720000000000e+03 +2.235000000000000e+02
+1.660070000000000e+03 +3.250000000000000e+02
+1.621650000000000e+03 +2.340000000000000e+02
+1.101930000000000e+03 +3.335000000000000e+02
+1.071180000000000e+03 +2.640000000000000e+02
+1.282500000000000e+03 +3.355000000000000e+02
+1.009530000000000e+03 +2.680000000000000e+02
+1.038030000000000e+03 +2.170000000000000e+02
+1.926340000000000e+03 +3.780000000000000e+02
+2.746240000000000e+02 +5.850000000000000e+01
+1.423480000000000e+03 +4.590000000000000e+02
+9.780210000000000e+02 +1.740000000000000e+02
+6.183470000000000e+02 +1.470000000000000e+02
+1.068470000000000e+03 +2.310000000000000e+02
+9.903010000000000e+02 +1.890000000000000e+02
+1.384280000000000e+03 +4.595000000000000e+02
+9.148120000000000e+02 +3.095000000000000e+02
+6.210250000000000e+02 +1.280000000000000e+02
+1.648380000000000e+03 +1.905000000000000e+02
+7.053320000000000e+02 +9.100000000000000e+01
+1.016440000000000e+03 +3.160000000000000e+02
+1.145010000000000e+03 +3.440000000000000e+02
+1.680920000000000e+03 +3.770000000000000e+02
+2.876790000000001e+02 +6.150000000000000e+01
+1.083960000000000e+03 +2.260000000000000e+02
+1.421300000000000e+03 +5.335000000000000e+02
+6.248110000000000e+02 +7.350000000000000e+01
+6.647370000000000e+02 +1.470000000000000e+02
+1.006260000000000e+03 +2.065000000000000e+02
+6.233910000000000e+02 +1.410000000000000e+02
+1.505170000000000e+03 +4.480000000000000e+02
+7.873639999999998e+02 +1.265000000000000e+02
+1.589310000000000e+03 +2.500000000000000e+02
+1.067250000000000e+03 +1.775000000000000e+02
+9.204990000000000e+02 +3.060000000000000e+02
+1.545670000000000e+03 +2.770000000000000e+02
+1.270650000000000e+03 +2.715000000000000e+02
+6.195760000000000e+02 +1.000000000000000e+02
+6.183940000000000e+02 +7.450000000000000e+01
+6.592130000000002e+02 +9.700000000000000e+01
+9.934010000000000e+02 +1.765000000000000e+02
+1.063150000000000e+03 +2.015000000000000e+02
+9.962660000000000e+02 +1.905000000000000e+02
+2.747700000000000e+02 +3.800000000000000e+01
+8.909160000000001e+02 +3.055000000000000e+02
+1.289560000000000e+03 +1.635000000000000e+02
+1.088080000000000e+03 +2.840000000000000e+02
+8.990880000000002e+02 +4.225000000000000e+02
+6.199150000000000e+02 +5.550000000000000e+01
+6.631010000000001e+02 +1.100000000000000e+02
+5.138310000000000e+02 +2.975000000000000e+02
+1.299770000000000e+03 +2.525000000000000e+02
+6.208280000000000e+02 +7.150000000000000e+01
+7.786120000000000e+02 +9.750000000000000e+01
+1.358380000000000e+03 +2.620000000000000e+02
+9.684750000000000e+02 +1.120000000000000e+02
+9.749070000000000e+02 +1.745000000000000e+02
+9.535770000000000e+02 +1.110000000000000e+02
+3.176000000000000e+02 +9.150000000000000e+01
+4.991090000000000e+02 +2.905000000000000e+02
+1.391890000000000e+03 +2.385000000000000e+02
+4.807600000000000e+02 +6.450000000000000e+01
+1.858670000000000e+03 +5.505000000000000e+02
+6.581669999999998e+02 +5.250000000000000e+01
+1.264860000000000e+03 +2.575000000000000e+02
+1.828720000000000e+03 +5.675000000000000e+02
+1.396440000000000e+03 +2.325000000000000e+02
+1.073180000000000e+03 +2.800000000000000e+02
+1.480670000000000e+03 +4.860000000000000e+02
+5.001600000000000e+02 +2.985000000000000e+02
+9.934170000000000e+02 +1.565000000000000e+02
+1.384000000000000e+03 +2.020000000000000e+02
+4.184730000000000e+02 +6.450000000000000e+01
+1.309220000000000e+03 +2.650000000000000e+02
+9.595990000000000e+02 +1.650000000000000e+02
+1.465390000000000e+03 +4.780000000000000e+02
+4.989940000000000e+02 +2.765000000000000e+02
+3.394240000000001e+02 +8.950000000000000e+01
+1.758530000000000e+03 +5.835000000000000e+02
+2.327120000000000e+03 +3.640000000000000e+02
+9.046110000000000e+02 +4.280000000000000e+02
+7.836239999999998e+02 +1.565000000000000e+02
+1.878320000000000e+03 +5.560000000000000e+02
+4.919940000000000e+02 +2.495000000000000e+02
+1.617150000000000e+03 +3.010000000000000e+02
+7.788939999999999e+02 +1.605000000000000e+02
+9.630000000000000e+02 +1.570000000000000e+02
+9.949840000000000e+02 +1.550000000000000e+02
+9.135300000000000e+02 +2.955000000000000e+02
+7.809610000000000e+02 +1.595000000000000e+02
+7.762370000000000e+02 +1.480000000000000e+02
+7.804320000000000e+02 +1.700000000000000e+02
+1.627050000000000e+03 +2.955000000000000e+02
+7.860910000000000e+02 +2.940000000000000e+02
+4.875800000000000e+02 +2.825000000000000e+02
+2.822160000000000e+02 +2.100000000000000e+01
+7.731619999999998e+02 +1.665000000000000e+02
+7.770549999999999e+02 +1.455000000000000e+02
+7.657320000000000e+02 +1.565000000000000e+02
+7.793210000000000e+02 +2.415000000000000e+02
+3.593769999999999e+02 +8.150000000000000e+01
+6.878789999999998e+02 +2.530000000000000e+02
+4.887830000000000e+02 +2.390000000000000e+02
+1.627590000000000e+03 +2.800000000000000e+02
+7.874169999999998e+02 +1.390000000000000e+02
+7.765200000000000e+02 +1.395000000000000e+02
+1.355810000000000e+03 +1.470000000000000e+02
+8.788020000000000e+02 +4.895000000000000e+02
+1.590510000000000e+03 +2.140000000000000e+02
+3.325250000000000e+02 +6.800000000000000e+01
+9.788410000000000e+02 +1.780000000000000e+02
+7.812500000000000e+02 +1.635000000000000e+02
+6.901810000000000e+02 +2.760000000000000e+02
+7.474119999999998e+02 +1.185000000000000e+02
+7.847310000000001e+02 +1.475000000000000e+02
+1.091890000000000e+03 +2.720000000000000e+02
+5.043640000000000e+02 +2.830000000000000e+02
+7.960160000000002e+02 +1.620000000000000e+02
+1.604150000000000e+03 +2.175000000000000e+02
+2.291900000000000e+03 +6.845000000000000e+02
+7.514620000000000e+02 +1.275000000000000e+02
+9.908780000000000e+02 +1.585000000000000e+02
+7.627170000000000e+02 +1.185000000000000e+02
+7.897239999999998e+02 +1.405000000000000e+02
+1.874020000000000e+03 +5.585000000000000e+02
+4.976250000000000e+02 +2.465000000000000e+02
+3.365760000000000e+02 +6.450000000000000e+01
+9.792340000000000e+02 +1.700000000000000e+02
+1.614700000000000e+03 +2.820000000000000e+02
+7.901469999999998e+02 +1.410000000000000e+02
+1.386760000000000e+03 +4.125000000000000e+02
+1.339700000000000e+03 +4.905000000000000e+02
+7.934270000000000e+02 +1.640000000000000e+02
+1.624350000000000e+03 +4.920000000000000e+02
+9.925700000000001e+02 +1.695000000000000e+02
+1.865050000000000e+03 +5.665000000000000e+02
+9.873000000000000e+02 +1.510000000000000e+02
+7.788430000000002e+02 +1.300000000000000e+02
+7.885989999999998e+02 +1.485000000000000e+02
+1.660270000000000e+03 +2.370000000000000e+02
+7.938330000000002e+02 +1.340000000000000e+02
+1.003010000000000e+03 +1.705000000000000e+02
+1.097770000000000e+03 +3.425000000000000e+02
+1.378320000000000e+03 +4.085000000000000e+02
+1.876120000000000e+03 +5.435000000000000e+02
+4.888150000000000e+02 +2.445000000000000e+02
+7.720700000000001e+02 +1.260000000000000e+02
+9.950200000000000e+02 +1.485000000000000e+02
+7.959780000000002e+02 +1.245000000000000e+02
+7.707280000000002e+02 +1.275000000000000e+02
+4.886710000000000e+02 +9.900000000000000e+01
+6.940599999999999e+02 +2.755000000000000e+02
+1.259590000000000e+03 +2.825000000000000e+02
+1.872320000000000e+03 +5.545000000000000e+02
+1.306720000000000e+03 +2.130000000000000e+02
+7.896960000000000e+02 +1.500000000000000e+02
+8.560219999999998e+02 +1.520000000000000e+02
+1.355530000000000e+03 +3.925000000000000e+02
+1.844070000000000e+03 +5.620000000000000e+02
+5.023130000000001e+02 +2.470000000000000e+02
+7.840039999999998e+02 +1.380000000000000e+02
+1.170460000000000e+03 +3.045000000000000e+02
+7.981350000000000e+02 +1.360000000000000e+02
+6.689310000000000e+02 +1.080000000000000e+02
+7.836039999999998e+02 +1.165000000000000e+02
+1.335480000000000e+03 +3.645000000000000e+02
+1.375960000000000e+03 +3.905000000000000e+02
+1.487860000000000e+03 +4.305000000000000e+02
+4.845500000000000e+02 +2.325000000000000e+02
+1.314730000000000e+03 +1.990000000000000e+02
+1.319770000000000e+03 +2.235000000000000e+02
+7.437410000000001e+02 +1.095000000000000e+02
+3.383819999999999e+02 +2.650000000000000e+01
+1.391420000000000e+03 +3.755000000000000e+02
+7.949710000000000e+02 +1.475000000000000e+02
+9.993430000000000e+02 +1.500000000000000e+02
+7.975410000000001e+02 +1.185000000000000e+02
+1.101300000000000e+03 +2.685000000000000e+02
+9.882850000000000e+02 +1.660000000000000e+02
+1.299180000000000e+03 +2.225000000000000e+02
+1.083380000000000e+03 +1.970000000000000e+02
+8.892489999999998e+02 +9.500000000000000e+01
+1.786740000000000e+03 +4.580000000000000e+02
+6.730720000000000e+02 +5.190000000000000e+02
+1.294630000000000e+03 +2.110000000000000e+02
+1.300890000000000e+03 +2.055000000000000e+02
+7.544390000000000e+02 +1.180000000000000e+02
+1.371440000000000e+03 +4.295000000000000e+02
+8.937400000000000e+02 +1.735000000000000e+02
+1.299700000000000e+03 +1.630000000000000e+02
+1.299160000000000e+03 +2.015000000000000e+02
+7.795580000000000e+02 +1.140000000000000e+02
+1.090690000000000e+03 +2.305000000000000e+02
+1.092100000000000e+03 +1.595000000000000e+02
+3.388640000000001e+02 +4.200000000000000e+01
+1.080530000000000e+03 +4.650000000000000e+02
+8.017320000000000e+02 +1.470000000000000e+02
+1.067960000000000e+03 +1.815000000000000e+02
+8.699710000000000e+02 +9.400000000000000e+01
+9.869790000000000e+02 +1.455000000000000e+02
+7.810770000000000e+02 +1.115000000000000e+02
+1.307680000000000e+03 +1.795000000000000e+02
+1.009020000000000e+03 +1.960000000000000e+02
+8.619080000000000e+02 +1.345000000000000e+02
+1.587560000000000e+03 +4.530000000000000e+02
+9.792140000000001e+02 +1.330000000000000e+02
+1.312590000000000e+03 +1.720000000000000e+02
+7.870599999999999e+02 +1.265000000000000e+02
+1.665810000000000e+03 +3.625000000000000e+02
+6.728930000000000e+02 +4.965000000000000e+02
+1.104540000000000e+03 +2.275000000000000e+02
+1.169710000000000e+03 +3.195000000000000e+02
+7.698120000000000e+02 +1.090000000000000e+02
+3.465760000000000e+02 +2.900000000000000e+01
+1.069490000000000e+03 +3.415000000000000e+02
+9.597290000000000e+02 +1.610000000000000e+02
+1.312630000000000e+03 +1.985000000000000e+02
+1.310820000000000e+03 +1.525000000000000e+02
+1.584250000000000e+03 +4.845000000000000e+02
+7.483380000000002e+02 +1.480000000000000e+02
+7.920560000000000e+02 +1.270000000000000e+02
+1.937140000000000e+03 +4.190000000000000e+02
+6.645269999999998e+02 +5.200000000000000e+02
+6.601360000000002e+02 +6.350000000000000e+01
+1.022300000000000e+03 +1.200000000000000e+02
+1.687150000000000e+03 +3.840000000000000e+02
+1.296940000000000e+03 +1.380000000000000e+02
+7.820400000000000e+02 +1.195000000000000e+02
+9.942180000000000e+02 +1.825000000000000e+02
+9.072870000000000e+02 +1.465000000000000e+02
+1.032410000000000e+03 +1.635000000000000e+02
+7.423910000000002e+02 +1.450000000000000e+02
+7.658880000000000e+02 +1.060000000000000e+02
+9.743150000000001e+02 +1.285000000000000e+02
+8.609570000000000e+02 +7.700000000000000e+01
+7.802980000000000e+02 +1.175000000000000e+02
+1.098510000000000e+03 +1.215000000000000e+02
+1.065110000000000e+03 +2.435000000000000e+02
+1.658610000000000e+03 +4.115000000000000e+02
+6.677430000000001e+02 +4.590000000000000e+02
+1.069370000000000e+03 +1.100000000000000e+02
+1.044720000000000e+03 +1.745000000000000e+02
+7.451660000000001e+02 +1.375000000000000e+02
+7.819889999999998e+02 +1.065000000000000e+02
+7.452020000000000e+02 +1.105000000000000e+02
+1.317470000000000e+03 +1.400000000000000e+02
+1.319750000000000e+03 +1.595000000000000e+02
+1.068790000000000e+03 +2.105000000000000e+02
+7.736310000000002e+02 +9.550000000000000e+01
+1.128200000000000e+03 +1.560000000000000e+02
+9.736330000000000e+02 +1.280000000000000e+02
+1.680080000000000e+03 +3.895000000000000e+02
+7.784660000000000e+02 +1.165000000000000e+02
+7.747669999999998e+02 +1.045000000000000e+02
+7.742130000000002e+02 +9.350000000000000e+01
+1.081280000000000e+03 +1.440000000000000e+02
+1.054730000000000e+03 +2.065000000000000e+02
+8.783290000000000e+02 +1.140000000000000e+02
+7.456860000000000e+02 +1.090000000000000e+02
+7.789100000000000e+02 +1.135000000000000e+02
+8.792970000000000e+02 +1.015000000000000e+02
+7.759060000000002e+02 +8.800000000000000e+01
+1.672250000000000e+03 +4.090000000000000e+02
+1.310080000000000e+03 +1.425000000000000e+02
+7.773630000000001e+02 +1.045000000000000e+02
+1.089600000000000e+03 +1.665000000000000e+02
+4.949320000000000e+02 +3.450000000000000e+01
+7.455770000000000e+02 +1.050000000000000e+02
+1.316740000000000e+03 +1.545000000000000e+02
+1.078380000000000e+03 +2.110000000000000e+02
+1.956160000000000e+03 +4.290000000000000e+02
+7.686180000000001e+02 +8.050000000000000e+01
+1.678030000000000e+03 +3.910000000000000e+02
+9.024870000000000e+02 +9.950000000000000e+01
+8.613099999999999e+02 +8.550000000000000e+01
+7.443110000000000e+02 +1.025000000000000e+02
+7.727950000000000e+02 +1.035000000000000e+02
+7.820350000000000e+02 +8.100000000000000e+01
+2.976640000000000e+02 +1.885000000000000e+02
+7.692739999999999e+02 +9.900000000000000e+01
+8.789220000000000e+02 +1.045000000000000e+02
+9.946600000000000e+02 +1.210000000000000e+02
+7.448439999999998e+02 +9.500000000000000e+01
+7.871410000000002e+02 +7.950000000000000e+01
+7.819480000000000e+02 +9.850000000000000e+01
+1.061000000000000e+03 +1.855000000000000e+02
+1.305290000000000e+03 +1.320000000000000e+02
+8.344010000000002e+02 +9.850000000000000e+01
+1.099860000000000e+03 +2.070000000000000e+02
+9.031670000000000e+02 +9.250000000000000e+01
+7.463270000000000e+02 +9.100000000000000e+01
+1.312290000000000e+03 +1.445000000000000e+02
+7.838880000000000e+02 +7.850000000000000e+01
+7.826210000000002e+02 +9.850000000000000e+01
+7.464260000000000e+02 +7.750000000000000e+01
+7.817780000000000e+02 +7.100000000000000e+01
+9.087770000000000e+02 +8.550000000000000e+01
+7.595630000000000e+02 +8.200000000000000e+01
+9.836480000000000e+02 +9.850000000000000e+01
+1.139230000000000e+03 +4.360000000000000e+02
+7.442139999999998e+02 +6.100000000000000e+01
+7.829050000000000e+02 +6.850000000000000e+01
+7.897370000000000e+02 +9.550000000000000e+01
+7.051330000000000e+02 +3.580000000000000e+02
+7.698860000000002e+02 +8.650000000000000e+01
+8.852360000000001e+02 +1.070000000000000e+02
+8.716089999999998e+02 +9.600000000000000e+01
+1.986940000000000e+03 +4.370000000000000e+02
+6.992010000000000e+02 +2.495000000000000e+02
+7.756569999999998e+02 +6.550000000000000e+01
+7.758489999999998e+02 +6.150000000000000e+01
+7.793489999999998e+02 +8.700000000000000e+01
+8.584900000000000e+02 +1.005000000000000e+02
+7.064240000000000e+02 +2.255000000000000e+02
+8.966080000000002e+02 +9.450000000000000e+01
+7.699930000000001e+02 +7.900000000000000e+01
+9.048300000000000e+02 +1.075000000000000e+02
+7.721950000000001e+02 +2.780000000000000e+02
+7.035440000000000e+02 +2.075000000000000e+02
+8.294900000000000e+02 +9.150000000000000e+01
+9.662089999999999e+02 +1.765000000000000e+02
+8.926700000000000e+02 +9.550000000000000e+01
+7.596160000000001e+02 +2.590000000000000e+02
+6.995410000000001e+02 +3.135000000000000e+02
+1.014400000000000e+03 +4.330000000000000e+02
+8.499950000000000e+02 +9.500000000000000e+01
+1.325740000000000e+03 +3.560000000000000e+02
+1.378190000000000e+03 +2.810000000000000e+02
+1.080970000000000e+03 +4.515000000000000e+02
+1.061800000000000e+03 +5.525000000000000e+02
+8.970380000000000e+02 +9.500000000000000e+01
+7.195750000000000e+02 +3.320000000000000e+02
+6.597339999999998e+02 +2.095000000000000e+02
+1.408150000000000e+03 +1.980000000000000e+02
+8.723070000000000e+02 +3.060000000000000e+02
+1.464490000000000e+03 +4.165000000000000e+02
+7.214069999999998e+02 +3.495000000000000e+02
+1.231020000000000e+03 +3.065000000000000e+02
+1.014160000000000e+03 +4.355000000000000e+02
+8.976530000000000e+02 +1.365000000000000e+02
+9.169340000000000e+02 +3.370000000000000e+02
+6.884299999999999e+02 +2.570000000000000e+02
+6.562250000000000e+02 +1.325000000000000e+02
+6.922110000000000e+02 +1.595000000000000e+02
+1.809030000000000e+03 +6.000000000000000e+02
+8.730430000000000e+02 +8.750000000000000e+01
+7.705050000000000e+02 +2.755000000000000e+02
+1.235600000000000e+03 +2.665000000000000e+02
+6.689560000000000e+02 +1.715000000000000e+02
+6.448110000000000e+02 +2.155000000000000e+02
+1.076760000000000e+03 +3.605000000000000e+02
+8.833099999999999e+02 +3.560000000000000e+02
+1.845720000000000e+03 +6.245000000000000e+02
+8.271669999999998e+02 +8.050000000000000e+01
+9.001020000000000e+02 +8.150000000000000e+01
+9.920240000000000e+02 +3.395000000000000e+02
+1.487300000000000e+03 +4.170000000000000e+02
+1.014580000000000e+03 +4.125000000000000e+02
+1.820640000000000e+03 +6.450000000000000e+02
+1.747480000000000e+03 +4.770000000000000e+02
+6.415630000000000e+02 +2.140000000000000e+02
+6.787060000000000e+02 +1.255000000000000e+02
+6.612389999999998e+02 +1.750000000000000e+02
+6.497690000000000e+02 +1.645000000000000e+02
+8.884510000000000e+02 +8.900000000000000e+01
+8.918960000000002e+02 +5.180000000000000e+02
+9.076930000000000e+02 +3.540000000000000e+02
+1.078290000000000e+03 +4.385000000000000e+02
+7.077150000000000e+02 +2.550000000000000e+02
+4.998880000000000e+02 +9.300000000000000e+01
+9.808420000000000e+02 +4.395000000000000e+02
+6.397100000000000e+02 +1.580000000000000e+02
+6.571650000000000e+02 +1.315000000000000e+02
+9.415990000000000e+02 +3.540000000000000e+02
+1.051630000000000e+03 +4.295000000000000e+02
+8.968330000000002e+02 +8.050000000000000e+01
+1.221580000000000e+03 +2.750000000000000e+02
+6.822950000000000e+02 +3.250000000000000e+02
+6.352980000000000e+02 +9.800000000000000e+01
+1.640340000000000e+03 +6.645000000000000e+02
+9.216110000000000e+02 +3.265000000000000e+02
+5.123430000000002e+02 +8.850000000000000e+01
+8.946750000000000e+02 +8.850000000000000e+01
+6.699800000000000e+02 +1.680000000000000e+02
+9.274700000000000e+02 +3.115000000000000e+02
+9.197190000000001e+02 +3.370000000000000e+02
+1.620320000000000e+03 +4.215000000000000e+02
+6.369030000000000e+02 +9.200000000000000e+01
+6.794989999999998e+02 +9.700000000000000e+01
+6.358640000000000e+02 +1.475000000000000e+02
+6.379400000000001e+02 +1.590000000000000e+02
+4.552430000000001e+02 +5.100000000000000e+01
+1.294930000000000e+03 +3.525000000000000e+02
+8.820050000000000e+02 +8.350000000000000e+01
+1.071700000000000e+03 +5.555000000000000e+02
+6.338000000000000e+02 +8.400000000000000e+01
+8.531500000000000e+02 +3.535000000000000e+02
+1.589060000000000e+03 +4.360000000000000e+02
+6.579970000000000e+02 +1.395000000000000e+02
+4.566190000000000e+02 +3.700000000000000e+01
+8.606110000000001e+02 +3.235000000000000e+02
+1.053370000000000e+03 +5.300000000000000e+02
+6.398910000000000e+02 +1.450000000000000e+02
+6.572350000000000e+02 +8.300000000000000e+01
+9.063830000000000e+02 +2.985000000000000e+02
+1.586980000000000e+03 +2.940000000000000e+02
+8.914360000000000e+02 +8.450000000000000e+01
+1.791190000000000e+03 +4.910000000000000e+02
+6.195440000000000e+02 +1.235000000000000e+02
+6.479600000000000e+02 +1.430000000000000e+02
+6.581230000000000e+02 +6.450000000000000e+01
+8.933560000000001e+02 +2.985000000000000e+02
+1.562510000000000e+03 +2.670000000000000e+02
+1.779060000000000e+03 +2.605000000000000e+02
+8.702200000000000e+02 +6.750000000000000e+01
+6.594820000000000e+02 +1.270000000000000e+02
+6.490169999999998e+02 +1.220000000000000e+02
+6.558819999999999e+02 +6.250000000000000e+01
+8.054910000000001e+02 +2.565000000000000e+02
+1.567930000000000e+03 +2.415000000000000e+02
+6.905610000000000e+02 +2.950000000000000e+02
+8.672660000000002e+02 +1.200000000000000e+02
+1.577380000000000e+03 +2.665000000000000e+02
+6.659630000000002e+02 +1.140000000000000e+02
+6.624140000000000e+02 +1.270000000000000e+02
+8.876120000000000e+02 +6.850000000000000e+01
+1.073790000000000e+03 +3.305000000000000e+02
+6.512580000000000e+02 +8.950000000000000e+01
+1.626330000000000e+03 +5.490000000000000e+02
+8.970230000000000e+02 +6.700000000000000e+01
+1.665500000000000e+03 +2.305000000000000e+02
+1.374150000000000e+03 +5.330000000000000e+02
+1.619480000000000e+03 +5.135000000000000e+02
+6.370780000000000e+02 +9.150000000000000e+01
+1.632690000000000e+03 +5.720000000000000e+02
+1.069730000000000e+03 +3.375000000000000e+02
+6.402370000000000e+02 +1.180000000000000e+02
+6.361860000000000e+02 +7.050000000000000e+01
+1.623080000000000e+03 +5.760000000000000e+02
+1.611610000000000e+03 +5.820000000000000e+02
+1.083000000000000e+03 +3.615000000000000e+02
+6.502859999999999e+02 +1.150000000000000e+02
+1.568760000000000e+03 +2.265000000000000e+02
+6.371690000000000e+02 +1.085000000000000e+02
+8.072460000000002e+02 +2.220000000000000e+02
+6.367700000000000e+02 +7.850000000000000e+01
+1.600310000000000e+03 +1.985000000000000e+02
+1.562020000000000e+03 +2.870000000000000e+02
+1.081370000000000e+03 +2.850000000000000e+02
+6.449200000000000e+02 +1.065000000000000e+02
+1.586250000000000e+03 +4.955000000000000e+02
+6.800089999999999e+02 +2.060000000000000e+02
+6.502040000000002e+02 +6.650000000000000e+01
+1.561590000000000e+03 +2.025000000000000e+02
+6.572230000000002e+02 +8.200000000000000e+01
+8.963310000000000e+02 +6.850000000000000e+01
+1.072710000000000e+03 +3.550000000000000e+02
+6.385069999999999e+02 +9.950000000000000e+01
+6.524780000000002e+02 +2.065000000000000e+02
+1.557180000000000e+03 +1.905000000000000e+02
+2.318550000000000e+03 +5.560000000000000e+02
+6.343030000000000e+02 +7.150000000000000e+01
+1.005140000000000e+03 +3.560000000000000e+02
+1.664100000000000e+03 +2.280000000000000e+02
+1.058060000000000e+03 +2.245000000000000e+02
+8.026160000000001e+02 +2.380000000000000e+02
+7.089789999999998e+02 +2.945000000000000e+02
+7.184900000000000e+02 +3.665000000000000e+02
+8.201130000000001e+02 +2.995000000000000e+02
+1.389770000000000e+03 +4.975000000000000e+02
+6.471200000000000e+02 +7.000000000000000e+01
+7.080860000000000e+02 +2.010000000000000e+02
+1.082550000000000e+03 +2.240000000000000e+02
+8.180160000000002e+02 +2.395000000000000e+02
+1.068790000000000e+03 +2.150000000000000e+02
+1.080050000000000e+03 +2.170000000000000e+02
+1.334000000000000e+03 +3.635000000000000e+02
+6.642539999999998e+02 +2.420000000000000e+02
+6.709839999999998e+02 +1.775000000000000e+02
+1.060480000000000e+03 +1.875000000000000e+02
+2.322630000000000e+03 +4.915000000000000e+02
+7.216360000000002e+02 +2.270000000000000e+02
+1.574680000000000e+03 +2.840000000000000e+02
+1.062380000000000e+03 +1.840000000000000e+02
+7.097110000000000e+02 +2.780000000000000e+02
+1.875830000000000e+03 +4.595000000000000e+02
+1.063610000000000e+03 +2.085000000000000e+02
+1.847210000000000e+03 +5.825000000000000e+02
+1.635190000000000e+03 +4.865000000000000e+02
+1.064990000000000e+03 +1.130000000000000e+02
+2.155180000000000e+03 +5.010000000000000e+02
+1.081460000000000e+03 +2.270000000000000e+02
+9.743840000000000e+02 +1.810000000000000e+02
+9.924220000000000e+02 +7.750000000000000e+01
+8.513980000000000e+02 +1.040000000000000e+02
+1.855520000000000e+03 +4.425000000000000e+02
+1.611140000000000e+03 +5.480000000000000e+02
+1.064850000000000e+03 +1.140000000000000e+02
+1.085130000000000e+03 +2.210000000000000e+02
+1.563060000000000e+03 +2.575000000000000e+02
+5.919200000000000e+02 +1.155000000000000e+02
+1.494010000000000e+03 +6.345000000000000e+02
+9.811910000000000e+02 +1.825000000000000e+02
+1.088460000000000e+03 +2.135000000000000e+02
+8.975700000000001e+02 +6.750000000000000e+01
+1.270440000000000e+03 +3.510000000000000e+02
+1.333040000000000e+03 +4.460000000000000e+02
+1.225610000000000e+03 +1.340000000000000e+02
+1.732080000000000e+03 +2.970000000000000e+02
+1.004130000000000e+03 +3.135000000000000e+02
+1.013640000000000e+03 +2.970000000000000e+02
+1.006060000000000e+03 +3.105000000000000e+02
+1.005820000000000e+03 +3.065000000000000e+02
+1.064250000000000e+03 +1.305000000000000e+02
+9.746570000000000e+02 +1.765000000000000e+02
+1.289020000000000e+03 +2.850000000000000e+02
+5.700990000000000e+02 +8.550000000000000e+01
+9.759130000000000e+02 +1.270000000000000e+02
+1.589410000000000e+03 +5.730000000000000e+02
+1.296200000000000e+03 +4.890000000000000e+02
+1.044920000000000e+03 +1.135000000000000e+02
+1.094820000000000e+03 +3.415000000000000e+02
+1.010620000000000e+03 +3.190000000000000e+02
+1.011580000000000e+03 +2.965000000000000e+02
+1.024030000000000e+03 +4.275000000000000e+02
+9.809670000000000e+02 +1.640000000000000e+02
+6.551820000000000e+02 +1.095000000000000e+02
+1.017820000000000e+03 +2.870000000000000e+02
+1.559960000000000e+03 +5.065000000000000e+02
+1.217010000000000e+03 +4.155000000000000e+02
+2.822470000000000e+02 +8.150000000000000e+01
+1.080270000000000e+03 +1.170000000000000e+02
+1.071180000000000e+03 +2.010000000000000e+02
+1.406790000000000e+03 +5.945000000000000e+02
+1.007330000000000e+03 +4.070000000000000e+02
+1.329890000000000e+03 +3.135000000000000e+02
+1.011760000000000e+03 +2.745000000000000e+02
+1.012270000000000e+03 +3.550000000000000e+02
+1.448820000000000e+03 +5.055000000000000e+02
+1.085590000000000e+03 +3.055000000000000e+02
+2.783490000000000e+02 +7.750000000000000e+01
+1.065970000000000e+03 +1.505000000000000e+02
+6.550369999999998e+02 +4.850000000000000e+01
+1.005790000000000e+03 +3.230000000000000e+02
+9.806260000000000e+02 +1.945000000000000e+02
+9.755450000000000e+02 +2.485000000000000e+02
+1.007830000000000e+03 +2.910000000000000e+02
+2.753950000000000e+02 +6.150000000000000e+01
+1.078410000000000e+03 +1.240000000000000e+02
+1.618640000000000e+03 +5.855000000000000e+02
+1.414760000000000e+03 +6.025000000000000e+02
+5.046700000000000e+02 +3.320000000000000e+02
+8.991020000000000e+02 +5.070000000000000e+02
+1.116810000000000e+03 +3.150000000000000e+02
+1.066730000000000e+03 +1.495000000000000e+02
+1.442600000000000e+03 +4.790000000000000e+02
+9.830010000000000e+02 +1.875000000000000e+02
+8.823200000000001e+02 +1.295000000000000e+02
+1.845440000000000e+03 +5.485000000000000e+02
+2.778890000000000e+02 +4.450000000000000e+01
+8.704290000000000e+02 +4.920000000000000e+02
+1.023050000000000e+03 +2.520000000000000e+02
+6.241070000000000e+02 +1.540000000000000e+02
+4.867180000000000e+02 +3.130000000000000e+02
+1.045770000000000e+03 +1.155000000000000e+02
+6.220150000000000e+02 +1.470000000000000e+02
+6.679839999999998e+02 +4.800000000000000e+01
+1.012410000000000e+03 +2.725000000000000e+02
+9.779820000000000e+02 +1.530000000000000e+02
+9.905380000000000e+02 +2.390000000000000e+02
+1.467120000000000e+03 +4.755000000000000e+02
+1.206740000000000e+03 +3.905000000000000e+02
+1.292160000000000e+03 +3.475000000000000e+02
+6.648150000000001e+02 +1.480000000000000e+02
+2.486760000000000e+03 +6.690000000000000e+02
+1.420100000000000e+03 +3.525000000000000e+02
+9.804870000000000e+02 +1.320000000000000e+02
+6.221849999999999e+02 +1.575000000000000e+02
+1.004280000000000e+03 +2.735000000000000e+02
+1.612420000000000e+03 +5.410000000000000e+02
+2.820590000000000e+02 +7.500000000000000e+01
+6.263790000000000e+02 +9.550000000000000e+01
+5.119470000000000e+02 +3.365000000000000e+02
+1.399710000000000e+03 +5.630000000000000e+02
+6.222380000000001e+02 +1.175000000000000e+02
+8.685020000000000e+02 +9.800000000000000e+01
+2.683440000000000e+03 +6.970000000000000e+02
+6.643950000000000e+02 +1.455000000000000e+02
+6.678330000000002e+02 +1.380000000000000e+02
+8.778900000000000e+02 +9.750000000000000e+01
+9.926340000000000e+02 +1.530000000000000e+02
+6.268480000000002e+02 +9.450000000000000e+01
+3.343980000000000e+02 +7.950000000000000e+01
+6.230680000000000e+02 +9.050000000000000e+01
+6.616050000000000e+02 +7.700000000000000e+01
+8.800530000000000e+02 +9.500000000000000e+01
+7.600210000000002e+02 +1.040000000000000e+02
+8.786080000000002e+02 +9.150000000000000e+01
+1.134930000000000e+03 +2.410000000000000e+02
+1.081890000000000e+03 +1.810000000000000e+02
+1.362780000000000e+03 +2.310000000000000e+02
+9.298880000000000e+02 +3.925000000000000e+02
+9.885220000000000e+02 +1.545000000000000e+02
+4.877800000000000e+02 +3.000000000000000e+02
+1.590520000000000e+03 +2.870000000000000e+02
+6.234270000000000e+02 +6.700000000000000e+01
+6.208480000000002e+02 +7.700000000000000e+01
+1.376980000000000e+03 +2.340000000000000e+02
+1.664880000000000e+03 +4.335000000000000e+02
+2.664660000000000e+03 +6.700000000000000e+02
+3.374390000000000e+02 +7.600000000000000e+01
+6.626870000000000e+02 +7.250000000000000e+01
+4.920720000000000e+02 +3.070000000000000e+02
+1.078330000000000e+03 +2.020000000000000e+02
+1.076430000000000e+03 +2.025000000000000e+02
+1.750400000000000e+03 +7.190000000000000e+02
+1.455560000000000e+03 +4.785000000000000e+02
+7.480520000000000e+02 +5.700000000000000e+01
+1.580630000000000e+03 +5.415000000000000e+02
+1.008620000000000e+03 +1.520000000000000e+02
+9.815690000000000e+02 +1.255000000000000e+02
+9.945480000000000e+02 +1.330000000000000e+02
+6.399970000000000e+02 +7.350000000000000e+01
+6.357710000000000e+02 +6.850000000000000e+01
+4.841650000000000e+02 +2.850000000000000e+02
+1.278520000000000e+03 +3.170000000000000e+02
+1.069670000000000e+03 +2.290000000000000e+02
+8.930119999999999e+02 +1.205000000000000e+02
+6.190670000000000e+02 +5.700000000000000e+01
+4.891080000000000e+02 +2.975000000000000e+02
+2.669520000000000e+03 +6.770000000000000e+02
+1.209940000000000e+03 +1.015000000000000e+02
+1.759230000000000e+03 +5.810000000000000e+02
+1.857380000000000e+03 +6.240000000000000e+02
+1.410570000000000e+03 +1.500000000000000e+02
+1.054600000000000e+03 +1.485000000000000e+02
+1.089760000000000e+03 +2.225000000000000e+02
+9.792150000000000e+02 +1.245000000000000e+02
+8.915820000000000e+02 +3.395000000000000e+02
+1.091970000000000e+03 +2.350000000000000e+02
+3.333490000000000e+02 +6.550000000000000e+01
+6.204209999999998e+02 +5.400000000000000e+01
+8.933720000000000e+02 +9.050000000000000e+01
+4.855590000000000e+02 +2.845000000000000e+02
+1.108280000000000e+03 +1.585000000000000e+02
+1.865840000000000e+03 +6.245000000000000e+02
+7.728600000000000e+02 +1.410000000000000e+02
+1.061110000000000e+03 +1.800000000000000e+02
+7.871500000000000e+02 +3.470000000000000e+02
+6.574430000000000e+02 +4.950000000000000e+01
+1.634420000000000e+03 +3.030000000000000e+02
+7.885290000000000e+02 +1.410000000000000e+02
+7.726489999999999e+02 +1.255000000000000e+02
+8.927589999999999e+02 +9.400000000000000e+01
+1.082510000000000e+03 +1.730000000000000e+02
+7.752530000000000e+02 +1.335000000000000e+02
+1.113670000000000e+03 +1.530000000000000e+02
+6.356170000000000e+02 +1.180000000000000e+02
+1.588440000000000e+03 +2.155000000000000e+02
+7.856260000000002e+02 +3.840000000000000e+02
+1.071270000000000e+03 +2.540000000000000e+02
+1.227160000000000e+03 +2.585000000000000e+02
+9.820700000000001e+02 +1.225000000000000e+02
+7.789370000000000e+02 +1.260000000000000e+02
+1.778940000000000e+03 +5.145000000000000e+02
+4.831750000000000e+02 +7.300000000000000e+01
+1.377250000000000e+03 +4.580000000000000e+02
+1.591620000000000e+03 +2.230000000000000e+02
+9.237230000000000e+02 +3.915000000000000e+02
+4.978180000000000e+02 +3.000000000000000e+02
+7.842300000000000e+02 +1.340000000000000e+02
+9.776540000000000e+02 +1.235000000000000e+02
+7.857669999999998e+02 +1.375000000000000e+02
+1.350620000000000e+03 +4.070000000000000e+02
+9.797340000000000e+02 +1.195000000000000e+02
+1.476300000000000e+03 +5.020000000000000e+02
+7.823839999999999e+02 +1.400000000000000e+02
+1.798820000000000e+03 +4.950000000000000e+02
+8.932289999999998e+02 +8.900000000000000e+01
+6.898470000000000e+02 +3.815000000000000e+02
+1.869740000000000e+03 +6.700000000000000e+02
+4.958330000000000e+02 +2.885000000000000e+02
+7.856849999999999e+02 +1.155000000000000e+02
+7.767719999999998e+02 +1.320000000000000e+02
+7.714349999999999e+02 +1.175000000000000e+02
+8.681810000000000e+02 +3.250000000000000e+02
+4.814680000000000e+02 +9.850000000000000e+01
+1.067490000000000e+03 +2.030000000000000e+02
+1.330640000000000e+03 +4.325000000000000e+02
+7.550440000000000e+02 +1.235000000000000e+02
+8.846480000000000e+02 +9.400000000000000e+01
+1.479970000000000e+03 +5.095000000000000e+02
+7.887600000000000e+02 +1.260000000000000e+02
+7.921000000000000e+02 +1.215000000000000e+02
+1.680780000000000e+03 +4.420000000000000e+02
+8.494860000000001e+02 +7.700000000000000e+01
+1.083580000000000e+03 +1.640000000000000e+02
+1.262920000000000e+03 +2.740000000000000e+02
+1.486280000000000e+03 +4.970000000000000e+02
+1.331130000000000e+03 +3.940000000000000e+02
+9.293240000000000e+02 +3.130000000000000e+02
+4.898690000000000e+02 +2.920000000000000e+02
+8.978480000000002e+02 +9.700000000000000e+01
+7.506510000000002e+02 +1.030000000000000e+02
+1.122080000000000e+03 +1.920000000000000e+02
+1.490830000000000e+03 +4.660000000000000e+02
+8.990810000000000e+02 +8.050000000000000e+01
+8.882350000000000e+02 +3.430000000000000e+02
+4.851450000000000e+02 +2.800000000000000e+02
+7.883810000000002e+02 +1.170000000000000e+02
+7.897669999999998e+02 +1.305000000000000e+02
+1.447860000000000e+03 +1.745000000000000e+02
+1.387410000000000e+03 +4.305000000000000e+02
+7.674110000000002e+02 +9.700000000000000e+01
+1.574970000000000e+03 +4.715000000000000e+02
+1.323760000000000e+03 +4.370000000000000e+02
+2.303800000000000e+03 +6.265000000000000e+02
+7.688270000000000e+02 +1.130000000000000e+02
+1.106890000000000e+03 +1.435000000000000e+02
+6.645069999999999e+02 +5.925000000000000e+02
+7.523550000000000e+02 +9.300000000000000e+01
+7.819510000000000e+02 +1.265000000000000e+02
+7.490310000000002e+02 +3.855000000000000e+02
+1.570120000000000e+03 +2.990000000000000e+02
+2.285540000000000e+03 +6.725000000000000e+02
+9.091660000000001e+02 +9.900000000000000e+01
+8.961500000000000e+02 +9.200000000000000e+01
+9.351220000000000e+02 +3.975000000000000e+02
+2.288310000000000e+03 +6.590000000000000e+02
+7.734989999999998e+02 +1.180000000000000e+02
+1.331520000000000e+03 +4.080000000000000e+02
+1.303500000000000e+03 +4.185000000000000e+02
+7.708600000000000e+02 +1.035000000000000e+02
+8.711430000000000e+02 +9.550000000000000e+01
+7.612539999999998e+02 +3.405000000000000e+02
+7.847160000000000e+02 +9.000000000000000e+01
+4.840740000000000e+02 +5.800000000000000e+01
+1.848500000000000e+03 +6.090000000000000e+02
+9.162560000000000e+02 +1.215000000000000e+02
+1.306430000000000e+03 +2.185000000000000e+02
+8.760230000000000e+02 +8.650000000000000e+01
+1.324270000000000e+03 +3.870000000000000e+02
+7.858980000000000e+02 +9.800000000000000e+01
+7.905730000000000e+02 +1.195000000000000e+02
+1.066090000000000e+03 +4.585000000000000e+02
+1.294980000000000e+03 +1.925000000000000e+02
+8.295260000000002e+02 +6.300000000000000e+01
+1.490780000000000e+03 +4.085000000000000e+02
+9.014610000000000e+02 +7.900000000000000e+01
+7.802000000000000e+02 +8.600000000000000e+01
+1.594770000000000e+03 +5.610000000000000e+02
+1.858860000000000e+03 +6.660000000000000e+02
+9.581900000000001e+02 +1.275000000000000e+02
+7.817189999999998e+02 +9.500000000000000e+01
+7.744240000000000e+02 +1.145000000000000e+02
+1.305670000000000e+03 +1.790000000000000e+02
+1.462350000000000e+03 +4.200000000000000e+02
+1.315860000000000e+03 +1.875000000000000e+02
+7.817780000000000e+02 +1.140000000000000e+02
+9.116750000000000e+02 +7.150000000000000e+01
+7.860480000000000e+02 +9.200000000000000e+01
+1.294830000000000e+03 +1.945000000000000e+02
+8.862960000000000e+02 +7.550000000000000e+01
+1.298210000000000e+03 +1.945000000000000e+02
+9.696079999999999e+02 +1.440000000000000e+02
+7.572710000000002e+02 +8.300000000000000e+01
+6.451870000000000e+02 +5.160000000000000e+02
+1.320640000000000e+03 +1.945000000000000e+02
+8.875950000000000e+02 +7.550000000000000e+01
+7.806450000000000e+02 +1.095000000000000e+02
+7.766790000000000e+02 +8.500000000000000e+01
+8.641189999999998e+02 +1.010000000000000e+02
+8.695549999999999e+02 +7.250000000000000e+01
+3.250050000000000e+02 +2.700000000000000e+01
+7.675219999999998e+02 +7.900000000000000e+01
+7.810360000000002e+02 +1.080000000000000e+02
+1.309880000000000e+03 +2.235000000000000e+02
+8.779540000000000e+02 +7.100000000000000e+01
+7.816180000000001e+02 +7.800000000000000e+01
+7.910930000000002e+02 +9.350000000000000e+01
+9.779950000000000e+02 +1.180000000000000e+02
+1.299220000000000e+03 +1.430000000000000e+02
+9.750280000000000e+02 +1.125000000000000e+02
+1.306830000000000e+03 +2.000000000000000e+02
+7.471630000000000e+02 +7.350000000000000e+01
+7.688489999999998e+02 +1.030000000000000e+02
+1.296910000000000e+03 +1.375000000000000e+02
+1.020450000000000e+03 +8.050000000000000e+02
+2.174620000000000e+03 +7.980000000000000e+02
+1.394640000000000e+03 +7.955000000000000e+02
+1.415470000000000e+03 +7.930000000000000e+02
+1.699310000000000e+03 +7.885000000000000e+02
+9.874700000000000e+02 +7.850000000000000e+02
+9.637840000000000e+02 +7.845000000000000e+02
+2.358990000000000e+03 +7.800000000000000e+02
+9.903240000000000e+02 +7.780000000000000e+02
+1.360800000000000e+03 +7.775000000000000e+02
+1.655260000000000e+03 +7.695000000000000e+02
+1.367260000000000e+03 +7.585000000000000e+02
+1.809830000000000e+03 +7.460000000000000e+02
+1.362030000000000e+03 +7.385000000000000e+02
+1.789210000000000e+03 +7.370000000000000e+02
+1.829130000000000e+02 +7.275000000000000e+02
+1.350870000000000e+03 +7.240000000000000e+02
+8.403130000000000e+02 +7.220000000000000e+02
+2.621520000000000e+03 +7.200000000000000e+02
+3.283630000000001e+02 +7.200000000000000e+02
+9.705810000000000e+02 +7.200000000000000e+02
+1.268130000000000e+03 +7.180000000000000e+02
+1.797950000000000e+03 +7.165000000000000e+02
+1.714840000000000e+02 +7.160000000000000e+02
+1.358360000000000e+03 +7.160000000000000e+02
+1.235910000000000e+03 +7.155000000000000e+02
+1.446990000000000e+03 +7.130000000000000e+02
+1.727790000000000e+03 +7.020000000000000e+02
+9.973840000000000e+02 +7.015000000000000e+02
+1.325460000000000e+03 +6.960000000000000e+02
+1.476700000000000e+03 +6.885000000000000e+02
+1.181600000000000e+03 +6.835000000000000e+02
+8.411560000000002e+02 +6.825000000000000e+02
+1.538320000000000e+03 +6.740000000000000e+02
+1.825580000000000e+03 +6.725000000000000e+02
+8.380599999999999e+02 +6.695000000000000e+02
+1.234830000000000e+03 +6.670000000000000e+02
+1.804490000000000e+03 +6.665000000000000e+02
+1.230570000000000e+03 +6.650000000000000e+02
+1.035990000000000e+03 +6.625000000000000e+02
+3.029180000000000e+02 +6.610000000000000e+02
+1.915630000000000e+03 +6.600000000000000e+02
+1.582980000000000e+03 +6.590000000000000e+02
+1.341550000000000e+03 +6.580000000000000e+02
+1.846260000000000e+03 +6.560000000000000e+02
+1.835020000000000e+03 +6.545000000000000e+02
+8.382160000000000e+02 +6.540000000000000e+02
+1.075590000000000e+03 +6.465000000000000e+02
+9.359710000000000e+02 +6.445000000000000e+02
+9.836330000000000e+02 +6.440000000000000e+02
+2.722150000000000e+02 +6.405000000000000e+02
+9.878070000000000e+02 +6.405000000000000e+02
+1.197420000000000e+03 +6.380000000000000e+02
+9.662440000000000e+02 +6.360000000000000e+02
+9.511480000000000e+02 +6.355000000000000e+02
+9.206860000000000e+02 +6.350000000000000e+02
+1.041490000000000e+03 +6.285000000000000e+02
+9.164100000000000e+02 +6.235000000000000e+02
+1.025440000000000e+03 +6.235000000000000e+02
+1.485660000000000e+03 +6.230000000000000e+02
+1.419000000000000e+03 +6.230000000000000e+02
+1.208660000000000e+03 +6.210000000000000e+02
+9.509860000000000e+02 +6.205000000000000e+02
+9.277120000000000e+02 +6.170000000000000e+02
+1.331510000000000e+03 +6.160000000000000e+02
+8.611910000000000e+02 +6.160000000000000e+02
+2.901320000000000e+02 +6.160000000000000e+02
+1.865600000000000e+03 +6.150000000000000e+02
+9.188060000000000e+02 +6.120000000000000e+02
+8.358180000000000e+02 +6.085000000000000e+02
+9.010800000000000e+02 +6.085000000000000e+02
+8.972930000000000e+02 +6.080000000000000e+02
+1.247560000000000e+03 +6.075000000000000e+02
+7.816780000000000e+02 +6.075000000000000e+02
+9.221470000000000e+02 +6.070000000000000e+02
+1.166420000000000e+03 +6.045000000000000e+02
+6.741630000000000e+02 +6.040000000000000e+02
+1.310110000000000e+03 +6.030000000000000e+02
+1.538580000000000e+03 +6.020000000000000e+02
+8.640530000000000e+02 +6.005000000000000e+02
+1.896150000000000e+03 +6.000000000000000e+02
+2.841880000000000e+02 +5.970000000000000e+02
+2.585660000000000e+02 +5.965000000000000e+02
+1.431180000000000e+03 +5.955000000000000e+02
+6.522430000000001e+02 +5.945000000000000e+02
+1.849430000000000e+03 +5.935000000000000e+02
+2.722010000000000e+02 +5.925000000000000e+02
+1.370630000000000e+03 +5.925000000000000e+02
+1.864220000000000e+03 +5.910000000000000e+02
+1.102730000000000e+03 +5.900000000000000e+02
+6.696350000000000e+02 +5.895000000000000e+02
+1.279040000000000e+03 +5.895000000000000e+02
+1.300090000000000e+03 +5.890000000000000e+02
+1.894520000000000e+03 +5.870000000000000e+02
+7.828420000000000e+02 +5.865000000000000e+02
+1.285720000000000e+03 +5.850000000000000e+02
+9.953500000000000e+02 +5.840000000000000e+02
+9.021920000000000e+02 +5.815000000000000e+02
+1.481420000000000e+03 +5.815000000000000e+02
+1.091760000000000e+03 +5.805000000000000e+02
+1.880570000000000e+03 +5.800000000000000e+02
+8.520169999999998e+02 +5.790000000000000e+02
+7.703370000000000e+02 +5.785000000000000e+02
+8.607830000000000e+02 +5.770000000000000e+02
+1.273030000000000e+03 +5.770000000000000e+02
+8.046730000000000e+02 +5.750000000000000e+02
+2.654310000000000e+02 +5.750000000000000e+02
+8.128700000000000e+02 +5.740000000000000e+02
+1.075850000000000e+03 +5.740000000000000e+02
+1.267240000000000e+03 +5.730000000000000e+02
+8.083489999999998e+02 +5.710000000000000e+02
+1.196520000000000e+03 +5.710000000000000e+02
+7.733339999999999e+02 +5.705000000000000e+02
+1.688430000000000e+03 +5.705000000000000e+02
+1.259170000000000e+03 +5.685000000000000e+02
+8.001790000000000e+02 +5.675000000000000e+02
+1.292170000000000e+03 +5.670000000000000e+02
+7.711389999999999e+02 +5.660000000000000e+02
+1.300890000000000e+03 +5.660000000000000e+02
+1.285210000000000e+03 +5.650000000000000e+02
+7.847370000000000e+02 +5.650000000000000e+02
+7.891310000000002e+02 +5.645000000000000e+02
+1.151220000000000e+03 +5.640000000000000e+02
+1.047530000000000e+03 +5.640000000000000e+02
+1.265880000000000e+03 +5.635000000000000e+02
+7.735790000000000e+02 +5.625000000000000e+02
+7.357410000000001e+02 +5.625000000000000e+02
+7.878620000000000e+02 +5.615000000000000e+02
+2.603160000000000e+02 +5.605000000000000e+02
+1.849670000000000e+03 +5.600000000000000e+02
+2.525920000000000e+02 +5.595000000000000e+02
+7.952470000000000e+02 +5.585000000000000e+02
+1.870290000000000e+03 +5.580000000000000e+02
+1.070460000000000e+03 +5.575000000000000e+02
+7.793850000000000e+02 +5.565000000000000e+02
+1.482070000000000e+03 +5.560000000000000e+02
+5.035680000000000e+02 +5.540000000000000e+02
+7.925760000000000e+01 +5.530000000000000e+02
+1.254900000000000e+03 +5.530000000000000e+02
+7.563320000000000e+02 +5.525000000000000e+02
+1.090080000000000e+03 +5.525000000000000e+02
+4.932260000000000e+02 +5.520000000000000e+02
+7.860080000000000e+02 +5.520000000000000e+02
+8.185110000000002e+02 +5.515000000000000e+02
+8.004970000000000e+02 +5.510000000000000e+02
+8.464480000000000e+02 +5.505000000000000e+02
+1.294390000000000e+03 +5.500000000000000e+02
+6.935139999999999e+02 +5.495000000000000e+02
+5.514840000000000e+02 +5.480000000000000e+02
+1.187400000000000e+03 +5.470000000000000e+02
+1.142480000000000e+03 +5.455000000000000e+02
+6.812360000000001e+02 +5.450000000000000e+02
+7.910910000000000e+02 +5.445000000000000e+02
+9.241350000000000e+02 +5.405000000000000e+02
+6.907589999999999e+02 +5.400000000000000e+02
+1.804930000000000e+03 +5.400000000000000e+02
+2.410080000000000e+02 +5.400000000000000e+02
+1.931210000000000e+03 +5.400000000000000e+02
+9.220740000000000e+02 +5.400000000000000e+02
+7.027760000000002e+02 +5.380000000000000e+02
+7.209130000000000e+02 +5.365000000000000e+02
+6.976720000000000e+02 +5.365000000000000e+02
+7.072539999999998e+02 +5.360000000000000e+02
+9.783440000000001e+02 +5.360000000000000e+02
+2.583200000000000e+02 +5.350000000000000e+02
+2.168880000000000e+02 +5.335000000000000e+02
+7.699090000000000e+02 +5.320000000000000e+02
+7.291089999999998e+02 +5.315000000000000e+02
+6.778480000000002e+02 +5.305000000000000e+02
+6.958889999999999e+02 +5.300000000000000e+02
+6.737010000000000e+02 +5.295000000000000e+02
+7.567869999999998e+02 +5.285000000000000e+02
+2.328130000000000e+02 +5.280000000000000e+02
+9.319349999999999e+02 +5.255000000000000e+02
+1.114290000000000e+03 +5.245000000000000e+02
+7.142739999999999e+02 +5.235000000000000e+02
+7.518170000000000e+02 +5.230000000000000e+02
+7.122350000000000e+02 +5.225000000000000e+02
+2.519970000000000e+02 +5.225000000000000e+02
+1.115800000000000e+03 +5.220000000000000e+02
+1.586350000000000e+03 +5.220000000000000e+02
+1.163740000000000e+03 +5.220000000000000e+02
+7.679630000000002e+02 +5.210000000000000e+02
+7.521669999999998e+02 +5.210000000000000e+02
+4.613470000000000e+02 +5.210000000000000e+02
+6.469119999999998e+02 +5.205000000000000e+02
+2.543990000000000e+02 +5.200000000000000e+02
+7.480419999999998e+02 +5.195000000000000e+02
+7.378680000000001e+02 +5.195000000000000e+02
+1.231270000000000e+03 +5.195000000000000e+02
+4.845190000000000e+02 +5.185000000000000e+02
+1.181750000000000e+03 +5.185000000000000e+02
+6.664980000000000e+02 +5.175000000000000e+02
+7.565760000000000e+02 +5.170000000000000e+02
+4.665940000000000e+02 +5.165000000000000e+02
+4.602380000000001e+02 +5.155000000000000e+02
+9.667180000000000e+02 +5.155000000000000e+02
+8.507930000000000e+02 +5.150000000000000e+02
+9.988400000000000e+02 +5.150000000000000e+02
+6.438910000000000e+02 +5.145000000000000e+02
+9.008360000000000e+02 +5.135000000000000e+02
+7.670750000000000e+02 +5.130000000000000e+02
+7.652160000000000e+02 +5.120000000000000e+02
+1.342270000000000e+03 +5.120000000000000e+02
+6.899430000000000e+02 +5.115000000000000e+02
+8.138160000000000e+02 +5.110000000000000e+02
+1.210190000000000e+03 +5.110000000000000e+02
+7.592619999999999e+02 +5.085000000000000e+02
+4.475740000000000e+02 +5.080000000000000e+02
+8.907470000000000e+02 +5.080000000000000e+02
+1.156600000000000e+03 +5.065000000000000e+02
+2.185480000000000e+02 +5.065000000000000e+02
+7.434340000000000e+02 +5.055000000000000e+02
+2.394020000000000e+02 +5.045000000000000e+02
+1.340860000000000e+03 +5.045000000000000e+02
+7.358580000000002e+02 +5.030000000000000e+02
+4.336640000000000e+02 +5.030000000000000e+02
+9.475560000000000e+02 +5.025000000000000e+02
+1.098050000000000e+03 +5.025000000000000e+02
+2.321540000000000e+02 +5.020000000000000e+02
+6.851750000000000e+02 +4.995000000000000e+02
+4.119740000000000e+02 +4.990000000000000e+02
+3.994740000000000e+02 +4.960000000000000e+02
+9.489860000000000e+02 +4.955000000000000e+02
+6.044930000000001e+02 +4.955000000000000e+02
+6.973020000000000e+02 +4.950000000000000e+02
+8.556020000000000e+02 +4.945000000000000e+02
+6.035860000000000e+02 +4.940000000000000e+02
+7.167850000000000e+02 +4.935000000000000e+02
+6.053750000000000e+02 +4.935000000000000e+02
+7.100119999999999e+02 +4.930000000000000e+02
+3.987690000000000e+02 +4.920000000000000e+02
+3.370390000000000e+02 +4.920000000000000e+02
+3.724370000000000e+02 +4.915000000000000e+02
+9.894290000000000e+02 +4.905000000000000e+02
+2.110120000000000e+02 +4.900000000000000e+02
+7.106239999999998e+02 +4.890000000000000e+02
+8.669270000000000e+02 +4.890000000000000e+02
+1.066710000000000e+03 +4.890000000000000e+02
+5.830050000000000e+02 +4.885000000000000e+02
+6.346040000000000e+02 +4.875000000000000e+02
+1.091920000000000e+03 +4.875000000000000e+02
+8.570269999999998e+02 +4.860000000000000e+02
+4.731080000000000e+02 +4.855000000000000e+02
+7.749810000000001e+02 +4.850000000000000e+02
+1.579520000000000e+03 +4.850000000000000e+02
+3.576110000000000e+02 +4.845000000000000e+02
+7.734530000000000e+02 +4.845000000000000e+02
+3.603360000000000e+02 +4.840000000000000e+02
+5.722070000000000e+02 +4.830000000000000e+02
+6.225190000000000e+02 +4.830000000000000e+02
+6.452790000000000e+02 +4.825000000000000e+02
+9.972770000000000e+02 +4.820000000000000e+02
+8.170530000000000e+02 +4.815000000000000e+02
+9.320430000000000e+02 +4.815000000000000e+02
+1.030080000000000e+03 +4.810000000000000e+02
+5.555710000000000e+02 +4.805000000000000e+02
+9.969720000000000e+02 +4.805000000000000e+02
+1.418140000000000e+03 +4.800000000000000e+02
+9.425100000000000e+02 +4.800000000000000e+02
+6.418090000000000e+02 +4.795000000000000e+02
+3.772340000000000e+02 +4.790000000000000e+02
+5.485419999999998e+02 +4.790000000000000e+02
+2.028690000000000e+02 +4.790000000000000e+02
+3.849690000000000e+02 +4.780000000000000e+02
+5.132840000000000e+02 +4.780000000000000e+02
+9.255250000000000e+02 +4.780000000000000e+02
+7.108410000000000e+02 +4.775000000000000e+02
+7.731369999999999e+02 +4.775000000000000e+02
+3.532680000000001e+02 +4.770000000000000e+02
+9.856470000000000e+02 +4.770000000000000e+02
+8.444750000000000e+02 +4.770000000000000e+02
+9.655820000000000e+02 +4.765000000000000e+02
+3.692870000000000e+02 +4.760000000000000e+02
+3.689780000000000e+02 +4.755000000000000e+02
+6.474500000000000e+02 +4.755000000000000e+02
+6.537180000000002e+02 +4.755000000000000e+02
+9.652540000000000e+02 +4.750000000000000e+02
+7.053210000000000e+02 +4.745000000000000e+02
+1.004270000000000e+03 +4.745000000000000e+02
+7.639330000000000e+02 +4.745000000000000e+02
+5.166940000000000e+02 +4.740000000000000e+02
+5.078040000000000e+02 +4.740000000000000e+02
+3.974870000000000e+02 +4.730000000000000e+02
+5.200040000000000e+02 +4.730000000000000e+02
+8.615210000000002e+02 +4.730000000000000e+02
+3.310740000000000e+02 +4.725000000000000e+02
+5.209150000000000e+02 +4.720000000000000e+02
+3.556260000000000e+02 +4.715000000000000e+02
+1.309510000000000e+03 +4.710000000000000e+02
+6.598720000000000e+02 +4.705000000000000e+02
+3.432680000000001e+02 +4.700000000000000e+02
+8.677660000000002e+02 +4.685000000000000e+02
+9.425270000000000e+02 +4.685000000000000e+02
+3.579010000000000e+02 +4.680000000000000e+02
+6.515180000000000e+02 +4.670000000000000e+02
+6.444780000000002e+02 +4.670000000000000e+02
+9.013900000000000e+02 +4.660000000000000e+02
+7.943620000000000e+02 +4.655000000000000e+02
+6.895239999999999e+02 +4.650000000000000e+02
+3.260870000000000e+02 +4.640000000000000e+02
+5.797909999999998e+02 +4.640000000000000e+02
+6.943310000000000e+02 +4.610000000000000e+02
+9.348760000000000e+02 +4.635000000000000e+02
+9.427120000000000e+02 +4.630000000000000e+02
+5.224790000000000e+02 +4.625000000000000e+02
+5.015290000000000e+02 +4.620000000000000e+02
+4.720980000000000e+02 +4.610000000000000e+02
+7.803300000000000e+02 +4.605000000000000e+02
+1.004510000000000e+03 +4.605000000000000e+02
+9.297240000000000e+02 +4.600000000000000e+02
+6.818090000000000e+02 +4.575000000000000e+02
+1.053190000000000e+03 +4.590000000000000e+02
+3.409060000000000e+02 +4.540000000000000e+02
+7.831840000000000e+02 +4.575000000000000e+02
+5.461600000000000e+02 +4.565000000000000e+02
+7.547639999999999e+02 +4.560000000000000e+02
+6.812830000000000e+02 +4.540000000000000e+02
+3.340600000000000e+02 +4.535000000000000e+02
+6.735839999999999e+02 +4.530000000000000e+02
+5.204670000000000e+02 +4.530000000000000e+02
+7.552730000000000e+02 +4.530000000000000e+02
+6.565830000000002e+02 +4.490000000000000e+02
+9.033560000000000e+02 +4.525000000000000e+02
+7.439950000000000e+02 +4.520000000000000e+02
+4.722760000000000e+02 +4.515000000000000e+02
+1.053110000000000e+03 +4.515000000000000e+02
+5.216990000000002e+02 +4.510000000000000e+02
+7.373700000000000e+02 +4.495000000000000e+02
+6.691920000000000e+02 +4.490000000000000e+02
+7.965369999999998e+02 +4.485000000000000e+02
+3.470790000000000e+02 +4.480000000000000e+02
+1.250350000000000e+03 +4.480000000000000e+02
+7.257250000000000e+02 +4.475000000000000e+02
+1.016720000000000e+03 +4.475000000000000e+02
+7.078700000000000e+02 +4.470000000000000e+02
+9.279640000000001e+02 +4.470000000000000e+02
+5.084970000000000e+02 +4.465000000000000e+02
+1.363510000000000e+03 +4.465000000000000e+02
+9.104240000000000e+02 +4.465000000000000e+02
+6.512250000000000e+02 +4.460000000000000e+02
+5.142270000000000e+02 +4.450000000000000e+02
+6.213460000000000e+02 +4.440000000000000e+02
+4.676730000000000e+02 +4.435000000000000e+02
+4.411210000000000e+02 +4.430000000000000e+02
+7.083520000000000e+02 +4.430000000000000e+02
+7.236860000000000e+02 +4.425000000000000e+02
+8.719739999999998e+02 +4.420000000000000e+02
+4.845780000000000e+02 +4.410000000000000e+02
+1.210740000000000e+03 +4.400000000000000e+02
+7.044639999999998e+02 +4.400000000000000e+02
+9.676470000000000e+02 +4.395000000000000e+02
+3.293630000000001e+02 +4.370000000000000e+02
+7.683220000000000e+02 +4.365000000000000e+02
+6.218610000000000e+02 +4.360000000000000e+02
+7.377639999999999e+02 +4.355000000000000e+02
+5.625549999999999e+02 +4.350000000000000e+02
+9.015560000000000e+02 +4.350000000000000e+02
+1.219510000000000e+03 +4.345000000000000e+02
+3.690850000000000e+02 +4.340000000000000e+02
+1.067090000000000e+03 +4.340000000000000e+02
+6.840860000000000e+02 +4.330000000000000e+02
+3.253750000000000e+02 +4.320000000000000e+02
+8.203500000000000e+02 +4.315000000000000e+02
+9.004580000000002e+02 +4.315000000000000e+02
+9.335660000000000e+02 +4.310000000000000e+02
+5.295090000000000e+02 +4.300000000000000e+02
+7.318049999999999e+02 +4.300000000000000e+02
+7.874260000000000e+02 +4.290000000000000e+02
+9.140800000000000e+02 +4.290000000000000e+02
+6.007240000000000e+02 +4.285000000000000e+02
+1.302880000000000e+03 +4.275000000000000e+02
+6.172890000000000e+02 +4.270000000000000e+02
+6.397480000000000e+02 +4.250000000000000e+02
+7.485780000000000e+02 +4.245000000000000e+02
+1.241700000000000e+02 +4.245000000000000e+02
+3.074270000000000e+02 +4.240000000000000e+02
+9.469650000000000e+02 +4.235000000000000e+02
+8.945690000000000e+02 +4.235000000000000e+02
+3.056180000000000e+02 +4.215000000000000e+02
+6.013400000000000e+02 +4.215000000000000e+02
+5.471080000000002e+02 +4.210000000000000e+02
+1.164380000000000e+03 +4.210000000000000e+02
+6.386709999999998e+02 +4.200000000000000e+02
+2.919350000000000e+02 +4.200000000000000e+02
+1.421390000000000e+03 +4.200000000000000e+02
+1.810920000000000e+03 +4.200000000000000e+02
+4.926250000000000e+02 +4.190000000000000e+02
+7.700330000000000e+02 +4.185000000000000e+02
+8.888720000000000e+02 +4.185000000000000e+02
+6.543520000000000e+02 +4.180000000000000e+02
+9.182700000000000e+02 +4.175000000000000e+02
+7.525939999999998e+02 +4.170000000000000e+02
+1.278490000000000e+03 +4.160000000000000e+02
+8.642189999999998e+02 +4.155000000000000e+02
+1.031740000000000e+03 +4.155000000000000e+02
+7.645230000000000e+02 +4.155000000000000e+02
+7.070180000000000e+02 +4.150000000000000e+02
+1.053600000000000e+03 +4.145000000000000e+02
+5.247390000000000e+02 +4.135000000000000e+02
+6.242480000000000e+02 +4.135000000000000e+02
+7.612730000000000e+02 +4.130000000000000e+02
+5.644410000000000e+02 +4.125000000000000e+02
+2.855340000000000e+02 +4.120000000000000e+02
+6.011910000000000e+02 +4.115000000000000e+02
+3.267320000000000e+02 +4.105000000000000e+02
+2.848990000000000e+02 +4.085000000000000e+02
+9.680990000000000e+02 +4.075000000000000e+02
+1.926670000000000e+02 +4.075000000000000e+02
+2.705740000000000e+02 +4.060000000000000e+02
+7.264080000000000e+02 +4.060000000000000e+02
+8.040490000000000e+01 +4.045000000000000e+02
+1.874710000000000e+02 +4.040000000000000e+02
+1.178580000000000e+02 +4.035000000000000e+02
+2.320450000000000e+02 +4.025000000000000e+02
+9.398430000000000e+02 +4.025000000000000e+02
+1.340610000000000e+02 +4.015000000000000e+02
+7.230920000000000e+02 +4.015000000000000e+02
+5.771210000000000e+02 +4.005000000000000e+02
+2.026470000000000e+02 +4.000000000000000e+02
+6.790150000000000e+02 +4.000000000000000e+02
+7.737239999999998e+02 +3.995000000000000e+02
+8.110419999999998e+02 +3.990000000000000e+02
+4.747360000000000e+02 +3.985000000000000e+02
+9.228690000000000e+02 +3.985000000000000e+02
+7.636920000000000e+02 +3.985000000000000e+02
+1.075200000000000e+02 +3.975000000000000e+02
+7.695150000000000e+02 +3.975000000000000e+02
+5.663500000000000e+01 +3.975000000000000e+02
+6.860280000000000e+02 +3.970000000000000e+02
+2.256130000000000e+02 +3.970000000000000e+02
+1.611750000000000e+02 +3.965000000000000e+02
+7.501980000000000e+02 +3.965000000000000e+02
+1.004220000000000e+02 +3.950000000000000e+02
+9.225970000000000e+02 +3.950000000000000e+02
+1.043190000000000e+03 +3.940000000000000e+02
+6.026100000000000e+02 +3.935000000000000e+02
+3.038180000000000e+01 +3.930000000000000e+02
+9.430839999999999e+02 +3.930000000000000e+02
+6.310610000000000e+02 +3.925000000000000e+02
+2.004960000000000e+01 +3.910000000000000e+02
+1.672030000000000e+02 +3.910000000000000e+02
+8.678860000000002e+02 +3.910000000000000e+02
+1.111410000000000e+03 +3.910000000000000e+02
+1.305700000000000e+02 +3.905000000000000e+02
+1.752030000000000e+02 +3.900000000000000e+02
+3.583270000000000e+02 +3.895000000000000e+02
+2.184000000000000e+02 +3.895000000000000e+02
+5.371080000000002e+02 +3.890000000000000e+02
+1.089940000000000e+03 +3.885000000000000e+02
+6.587680000000000e+02 +3.885000000000000e+02
+8.964789999999998e+02 +3.885000000000000e+02
+7.736900000000001e+02 +3.880000000000000e+02
+1.420170000000000e+01 +3.875000000000000e+02
+2.162840000000000e+02 +3.870000000000000e+02
+8.273080000000000e+02 +3.870000000000000e+02
+6.580380000000000e+02 +3.870000000000000e+02
+8.317960000000000e+02 +3.865000000000000e+02
+7.465660000000000e+02 +3.865000000000000e+02
+9.086290000000000e+02 +3.860000000000000e+02
+6.566830000000000e+02 +3.850000000000000e+02
+4.105240000000000e+01 +3.850000000000000e+02
+5.459280000000000e+01 +3.845000000000000e+02
+8.342230000000002e+02 +3.845000000000000e+02
+8.205330000000000e+02 +3.845000000000000e+02
+3.765330000000000e+01 +3.840000000000000e+02
+5.740069999999999e+02 +3.835000000000000e+02
+7.072320000000000e+02 +3.835000000000000e+02
+6.250520000000000e+02 +3.830000000000000e+02
+1.139760000000000e+03 +3.815000000000000e+02
+9.149680000000000e+02 +3.815000000000000e+02
+1.209950000000000e+01 +3.810000000000000e+02
+6.849130000000000e+02 +3.810000000000000e+02
+2.398020000000000e+01 +3.805000000000000e+02
+9.137590000000000e+02 +3.805000000000000e+02
+8.631150000000000e+02 +3.805000000000000e+02
+1.867090000000000e+01 +3.800000000000000e+02
+8.544589999999999e+02 +3.795000000000000e+02
+1.318860000000000e+03 +3.785000000000000e+02
+8.727450000000000e+02 +3.780000000000000e+02
+8.613500000000000e+02 +3.780000000000000e+02
+1.112030000000000e+03 +3.775000000000000e+02
+6.270040000000000e+02 +3.770000000000000e+02
+8.519430000000000e+02 +3.770000000000000e+02
+6.635030000000000e+02 +3.765000000000000e+02
+2.328670000000000e+02 +3.760000000000000e+02
+2.239760000000000e+02 +3.760000000000000e+02
+1.172010000000000e+03 +3.760000000000000e+02
+5.475180000000000e+02 +3.750000000000000e+02
+6.968230000000000e+02 +3.750000000000000e+02
+2.552480000000000e+01 +3.735000000000000e+02
+6.121280000000000e+02 +3.730000000000000e+02
+5.977890000000000e+02 +3.725000000000000e+02
+1.754970000000000e+02 +3.125000000000000e+02
+6.978969999999998e+02 +3.720000000000000e+02
+7.907760000000002e+02 +3.715000000000000e+02
+1.662710000000000e+02 +2.920000000000000e+02
+1.545500000000000e+02 +3.705000000000000e+02
+8.847940000000000e+02 +3.705000000000000e+02
+1.515170000000000e+02 +3.700000000000000e+02
+8.398639999999998e+02 +3.700000000000000e+02
+7.331619999999998e+02 +3.695000000000000e+02
+1.068030000000000e+02 +1.925000000000000e+02
+6.333560000000000e+02 +3.680000000000000e+02
+7.344250000000000e+02 +3.675000000000000e+02
+1.479710000000000e+02 +3.675000000000000e+02
+8.538660000000001e+02 +3.675000000000000e+02
+1.663920000000000e+02 +3.670000000000000e+02
+8.322389999999998e+02 +3.665000000000000e+02
+6.555350000000000e+02 +3.655000000000000e+02
+1.155210000000000e+01 +2.295000000000000e+02
+1.747460000000000e+02 +3.650000000000000e+02
+6.505440000000000e+02 +3.645000000000000e+02
+5.952580000000000e+02 +3.645000000000000e+02
+7.130419999999998e+02 +3.640000000000000e+02
+7.170760000000000e+02 +3.635000000000000e+02
+2.577220000000000e+02 +3.595000000000000e+02
+5.578500000000000e+02 +3.625000000000000e+02
+1.493040000000000e+02 +3.625000000000000e+02
+3.019980000000000e+01 +3.620000000000000e+02
+6.827120000000000e+02 +3.620000000000000e+02
+1.949910000000000e+01 +3.620000000000000e+02
+2.386600000000000e+02 +3.375000000000000e+02
+7.998989999999999e+02 +3.600000000000000e+02
+1.387760000000000e+03 +3.600000000000000e+02
+1.382070000000000e+02 +3.600000000000000e+02
+1.293960000000000e+03 +3.600000000000000e+02
+1.602460000000000e+03 +3.600000000000000e+02
+1.905020000000000e+03 +3.600000000000000e+02
+1.309860000000000e+03 +3.600000000000000e+02
+1.345880000000000e+03 +3.600000000000000e+02
+3.283090000000000e+02 +3.600000000000000e+02
+2.266570000000000e+01 +3.595000000000000e+02
+6.085549999999999e+02 +3.590000000000000e+02
+3.117710000000000e+01 +3.580000000000000e+02
+1.121370000000000e+03 +3.580000000000000e+02
+6.517540000000000e+02 +3.570000000000000e+02
+1.136940000000000e+03 +3.565000000000000e+02
+4.531090000000000e+02 +3.555000000000000e+02
+1.437320000000000e+01 +1.975000000000000e+02
+5.657260000000000e+02 +3.550000000000000e+02
+5.952950000000000e+02 +3.550000000000000e+02
+6.990030000000000e+02 +3.545000000000000e+02
+8.369950000000000e+02 +3.540000000000000e+02
+1.084310000000000e+03 +3.540000000000000e+02
+6.459970000000000e+02 +3.535000000000000e+02
+1.151980000000000e+03 +3.535000000000000e+02
+2.051950000000000e+02 +3.165000000000000e+02
+7.572630000000000e+02 +3.520000000000000e+02
+1.766300000000000e+02 +2.970000000000000e+02
+7.241130000000001e+02 +3.505000000000000e+02
+1.615090000000000e+01 +3.505000000000000e+02
+8.567800000000000e+02 +3.505000000000000e+02
+8.952539999999998e+02 +3.505000000000000e+02
+9.691220000000000e+02 +3.505000000000000e+02
+6.050300000000000e+02 +3.495000000000000e+02
+7.067050000000000e+02 +3.490000000000000e+02
+5.246200000000000e+02 +3.490000000000000e+02
+6.199460000000000e+02 +3.490000000000000e+02
+1.681160000000000e+02 +3.485000000000000e+02
+7.517650000000000e+02 +3.485000000000000e+02
+5.869230000000000e+02 +3.480000000000000e+02
+6.108490000000000e+02 +3.480000000000000e+02
+1.170730000000000e+01 +3.475000000000000e+02
+5.481810000000000e+02 +3.465000000000000e+02
+5.161310000000000e+02 +3.460000000000000e+02
+5.344200000000000e+02 +3.455000000000000e+02
+6.453510000000000e+02 +3.455000000000000e+02
+2.493730000000000e+02 +3.455000000000000e+02
+1.088070000000000e+03 +3.450000000000000e+02
+1.111620000000000e+02 +3.450000000000000e+02
+1.536320000000000e+01 +3.085000000000000e+02
+8.978429999999999e+00 +2.375000000000000e+02
+2.392920000000000e+01 +3.440000000000000e+02
+1.093490000000000e+03 +3.440000000000000e+02
+1.245310000000000e+02 +3.430000000000000e+02
+1.255120000000000e+02 +2.110000000000000e+02
+4.708770000000000e+02 +3.415000000000000e+02
+5.149209999999998e+02 +3.415000000000000e+02
+8.595180000000000e+02 +3.415000000000000e+02
+6.323020000000000e+02 +3.410000000000000e+02
+5.972170000000000e+02 +3.410000000000000e+02
+1.068380000000000e+03 +3.410000000000000e+02
+6.865630000000000e+01 +3.410000000000000e+02
+1.181130000000000e+02 +3.400000000000000e+02
+1.413720000000000e+01 +3.400000000000000e+02
+4.719500000000000e+02 +3.395000000000000e+02
+5.541740000000000e+02 +3.390000000000000e+02
+4.645580000000000e+02 +3.390000000000000e+02
+1.544940000000000e+01 +3.385000000000000e+02
+1.088470000000000e+02 +3.380000000000000e+02
+1.264580000000000e+01 +3.375000000000000e+02
+5.334410000000000e+02 +3.360000000000000e+02
+5.146540000000000e+02 +3.360000000000000e+02
+7.192930000000000e+02 +3.350000000000000e+02
+1.015690000000000e+02 +3.350000000000000e+02
+1.410020000000000e+02 +2.320000000000000e+02
+6.388490000000000e+02 +3.345000000000000e+02
+1.452300000000000e+03 +3.345000000000000e+02
+6.144500000000000e+02 +3.340000000000000e+02
+5.926540000000000e+02 +3.335000000000000e+02
+3.456100000000000e+02 +3.335000000000000e+02
+1.618440000000000e+02 +3.330000000000000e+02
+8.944060000000002e+02 +3.325000000000000e+02
+5.943670000000000e+02 +3.325000000000000e+02
+5.735610000000000e+02 +3.325000000000000e+02
+7.207110000000000e+02 +3.315000000000000e+02
+4.931610000000000e+02 +3.310000000000000e+02
+6.590640000000000e+01 +1.950000000000000e+02
+6.416110000000000e+02 +3.305000000000000e+02
+9.284930000000001e+02 +3.305000000000000e+02
+1.348620000000000e+02 +3.295000000000000e+02
+8.808930000000000e+02 +3.295000000000000e+02
+1.057310000000000e+02 +3.290000000000000e+02
+5.666830000000000e+02 +3.285000000000000e+02
+2.112340000000000e+02 +3.280000000000000e+02
+8.008380000000002e+02 +3.275000000000000e+02
+1.287990000000000e+02 +3.275000000000000e+02
+5.877100000000000e+02 +3.270000000000000e+02
+1.046200000000000e+02 +3.270000000000000e+02
+2.564270000000000e+02 +3.255000000000000e+02
+1.523280000000000e+02 +2.555000000000000e+02
+5.829330000000000e+02 +3.250000000000000e+02
+6.598030000000000e+02 +3.245000000000000e+02
+2.012010000000000e+01 +3.245000000000000e+02
+7.848099999999999e+02 +3.240000000000000e+02
+8.557180000000002e+02 +3.240000000000000e+02
+4.984740000000000e+02 +3.240000000000000e+02
+9.332100000000000e+01 +3.240000000000000e+02
+6.225100000000000e+02 +3.240000000000000e+02
+8.000490000000000e+02 +3.240000000000000e+02
+4.164150000000000e+02 +3.235000000000000e+02
+3.928830000000000e+02 +3.235000000000000e+02
+9.353420000000000e+01 +3.235000000000000e+02
+6.078260000000000e+02 +3.235000000000000e+02
+7.996339999999999e+02 +3.230000000000000e+02
+8.105690000000000e+02 +3.225000000000000e+02
+3.700970000000000e+02 +3.225000000000000e+02
+8.784980000000000e+02 +3.220000000000000e+02
+3.872470000000000e+02 +3.215000000000000e+02
+7.531380000000000e+02 +3.215000000000000e+02
+2.109980000000000e+02 +3.215000000000000e+02
+1.188590000000000e+02 +1.960000000000000e+02
+5.310630000000000e+02 +3.205000000000000e+02
+5.764990000000000e+02 +3.205000000000000e+02
+5.875380000000000e+02 +3.200000000000000e+02
+7.285200000000000e+02 +3.195000000000000e+02
+2.806230000000000e+02 +3.190000000000000e+02
+6.194290000000000e+02 +3.190000000000000e+02
+1.509010000000000e+02 +2.370000000000000e+02
+4.740890000000000e+02 +3.185000000000000e+02
+8.601030000000000e+01 +3.185000000000000e+02
+8.756260000000002e+02 +3.185000000000000e+02
+9.085359999999999e+02 +3.180000000000000e+02
+5.032530000000000e+02 +3.175000000000000e+02
+6.728620000000000e+02 +3.175000000000000e+02
+8.710630000000000e+01 +3.170000000000000e+02
+2.208770000000000e+02 +3.165000000000000e+02
+8.803220000000000e+01 +3.160000000000000e+02
+1.398430000000000e+01 +3.160000000000000e+02
+3.834820000000000e+02 +3.155000000000000e+02
+5.150720000000000e+02 +3.155000000000000e+02
+3.067360000000000e+02 +3.155000000000000e+02
+1.890140000000000e+02 +2.735000000000000e+02
+3.808890000000000e+02 +3.150000000000000e+02
+6.287120000000000e+02 +3.140000000000000e+02
+3.336900000000000e+01 +3.135000000000000e+02
+7.174440000000000e+02 +3.135000000000000e+02
+8.778789999999999e+01 +3.135000000000000e+02
+3.029580000000000e+02 +3.135000000000000e+02
+1.419900000000000e+03 +3.130000000000000e+02
+9.101880000000000e+01 +3.125000000000000e+02
+3.547020000000000e+02 +3.120000000000000e+02
+4.187490000000000e+02 +3.120000000000000e+02
+9.807089999999999e+01 +2.010000000000000e+02
+7.010770000000000e+02 +3.115000000000000e+02
+2.277820000000000e+02 +2.385000000000000e+02
+7.509450000000001e+02 +3.110000000000000e+02
+1.033480000000000e+03 +3.105000000000000e+02
+3.838650000000000e+02 +3.100000000000000e+02
+1.105660000000000e+03 +3.100000000000000e+02
+4.894600000000000e+02 +3.100000000000000e+02
+6.196569999999998e+02 +3.100000000000000e+02
+1.087830000000000e+03 +3.100000000000000e+02
+5.129280000000000e+02 +3.070000000000000e+02
+6.206669999999998e+02 +3.085000000000000e+02
+6.345050000000000e+02 +3.085000000000000e+02
+1.115830000000000e+02 +3.085000000000000e+02
+9.669040000000000e+01 +3.080000000000000e+02
+6.597970000000000e+02 +3.075000000000000e+02
+2.517190000000000e+02 +3.070000000000000e+02
+4.448560000000000e+02 +3.070000000000000e+02
+1.731530000000000e+02 +2.360000000000000e+02
+6.152960000000000e+00 +2.935000000000000e+02
+1.452320000000000e+03 +3.060000000000000e+02
+3.369510000000000e+02 +3.055000000000000e+02
+5.346170000000000e+02 +3.040000000000000e+02
+3.188010000000000e+02 +3.030000000000000e+02
+5.729610000000000e+02 +3.025000000000000e+02
+6.413049999999999e+02 +3.020000000000000e+02
+4.875960000000000e+02 +3.020000000000000e+02
+7.122070000000000e+02 +3.020000000000000e+02
+5.366319999999999e+02 +3.010000000000000e+02
+7.944720000000000e+01 +3.005000000000000e+02
+1.165390000000000e+03 +3.000000000000000e+02
+1.603220000000000e+03 +3.000000000000000e+02
+1.157820000000000e+03 +3.000000000000000e+02
+1.291380000000000e+03 +3.000000000000000e+02
+1.309970000000000e+03 +3.000000000000000e+02
+2.470190000000000e+02 +2.995000000000000e+02
+7.409100000000002e+01 +2.995000000000000e+02
+1.039750000000000e+03 +2.995000000000000e+02
+1.367300000000000e+02 +2.035000000000000e+02
+3.373400000000000e+02 +2.990000000000000e+02
+8.969020000000000e+01 +2.020000000000000e+02
+4.109930000000001e+02 +2.985000000000000e+02
+1.866630000000000e+01 +6.900000000000000e+01
+1.134160000000000e+01 +2.980000000000000e+02
+7.298950000000001e+01 +2.975000000000000e+02
+3.030800000000000e+02 +2.970000000000000e+02
+7.697660000000002e+02 +2.970000000000000e+02
+3.229320000000000e+02 +2.965000000000000e+02
+3.170850000000000e+02 +2.960000000000000e+02
+2.473100000000000e+02 +2.955000000000000e+02
+5.681350000000000e+02 +2.955000000000000e+02
+4.412540000000000e+02 +2.955000000000000e+02
+7.514130000000000e+02 +2.950000000000000e+02
+9.261870000000000e+00 +2.950000000000000e+02
+4.422970000000000e+02 +2.945000000000000e+02
+6.196830000000000e+02 +2.945000000000000e+02
+4.432640000000000e+01 +2.945000000000000e+02
+1.059030000000000e+03 +2.940000000000000e+02
+5.920240000000000e+02 +2.940000000000000e+02
+1.071620000000000e+03 +2.940000000000000e+02
+2.676140000000000e+02 +2.935000000000000e+02
+6.763920000000000e+01 +2.935000000000000e+02
+3.439420000000000e+02 +2.930000000000000e+02
+4.066970000000000e+02 +2.925000000000000e+02
+7.524050000000000e+01 +2.925000000000000e+02
+4.629170000000000e+02 +2.920000000000000e+02
+5.278969999999998e+02 +2.920000000000000e+02
+3.375260000000000e+00 +2.920000000000000e+02
+1.611810000000000e+02 +2.330000000000000e+02
+5.389059999999999e+02 +2.915000000000000e+02
+6.675600000000000e+01 +2.915000000000000e+02
+1.232590000000000e+03 +2.915000000000000e+02
+6.729789999999998e+02 +2.910000000000000e+02
+7.255329999999999e+01 +2.910000000000000e+02
+6.948560000000001e+01 +2.310000000000000e+02
+6.051530000000000e+02 +2.895000000000000e+02
+9.130050000000000e+02 +2.895000000000000e+02
+8.518320000000000e+02 +2.890000000000000e+02
+6.602139999999998e+02 +2.885000000000000e+02
+6.724450000000000e+01 +2.885000000000000e+02
+2.673230000000000e+02 +2.545000000000000e+02
+5.000650000000000e+02 +2.885000000000000e+02
+5.045800000000000e+02 +2.880000000000000e+02
+2.125030000000000e+02 +2.880000000000000e+02
+7.821930000000000e+02 +2.870000000000000e+02
+6.877460000000001e+01 +2.865000000000000e+02
+1.128690000000000e+03 +2.865000000000000e+02
+4.482290000000000e+02 +2.865000000000000e+02
+5.117870000000000e+02 +2.860000000000000e+02
+4.287780000000000e+02 +2.855000000000000e+02
+4.429820000000000e+02 +2.850000000000000e+02
+8.766170000000000e+02 +2.845000000000000e+02
+4.598470000000000e+02 +2.840000000000000e+02
+5.472930000000000e+01 +2.840000000000000e+02
+1.035620000000000e+03 +2.835000000000000e+02
+2.796840000000000e+01 +6.700000000000000e+01
+2.396510000000000e+02 +2.830000000000000e+02
+4.410980000000000e+02 +2.825000000000000e+02
+8.383560000000000e+01 +2.355000000000000e+02
+6.018120000000000e+01 +2.820000000000000e+02
+1.258190000000000e+01 +2.820000000000000e+02
+2.733700000000000e+02 +2.815000000000000e+02
+4.545200000000000e+02 +2.810000000000000e+02
+1.276910000000000e+03 +2.810000000000000e+02
+4.696420000000000e+02 +2.805000000000000e+02
+1.163870000000000e+02 +2.220000000000000e+02
+7.884540000000000e+02 +2.800000000000000e+02
+4.546450000000000e+02 +2.795000000000000e+02
+2.168330000000000e+02 +2.790000000000000e+02
+8.209180000000000e+02 +2.790000000000000e+02
+1.466410000000000e+03 +2.790000000000000e+02
+3.465820000000000e+02 +2.785000000000000e+02
+2.513680000000000e+02 +2.780000000000000e+02
+5.468090000000000e+02 +2.780000000000000e+02
+2.942960000000000e+02 +2.775000000000000e+02
+3.155970000000000e+02 +2.770000000000000e+02
+5.576490000000000e+01 +2.770000000000000e+02
+7.880060000000002e+02 +2.770000000000000e+02
+4.895070000000000e+02 +2.765000000000000e+02
+4.461190000000000e+01 +2.765000000000000e+02
+5.611830000000000e+02 +2.760000000000000e+02
+5.371880000000000e+01 +2.760000000000000e+02
+2.454670000000000e+02 +2.755000000000000e+02
+5.043630000000001e+02 +2.755000000000000e+02
+4.897560000000000e+02 +2.740000000000000e+02
+8.640300000000000e+02 +2.750000000000000e+02
+9.087750000000000e+02 +2.750000000000000e+02
+4.218160000000000e+02 +2.750000000000000e+02
+9.638410000000000e+02 +2.745000000000000e+02
+4.912970000000000e+01 +2.745000000000000e+02
+4.564580000000000e+02 +2.740000000000000e+02
+6.962810000000002e+02 +2.730000000000000e+02
+9.981140000000000e+02 +2.725000000000000e+02
+2.451530000000000e+02 +2.530000000000000e+02
+4.763080000000000e+02 +2.720000000000000e+02
+7.682719999999998e+02 +2.715000000000000e+02
+4.385240000000000e+02 +2.715000000000000e+02
+3.950660000000000e+02 +2.710000000000000e+02
+1.009270000000000e+03 +2.710000000000000e+02
+1.529340000000000e+02 +2.290000000000000e+02
+1.355050000000000e+02 +2.285000000000000e+02
+7.782510000000002e+02 +2.700000000000000e+02
+2.135040000000000e+02 +2.695000000000000e+02
+2.331730000000000e+01 +2.690000000000000e+02
+1.038300000000000e+03 +2.690000000000000e+02
+3.216110000000000e+02 +2.690000000000000e+02
+4.577730000000000e+02 +2.685000000000000e+02
+1.845050000000000e+01 +2.680000000000000e+02
+3.917380000000001e+02 +2.680000000000000e+02
+4.697700000000000e+02 +2.675000000000000e+02
+5.994560000000000e+02 +2.675000000000000e+02
+3.926770000000000e+02 +2.670000000000000e+02
+5.210720000000000e+02 +2.670000000000000e+02
+4.717590000000000e+01 +2.670000000000000e+02
+7.045280000000000e+02 +2.670000000000000e+02
+8.563180000000000e+02 +2.670000000000000e+02
+2.608480000000000e+02 +2.665000000000000e+02
+4.827570000000000e+01 +2.665000000000000e+02
+2.579320000000000e+02 +2.660000000000000e+02
+4.296170000000000e+01 +2.655000000000000e+02
+7.868140000000000e+02 +2.645000000000000e+02
+7.596880000000000e+02 +2.645000000000000e+02
+8.385530000000000e+02 +2.645000000000000e+02
+1.690440000000000e+02 +2.640000000000000e+02
+5.579660000000000e+02 +2.640000000000000e+02
+7.578070000000000e+02 +2.640000000000000e+02
+3.573200000000000e+01 +6.600000000000000e+01
+2.157550000000000e+02 +2.635000000000000e+02
+4.185630000000001e+02 +2.635000000000000e+02
+5.667809999999999e+02 +2.635000000000000e+02
+4.094030000000000e+00 +2.635000000000000e+02
+5.113080000000000e+02 +2.630000000000000e+02
+2.708910000000000e+02 +2.545000000000000e+02
+3.650540000000000e+01 +2.625000000000000e+02
+8.317669999999998e+02 +2.625000000000000e+02
+1.157330000000000e+02 +2.385000000000000e+02
+5.065670000000000e+01 +2.620000000000000e+02
+5.564450000000001e+02 +2.610000000000000e+02
+8.205430000000000e+02 +2.610000000000000e+02
+2.239790000000000e+01 +2.605000000000000e+02
+4.191700000000000e+01 +2.605000000000000e+02
+5.436910000000000e+02 +2.600000000000000e+02
+1.295800000000000e+01 +2.600000000000000e+02
+4.497930000000000e+01 +2.600000000000000e+02
+3.508350000000000e+02 +2.595000000000000e+02
+1.683690000000000e+03 +2.595000000000000e+02
+3.497800000000000e+02 +2.595000000000000e+02
+4.780890000000000e+02 +2.590000000000000e+02
+7.967750000000000e+00 +2.590000000000000e+02
+2.425410000000000e+02 +2.590000000000000e+02
+5.370950000000000e+02 +2.585000000000000e+02
+6.699950000000000e+01 +1.915000000000000e+02
+4.086200000000000e+02 +2.580000000000000e+02
+3.813720000000000e+01 +2.580000000000000e+02
+3.080720000000000e+02 +2.580000000000000e+02
+4.274890000000000e+02 +2.575000000000000e+02
+3.167650000000000e+02 +2.575000000000000e+02
+3.864240000000000e+01 +2.575000000000000e+02
+1.093010000000000e+03 +2.570000000000000e+02
+7.878610000000001e+02 +2.565000000000000e+02
+2.750230000000000e+02 +2.565000000000000e+02
+2.556500000000000e+01 +2.555000000000000e+02
+3.887400000000000e+01 +2.555000000000000e+02
+1.825100000000000e+02 +2.550000000000000e+02
+3.926270000000000e+02 +2.545000000000000e+02
+3.795200000000000e+02 +2.545000000000000e+02
+7.919670000000000e+02 +2.545000000000000e+02
+4.439010000000000e+02 +2.545000000000000e+02
+2.569690000000000e+02 +2.540000000000000e+02
+1.235690000000000e+02 +2.535000000000000e+02
+2.915940000000000e+02 +2.530000000000000e+02
+4.425990000000001e+00 +2.530000000000000e+02
+2.384540000000000e+02 +2.525000000000000e+02
+8.029620000000000e+02 +2.525000000000000e+02
+3.668200000000000e+02 +2.520000000000000e+02
+6.477809999999999e+02 +2.520000000000000e+02
+8.668140000000000e+01 +2.520000000000000e+02
+2.403600000000000e+01 +2.515000000000000e+02
+5.939680000000002e+02 +2.510000000000000e+02
+7.859780000000002e+02 +2.505000000000000e+02
+4.565540000000000e+00 +2.500000000000000e+02
+7.776270000000000e+02 +2.500000000000000e+02
+2.211860000000000e+02 +2.495000000000000e+02
+2.792160000000000e+01 +2.495000000000000e+02
+9.444760000000000e+02 +2.495000000000000e+02
+7.492060000000000e+02 +2.490000000000000e+02
+9.929200000000000e+00 +2.485000000000000e+02
+7.889500000000000e+01 +2.480000000000000e+02
+2.021900000000000e+00 +2.475000000000000e+02
+5.347290000000000e+02 +2.470000000000000e+02
+2.321930000000000e+02 +2.465000000000000e+02
+1.175130000000000e+02 +2.465000000000000e+02
+4.417880000000000e+02 +2.465000000000000e+02
+7.114570000000000e+02 +2.460000000000000e+02
+9.240670000000000e+02 +2.460000000000000e+02
+5.343580000000002e+02 +2.455000000000000e+02
+2.615660000000000e+02 +2.450000000000000e+02
+3.372050000000000e+02 +2.445000000000000e+02
+1.244660000000000e+03 +2.445000000000000e+02
+2.555290000000000e+02 +2.435000000000000e+02
+2.406180000000000e+02 +2.430000000000000e+02
+1.978170000000000e+01 +2.430000000000000e+02
+2.352910000000000e+02 +2.420000000000000e+02
+2.614870000000000e+01 +2.420000000000000e+02
+2.307510000000000e+02 +2.420000000000000e+02
+3.209900000000000e+02 +2.420000000000000e+02
+1.486080000000000e+01 +2.420000000000000e+02
+2.146990000000000e+02 +2.410000000000000e+02
+2.340420000000000e+02 +2.410000000000000e+02
+4.506640000000000e+02 +2.405000000000000e+02
+4.136320000000000e+02 +2.405000000000000e+02
+1.895930000000000e+01 +2.395000000000000e+02
+2.844300000000000e+01 +2.395000000000000e+02
+8.387130000000002e+02 +2.395000000000000e+02
+5.065430000000000e+01 +1.945000000000000e+02
+2.638670000000000e+01 +2.390000000000000e+02
+9.148510000000000e+01 +2.385000000000000e+02
+7.806730000000000e+02 +2.385000000000000e+02
+2.421750000000000e+01 +2.380000000000000e+02
+8.942439999999998e+02 +2.370000000000000e+02
+1.367510000000000e+03 +2.370000000000000e+02
+2.954400000000000e+02 +2.365000000000000e+02
+1.740760000000000e+02 +2.325000000000000e+02
+6.675450000000000e+02 +2.360000000000000e+02
+1.025440000000000e+03 +2.360000000000000e+02
+4.107920000000000e+02 +2.350000000000000e+02
+5.338030000000000e+01 +1.900000000000000e+02
+3.199400000000000e+00 +2.345000000000000e+02
+3.457560000000000e+02 +2.345000000000000e+02
+2.511920000000000e+02 +2.345000000000000e+02
+4.440820000000000e+02 +2.340000000000000e+02
+1.038170000000000e+03 +2.340000000000000e+02
+4.657770000000000e+02 +2.330000000000000e+02
+9.500610000000000e+01 +1.875000000000000e+02
+2.295120000000000e+02 +2.320000000000000e+02
+9.670280000000000e-01 +2.320000000000000e+02
+8.842910000000001e+02 +2.320000000000000e+02
+1.525730000000000e+01 +2.320000000000000e+02
+2.255050000000000e+02 +2.315000000000000e+02
+3.954590000000000e+02 +2.310000000000000e+02
+1.323940000000000e+02 +2.305000000000000e+02
+2.716720000000000e+02 +2.305000000000000e+02
+2.134110000000000e+02 +2.305000000000000e+02
+1.154070000000000e+01 +2.300000000000000e+02
+6.624589999999999e+02 +2.300000000000000e+02
+7.974020000000000e+01 +2.300000000000000e+02
+1.126380000000000e+02 +2.070000000000000e+02
+1.376640000000000e+01 +2.295000000000000e+02
+2.314830000000000e+02 +2.295000000000000e+02
+5.827640000000000e+02 +2.295000000000000e+02
+3.419560000000000e+02 +2.290000000000000e+02
+4.277630000000000e+02 +2.290000000000000e+02
+2.139950000000000e+02 +2.285000000000000e+02
+6.667280000000000e+00 +2.285000000000000e+02
+2.538410000000000e+02 +2.285000000000000e+02
+1.063060000000000e+02 +2.280000000000000e+02
+3.785750000000000e+02 +2.280000000000000e+02
+1.158220000000000e+01 +2.275000000000000e+02
+6.109620000000000e+02 +2.275000000000000e+02
+6.629160000000001e+00 +2.270000000000000e+02
+4.238300000000000e+02 +2.270000000000000e+02
+2.140130000000000e+02 +2.270000000000000e+02
+3.325470000000000e+02 +2.265000000000000e+02
+2.621260000000000e+02 +2.260000000000000e+02
+5.359349999999999e+02 +2.260000000000000e+02
+1.187550000000000e+01 +2.260000000000000e+02
+1.147320000000000e+01 +2.255000000000000e+02
+5.248650000000000e+02 +2.255000000000000e+02
+6.628880000000000e+02 +2.250000000000000e+02
+6.550580000000000e+02 +2.250000000000000e+02
+1.998510000000000e+01 +1.785000000000000e+02
+4.781020000000000e+02 +2.240000000000000e+02
+2.270760000000000e+02 +2.235000000000000e+02
+1.033040000000000e+02 +2.235000000000000e+02
+8.859889999999998e+02 +2.235000000000000e+02
+8.634620000000001e-01 +2.230000000000000e+02
+2.005710000000000e+02 +2.230000000000000e+02
+9.239530000000000e+00 +2.230000000000000e+02
+2.403050000000000e+01 +2.225000000000000e+02
+3.201690000000001e+02 +2.220000000000000e+02
+6.169000000000000e+02 +2.220000000000000e+02
+1.032930000000000e+02 +2.215000000000000e+02
+4.424840000000000e+02 +2.115000000000000e+02
+2.927480000000000e+02 +2.210000000000000e+02
+9.402780000000000e+02 +2.210000000000000e+02
+2.030770000000000e+02 +2.205000000000000e+02
+3.931260000000000e+02 +2.195000000000000e+02
+2.901210000000000e+02 +2.195000000000000e+02
+5.052350000000000e+00 +2.170000000000000e+02
+7.896510000000002e+02 +2.185000000000000e+02
+5.110720000000000e+02 +2.180000000000000e+02
+1.529940000000000e+02 +2.100000000000000e+02
+4.328200000000000e+02 +2.175000000000000e+02
+2.754400000000000e+02 +2.175000000000000e+02
+3.249340000000000e+00 +2.175000000000000e+02
+7.477510000000002e+02 +2.140000000000000e+02
+9.210950000000000e+02 +2.170000000000000e+02
+4.426970000000000e+02 +2.165000000000000e+02
+5.450630000000000e+01 +2.155000000000000e+02
+8.032150000000000e+00 +2.155000000000000e+02
+3.480220000000000e+02 +2.100000000000000e+02
+2.522900000000000e+02 +2.150000000000000e+02
+1.912530000000000e+02 +2.150000000000000e+02
+2.378880000000000e+01 +2.150000000000000e+02
+3.440340000000000e+02 +2.145000000000000e+02
+4.567950000000000e+02 +2.140000000000000e+02
+8.082830000000000e+02 +2.130000000000000e+02
+6.832370000000000e+02 +2.130000000000000e+02
+9.262820000000000e+02 +2.130000000000000e+02
+1.242500000000000e+02 +2.100000000000000e+02
+7.768980000000000e+02 +2.125000000000000e+02
+7.077930000000000e+02 +2.125000000000000e+02
+4.049480000000000e+02 +2.120000000000000e+02
+2.656550000000000e+02 +2.120000000000000e+02
+4.724570000000000e+02 +2.115000000000000e+02
+7.487510000000002e+02 +2.115000000000000e+02
+1.567140000000000e+02 +2.115000000000000e+02
+2.763200000000000e+02 +2.115000000000000e+02
+2.360500000000000e+00 +2.110000000000000e+02
+3.185760000000000e+02 +2.105000000000000e+02
+2.272640000000000e+02 +2.095000000000000e+02
+4.416730000000000e+02 +2.095000000000000e+02
+6.752910000000001e+02 +2.090000000000000e+02
+3.723540000000000e+02 +1.810000000000000e+02
+9.720880000000000e+02 +2.085000000000000e+02
+4.578430000000000e+02 +2.085000000000000e+02
+4.911420000000000e+02 +2.085000000000000e+02
+4.129640000000000e+02 +1.840000000000000e+02
+8.172180000000000e+01 +1.840000000000000e+02
+5.075470000000000e+02 +1.825000000000000e+02
+1.566840000000000e+01 +2.065000000000000e+02
+6.748260000000000e+02 +2.065000000000000e+02
+3.532710000000000e+02 +2.060000000000000e+02
+3.974510000000000e+02 +1.855000000000000e+02
+7.178639999999998e+02 +2.060000000000000e+02
+7.352240000000000e+01 +2.060000000000000e+02
+9.317569999999999e+02 +2.055000000000000e+02
+4.374830000000000e+02 +2.055000000000000e+02
+8.461559999999999e-01 +2.050000000000000e+02
+3.814150000000000e+02 +1.860000000000000e+02
+1.061090000000000e+03 +2.050000000000000e+02
+8.292500000000000e+02 +2.050000000000000e+02
+9.724840000000000e+02 +2.045000000000000e+02
+2.095560000000000e+02 +2.045000000000000e+02
+4.256290000000000e+02 +1.850000000000000e+02
+6.705269999999998e+02 +2.040000000000000e+02
+5.128099999999999e+02 +1.870000000000000e+02
+4.899730000000000e+02 +2.035000000000000e+02
+8.332350000000000e+02 +2.025000000000000e+02
+1.056200000000000e+01 +2.015000000000000e+02
+3.570810000000000e+02 +1.830000000000000e+02
+7.066240000000000e+00 +2.015000000000000e+02
+2.187930000000000e+02 +2.005000000000000e+02
+7.268320000000000e+02 +2.005000000000000e+02
+7.300080000000000e+02 +2.005000000000000e+02
+4.272130000000000e+02 +2.000000000000000e+02
+1.320490000000000e+03 +2.000000000000000e+02
+7.846000000000000e+02 +1.995000000000000e+02
+3.865430000000000e+00 +1.980000000000000e+02
+2.480550000000000e+02 +1.885000000000000e+02
+7.428240000000000e+02 +1.980000000000000e+02
+2.518240000000000e+02 +1.970000000000000e+02
+9.459510000000000e+02 +1.890000000000000e+02
+7.425780000000000e+02 +1.965000000000000e+02
+4.939650000000000e+02 +1.960000000000000e+02
+8.031790000000000e+02 +1.960000000000000e+02
+4.305760000000000e+02 +1.955000000000000e+02
+2.538680000000000e+02 +1.865000000000000e+02
+8.865730000000000e+02 +1.955000000000000e+02
+1.776559999999999e+02 +1.950000000000000e+02
+1.450720000000000e+03 +1.950000000000000e+02
+3.077580000000000e+01 +1.020000000000000e+02
+4.452270000000000e+02 +1.945000000000000e+02
+4.421140000000000e+02 +1.945000000000000e+02
+2.851740000000001e+02 +1.945000000000000e+02
+5.074850000000000e+00 +1.940000000000000e+02
+2.009170000000000e+02 +1.935000000000000e+02
+7.255140000000000e+01 +1.935000000000000e+02
+4.472200000000000e+02 +1.935000000000000e+02
+2.471790000000000e+02 +1.935000000000000e+02
+1.944580000000000e+00 +1.930000000000000e+02
+1.541750000000000e+02 +1.930000000000000e+02
+2.268620000000000e+02 +1.930000000000000e+02
+3.658960000000000e+02 +1.925000000000000e+02
+6.103640000000000e+02 +1.920000000000000e+02
+2.275130000000000e+02 +1.920000000000000e+02
+4.009380000000000e+00 +1.920000000000000e+02
+1.629370000000000e+03 +1.920000000000000e+02
+6.132400000000000e+00 +1.915000000000000e+02
+8.140400000000000e+02 +1.910000000000000e+02
+2.119600000000000e+02 +1.905000000000000e+02
+2.330070000000000e+02 +1.810000000000000e+02
+4.244630000000000e+02 +1.900000000000000e+02
+2.289420000000000e+02 +1.900000000000000e+02
+7.934580000000000e-01 +1.900000000000000e+02
+7.882040000000000e+02 +1.900000000000000e+02
+7.668770000000000e+02 +1.900000000000000e+02
+8.727780000000001e+00 +1.895000000000000e+02
+1.114970000000000e+00 +1.895000000000000e+02
+4.316070000000000e+02 +1.885000000000000e+02
+9.461540000000000e+02 +1.885000000000000e+02
+4.060190000000000e+02 +1.880000000000000e+02
+3.932010000000000e+02 +1.815000000000000e+02
+6.409940000000000e+02 +1.880000000000000e+02
+7.095450000000000e+02 +1.880000000000000e+02
+8.612680000000000e+02 +1.880000000000000e+02
+1.619860000000000e+02 +1.875000000000000e+02
+7.155100000000000e+02 +1.875000000000000e+02
+7.534349999999999e+02 +1.870000000000000e+02
+8.213420000000000e+02 +1.865000000000000e+02
+2.198490000000000e+02 +1.860000000000000e+02
+1.491820000000000e-01 +1.860000000000000e+02
+6.075740000000001e+00 +1.255000000000000e+02
+3.652880000000000e+02 +1.855000000000000e+02
+1.604000000000000e+02 +1.855000000000000e+02
+7.901020000000000e+02 +1.855000000000000e+02
+1.250370000000000e+03 +1.850000000000000e+02
+3.254940000000000e+01 +1.845000000000000e+02
+2.394200000000000e+02 +1.820000000000000e+02
+4.575920000000000e+02 +1.825000000000000e+02
+2.240350000000000e+02 +1.835000000000000e+02
+2.294780000000000e+02 +1.825000000000000e+02
+9.685790000000000e+02 +1.830000000000000e+02
+1.082980000000000e+03 +1.825000000000000e+02
+6.398210000000000e+02 +1.820000000000000e+02
+1.056390000000000e+00 +1.815000000000000e+02
+4.030600000000000e+02 +1.790000000000000e+02
+1.589540000000000e+02 +1.815000000000000e+02
+9.142370000000000e-01 +1.815000000000000e+02
+2.417470000000000e+02 +1.770000000000000e+02
+3.637690000000000e+02 +1.800000000000000e+02
+5.015130000000000e+02 +1.810000000000000e+02
+8.638190000000000e+02 +1.810000000000000e+02
+1.263010000000000e+03 +1.800000000000000e+02
+7.582200000000000e+02 +1.800000000000000e+02
+9.696020000000000e+02 +1.800000000000000e+02
+7.459900000000000e+02 +1.800000000000000e+02
+6.568880000000000e-01 +1.800000000000000e+02
+1.154980000000000e+03 +1.800000000000000e+02
+9.486050000000000e+02 +1.800000000000000e+02
+1.028400000000000e+01 +1.800000000000000e+02
+5.355910000000000e+02 +1.800000000000000e+02
+7.391980000000000e+02 +1.800000000000000e+02
+1.161480000000000e+03 +1.800000000000000e+02
+9.248860000000000e+02 +1.800000000000000e+02
+1.262830000000000e+02 +1.800000000000000e+02
+1.245510000000000e+03 +1.795000000000000e+02
+4.080070000000000e+02 +1.785000000000000e+02
+1.200890000000000e+03 +1.780000000000000e+02
+1.921390000000000e+02 +1.780000000000000e+02
+1.935260000000000e+02 +1.775000000000000e+02
+1.296040000000000e+02 +1.775000000000000e+02
+5.801460000000000e+01 +1.110000000000000e+02
+3.513860000000000e+02 +1.770000000000000e+02
+6.684230000000000e+02 +1.770000000000000e+02
+2.338960000000000e+02 +1.765000000000000e+02
+1.158710000000000e+03 +1.760000000000000e+02
+2.102990000000000e+02 +1.755000000000000e+02
+5.125630000000000e+02 +1.745000000000000e+02
+9.274540000000000e+02 +1.745000000000000e+02
+2.596470000000000e+01 +1.740000000000000e+02
+1.292380000000000e+02 +1.735000000000000e+02
+3.286780000000001e+02 +1.735000000000000e+02
+4.593460000000000e+02 +1.715000000000000e+02
+2.049300000000000e+02 +1.710000000000000e+02
+2.418600000000000e+01 +1.710000000000000e+02
+1.701150000000000e+02 +1.705000000000000e+02
+3.471319999999999e+02 +1.695000000000000e+02
+4.692790000000000e+02 +1.695000000000000e+02
+3.258540000000001e+02 +1.695000000000000e+02
+3.324940000000000e+02 +1.690000000000000e+02
+7.413960000000002e+02 +1.685000000000000e+02
+4.402780000000000e+02 +1.685000000000000e+02
+5.640040000000000e+00 +6.500000000000000e+00
+3.671520000000000e+02 +1.680000000000000e+02
+1.868360000000000e+02 +1.680000000000000e+02
+2.603090000000000e+00 +1.675000000000000e+02
+6.372450000000000e+02 +1.675000000000000e+02
+5.603350000000000e+02 +1.670000000000000e+02
+1.254470000000000e+03 +1.665000000000000e+02
+7.903300000000000e+02 +1.665000000000000e+02
+1.639500000000000e+02 +1.600000000000000e+02
+1.652490000000000e+01 +1.520000000000000e+02
+1.870300000000000e+02 +1.645000000000000e+02
+1.163660000000000e+02 +1.175000000000000e+02
+3.052050000000000e+02 +1.645000000000000e+02
+1.612300000000000e+00 +1.640000000000000e+02
+5.431580000000000e+02 +1.640000000000000e+02
+8.337040000000000e+02 +1.230000000000000e+02
+9.605440000000000e+01 +8.750000000000000e+01
+1.179170000000000e+03 +1.635000000000000e+02
+6.982619999999999e+02 +1.635000000000000e+02
+9.409080000000000e+01 +1.305000000000000e+02
+2.033710000000000e+02 +1.630000000000000e+02
+1.450900000000000e+02 +1.205000000000000e+02
+8.898860000000002e+02 +1.615000000000000e+02
+6.051830000000000e+02 +1.610000000000000e+02
+1.133390000000000e+02 +1.610000000000000e+02
+1.833420000000000e+01 +1.605000000000000e+02
+2.102660000000000e+02 +1.505000000000000e+02
+1.787310000000000e+02 +1.600000000000000e+02
+3.678500000000000e+02 +1.595000000000000e+02
+8.756480000000000e+02 +1.335000000000000e+02
+1.862290000000000e+02 +1.590000000000000e+02
+3.330340000000000e+02 +1.590000000000000e+02
+3.861120000000000e+01 +8.250000000000000e+01
+6.087840000000000e+02 +1.585000000000000e+02
+1.712550000000000e+02 +1.585000000000000e+02
+1.579120000000000e+02 +1.260000000000000e+02
+1.685050000000000e+02 +1.005000000000000e+02
+6.811260000000000e+01 +1.560000000000000e+02
+4.617120000000000e+02 +1.575000000000000e+02
+8.003960000000002e+02 +1.570000000000000e+02
+1.671640000000000e+02 +1.565000000000000e+02
+1.500950000000000e+00 +1.530000000000000e+02
+1.140470000000000e+02 +1.210000000000000e+02
+1.119150000000000e+03 +1.550000000000000e+02
+1.189660000000000e+02 +1.545000000000000e+02
+2.408570000000000e+02 +1.540000000000000e+02
+9.306900000000000e+01 +1.320000000000000e+02
+1.972720000000000e+02 +1.535000000000000e+02
+1.082300000000000e+02 +1.535000000000000e+02
+7.647950000000000e+02 +1.345000000000000e+02
+8.066070000000000e+02 +1.525000000000000e+02
+1.631330000000000e+02 +1.525000000000000e+02
+1.629340000000000e+01 +8.450000000000000e+01
+3.195250000000000e+02 +1.520000000000000e+02
+6.942410000000000e+01 +1.180000000000000e+02
+1.793720000000000e+02 +1.515000000000000e+02
+2.390460000000000e+02 +9.850000000000000e+01
+2.618260000000000e+02 +1.515000000000000e+02
+2.801250000000000e+02 +1.515000000000000e+02
+6.523450000000000e+01 +1.515000000000000e+02
+1.468790000000000e+02 +1.510000000000000e+02
+4.007440000000000e+02 +1.510000000000000e+02
+1.112740000000000e+02 +1.510000000000000e+02
+2.130900000000000e+02 +1.505000000000000e+02
+2.944330000000000e+02 +1.500000000000000e+02
+5.646580000000000e+02 +1.500000000000000e+02
+6.809390000000000e+02 +1.490000000000000e+02
+1.729550000000000e+02 +1.480000000000000e+02
+1.672350000000000e+02 +1.475000000000000e+02
+1.371270000000000e+02 +1.470000000000000e+02
+1.961620000000000e+02 +1.470000000000000e+02
+5.585160000000000e+02 +1.470000000000000e+02
+9.923050000000001e+01 +1.465000000000000e+02
+3.258180000000000e+02 +1.460000000000000e+02
+5.085400000000000e+01 +7.350000000000000e+01
+8.110730000000000e+02 +1.450000000000000e+02
+5.887200000000000e+02 +1.445000000000000e+02
+8.245139999999999e+02 +1.445000000000000e+02
+2.368580000000000e+02 +1.440000000000000e+02
+2.215060000000000e+00 +1.800000000000000e+01
+7.120360000000002e+02 +1.010000000000000e+02
+8.012869999999998e+02 +1.435000000000000e+02
+9.762500000000000e+01 +1.425000000000000e+02
+1.148610000000000e+02 +8.850000000000000e+01
+1.876710000000000e+02 +1.420000000000000e+02
+8.263920000000000e+01 +1.325000000000000e+02
+1.792340000000000e+02 +1.415000000000000e+02
+6.371290000000000e+02 +1.410000000000000e+02
+8.957769999999998e+02 +1.315000000000000e+02
+2.031250000000000e+02 +9.400000000000000e+01
+5.412280000000002e+02 +1.405000000000000e+02
+3.472510000000000e+02 +1.405000000000000e+02
+1.668410000000000e+02 +1.400000000000000e+02
+8.213470000000000e+02 +1.400000000000000e+02
+1.609550000000000e+02 +1.385000000000000e+02
+1.090690000000000e+02 +1.300000000000000e+02
+7.021289999999998e+02 +1.380000000000000e+02
+1.149030000000000e+03 +1.380000000000000e+02
+8.863860000000002e+02 +1.380000000000000e+02
+8.946650000000000e+01 +1.375000000000000e+02
+7.957230000000002e+02 +1.375000000000000e+02
+6.406800000000000e+00 +1.370000000000000e+02
+6.000910000000000e+02 +1.370000000000000e+02
+8.256880000000000e+02 +1.365000000000000e+02
+1.715180000000000e+02 +1.365000000000000e+02
+5.673870000000000e+00 +6.000000000000000e+00
+6.800960000000000e+02 +1.360000000000000e+02
+1.099520000000000e+01 +5.500000000000000e+00
+1.301630000000000e+02 +1.350000000000000e+02
+1.973540000000000e+02 +1.345000000000000e+02
+1.542180000000000e+02 +1.345000000000000e+02
+9.852910000000001e+02 +1.260000000000000e+02
+4.411010000000000e+01 +9.000000000000000e+01
+6.304400000000001e+02 +1.340000000000000e+02
+1.910340000000000e+02 +1.335000000000000e+02
+6.748980000000000e+01 +1.335000000000000e+02
+3.940520000000000e+02 +1.330000000000000e+02
+1.501160000000000e+02 +1.320000000000000e+02
+4.338470000000000e+01 +1.100000000000000e+02
+1.474470000000000e+02 +1.320000000000000e+02
+9.025920000000001e+01 +1.315000000000000e+02
+8.403100000000001e+01 +1.315000000000000e+02
+3.492300000000000e+00 +1.310000000000000e+02
+1.193920000000000e+01 +1.310000000000000e+02
+4.617080000000000e+02 +1.310000000000000e+02
+8.094789999999998e+02 +1.305000000000000e+02
+4.609610000000000e+02 +1.305000000000000e+02
+1.441070000000000e+02 +1.300000000000000e+02
+7.716450000000000e+01 +1.300000000000000e+02
+5.092550000000000e+02 +1.300000000000000e+02
+2.190580000000000e+02 +1.300000000000000e+02
+5.670000000000000e+01 +9.200000000000000e+01
+2.906680000000000e+02 +1.295000000000000e+02
+4.621150000000000e+00 +6.750000000000000e+01
+3.213710000000000e+01 +1.290000000000000e+02
+8.127400000000000e+01 +1.290000000000000e+02
+4.915660000000000e+02 +1.285000000000000e+02
+3.440340000000000e-01 +1.275000000000000e+02
+4.400990000000000e+02 +1.275000000000000e+02
+5.694040000000000e+01 +9.750000000000000e+01
+4.917340000000000e+02 +1.265000000000000e+02
+7.714700000000000e+02 +1.265000000000000e+02
+2.361090000000000e+00 +5.600000000000000e+01
+7.632210000000000e+02 +1.260000000000000e+02
+4.984370000000000e+02 +1.260000000000000e+02
+4.362400000000000e+02 +1.260000000000000e+02
+1.279660000000000e+02 +9.450000000000000e+01
+4.896850000000000e+02 +1.255000000000000e+02
+6.331850000000000e+01 +8.450000000000000e+01
+2.466880000000000e+01 +1.250000000000000e+02
+1.878290000000000e+02 +9.250000000000000e+01
+2.174600000000000e+02 +1.245000000000000e+02
+1.270430000000000e+02 +1.245000000000000e+02
+3.744950000000000e+02 +1.105000000000000e+02
+7.589560000000000e+02 +1.240000000000000e+02
+8.517340000000000e+01 +1.240000000000000e+02
+1.206350000000000e+02 +1.235000000000000e+02
+4.847480000000001e+02 +1.235000000000000e+02
+2.245070000000000e+01 +1.235000000000000e+02
+1.171640000000000e+03 +1.235000000000000e+02
+1.120940000000000e+03 +1.230000000000000e+02
+1.054190000000000e+02 +1.225000000000000e+02
+1.142690000000000e+03 +1.225000000000000e+02
+1.827120000000000e+01 +1.220000000000000e+02
+3.381869999999999e+02 +1.220000000000000e+02
+7.182140000000000e+01 +1.215000000000000e+02
+4.346340000000000e+02 +1.215000000000000e+02
+6.403810000000000e+01 +7.350000000000000e+01
+6.885260000000002e+02 +1.210000000000000e+02
+6.336559999999999e+02 +1.205000000000000e+02
+3.848240000000000e+01 +7.350000000000000e+01
+1.312620000000000e+01 +1.195000000000000e+02
+6.920340000000000e+01 +1.195000000000000e+02
+4.656430000000000e+02 +1.190000000000000e+02
+2.172070000000000e+02 +1.190000000000000e+02
+6.621530000000000e+02 +1.185000000000000e+02
+4.664290000000000e+02 +1.185000000000000e+02
+9.560010000000000e+01 +1.185000000000000e+02
+4.265850000000000e+01 +1.185000000000000e+02
+5.095990000000000e+02 +1.185000000000000e+02
+6.143250000000000e+02 +1.180000000000000e+02
+5.289320000000000e+02 +1.180000000000000e+02
+6.073240000000002e+02 +1.180000000000000e+02
+7.250590000000000e+01 +8.650000000000000e+01
+4.558580000000000e+02 +1.175000000000000e+02
+1.102120000000000e+03 +1.175000000000000e+02
+8.878810000000002e+02 +1.175000000000000e+02
+8.281480000000001e+01 +8.600000000000000e+01
+5.357850000000000e+02 +1.165000000000000e+02
+4.718820000000000e+01 +9.300000000000000e+01
+8.347380000000000e+01 +1.160000000000000e+02
+7.737030000000000e+02 +1.155000000000000e+02
+5.978190000000000e+01 +1.155000000000000e+02
+4.631470000000000e+02 +1.150000000000000e+02
+7.883220000000000e+01 +1.145000000000000e+02
+6.413470000000000e+02 +1.145000000000000e+02
+2.664310000000000e+01 +6.750000000000000e+01
+4.595050000000000e+02 +1.140000000000000e+02
+1.992530000000000e+01 +1.140000000000000e+02
+1.151150000000000e+02 +1.010000000000000e+02
+4.399230000000000e+02 +1.135000000000000e+02
+3.789140000000000e+02 +1.135000000000000e+02
+3.999920000000000e+01 +1.130000000000000e+02
+4.393540000000000e+02 +1.130000000000000e+02
+7.207910000000001e+02 +1.130000000000000e+02
+2.321870000000000e+02 +1.130000000000000e+02
+9.635240000000000e+01 +9.100000000000000e+01
+7.311870000000000e+01 +1.125000000000000e+02
+1.826420000000000e+02 +1.120000000000000e+02
+4.068300000000000e+02 +1.120000000000000e+02
+1.627580000000000e+02 +1.120000000000000e+02
+7.596120000000001e+01 +9.900000000000000e+01
+4.779340000000000e+02 +1.115000000000000e+02
+4.422850000000000e+02 +1.115000000000000e+02
+6.254270000000000e+02 +1.115000000000000e+02
+4.205700000000000e+02 +1.110000000000000e+02
+1.498230000000000e+02 +1.105000000000000e+02
+4.285520000000000e+02 +1.105000000000000e+02
+5.426369999999999e+02 +1.105000000000000e+02
+1.155380000000000e+00 +1.085000000000000e+02
+5.680330000000000e+01 +1.100000000000000e+02
+4.328290000000000e+02 +1.100000000000000e+02
+3.031070000000000e+02 +1.100000000000000e+02
+1.326520000000000e+02 +1.090000000000000e+02
+6.366310000000000e+02 +1.090000000000000e+02
+1.801170000000000e+02 +1.085000000000000e+02
+4.577820000000000e+02 +1.085000000000000e+02
+1.120080000000000e+01 +9.000000000000000e+01
+4.135260000000000e+01 +1.085000000000000e+02
+3.174890000000000e+01 +1.080000000000000e+02
+2.568060000000000e+02 +1.080000000000000e+02
+3.702010000000000e+01 +9.600000000000000e+01
+3.293530000000000e+02 +1.075000000000000e+02
+4.338340000000000e+02 +1.075000000000000e+02
+6.696650000000000e+02 +1.075000000000000e+02
+1.795650000000000e+01 +6.300000000000000e+01
+4.627280000000000e+02 +1.070000000000000e+02
+3.607190000000000e+02 +1.070000000000000e+02
+1.882210000000000e+02 +6.950000000000000e+01
+6.226600000000000e+02 +1.060000000000000e+02
+4.349070000000000e+02 +1.055000000000000e+02
+4.066470000000000e+00 +1.055000000000000e+02
+4.121090000000000e+01 +6.050000000000000e+01
+4.328390000000000e+02 +1.050000000000000e+02
+1.939510000000000e+01 +1.050000000000000e+02
+4.523950000000000e+02 +1.045000000000000e+02
+1.952110000000000e+02 +7.400000000000000e+01
+3.502870000000000e+02 +1.035000000000000e+02
+4.110740000000000e+02 +1.035000000000000e+02
+7.786810000000000e+02 +1.025000000000000e+02
+7.723430000000000e+01 +9.200000000000000e+01
+4.356040000000000e+02 +1.020000000000000e+02
+6.253960000000000e+02 +1.020000000000000e+02
+3.137420000000000e+02 +1.015000000000000e+02
+4.427580000000000e+02 +1.015000000000000e+02
+4.137220000000000e+02 +1.010000000000000e+02
+1.229970000000000e+02 +5.850000000000000e+01
+6.204720000000000e+02 +1.010000000000000e+02
+3.569250000000000e+01 +8.300000000000000e+01
+5.663650000000000e+02 +1.005000000000000e+02
+5.884950000000000e+02 +1.000000000000000e+02
+4.013890000000000e+02 +1.000000000000000e+02
+5.408270000000000e+02 +1.000000000000000e+02
+1.391680000000000e+02 +9.000000000000000e+01
+7.398730000000000e+02 +9.950000000000000e+01
+6.231220000000000e+02 +9.950000000000000e+01
+5.264780000000000e+00 +9.950000000000000e+01
+3.153850000000000e+02 +9.900000000000000e+01
+7.483989999999999e+02 +9.850000000000000e+01
+4.357240000000000e+02 +9.850000000000000e+01
+7.410069999999999e+02 +9.850000000000000e+01
+4.601070000000000e+02 +9.850000000000000e+01
+8.432180000000000e+01 +9.800000000000000e+01
+7.237669999999998e+02 +9.800000000000000e+01
+1.831840000000000e+02 +9.800000000000000e+01
+3.592120000000000e+02 +9.800000000000000e+01
+1.523400000000000e+02 +5.950000000000000e+01
+7.493360000000000e+01 +9.750000000000000e+01
+4.738200000000000e+02 +9.750000000000000e+01
+4.394450000000000e+02 +9.700000000000000e+01
+5.872430000000001e+02 +9.650000000000000e+01
+5.580080000000000e+02 +9.650000000000000e+01
+1.057510000000000e+02 +8.100000000000000e+01
+3.838490000000000e+01 +9.600000000000000e+01
+4.217870000000000e+02 +9.600000000000000e+01
+5.296830000000000e+02 +9.600000000000000e+01
+9.197190000000001e+02 +9.600000000000000e+01
+2.849620000000000e+02 +9.600000000000000e+01
+9.083910000000000e+01 +9.600000000000000e+01
+3.509700000000000e+01 +6.950000000000000e+01
+2.125900000000000e+02 +8.800000000000000e+01
+4.202920000000000e+02 +9.500000000000000e+01
+5.283030000000000e+02 +9.500000000000000e+01
+6.513890000000000e+00 +9.500000000000000e+01
+5.394240000000000e+00 +5.450000000000000e+01
+5.619460000000000e+02 +9.450000000000000e+01
+6.928260000000000e+01 +9.450000000000000e+01
+5.944770000000000e+02 +9.450000000000000e+01
+1.156160000000000e+02 +9.450000000000000e+01
+5.898180000000000e+01 +6.800000000000000e+01
+8.790219999999998e+02 +9.400000000000000e+01
+6.386600000000001e+01 +9.400000000000000e+01
+2.432660000000000e+02 +9.400000000000000e+01
+2.829140000000000e+02 +9.350000000000000e+01
+2.010460000000000e+02 +7.550000000000000e+01
+9.856479999999999e+00 +9.350000000000000e+01
+1.706920000000000e+02 +9.350000000000000e+01
+5.240900000000000e+02 +9.300000000000000e+01
+2.373120000000000e+01 +9.250000000000000e+01
+3.415719999999999e+02 +8.680000000000000e+02
+7.039140000000000e+02 +9.250000000000000e+01
+2.751380000000000e+02 +9.200000000000000e+01
+3.624250000000000e+01 +9.200000000000000e+01
+6.281700000000000e+02 +9.200000000000000e+01
+1.138990000000000e+02 +9.150000000000000e+01
+4.196700000000000e+02 +9.150000000000000e+01
+1.630350000000000e+02 +5.750000000000000e+01
+4.776970000000000e+02 +9.100000000000000e+01
+5.467490000000000e+02 +9.100000000000000e+01
+8.575150000000000e+02 +9.050000000000000e+01
+4.137080000000000e+02 +9.050000000000000e+01
+2.052510000000000e+01 +9.050000000000000e+01
+1.010030000000000e+02 +9.050000000000000e+01
+4.018310000000000e+02 +9.000000000000000e+01
+4.622510000000000e+02 +9.000000000000000e+01
+5.752160000000000e+02 +9.000000000000000e+01
+1.165240000000000e+02 +8.950000000000000e+01
+1.002400000000000e+02 +5.700000000000000e+01
+2.685070000000000e+01 +5.100000000000000e+01
+1.364490000000000e+02 +5.600000000000000e+01
+4.273450000000000e+02 +8.900000000000000e+01
+1.904080000000000e+01 +4.850000000000000e+01
+7.244290000000000e+02 +8.850000000000000e+01
+4.435900000000000e+02 +8.800000000000000e+01
+9.674450000000000e+01 +5.450000000000000e+01
+3.739770000000000e+02 +8.800000000000000e+01
+8.969360000000000e+01 +8.750000000000000e+01
+5.623440000000001e+02 +8.700000000000000e+01
+3.968960000000000e+02 +8.700000000000000e+01
+6.522810000000000e+00 +8.700000000000000e+01
+4.261230000000001e+02 +8.650000000000000e+01
+3.940350000000000e+02 +8.600000000000000e+01
+4.761710000000000e-01 +8.600000000000000e+01
+1.257160000000000e+02 +8.550000000000000e+01
+1.413820000000000e+02 +5.350000000000000e+01
+1.064520000000000e+02 +8.500000000000000e+01
+2.872290000000001e+02 +8.500000000000000e+01
+2.181800000000000e+01 +4.700000000000000e+01
+6.651000000000000e+02 +8.500000000000000e+01
+1.111390000000000e+02 +8.450000000000000e+01
+7.693300000000001e+01 +5.300000000000000e+01
+3.049470000000000e+02 +8.400000000000000e+01
+2.747060000000000e+02 +8.400000000000000e+01
+2.595270000000000e+01 +5.850000000000000e+01
+4.247720000000000e+02 +8.300000000000000e+01
+5.392110000000000e+02 +8.250000000000000e+01
+6.428770000000000e+02 +8.250000000000000e+01
+1.112600000000000e+03 +8.200000000000000e+01
+2.714890000000000e+01 +8.200000000000000e+01
+1.941680000000000e-02 +8.200000000000000e+01
+2.248490000000000e+02 +8.200000000000000e+01
+2.062050000000000e+01 +7.300000000000000e+01
+3.971940000000000e+02 +8.100000000000000e+01
+1.213240000000000e+02 +6.150000000000000e+01
+3.775990000000000e+02 +8.100000000000000e+01
+7.213970000000000e+01 +8.050000000000000e+01
+6.146770000000000e+02 +8.050000000000000e+01
+4.142220000000000e+02 +8.050000000000000e+01
+4.203710000000000e+02 +8.000000000000000e+01
+4.946020000000000e+01 +6.300000000000000e+01
+1.125310000000000e+02 +7.900000000000000e+01
+3.290380000000000e+02 +7.900000000000000e+01
+7.223530000000002e+02 +7.850000000000000e+01
+5.943800000000000e+02 +7.850000000000000e+01
+3.524469999999999e+02 +7.800000000000000e+01
+2.596380000000000e+01 +4.700000000000000e+01
+5.954730000000002e+02 +7.800000000000000e+01
+3.779660000000000e+00 +5.450000000000000e+01
+4.120160000000000e+02 +7.750000000000000e+01
+1.414940000000000e+02 +7.100000000000000e+01
+7.724600000000000e+02 +7.750000000000000e+01
+1.148350000000000e+02 +4.850000000000000e+01
+2.766880000000000e+01 +7.700000000000000e+01
+7.157520000000000e+02 +7.700000000000000e+01
+5.661659999999998e+02 +7.650000000000000e+01
+1.130270000000000e+00 +7.650000000000000e+01
+1.180560000000000e+02 +4.650000000000000e+01
+1.225830000000000e+01 +7.600000000000000e+01
+4.357800000000000e+02 +7.600000000000000e+01
+2.518330000000000e+02 +7.600000000000000e+01
+3.245500000000000e+02 +7.600000000000000e+01
+5.848840000000000e+02 +7.550000000000000e+01
+2.765920000000000e+02 +7.550000000000000e+01
+1.549680000000000e+02 +6.150000000000000e+01
+8.730490000000000e+02 +7.550000000000000e+01
+4.188200000000000e+02 +7.500000000000000e+01
+4.922050000000000e+01 +6.750000000000000e+01
+4.197470000000000e+02 +7.450000000000000e+01
+9.499290000000000e+00 +7.450000000000000e+01
+6.442480000000000e+02 +7.450000000000000e+01
+3.703000000000000e+02 +7.450000000000000e+01
+3.993480000000000e+02 +7.450000000000000e+01
+1.808930000000000e+02 +6.650000000000000e+01
+8.981560000000002e+02 +7.400000000000000e+01
+3.208030000000000e+02 +7.350000000000000e+01
+1.405230000000000e+00 +5.650000000000000e+01
+3.659610000000000e+02 +7.300000000000000e+01
+4.253110000000000e+02 +7.250000000000000e+01
+3.299110000000000e+02 +7.200000000000000e+01
+1.740040000000000e+02 +6.450000000000000e+01
+5.423680000000001e+02 +7.200000000000000e+01
+1.885270000000000e+01 +7.000000000000000e+00
+4.003800000000000e+02 +7.100000000000000e+01
+6.871310000000002e+02 +7.100000000000000e+01
+8.269570000000000e+02 +7.000000000000000e+01
+2.981560000000000e+02 +7.000000000000000e+01
+2.986460000000000e+02 +1.950000000000000e+01
+4.144340000000000e+02 +6.950000000000000e+01
+3.604090000000000e+02 +6.950000000000000e+01
+8.351500000000000e+02 +6.950000000000000e+01
+1.077170000000000e+02 +6.150000000000000e+01
+1.266580000000000e+02 +6.900000000000000e+01
+6.663589999999998e+02 +6.900000000000000e+01
+5.774650000000000e+01 +6.900000000000000e+01
+1.059450000000000e+02 +6.850000000000000e+01
+4.761190000000000e+02 +6.850000000000000e+01
+3.669970000000000e+01 +1.650000000000000e+01
+1.798740000000001e+02 +6.850000000000000e+01
+3.886120000000000e+02 +6.800000000000000e+01
+1.587510000000000e+02 +6.800000000000000e+01
+5.603800000000000e-01 +6.750000000000000e+01
+5.299300000000000e+00 +6.750000000000000e+01
+8.223830000000000e+02 +6.750000000000000e+01
+2.909480000000000e+02 +6.700000000000000e+01
+1.024840000000000e+00 +6.050000000000000e+01
+8.016870000000000e+02 +6.700000000000000e+01
+7.016580000000000e+02 +6.650000000000000e+01
+4.353190000000000e+02 +6.650000000000000e+01
+1.529620000000000e+02 +6.650000000000000e+01
+4.821530000000000e+02 +6.600000000000000e+01
+4.279060000000000e+02 +6.600000000000000e+01
+1.467300000000000e+01 +5.750000000000000e+01
+3.732300000000000e+02 +6.550000000000000e+01
+1.264990000000000e+02 +4.950000000000000e+01
+3.977930000000000e+02 +6.550000000000000e+01
+7.442850000000000e+02 +6.500000000000000e+01
+4.462720000000000e-01 +6.500000000000000e+01
+3.369000000000000e+01 +6.500000000000000e+01
+6.076220000000000e+02 +6.500000000000000e+01
+3.854590000000000e+02 +6.500000000000000e+01
+5.225570000000000e+02 +6.450000000000000e+01
+3.447360000000000e+02 +6.450000000000000e+01
+2.844020000000000e+02 +6.400000000000000e+01
+1.342240000000000e+01 +6.400000000000000e+01
+7.115369999999998e+02 +6.350000000000000e+01
+3.136690000000001e+02 +6.350000000000000e+01
+6.275200000000000e+02 +6.350000000000000e+01
+4.094200000000000e+00 +8.000000000000000e+00
+2.703030000000000e+02 +6.300000000000000e+01
+1.201570000000000e+02 +6.300000000000000e+01
+9.948699999999999e+01 +6.250000000000000e+01
+5.370010000000000e+02 +6.250000000000000e+01
+3.787280000000000e+02 +6.250000000000000e+01
+9.463190000000000e+01 +6.150000000000000e+01
+8.031680000000000e-01 +6.150000000000000e+01
+1.266420000000000e+02 +6.150000000000000e+01
+5.909600000000000e+02 +6.150000000000000e+01
+7.224140000000000e+02 +6.100000000000000e+01
+2.381030000000000e+02 +6.100000000000000e+01
+7.556180000000000e+00 +5.900000000000000e+01
+1.179010000000000e+00 +4.500000000000000e+01
+1.258710000000000e+02 +6.050000000000000e+01
+3.868640000000000e+02 +6.050000000000000e+01
+6.989440000000000e+02 +6.050000000000000e+01
+2.578260000000000e+02 +6.050000000000000e+01
+1.006420000000000e+02 +6.000000000000000e+01
+7.051260000000002e+02 +6.000000000000000e+01
+1.536300000000000e+02 +5.550000000000000e+01
+5.011560000000000e+02 +6.000000000000000e+01
+4.763490000000000e+02 +6.000000000000000e+01
+4.040420000000000e+02 +6.000000000000000e+01
+3.593980000000000e+02 +6.000000000000000e+01
+2.081790000000000e+02 +6.000000000000000e+01
+3.990120000000000e+00 +1.150000000000000e+01
+6.911980000000000e+02 +5.950000000000000e+01
+2.613880000000000e+02 +5.950000000000000e+01
+5.743480000000002e+02 +5.950000000000000e+01
+4.073880000000000e+02 +5.950000000000000e+01
+2.367210000000000e+02 +5.950000000000000e+01
+7.231950000000001e+02 +5.900000000000000e+01
+7.062869999999999e+01 +1.750000000000000e+01
+5.875040000000000e+01 +5.900000000000000e+01
+2.264630000000000e+02 +5.850000000000000e+01
+1.297700000000000e+02 +5.450000000000000e+01
+2.425720000000000e+02 +5.850000000000000e+01
+6.956750000000000e+02 +5.750000000000000e+01
+2.424870000000000e+02 +5.750000000000000e+01
+4.056000000000000e+02 +5.750000000000000e+01
+2.120400000000000e+01 +5.700000000000000e+01
+3.600780000000000e+00 +4.850000000000000e+01
+4.014510000000000e+02 +5.650000000000000e+01
+3.690230000000000e+02 +5.650000000000000e+01
+1.790280000000000e+02 +5.650000000000000e+01
+1.612500000000000e+02 +5.600000000000000e+01
+5.995470000000000e+02 +5.550000000000000e+01
+1.486710000000000e+02 +5.250000000000000e+01
+3.684590000000000e+02 +5.550000000000000e+01
+5.437370000000000e+01 +1.250000000000000e+01
+8.688799999999998e+01 +5.500000000000000e+01
+2.044490000000000e+01 +4.700000000000000e+01
+5.438380000000002e+02 +5.500000000000000e+01
+3.221180000000000e+02 +5.500000000000000e+01
+6.167480000000000e+02 +5.500000000000000e+01
+1.063460000000000e+01 +4.500000000000000e+01
+3.858000000000000e+02 +5.450000000000000e+01
+1.219810000000000e+02 +5.000000000000000e+01
+2.772370000000000e+01 +5.400000000000000e+01
+6.514720000000000e+02 +5.350000000000000e+01
+2.429960000000000e+02 +5.350000000000000e+01
+6.246420000000000e+00 +3.350000000000000e+01
+5.317490000000000e+02 +5.300000000000000e+01
+4.010040000000000e+02 +5.300000000000000e+01
+2.347100000000000e+02 +5.250000000000000e+01
+2.669950000000000e+02 +5.250000000000000e+01
+5.234450000000001e+02 +5.250000000000000e+01
+3.841580000000000e+02 +5.200000000000000e+01
+4.236710000000000e+02 +5.200000000000000e+01
+6.184460000000000e+02 +5.200000000000000e+01
+8.786399999999998e+01 +5.150000000000000e+01
+6.354109999999999e+02 +5.150000000000000e+01
+2.418760000000000e+02 +5.150000000000000e+01
+3.818800000000000e+02 +5.150000000000000e+01
+2.936290000000000e+02 +5.150000000000000e+01
+2.598250000000000e+02 +5.100000000000000e+01
+1.075010000000000e+02 +4.650000000000000e+01
+5.872340000000000e+02 +5.100000000000000e+01
+2.121450000000000e+02 +5.050000000000000e+01
+2.328620000000000e+02 +5.050000000000000e+01
+3.237240000000000e+02 +5.050000000000000e+01
+4.813380000000000e+02 +5.050000000000000e+01
+3.324610000000000e+02 +5.050000000000000e+01
+7.027450000000000e+01 +5.000000000000000e+01
+8.869799999999999e+01 +4.950000000000000e+01
+2.305580000000000e+02 +4.950000000000000e+01
+2.641090000000000e+00 +4.500000000000000e+00
+6.487950000000000e+02 +4.900000000000000e+01
+1.188860000000000e+02 +4.900000000000000e+01
+8.000850000000000e+01 +4.900000000000000e+01
+4.926110000000000e+02 +4.850000000000000e+01
+3.976070000000000e+02 +4.850000000000000e+01
+2.393200000000000e+01 +4.500000000000000e+00
+5.488470000000000e+01 +4.850000000000000e+01
+5.827380000000001e+02 +4.800000000000000e+01
+1.824710000000000e+02 +4.800000000000000e+01
+3.519280000000001e+02 +4.750000000000000e+01
+4.697630000000000e+02 +4.750000000000000e+01
+3.691800000000000e+01 +4.750000000000000e+01
+2.535420000000000e+02 +4.750000000000000e+01
+3.222990000000001e+02 +4.700000000000000e+01
+1.226620000000000e+02 +4.600000000000000e+01
+2.483640000000000e+01 +4.500000000000000e+01
+1.086220000000000e+02 +4.300000000000000e+01
+3.751290000000000e+02 +4.650000000000000e+01
+2.303440000000000e+02 +4.600000000000000e+01
+1.882150000000000e+02 +4.550000000000000e+01
+3.153500000000000e+02 +4.550000000000000e+01
+8.098720000000000e+01 +4.500000000000000e+01
+6.461510000000000e+01 +4.500000000000000e+01
+3.405000000000000e+02 +4.400000000000000e+01
+2.228250000000000e+02 +4.350000000000000e+01
+4.708140000000000e+02 +4.350000000000000e+01
+2.188260000000000e+02 +4.300000000000000e+01
+6.149330000000000e+01 +4.250000000000000e+01
+2.261420000000000e+02 +4.300000000000000e+01
+7.569810000000000e+01 +4.250000000000000e+01
+9.631870000000001e+01 +4.150000000000000e+01
+5.724720000000000e+02 +4.250000000000000e+01
+2.115610000000000e+02 +4.250000000000000e+01
+3.765900000000000e+01 +3.850000000000000e+01
+4.018780000000000e+02 +4.200000000000000e+01
+7.664480000000000e+00 +2.250000000000000e+01
+4.420570000000000e+02 +4.150000000000000e+01
+1.352200000000000e+00 +1.150000000000000e+01
+3.676800000000000e+02 +4.100000000000000e+01
+2.267530000000000e+01 +9.500000000000000e+00
+1.453820000000000e+02 +4.050000000000000e+01
+2.633210000000000e+02 +4.050000000000000e+01
+2.843950000000000e+02 +3.950000000000000e+01
+2.336900000000000e+02 +3.950000000000000e+01
+3.292980000000000e+02 +3.900000000000000e+01
+2.507630000000000e+02 +3.900000000000000e+01
+8.671570000000000e+01 +3.750000000000000e+01
+2.378510000000000e+02 +3.850000000000000e+01
+9.106670000000000e+01 +3.700000000000000e+01
+2.902530000000000e+02 +3.750000000000000e+01
+3.143940000000000e+02 +3.700000000000000e+01
+4.339780000000000e+02 +3.700000000000000e+01
+1.369030000000000e+02 +3.650000000000000e+01
+9.613310000000000e+01 +3.650000000000000e+01
+2.038470000000000e+02 +3.650000000000000e+01
+3.626900000000000e+02 +3.650000000000000e+01
+7.269050000000000e+01 +3.600000000000000e+01
+5.285050000000000e+02 +3.600000000000000e+01
+1.024570000000000e+02 +3.600000000000000e+01
+1.936150000000000e+02 +3.600000000000000e+01
+3.668730000000001e+02 +3.600000000000000e+01
+6.703570000000001e+01 +3.550000000000000e+01
+9.595530000000000e+00 +1.000000000000000e+01
+2.616100000000000e+02 +3.550000000000000e+01
+1.731500000000000e+02 +3.550000000000000e+01
+1.023420000000000e+02 +3.500000000000000e+01
+4.223830000000000e+02 +3.500000000000000e+01
+1.152360000000000e+01 +3.450000000000000e+01
+8.230549999999999e+00 +4.000000000000000e+00
+1.411380000000000e+02 +3.450000000000000e+01
+1.916220000000000e+02 +3.400000000000000e+01
+2.692610000000000e+02 +3.400000000000000e+01
+1.196780000000000e+02 +3.350000000000000e+01
+3.562790000000000e+02 +3.350000000000000e+01
+1.946400000000000e+02 +3.350000000000000e+01
+1.035420000000000e+02 +3.350000000000000e+01
+6.545460000000000e+00 +5.500000000000000e+00
+4.490750000000000e+02 +3.300000000000000e+01
+6.034870000000000e+00 +2.200000000000000e+01
+2.475200000000000e+02 +3.300000000000000e+01
+2.430310000000000e+02 +3.300000000000000e+01
+6.359580000000000e+00 +3.000000000000000e+00
+1.800740000000000e+02 +3.250000000000000e+01
+3.417940000000001e+02 +3.250000000000000e+01
+1.158350000000000e+02 +3.200000000000000e+01
+8.516450000000000e+01 +3.200000000000000e+01
+2.640840000000000e+02 +3.200000000000000e+01
+3.774280000000001e+02 +3.200000000000000e+01
+3.750860000000000e+02 +3.150000000000000e+01
+8.922250000000000e-01 +2.000000000000000e+00
+2.991950000000000e+01 +3.100000000000000e+01
+2.523480000000000e+02 +3.100000000000000e+01
+2.111350000000000e+02 +3.100000000000000e+01
+3.978660000000000e+00 +3.100000000000000e+01
+1.913380000000000e+02 +3.050000000000000e+01
+1.074550000000000e+02 +3.050000000000000e+01
+1.192380000000000e+02 +3.050000000000000e+01
+1.681120000000000e+02 +3.050000000000000e+01
+2.373340000000000e+02 +3.000000000000000e+01
+1.579440000000000e+02 +1.350000000000000e+01
+2.264740000000000e+02 +2.950000000000000e+01
+3.154800000000000e+02 +2.950000000000000e+01
+1.719430000000000e+01 +2.900000000000000e+01
+3.096590000000000e+01 +2.850000000000000e+01
+3.458360000000000e+02 +2.850000000000000e+01
+2.014100000000000e+02 +2.800000000000000e+01
+7.627090000000000e+01 +2.800000000000000e+01
+5.962080000000000e+01 +2.750000000000000e+01
+4.036100000000000e+02 +2.750000000000000e+01
+7.901260000000001e+01 +2.750000000000000e+01
+3.387569999999999e+02 +2.750000000000000e+01
+3.333850000000000e+01 +2.750000000000000e+01
+1.255960000000000e+02 +2.700000000000000e+01
+2.850270000000000e+02 +2.700000000000000e+01
+1.605700000000000e+02 +2.650000000000000e+01
+2.094100000000000e+02 +2.600000000000000e+01
+3.202870000000001e+02 +2.600000000000000e+01
+2.176250000000000e+02 +2.550000000000000e+01
+3.963080000000000e+01 +9.500000000000000e+00
+2.142030000000000e+02 +2.500000000000000e+01
+2.564970000000000e+02 +2.500000000000000e+01
+1.043340000000000e+02 +2.450000000000000e+01
+6.539630000000000e+01 +2.400000000000000e+01
+3.210530000000000e+02 +2.400000000000000e+01
+1.540690000000000e+02 +2.400000000000000e+01
+2.450250000000000e+01 +2.350000000000000e+01
+2.930170000000000e+00 +2.300000000000000e+01
+2.098080000000000e+02 +2.300000000000000e+01
+2.225990000000000e+02 +2.300000000000000e+01
+3.084060000000000e+02 +2.250000000000000e+01
+1.684160000000000e+02 +2.250000000000000e+01
+3.013230000000000e+00 +2.250000000000000e+01
+1.804090000000000e+02 +2.250000000000000e+01
+1.912900000000000e+02 +2.250000000000000e+01
+6.327500000000000e+00 +7.000000000000000e+00
+5.144750000000000e-01 +2.200000000000000e+01
+4.398600000000000e+01 +2.150000000000000e+01
+2.027290000000000e+02 +2.150000000000000e+01
+5.197480000000000e+01 +2.100000000000000e+01
+8.888790000000000e+01 +2.100000000000000e+01
+2.208140000000000e+02 +2.100000000000000e+01
+2.314420000000000e+02 +2.100000000000000e+01
+3.012490000000000e+02 +2.050000000000000e+01
+2.421250000000000e+02 +2.050000000000000e+01
+8.258499999999999e+01 +2.000000000000000e+01
+1.389370000000000e+02 +2.000000000000000e+01
+1.496960000000000e+02 +2.000000000000000e+01
+2.762730000000000e+02 +1.950000000000000e+01
+2.310180000000000e+00 +1.950000000000000e+01
+2.659820000000000e+02 +1.950000000000000e+01
+6.236010000000000e+01 +1.900000000000000e+01
+1.770580000000000e+02 +1.850000000000000e+01
+1.635550000000000e+02 +1.800000000000000e+01
+1.242290000000000e+02 +1.800000000000000e+01
+1.470630000000000e+02 +1.750000000000000e+01
+4.472060000000000e+01 +1.700000000000000e+01
+2.585100000000000e+02 +1.700000000000000e+01
+5.806350000000000e+01 +1.650000000000000e+01
+2.080500000000000e+02 +1.650000000000000e+01
+2.404970000000000e+02 +1.650000000000000e+01
+2.166710000000000e+02 +1.600000000000000e+01
+1.291000000000000e+02 +1.600000000000000e+01
+3.346520000000000e-01 +2.000000000000000e+00
+1.974340000000000e+02 +1.550000000000000e+01
+1.069830000000000e+02 +1.550000000000000e+01
+2.236380000000000e+02 +1.550000000000000e+01
+2.127860000000000e+02 +1.500000000000000e+01
+2.089320000000000e+02 +1.450000000000000e+01
+1.377000000000000e+02 +1.450000000000000e+01
+1.035810000000000e+02 +1.400000000000000e+01
+1.544480000000000e+02 +1.400000000000000e+01
+1.637480000000000e+02 +1.350000000000000e+01
+5.647210000000000e+01 +1.350000000000000e+01
+1.971780000000000e+02 +1.350000000000000e+01
+4.224290000000000e+01 +1.300000000000000e+01
+4.534940000000000e+01 +1.300000000000000e+01
+2.401010000000000e+01 +1.300000000000000e+01
+1.243900000000000e+02 +1.300000000000000e+01
+9.850650000000000e+01 +1.300000000000000e+01
+7.094459999999999e+01 +1.300000000000000e+01
+4.762130000000000e+01 +1.250000000000000e+01
+2.133360000000000e+01 +1.200000000000000e+01
+5.515400000000000e+01 +1.150000000000000e+01
+8.368770000000001e+01 +1.150000000000000e+01
+1.414360000000000e+02 +1.150000000000000e+01
+7.140200000000000e+01 +1.100000000000000e+01
+9.516990000000000e+01 +1.100000000000000e+01
+1.125180000000000e+02 +1.050000000000000e+01
+8.174079999999999e+01 +1.050000000000000e+01
+1.501470000000000e+02 +1.050000000000000e+01
+8.192540000000000e+01 +1.050000000000000e+01
+9.256900000000000e+01 +1.050000000000000e+01
+3.849120000000000e+01 +1.000000000000000e+01
+1.033690000000000e+02 +1.000000000000000e+01
+1.255790000000000e+02 +1.000000000000000e+01
+1.405960000000000e+01 +9.500000000000000e+00
+1.391200000000000e+02 +9.500000000000000e+00
+4.220250000000000e+01 +9.000000000000000e+00
+6.852900000000000e+01 +8.500000000000000e+00
+6.890630000000000e+01 +8.500000000000000e+00
+9.432920000000000e+01 +8.500000000000000e+00
+9.147230000000000e+01 +8.500000000000000e+00
+6.357150000000000e+01 +8.000000000000000e+00
+7.075109999999999e+01 +8.000000000000000e+00
+8.456170000000000e+01 +8.000000000000000e+00
+2.581800000000000e+01 +7.500000000000000e+00
+6.587350000000001e+01 +7.500000000000000e+00
+6.751620000000000e+01 +7.500000000000000e+00
+9.074240000000000e+01 +7.500000000000000e+00
+1.006960000000000e+02 +7.500000000000000e+00
+1.553180000000000e+01 +7.000000000000000e+00
+5.666550000000000e+01 +7.000000000000000e+00
+1.018180000000000e+01 +7.000000000000000e+00
+8.089210000000000e+01 +6.500000000000000e+00
+7.364319999999998e+00 +6.000000000000000e+00
+5.427690000000000e+01 +6.000000000000000e+00
+2.825420000000000e+01 +5.500000000000000e+00
+4.044720000000000e-01 +5.500000000000000e+00
+6.650620000000001e+01 +5.500000000000000e+00
+5.966140000000000e+01 +5.500000000000000e+00
+3.285420000000000e+01 +5.000000000000000e+00
+2.477780000000000e+01 +4.500000000000000e+00
+3.570510000000000e+01 +4.500000000000000e+00
+3.884440000000000e+00 +4.500000000000000e+00
+1.589760000000000e+01 +4.500000000000000e+00
+4.394050000000000e+01 +4.000000000000000e+00
+4.148580000000000e+01 +4.000000000000000e+00
+3.787390000000000e+01 +4.000000000000000e+00
+2.865460000000000e+01 +3.500000000000000e+00
+1.366360000000000e+01 +2.500000000000000e+00
+6.474750000000000e-01 +2.500000000000000e+00
+1.492360000000000e+01 +2.500000000000000e+00
+2.287520000000000e+01 +2.500000000000000e+00
+1.843750000000000e+01 +2.500000000000000e+00
+1.338260000000000e+01 +2.000000000000000e+00
+4.971600000000000e+00 +1.000000000000000e+00
+4.197890000000000e+00 +1.000000000000000e+00
+2.574990000000000e+00 +5.000000000000000e-01
+0.000000000000000e+00 +5.000000000000000e-01
+0.000000000000000e+00 +0.000000000000000e+00
+0.000000000000000e+00 +0.000000000000000e+00
+0.000000000000000e+00 +0.000000000000000e+00
+0.000000000000000e+00 +0.000000000000000e+00
+0.000000000000000e+00 +0.000000000000000e+00
+0.000000000000000e+00 +0.000000000000000e+00
+0.000000000000000e+00 +0.000000000000000e+00
+0.000000000000000e+00 +0.000000000000000e+00
+0.000000000000000e+00 +0.000000000000000e+00
+0.000000000000000e+00 +0.000000000000000e+00
+0.000000000000000e+00 +0.000000000000000e+00
+0.000000000000000e+00 +0.000000000000000e+00
+2.156010000000000e+03 +8.450000000000000e+02
+1.058870000000000e+03 +8.420000000000000e+02
+1.710230000000000e+03 +8.215000000000000e+02
};
\addplot [only marks, draw=color0, mark size=1.0, fill=color0, opacity=0.75, colormap/viridis]
table{%
x                      y
+1.855480000000000e+03 +2.505000000000000e+02
+6.929910000000001e+02 +5.200000000000000e+01
+1.358890000000000e+03 +1.365000000000000e+02
+1.082450000000000e+03 +1.000000000000000e+02
+1.000600000000000e+03 +1.280000000000000e+02
+2.789250000000000e+02 +2.100000000000000e+01
+1.005390000000000e+03 +1.040000000000000e+02
+9.784960000000000e+02 +1.245000000000000e+02
+6.599019999999998e+02 +8.450000000000000e+01
+1.035650000000000e+03 +1.225000000000000e+02
+9.931680000000000e+02 +1.275000000000000e+02
+1.063840000000000e+03 +1.405000000000000e+02
+8.571039999999998e+02 +8.350000000000000e+01
+4.916920000000000e+02 +7.000000000000000e+01
+9.213240000000000e+02 +1.510000000000000e+02
+6.937250000000000e+02 +5.050000000000000e+01
+8.678980000000000e+02 +8.750000000000000e+01
+2.927260000000000e+02 +3.550000000000000e+01
+7.761550000000000e+02 +4.700000000000000e+01
+1.009020000000000e+03 +1.285000000000000e+02
+9.732370000000000e+02 +8.965000000000000e+02
+7.702950000000000e+02 +4.750000000000000e+01
+2.751700000000000e+02 +2.050000000000000e+01
+7.883260000000000e+02 +4.800000000000000e+01
+1.405180000000000e+03 +1.480000000000000e+02
+1.314050000000000e+03 +1.655000000000000e+02
+1.128820000000000e+03 +1.695000000000000e+02
+6.205850000000000e+02 +1.335000000000000e+02
+1.113520000000000e+03 +1.205000000000000e+02
+9.923200000000001e+02 +1.265000000000000e+02
+2.688120000000000e+03 +3.845000000000000e+02
+1.861630000000000e+03 +2.520000000000000e+02
+1.787660000000000e+03 +3.660000000000000e+02
+1.419050000000000e+03 +1.625000000000000e+02
+2.827610000000000e+02 +2.050000000000000e+01
+2.673470000000000e+03 +3.800000000000000e+02
+3.244580000000000e+02 +2.250000000000000e+01
+9.149010000000000e+02 +6.750000000000000e+01
+6.220290000000000e+02 +1.320000000000000e+02
+7.652639999999999e+02 +4.650000000000000e+01
+9.018980000000000e+02 +9.700000000000000e+01
+1.007110000000000e+03 +1.220000000000000e+02
+1.316050000000000e+03 +1.510000000000000e+02
+1.333760000000000e+03 +1.305000000000000e+02
+2.332510000000000e+03 +4.645000000000000e+02
+9.943860000000000e+02 +9.300000000000000e+01
+1.555860000000000e+03 +1.980000000000000e+02
+7.528860000000002e+02 +8.600000000000000e+01
+7.750030000000000e+02 +4.700000000000000e+01
+9.961030000000000e+02 +1.435000000000000e+02
+1.076600000000000e+03 +1.085000000000000e+02
+9.726490000000000e+02 +1.200000000000000e+02
+1.407650000000000e+03 +2.450000000000000e+02
+6.405459999999998e+02 +7.850000000000000e+01
+2.311750000000000e+03 +2.770000000000000e+02
+1.037340000000000e+03 +1.320000000000000e+02
+1.544140000000000e+03 +2.160000000000000e+02
+8.793720000000000e+02 +6.250000000000000e+01
+6.769030000000000e+02 +4.500000000000000e+01
+1.090250000000000e+03 +1.595000000000000e+02
+1.077230000000000e+03 +1.330000000000000e+02
+1.615110000000000e+03 +2.615000000000000e+02
+1.853750000000000e+03 +2.575000000000000e+02
+1.281020000000000e+03 +1.220000000000000e+02
+7.615050000000000e+02 +5.450000000000000e+01
+1.561150000000000e+03 +1.920000000000000e+02
+1.718540000000000e+03 +1.925000000000000e+02
+1.521870000000000e+03 +2.120000000000000e+02
+6.814839999999998e+02 +7.850000000000000e+01
+1.367550000000000e+03 +3.030000000000000e+02
+8.424169999999998e+02 +6.050000000000000e+01
+9.092920000000000e+02 +8.900000000000000e+01
+9.142070000000000e+02 +1.420000000000000e+02
+1.411260000000000e+03 +1.445000000000000e+02
+1.783500000000000e+03 +2.295000000000000e+02
+8.813600000000000e+02 +6.400000000000000e+01
+6.657650000000000e+02 +7.450000000000000e+01
+9.034400000000001e+02 +1.220000000000000e+02
+4.922770000000000e+02 +6.350000000000000e+01
+6.927030000000000e+02 +1.450000000000000e+02
+7.652170000000000e+02 +4.550000000000000e+01
+1.120120000000000e+03 +1.115000000000000e+02
+6.450870000000000e+02 +1.820000000000000e+02
+6.250269999999998e+02 +7.200000000000000e+01
+1.033020000000000e+03 +1.120000000000000e+02
+8.701080000000002e+02 +6.150000000000000e+01
+4.859460000000000e+02 +5.950000000000000e+01
+6.028950000000000e+02 +7.250000000000000e+01
+6.742680000000000e+02 +8.950000000000000e+01
+3.369630000000000e+02 +8.845000000000000e+02
+7.680560000000000e+02 +4.750000000000000e+01
+1.349740000000000e+03 +1.880000000000000e+02
+6.668960000000002e+02 +1.260000000000000e+02
+1.574100000000000e+03 +1.935000000000000e+02
+1.303820000000000e+03 +2.305000000000000e+02
+1.864100000000000e+03 +2.575000000000000e+02
+1.446080000000000e+03 +2.025000000000000e+02
+7.516039999999998e+02 +5.500000000000000e+01
+1.297880000000000e+03 +1.540000000000000e+02
+1.059400000000000e+03 +9.800000000000000e+01
+1.029620000000000e+03 +1.130000000000000e+02
+8.765510000000000e+02 +6.650000000000000e+01
+6.209950000000000e+02 +1.235000000000000e+02
+7.729119999999998e+02 +4.650000000000000e+01
+1.392970000000000e+03 +1.865000000000000e+02
+8.909920000000000e+02 +6.400000000000000e+01
+8.542030000000000e+02 +1.225000000000000e+02
+6.794160000000001e+02 +7.750000000000000e+01
+8.859299999999999e+02 +6.600000000000000e+01
+9.172310000000000e+02 +1.215000000000000e+02
+1.004570000000000e+03 +1.115000000000000e+02
+1.309740000000000e+03 +1.325000000000000e+02
+9.760050000000000e+02 +1.165000000000000e+02
+7.774040000000000e+02 +6.800000000000000e+01
+1.082910000000000e+03 +1.000000000000000e+02
+7.743650000000000e+02 +6.100000000000000e+01
+9.081860000000000e+02 +1.230000000000000e+02
+1.045950000000000e+03 +1.065000000000000e+02
+1.289490000000000e+03 +1.335000000000000e+02
+3.345040000000000e+02 +2.450000000000000e+01
+1.230000000000000e+03 +1.275000000000000e+02
+1.061130000000000e+03 +9.450000000000000e+01
+1.127560000000000e+03 +1.160000000000000e+02
+6.511390000000000e+02 +6.950000000000000e+01
+9.662790000000000e+02 +8.795000000000000e+02
+4.920510000000000e+02 +2.950000000000000e+01
+1.412390000000000e+03 +2.360000000000000e+02
+8.554490000000000e+02 +6.700000000000000e+01
+6.699600000000000e+02 +6.950000000000000e+01
+1.322320000000000e+03 +1.850000000000000e+02
+7.695260000000002e+02 +4.750000000000000e+01
+1.302550000000000e+03 +1.510000000000000e+02
+1.475220000000000e+03 +2.010000000000000e+02
+1.364020000000000e+03 +2.035000000000000e+02
+4.942410000000000e+02 +3.850000000000000e+01
+9.976070000000000e+02 +1.135000000000000e+02
+1.755750000000000e+03 +1.865000000000000e+02
+1.335690000000000e+03 +2.270000000000000e+02
+6.873520000000000e+02 +1.645000000000000e+02
+1.055960000000000e+03 +1.100000000000000e+02
+1.679330000000000e+03 +2.465000000000000e+02
+6.679220000000000e+02 +8.200000000000000e+01
+1.125690000000000e+03 +1.545000000000000e+02
+1.321400000000000e+03 +1.325000000000000e+02
+1.151700000000000e+03 +2.105000000000000e+02
+6.867030000000000e+02 +1.130000000000000e+02
+1.868770000000000e+03 +2.510000000000000e+02
+1.061180000000000e+03 +9.500000000000000e+01
+3.276340000000000e+02 +8.770000000000000e+02
+7.820750000000000e+02 +4.850000000000000e+01
+1.593120000000000e+03 +2.525000000000000e+02
+4.937380000000001e+02 +5.650000000000000e+01
+7.530640000000000e+02 +7.700000000000000e+01
+9.970080000000000e+02 +1.050000000000000e+02
+1.083920000000000e+03 +9.650000000000000e+01
+1.138390000000000e+03 +1.600000000000000e+02
+1.306300000000000e+03 +1.320000000000000e+02
+7.503439999999998e+02 +4.550000000000000e+01
+1.112550000000000e+03 +1.225000000000000e+02
+1.409210000000000e+03 +2.340000000000000e+02
+9.030790000000000e+02 +1.425000000000000e+02
+9.929140000000000e+02 +1.130000000000000e+02
+1.624530000000000e+03 +2.515000000000000e+02
+1.548120000000000e+03 +1.805000000000000e+02
+1.330140000000000e+03 +2.930000000000000e+02
+2.774470000000000e+02 +2.100000000000000e+01
+9.401740000000000e+02 +1.390000000000000e+02
+1.289110000000000e+03 +1.795000000000000e+02
+7.782580000000000e+02 +4.700000000000000e+01
+1.296930000000000e+03 +1.395000000000000e+02
+8.826810000000000e+02 +6.450000000000000e+01
+9.219180000000000e+02 +1.175000000000000e+02
+1.004130000000000e+03 +8.150000000000000e+01
+8.392420000000000e+02 +6.100000000000000e+01
+9.895470000000000e+02 +1.075000000000000e+02
+1.056690000000000e+03 +1.145000000000000e+02
+6.888700000000000e+02 +1.340000000000000e+02
+9.011180000000001e+02 +6.300000000000000e+01
+1.251460000000000e+03 +2.085000000000000e+02
+1.854570000000000e+03 +2.485000000000000e+02
+9.510910000000000e+02 +7.050000000000000e+01
+7.621510000000002e+02 +1.495000000000000e+02
+1.687820000000000e+03 +2.610000000000000e+02
+9.252900000000000e+02 +6.500000000000000e+01
+1.068550000000000e+03 +2.115000000000000e+02
+1.285780000000000e+03 +2.360000000000000e+02
+2.799990000000000e+02 +2.100000000000000e+01
+8.868789999999998e+02 +1.290000000000000e+02
+1.488760000000000e+03 +1.990000000000000e+02
+1.506670000000000e+03 +1.915000000000000e+02
+7.533420000000000e+02 +1.540000000000000e+02
+8.638570000000000e+02 +6.100000000000000e+01
+1.050970000000000e+03 +2.080000000000000e+02
+6.372060000000000e+02 +6.200000000000000e+01
+8.693700000000000e+02 +6.150000000000000e+01
+1.463790000000000e+03 +2.370000000000000e+02
+6.746410000000002e+02 +7.900000000000000e+01
+8.406660000000001e+02 +6.000000000000000e+01
+1.306350000000000e+03 +1.930000000000000e+02
+6.915980000000002e+02 +1.350000000000000e+02
+1.555840000000000e+03 +1.740000000000000e+02
+8.665139999999999e+02 +2.200000000000000e+02
+1.079740000000000e+03 +2.045000000000000e+02
+1.107910000000000e+03 +1.155000000000000e+02
+9.343110000000000e+02 +6.400000000000000e+01
+4.536680000000000e+02 +3.350000000000000e+01
+1.375300000000000e+03 +1.960000000000000e+02
+1.371760000000000e+03 +2.175000000000000e+02
+9.590110000000000e+02 +8.680000000000000e+02
+8.587040000000000e+02 +1.235000000000000e+02
+3.163020000000000e+02 +8.680000000000000e+02
+3.415719999999999e+02 +8.680000000000000e+02
+6.427090000000002e+02 +6.300000000000000e+01
+8.410200000000000e+02 +1.200000000000000e+02
+6.929989999999998e+02 +1.310000000000000e+02
+7.707869999999998e+02 +4.550000000000000e+01
+1.074230000000000e+03 +2.045000000000000e+02
+6.367410000000000e+02 +6.100000000000000e+01
+3.374630000000000e+02 +2.450000000000000e+01
+7.727520000000000e+02 +5.550000000000000e+01
+7.538180000000000e+02 +7.450000000000000e+01
+1.063320000000000e+03 +1.160000000000000e+02
+1.289720000000000e+03 +2.285000000000000e+02
+6.403869999999999e+02 +6.000000000000000e+01
+1.728760000000000e+03 +2.910000000000000e+02
+6.798819999999999e+02 +1.545000000000000e+02
+6.556920000000000e+02 +5.700000000000000e+01
+1.871470000000000e+03 +3.615000000000000e+02
+9.569390000000000e+02 +1.030000000000000e+02
+9.852000000000000e+02 +1.365000000000000e+02
+1.143960000000000e+03 +1.525000000000000e+02
+1.095920000000000e+03 +1.690000000000000e+02
+9.154320000000000e+02 +1.345000000000000e+02
+1.840350000000000e+03 +3.620000000000000e+02
+1.018190000000000e+03 +1.095000000000000e+02
+1.717270000000000e+03 +2.120000000000000e+02
+9.112089999999999e+02 +1.095000000000000e+02
+1.860700000000000e+03 +3.595000000000000e+02
+4.910120000000000e+02 +4.550000000000000e+01
+6.522100000000000e+02 +1.720000000000000e+02
+6.853520000000000e+02 +1.555000000000000e+02
+1.101700000000000e+03 +1.370000000000000e+02
+9.594070000000000e+02 +8.635000000000000e+02
+7.663120000000000e+02 +1.160000000000000e+02
+1.134520000000000e+03 +1.945000000000000e+02
+6.645080000000000e+02 +5.750000000000000e+01
+7.520870000000000e+02 +1.495000000000000e+02
+1.363140000000000e+03 +2.070000000000000e+02
+1.867270000000000e+03 +3.600000000000000e+02
+1.146150000000000e+03 +1.730000000000000e+02
+7.458620000000000e+02 +1.285000000000000e+02
+7.681710000000000e+02 +1.155000000000000e+02
+1.300960000000000e+03 +1.295000000000000e+02
+8.795350000000000e+02 +5.950000000000000e+01
+8.732840000000000e+02 +1.080000000000000e+02
+2.326670000000000e+03 +3.460000000000000e+02
+1.320140000000000e+03 +2.895000000000000e+02
+7.746770000000000e+02 +1.175000000000000e+02
+1.865620000000000e+03 +2.375000000000000e+02
+1.060150000000000e+03 +1.830000000000000e+02
+1.295920000000000e+03 +1.360000000000000e+02
+1.932970000000000e+03 +3.175000000000000e+02
+1.753470000000000e+03 +2.130000000000000e+02
+1.388780000000000e+03 +2.215000000000000e+02
+1.094170000000000e+03 +2.190000000000000e+02
+6.664250000000000e+02 +1.705000000000000e+02
+2.818490000000000e+02 +9.350000000000000e+01
+1.017310000000000e+03 +7.250000000000000e+01
+3.224120000000001e+02 +8.605000000000000e+02
+9.808030000000000e+02 +1.030000000000000e+02
+1.215670000000000e+03 +2.090000000000000e+02
+1.064130000000000e+03 +1.985000000000000e+02
+6.876990000000000e+02 +7.450000000000000e+01
+1.340300000000000e+03 +2.755000000000000e+02
+8.938839999999999e+02 +1.495000000000000e+02
+1.108810000000000e+03 +1.590000000000000e+02
+6.311300000000000e+02 +5.700000000000000e+01
+1.156770000000000e+03 +2.500000000000000e+02
+1.874760000000000e+03 +3.635000000000000e+02
+6.782950000000000e+02 +1.520000000000000e+02
+1.017980000000000e+03 +1.715000000000000e+02
+4.870110000000000e+02 +4.500000000000000e+01
+2.771970000000000e+02 +9.250000000000000e+01
+7.673850000000000e+02 +1.140000000000000e+02
+1.868780000000000e+03 +2.360000000000000e+02
+6.868489999999998e+02 +1.535000000000000e+02
+6.226500000000000e+02 +9.850000000000000e+01
+9.735540000000000e+02 +9.850000000000000e+01
+8.776870000000000e+02 +6.150000000000000e+01
+1.074710000000000e+03 +1.955000000000000e+02
+6.854330000000000e+02 +7.350000000000000e+01
+9.873360000000000e+02 +7.200000000000000e+01
+1.084100000000000e+03 +1.050000000000000e+02
+1.066170000000000e+03 +2.000000000000000e+02
+6.612380000000001e+02 +5.300000000000000e+01
+1.853290000000000e+03 +3.550000000000000e+02
+7.751060000000001e+02 +1.115000000000000e+02
+1.885520000000000e+03 +3.490000000000000e+02
+7.634880000000001e+02 +1.440000000000000e+02
+5.886840000000000e+02 +9.150000000000000e+01
+6.051230000000000e+02 +5.050000000000000e+01
+1.301590000000000e+03 +2.845000000000000e+02
+1.740150000000000e+03 +2.835000000000000e+02
+1.072860000000000e+03 +1.890000000000000e+02
+3.017960000000000e+02 +8.545000000000000e+02
+9.745210000000000e+02 +9.950000000000000e+01
+1.326760000000000e+03 +2.765000000000000e+02
+1.131450000000000e+03 +8.650000000000000e+01
+1.079120000000000e+03 +1.940000000000000e+02
+7.544160000000001e+02 +6.250000000000000e+01
+2.667070000000000e+03 +2.850000000000000e+02
+1.115000000000000e+03 +1.025000000000000e+02
+1.412080000000000e+03 +2.125000000000000e+02
+7.684710000000000e+02 +1.075000000000000e+02
+1.385550000000000e+03 +2.030000000000000e+02
+2.340510000000000e+03 +3.210000000000000e+02
+8.846790000000000e+02 +9.550000000000000e+01
+1.478670000000000e+03 +3.640000000000000e+02
+1.371670000000000e+03 +2.150000000000000e+02
+7.862030000000000e+02 +1.040000000000000e+02
+9.374980000000000e+02 +1.610000000000000e+02
+1.294660000000000e+03 +1.210000000000000e+02
+6.514410000000000e+02 +5.200000000000000e+01
+1.314960000000000e+03 +1.295000000000000e+02
+1.736100000000000e+03 +2.075000000000000e+02
+2.819530000000000e+02 +8.600000000000000e+01
+1.868040000000000e+03 +3.500000000000000e+02
+1.195210000000000e+03 +8.400000000000000e+01
+6.896230000000000e+02 +1.255000000000000e+02
+1.767740000000000e+03 +2.040000000000000e+02
+9.855599999999999e+02 +1.550000000000000e+02
+9.644340000000000e+02 +8.510000000000000e+02
+3.122490000000000e+02 +8.510000000000000e+02
+7.724630000000002e+02 +1.090000000000000e+02
+1.432610000000000e+03 +2.015000000000000e+02
+1.485030000000000e+03 +3.195000000000000e+02
+7.563610000000001e+02 +1.080000000000000e+02
+8.797869999999998e+02 +9.300000000000000e+01
+1.070180000000000e+03 +1.975000000000000e+02
+4.980300000000000e+02 +4.000000000000000e+01
+9.409660000000000e+02 +8.495000000000000e+02
+2.910620000000000e+02 +8.495000000000000e+02
+1.553080000000000e+03 +1.580000000000000e+02
+9.848819999999999e+02 +8.490000000000000e+02
+7.644370000000000e+02 +1.075000000000000e+02
+1.154520000000000e+03 +1.930000000000000e+02
+2.761120000000000e+02 +8.500000000000000e+01
+6.336230000000000e+02 +5.050000000000000e+01
+1.642100000000000e+03 +3.240000000000000e+02
+9.473900000000000e+02 +1.225000000000000e+02
+1.099940000000000e+03 +1.625000000000000e+02
+1.137910000000000e+03 +2.055000000000000e+02
+1.040900000000000e+03 +1.615000000000000e+02
+7.773270000000000e+02 +1.075000000000000e+02
+1.095900000000000e+03 +1.535000000000000e+02
+9.908400000000000e+02 +1.965000000000000e+02
+1.383490000000000e+03 +2.705000000000000e+02
+1.621390000000000e+03 +2.240000000000000e+02
+8.854490000000000e+02 +1.565000000000000e+02
+5.109180000000000e+02 +8.470000000000000e+02
+7.477020000000000e+02 +1.170000000000000e+02
+7.006310000000002e+02 +1.235000000000000e+02
+1.994490000000000e+03 +3.200000000000000e+02
+6.752819999999998e+02 +1.200000000000000e+02
+1.853440000000000e+03 +2.235000000000000e+02
+8.630560000000000e+02 +1.185000000000000e+02
+1.100030000000000e+03 +1.390000000000000e+02
+7.790549999999999e+02 +1.075000000000000e+02
+9.754500000000000e+02 +1.180000000000000e+02
+9.395630000000000e+02 +1.585000000000000e+02
+1.062900000000000e+03 +9.150000000000000e+01
+7.450100000000000e+02 +1.000000000000000e+02
+2.346410000000000e+03 +4.015000000000000e+02
+8.733410000000000e+02 +1.980000000000000e+02
+9.422270000000000e+02 +8.445000000000000e+02
+2.793860000000000e+02 +8.445000000000000e+02
+9.775520000000000e+02 +1.180000000000000e+02
+6.353480000000002e+02 +1.525000000000000e+02
+8.616540000000000e+02 +1.065000000000000e+02
+7.708620000000000e+02 +1.050000000000000e+02
+1.286160000000000e+03 +2.090000000000000e+02
+2.769890000000000e+02 +8.200000000000000e+01
+9.895940000000001e+02 +1.790000000000000e+02
+7.536330000000000e+02 +1.335000000000000e+02
+9.842980000000000e+02 +1.485000000000000e+02
+6.613760000000002e+02 +8.400000000000000e+01
+6.880050000000000e+02 +1.090000000000000e+02
+9.079470000000000e+02 +9.800000000000000e+01
+4.959750000000000e+02 +3.900000000000000e+01
+8.044800000000000e+02 +1.025000000000000e+02
+6.333490000000000e+02 +1.545000000000000e+02
+3.053900000000000e+02 +8.425000000000000e+02
+1.957680000000000e+03 +3.115000000000000e+02
+1.061520000000000e+03 +2.830000000000000e+02
+6.787100000000000e+02 +1.440000000000000e+02
+9.757900000000000e+02 +8.420000000000000e+02
+7.810250000000000e+02 +1.065000000000000e+02
+1.059000000000000e+03 +1.830000000000000e+02
+1.008060000000000e+03 +1.670000000000000e+02
+9.093170000000000e+02 +9.000000000000000e+01
+9.908130000000000e+02 +1.915000000000000e+02
+1.071600000000000e+03 +1.815000000000000e+02
+7.005340000000000e+02 +1.325000000000000e+02
+7.563689999999998e+02 +1.025000000000000e+02
+9.069530000000000e+02 +8.700000000000000e+01
+8.522619999999999e+02 +9.100000000000000e+01
+6.421030000000002e+02 +1.605000000000000e+02
+3.368600000000000e+02 +2.400000000000000e+01
+1.559630000000000e+03 +3.090000000000000e+02
+1.071320000000000e+03 +1.895000000000000e+02
+1.217790000000000e+03 +2.740000000000000e+02
+1.083850000000000e+02 +1.250000000000000e+01
+1.312460000000000e+03 +1.290000000000000e+02
+8.718589999999998e+02 +9.250000000000000e+01
+6.895599999999999e+02 +1.355000000000000e+02
+7.652080000000002e+02 +1.015000000000000e+02
+1.995420000000000e+03 +3.095000000000000e+02
+1.857680000000000e+03 +3.395000000000000e+02
+6.366369999999999e+02 +1.520000000000000e+02
+7.874839999999998e+02 +1.035000000000000e+02
+1.834420000000000e+03 +3.830000000000000e+02
+6.779220000000000e+02 +6.200000000000000e+01
+9.775950000000000e+02 +1.145000000000000e+02
+1.118680000000000e+03 +1.905000000000000e+02
+9.545810000000000e+02 +1.450000000000000e+02
+8.599119999999998e+02 +1.035000000000000e+02
+1.269720000000000e+03 +1.720000000000000e+02
+6.837869999999998e+02 +1.480000000000000e+02
+6.739550000000000e+02 +1.430000000000000e+02
+1.568390000000000e+03 +3.060000000000000e+02
+6.617710000000002e+02 +1.005000000000000e+02
+1.001510000000000e+03 +1.725000000000000e+02
+1.571630000000000e+03 +1.750000000000000e+02
+7.621860000000000e+02 +9.550000000000000e+01
+1.074730000000000e+03 +1.680000000000000e+02
+8.081730000000000e+02 +9.950000000000000e+01
+2.841130000000000e+02 +7.600000000000000e+01
+1.107510000000000e+03 +2.240000000000000e+02
+9.897880000000000e+02 +1.135000000000000e+02
+7.647830000000000e+02 +1.460000000000000e+02
+6.474290000000000e+02 +1.510000000000000e+02
+7.767300000000000e+02 +1.000000000000000e+02
+1.109260000000000e+03 +1.725000000000000e+02
+1.284240000000000e+03 +1.830000000000000e+02
+6.338350000000000e+02 +1.480000000000000e+02
+1.372080000000000e+03 +2.895000000000000e+02
+8.695520000000000e+02 +1.830000000000000e+02
+6.570790000000000e+02 +1.495000000000000e+02
+6.935280000000000e+02 +1.075000000000000e+02
+9.783480000000000e+02 +1.100000000000000e+02
+7.813930000000000e+02 +9.750000000000000e+01
+1.068750000000000e+03 +1.675000000000000e+02
+1.215980000000000e+03 +1.800000000000000e+02
+1.065360000000000e+03 +1.860000000000000e+02
+1.010290000000000e+03 +1.630000000000000e+02
+2.316700000000000e+03 +4.110000000000000e+02
+7.519169999999998e+02 +9.900000000000000e+01
+8.780500000000000e+02 +8.900000000000000e+01
+1.849450000000000e+03 +3.210000000000000e+02
+2.741560000000000e+02 +6.150000000000000e+01
+9.541310000000000e+02 +8.315000000000000e+02
+8.836020000000000e+02 +8.500000000000000e+01
+2.894710000000000e+02 +8.315000000000000e+02
+2.722100000000000e+02 +5.900000000000000e+01
+7.520630000000000e+02 +1.070000000000000e+02
+9.085930000000000e+02 +8.150000000000000e+01
+1.447420000000000e+03 +2.755000000000000e+02
+1.571700000000000e+03 +2.195000000000000e+02
+1.300200000000000e+03 +1.975000000000000e+02
+8.685360000000002e+02 +1.110000000000000e+02
+6.173750000000000e+02 +1.490000000000000e+02
+1.568770000000000e+03 +1.735000000000000e+02
+7.833930000000000e+02 +9.800000000000000e+01
+8.927630000000000e+02 +8.850000000000000e+01
+1.124050000000000e+03 +1.885000000000000e+02
+4.898910000000000e+02 +1.135000000000000e+02
+6.616050000000000e+02 +9.600000000000000e+01
+6.763070000000000e+02 +1.375000000000000e+02
+4.463600000000000e+02 +8.450000000000000e+01
+7.750110000000002e+02 +9.150000000000000e+01
+1.537790000000000e+03 +3.005000000000000e+02
+4.580520000000000e+02 +6.650000000000000e+01
+2.655820000000000e+02 +8.285000000000000e+02
+3.224100000000000e+02 +8.550000000000000e+01
+1.052960000000000e+03 +1.695000000000000e+02
+2.810300000000000e+02 +6.700000000000000e+01
+6.608700000000000e+02 +1.385000000000000e+02
+6.856400000000000e+02 +1.345000000000000e+02
+1.001390000000000e+03 +1.795000000000000e+02
+1.038310000000000e+03 +1.440000000000000e+02
+1.119310000000000e+03 +1.645000000000000e+02
+7.711050000000000e+02 +1.775000000000000e+02
+8.625419999999998e+02 +1.780000000000000e+02
+1.074490000000000e+03 +1.725000000000000e+02
+1.000060000000000e+03 +1.600000000000000e+02
+6.918090000000000e+02 +9.850000000000000e+01
+1.929100000000000e+03 +2.995000000000000e+02
+1.302610000000000e+03 +2.135000000000000e+02
+1.401780000000000e+03 +3.045000000000000e+02
+1.057420000000000e+03 +1.535000000000000e+02
+3.251410000000000e+02 +8.550000000000000e+01
+1.109320000000000e+03 +1.625000000000000e+02
+1.491390000000000e+03 +4.180000000000000e+02
+1.004920000000000e+03 +1.765000000000000e+02
+1.043590000000000e+03 +1.695000000000000e+02
+1.737300000000000e+03 +4.345000000000000e+02
+9.064880000000001e+02 +1.825000000000000e+02
+6.397780000000000e+02 +1.460000000000000e+02
+1.306460000000000e+03 +1.505000000000000e+02
+1.078750000000000e+02 +2.000000000000000e+01
+6.927280000000002e+02 +1.310000000000000e+02
+6.243390000000001e+02 +6.800000000000000e+01
+1.554790000000000e+03 +3.020000000000000e+02
+9.001080000000002e+02 +8.950000000000000e+01
+1.643010000000000e+03 +1.645000000000000e+02
+6.570520000000000e+02 +1.485000000000000e+02
+6.659150000000000e+02 +1.325000000000000e+02
+1.147470000000000e+03 +1.390000000000000e+02
+6.801230000000000e+02 +9.850000000000000e+01
+2.289960000000000e+03 +4.880000000000000e+02
+9.885520000000000e+02 +1.780000000000000e+02
+1.048400000000000e+03 +1.810000000000000e+02
+6.509550000000000e+02 +1.440000000000000e+02
+7.665280000000000e+02 +8.050000000000000e+01
+8.564400000000001e+02 +8.500000000000000e+01
+1.473170000000000e+03 +1.935000000000000e+02
+1.325310000000000e+03 +2.545000000000000e+02
+9.779730000000000e+02 +1.670000000000000e+02
+1.076090000000000e+03 +1.660000000000000e+02
+2.697470000000000e+03 +5.210000000000000e+02
+2.353080000000000e+03 +4.100000000000000e+02
+8.828780000000000e+02 +8.750000000000000e+01
+1.100910000000000e+03 +2.185000000000000e+02
+7.645510000000000e+02 +8.950000000000000e+01
+1.747960000000000e+03 +4.320000000000000e+02
+1.302030000000000e+03 +2.135000000000000e+02
+2.013590000000000e+03 +2.965000000000000e+02
+8.437950000000000e+02 +8.100000000000000e+01
+8.679060000000002e+02 +1.725000000000000e+02
+7.894340000000000e+02 +1.740000000000000e+02
+6.202510000000000e+02 +1.380000000000000e+02
+7.513789999999998e+02 +1.005000000000000e+02
+7.745980000000002e+02 +8.350000000000000e+01
+9.000630000000000e+02 +8.350000000000000e+01
+8.934250000000000e+02 +1.275000000000000e+02
+4.822860000000000e+02 +1.060000000000000e+02
+9.755839999999999e+02 +1.650000000000000e+02
+4.800760000000000e+02 +8.400000000000000e+01
+6.452130000000002e+02 +1.410000000000000e+02
+6.550530000000000e+02 +1.290000000000000e+02
+9.047310000000000e+02 +8.850000000000000e+01
+1.281020000000000e+03 +2.485000000000000e+02
+1.426250000000000e+02 +8.180000000000000e+02
+1.294290000000000e+03 +1.990000000000000e+02
+1.317090000000000e+03 +2.500000000000000e+02
+7.918639999999998e+02 +1.730000000000000e+02
+8.051940000000000e+02 +1.865000000000000e+02
+1.305950000000000e+03 +2.050000000000000e+02
+1.362340000000000e+03 +2.910000000000000e+02
+9.026230000000000e+02 +2.015000000000000e+02
+1.745480000000000e+03 +3.040000000000000e+02
+1.006470000000000e+03 +1.685000000000000e+02
+7.564780000000002e+02 +9.000000000000000e+01
+9.337050000000000e+02 +1.945000000000000e+02
+6.186330000000000e+02 +5.900000000000000e+01
+2.008270000000000e+03 +3.645000000000000e+02
+1.010440000000000e+03 +1.540000000000000e+02
+6.715030000000000e+02 +1.400000000000000e+02
+1.321370000000000e+03 +2.460000000000000e+02
+6.829069999999998e+02 +1.350000000000000e+02
+6.200520000000000e+02 +1.270000000000000e+02
+1.308850000000000e+03 +2.035000000000000e+02
+9.701990000000000e+02 +1.115000000000000e+02
+6.625000000000000e+02 +1.280000000000000e+02
+6.452040000000002e+02 +2.330000000000000e+02
+1.562040000000000e+03 +1.615000000000000e+02
+7.685910000000000e+02 +8.200000000000000e+01
+1.111010000000000e+03 +1.485000000000000e+02
+1.074890000000000e+03 +2.565000000000000e+02
+6.816640000000000e+02 +1.435000000000000e+02
+1.065210000000000e+03 +1.650000000000000e+02
+6.612930000000000e+02 +1.395000000000000e+02
+2.326120000000000e+03 +4.740000000000000e+02
+7.869800000000000e+02 +9.050000000000000e+01
+1.615310000000000e+03 +3.125000000000000e+02
+8.762460000000002e+02 +8.850000000000000e+01
+1.319820000000000e+03 +2.320000000000000e+02
+8.487940000000000e+02 +1.695000000000000e+02
+1.081300000000000e+03 +1.545000000000000e+02
+8.049220000000000e+02 +1.750000000000000e+02
+1.032840000000000e+03 +1.320000000000000e+02
+1.325730000000000e+03 +2.470000000000000e+02
+7.848960000000002e+02 +8.500000000000000e+01
+1.768030000000000e+03 +2.740000000000000e+02
+1.471360000000000e+03 +1.805000000000000e+02
+1.037910000000000e+03 +1.585000000000000e+02
+7.474190000000000e+02 +7.850000000000000e+01
+1.776370000000000e+03 +2.740000000000000e+02
+1.010250000000000e+03 +1.635000000000000e+02
+2.833470000000000e+02 +5.350000000000000e+01
+8.427150000000000e+02 +8.950000000000000e+01
+6.666430000000000e+02 +4.600000000000000e+01
+1.853830000000000e+03 +3.070000000000000e+02
+1.095390000000000e+03 +2.050000000000000e+02
+1.487440000000000e+03 +2.535000000000000e+02
+6.606139999999998e+02 +5.400000000000000e+01
+7.552110000000000e+02 +8.000000000000000e+01
+1.081190000000000e+03 +1.485000000000000e+02
+4.895660000000000e+02 +1.005000000000000e+02
+1.362440000000000e+03 +1.980000000000000e+02
+1.010850000000000e+03 +1.575000000000000e+02
+6.727919999999998e+02 +1.195000000000000e+02
+1.877950000000000e+03 +3.120000000000000e+02
+2.830630000000000e+02 +8.095000000000000e+02
+1.321620000000000e+03 +1.465000000000000e+02
+1.433640000000000e+03 +1.945000000000000e+02
+1.206440000000000e+03 +2.460000000000000e+02
+6.973700000000000e+02 +1.210000000000000e+02
+1.618460000000000e+03 +3.110000000000000e+02
+1.742300000000000e+03 +2.385000000000000e+02
+6.657360000000001e+02 +1.345000000000000e+02
+7.857339999999998e+02 +8.300000000000000e+01
+1.297900000000000e+03 +2.025000000000000e+02
+1.117060000000000e+03 +1.475000000000000e+02
+9.117630000000000e+02 +7.850000000000000e+01
+6.891310000000002e+02 +7.600000000000000e+01
+7.747790000000000e+02 +7.700000000000000e+01
+9.789890000000000e+02 +1.590000000000000e+02
+1.942310000000000e+03 +3.620000000000000e+02
+6.545140000000000e+02 +1.300000000000000e+02
+1.241810000000000e+03 +1.455000000000000e+02
+1.763760000000000e+03 +3.835000000000000e+02
+7.488099999999999e+02 +8.500000000000000e+01
+2.319060000000000e+03 +3.885000000000000e+02
+6.902739999999999e+02 +9.250000000000000e+01
+3.332760000000000e+02 +6.800000000000000e+01
+1.782100000000000e+03 +2.705000000000000e+02
+1.064850000000000e+03 +1.535000000000000e+02
+9.627830000000000e+02 +8.060000000000000e+02
+1.573340000000000e+03 +1.635000000000000e+02
+7.031540000000000e+02 +9.850000000000000e+01
+1.569280000000000e+03 +2.835000000000000e+02
+1.967380000000000e+03 +2.715000000000000e+02
+1.426260000000000e+03 +2.895000000000000e+02
+7.562210000000000e+02 +8.650000000000000e+01
+6.400960000000000e+02 +1.345000000000000e+02
+6.230760000000000e+02 +5.050000000000000e+01
+9.876860000000000e+02 +1.570000000000000e+02
+1.061240000000000e+03 +2.575000000000000e+02
+2.887470000000000e+02 +5.200000000000000e+01
+9.199310000000000e+02 +8.050000000000000e+02
+1.860820000000000e+03 +3.770000000000000e+02
+4.912730000000000e+02 +9.350000000000000e+01
+2.654390000000000e+02 +5.100000000000000e+01
+1.321260000000000e+03 +1.805000000000000e+02
+2.476750000000000e+02 +8.045000000000000e+02
+1.782300000000000e+03 +3.195000000000000e+02
+9.179360000000000e+02 +8.500000000000000e+01
+1.081360000000000e+03 +1.500000000000000e+02
+6.314010000000000e+02 +5.250000000000000e+01
+1.148280000000000e+02 +1.250000000000000e+01
+1.297950000000000e+03 +1.905000000000000e+02
+1.331810000000000e+03 +3.335000000000000e+02
+1.855760000000000e+03 +2.945000000000000e+02
+6.523640000000000e+02 +1.345000000000000e+02
+7.710790000000000e+02 +9.550000000000000e+01
+6.229310000000000e+02 +4.950000000000000e+01
+9.762480000000000e+02 +1.555000000000000e+02
+6.662869999999998e+02 +1.280000000000000e+02
+8.259540000000000e+02 +1.695000000000000e+02
+6.785820000000000e+02 +1.190000000000000e+02
+6.062000000000000e+02 +4.900000000000000e+01
+2.946740000000001e+02 +5.200000000000000e+01
+1.875040000000000e+03 +3.020000000000000e+02
+1.126730000000000e+03 +2.100000000000000e+02
+8.970450000000000e+02 +7.350000000000000e+01
+1.331490000000000e+03 +3.595000000000000e+02
+6.761139999999998e+02 +1.080000000000000e+02
+9.514600000000000e+02 +8.005000000000000e+02
+7.684510000000000e+02 +7.850000000000000e+01
+9.783760000000000e+02 +1.560000000000000e+02
+9.476210000000000e+02 +2.180000000000000e+02
+1.321000000000000e+03 +1.735000000000000e+02
+1.308710000000000e+03 +1.925000000000000e+02
+9.690110000000000e+02 +1.415000000000000e+02
+1.862610000000000e+03 +3.745000000000000e+02
+7.690850000000000e+02 +7.200000000000000e+01
+1.089790000000000e+03 +1.400000000000000e+02
+1.846800000000000e+03 +2.915000000000000e+02
+9.983230000000000e+02 +1.550000000000000e+02
+6.490640000000000e+02 +1.320000000000000e+02
+6.726389999999999e+02 +1.440000000000000e+02
+1.283460000000000e+03 +1.945000000000000e+02
+9.796280000000000e+02 +1.530000000000000e+02
+1.340630000000000e+03 +1.390000000000000e+02
+7.791419999999998e+02 +7.900000000000000e+01
+9.294420000000000e+02 +9.400000000000000e+01
+2.335980000000000e+03 +3.875000000000000e+02
+7.763720000000000e+02 +7.100000000000000e+01
+1.590060000000000e+03 +2.715000000000000e+02
+1.408370000000000e+03 +2.745000000000000e+02
+9.707110000000000e+02 +1.495000000000000e+02
+1.054800000000000e+03 +1.425000000000000e+02
+2.605300000000000e+02 +7.970000000000000e+02
+4.864470000000000e+02 +6.450000000000000e+01
+9.096380000000000e+02 +7.450000000000000e+01
+1.846660000000000e+03 +2.995000000000000e+02
+1.065540000000000e+03 +1.465000000000000e+02
+1.051100000000000e+03 +1.475000000000000e+02
+1.311820000000000e+03 +1.935000000000000e+02
+1.003340000000000e+03 +1.495000000000000e+02
+7.821189999999998e+02 +7.800000000000000e+01
+6.536900000000001e+02 +1.300000000000000e+02
+2.359510000000000e+03 +4.845000000000000e+02
+1.057460000000000e+03 +2.550000000000000e+02
+6.513410000000000e+02 +1.350000000000000e+02
+1.852630000000000e+03 +3.675000000000000e+02
+2.722880000000000e+02 +7.950000000000000e+02
+1.861550000000000e+03 +2.995000000000000e+02
+6.614330000000000e+02 +2.010000000000000e+02
+1.308010000000000e+03 +2.285000000000000e+02
+6.251120000000000e+02 +4.750000000000000e+01
+1.311600000000000e+03 +1.825000000000000e+02
+1.004200000000000e+03 +2.245000000000000e+02
+7.458800000000000e+02 +7.900000000000000e+01
+1.130830000000000e+03 +2.640000000000000e+02
+1.308450000000000e+03 +2.695000000000000e+02
+1.006730000000000e+03 +1.485000000000000e+02
+1.083930000000000e+03 +1.900000000000000e+02
+6.880939999999998e+02 +7.050000000000000e+01
+1.037360000000000e+03 +1.315000000000000e+02
+1.104100000000000e+03 +2.240000000000000e+02
+6.782530000000000e+02 +1.315000000000000e+02
+2.513040000000000e+02 +7.930000000000000e+02
+2.366290000000000e+03 +4.800000000000000e+02
+6.313009999999998e+02 +9.000000000000000e+01
+6.882170000000000e+02 +6.750000000000000e+01
+1.337650000000000e+03 +2.335000000000000e+02
+1.159280000000000e+03 +2.355000000000000e+02
+5.727900000000000e+02 +7.050000000000000e+01
+1.868090000000000e+03 +2.875000000000000e+02
+6.536830000000000e+02 +1.190000000000000e+02
+1.484340000000000e+03 +2.320000000000000e+02
+1.346140000000000e+03 +2.120000000000000e+02
+1.751490000000000e+03 +2.840000000000000e+02
+1.277630000000000e+03 +2.655000000000000e+02
+4.896330000000000e+02 +8.600000000000000e+01
+6.875330000000000e+02 +1.045000000000000e+02
+2.312500000000000e+03 +4.510000000000000e+02
+2.523460000000000e+02 +7.915000000000000e+02
+7.623500000000000e+02 +6.650000000000000e+01
+1.063970000000000e+03 +1.325000000000000e+02
+1.469460000000000e+03 +2.875000000000000e+02
+1.308310000000000e+03 +1.825000000000000e+02
+1.857400000000000e+03 +2.890000000000000e+02
+1.068510000000000e+03 +1.465000000000000e+02
+1.298660000000000e+03 +1.915000000000000e+02
+4.859700000000000e+02 +8.250000000000000e+01
+7.008680000000001e+02 +6.950000000000000e+01
+9.885900000000000e+02 +1.505000000000000e+02
+1.219640000000000e+03 +2.535000000000000e+02
+1.313000000000000e+03 +1.600000000000000e+02
+7.762100000000000e+02 +7.450000000000000e+01
+1.557650000000000e+03 +2.710000000000000e+02
+2.778350000000000e+02 +7.895000000000000e+02
+9.492140000000001e+02 +1.845000000000000e+02
+1.570870000000000e+03 +2.695000000000000e+02
+2.353210000000000e+03 +4.745000000000000e+02
+7.508240000000000e+02 +7.100000000000000e+01
+4.913500000000000e+02 +7.880000000000000e+02
+1.069030000000000e+03 +1.370000000000000e+02
+1.010190000000000e+03 +1.370000000000000e+02
+7.631610000000002e+02 +6.450000000000000e+01
+1.774670000000000e+03 +2.540000000000000e+02
+4.940560000000000e+02 +5.400000000000000e+01
+6.295010000000000e+02 +1.155000000000000e+02
+7.527300000000000e+02 +8.200000000000000e+01
+6.340219999999998e+02 +1.045000000000000e+02
+7.840300000000000e+02 +7.300000000000000e+01
+1.300860000000000e+03 +1.855000000000000e+02
+1.020390000000000e+03 +1.260000000000000e+02
+1.737210000000000e+03 +2.645000000000000e+02
+1.158220000000000e+03 +2.315000000000000e+02
+1.309160000000000e+03 +1.835000000000000e+02
+1.088470000000000e+03 +1.445000000000000e+02
+6.372060000000000e+02 +1.215000000000000e+02
+2.631070000000000e+03 +4.895000000000000e+02
+1.319580000000000e+03 +2.310000000000000e+02
+9.557850000000000e+02 +7.845000000000000e+02
+1.309850000000000e+03 +1.880000000000000e+02
+1.294470000000000e+03 +1.560000000000000e+02
+9.941609999999999e+02 +1.415000000000000e+02
+9.360630000000000e+02 +1.985000000000000e+02
+1.227580000000000e+03 +2.510000000000000e+02
+9.948330000000000e+02 +1.380000000000000e+02
+1.570060000000000e+03 +2.675000000000000e+02
+7.658639999999998e+02 +6.200000000000000e+01
+6.900039999999998e+02 +6.800000000000000e+01
+1.037520000000000e+03 +1.225000000000000e+02
+1.847270000000000e+03 +2.910000000000000e+02
+1.470490000000000e+03 +1.540000000000000e+02
+9.008620000000000e+02 +1.190000000000000e+02
+1.612110000000000e+03 +2.855000000000000e+02
+8.671430000000000e+02 +1.370000000000000e+02
+6.454320000000000e+02 +1.015000000000000e+02
+6.182809999999999e+02 +1.395000000000000e+02
+1.135130000000000e+02 +1.150000000000000e+01
+8.936519999999998e+02 +1.720000000000000e+02
+6.804019999999998e+02 +1.140000000000000e+02
+7.864060000000002e+02 +6.750000000000000e+01
+8.824570000000000e+02 +1.135000000000000e+02
+1.402120000000000e+03 +2.590000000000000e+02
+6.817170000000000e+02 +9.950000000000000e+01
+1.864830000000000e+03 +2.925000000000000e+02
+6.885540000000000e+02 +6.050000000000000e+01
+1.064660000000000e+03 +1.235000000000000e+02
+1.924940000000000e+03 +3.025000000000000e+02
+1.913800000000000e+03 +2.575000000000000e+02
+2.066380000000000e+03 +7.800000000000000e+02
+6.788910000000002e+02 +1.040000000000000e+02
+8.093889999999999e+02 +1.375000000000000e+02
+6.365750000000000e+02 +1.135000000000000e+02
+1.132360000000000e+03 +1.855000000000000e+02
+6.185010000000000e+02 +1.390000000000000e+02
+1.364020000000000e+03 +2.045000000000000e+02
+1.502090000000000e+03 +3.670000000000000e+02
+4.889190000000000e+02 +7.500000000000000e+01
+1.862410000000000e+03 +3.645000000000000e+02
+7.476840000000000e+02 +5.900000000000000e+01
+6.667850000000000e+02 +2.060000000000000e+02
+4.864290000000000e+02 +7.800000000000000e+01
+6.406510000000000e+02 +1.165000000000000e+02
+1.478140000000000e+03 +2.770000000000000e+02
+9.808390000000001e+02 +1.430000000000000e+02
+8.970670000000000e+02 +1.135000000000000e+02
+2.651520000000000e+03 +4.725000000000000e+02
+6.867660000000002e+02 +1.560000000000000e+02
+6.793040000000000e+02 +1.190000000000000e+02
+2.336300000000000e+03 +4.715000000000000e+02
+6.788439999999998e+02 +9.250000000000000e+01
+7.629620000000000e+02 +5.850000000000000e+01
+1.421190000000000e+03 +1.245000000000000e+02
+7.447760000000002e+02 +6.250000000000000e+01
+8.980400000000000e+02 +1.225000000000000e+02
+1.080780000000000e+03 +1.365000000000000e+02
+7.073489999999998e+02 +1.190000000000000e+02
+1.401420000000000e+03 +2.570000000000000e+02
+1.001490000000000e+03 +1.270000000000000e+02
+7.533819999999999e+02 +7.500000000000000e+01
+2.726200000000000e+02 +1.165000000000000e+02
+1.146710000000000e+03 +1.870000000000000e+02
+6.640269999999998e+02 +1.355000000000000e+02
+7.714620000000000e+02 +6.500000000000000e+01
+1.352740000000000e+03 +2.010000000000000e+02
+1.589940000000000e+03 +3.560000000000000e+02
+8.841540000000000e+02 +1.745000000000000e+02
+1.303180000000000e+03 +3.015000000000000e+02
+8.713450000000000e+02 +1.185000000000000e+02
+8.778889999999999e+02 +1.685000000000000e+02
+7.650580000000000e+02 +5.800000000000000e+01
+1.370060000000000e+03 +1.615000000000000e+02
+6.725180000000000e+02 +1.190000000000000e+02
+1.402330000000000e+03 +2.580000000000000e+02
+8.729510000000000e+02 +9.450000000000000e+01
+9.443240000000000e+02 +7.725000000000000e+02
+7.527800000000000e+02 +7.100000000000000e+01
+1.429070000000000e+03 +2.595000000000000e+02
+1.867390000000000e+03 +2.790000000000000e+02
+9.995280000000000e+02 +1.255000000000000e+02
+6.409059999999999e+02 +1.035000000000000e+02
+7.789989999999998e+02 +5.800000000000000e+01
+9.407150000000000e+02 +1.310000000000000e+02
+9.775260000000000e+02 +1.260000000000000e+02
+6.370630000000000e+02 +1.115000000000000e+02
+7.589420000000000e+02 +6.550000000000000e+01
+1.291860000000000e+03 +2.450000000000000e+02
+1.865570000000000e+03 +2.815000000000000e+02
+4.880160000000000e+02 +7.250000000000000e+01
+9.922450000000000e+02 +1.280000000000000e+02
+1.079690000000000e+03 +1.720000000000000e+02
+8.994639999999998e+02 +1.165000000000000e+02
+1.010900000000000e+03 +1.945000000000000e+02
+9.031060000000000e+02 +1.165000000000000e+02
+6.947030000000000e+02 +1.470000000000000e+02
+1.747900000000000e+03 +2.945000000000000e+02
+8.523600000000000e+02 +1.020000000000000e+02
+2.359710000000000e+03 +4.625000000000000e+02
+1.054600000000000e+03 +1.320000000000000e+02
+9.056500000000000e+02 +1.195000000000000e+02
+1.178740000000000e+03 +1.260000000000000e+02
+1.864550000000000e+03 +2.815000000000000e+02
+6.634250000000000e+02 +5.750000000000000e+01
+9.259770000000000e+02 +1.695000000000000e+02
+8.585670000000000e+02 +6.650000000000000e+01
+9.746860000000000e+02 +1.305000000000000e+02
+1.333550000000000e+03 +1.910000000000000e+02
+9.093210000000000e+02 +1.180000000000000e+02
+8.900020000000000e+02 +1.740000000000000e+02
+1.012040000000000e+03 +1.945000000000000e+02
+6.233500000000000e+02 +1.290000000000000e+02
+7.712010000000000e+02 +5.950000000000000e+01
+1.498560000000000e+03 +2.145000000000000e+02
+8.518250000000000e+02 +1.165000000000000e+02
+1.073500000000000e+03 +2.320000000000000e+02
+1.567260000000000e+03 +1.580000000000000e+02
+9.758620000000000e+02 +1.315000000000000e+02
+9.199880000000001e+02 +1.145000000000000e+02
+1.372590000000000e+03 +2.265000000000000e+02
+7.826060000000001e+02 +5.900000000000000e+01
+1.023740000000000e+03 +1.100000000000000e+02
+9.213690000000000e+02 +7.655000000000000e+02
+9.104800000000000e+02 +1.185000000000000e+02
+1.416280000000000e+03 +2.515000000000000e+02
+1.334160000000000e+03 +2.465000000000000e+02
+6.345920000000000e+02 +9.500000000000000e+01
+1.118250000000000e+03 +1.830000000000000e+02
+7.646480000000000e+02 +5.100000000000000e+01
+9.010200000000000e+02 +8.550000000000000e+01
+6.846619999999998e+02 +1.750000000000000e+02
+6.580400000000000e+02 +1.315000000000000e+02
+1.579590000000000e+03 +2.465000000000000e+02
+1.325520000000000e+03 +2.230000000000000e+02
+1.328920000000000e+03 +2.120000000000000e+02
+1.005230000000000e+03 +1.935000000000000e+02
+1.035940000000000e+03 +1.285000000000000e+02
+2.308840000000000e+03 +4.235000000000000e+02
+1.075150000000000e+03 +2.250000000000000e+02
+4.885090000000000e+02 +6.800000000000000e+01
+7.768130000000000e+02 +5.350000000000000e+01
+1.395640000000000e+03 +2.915000000000000e+02
+1.010970000000000e+03 +1.880000000000000e+02
+1.152590000000000e+03 +1.830000000000000e+02
+7.866480000000000e+02 +5.350000000000000e+01
+1.415770000000000e+03 +2.465000000000000e+02
+4.915810000000000e+02 +6.550000000000000e+01
+6.404740000000000e+02 +9.450000000000000e+01
+7.603680000000001e+02 +6.350000000000000e+01
+7.778380000000002e+02 +5.600000000000000e+01
+6.718480000000002e+02 +1.950000000000000e+02
+6.693720000000000e+02 +1.335000000000000e+02
+1.083920000000000e+03 +1.650000000000000e+02
+1.047120000000000e+03 +1.295000000000000e+02
+4.759490000000000e+02 +3.150000000000000e+01
+9.579780000000000e+02 +1.730000000000000e+02
+7.876750000000000e+02 +2.280000000000000e+02
+1.850980000000000e+03 +2.755000000000000e+02
+8.123710000000002e+02 +1.310000000000000e+02
+6.633489999999998e+02 +1.225000000000000e+02
+1.301070000000000e+03 +1.645000000000000e+02
+6.701810000000000e+02 +1.235000000000000e+02
+1.159650000000000e+03 +2.965000000000000e+02
+8.557760000000002e+02 +1.445000000000000e+02
+6.794019999999998e+02 +1.340000000000000e+02
+4.712450000000000e+02 +3.100000000000000e+01
+1.240080000000000e+03 +3.180000000000000e+02
+1.502600000000000e+03 +3.415000000000000e+02
+2.331090000000000e+03 +3.490000000000000e+02
+1.285760000000000e+03 +2.440000000000000e+02
+8.177990000000000e+02 +1.310000000000000e+02
+1.856020000000000e+03 +3.380000000000000e+02
+6.213270000000000e+02 +1.190000000000000e+02
+4.783400000000000e+02 +3.200000000000000e+01
+8.894210000000000e+02 +1.060000000000000e+02
+1.489490000000000e+03 +2.580000000000000e+02
+6.418790000000000e+02 +7.900000000000000e+01
+1.240520000000000e+03 +2.155000000000000e+02
+4.904700000000000e+02 +4.050000000000000e+01
+7.502430000000001e+02 +6.050000000000000e+01
+1.128850000000000e+03 +2.375000000000000e+02
+1.300690000000000e+03 +3.510000000000000e+02
+1.100230000000000e+03 +2.690000000000000e+02
+2.487260000000000e+02 +7.560000000000000e+02
+1.044020000000000e+03 +1.765000000000000e+02
+1.857940000000000e+03 +3.755000000000000e+02
+6.350230000000000e+02 +9.100000000000000e+01
+7.633099999999999e+02 +6.050000000000000e+01
+6.249880000000001e+02 +1.200000000000000e+02
+8.931239999999998e+02 +2.185000000000000e+02
+1.189650000000000e+03 +1.420000000000000e+02
+6.926669999999998e+02 +1.365000000000000e+02
+1.578920000000000e+03 +2.425000000000000e+02
+1.290190000000000e+03 +2.340000000000000e+02
+1.115010000000000e+03 +1.770000000000000e+02
+1.114960000000000e+03 +3.040000000000000e+02
+8.539900000000000e+02 +1.055000000000000e+02
+1.069660000000000e+03 +2.105000000000000e+02
+1.563440000000000e+03 +2.395000000000000e+02
+1.750210000000000e+03 +2.625000000000000e+02
+1.078640000000000e+03 +2.090000000000000e+02
+2.557130000000000e+02 +7.535000000000000e+02
+1.300870000000000e+03 +1.595000000000000e+02
+8.997160000000000e+02 +1.080000000000000e+02
+8.966810000000000e+02 +7.700000000000000e+01
+1.836470000000000e+03 +3.715000000000000e+02
+2.737550000000000e+02 +9.150000000000000e+01
+1.082480000000000e+03 +2.080000000000000e+02
+6.236730000000000e+02 +1.150000000000000e+02
+7.711289999999998e+02 +4.950000000000000e+01
+6.743170000000000e+02 +1.785000000000000e+02
+7.573520000000000e+02 +1.575000000000000e+02
+9.889320000000000e+02 +1.195000000000000e+02
+1.436610000000000e+03 +2.440000000000000e+02
+1.848410000000000e+03 +3.400000000000000e+02
+6.194620000000000e+02 +1.140000000000000e+02
+1.751760000000000e+03 +3.585000000000000e+02
+1.423420000000000e+03 +3.485000000000000e+02
+1.003970000000000e+03 +2.210000000000000e+02
+1.127060000000000e+03 +2.610000000000000e+02
+6.962020000000000e+02 +1.475000000000000e+02
+3.314580000000000e+02 +2.450000000000000e+01
+1.348450000000000e+03 +2.800000000000000e+02
+6.595630000000000e+02 +8.700000000000000e+01
+1.585940000000000e+03 +3.340000000000000e+02
+6.620810000000000e+02 +1.860000000000000e+02
+6.541960000000000e+02 +8.400000000000000e+01
+9.765309999999999e+02 +1.215000000000000e+02
+1.542300000000000e+03 +2.225000000000000e+02
+1.045600000000000e+03 +1.825000000000000e+02
+9.941750000000000e+02 +1.165000000000000e+02
+1.140840000000000e+03 +1.975000000000000e+02
+1.428310000000000e+03 +1.775000000000000e+02
+4.926440000000000e+02 +5.550000000000000e+01
+6.404150000000000e+02 +9.100000000000000e+01
+1.315690000000000e+03 +2.545000000000000e+02
+1.119370000000000e+03 +2.995000000000000e+02
+9.111380000000000e+02 +1.725000000000000e+02
+1.468690000000000e+03 +2.510000000000000e+02
+7.686120000000000e+02 +4.800000000000000e+01
+9.856920000000000e+02 +1.205000000000000e+02
+7.816270000000000e+02 +1.220000000000000e+02
+1.329390000000000e+03 +2.815000000000000e+02
+7.624750000000000e+02 +4.550000000000000e+01
+6.791560000000002e+02 +1.740000000000000e+02
+9.167790000000000e+02 +1.080000000000000e+02
+4.857490000000000e+02 +5.400000000000000e+01
+1.008460000000000e+03 +2.160000000000000e+02
+6.537020000000000e+02 +7.000000000000000e+01
+1.057420000000000e+03 +1.875000000000000e+02
+1.331670000000000e+03 +2.775000000000000e+02
+9.319020000000000e+02 +2.810000000000000e+02
+5.067460000000000e+02 +5.600000000000000e+01
+6.605839999999999e+02 +1.145000000000000e+02
+2.010580000000000e+03 +3.305000000000000e+02
+1.617780000000000e+03 +3.625000000000000e+02
+7.746720000000000e+02 +1.190000000000000e+02
+1.301610000000000e+03 +2.395000000000000e+02
+6.810210000000002e+02 +1.535000000000000e+02
+8.990500000000000e+02 +1.155000000000000e+02
+2.642860000000000e+02 +7.445000000000000e+02
+9.767030000000000e+02 +1.400000000000000e+02
+1.558400000000000e+03 +2.285000000000000e+02
+8.857310000000001e+02 +1.080000000000000e+02
+1.847040000000000e+03 +3.655000000000000e+02
+1.374910000000000e+03 +3.085000000000000e+02
+7.783819999999999e+02 +1.180000000000000e+02
+1.292770000000000e+03 +1.465000000000000e+02
+8.768800000000000e+02 +1.085000000000000e+02
+1.306070000000000e+03 +1.500000000000000e+02
+1.168290000000000e+03 +2.740000000000000e+02
+9.909610000000000e+02 +2.150000000000000e+02
+9.060830000000000e+02 +1.150000000000000e+02
+6.718310000000000e+02 +2.745000000000000e+02
+1.032230000000000e+03 +1.845000000000000e+02
+6.574720000000000e+02 +1.140000000000000e+02
+1.575010000000000e+03 +2.270000000000000e+02
+1.066830000000000e+03 +1.950000000000000e+02
+3.454010000000000e+02 +2.750000000000000e+01
+2.758740000000000e+02 +8.500000000000000e+01
+6.342180000000002e+02 +8.400000000000000e+01
+1.078810000000000e+03 +1.900000000000000e+02
+7.871480000000000e+02 +1.175000000000000e+02
+6.565239999999999e+02 +7.700000000000000e+01
+1.548140000000000e+03 +2.230000000000000e+02
+1.870990000000000e+03 +4.040000000000000e+02
+2.383640000000000e+02 +7.395000000000000e+02
+9.072130000000000e+02 +1.060000000000000e+02
+1.400770000000000e+03 +3.355000000000000e+02
+8.460219999999998e+02 +2.045000000000000e+02
+6.873610000000001e+02 +1.570000000000000e+02
+1.079840000000000e+03 +1.955000000000000e+02
+7.687790000000000e+02 +1.140000000000000e+02
+9.855930000000000e+02 +1.150000000000000e+02
+1.869570000000000e+03 +4.685000000000000e+02
+9.863740000000000e+02 +2.140000000000000e+02
+1.142240000000000e+03 +2.470000000000000e+02
+2.343850000000000e+03 +6.785000000000000e+02
+6.592880000000000e+02 +1.150000000000000e+02
+9.718690000000000e+02 +1.120000000000000e+02
+1.739660000000000e+03 +3.500000000000000e+02
+9.058440000000001e+02 +6.950000000000000e+01
+1.307190000000000e+03 +1.465000000000000e+02
+4.921650000000000e+02 +1.330000000000000e+02
+1.129160000000000e+03 +2.490000000000000e+02
+1.295810000000000e+03 +2.440000000000000e+02
+7.801660000000001e+02 +1.155000000000000e+02
+1.985380000000000e+03 +3.175000000000000e+02
+9.126310000000000e+02 +1.010000000000000e+02
+1.872910000000000e+03 +3.615000000000000e+02
+6.894650000000000e+02 +1.465000000000000e+02
+7.704960000000002e+02 +1.130000000000000e+02
+1.567530000000000e+03 +2.250000000000000e+02
+9.501730000000000e+02 +2.410000000000000e+02
+1.304780000000000e+03 +2.715000000000000e+02
+6.198830000000000e+02 +1.020000000000000e+02
+7.865480000000000e+02 +1.170000000000000e+02
+9.948630000000001e+02 +1.140000000000000e+02
+1.327350000000000e+03 +2.955000000000000e+02
+1.302190000000000e+03 +1.525000000000000e+02
+1.331270000000000e+03 +3.065000000000000e+02
+9.298600000000000e+02 +1.010000000000000e+02
+7.910039999999998e+02 +1.070000000000000e+02
+9.558530000000000e+02 +2.360000000000000e+02
+7.787930000000000e+02 +1.150000000000000e+02
+1.720230000000000e+03 +3.050000000000000e+02
+6.911080000000002e+02 +1.700000000000000e+02
+2.678710000000000e+03 +4.440000000000000e+02
+7.869180000000000e+02 +1.165000000000000e+02
+1.581940000000000e+03 +2.180000000000000e+02
+1.002240000000000e+03 +2.110000000000000e+02
+1.858720000000000e+03 +3.415000000000000e+02
+6.308350000000000e+02 +9.850000000000000e+01
+1.002480000000000e+03 +1.155000000000000e+02
+9.197619999999999e+02 +1.670000000000000e+02
+1.609050000000000e+03 +2.380000000000000e+02
+4.836950000000000e+02 +1.310000000000000e+02
+6.374430000000000e+02 +6.150000000000000e+01
+1.158370000000000e+03 +2.550000000000000e+02
+1.572840000000000e+03 +3.010000000000000e+02
+1.109500000000000e+03 +1.620000000000000e+02
+1.295450000000000e+03 +3.235000000000000e+02
+8.714050000000000e+02 +7.320000000000000e+02
+1.754730000000000e+03 +4.635000000000000e+02
+1.223680000000000e+03 +2.010000000000000e+02
+1.064160000000000e+03 +1.900000000000000e+02
+7.428020000000000e+02 +1.195000000000000e+02
+1.145970000000000e+03 +2.490000000000000e+02
+6.899930000000001e+02 +1.520000000000000e+02
+7.739610000000000e+02 +1.130000000000000e+02
+8.920740000000000e+02 +1.085000000000000e+02
+1.247170000000000e+03 +2.090000000000000e+02
+8.589240000000000e+02 +2.035000000000000e+02
+1.546190000000000e+03 +2.200000000000000e+02
+6.691230000000000e+02 +2.110000000000000e+02
+1.308370000000000e+03 +2.270000000000000e+02
+1.055770000000000e+03 +1.720000000000000e+02
+1.550790000000000e+03 +2.170000000000000e+02
+1.855040000000000e+03 +3.655000000000000e+02
+6.848190000000000e+02 +1.505000000000000e+02
+6.490590000000000e+02 +2.125000000000000e+02
+1.394060000000000e+03 +2.630000000000000e+02
+6.342780000000000e+02 +6.150000000000000e+01
+1.853670000000000e+03 +3.405000000000000e+02
+9.962300000000000e+02 +2.175000000000000e+02
+1.282220000000000e+03 +2.765000000000000e+02
+6.250790000000002e+02 +9.550000000000000e+01
+1.863020000000000e+03 +4.550000000000000e+02
+1.156610000000000e+03 +2.520000000000000e+02
+1.767210000000000e+03 +2.545000000000000e+02
+8.882030000000000e+02 +1.095000000000000e+02
+6.733140000000000e+02 +2.645000000000000e+02
+4.821910000000000e+02 +1.185000000000000e+02
+7.727970000000000e+02 +1.965000000000000e+02
+8.755119999999999e+02 +7.265000000000000e+02
+7.662819999999998e+02 +1.085000000000000e+02
+1.302110000000000e+03 +1.370000000000000e+02
+1.140040000000000e+03 +2.815000000000000e+02
+4.744540000000000e+02 +7.850000000000000e+01
+1.460170000000000e+03 +5.520000000000000e+02
+1.304000000000000e+03 +2.410000000000000e+02
+4.856240000000000e+02 +1.300000000000000e+02
+1.072860000000000e+03 +1.505000000000000e+02
+9.473460000000000e+02 +2.650000000000000e+02
+8.580560000000000e+02 +1.160000000000000e+02
+6.417430000000001e+02 +1.575000000000000e+02
+7.641880000000000e+02 +1.320000000000000e+02
+1.529990000000000e+03 +3.125000000000000e+02
+7.834580000000002e+02 +1.120000000000000e+02
+7.870570000000000e+02 +1.140000000000000e+02
+1.054010000000000e+03 +2.650000000000000e+02
+7.823919999999998e+02 +7.240000000000000e+02
+2.343770000000000e+03 +4.345000000000000e+02
+7.698839999999999e+02 +1.020000000000000e+02
+1.337430000000000e+03 +3.650000000000000e+02
+1.341110000000000e+03 +3.195000000000000e+02
+5.044190000000000e+02 +1.245000000000000e+02
+9.784970000000000e+02 +1.240000000000000e+02
+6.378680000000001e+02 +1.590000000000000e+02
+7.351760000000000e+02 +1.380000000000000e+02
+1.294770000000000e+03 +2.230000000000000e+02
+1.290030000000000e+03 +3.645000000000000e+02
+6.540520000000000e+02 +2.640000000000000e+02
+4.921630000000000e+02 +1.315000000000000e+02
+1.089440000000000e+03 +2.595000000000000e+02
+9.118110000000000e+02 +9.050000000000000e+01
+9.904180000000000e+02 +2.015000000000000e+02
+6.740210000000002e+02 +1.485000000000000e+02
+2.300310000000000e+02 +7.220000000000000e+02
+7.488110000000000e+02 +1.065000000000000e+02
+1.756150000000000e+03 +3.370000000000000e+02
+4.776780000000001e+02 +7.600000000000000e+01
+1.298900000000000e+03 +3.205000000000000e+02
+1.066960000000000e+02 +1.700000000000000e+01
+7.743090000000000e+02 +1.045000000000000e+02
+9.836130000000001e+02 +1.150000000000000e+02
+9.430870000000000e+02 +9.050000000000000e+01
+6.588739999999998e+02 +1.580000000000000e+02
+8.845410000000001e+02 +7.210000000000000e+02
+1.871110000000000e+03 +3.355000000000000e+02
+6.200419999999998e+02 +8.900000000000000e+01
+1.316380000000000e+03 +2.185000000000000e+02
+9.492590000000000e+02 +2.585000000000000e+02
+9.046210000000000e+02 +1.950000000000000e+02
+7.843730000000000e+02 +1.055000000000000e+02
+1.829190000000000e+03 +3.485000000000000e+02
+1.591560000000000e+03 +4.380000000000000e+02
+7.926870000000000e+02 +1.025000000000000e+02
+1.810230000000000e+03 +7.200000000000000e+02
+2.176260000000000e+02 +7.200000000000000e+02
+1.939680000000000e+03 +3.630000000000000e+02
+1.911930000000000e+03 +3.115000000000000e+02
+1.322270000000000e+03 +2.075000000000000e+02
+1.042180000000000e+03 +1.765000000000000e+02
+1.951580000000000e+03 +3.710000000000000e+02
+1.924320000000000e+03 +3.490000000000000e+02
+2.129750000000000e+03 +4.185000000000000e+02
+9.344230000000000e+02 +7.200000000000000e+02
+2.535750000000000e+02 +7.195000000000000e+02
+1.322590000000000e+03 +4.310000000000000e+02
+4.913140000000000e+02 +1.260000000000000e+02
+1.673960000000000e+03 +3.070000000000000e+02
+6.962769999999998e+02 +1.430000000000000e+02
+1.853480000000000e+03 +3.265000000000000e+02
+1.148560000000000e+03 +2.730000000000000e+02
+7.423310000000000e+02 +1.070000000000000e+02
+8.913860000000002e+02 +9.900000000000000e+01
+6.418690000000000e+02 +1.470000000000000e+02
+2.278390000000000e+02 +7.180000000000000e+02
+1.314490000000000e+03 +2.300000000000000e+02
+1.380490000000000e+03 +4.425000000000000e+02
+1.951830000000000e+03 +5.325000000000000e+02
+8.568800000000000e+02 +1.915000000000000e+02
+1.858200000000000e+03 +3.580000000000000e+02
+1.607310000000000e+03 +3.445000000000000e+02
+1.166740000000000e+03 +2.745000000000000e+02
+1.205220000000000e+03 +2.685000000000000e+02
+6.537600000000000e+02 +1.455000000000000e+02
+1.952100000000000e+03 +2.925000000000000e+02
+1.060180000000000e+03 +1.740000000000000e+02
+6.403840000000000e+02 +1.500000000000000e+02
+1.794320000000000e+03 +4.195000000000000e+02
+2.312590000000000e+03 +4.065000000000000e+02
+9.737940000000000e+02 +1.200000000000000e+02
+1.578740000000000e+03 +2.010000000000000e+02
+1.768270000000000e+03 +2.995000000000000e+02
+1.284530000000000e+03 +3.160000000000000e+02
+1.428140000000000e+03 +2.095000000000000e+02
+6.249600000000000e+02 +8.450000000000000e+01
+1.307050000000000e+03 +1.300000000000000e+02
+1.256600000000000e+03 +3.010000000000000e+02
+2.853860000000000e+02 +7.150000000000000e+02
+1.422650000000000e+03 +1.340000000000000e+02
+8.590520000000000e+02 +7.145000000000000e+02
+1.856990000000000e+03 +3.985000000000000e+02
+9.643740000000000e+02 +1.160000000000000e+02
+8.985210000000002e+02 +1.000000000000000e+02
+9.892100000000000e+02 +1.895000000000000e+02
+1.380590000000000e+03 +2.005000000000000e+02
+7.409420000000000e+02 +1.020000000000000e+02
+7.764500000000000e+02 +7.135000000000000e+02
+1.368330000000000e+03 +2.860000000000000e+02
+8.969950000000000e+02 +1.385000000000000e+02
+1.065000000000000e+03 +1.785000000000000e+02
+9.325380000000000e+02 +7.130000000000000e+02
+2.348740000000000e+03 +5.155000000000000e+02
+6.956280000000000e+02 +1.405000000000000e+02
+1.114120000000000e+03 +3.670000000000000e+02
+6.473020000000000e+02 +2.515000000000000e+02
+4.950630000000001e+02 +1.235000000000000e+02
+1.433040000000000e+03 +3.100000000000000e+02
+4.887560000000000e+02 +1.245000000000000e+02
+1.323240000000000e+03 +2.465000000000000e+02
+9.000920000000000e+02 +1.335000000000000e+02
+9.897859999999999e+02 +1.830000000000000e+02
+6.411130000000001e+02 +1.445000000000000e+02
+8.636780000000000e+02 +1.280000000000000e+02
+1.004580000000000e+03 +1.840000000000000e+02
+1.570040000000000e+03 +1.830000000000000e+02
+9.840050000000000e+02 +1.860000000000000e+02
+2.818940000000000e+02 +6.250000000000000e+01
+1.428800000000000e+03 +4.295000000000000e+02
+1.003040000000000e+03 +1.890000000000000e+02
+6.350160000000000e+02 +1.470000000000000e+02
+9.492809999999999e+02 +2.225000000000000e+02
+1.552820000000000e+03 +1.795000000000000e+02
+1.116810000000000e+03 +1.405000000000000e+02
+9.256910000000000e+02 +1.355000000000000e+02
+4.831970000000000e+02 +1.230000000000000e+02
+8.258900000000000e+02 +7.085000000000000e+02
+1.878480000000000e+03 +4.100000000000000e+02
+7.711820000000000e+02 +1.000000000000000e+02
+7.806500000000000e+02 +1.005000000000000e+02
+2.027390000000000e+02 +7.075000000000000e+02
+6.250409999999998e+02 +7.850000000000000e+01
+7.643830000000000e+02 +9.350000000000000e+01
+1.068750000000000e+03 +1.740000000000000e+02
+2.691470000000000e+03 +5.930000000000000e+02
+7.890219999999998e+02 +1.025000000000000e+02
+1.296830000000000e+03 +2.235000000000000e+02
+1.390050000000000e+03 +3.460000000000000e+02
+8.981940000000000e+02 +1.305000000000000e+02
+1.417380000000000e+03 +3.060000000000000e+02
+1.472380000000000e+03 +4.185000000000000e+02
+9.911720000000000e+02 +1.910000000000000e+02
+1.057360000000000e+03 +1.795000000000000e+02
+1.869260000000000e+03 +3.930000000000000e+02
+2.206570000000000e+02 +7.055000000000000e+02
+8.625380000000000e+02 +1.055000000000000e+02
+6.566200000000000e+02 +1.445000000000000e+02
+1.292250000000000e+03 +2.135000000000000e+02
+1.052980000000000e+03 +2.785000000000000e+02
+6.827280000000002e+02 +2.415000000000000e+02
+9.887809999999999e+02 +1.345000000000000e+02
+1.659120000000000e+03 +2.705000000000000e+02
+7.571039999999998e+02 +1.480000000000000e+02
+7.891080000000002e+02 +1.015000000000000e+02
+8.486860000000000e+02 +2.860000000000000e+02
+1.859430000000000e+03 +3.700000000000000e+02
+7.663350000000000e+02 +7.035000000000000e+02
+8.696790000000000e+02 +7.035000000000000e+02
+1.876780000000000e+03 +3.425000000000000e+02
+4.824150000000000e+02 +8.550000000000000e+01
+1.861310000000000e+03 +4.325000000000000e+02
+9.144220000000000e+02 +1.300000000000000e+02
+1.961400000000000e+03 +3.820000000000000e+02
+9.310480000000000e+02 +1.435000000000000e+02
+7.670730000000000e+02 +1.185000000000000e+02
+7.666900000000001e+02 +8.500000000000000e+01
+1.364010000000000e+03 +3.410000000000000e+02
+1.946930000000000e+03 +5.190000000000000e+02
+2.031840000000000e+03 +3.645000000000000e+02
+4.892110000000000e+02 +1.110000000000000e+02
+6.751780000000000e+02 +1.280000000000000e+02
+1.081430000000000e+02 +2.550000000000000e+01
+6.892160000000000e+02 +1.570000000000000e+02
+7.437689999999999e+02 +9.100000000000000e+01
+1.058410000000000e+03 +1.735000000000000e+02
+1.097680000000000e+02 +2.250000000000000e+01
+1.289860000000000e+03 +4.085000000000000e+02
+5.872760000000000e+02 +8.800000000000000e+01
+6.704030000000000e+02 +1.500000000000000e+02
+1.154970000000000e+03 +2.585000000000000e+02
+8.823630000000001e+02 +1.270000000000000e+02
+1.388970000000000e+03 +4.145000000000000e+02
+8.461439999999999e+02 +6.990000000000000e+02
+7.762270000000000e+02 +8.900000000000000e+01
+8.930590000000000e+02 +3.900000000000000e+02
+1.573580000000000e+03 +2.720000000000000e+02
+1.382260000000000e+03 +3.430000000000000e+02
+9.190430000000000e+02 +1.355000000000000e+02
+1.129110000000000e+03 +1.690000000000000e+02
+1.065020000000000e+03 +1.600000000000000e+02
+9.315470000000000e+02 +1.325000000000000e+02
+1.024910000000000e+03 +2.665000000000000e+02
+2.765450000000000e+02 +5.200000000000000e+01
+7.883420000000000e+02 +8.950000000000000e+01
+1.299680000000000e+03 +2.110000000000000e+02
+1.044610000000000e+03 +1.305000000000000e+02
+1.873970000000000e+03 +3.270000000000000e+02
+8.547719999999998e+02 +6.965000000000000e+02
+1.893110000000000e+02 +6.965000000000000e+02
+7.726110000000001e+02 +9.500000000000000e+01
+9.787160000000000e+02 +1.625000000000000e+02
+9.155860000000000e+02 +1.725000000000000e+02
+8.673130000000000e+02 +2.190000000000000e+02
+6.142250000000000e+02 +2.425000000000000e+02
+1.079810000000000e+03 +1.580000000000000e+02
+2.229350000000000e+02 +6.960000000000000e+02
+1.188650000000000e+03 +2.640000000000000e+02
+1.342160000000000e+03 +6.960000000000000e+02
+8.913220000000000e+02 +1.500000000000000e+02
+7.495210000000002e+02 +8.900000000000000e+01
+7.918680000000001e+02 +9.350000000000000e+01
+7.883739999999998e+02 +1.655000000000000e+02
+7.538910000000002e+02 +1.425000000000000e+02
+8.763950000000000e+02 +6.950000000000000e+02
+1.948270000000000e+02 +6.950000000000000e+02
+7.469810000000001e+02 +8.500000000000000e+01
+1.352520000000000e+03 +3.415000000000000e+02
+1.077270000000000e+03 +3.770000000000000e+02
+1.137360000000000e+03 +3.360000000000000e+02
+1.846960000000000e+03 +4.280000000000000e+02
+1.010650000000000e+03 +1.715000000000000e+02
+1.744390000000000e+03 +3.745000000000000e+02
+1.426280000000000e+03 +1.865000000000000e+02
+9.915160000000000e+02 +1.515000000000000e+02
+1.380020000000000e+03 +2.950000000000000e+02
+1.594960000000000e+03 +2.765000000000000e+02
+1.421890000000000e+03 +4.155000000000000e+02
+2.729040000000000e+02 +1.935000000000000e+02
+6.532170000000000e+02 +1.280000000000000e+02
+1.298420000000000e+03 +2.145000000000000e+02
+9.071810000000000e+02 +1.335000000000000e+02
+1.872110000000000e+03 +5.395000000000000e+02
+6.908910000000002e+02 +1.225000000000000e+02
+6.969939999999998e+02 +1.380000000000000e+02
+1.071380000000000e+03 +1.535000000000000e+02
+7.812890000000000e+02 +8.700000000000000e+01
+1.556920000000000e+03 +2.620000000000000e+02
+2.512230000000000e+02 +6.915000000000000e+02
+6.424590000000002e+02 +1.325000000000000e+02
+7.899480000000000e+02 +9.150000000000000e+01
+9.963650000000000e+02 +1.640000000000000e+02
+1.554410000000000e+03 +2.585000000000000e+02
+6.560269999999998e+02 +1.260000000000000e+02
+2.666070000000000e+03 +6.560000000000000e+02
+1.376510000000000e+03 +2.840000000000000e+02
+7.756039999999998e+02 +6.890000000000000e+02
+7.581680000000000e+02 +8.250000000000000e+01
+1.298950000000000e+03 +2.120000000000000e+02
+1.064640000000000e+03 +1.275000000000000e+02
+6.920139999999999e+02 +2.145000000000000e+02
+1.983900000000000e+03 +2.740000000000000e+02
+1.010940000000000e+03 +2.775000000000000e+02
+6.530840000000002e+02 +1.315000000000000e+02
+1.866310000000000e+03 +3.050000000000000e+02
+1.564710000000000e+03 +2.540000000000000e+02
+2.287250000000000e+03 +4.345000000000000e+02
+1.868530000000000e+03 +3.190000000000000e+02
+6.881770000000000e+02 +1.170000000000000e+02
+1.130730000000000e+03 +2.750000000000000e+02
+1.489400000000000e+03 +6.460000000000000e+02
+6.935360000000002e+02 +7.250000000000000e+01
+8.976020000000000e+02 +1.215000000000000e+02
+6.724889999999998e+02 +3.370000000000000e+02
+1.479120000000000e+03 +1.925000000000000e+02
+8.606489999999999e+02 +9.150000000000000e+01
+6.542080000000002e+02 +1.265000000000000e+02
+7.899720000000000e+02 +1.555000000000000e+02
+8.975570000000000e+02 +1.990000000000000e+02
+1.463360000000000e+03 +2.960000000000000e+02
+7.790950000000000e+02 +8.250000000000000e+01
+9.883260000000000e+02 +1.600000000000000e+02
+6.078450000000000e+02 +7.350000000000000e+01
+7.879860000000001e+02 +8.300000000000000e+01
+8.990930000000002e+02 +1.280000000000000e+02
+4.902260000000000e+02 +1.000000000000000e+02
+8.252089999999999e+02 +6.845000000000000e+02
+7.823660000000001e+02 +8.750000000000000e+01
+1.483570000000000e+03 +1.890000000000000e+02
+9.753140000000000e+02 +1.000000000000000e+02
+1.755180000000000e+03 +2.975000000000000e+02
+1.075070000000000e+03 +3.600000000000000e+02
+6.602139999999998e+02 +1.265000000000000e+02
+6.755839999999999e+02 +1.130000000000000e+02
+1.141150000000000e+03 +2.400000000000000e+02
+2.778890000000000e+02 +3.950000000000000e+01
+1.780310000000000e+03 +2.700000000000000e+02
+1.405830000000000e+03 +4.095000000000000e+02
+8.838439999999998e+02 +3.690000000000000e+02
+1.000090000000000e+03 +2.730000000000000e+02
+7.574180000000000e+02 +9.300000000000000e+01
+3.261000000000000e+02 +6.000000000000000e+01
+8.963900000000000e+02 +1.145000000000000e+02
+1.127560000000000e+03 +2.075000000000000e+02
+1.942590000000000e+02 +6.815000000000000e+02
+7.832210000000000e+02 +8.250000000000000e+01
+9.002040000000000e+02 +1.270000000000000e+02
+1.410900000000000e+03 +5.180000000000000e+02
+6.175730000000000e+02 +7.150000000000000e+01
+1.006620000000000e+03 +1.650000000000000e+02
+8.599650000000000e+02 +8.800000000000000e+01
+7.773660000000001e+02 +8.550000000000000e+01
+6.056180000000001e+02 +1.545000000000000e+02
+1.311920000000000e+03 +2.080000000000000e+02
+1.334140000000000e+03 +3.305000000000000e+02
+6.911810000000000e+02 +6.600000000000000e+01
+7.879900000000000e+02 +8.300000000000000e+01
+1.213960000000000e+03 +2.580000000000000e+02
+6.618450000000000e+02 +3.360000000000000e+02
+1.072660000000000e+03 +1.440000000000000e+02
+1.002330000000000e+03 +1.425000000000000e+02
+1.913290000000000e+02 +6.790000000000000e+02
+1.450500000000000e+03 +1.850000000000000e+02
+1.078610000000000e+03 +1.470000000000000e+02
+1.291910000000000e+03 +1.905000000000000e+02
+1.001780000000000e+03 +1.550000000000000e+02
+1.093440000000000e+03 +1.415000000000000e+02
+6.844349999999999e+02 +6.200000000000000e+01
+7.841410000000002e+02 +8.450000000000000e+01
+8.965889999999998e+02 +1.250000000000000e+02
+1.872180000000000e+03 +3.185000000000000e+02
+1.307770000000000e+03 +2.180000000000000e+02
+7.850630000000000e+02 +8.450000000000000e+01
+1.313300000000000e+03 +1.985000000000000e+02
+9.467650000000000e+02 +2.215000000000000e+02
+1.007320000000000e+03 +2.685000000000000e+02
+8.772089999999999e+02 +8.950000000000000e+01
+7.442189999999998e+02 +7.700000000000000e+01
+9.921350000000000e+02 +1.510000000000000e+02
+8.962300000000000e+02 +1.095000000000000e+02
+1.859540000000000e+03 +5.250000000000000e+02
+1.001310000000000e+03 +2.230000000000000e+02
+9.018740000000000e+02 +1.165000000000000e+02
+1.048480000000000e+03 +1.490000000000000e+02
+6.746720000000000e+02 +1.380000000000000e+02
+9.371600000000000e+02 +2.210000000000000e+02
+4.662390000000000e+02 +8.450000000000000e+01
+7.925200000000000e+02 +8.500000000000000e+01
+6.611170000000000e+02 +3.345000000000000e+02
+4.853040000000000e+02 +9.450000000000000e+01
+6.373770000000000e+02 +1.195000000000000e+02
+1.885600000000000e+02 +6.740000000000000e+02
+7.608350000000000e+02 +8.150000000000000e+01
+1.952480000000000e+03 +2.520000000000000e+02
+9.156340000000000e+02 +1.115000000000000e+02
+6.853190000000000e+02 +1.450000000000000e+02
+1.086950000000000e+03 +1.430000000000000e+02
+2.281470000000000e+03 +4.190000000000000e+02
+4.934820000000000e+02 +8.900000000000000e+01
+7.832919999999998e+02 +8.150000000000000e+01
+9.545400000000000e+02 +6.725000000000000e+02
+9.893950000000000e+02 +6.720000000000000e+02
+1.066960000000000e+03 +1.010000000000000e+02
+6.544180000000000e+02 +3.305000000000000e+02
+6.539480000000000e+02 +1.175000000000000e+02
+1.306100000000000e+03 +2.100000000000000e+02
+1.294790000000000e+03 +5.060000000000000e+02
+9.045300000000000e+02 +1.205000000000000e+02
+7.505480000000000e+02 +6.705000000000000e+02
+1.308810000000000e+03 +1.715000000000000e+02
+7.805260000000002e+02 +8.350000000000000e+01
+1.300990000000000e+03 +1.930000000000000e+02
+1.580690000000000e+03 +2.445000000000000e+02
+1.252120000000000e+03 +3.845000000000000e+02
+9.254770000000000e+02 +2.195000000000000e+02
+8.672769999999998e+02 +9.200000000000000e+01
+6.894530000000000e+02 +6.050000000000000e+01
+9.621150000000000e+02 +1.370000000000000e+02
+6.394670000000000e+02 +1.165000000000000e+02
+7.882840000000000e+02 +8.000000000000000e+01
+6.921740000000000e+02 +1.370000000000000e+02
+1.013510000000000e+03 +1.325000000000000e+02
+6.731550000000000e+02 +1.000000000000000e+02
+1.304060000000000e+03 +1.705000000000000e+02
+1.301640000000000e+03 +4.975000000000000e+02
+8.614320000000000e+02 +1.335000000000000e+02
+1.580840000000000e+03 +2.450000000000000e+02
+1.143150000000000e+03 +3.770000000000000e+02
+1.061780000000000e+03 +4.745000000000000e+02
+6.940010000000002e+02 +9.750000000000000e+01
+6.908420000000000e+02 +6.800000000000000e+01
+1.319990000000000e+03 +2.035000000000000e+02
+2.410700000000000e+03 +5.740000000000000e+02
+1.287430000000000e+03 +2.710000000000000e+02
+6.527390000000000e+02 +1.145000000000000e+02
+9.250010000000000e+02 +6.665000000000000e+02
+7.891890000000000e+02 +8.200000000000000e+01
+4.506030000000000e+02 +3.450000000000000e+01
+6.955450000000000e+02 +2.300000000000000e+02
+6.363670000000000e+02 +1.035000000000000e+02
+7.461790000000000e+02 +7.500000000000000e+01
+1.586420000000000e+03 +3.600000000000000e+02
+9.386340000000000e+02 +2.190000000000000e+02
+1.304170000000000e+03 +2.585000000000000e+02
+6.950300000000000e+02 +9.800000000000000e+01
+6.245580000000000e+02 +1.445000000000000e+02
+4.635030000000000e+02 +3.550000000000000e+01
+1.051470000000000e+03 +3.400000000000000e+02
+7.658160000000000e+02 +6.650000000000000e+02
+8.996139999999998e+02 +1.150000000000000e+02
+1.277450000000000e+03 +6.650000000000000e+02
+8.586239999999998e+02 +1.325000000000000e+02
+1.107190000000000e+03 +4.455000000000000e+02
+8.700640000000000e+02 +1.360000000000000e+02
+8.590989999999998e+02 +1.315000000000000e+02
+6.498650000000000e+02 +1.040000000000000e+02
+8.576950000000001e+02 +8.200000000000000e+01
+7.769540000000000e+02 +7.750000000000000e+01
+9.328770000000000e+02 +1.445000000000000e+02
+8.937970000000000e+02 +2.115000000000000e+02
+1.000580000000000e+03 +1.080000000000000e+02
+1.368860000000000e+03 +3.800000000000000e+02
+7.757460000000002e+02 +6.630000000000000e+02
+9.832859999999999e+02 +1.075000000000000e+02
+1.013910000000000e+03 +2.520000000000000e+02
+1.077250000000000e+03 +1.335000000000000e+02
+7.447589999999999e+02 +5.900000000000000e+01
+1.037960000000000e+03 +1.940000000000000e+02
+7.810250000000000e+02 +7.850000000000000e+01
+1.311070000000000e+03 +1.865000000000000e+02
+1.208980000000000e+03 +1.290000000000000e+02
+7.530700000000001e+02 +7.650000000000000e+01
+8.835650000000001e+02 +1.675000000000000e+02
+8.316070000000000e+02 +6.615000000000000e+02
+1.331370000000000e+03 +3.050000000000000e+02
+4.932870000000000e+02 +7.900000000000000e+01
+6.576750000000000e+02 +1.000000000000000e+02
+1.805470000000000e+02 +6.610000000000000e+02
+7.772919999999998e+02 +7.850000000000000e+01
+1.522610000000000e+03 +2.345000000000000e+02
+1.155910000000000e+03 +3.225000000000000e+02
+4.855390000000000e+02 +1.710000000000000e+02
+8.116210000000002e+02 +6.605000000000000e+02
+1.563710000000000e+03 +2.330000000000000e+02
+9.072220000000000e+02 +1.135000000000000e+02
+4.960260000000000e+02 +1.770000000000000e+02
+1.699170000000000e+03 +2.315000000000000e+02
+1.455740000000000e+03 +3.430000000000000e+02
+8.761360000000002e+02 +8.800000000000000e+01
+9.342500000000000e+02 +1.450000000000000e+02
+1.918550000000000e+03 +4.290000000000000e+02
+1.910920000000000e+03 +2.970000000000000e+02
+1.809360000000000e+03 +6.600000000000000e+02
+6.767330000000002e+02 +1.055000000000000e+02
+1.004730000000000e+03 +2.550000000000000e+02
+1.078030000000000e+03 +2.345000000000000e+02
+1.556710000000000e+03 +2.355000000000000e+02
+6.555160000000002e+02 +3.120000000000000e+02
+1.016710000000000e+03 +2.045000000000000e+02
+1.140280000000000e+03 +2.940000000000000e+02
+7.868730000000000e+02 +7.800000000000000e+01
+1.316830000000000e+03 +1.920000000000000e+02
+1.778060000000000e+03 +2.420000000000000e+02
+4.800520000000000e+02 +5.900000000000000e+01
+6.237869999999998e+02 +1.365000000000000e+02
+8.702700000000000e+02 +1.105000000000000e+02
+2.409430000000000e+03 +5.700000000000000e+02
+7.749280000000000e+02 +6.575000000000000e+02
+1.157790000000000e+03 +2.135000000000000e+02
+7.895239999999999e+02 +1.515000000000000e+02
+6.128580000000002e+02 +5.150000000000000e+01
+1.748120000000000e+02 +6.570000000000000e+02
+7.977339999999998e+02 +1.525000000000000e+02
+7.819019999999998e+02 +7.150000000000000e+01
+9.380780000000000e+02 +6.560000000000000e+02
+7.949130000000000e+02 +7.550000000000000e+01
+9.936510000000000e+02 +1.330000000000000e+02
+1.448840000000000e+03 +1.650000000000000e+02
+1.687940000000000e+03 +4.045000000000000e+02
+6.455440000000000e+02 +1.075000000000000e+02
+1.073310000000000e+03 +1.995000000000000e+02
+8.583170000000000e+02 +8.650000000000000e+01
+1.555700000000000e+03 +2.255000000000000e+02
+7.061920000000000e+02 +1.855000000000000e+02
+6.672660000000002e+02 +1.050000000000000e+02
+1.083160000000000e+03 +1.255000000000000e+02
+2.782850000000000e+02 +2.050000000000000e+01
+9.237490000000000e+02 +6.545000000000000e+02
+7.602089999999999e+02 +7.050000000000000e+01
+8.780860000000000e+02 +1.200000000000000e+02
+6.598989999999999e+02 +1.005000000000000e+02
+7.911750000000000e+02 +6.540000000000000e+02
+6.243000000000000e+02 +1.345000000000000e+02
+1.102400000000000e+03 +1.730000000000000e+02
+7.903700000000000e+02 +6.540000000000000e+02
+4.945090000000000e+02 +1.625000000000000e+02
+1.314010000000000e+03 +3.735000000000000e+02
+6.350040000000000e+02 +1.035000000000000e+02
+7.910770000000000e+02 +1.500000000000000e+02
+1.046220000000000e+03 +1.750000000000000e+02
+1.851130000000000e+03 +3.960000000000000e+02
+7.589169999999998e+02 +6.650000000000000e+01
+9.067430000000001e+02 +1.355000000000000e+02
+8.738510000000001e+02 +3.455000000000000e+02
+6.999980000000000e+02 +2.110000000000000e+02
+9.000450000000000e+02 +1.140000000000000e+02
+2.853980000000000e+02 +2.150000000000000e+01
+7.369260000000000e+02 +7.150000000000000e+01
+8.939689999999998e+02 +2.740000000000000e+02
+6.899240000000000e+02 +1.795000000000000e+02
+6.861330000000000e+02 +6.100000000000000e+01
+7.690780000000000e+02 +6.520000000000000e+02
+1.864200000000000e+03 +4.010000000000000e+02
+9.808400000000000e+02 +1.950000000000000e+02
+6.680100000000000e+02 +1.035000000000000e+02
+7.919720000000000e+02 +1.450000000000000e+02
+1.215620000000000e+03 +2.620000000000000e+02
+5.018070000000000e+02 +1.705000000000000e+02
+8.833810000000002e+02 +1.630000000000000e+02
+1.041330000000000e+02 +1.150000000000000e+01
+8.825889999999998e+02 +1.090000000000000e+02
+9.195330000000000e+02 +6.505000000000000e+02
+1.316490000000000e+03 +1.685000000000000e+02
+1.308690000000000e+03 +4.815000000000000e+02
+9.969400000000001e+02 +2.090000000000000e+02
+6.586550000000000e+02 +1.370000000000000e+02
+9.755660000000000e+02 +1.325000000000000e+02
+8.939760000000001e+02 +1.795000000000000e+02
+8.996740000000000e+02 +2.710000000000000e+02
+9.547770000000000e+02 +1.220000000000000e+02
+1.740960000000000e+02 +6.495000000000000e+02
+9.056750000000000e+02 +1.860000000000000e+02
+6.688839999999999e+02 +1.030000000000000e+02
+8.490989999999998e+02 +7.400000000000000e+01
+1.055850000000000e+03 +3.120000000000000e+02
+1.863350000000000e+03 +3.360000000000000e+02
+6.190180000000000e+02 +1.315000000000000e+02
+2.340170000000000e+02 +6.485000000000000e+02
+7.756310000000002e+02 +6.900000000000000e+01
+9.390560000000000e+02 +1.225000000000000e+02
+1.405000000000000e+03 +4.870000000000000e+02
+6.515130000000000e+02 +9.950000000000000e+01
+7.607230000000002e+02 +6.480000000000000e+02
+7.931519999999998e+02 +1.450000000000000e+02
+1.066370000000000e+03 +1.605000000000000e+02
+8.859800000000000e+02 +1.085000000000000e+02
+6.746230000000000e+02 +1.105000000000000e+02
+8.931200000000000e+02 +1.095000000000000e+02
+1.007290000000000e+03 +1.970000000000000e+02
+1.092690000000000e+03 +2.245000000000000e+02
+7.862840000000000e+02 +1.435000000000000e+02
+1.054750000000000e+03 +5.470000000000000e+02
+1.080940000000000e+03 +1.115000000000000e+02
+9.093390000000001e+02 +6.465000000000000e+02
+3.936290000000000e+02 +3.050000000000000e+01
+6.907700000000000e+02 +1.440000000000000e+02
+1.574350000000000e+03 +3.165000000000000e+02
+9.166910000000000e+02 +1.590000000000000e+02
+6.295130000000000e+02 +4.215000000000000e+02
+8.824040000000000e+02 +1.560000000000000e+02
+1.055050000000000e+03 +1.595000000000000e+02
+1.416780000000000e+03 +4.860000000000000e+02
+7.781310000000002e+02 +1.390000000000000e+02
+1.697710000000000e+02 +6.455000000000000e+02
+7.899390000000000e+02 +1.430000000000000e+02
+9.775170000000001e+02 +1.265000000000000e+02
+1.108700000000000e+03 +1.615000000000000e+02
+1.773650000000000e+02 +6.450000000000000e+02
+1.006810000000000e+03 +1.590000000000000e+02
+9.007610000000000e+02 +1.125000000000000e+02
+1.058510000000000e+03 +1.200000000000000e+02
+9.186780000000000e+02 +6.445000000000000e+02
+1.874650000000000e+03 +3.335000000000000e+02
+6.197650000000000e+02 +1.245000000000000e+02
+7.610830000000002e+02 +1.375000000000000e+02
+6.478750000000000e+02 +9.950000000000000e+01
+9.377550000000000e+02 +6.440000000000000e+02
+9.765760000000000e+02 +1.285000000000000e+02
+9.109430000000000e+02 +1.700000000000000e+02
+6.538960000000000e+02 +9.700000000000000e+01
+1.295600000000000e+03 +1.705000000000000e+02
+9.257380000000001e+02 +1.565000000000000e+02
+6.708869999999999e+02 +8.550000000000000e+01
+1.006160000000000e+03 +2.280000000000000e+02
+2.297830000000000e+03 +5.135000000000000e+02
+8.697669999999998e+02 +8.600000000000000e+01
+6.550880000000002e+02 +2.970000000000000e+02
+9.988070000000000e+02 +1.625000000000000e+02
+7.882510000000002e+02 +1.415000000000000e+02
+1.463860000000000e+03 +2.675000000000000e+02
+7.477210000000000e+02 +5.350000000000000e+01
+7.525820000000000e+02 +6.300000000000000e+01
+1.404380000000000e+03 +4.760000000000000e+02
+6.488850000000000e+02 +4.145000000000000e+02
+1.584330000000000e+03 +2.615000000000000e+02
+6.598730000000000e+02 +9.850000000000000e+01
+6.844750000000000e+02 +1.045000000000000e+02
+4.821550000000000e+02 +1.010000000000000e+02
+1.081750000000000e+03 +2.170000000000000e+02
+1.083050000000000e+03 +1.865000000000000e+02
+6.248099999999999e+02 +1.250000000000000e+02
+1.660940000000001e+02 +6.405000000000000e+02
+1.311140000000000e+03 +1.695000000000000e+02
+1.404320000000000e+03 +5.965000000000000e+02
+4.941960000000000e+02 +1.535000000000000e+02
+1.396000000000000e+03 +1.560000000000000e+02
+6.949380000000000e+02 +1.050000000000000e+02
+1.531050000000000e+02 +6.400000000000000e+02
+8.022530000000000e+02 +1.310000000000000e+02
+2.824030000000000e+02 +2.100000000000000e+01
+1.066500000000000e+03 +1.955000000000000e+02
+8.717810000000002e+02 +7.350000000000000e+01
+1.281710000000000e+03 +3.165000000000000e+02
+1.847380000000000e+03 +2.535000000000000e+02
+9.130370000000000e+02 +1.615000000000000e+02
+4.595590000000000e+02 +6.385000000000000e+02
+1.284930000000000e+03 +3.605000000000000e+02
+6.628120000000000e+02 +9.750000000000000e+01
+1.960140000000000e+03 +3.355000000000000e+02
+1.063990000000000e+03 +5.425000000000000e+02
+1.217760000000000e+03 +1.915000000000000e+02
+1.045450000000000e+03 +2.000000000000000e+02
+7.547790000000000e+02 +6.050000000000000e+01
+1.588790000000000e+03 +3.350000000000000e+02
+1.247180000000000e+03 +2.175000000000000e+02
+1.485350000000000e+03 +3.685000000000000e+02
+1.115710000000000e+03 +2.705000000000000e+02
+1.338120000000000e+03 +3.185000000000000e+02
+1.184070000000000e+03 +2.900000000000000e+02
+7.862460000000002e+02 +1.400000000000000e+02
+7.921890000000000e+02 +1.100000000000000e+02
+6.893450000000000e+02 +2.230000000000000e+02
+8.418270000000000e+02 +6.360000000000000e+02
+6.808510000000001e+02 +1.705000000000000e+02
+1.081100000000000e+03 +2.155000000000000e+02
+8.566039999999998e+02 +6.355000000000000e+02
+1.552740000000000e+02 +6.355000000000000e+02
+7.816799999999999e+02 +1.365000000000000e+02
+9.761110000000000e+02 +1.225000000000000e+02
+8.778900000000000e+02 +1.500000000000000e+02
+8.206350000000000e+02 +6.350000000000000e+02
+7.965230000000000e+02 +1.380000000000000e+02
+1.103780000000000e+03 +1.515000000000000e+02
+9.483920000000001e+02 +2.945000000000000e+02
+8.722260000000001e+02 +3.245000000000000e+02
+6.355459999999998e+02 +9.100000000000000e+01
+6.901130000000001e+02 +1.390000000000000e+02
+7.848800000000000e+02 +1.390000000000000e+02
+1.323290000000000e+03 +3.190000000000000e+02
+1.025060000000000e+03 +3.045000000000000e+02
+7.651030000000002e+02 +1.350000000000000e+02
+7.902360000000001e+02 +6.335000000000000e+02
+8.711080000000002e+02 +7.550000000000000e+01
+1.695870000000000e+03 +3.845000000000000e+02
+7.924989999999998e+02 +1.370000000000000e+02
+8.822569999999999e+02 +1.860000000000000e+02
+1.644070000000000e+03 +3.850000000000000e+02
+1.863390000000000e+03 +3.340000000000000e+02
+1.298440000000000e+03 +2.460000000000000e+02
+1.607190000000000e+03 +3.280000000000000e+02
+6.982439999999998e+02 +9.900000000000000e+01
+1.013390000000000e+03 +2.245000000000000e+02
+1.040350000000000e+03 +2.055000000000000e+02
+6.478390000000001e+02 +9.200000000000000e+01
+8.796750000000000e+02 +3.190000000000000e+02
+1.122000000000000e+03 +2.255000000000000e+02
+7.875150000000000e+02 +1.320000000000000e+02
+1.292180000000000e+03 +5.790000000000000e+02
+1.082230000000000e+03 +2.090000000000000e+02
+9.901330000000000e+02 +1.570000000000000e+02
+7.955110000000002e+02 +6.300000000000000e+02
+2.797730000000000e+02 +1.025000000000000e+02
+1.364190000000000e+03 +3.835000000000000e+02
+8.731489999999999e+02 +1.520000000000000e+02
+1.082940000000000e+03 +1.590000000000000e+02
+7.770520000000000e+02 +6.290000000000000e+02
+8.613389999999998e+02 +9.950000000000000e+01
+7.016960000000000e+02 +1.750000000000000e+02
+1.211050000000000e+03 +2.010000000000000e+02
+9.899600000000000e+02 +2.210000000000000e+02
+8.792410000000001e+02 +6.280000000000000e+02
+7.455889999999998e+02 +5.150000000000000e+01
+6.806799999999999e+02 +1.730000000000000e+02
+1.089550000000000e+03 +2.080000000000000e+02
+1.613290000000000e+03 +3.645000000000000e+02
+7.805960000000000e+02 +9.900000000000000e+01
+1.473860000000000e+03 +3.560000000000000e+02
+1.019910000000000e+03 +1.970000000000000e+02
+1.171510000000000e+03 +3.785000000000000e+02
+1.344600000000000e+03 +4.905000000000000e+02
+7.555350000000000e+02 +1.025000000000000e+02
+9.956319999999999e+02 +7.650000000000000e+01
+6.413810000000000e+02 +8.150000000000000e+01
+1.541690000000000e+02 +6.265000000000000e+02
+9.030549999999999e+02 +1.675000000000000e+02
+6.632689999999999e+02 +8.450000000000000e+01
+1.856410000000000e+03 +3.230000000000000e+02
+9.851420000000001e+02 +1.150000000000000e+02
+5.915369999999998e+02 +9.350000000000000e+01
+6.883439999999998e+02 +1.270000000000000e+02
+7.861319999999999e+02 +1.290000000000000e+02
+9.104610000000000e+02 +1.120000000000000e+02
+8.794180000000000e+02 +1.410000000000000e+02
+7.118539999999998e+02 +6.250000000000000e+02
+8.043739999999998e+02 +6.250000000000000e+02
+1.855000000000000e+03 +3.455000000000000e+02
+1.598670000000000e+03 +4.335000000000000e+02
+6.549470000000000e+02 +2.005000000000000e+02
+7.477339999999998e+02 +6.245000000000000e+02
+7.812960000000000e+02 +1.305000000000000e+02
+1.592860000000000e+03 +3.235000000000000e+02
+1.288520000000000e+03 +3.540000000000000e+02
+4.722520000000000e+02 +4.450000000000000e+01
+9.079520000000000e+02 +1.720000000000000e+02
+7.947739999999999e+02 +2.050000000000000e+02
+9.493440000000001e+02 +6.235000000000000e+02
+8.086820000000000e+02 +6.235000000000000e+02
+1.478000000000000e+02 +6.235000000000000e+02
+1.160050000000000e+03 +9.600000000000000e+01
+9.907960000000000e+02 +1.145000000000000e+02
+1.078240000000000e+03 +1.410000000000000e+02
+1.847870000000000e+03 +4.870000000000000e+02
+6.210740000000002e+02 +8.300000000000000e+01
+7.836860000000000e+02 +1.285000000000000e+02
+1.995830000000000e+03 +3.305000000000000e+02
+9.099950000000000e+02 +1.515000000000000e+02
+6.201870000000000e+02 +6.650000000000000e+01
+2.314230000000000e+03 +4.610000000000000e+02
+1.846820000000000e+03 +4.865000000000000e+02
+6.575080000000000e+02 +8.350000000000000e+01
+7.587980000000000e+02 +6.215000000000000e+02
+6.933980000000000e+02 +1.210000000000000e+02
+7.944370000000000e+02 +1.270000000000000e+02
+9.169310000000000e+02 +1.515000000000000e+02
+1.099920000000000e+03 +4.085000000000000e+02
+1.386840000000000e+03 +5.720000000000000e+02
+1.953390000000000e+03 +3.190000000000000e+02
+1.421030000000000e+03 +5.810000000000000e+02
+8.333730000000000e+02 +1.715000000000000e+02
+4.875220000000000e+02 +1.415000000000000e+02
+6.672280000000002e+02 +7.450000000000000e+01
+8.137160000000000e+02 +6.205000000000000e+02
+1.080200000000000e+02 +1.350000000000000e+01
+6.701189999999998e+02 +3.905000000000000e+02
+1.411320000000000e+03 +2.345000000000000e+02
+8.599680000000002e+02 +1.565000000000000e+02
+6.224420000000000e+02 +1.060000000000000e+02
+1.173250000000000e+02 +1.500000000000000e+01
+1.143810000000000e+03 +2.950000000000000e+02
+8.405150000000000e+02 +6.550000000000000e+01
+9.822140000000001e+02 +1.500000000000000e+02
+1.057450000000000e+03 +1.970000000000000e+02
+7.783510000000001e+02 +1.270000000000000e+02
+6.302950000000000e+02 +1.010000000000000e+02
+1.551530000000000e+03 +1.890000000000000e+02
+6.446930000000000e+02 +1.000000000000000e+02
+1.753160000000000e+03 +4.225000000000000e+02
+4.452200000000000e+02 +8.200000000000000e+01
+1.430010000000000e+03 +2.465000000000000e+02
+1.862790000000000e+03 +4.820000000000000e+02
+1.008580000000000e+03 +2.115000000000000e+02
+6.600880000000002e+02 +7.900000000000000e+01
+1.055500000000000e+03 +2.200000000000000e+02
+7.283760000000002e+02 +6.160000000000000e+02
+2.333310000000000e+02 +6.160000000000000e+02
+9.587940000000000e+02 +1.485000000000000e+02
+1.009370000000000e+03 +1.575000000000000e+02
+1.752730000000000e+03 +2.995000000000000e+02
+6.604490000000000e+02 +6.300000000000000e+01
+1.858040000000000e+03 +3.080000000000000e+02
+7.807400000000000e+02 +1.225000000000000e+02
+1.339080000000000e+03 +3.725000000000000e+02
+9.524850000000000e+02 +2.780000000000000e+02
+7.581460000000002e+02 +6.145000000000000e+02
+7.807339999999998e+02 +1.250000000000000e+02
+1.536240000000000e+03 +2.815000000000000e+02
+1.290190000000000e+03 +5.700000000000000e+02
+6.610300000000000e+02 +6.950000000000000e+01
+7.296630000000000e+02 +6.140000000000000e+02
+1.347220000000000e+03 +3.000000000000000e+02
+2.815090000000000e+02 +8.650000000000000e+01
+6.867420000000000e+02 +1.600000000000000e+02
+7.893480000000002e+02 +1.250000000000000e+02
+1.141340000000000e+03 +2.970000000000000e+02
+7.939939999999998e+02 +6.120000000000000e+02
+7.530660000000000e+02 +1.195000000000000e+02
+1.296850000000000e+03 +1.445000000000000e+02
+9.783780000000000e+02 +1.485000000000000e+02
+6.223230000000000e+02 +9.650000000000000e+01
+7.739920000000000e+02 +1.205000000000000e+02
+1.296930000000000e+03 +2.305000000000000e+02
+9.691090000000000e+02 +1.460000000000000e+02
+1.363580000000000e+02 +6.105000000000000e+02
+3.250260000000000e+02 +2.300000000000000e+01
+1.047330000000000e+03 +5.175000000000000e+02
+1.423610000000000e+03 +2.395000000000000e+02
+1.751280000000000e+03 +3.540000000000000e+02
+6.456180000000001e+02 +1.575000000000000e+02
+1.142340000000000e+03 +2.795000000000000e+02
+8.920970000000000e+02 +1.610000000000000e+02
+6.138090000000000e+02 +5.850000000000000e+01
+7.810230000000000e+02 +1.185000000000000e+02
+1.097180000000000e+03 +1.275000000000000e+02
+1.137120000000000e+03 +3.285000000000000e+02
+6.620880000000002e+02 +4.760000000000000e+02
+7.025810000000000e+02 +6.085000000000000e+02
+6.629450000000001e+02 +9.700000000000000e+01
+1.759090000000000e+03 +4.690000000000000e+02
+9.992820000000000e+02 +1.990000000000000e+02
+6.608070000000000e+02 +1.620000000000000e+02
+7.951540000000000e+02 +6.080000000000000e+02
+7.798080000000000e+02 +1.215000000000000e+02
+1.834840000000000e+03 +4.750000000000000e+02
+7.877360000000001e+02 +1.180000000000000e+02
+1.722570000000000e+03 +3.040000000000000e+02
+1.385640000000000e+03 +4.100000000000000e+02
+4.939470000000000e+02 +1.305000000000000e+02
+7.003860000000002e+02 +1.580000000000000e+02
+1.641360000000000e+03 +3.515000000000000e+02
+8.234520000000000e+02 +6.070000000000000e+02
+1.057430000000000e+01 +6.070000000000000e+02
+7.458980000000000e+02 +5.250000000000000e+01
+8.827339999999998e+02 +3.490000000000000e+02
+9.836200000000000e+02 +1.455000000000000e+02
+1.014500000000000e+03 +2.130000000000000e+02
+1.309480000000000e+03 +2.600000000000000e+02
+1.324520000000000e+03 +3.615000000000000e+02
+9.130030000000000e+02 +1.560000000000000e+02
+1.344440000000000e+03 +2.780000000000000e+02
+6.316170000000000e+02 +9.300000000000000e+01
+7.912170000000000e+02 +1.165000000000000e+02
+1.307020000000000e+03 +3.360000000000000e+02
+1.856920000000000e+03 +3.265000000000000e+02
+9.450910000000000e+02 +7.900000000000000e+01
+1.869440000000000e+03 +4.830000000000000e+02
+6.300890000000001e+02 +4.735000000000000e+02
+8.970829999999999e+01 +6.040000000000000e+02
+7.938410000000000e+02 +1.845000000000000e+02
+7.467739999999999e+02 +1.215000000000000e+02
+9.951799999999999e+02 +1.840000000000000e+02
+7.870599999999999e+02 +1.215000000000000e+02
+1.863910000000000e+03 +4.730000000000000e+02
+1.070630000000000e+03 +3.125000000000000e+02
+7.142580000000000e+02 +1.890000000000000e+02
+6.911590000000000e+02 +1.550000000000000e+02
+1.906070000000000e+02 +6.025000000000000e+02
+1.296660000000000e+03 +4.355000000000000e+02
+8.455050000000000e+02 +7.300000000000000e+01
+6.950580000000000e+02 +2.595000000000000e+02
+1.002760000000000e+03 +2.485000000000000e+02
+1.855460000000000e+03 +3.250000000000000e+02
+7.552810000000002e+02 +1.185000000000000e+02
+7.446469999999998e+02 +6.015000000000000e+02
+1.834930000000000e+02 +6.015000000000000e+02
+9.833380000000000e+02 +1.105000000000000e+02
+1.189730000000000e+03 +2.110000000000000e+02
+8.966220000000000e+02 +3.340000000000000e+02
+7.808760000000002e+02 +1.150000000000000e+02
+1.525420000000000e+03 +2.860000000000000e+02
+8.387230000000002e+02 +6.850000000000000e+01
+1.407330000000000e+03 +5.655000000000000e+02
+1.322060000000000e+03 +2.130000000000000e+02
+1.877420000000000e+03 +3.255000000000000e+02
+1.532320000000000e+03 +3.485000000000000e+02
+8.673939999999999e+02 +6.500000000000000e+01
+6.796940000000000e+02 +1.560000000000000e+02
+1.921080000000000e+03 +3.580000000000000e+02
+1.907810000000000e+03 +3.405000000000000e+02
+6.370119999999999e+02 +1.520000000000000e+02
+7.920430000000000e+02 +1.565000000000000e+02
+6.188660000000000e+02 +8.750000000000000e+01
+7.892930000000000e+02 +1.130000000000000e+02
+8.805360000000002e+02 +1.850000000000000e+02
+6.969019999999998e+02 +1.455000000000000e+02
+1.522840000000000e+03 +1.440000000000000e+02
+8.118510000000001e+02 +1.815000000000000e+02
+6.370840000000002e+02 +1.535000000000000e+02
+1.327540000000000e+03 +2.525000000000000e+02
+9.775860000000000e+02 +1.190000000000000e+02
+1.031970000000000e+03 +1.215000000000000e+02
+5.825850000000000e+02 +1.005000000000000e+02
+9.726830000000000e+02 +1.830000000000000e+02
+8.492030000000000e+02 +7.000000000000000e+01
+1.204480000000000e+03 +1.835000000000000e+02
+6.207619999999999e+02 +5.965000000000000e+02
+1.486230000000000e+02 +5.965000000000000e+02
+1.322980000000000e+03 +2.765000000000000e+02
+4.874680000000000e+02 +1.205000000000000e+02
+2.921790000000001e+02 +1.110000000000000e+02
+1.496970000000000e+03 +3.340000000000000e+02
+9.757130000000000e+02 +1.150000000000000e+02
+1.376310000000000e+03 +3.475000000000000e+02
+7.007010000000000e+02 +1.440000000000000e+02
+1.364930000000000e+02 +5.955000000000000e+02
+2.384680000000000e+03 +4.060000000000000e+02
+6.775219999999998e+02 +1.200000000000000e+02
+1.365580000000000e+03 +5.475000000000000e+02
+1.364730000000000e+03 +2.620000000000000e+02
+6.356010000000000e+02 +1.410000000000000e+02
+1.040850000000000e+03 +2.120000000000000e+02
+6.839520000000000e+02 +1.500000000000000e+02
+6.770440000000000e+02 +1.170000000000000e+02
+1.942340000000000e+02 +5.940000000000000e+02
+6.393400000000000e+02 +1.510000000000000e+02
+1.852740000000000e+03 +3.895000000000000e+02
+1.668470000000000e+03 +3.025000000000000e+02
+1.285150000000000e+03 +3.230000000000000e+02
+1.008840000000000e+03 +1.855000000000000e+02
+7.137589999999999e+02 +5.925000000000000e+02
+1.467750000000000e+02 +5.925000000000000e+02
+1.922450000000000e+03 +2.925000000000000e+02
+8.633160000000000e+02 +1.220000000000000e+02
+9.132360000000000e+02 +1.885000000000000e+02
+9.849290000000000e+02 +1.830000000000000e+02
+2.753600000000000e+02 +6.650000000000000e+01
+1.125590000000000e+03 +2.640000000000000e+02
+1.816460000000000e+03 +2.860000000000000e+02
+7.848789999999998e+02 +5.915000000000000e+02
+1.092240000000000e+02 +3.150000000000000e+01
+8.761760000000000e+02 +5.025000000000000e+02
+6.706660000000001e+02 +1.180000000000000e+02
+6.733670000000000e+02 +1.415000000000000e+02
+1.861270000000000e+03 +3.870000000000000e+02
+1.705490000000000e+02 +5.910000000000000e+02
+9.756220000000000e+02 +1.770000000000000e+02
+4.542820000000000e+02 +6.350000000000000e+01
+1.364280000000000e+03 +5.375000000000000e+02
+7.801419999999998e+02 +5.900000000000000e+02
+1.375910000000000e+02 +5.900000000000000e+02
+7.718770000000000e+02 +1.100000000000000e+02
+1.602360000000000e+03 +2.900000000000000e+02
+6.651980000000000e+02 +4.680000000000000e+02
+1.087220000000000e+03 +1.755000000000000e+02
+6.405319999999998e+02 +1.400000000000000e+02
+8.819270000000000e+02 +1.355000000000000e+02
+1.333890000000000e+03 +5.895000000000000e+02
+1.834990000000000e+03 +4.585000000000000e+02
+9.985880000000000e+02 +1.820000000000000e+02
+7.742990000000000e+02 +5.890000000000000e+02
+9.784480000000000e+02 +1.090000000000000e+02
+6.122010000000000e+02 +6.000000000000000e+01
+2.816580000000000e+02 +6.500000000000000e+01
+1.286620000000000e+03 +3.400000000000000e+02
+1.616280000000000e+03 +3.230000000000000e+02
+7.767089999999999e+02 +1.040000000000000e+02
+4.699100000000000e+02 +1.100000000000000e+02
+8.417400000000000e+02 +1.065000000000000e+02
+1.460740000000000e+03 +4.225000000000000e+02
+8.115039999999998e+02 +1.475000000000000e+02
+1.066400000000000e+03 +1.080000000000000e+02
+1.628200000000000e+03 +3.180000000000000e+02
+1.873670000000000e+03 +3.155000000000000e+02
+1.158880000000000e+02 +6.300000000000000e+01
+1.372450000000000e+03 +3.425000000000000e+02
+6.815989999999998e+02 +1.465000000000000e+02
+1.406780000000000e+03 +2.565000000000000e+02
+6.485800000000000e+02 +1.460000000000000e+02
+7.397900000000000e+02 +5.865000000000000e+02
+1.163340000000000e+02 +5.865000000000000e+02
+7.738000000000000e+02 +1.075000000000000e+02
+8.090160000000002e+02 +1.760000000000000e+02
+8.384310000000000e+02 +1.720000000000000e+02
+1.075360000000000e+03 +1.645000000000000e+02
+6.645630000000000e+02 +7.750000000000000e+01
+7.874180000000000e+02 +1.090000000000000e+02
+1.057910000000000e+03 +1.060000000000000e+02
+1.797020000000000e+03 +4.660000000000000e+02
+7.854119999999998e+02 +1.030000000000000e+02
+1.864040000000000e+03 +4.605000000000000e+02
+1.070270000000000e+03 +1.690000000000000e+02
+1.235290000000000e+02 +5.850000000000000e+02
+7.458589999999998e+02 +1.000000000000000e+02
+9.247340000000000e+02 +1.890000000000000e+02
+6.233900000000000e+02 +7.500000000000000e+01
+1.164400000000000e+03 +1.695000000000000e+02
+7.253339999999999e+02 +5.835000000000000e+02
+7.391619999999998e+02 +5.835000000000000e+02
+3.258570000000000e+02 +1.010000000000000e+02
+1.394630000000000e+03 +2.155000000000000e+02
+6.441910000000000e+02 +1.355000000000000e+02
+1.159600000000000e+02 +5.830000000000000e+02
+1.159860000000000e+03 +2.505000000000000e+02
+1.010890000000000e+03 +1.725000000000000e+02
+7.943830000000000e+02 +1.065000000000000e+02
+6.899260000000000e+02 +2.410000000000000e+02
+7.457339999999998e+02 +1.015000000000000e+02
+6.127640000000000e+02 +1.335000000000000e+02
+1.073200000000000e+03 +1.705000000000000e+02
+7.811619999999998e+02 +5.815000000000000e+02
+7.094490000000000e+02 +5.815000000000000e+02
+6.248360000000000e+02 +7.000000000000000e+01
+1.269050000000000e+02 +5.815000000000000e+02
+1.111200000000000e+03 +1.845000000000000e+02
+7.520110000000002e+02 +1.445000000000000e+02
+9.025390000000000e+02 +1.320000000000000e+02
+1.154900000000000e+03 +5.805000000000000e+02
+6.358700000000000e+02 +1.365000000000000e+02
+1.553850000000000e+03 +2.775000000000000e+02
+1.857540000000000e+03 +3.040000000000000e+02
+4.878150000000000e+02 +1.080000000000000e+02
+7.434440000000000e+02 +5.795000000000000e+02
+1.315850000000000e+03 +3.450000000000000e+02
+7.753150000000001e+02 +1.000000000000000e+02
+7.153750000000000e+02 +5.790000000000000e+02
+7.175310000000002e+02 +5.790000000000000e+02
+1.092400000000000e+02 +5.790000000000000e+02
+1.497790000000000e+03 +2.075000000000000e+02
+1.171590000000000e+03 +5.785000000000000e+02
+4.818390000000000e+02 +1.055000000000000e+02
+1.069500000000000e+03 +1.600000000000000e+02
+9.059230000000000e+02 +3.800000000000000e+02
+9.110970000000000e+02 +1.800000000000000e+02
+1.073160000000000e+03 +1.590000000000000e+02
+7.070700000000001e+02 +5.775000000000000e+02
+7.913480000000002e+02 +5.775000000000000e+02
+7.885380000000000e+02 +1.035000000000000e+02
+1.673200000000000e+03 +2.845000000000000e+02
+8.969150000000000e+02 +1.095000000000000e+02
+6.582330000000002e+02 +4.585000000000000e+02
+4.987620000000000e+02 +1.075000000000000e+02
+1.373820000000000e+03 +3.380000000000000e+02
+1.713750000000000e+03 +3.305000000000000e+02
+2.386740000000000e+03 +5.135000000000000e+02
+1.302920000000000e+03 +2.260000000000000e+02
+1.445480000000000e+03 +2.980000000000000e+02
+1.119840000000000e+03 +2.950000000000000e+02
+9.006590000000000e+02 +4.900000000000000e+02
+9.817080000000000e+02 +1.690000000000000e+02
+7.456230000000000e+02 +9.700000000000000e+01
+7.222489999999998e+02 +5.760000000000000e+02
+1.722860000000000e+03 +3.820000000000000e+02
+6.944100000000000e+02 +1.575000000000000e+02
+8.451189999999998e+02 +1.145000000000000e+02
+9.211100000000000e+02 +1.115000000000000e+02
+1.459300000000000e+03 +2.985000000000000e+02
+1.603440000000000e+03 +3.235000000000000e+02
+7.196220000000000e+02 +5.750000000000000e+02
+6.249180000000000e+02 +6.600000000000000e+01
+9.955870000000000e+01 +5.750000000000000e+02
+7.816389999999999e+02 +1.005000000000000e+02
+9.856590000000000e+02 +1.975000000000000e+02
+7.039750000000000e+02 +1.510000000000000e+02
+7.466519999999998e+02 +9.700000000000000e+01
+7.313439999999998e+02 +5.740000000000000e+02
+1.306310000000000e+03 +1.885000000000000e+02
+1.991750000000000e+03 +2.795000000000000e+02
+1.557610000000000e+03 +2.700000000000000e+02
+8.612270000000000e+02 +2.020000000000000e+02
+6.512450000000000e+02 +1.305000000000000e+02
+7.093220000000000e+02 +5.735000000000000e+02
+1.869610000000000e+03 +5.470000000000000e+02
+9.811920000000000e+02 +1.660000000000000e+02
+9.891040000000000e+01 +5.725000000000000e+02
+9.874250000000000e+02 +1.845000000000000e+02
+1.070350000000000e+03 +1.775000000000000e+02
+6.916189999999998e+02 +2.210000000000000e+02
+1.524980000000000e+03 +2.615000000000000e+02
+1.316230000000000e+03 +2.275000000000000e+02
+7.832880000000000e+02 +1.005000000000000e+02
+1.393470000000000e+03 +3.645000000000000e+02
+6.990230000000000e+02 +5.710000000000000e+02
+1.188830000000000e+02 +5.710000000000000e+02
+7.459860000000001e+02 +9.500000000000000e+01
+1.701660000000000e+03 +2.705000000000000e+02
+1.168140000000000e+03 +5.705000000000000e+02
+8.587410000000001e+02 +2.015000000000000e+02
+8.899770000000000e+02 +5.705000000000000e+02
+1.789040000000000e+03 +4.885000000000000e+02
+4.829530000000000e+02 +1.025000000000000e+02
+6.213530000000002e+02 +6.300000000000000e+01
+9.156910000000000e+02 +1.760000000000000e+02
+4.949270000000000e+02 +1.950000000000000e+02
+8.587030000000000e+02 +1.950000000000000e+02
+8.075150000000000e+02 +2.185000000000000e+02
+5.006640000000000e+02 +8.300000000000000e+01
+1.034200000000000e+03 +1.635000000000000e+02
+9.068160000000000e+02 +1.065000000000000e+02
+7.145920000000000e+02 +5.685000000000000e+02
+6.800770000000000e+02 +5.685000000000000e+02
+7.708580000000002e+02 +9.700000000000000e+01
+1.658100000000000e+03 +2.865000000000000e+02
+1.224100000000000e+03 +2.800000000000000e+02
+1.761340000000000e+03 +5.685000000000000e+02
+5.005240000000000e+02 +1.885000000000000e+02
+1.675300000000000e+03 +2.790000000000000e+02
+9.394650000000000e+02 +1.845000000000000e+02
+6.599290000000000e+02 +5.675000000000000e+02
+7.741920000000000e+02 +9.900000000000000e+01
+1.302580000000000e+03 +2.000000000000000e+02
+1.097670000000000e+03 +1.715000000000000e+02
+1.869230000000000e+03 +5.440000000000000e+02
+5.036340000000000e+02 +1.985000000000000e+02
+9.197960000000000e+01 +5.670000000000000e+02
+1.466680000000000e+03 +2.630000000000000e+02
+1.130210000000000e+03 +2.405000000000000e+02
+1.743070000000000e+03 +3.675000000000000e+02
+6.383670000000000e+02 +1.220000000000000e+02
+9.956460000000000e+02 +1.650000000000000e+02
+1.148010000000000e+03 +5.660000000000000e+02
+6.986450000000000e+02 +2.605000000000000e+02
+7.830510000000000e+02 +9.550000000000000e+01
+9.943660000000000e+02 +1.825000000000000e+02
+1.973340000000000e+03 +3.725000000000000e+02
+5.056770000000000e+02 +2.020000000000000e+02
+7.998560000000001e+02 +1.570000000000000e+02
+7.807200000000000e+02 +9.850000000000000e+01
+1.292280000000000e+03 +1.995000000000000e+02
+1.869790000000000e+03 +5.400000000000000e+02
+6.503240000000002e+02 +5.650000000000000e+02
+7.592600000000000e+02 +9.550000000000000e+01
+9.152320000000000e+02 +1.065000000000000e+02
+9.095890000000001e+02 +1.745000000000000e+02
+1.306710000000000e+03 +2.915000000000000e+02
+1.073940000000000e+03 +2.625000000000000e+02
+8.529920000000000e+02 +5.645000000000000e+02
+1.856280000000000e+03 +3.645000000000000e+02
+9.864780000000000e+02 +1.545000000000000e+02
+1.313720000000000e+03 +1.940000000000000e+02
+1.085030000000000e+03 +2.610000000000000e+02
+6.374230000000000e+02 +1.265000000000000e+02
+6.228500000000000e+02 +5.550000000000000e+01
+1.390050000000000e+03 +3.280000000000000e+02
+1.745180000000000e+03 +2.675000000000000e+02
+1.874400000000000e+03 +5.425000000000000e+02
+7.843530000000002e+02 +9.800000000000000e+01
+1.381660000000000e+03 +3.590000000000000e+02
+8.785590000000000e+02 +1.065000000000000e+02
+5.011920000000000e+02 +1.920000000000000e+02
+1.035190000000000e+03 +2.135000000000000e+02
+9.349930000000001e+01 +5.630000000000000e+02
+7.950250000000000e+02 +9.900000000000000e+01
+6.508040000000000e+02 +5.625000000000000e+02
+1.872860000000000e+03 +3.970000000000000e+02
+1.308570000000000e+03 +2.050000000000000e+02
+1.002620000000000e+03 +2.840000000000000e+02
+4.753100000000000e+02 +1.850000000000000e+02
+6.310620000000000e+02 +5.620000000000000e+02
+8.140009999999999e+01 +5.620000000000000e+02
+7.784530000000000e+02 +9.450000000000000e+01
+1.183730000000000e+03 +5.615000000000000e+02
+6.165500000000000e+02 +5.615000000000000e+02
+6.032310000000000e+02 +5.615000000000000e+02
+1.074110000000000e+03 +2.550000000000000e+02
+7.951330000000000e+02 +9.950000000000000e+01
+1.609390000000000e+03 +3.010000000000000e+02
+1.493480000000000e+03 +2.945000000000000e+02
+1.206430000000000e+03 +2.530000000000000e+02
+4.960080000000000e+02 +1.935000000000000e+02
+1.013790000000000e+03 +1.630000000000000e+02
+6.709639999999998e+02 +1.800000000000000e+02
+1.864860000000000e+03 +3.640000000000000e+02
+7.436510000000002e+02 +8.450000000000000e+01
+9.991840000000001e+01 +5.595000000000000e+02
+9.891900000000001e+02 +1.615000000000000e+02
+7.964650000000000e+02 +9.400000000000000e+01
+8.943200000000001e+02 +1.190000000000000e+02
+8.318439999999998e+02 +5.585000000000000e+02
+7.040710000000000e+02 +5.585000000000000e+02
+1.632420000000000e+03 +2.745000000000000e+02
+1.704760000000000e+03 +3.100000000000000e+02
+1.839300000000000e+03 +3.580000000000000e+02
+7.798400000000000e+02 +9.150000000000000e+01
+9.923370000000000e+02 +1.540000000000000e+02
+9.021180000000001e+02 +1.120000000000000e+02
+1.277850000000000e+03 +2.535000000000000e+02
+8.895360000000002e+02 +2.280000000000000e+02
+1.042460000000000e+03 +1.665000000000000e+02
+2.003500000000000e+03 +3.730000000000000e+02
+6.644000000000000e+02 +1.785000000000000e+02
+7.025000000000000e+02 +1.360000000000000e+02
+4.678090000000000e+02 +8.900000000000000e+01
+1.297370000000000e+03 +3.165000000000000e+02
+6.363220000000000e+02 +1.100000000000000e+02
+1.850560000000000e+03 +5.565000000000000e+02
+1.872500000000000e+03 +3.535000000000000e+02
+1.491480000000000e+03 +3.420000000000000e+02
+9.900340000000000e+02 +1.545000000000000e+02
+1.563970000000000e+03 +2.240000000000000e+02
+1.316870000000000e+03 +1.910000000000000e+02
+7.029230000000000e+02 +1.365000000000000e+02
+1.366050000000000e+03 +4.230000000000000e+02
+1.068020000000000e+03 +2.570000000000000e+02
+8.706110000000000e+01 +5.545000000000000e+02
+1.576650000000000e+03 +3.115000000000000e+02
+6.789050000000000e+02 +5.540000000000000e+02
+6.781880000000000e+02 +1.775000000000000e+02
+6.365690000000000e+02 +1.125000000000000e+02
+7.901369999999999e+02 +1.455000000000000e+02
+1.073690000000000e+03 +2.530000000000000e+02
+8.116380000000000e+02 +1.435000000000000e+02
+6.996260000000002e+02 +5.535000000000000e+02
+1.309990000000000e+03 +2.860000000000000e+02
+1.294270000000000e+03 +1.860000000000000e+02
+1.353700000000000e+03 +4.170000000000000e+02
+9.027320000000000e+02 +1.100000000000000e+02
+1.296870000000000e+03 +2.775000000000000e+02
+6.496110000000000e+02 +1.150000000000000e+02
+1.463310000000000e+00 +5.530000000000000e+02
+7.787589999999999e+02 +9.250000000000000e+01
+1.725930000000000e+03 +5.530000000000000e+02
+1.874920000000000e+03 +5.335000000000000e+02
+8.813610000000001e+02 +1.935000000000000e+02
+1.315410000000000e+03 +1.840000000000000e+02
+1.102070000000000e+03 +1.590000000000000e+02
+1.738030000000000e+03 +2.540000000000000e+02
+6.390180000000000e+02 +5.385000000000000e+02
+4.772200000000000e+02 +7.050000000000000e+01
+7.092060000000000e+02 +5.520000000000000e+02
+9.545510000000000e+02 +1.480000000000000e+02
+1.058420000000000e+03 +3.535000000000000e+02
+1.444150000000000e+03 +5.075000000000000e+02
+1.600620000000000e+03 +4.080000000000000e+02
+6.070970000000000e+02 +5.510000000000000e+02
+8.744210000000000e+02 +1.005000000000000e+02
+1.382600000000000e+03 +4.165000000000000e+02
+1.348450000000000e+03 +2.395000000000000e+02
+2.319180000000000e+03 +4.805000000000000e+02
+7.794299999999999e+02 +8.850000000000000e+01
+1.755460000000000e+03 +4.130000000000000e+02
+8.740419999999998e+02 +8.950000000000000e+01
+9.006780000000000e+02 +5.070000000000000e+02
+1.570940000000000e+03 +3.040000000000000e+02
+1.998620000000000e+03 +3.600000000000000e+02
+8.412320000000000e+02 +8.150000000000000e+01
+1.383570000000000e+03 +2.615000000000000e+02
+7.417580000000000e+02 +7.650000000000000e+01
+6.774480000000000e+02 +5.495000000000000e+02
+4.520880000000000e+01 +5.495000000000000e+02
+1.136890000000000e+03 +2.585000000000000e+02
+6.835130000000000e+02 +2.030000000000000e+02
+6.921280000000000e+02 +1.300000000000000e+02
+1.394660000000000e+03 +4.535000000000000e+02
+1.299980000000000e+03 +2.590000000000000e+02
+1.234610000000000e+03 +3.605000000000000e+02
+1.872360000000000e+03 +5.295000000000000e+02
+4.823500000000000e+02 +1.880000000000000e+02
+1.082600000000000e+03 +2.525000000000000e+02
+6.923530000000002e+02 +5.480000000000000e+02
+1.303590000000000e+03 +1.880000000000000e+02
+1.137690000000000e+03 +2.430000000000000e+02
+7.762410000000001e+02 +1.620000000000000e+02
+4.886330000000000e+02 +8.150000000000000e+01
+8.910050000000000e+02 +2.170000000000000e+02
+1.581020000000000e+03 +2.150000000000000e+02
+2.011490000000000e+03 +3.625000000000000e+02
+9.032140000000001e+02 +1.090000000000000e+02
+9.795020000000000e+02 +1.505000000000000e+02
+1.288210000000000e+03 +2.755000000000000e+02
+7.418589999999998e+02 +5.460000000000000e+02
+1.863360000000000e+03 +5.235000000000000e+02
+6.230169999999998e+02 +1.440000000000000e+02
+1.073790000000000e+03 +4.540000000000000e+02
+9.663500000000000e+02 +1.430000000000000e+02
+7.906350000000000e+02 +1.345000000000000e+02
+5.945300000000000e+02 +5.445000000000000e+02
+7.720570000000000e+02 +8.150000000000000e+01
+9.235580000000000e+02 +2.185000000000000e+02
+1.665360000000000e+03 +3.265000000000000e+02
+1.085660000000000e+03 +2.475000000000000e+02
+4.923280000000000e+02 +1.860000000000000e+02
+2.986780000000000e+02 +1.065000000000000e+02
+1.037030000000000e+03 +2.495000000000000e+02
+6.208830000000000e+02 +1.420000000000000e+02
+1.382990000000000e+03 +4.105000000000000e+02
+9.127630000000000e+02 +1.605000000000000e+02
+1.593830000000000e+03 +2.715000000000000e+02
+1.079060000000000e+03 +2.500000000000000e+02
+8.288480000000002e+02 +5.420000000000000e+02
+1.080590000000000e+03 +9.850000000000000e+01
+6.251350000000000e+02 +1.450000000000000e+02
+1.308050000000000e+03 +1.580000000000000e+02
+1.015460000000000e+03 +2.690000000000000e+02
+6.399610000000000e+02 +1.075000000000000e+02
+6.873180000000000e+02 +5.410000000000000e+02
+7.485180000000000e+01 +5.410000000000000e+02
+7.627320000000000e+02 +1.565000000000000e+02
+6.888339999999999e+02 +1.925000000000000e+02
+6.381410000000000e+02 +1.040000000000000e+02
+6.713260000000000e+02 +5.405000000000000e+02
+7.785100000000000e+02 +8.000000000000000e+01
+1.375540000000000e+03 +3.375000000000000e+02
+1.139690000000000e+03 +4.350000000000000e+02
+1.299480000000000e+03 +1.775000000000000e+02
+1.339630000000000e+03 +2.340000000000000e+02
+1.379760000000000e+03 +2.305000000000000e+02
+1.817900000000000e+03 +5.400000000000000e+02
+7.620780000000001e+01 +5.400000000000000e+02
+1.936860000000000e+03 +4.190000000000000e+02
+1.911930000000000e+03 +3.555000000000000e+02
+1.957040000000000e+03 +4.285000000000000e+02
+1.928000000000000e+03 +3.705000000000000e+02
+1.322320000000000e+03 +2.545000000000000e+02
+1.039620000000000e+03 +1.850000000000000e+02
+1.863730000000000e+03 +5.400000000000000e+02
+6.747180000000002e+02 +1.030000000000000e+02
+1.988100000000000e+03 +4.360000000000000e+02
+1.946130000000000e+03 +3.630000000000000e+02
+8.642919999999998e+02 +5.400000000000000e+02
+8.144000000000000e+02 +1.385000000000000e+02
+6.942610000000001e+01 +5.395000000000000e+02
+7.799349999999999e+02 +1.530000000000000e+02
+8.971750000000000e+02 +1.060000000000000e+02
+9.059410000000000e+02 +2.105000000000000e+02
+1.015400000000000e+03 +1.480000000000000e+02
+7.076130000000001e+02 +2.055000000000000e+02
+1.681040000000000e+03 +3.230000000000000e+02
+1.606720000000000e+03 +4.115000000000000e+02
+6.556089999999998e+02 +2.055000000000000e+02
+1.657790000000000e+03 +2.940000000000000e+02
+6.520650000000001e+02 +1.025000000000000e+02
+1.852910000000000e+03 +3.720000000000000e+02
+9.609299999999999e+02 +1.455000000000000e+02
+9.083990000000000e+02 +1.030000000000000e+02
+1.343500000000000e+03 +2.285000000000000e+02
+1.693170000000000e+03 +3.795000000000000e+02
+1.295980000000000e+03 +3.690000000000000e+02
+1.088280000000000e+03 +1.430000000000000e+02
+2.745690000000000e+02 +2.050000000000000e+01
+6.185010000000000e+02 +9.750000000000000e+01
+5.430800000000000e+02 +5.365000000000000e+02
+6.774589999999999e+02 +5.365000000000000e+02
+1.269500000000000e+03 +2.895000000000000e+02
+8.253370000000000e+02 +5.360000000000000e+02
+6.674960000000002e+02 +5.360000000000000e+02
+1.410890000000000e+03 +4.980000000000000e+02
+1.236880000000000e+03 +2.290000000000000e+02
+1.065740000000000e+03 +1.460000000000000e+02
+1.300630000000000e+03 +1.740000000000000e+02
+7.946840000000000e+02 +1.605000000000000e+02
+8.036810000000000e+02 +1.365000000000000e+02
+1.088860000000000e+03 +2.420000000000000e+02
+6.839780000000000e+01 +5.350000000000000e+02
+1.584410000000000e+03 +2.825000000000000e+02
+8.918610000000000e+01 +5.345000000000000e+02
+8.787739999999999e+02 +4.495000000000000e+02
+9.378790000000000e+02 +1.650000000000000e+02
+9.799550000000000e+02 +1.425000000000000e+02
+9.231910000000000e+02 +1.370000000000000e+02
+1.395340000000000e+03 +2.430000000000000e+02
+6.249980000000000e+02 +1.350000000000000e+02
+5.396290000000000e+01 +5.335000000000000e+02
+7.652150000000000e+02 +1.505000000000000e+02
+9.199550000000000e+02 +2.505000000000000e+02
+7.816610000000002e+02 +1.475000000000000e+02
+1.134170000000000e+03 +2.440000000000000e+02
+5.877800000000000e+02 +5.320000000000000e+02
+6.031630000000000e+02 +5.320000000000000e+02
+9.190490000000000e+02 +1.345000000000000e+02
+1.029250000000000e+03 +1.515000000000000e+02
+1.869130000000000e+03 +4.440000000000000e+02
+2.817010000000000e+02 +2.150000000000000e+01
+7.015390000000000e+02 +1.970000000000000e+02
+6.577869999999998e+02 +5.305000000000000e+02
+9.259990000000000e+02 +2.505000000000000e+02
+1.478630000000000e+03 +3.770000000000000e+02
+1.844360000000000e+03 +3.330000000000000e+02
+1.673650000000000e+03 +4.075000000000000e+02
+6.827869999999998e+02 +5.295000000000000e+02
+5.581150000000000e+02 +5.295000000000000e+02
+1.079690000000000e+03 +2.240000000000000e+02
+4.951210000000000e+02 +2.620000000000000e+02
+4.958500000000000e+02 +5.200000000000000e+01
+8.385880000000002e+02 +8.200000000000000e+01
+9.750700000000001e+02 +1.375000000000000e+02
+1.017230000000000e+03 +1.080000000000000e+02
+5.710910000000000e+02 +5.285000000000000e+02
+1.331260000000000e+03 +2.685000000000000e+02
+6.227220000000000e+02 +1.295000000000000e+02
+1.012970000000000e+03 +2.415000000000000e+02
+2.127360000000000e+02 +5.280000000000000e+02
+6.796890000000000e+02 +5.275000000000000e+02
+8.886920000000000e+02 +1.490000000000000e+02
+1.050850000000000e+03 +2.380000000000000e+02
+2.812460000000000e+02 +2.200000000000000e+01
+7.734360000000000e+02 +1.460000000000000e+02
+1.032840000000000e+03 +1.935000000000000e+02
+9.757280000000000e+02 +3.370000000000000e+02
+7.958750000000000e+02 +2.365000000000000e+02
+6.157390000000000e+02 +9.200000000000000e+01
+1.090860000000000e+03 +1.370000000000000e+02
+7.158110000000000e+02 +5.260000000000000e+02
+6.311970000000000e+02 +5.260000000000000e+02
+1.081010000000000e+03 +2.380000000000000e+02
+7.793510000000001e+02 +5.255000000000000e+02
+6.609989999999998e+02 +1.265000000000000e+02
+7.852780000000000e+02 +1.420000000000000e+02
+6.212640000000000e+02 +9.500000000000000e+01
+1.295100000000000e+03 +1.610000000000000e+02
+1.140840000000000e+03 +1.985000000000000e+02
+9.305170000000001e+02 +2.070000000000000e+02
+1.475450000000000e+03 +2.595000000000000e+02
+1.262600000000000e+03 +3.065000000000000e+02
+1.326640000000000e+03 +2.865000000000000e+02
+5.633070000000000e+01 +5.245000000000000e+02
+1.367060000000000e+03 +3.945000000000000e+02
+9.105860000000000e+02 +1.295000000000000e+02
+1.070000000000000e+03 +2.395000000000000e+02
+6.591870000000000e+02 +9.200000000000000e+01
+9.466020000000000e+02 +2.075000000000000e+02
+6.702950000000000e+02 +9.300000000000000e+01
+8.253700000000000e+02 +5.235000000000000e+02
+1.863510000000000e+03 +4.315000000000000e+02
+1.292930000000000e+03 +1.620000000000000e+02
+1.024290000000000e+03 +2.180000000000000e+02
+1.354500000000000e+03 +4.645000000000000e+02
+6.916439999999999e+02 +1.695000000000000e+02
+1.632540000000000e+03 +3.575000000000000e+02
+8.915720000000000e+02 +1.300000000000000e+02
+6.430830000000002e+02 +3.745000000000000e+02
+1.315840000000000e+03 +2.030000000000000e+02
+8.273630000000000e+01 +5.225000000000000e+02
+1.873890000000000e+03 +5.095000000000000e+02
+8.565720000000000e+01 +5.220000000000000e+02
+1.952480000000000e+03 +3.345000000000000e+02
+9.069890000000000e+02 +1.330000000000000e+02
+8.678260000000000e+02 +1.115000000000000e+02
+4.868270000000000e+02 +1.670000000000000e+02
+1.307620000000000e+03 +2.565000000000000e+02
+8.199750000000000e+02 +1.160000000000000e+02
+1.571050000000000e+03 +2.225000000000000e+02
+1.671220000000000e+03 +3.990000000000000e+02
+9.061150000000000e+02 +1.295000000000000e+02
+1.012540000000000e+03 +5.215000000000000e+02
+1.120600000000000e+03 +5.210000000000000e+02
+1.404070000000000e+03 +2.305000000000000e+02
+6.404720000000000e+02 +8.950000000000000e+01
+5.368960000000000e+02 +5.210000000000000e+02
+1.166440000000000e+03 +2.070000000000000e+02
+7.947200000000000e+02 +2.310000000000000e+02
+5.473230000000000e+02 +5.205000000000000e+02
+1.371240000000000e+03 +4.965000000000000e+02
+9.082850000000000e+02 +1.290000000000000e+02
+6.695939999999998e+02 +9.150000000000000e+01
+8.803670000000000e+02 +1.435000000000000e+02
+6.204000000000000e+01 +5.200000000000000e+02
+1.140600000000000e+03 +2.215000000000000e+02
+9.263720000000000e+02 +2.470000000000000e+02
+1.177090000000000e+03 +5.195000000000000e+02
+5.360050000000000e+02 +5.195000000000000e+02
+7.928040000000000e+01 +5.195000000000000e+02
+2.391990000000000e+01 +5.195000000000000e+02
+9.745760000000000e+02 +1.345000000000000e+02
+1.006750000000000e+03 +2.305000000000000e+02
+6.498860000000000e+02 +7.800000000000000e+01
+9.382060000000000e+02 +2.425000000000000e+02
+7.719390000000000e+02 +5.185000000000000e+02
+1.000620000000000e+03 +3.480000000000000e+02
+8.707930000000000e+02 +1.635000000000000e+02
+1.016310000000000e+03 +1.120000000000000e+02
+1.882760000000000e+03 +4.430000000000000e+02
+6.591139999999998e+02 +1.210000000000000e+02
+4.924690000000000e+01 +5.185000000000000e+02
+9.943500000000000e+02 +1.455000000000000e+02
+6.595790000000000e+02 +1.660000000000000e+02
+6.584870000000000e+02 +1.860000000000000e+02
+1.590200000000000e+03 +2.215000000000000e+02
+1.598830000000000e+03 +3.280000000000000e+02
+2.781780000000000e+02 +2.100000000000000e+01
+5.196490000000000e+02 +5.175000000000000e+02
+7.693550000000000e+02 +1.350000000000000e+02
+1.165260000000000e+03 +5.170000000000000e+02
+1.492050000000000e+03 +3.665000000000000e+02
+1.061870000000000e+03 +2.310000000000000e+02
+7.684760000000001e+02 +1.355000000000000e+02
+6.714240000000000e+02 +5.165000000000000e+02
+1.278810000000000e+03 +2.495000000000000e+02
+9.789660000000000e+02 +1.350000000000000e+02
+1.603920000000000e+03 +4.060000000000000e+02
+7.039780000000002e+02 +5.160000000000000e+02
+1.873010000000000e+03 +3.535000000000000e+02
+7.672930000000000e+02 +5.155000000000000e+02
+7.508819999999999e+02 +5.600000000000000e+01
+8.558049999999999e+02 +5.155000000000000e+02
+6.619880000000001e+02 +1.205000000000000e+02
+7.818480000000002e+02 +1.375000000000000e+02
+1.133490000000000e+03 +2.165000000000000e+02
+1.467760000000000e+03 +4.810000000000000e+02
+1.306570000000000e+03 +2.265000000000000e+02
+6.398940000000000e+02 +5.150000000000000e+02
+1.314020000000000e+03 +2.000000000000000e+02
+6.367560000000000e+02 +1.230000000000000e+02
+1.119100000000000e+03 +4.215000000000000e+02
+1.075940000000000e+03 +2.345000000000000e+02
+5.804660000000000e+02 +5.145000000000000e+02
+6.152390000000000e+02 +1.195000000000000e+02
+1.303020000000000e+03 +1.560000000000000e+02
+1.372340000000000e+03 +2.260000000000000e+02
+9.905290000000000e+02 +1.405000000000000e+02
+1.724410000000000e+03 +3.500000000000000e+02
+1.064890000000000e+03 +2.295000000000000e+02
+6.336910000000000e+01 +5.135000000000000e+02
+7.903560000000001e+02 +1.350000000000000e+02
+1.948250000000000e+03 +4.370000000000000e+02
+1.144840000000000e+03 +5.130000000000000e+02
+6.424540000000002e+02 +1.230000000000000e+02
+8.902020000000000e+02 +3.620000000000000e+02
+8.069390000000000e+02 +2.065000000000000e+02
+6.336120000000000e+02 +8.350000000000000e+01
+1.861130000000000e+03 +4.195000000000000e+02
+2.331010000000000e+03 +3.485000000000000e+02
+2.794370000000000e+02 +1.045000000000000e+02
+1.862890000000000e+03 +4.360000000000000e+02
+8.877220000000000e+02 +1.260000000000000e+02
+1.608140000000000e+03 +2.530000000000000e+02
+1.481410000000000e+03 +2.460000000000000e+02
+6.396380000000000e+02 +5.110000000000000e+02
+1.867870000000000e+03 +4.350000000000000e+02
+2.818490000000000e+02 +1.035000000000000e+02
+6.761560000000002e+02 +1.790000000000000e+02
+7.713280000000000e+02 +1.305000000000000e+02
+7.913650000000000e+02 +2.065000000000000e+02
+1.301330000000000e+03 +1.535000000000000e+02
+9.004430000000000e+02 +1.210000000000000e+02
+1.163610000000000e+03 +2.300000000000000e+02
+4.946500000000000e+02 +2.475000000000000e+02
+1.485300000000000e+03 +2.465000000000000e+02
+6.554000000000000e+02 +8.700000000000000e+01
+1.053620000000000e+03 +2.115000000000000e+02
+1.551610000000000e+03 +2.625000000000000e+02
+8.891350000000000e+02 +4.255000000000000e+02
+1.065430000000000e+03 +2.140000000000000e+02
+6.317930000000000e+02 +1.135000000000000e+02
+5.819730000000000e+01 +5.085000000000000e+02
+1.303600000000000e+03 +1.545000000000000e+02
+1.173270000000000e+03 +5.080000000000000e+02
+6.147010000000000e+02 +8.100000000000000e+01
+7.042580000000000e+02 +1.780000000000000e+02
+7.783049999999999e+02 +1.310000000000000e+02
+4.858830000000000e+02 +5.250000000000000e+01
+2.005210000000000e+03 +3.205000000000000e+02
+1.007330000000000e+03 +2.240000000000000e+02
+9.844960000000000e+02 +1.310000000000000e+02
+1.085030000000000e+03 +1.205000000000000e+02
+1.494680000000000e+03 +2.475000000000000e+02
+2.790550000000000e+02 +1.015000000000000e+02
+6.343650000000000e+02 +7.250000000000000e+01
+6.796530000000000e+02 +5.070000000000000e+02
+1.035970000000000e+03 +1.195000000000000e+02
+1.306620000000000e+03 +2.720000000000000e+02
+4.284390000000000e+01 +5.065000000000000e+02
+6.688180000000000e+02 +5.060000000000000e+02
+7.899310000000000e+02 +1.305000000000000e+02
+9.956190000000000e+02 +1.340000000000000e+02
+8.936480000000000e+02 +4.595000000000000e+02
+4.902590000000000e+01 +5.055000000000000e+02
+1.549410000000000e+03 +2.570000000000000e+02
+8.997689999999999e+02 +1.150000000000000e+02
+6.371110000000000e+02 +8.050000000000000e+01
+7.753610000000001e+02 +1.275000000000000e+02
+1.651680000000000e+03 +3.060000000000000e+02
+6.776980000000000e+02 +1.775000000000000e+02
+5.047880000000000e+01 +5.045000000000000e+02
+1.992970000000000e+03 +3.230000000000000e+02
+1.158310000000000e+03 +2.275000000000000e+02
+1.673260000000000e+03 +3.790000000000000e+02
+1.368470000000000e+03 +4.795000000000000e+02
+9.142640000000000e+02 +1.900000000000000e+02
+1.011690000000000e+03 +2.255000000000000e+02
+9.710160000000000e+02 +2.160000000000000e+02
+1.043830000000000e+03 +2.160000000000000e+02
+7.807730000000000e+02 +1.275000000000000e+02
+1.131210000000000e+03 +5.030000000000000e+02
+1.210530000000000e+03 +1.985000000000000e+02
+4.846010000000000e+02 +2.370000000000000e+02
+6.735189999999999e+02 +5.025000000000000e+02
+8.023090000000000e+02 +5.025000000000000e+02
+7.884980000000000e+02 +1.280000000000000e+02
+1.763900000000000e+03 +3.240000000000000e+02
+1.021330000000000e+03 +9.400000000000000e+01
+4.257450000000000e+01 +5.020000000000000e+02
+7.514130000000000e+02 +1.240000000000000e+02
+1.682510000000000e+03 +3.030000000000000e+02
+1.162190000000000e+03 +4.595000000000000e+02
+1.082540000000000e+03 +2.115000000000000e+02
+1.097680000000000e+03 +2.010000000000000e+02
+9.939850000000000e+02 +1.360000000000000e+02
+6.143290000000002e+02 +6.050000000000000e+01
+1.526170000000000e+03 +2.450000000000000e+02
+6.371240000000000e+02 +7.250000000000000e+01
+1.855040000000000e+03 +4.165000000000000e+02
+1.315710000000000e+03 +1.450000000000000e+02
+9.900630000000000e+02 +1.325000000000000e+02
+1.344140000000000e+03 +4.590000000000000e+02
+8.784030000000000e+02 +1.630000000000000e+02
+1.189840000000000e+03 +1.910000000000000e+02
+1.646820000000000e+03 +2.975000000000000e+02
+1.221590000000000e+03 +1.940000000000000e+02
+1.110700000000000e+03 +1.910000000000000e+02
+9.218370000000000e+02 +1.125000000000000e+02
+8.688240000000000e+02 +2.035000000000000e+02
+2.868760000000000e+02 +1.035000000000000e+02
+6.405910000000000e+02 +4.975000000000000e+02
+7.810460000000000e+02 +1.245000000000000e+02
+6.514740000000000e+02 +6.500000000000000e+01
+6.769190000000000e+02 +1.650000000000000e+02
+3.640710000000000e+01 +4.970000000000000e+02
+1.099480000000000e+03 +2.175000000000000e+02
+6.712430000000001e+02 +6.650000000000000e+01
+1.852100000000000e+03 +4.060000000000000e+02
+1.337090000000000e+03 +4.475000000000000e+02
+8.839370000000000e+02 +4.135000000000000e+02
+1.050750000000000e+03 +2.040000000000000e+02
+7.503150000000001e+02 +1.385000000000000e+02
+6.742160000000000e+02 +1.660000000000000e+02
+9.056300000000000e+02 +7.000000000000000e+01
+4.954180000000000e+02 +2.395000000000000e+02
+6.841760000000000e+02 +4.955000000000000e+02
+4.134860000000000e+02 +4.955000000000000e+02
+1.308430000000000e+03 +1.475000000000000e+02
+1.605990000000000e+03 +2.390000000000000e+02
+8.741960000000000e+02 +4.540000000000000e+02
+1.025400000000000e+03 +9.100000000000000e+01
+1.085850000000000e+03 +1.925000000000000e+02
+6.333819999999999e+02 +6.900000000000000e+01
+7.765110000000002e+02 +1.205000000000000e+02
+1.318440000000000e+03 +1.450000000000000e+02
+1.329770000000000e+03 +4.030000000000000e+02
+1.234770000000000e+03 +4.515000000000000e+02
+4.164240000000000e+02 +4.940000000000000e+02
+9.074840000000000e+02 +1.635000000000000e+02
+1.119910000000000e+03 +4.935000000000000e+02
+1.074050000000000e+03 +2.120000000000000e+02
+6.637040000000000e+02 +6.500000000000000e+01
+4.098810000000000e+02 +4.935000000000000e+02
+8.941760000000000e+02 +6.950000000000000e+01
+1.855810000000000e+03 +4.050000000000000e+02
+7.798339999999999e+02 +1.215000000000000e+02
+9.213460000000000e+02 +2.240000000000000e+02
+6.668789999999998e+02 +1.365000000000000e+02
+9.660510000000000e+02 +1.410000000000000e+02
+6.195190000000000e+02 +1.045000000000000e+02
+6.559470000000000e+02 +4.920000000000000e+02
+9.325520000000000e+02 +1.200000000000000e+02
+1.424770000000000e+03 +2.005000000000000e+02
+7.672370000000000e+02 +4.915000000000000e+02
+6.468370000000000e+02 +7.000000000000000e+01
+9.997960000000000e+02 +2.160000000000000e+02
+8.742850000000000e+02 +1.990000000000000e+02
+1.787650000000000e+03 +3.305000000000000e+02
+7.775740000000000e+02 +1.185000000000000e+02
+9.846470000000000e+02 +1.270000000000000e+02
+1.302120000000000e+03 +4.905000000000000e+02
+3.611970000000000e+01 +4.900000000000000e+02
+7.932470000000000e+02 +1.210000000000000e+02
+6.785520000000000e+02 +1.445000000000000e+02
+7.728000000000000e+02 +4.895000000000000e+02
+1.552700000000000e+03 +2.460000000000000e+02
+1.140800000000000e+03 +4.740000000000000e+02
+1.673490000000000e+03 +3.255000000000000e+02
+6.491130000000001e+02 +1.645000000000000e+02
+6.827089999999999e+02 +4.890000000000000e+02
+6.228310000000000e+02 +9.600000000000000e+01
+4.392180000000000e+01 +4.890000000000000e+02
+7.778739999999998e+02 +1.195000000000000e+02
+9.718440000000001e+02 +2.025000000000000e+02
+3.966480000000000e+02 +4.885000000000000e+02
+1.575230000000000e+03 +3.550000000000000e+02
+6.213020000000000e+02 +9.100000000000000e+01
+6.452950000000000e+02 +1.595000000000000e+02
+5.158880000000000e+02 +4.875000000000000e+02
+1.338780000000000e+03 +4.650000000000000e+02
+9.019890000000000e+02 +1.085000000000000e+02
+1.237290000000000e+03 +1.815000000000000e+02
+2.814160000000000e+02 +1.085000000000000e+02
+8.992610000000002e+02 +1.050000000000000e+02
+6.466720000000000e+02 +4.860000000000000e+02
+1.669650000000000e+03 +4.525000000000000e+02
+5.054780000000000e+02 +4.855000000000000e+02
+1.004700000000000e+03 +2.135000000000000e+02
+9.137350000000000e+02 +2.195000000000000e+02
+1.072440000000000e+03 +2.125000000000000e+02
+6.367010000000000e+02 +1.570000000000000e+02
+4.259630000000000e+02 +4.850000000000000e+02
+1.865520000000000e+03 +3.985000000000000e+02
+7.740939999999998e+02 +1.175000000000000e+02
+1.071550000000000e+03 +1.925000000000000e+02
+6.488060000000000e+02 +4.845000000000000e+02
+1.488930000000000e+03 +2.725000000000000e+02
+8.687960000000000e+02 +1.905000000000000e+02
+6.004940000000000e+02 +4.845000000000000e+02
+8.654000000000000e+02 +1.255000000000000e+02
+9.736900000000001e+02 +1.915000000000000e+02
+9.019810000000000e+02 +9.900000000000000e+01
+5.018080000000000e+02 +4.840000000000000e+02
+4.935080000000000e+02 +2.330000000000000e+02
+1.092800000000000e+03 +2.085000000000000e+02
+8.770980000000002e+02 +1.200000000000000e+02
+1.088270000000000e+03 +2.045000000000000e+02
+7.001220000000000e+02 +2.555000000000000e+02
+3.485000000000000e+02 +4.830000000000000e+02
+5.240850000000000e+02 +4.830000000000000e+02
+1.056330000000000e+03 +1.965000000000000e+02
+1.707620000000000e+03 +4.825000000000000e+02
+6.636910000000000e+02 +9.000000000000000e+01
+7.727550000000000e+02 +1.155000000000000e+02
+7.814639999999998e+02 +1.155000000000000e+02
+1.375460000000000e+03 +4.635000000000000e+02
+1.453520000000000e+03 +4.500000000000000e+02
+3.345260000000000e+02 +9.950000000000000e+01
+1.771020000000000e+03 +4.065000000000000e+02
+1.098330000000000e+03 +3.905000000000000e+02
+8.926469999999998e+02 +1.695000000000000e+02
+9.048140000000000e+02 +1.200000000000000e+02
+1.023370000000000e+03 +4.810000000000000e+02
+6.951089999999998e+02 +2.680000000000000e+02
+2.778570000000000e+02 +8.600000000000000e+01
+5.207700000000000e+02 +4.805000000000000e+02
+7.878320000000000e+02 +1.145000000000000e+02
+1.012180000000000e+03 +4.805000000000000e+02
+6.361260000000000e+02 +1.500000000000000e+02
+9.023490000000000e+02 +9.650000000000000e+01
+1.610970000000000e+03 +2.240000000000000e+02
+1.759900000000000e+03 +4.620000000000000e+02
+6.354820000000000e+02 +4.800000000000000e+02
+5.606190000000000e+02 +4.795000000000000e+02
+9.911630000000000e+02 +1.870000000000000e+02
+9.094030000000000e+02 +1.095000000000000e+02
+9.367530000000000e+02 +2.170000000000000e+02
+7.580020000000000e+02 +4.790000000000000e+02
+3.454110000000000e+02 +4.790000000000000e+02
+2.994060000000000e+01 +4.790000000000000e+02
+4.862930000000000e+02 +2.195000000000000e+02
+6.406890000000000e+02 +1.525000000000000e+02
+7.402180000000002e+02 +4.780000000000000e+02
+5.000160000000000e+02 +4.780000000000000e+02
+8.874800000000000e+02 +4.780000000000000e+02
+1.077100000000000e+03 +1.925000000000000e+02
+4.816670000000000e+02 +4.775000000000000e+02
+5.584159999999998e+02 +4.775000000000000e+02
+1.288130000000000e+03 +2.195000000000000e+02
+7.726130000000001e+02 +1.120000000000000e+02
+1.292420000000000e+03 +2.260000000000000e+02
+9.286700000000000e+02 +2.110000000000000e+02
+4.905370000000000e+02 +4.770000000000000e+02
+8.058330000000002e+02 +4.770000000000000e+02
+1.043900000000000e+03 +1.760000000000000e+02
+1.227890000000000e+03 +4.285000000000000e+02
+1.008490000000000e+03 +3.115000000000000e+02
+7.117189999999998e+02 +1.430000000000000e+02
+2.743620000000000e+01 +4.765000000000000e+02
+9.600450000000000e+02 +4.765000000000000e+02
+1.174400000000000e+03 +3.465000000000000e+02
+8.957689999999999e+02 +4.430000000000000e+02
+1.018640000000000e+03 +3.155000000000000e+02
+1.304460000000000e+03 +2.290000000000000e+02
+5.291430000000000e+02 +4.755000000000000e+02
+4.283100000000000e+02 +4.755000000000000e+02
+5.297400000000000e+02 +4.755000000000000e+02
+1.002670000000000e+03 +3.065000000000000e+02
+8.110230000000000e+02 +2.080000000000000e+02
+1.572390000000000e+03 +1.870000000000000e+02
+1.556190000000000e+03 +3.945000000000000e+02
+1.411110000000000e+03 +4.750000000000000e+02
+6.437880000000000e+02 +1.070000000000000e+02
+4.814730000000000e+02 +4.745000000000000e+02
+7.690080000000000e+02 +1.115000000000000e+02
+1.304420000000000e+03 +2.290000000000000e+02
+9.748150000000001e+02 +4.745000000000000e+02
+7.708350000000000e+02 +4.745000000000000e+02
+9.933330000000000e+02 +1.875000000000000e+02
+3.538970000000000e+02 +4.740000000000000e+02
+4.825150000000000e+02 +4.740000000000000e+02
+3.468260000000000e+00 +4.735000000000000e+02
+9.859690000000001e+02 +1.845000000000000e+02
+5.507750000000000e+02 +4.730000000000000e+02
+2.820110000000000e+02 +8.100000000000000e+01
+5.235269999999998e+02 +4.730000000000000e+02
+7.737239999999998e+02 +1.120000000000000e+02
+1.778190000000000e+03 +2.935000000000000e+02
+1.382410000000000e+03 +2.550000000000000e+02
+4.257780000000000e+02 +4.725000000000000e+02
+6.439430000000000e+02 +1.495000000000000e+02
+7.035010000000002e+02 +2.590000000000000e+02
+1.716700000000000e+03 +4.720000000000000e+02
+5.176090000000000e+02 +4.715000000000000e+02
+6.579030000000000e+02 +1.535000000000000e+02
+7.899530000000000e+02 +1.125000000000000e+02
+1.066290000000000e+03 +1.965000000000000e+02
+1.880930000000000e+03 +3.920000000000000e+02
+7.455520000000000e+02 +1.075000000000000e+02
+4.872180000000000e+02 +2.175000000000000e+02
+1.079760000000000e+03 +1.925000000000000e+02
+4.373230000000000e+02 +4.705000000000000e+02
+4.230570000000000e+01 +4.705000000000000e+02
+7.777260000000001e+02 +1.090000000000000e+02
+9.133860000000000e+02 +2.060000000000000e+02
+1.116460000000000e+03 +4.700000000000000e+02
+9.941380000000000e+02 +1.860000000000000e+02
+7.467320000000000e+02 +1.140000000000000e+02
+6.570680000000000e+02 +1.435000000000000e+02
+3.154460000000000e+02 +4.700000000000000e+02
+4.961670000000000e+02 +2.240000000000000e+02
+7.454410000000000e+02 +4.695000000000000e+02
+6.967210000000000e+02 +1.730000000000000e+02
+7.805169999999998e+02 +1.095000000000000e+02
+3.358990000000000e+02 +9.150000000000000e+01
+1.145270000000000e+03 +1.830000000000000e+02
+7.006770000000000e+02 +2.450000000000000e+02
+6.669420000000000e+02 +1.410000000000000e+02
+5.840910000000000e+02 +4.690000000000000e+02
+6.190490000000000e+02 +7.850000000000000e+01
+6.259990000000000e+02 +4.685000000000000e+02
+1.775880000000000e+03 +3.965000000000000e+02
+9.891980000000000e+02 +4.685000000000000e+02
+2.607350000000000e+02 +2.050000000000000e+01
+9.033530000000000e+02 +4.380000000000000e+02
+1.054890000000000e+03 +1.800000000000000e+02
+6.247650000000000e+02 +1.480000000000000e+02
+6.591230000000000e+02 +4.680000000000000e+02
+6.305830000000002e+02 +6.050000000000000e+01
+4.693670000000000e+02 +4.670000000000000e+02
+1.863250000000000e+03 +3.930000000000000e+02
+1.563950000000000e+03 +3.065000000000000e+02
+4.899990000000000e+02 +2.255000000000000e+02
+7.041400000000000e+02 +2.555000000000000e+02
+1.300500000000000e+03 +2.205000000000000e+02
+9.181400000000000e+02 +9.900000000000000e+01
+9.975130000000000e+02 +1.795000000000000e+02
+1.985750000000000e+03 +3.915000000000000e+02
+6.088200000000001e+02 +4.655000000000000e+02
+4.566630000000000e+02 +4.650000000000000e+02
+1.072240000000000e+03 +1.815000000000000e+02
+1.502220000000000e+01 +4.645000000000000e+02
+7.728989999999999e+02 +1.045000000000000e+02
+9.110850000000000e+02 +3.030000000000000e+02
+7.447060000000000e+02 +4.640000000000000e+02
+8.740250000000000e+02 +2.120000000000000e+02
+3.424550000000000e+02 +4.640000000000000e+02
+9.894380000000000e+02 +1.825000000000000e+02
+1.228890000000000e+03 +1.820000000000000e+02
+7.471310000000002e+02 +4.635000000000000e+02
+7.032500000000000e+02 +2.425000000000000e+02
+6.636610000000002e+02 +7.350000000000000e+01
+2.978260000000000e+01 +4.635000000000000e+02
+9.947820000000000e+02 +1.785000000000000e+02
+1.015540000000000e+03 +4.635000000000000e+02
+4.566750000000000e+02 +4.800000000000000e+01
+1.009900000000000e+03 +1.805000000000000e+02
+1.376930000000000e+03 +2.460000000000000e+02
+1.028250000000000e+03 +4.630000000000000e+02
+8.819040000000000e+02 +2.085000000000000e+02
+4.936950000000000e+02 +4.625000000000000e+02
+1.683470000000000e+03 +3.565000000000000e+02
+1.230340000000000e+03 +2.975000000000000e+02
+7.087050000000000e+02 +3.570000000000000e+02
+1.754350000000000e+03 +4.620000000000000e+02
+1.019970000000000e+03 +2.930000000000000e+02
+2.905990000000000e+02 +4.615000000000000e+02
+8.927160000000000e+02 +8.500000000000000e+01
+9.367130000000000e+02 +3.040000000000000e+02
+2.730020000000000e+02 +4.610000000000000e+02
+2.126800000000000e+01 +4.610000000000000e+02
+7.711910000000000e+02 +1.040000000000000e+02
+1.594110000000000e+03 +2.670000000000000e+02
+6.208550000000000e+02 +4.605000000000000e+02
+7.683600000000000e+02 +1.190000000000000e+02
+1.015310000000000e+03 +4.605000000000000e+02
+7.045060000000002e+02 +1.600000000000000e+02
+2.577980000000000e+01 +4.600000000000000e+02
+3.101790000000000e+01 +4.595000000000000e+02
+4.703480000000000e+02 +4.590000000000000e+02
+2.822410000000000e+02 +7.150000000000000e+01
+6.443060000000000e+02 +1.325000000000000e+02
+3.233170000000000e+01 +4.590000000000000e+02
+1.184370000000000e+03 +3.540000000000000e+02
+4.794230000000000e+02 +4.500000000000000e+01
+7.145780000000000e+02 +2.445000000000000e+02
+8.068830000000000e+02 +1.700000000000000e+02
+4.237350000000000e+02 +4.575000000000000e+02
+1.859740000000000e+03 +3.795000000000000e+02
+7.781239999999998e+02 +1.025000000000000e+02
+6.388009999999998e+02 +1.330000000000000e+02
+8.754930000000001e+02 +1.180000000000000e+02
+6.249760000000000e+02 +6.850000000000000e+01
+7.460790000000000e+02 +1.050000000000000e+02
+3.803940000000000e+02 +4.565000000000000e+02
+1.300880000000000e+03 +2.140000000000000e+02
+5.461430000000000e+02 +4.560000000000000e+02
+6.232470000000000e+02 +6.950000000000000e+01
+1.223850000000000e+03 +2.900000000000000e+02
+4.034470000000000e+02 +4.555000000000000e+02
+4.899090000000000e+02 +3.010000000000000e+02
+7.660560000000000e+02 +9.750000000000000e+01
+9.859700000000000e+02 +1.800000000000000e+02
+1.022600000000000e+03 +2.890000000000000e+02
+7.118780000000000e+02 +4.545000000000000e+02
+8.233720000000000e+02 +1.530000000000000e+02
+5.648180000000000e+02 +4.545000000000000e+02
+7.817250000000000e+02 +1.020000000000000e+02
+4.228530000000000e+02 +4.150000000000000e+01
+6.959700000000000e+02 +4.540000000000000e+02
+1.009730000000000e+03 +2.870000000000000e+02
+6.571439999999999e+02 +2.405000000000000e+02
+6.284290000000000e+02 +6.850000000000000e+01
+1.337130000000000e+03 +4.345000000000000e+02
+5.035820000000000e+02 +3.025000000000000e+02
+6.380369999999998e+02 +1.375000000000000e+02
+6.951330000000000e+02 +1.510000000000000e+02
+1.321730000000000e+03 +4.310000000000000e+02
+5.106490000000000e+02 +3.035000000000000e+02
+7.823610000000001e+02 +1.485000000000000e+02
+4.964130000000000e+02 +4.530000000000000e+02
+6.595020000000000e+02 +6.450000000000000e+01
+8.015670000000000e+02 +4.530000000000000e+02
+5.508250000000000e+02 +4.525000000000000e+02
+1.499680000000000e+03 +3.205000000000000e+02
+2.783400000000000e+02 +5.100000000000000e+01
+4.763000000000000e+02 +4.525000000000000e+02
+7.849130000000000e+02 +1.010000000000000e+02
+8.998930000000000e+02 +8.750000000000000e+01
+1.004290000000000e+03 +4.525000000000000e+02
+4.824550000000000e+02 +2.040000000000000e+02
+1.353140000000000e+03 +2.930000000000000e+02
+1.338920000000000e+03 +4.340000000000000e+02
+8.294930000000001e+02 +4.520000000000000e+02
+6.189660000000000e+02 +4.515000000000000e+02
+4.718240000000000e+02 +4.515000000000000e+02
+1.574490000000000e+03 +2.955000000000000e+02
+1.150730000000000e+03 +2.495000000000000e+02
+1.014340000000000e+03 +2.820000000000000e+02
+8.621960000000000e+02 +2.030000000000000e+02
+6.043860000000000e+02 +1.285000000000000e+02
+3.817430000000001e+02 +4.510000000000000e+02
+1.106350000000000e+03 +1.510000000000000e+02
+9.891570000000000e+02 +1.665000000000000e+02
+6.730119999999999e+02 +1.880000000000000e+02
+7.838339999999999e+02 +9.500000000000000e+01
+7.020300000000000e+02 +1.490000000000000e+02
+1.535310000000000e+01 +4.500000000000000e+02
+5.051720000000000e+02 +3.055000000000000e+02
+1.577790000000000e+01 +4.495000000000000e+02
+1.299790000000000e+03 +2.110000000000000e+02
+1.779290000000000e+03 +3.830000000000000e+02
+4.187500000000000e+02 +4.490000000000000e+02
+5.232730000000000e+02 +4.485000000000000e+02
+8.586900000000001e+02 +1.075000000000000e+02
+1.093370000000000e+03 +1.520000000000000e+02
+8.476990000000000e+02 +8.950000000000000e+01
+5.435330000000000e+02 +4.480000000000000e+02
+4.861990000000000e+02 +2.055000000000000e+02
+1.270080000000000e+03 +4.480000000000000e+02
+1.310110000000000e+03 +4.320000000000000e+02
+2.818770000000000e+02 +4.800000000000000e+01
+4.527910000000000e+02 +4.475000000000000e+02
+6.186369999999999e+02 +6.050000000000000e+01
+7.782990000000000e+02 +9.300000000000000e+01
+1.382050000000000e+03 +4.340000000000000e+02
+8.893360000000000e+02 +9.050000000000000e+01
+1.611540000000000e+03 +3.035000000000000e+02
+2.948230000000000e+02 +5.450000000000000e+01
+1.309030000000000e+03 +3.185000000000000e+02
+1.015790000000000e+03 +4.470000000000000e+02
+6.516319999999999e+02 +1.260000000000000e+02
+4.777130000000000e+02 +4.465000000000000e+02
+1.177320000000000e+01 +4.465000000000000e+02
+9.460090000000000e+02 +4.465000000000000e+02
+8.965690000000000e+02 +2.360000000000000e+02
+4.992450000000000e+02 +2.900000000000000e+02
+1.332600000000000e+03 +3.200000000000000e+02
+9.792690000000000e+02 +1.780000000000000e+02
+1.081120000000000e+03 +1.550000000000000e+02
+9.768030000000000e+02 +2.050000000000000e+02
+3.740970000000000e+02 +4.450000000000000e+02
+8.948919999999998e+02 +8.200000000000000e+01
+9.985640000000000e+02 +1.665000000000000e+02
+7.573550000000000e+02 +9.550000000000000e+01
+7.705730000000000e+02 +9.450000000000000e+01
+6.430540000000000e+02 +4.300000000000000e+01
+4.997500000000000e+02 +2.910000000000000e+02
+4.229730000000000e+02 +4.440000000000000e+02
+7.806089999999998e+02 +9.250000000000000e+01
+1.109140000000000e+03 +1.530000000000000e+02
+7.095730000000000e+02 +3.375000000000000e+02
+7.639620000000000e+02 +9.700000000000000e+01
+2.336120000000000e+02 +4.435000000000000e+02
+3.036640000000000e+02 +6.150000000000000e+01
+4.386270000000000e+02 +4.430000000000000e+02
+1.563550000000000e+03 +2.305000000000000e+02
+2.443910000000000e+01 +4.430000000000000e+02
+7.581260000000002e+02 +9.000000000000000e+01
+1.748250000000000e+03 +4.150000000000000e+02
+4.234020000000000e+02 +3.350000000000000e+01
+1.068210000000000e+03 +2.740000000000000e+02
+2.662660000000000e+02 +4.425000000000000e+02
+1.328530000000000e+03 +3.165000000000000e+02
+8.635920000000000e+02 +8.650000000000000e+01
+1.752070000000000e+01 +4.420000000000000e+02
+7.842130000000002e+02 +9.400000000000000e+01
+1.957050000000000e+03 +3.580000000000000e+02
+9.040870000000000e+02 +2.340000000000000e+02
+1.063500000000000e+03 +2.560000000000000e+02
+6.495830000000002e+02 +1.240000000000000e+02
+9.187790000000000e+02 +2.910000000000000e+02
+3.887540000000000e+02 +4.410000000000000e+02
+6.237040000000002e+02 +5.600000000000000e+01
+8.698850000000000e+02 +2.290000000000000e+02
+1.626570000000000e+03 +2.950000000000000e+02
+1.661640000000000e+03 +4.025000000000000e+02
+1.773740000000000e+03 +3.705000000000000e+02
+1.066460000000000e+03 +1.970000000000000e+02
+6.132520000000000e+02 +4.395000000000000e+02
+7.007639999999999e+02 +1.520000000000000e+02
+7.067800000000000e+02 +2.060000000000000e+02
+7.580540000000000e+02 +9.700000000000000e+01
+6.414150000000000e+02 +1.260000000000000e+02
+7.092960000000000e+02 +1.340000000000000e+02
+8.999100000000000e+02 +7.200000000000000e+01
+1.069040000000000e+03 +2.540000000000000e+02
+4.175550000000000e+02 +5.550000000000000e+01
+7.744400000000001e+02 +8.950000000000000e+01
+9.885020000000000e+02 +1.565000000000000e+02
+4.437130000000000e+02 +4.370000000000000e+02
+7.817500000000000e+02 +8.900000000000000e+01
+6.205850000000000e+02 +1.555000000000000e+02
+7.741660000000001e+02 +9.200000000000000e+01
+9.627990000000000e+02 +1.590000000000000e+02
+1.064200000000000e+03 +1.330000000000000e+02
+1.781960000000000e+03 +2.545000000000000e+02
+8.239290000000000e+02 +1.505000000000000e+02
+6.541540000000000e+02 +1.135000000000000e+02
+5.873440000000001e+02 +4.360000000000000e+02
+1.866720000000000e+03 +3.640000000000000e+02
+1.317010000000000e+03 +1.985000000000000e+02
+4.858900000000000e+02 +4.355000000000000e+02
+6.226870000000000e+02 +1.540000000000000e+02
+7.451369999999999e+02 +8.600000000000000e+01
+1.563690000000000e+03 +2.805000000000000e+02
+1.097460000000000e+03 +1.390000000000000e+02
+8.508049999999999e+02 +3.535000000000000e+02
+6.364540000000002e+02 +1.105000000000000e+02
+3.351170000000000e+02 +4.350000000000000e+02
+4.117570000000000e+02 +4.350000000000000e+02
+1.316270000000000e+03 +1.930000000000000e+02
+9.233170000000000e+02 +4.350000000000000e+02
+6.628170000000000e+02 +1.410000000000000e+02
+1.177190000000000e+03 +4.345000000000000e+02
+9.096180000000001e+02 +9.000000000000000e+01
+9.806060000000000e+02 +2.670000000000000e+02
+4.746810000000000e+02 +4.340000000000000e+02
+1.064200000000000e+03 +2.500000000000000e+02
+6.397550000000000e+02 +1.185000000000000e+02
+1.172060000000000e+03 +3.250000000000000e+02
+6.071060000000000e+02 +4.335000000000000e+02
+8.501760000000000e+02 +7.600000000000000e+01
+2.957360000000000e+02 +5.150000000000000e+01
+6.623950000000000e+02 +1.415000000000000e+02
+6.552040000000002e+02 +1.225000000000000e+02
+4.915100000000000e+02 +4.330000000000000e+02
+1.076570000000000e+03 +2.510000000000000e+02
+1.313740000000000e+03 +3.110000000000000e+02
+8.775920000000000e+02 +7.900000000000000e+01
+6.687239999999998e+02 +2.130000000000000e+02
+4.504030000000000e+02 +4.320000000000000e+02
+1.271780000000000e+03 +2.800000000000000e+02
+7.773150000000001e+02 +8.850000000000000e+01
+8.983220000000000e+02 +7.750000000000000e+01
+1.026810000000000e+03 +2.625000000000000e+02
+5.733099999999999e+02 +4.315000000000000e+02
+1.558810000000000e+03 +2.585000000000000e+02
+6.622780000000000e+02 +1.515000000000000e+02
+9.294330000000000e+02 +4.315000000000000e+02
+1.066980000000000e+03 +2.485000000000000e+02
+2.777000000000000e+02 +4.000000000000000e+01
+2.209510000000000e+02 +4.310000000000000e+02
+4.339080000000000e+02 +4.310000000000000e+02
+7.741289999999998e+02 +8.600000000000000e+01
+1.300390000000000e+03 +1.985000000000000e+02
+1.610970000000000e+03 +3.510000000000000e+02
+2.927940000000001e+02 +4.300000000000000e+02
+1.858940000000000e+03 +3.545000000000000e+02
+7.605060000000001e+00 +4.300000000000000e+02
+7.789360000000000e+02 +8.800000000000000e+01
+1.133460000000000e+03 +3.860000000000000e+02
+6.407950000000000e+02 +1.090000000000000e+02
+6.224259999999998e+02 +1.500000000000000e+02
+1.137240000000000e+03 +4.295000000000000e+02
+1.037040000000000e+03 +3.260000000000000e+02
+1.086540000000000e+03 +2.825000000000000e+02
+7.183530000000002e+02 +2.195000000000000e+02
+7.809860000000001e+02 +8.650000000000000e+01
+1.301830000000000e+03 +1.890000000000000e+02
+9.892320000000000e+02 +4.290000000000000e+02
+8.656750000000000e+02 +1.405000000000000e+02
+9.053610000000000e+02 +2.230000000000000e+02
+5.141990000000002e+02 +4.285000000000000e+02
+9.942210000000000e+02 +1.675000000000000e+02
+1.623050000000000e+03 +3.210000000000000e+02
+7.834380000000000e+02 +8.700000000000000e+01
+1.610040000000000e+03 +4.275000000000000e+02
+6.882610000000002e+02 +1.180000000000000e+02
+4.832080000000000e+02 +2.800000000000000e+02
+1.063520000000000e+03 +2.455000000000000e+02
+5.288510000000000e+02 +4.270000000000000e+02
+1.055760000000000e+03 +1.395000000000000e+02
+9.261310000000000e+02 +3.920000000000000e+02
+1.073250000000000e+03 +2.565000000000000e+02
+6.346250000000000e+02 +1.175000000000000e+02
+5.009630000000000e+02 +2.880000000000000e+02
+2.660650000000000e+02 +4.255000000000000e+02
+1.310120000000000e+03 +1.925000000000000e+02
+7.778750000000000e+02 +4.255000000000000e+02
+1.324910000000000e+03 +3.005000000000000e+02
+7.789010000000002e+02 +8.350000000000000e+01
+9.980650000000001e+02 +1.540000000000000e+02
+7.802530000000000e+02 +4.245000000000000e+02
+2.006610000000000e+02 +4.245000000000000e+02
+3.933390000000000e+02 +4.240000000000000e+02
+1.306350000000000e+01 +4.240000000000000e+02
+7.747970000000000e+02 +1.590000000000000e+02
+1.297350000000000e+03 +1.850000000000000e+02
+1.027900000000000e+03 +1.350000000000000e+02
+6.973070000000000e+02 +1.890000000000000e+02
+8.636760000000000e+02 +1.845000000000000e+02
+7.859620000000000e+02 +8.350000000000000e+01
+9.747850000000000e+02 +4.235000000000000e+02
+1.778470000000000e+03 +2.465000000000000e+02
+4.976240000000000e+02 +2.830000000000000e+02
+4.926270000000000e+02 +4.500000000000000e+01
+6.002180000000002e+02 +4.230000000000000e+02
+9.124990000000000e+02 +1.150000000000000e+02
+1.628190000000000e+03 +2.835000000000000e+02
+5.489970000000000e+02 +4.220000000000000e+02
+7.839830000000002e+02 +1.580000000000000e+02
+1.113670000000000e+03 +4.220000000000000e+02
+4.185930000000000e+02 +4.215000000000000e+02
+1.273540000000000e+03 +2.740000000000000e+02
+1.860420000000000e+03 +3.515000000000000e+02
+2.676790000000000e+02 +4.210000000000000e+02
+1.684160000000000e+03 +4.005000000000000e+02
+9.038720000000000e+02 +1.140000000000000e+02
+9.101680000000000e+02 +2.110000000000000e+02
+9.944970000000000e+02 +1.305000000000000e+02
+1.135040000000000e+03 +1.310000000000000e+02
+1.486250000000000e+03 +2.855000000000000e+02
+4.208060000000000e+02 +4.200000000000000e+02
+8.345219999999999e+00 +4.200000000000000e+02
+1.402710000000000e+03 +4.200000000000000e+02
+1.911740000000000e+03 +2.730000000000000e+02
+1.818440000000000e+03 +4.200000000000000e+02
+6.794730000000002e+02 +9.800000000000000e+01
+1.626990000000000e+03 +4.005000000000000e+02
+2.577850000000000e+02 +4.190000000000000e+02
+6.401250000000000e+02 +9.450000000000000e+01
+1.072430000000000e+03 +2.400000000000000e+02
+5.942530000000000e+02 +4.185000000000000e+02
+8.793260000000000e+02 +4.185000000000000e+02
+7.510230000000000e+02 +1.280000000000000e+02
+3.917540000000000e+02 +4.180000000000000e+02
+1.303900000000000e+03 +1.825000000000000e+02
+1.114160000000000e+03 +1.230000000000000e+02
+9.349380000000000e+02 +4.175000000000000e+02
+6.453880000000000e+02 +1.050000000000000e+02
+8.651790000000000e+02 +9.250000000000000e+01
+1.750980000000000e+03 +3.880000000000000e+02
+1.363620000000000e+03 +2.635000000000000e+02
+6.629030000000000e+02 +1.055000000000000e+02
+1.754900000000000e+01 +4.165000000000000e+02
+9.828470000000000e+02 +1.565000000000000e+02
+1.089090000000000e+03 +1.345000000000000e+02
+6.351690000000000e+02 +1.090000000000000e+02
+1.680800000000000e+03 +3.915000000000000e+02
+5.057490000000000e+02 +4.155000000000000e+02
+1.135580000000000e+03 +4.155000000000000e+02
+8.077769999999998e+02 +4.155000000000000e+02
+6.108830000000000e+02 +1.010000000000000e+02
+5.663030000000000e+02 +4.150000000000000e+02
+1.189910000000000e+03 +2.805000000000000e+02
+9.384070000000000e+02 +3.860000000000000e+02
+8.817890000000000e+02 +1.820000000000000e+02
+6.556820000000000e+02 +1.050000000000000e+02
+3.024490000000000e+02 +4.145000000000000e+02
+1.666610000000000e+03 +3.880000000000000e+02
+1.302520000000000e+03 +2.645000000000000e+02
+7.671180000000001e+02 +1.530000000000000e+02
+9.943930000000000e+02 +1.210000000000000e+02
+6.510150000000000e+02 +1.080000000000000e+02
+2.456280000000000e+02 +4.135000000000000e+02
+1.606370000000000e+03 +4.135000000000000e+02
+7.688049999999999e+02 +1.490000000000000e+02
+1.116700000000000e+03 +4.135000000000000e+02
+9.066369999999999e+02 +1.350000000000000e+02
+5.687750000000000e+02 +4.130000000000000e+02
+2.326400000000000e+02 +4.125000000000000e+02
+6.210010000000000e+02 +1.355000000000000e+02
+9.352230000000000e+02 +3.870000000000000e+02
+4.255320000000000e+02 +4.120000000000000e+02
+1.091530000000000e+03 +2.530000000000000e+02
+8.628889999999999e+02 +8.500000000000000e+01
+9.898160000000000e+02 +1.365000000000000e+02
+9.256079999999999e+02 +2.080000000000000e+02
+1.611860000000000e+03 +4.115000000000000e+02
+7.775610000000000e+02 +1.485000000000000e+02
+5.298190000000000e+02 +4.110000000000000e+02
+1.300420000000000e+03 +1.750000000000000e+02
+3.725490000000000e+02 +4.105000000000000e+02
+6.562730000000000e+02 +2.255000000000000e+02
+2.128540000000000e+02 +4.105000000000000e+02
+1.095940000000000e+03 +1.155000000000000e+02
+1.370250000000000e+03 +1.960000000000000e+02
+1.320660000000000e+03 +1.800000000000000e+02
+9.001600000000000e+02 +1.235000000000000e+02
+1.263360000000000e+03 +1.455000000000000e+02
+6.528220000000000e+02 +9.250000000000000e+01
+9.909760000000000e+02 +1.435000000000000e+02
+6.507260000000000e+02 +1.055000000000000e+02
+7.886360000000002e+02 +1.495000000000000e+02
+1.540190000000000e+03 +2.515000000000000e+02
+9.056690000000000e+02 +1.240000000000000e+02
+3.979520000000000e+02 +4.085000000000000e+02
+6.176460000000000e+02 +8.950000000000000e+01
+4.053740000000000e+02 +4.085000000000000e+02
+1.006600000000000e+03 +3.575000000000000e+02
+8.021770000000000e+00 +4.080000000000000e+02
+7.675139999999999e+02 +1.485000000000000e+02
+1.255750000000000e+03 +2.395000000000000e+02
+1.288190000000000e+03 +2.530000000000000e+02
+4.025250000000000e+02 +4.075000000000000e+02
+1.320460000000000e+01 +4.075000000000000e+02
+6.636010000000001e+02 +1.050000000000000e+02
+3.707200000000000e+02 +4.060000000000000e+02
+1.733810000000000e+03 +3.795000000000000e+02
+5.698120000000000e+02 +4.055000000000000e+02
+6.235150000000000e+02 +1.285000000000000e+02
+5.849400000000001e+02 +4.050000000000000e+02
+7.785060000000002e+02 +1.465000000000000e+02
+1.313310000000000e+03 +1.710000000000000e+02
+2.776000000000000e+02 +4.045000000000000e+02
+4.103670000000000e+02 +4.040000000000000e+02
+1.270850000000000e+01 +4.040000000000000e+02
+7.790590000000000e+02 +1.425000000000000e+02
+9.191770000000000e+02 +1.960000000000000e+02
+1.016280000000000e+03 +2.370000000000000e+02
+2.789650000000000e+00 +4.035000000000000e+02
+7.844889999999998e+02 +1.450000000000000e+02
+8.588220000000000e+02 +1.235000000000000e+02
+1.208340000000000e+03 +2.115000000000000e+02
+7.576480000000000e+02 +1.400000000000000e+02
+3.892970000000000e+02 +4.025000000000000e+02
+1.235950000000000e+03 +4.025000000000000e+02
+6.803670000000000e+02 +1.770000000000000e+02
+1.219320000000000e+03 +2.040000000000000e+02
+3.330470000000000e+02 +4.015000000000000e+02
+1.008970000000000e+03 +2.370000000000000e+02
+6.381200000000000e+02 +9.600000000000000e+01
+1.767390000000000e+03 +2.260000000000000e+02
+5.915940000000001e+02 +4.010000000000000e+02
+3.488570000000000e+02 +4.010000000000000e+02
+7.874660000000000e+02 +1.440000000000000e+02
+8.617800000000000e+02 +1.625000000000000e+02
+7.498650000000000e+02 +5.550000000000000e+01
+4.884960000000000e+02 +4.005000000000000e+02
+7.796630000000000e+02 +1.400000000000000e+02
+9.845710000000000e+02 +1.380000000000000e+02
+3.165290000000000e+02 +4.000000000000000e+02
+1.304850000000000e+03 +1.695000000000000e+02
+1.775780000000000e+03 +2.280000000000000e+02
+4.844880000000001e+02 +3.850000000000000e+01
+5.489830000000002e+02 +3.995000000000000e+02
+4.124000000000000e+02 +3.995000000000000e+02
+1.108370000000000e+03 +3.995000000000000e+02
+7.035889999999998e+02 +2.975000000000000e+02
+5.229590000000002e+02 +3.990000000000000e+02
+1.282230000000000e+03 +1.655000000000000e+02
+1.232630000000000e+03 +3.060000000000000e+02
+5.295419999999998e+02 +3.985000000000000e+02
+3.812870000000000e+02 +3.985000000000000e+02
+1.868980000000000e+03 +3.170000000000000e+02
+1.308340000000000e+01 +3.980000000000000e+02
+3.161210000000000e+02 +3.975000000000000e+02
+7.034260000000000e+02 +1.825000000000000e+02
+8.720060000000002e+02 +1.665000000000000e+02
+5.057890000000000e+02 +3.975000000000000e+02
+1.630090000000000e+03 +3.535000000000000e+02
+3.663840000000000e+02 +3.975000000000000e+02
+6.955490000000000e+02 +3.970000000000000e+02
+4.202620000000000e+02 +3.970000000000000e+02
+3.640370000000000e-01 +3.970000000000000e+02
+7.864639999999998e+02 +1.415000000000000e+02
+8.466400000000000e+02 +1.220000000000000e+02
+3.001880000000000e+02 +3.965000000000000e+02
+4.867590000000000e+02 +2.610000000000000e+02
+5.614310000000000e+02 +3.965000000000000e+02
+6.581070000000000e+02 +1.240000000000000e+02
+7.852700000000000e+02 +1.380000000000000e+02
+9.042150000000000e+02 +1.210000000000000e+02
+8.712650000000000e+02 +3.960000000000000e+02
+9.260570000000000e+02 +1.940000000000000e+02
+5.049020000000000e+00 +3.955000000000000e+02
+8.998400000000000e+02 +8.650000000000000e+01
+2.496320000000000e+02 +3.950000000000000e+02
+7.011510000000002e+02 +1.840000000000000e+02
+1.000640000000000e+03 +1.090000000000000e+02
+7.898260000000000e+02 +1.415000000000000e+02
+9.228280000000000e+02 +3.950000000000000e+02
+6.355430000000000e+02 +9.050000000000000e+01
+2.753880000000000e+00 +3.940000000000000e+02
+3.535860000000000e+02 +3.935000000000000e+02
+9.380150000000000e-01 +3.935000000000000e+02
+1.937250000000000e+02 +3.930000000000000e+02
+6.661750000000000e+02 +1.750000000000000e+02
+1.063280000000000e+03 +2.165000000000000e+02
+6.345309999999999e+02 +8.350000000000000e+01
+7.924450000000001e+02 +1.370000000000000e+02
+1.369320000000000e+03 +3.930000000000000e+02
+1.011270000000000e+03 +2.275000000000000e+02
+1.067540000000000e+03 +2.255000000000000e+02
+4.138400000000000e+02 +3.925000000000000e+02
+3.845800000000000e+02 +3.925000000000000e+02
+1.039800000000000e+03 +1.745000000000000e+02
+1.429310000000000e+03 +1.705000000000000e+02
+1.403950000000000e+03 +3.575000000000000e+02
+4.410880000000000e+01 +3.910000000000000e+02
+1.005690000000000e+03 +3.435000000000000e+02
+6.248070000000000e+02 +1.130000000000000e+02
+1.239560000000000e+03 +3.910000000000000e+02
+1.093630000000000e+03 +3.910000000000000e+02
+1.101090000000000e+02 +3.905000000000000e+02
+5.400530000000000e+02 +3.905000000000000e+02
+1.294780000000000e+03 +1.650000000000000e+02
+1.164340000000000e+03 +3.415000000000000e+02
+8.638020000000000e+02 +2.155000000000000e+02
+3.336980000000001e+02 +3.895000000000000e+02
+1.108270000000000e+03 +1.795000000000000e+02
+6.405160000000000e+02 +8.750000000000000e+01
+1.862580000000000e+03 +3.400000000000000e+02
+7.852569999999999e+02 +1.350000000000000e+02
+1.027950000000000e+03 +3.885000000000000e+02
+1.727630000000000e+03 +3.565000000000000e+02
+9.191260000000000e+02 +3.885000000000000e+02
+5.526530000000000e+02 +3.880000000000000e+02
+5.339360000000000e+02 +3.880000000000000e+02
+7.625740000000000e+02 +1.310000000000000e+02
+9.796520000000000e+02 +1.325000000000000e+02
+2.844360000000000e+01 +3.875000000000000e+02
+1.001110000000000e+03 +2.245000000000000e+02
+3.360290000000000e+02 +2.850000000000000e+01
+2.153660000000000e+01 +3.870000000000000e+02
+1.299840000000000e+03 +2.415000000000000e+02
+5.231440000000000e+02 +3.870000000000000e+02
+6.213590000000000e+02 +1.115000000000000e+02
+1.289620000000000e+03 +1.585000000000000e+02
+1.766750000000000e+03 +3.160000000000000e+02
+8.626120000000000e+02 +1.590000000000000e+02
+5.412120000000000e+02 +3.865000000000000e+02
+7.934580000000002e+02 +1.320000000000000e+02
+1.700420000000000e+03 +3.590000000000000e+02
+1.155850000000000e+03 +1.915000000000000e+02
+1.086710000000000e+03 +2.255000000000000e+02
+1.002850000000000e+03 +3.860000000000000e+02
+6.501700000000000e+02 +7.000000000000000e+01
+7.911319999999999e+02 +1.330000000000000e+02
+2.794470000000000e+02 +9.800000000000000e+01
+6.645710000000000e+02 +7.650000000000000e+01
+1.318250000000000e+03 +3.850000000000000e+02
+5.005700000000000e+01 +3.850000000000000e+02
+3.963490000000000e+01 +3.845000000000000e+02
+4.833970000000000e+02 +2.540000000000000e+02
+6.890960000000000e+02 +1.670000000000000e+02
+2.818490000000000e+02 +9.750000000000000e+01
+5.551890000000000e+02 +3.845000000000000e+02
+3.893450000000000e+02 +3.845000000000000e+02
+3.461070000000000e+02 +3.050000000000000e+01
+9.096319999999999e+02 +3.845000000000000e+02
+8.725630000000000e+02 +9.850000000000000e+01
+2.130320000000000e+01 +3.840000000000000e+02
+1.395060000000000e+03 +2.265000000000000e+02
+6.228990000000000e+02 +7.800000000000000e+01
+5.307290000000000e+02 +3.835000000000000e+02
+1.865250000000000e+03 +3.340000000000000e+02
+6.988520000000000e+02 +2.045000000000000e+02
+3.540060000000000e+02 +3.830000000000000e+02
+7.965470000000000e+02 +1.340000000000000e+02
+8.919510000000000e+02 +1.085000000000000e+02
+6.580080000000000e+02 +6.900000000000000e+01
+2.650280000000000e+02 +3.820000000000000e+02
+8.897850000000000e+02 +1.090000000000000e+02
+4.927900000000000e+02 +2.485000000000000e+02
+7.998020000000000e+02 +1.330000000000000e+02
+1.374610000000000e+03 +3.665000000000000e+02
+1.340370000000000e+03 +3.310000000000000e+02
+1.350740000000000e+01 +3.810000000000000e+02
+7.440720000000000e+02 +1.355000000000000e+02
+1.319370000000000e+03 +2.580000000000000e+02
+1.595470000000000e+01 +3.805000000000000e+02
+9.990180000000000e+02 +3.805000000000000e+02
+9.009800000000000e+02 +1.095000000000000e+02
+3.761790000000000e+02 +3.805000000000000e+02
+7.526890000000000e+02 +1.380000000000000e+02
+7.923000000000000e+02 +1.310000000000000e+02
+1.563600000000000e+01 +3.795000000000000e+02
+1.029560000000000e+03 +2.200000000000000e+02
+1.296780000000000e+03 +1.535000000000000e+02
+8.791000000000000e+02 +3.795000000000000e+02
+4.988460000000000e+02 +3.355000000000000e+02
+4.861030000000000e+02 +3.380000000000000e+02
+5.450200000000000e+02 +3.785000000000000e+02
+1.880120000000000e+03 +3.315000000000000e+02
+1.302930000000000e+03 +1.500000000000000e+02
+8.911600000000000e+02 +1.770000000000000e+02
+7.645510000000000e+02 +1.255000000000000e+02
+8.766950000000001e+02 +3.780000000000000e+02
+8.909390000000000e+02 +3.780000000000000e+02
+4.234310000000000e+01 +3.775000000000000e+02
+1.374570000000000e+03 +2.210000000000000e+02
+3.773160000000000e+02 +3.770000000000000e+02
+1.362230000000000e+03 +3.770000000000000e+02
+8.948150000000001e+02 +9.650000000000000e+01
+8.907780000000000e+02 +1.710000000000000e+02
+5.233330000000002e+02 +3.765000000000000e+02
+3.361230000000001e+02 +3.765000000000000e+02
+7.931720000000000e+02 +1.285000000000000e+02
+3.778450000000000e+02 +3.760000000000000e+02
+1.879880000000000e+02 +3.760000000000000e+02
+1.027740000000000e+03 +3.760000000000000e+02
+4.975350000000000e+02 +3.440000000000000e+02
+7.967680000000000e+02 +1.280000000000000e+02
+4.792600000000000e+02 +3.365000000000000e+02
+1.808570000000000e+02 +3.750000000000000e+02
+1.321850000000000e+03 +3.750000000000000e+02
+6.351050000000000e+02 +6.750000000000000e+01
+9.756900000000001e+02 +1.165000000000000e+02
+1.438430000000000e+03 +3.740000000000000e+02
+2.774880000000000e+02 +9.650000000000000e+01
+3.315250000000000e+02 +3.735000000000000e+02
+8.511050000000000e+02 +3.735000000000000e+02
+2.939050000000000e+02 +8.750000000000000e+01
+1.184230000000000e+03 +3.730000000000000e+02
+3.907430000000001e+02 +3.725000000000000e+02
+3.984790000000000e+02 +3.725000000000000e+02
+7.882919999999998e+02 +1.265000000000000e+02
+1.888530000000000e+01 +3.720000000000000e+02
+4.892000000000000e+02 +3.720000000000000e+02
+6.619220000000000e+02 +1.030000000000000e+02
+1.664540000000000e+03 +2.350000000000000e+02
+3.167150000000000e+02 +2.250000000000000e+01
+5.387630000000000e+02 +3.715000000000000e+02
+1.032060000000000e+03 +1.605000000000000e+02
+7.057650000000000e+02 +2.735000000000000e+02
+9.783380000000000e+02 +1.180000000000000e+02
+1.555800000000000e+03 +2.165000000000000e+02
+8.429160000000001e+02 +1.010000000000000e+02
+1.063110000000000e+03 +1.915000000000000e+02
+7.925520000000000e+02 +1.275000000000000e+02
+9.940010000000000e+02 +1.185000000000000e+02
+8.992460000000002e+02 +1.040000000000000e+02
+8.624530000000000e+02 +3.700000000000000e+02
+5.410660000000000e+02 +3.695000000000000e+02
+7.907550000000000e+02 +1.230000000000000e+02
+5.297360000000000e+02 +3.690000000000000e+02
+7.666419999999998e+02 +1.210000000000000e+02
+8.603570000000000e+02 +1.000000000000000e+02
+6.400260000000000e+02 +1.595000000000000e+02
+9.844730000000000e+02 +9.600000000000000e+01
+9.077800000000000e+02 +9.600000000000000e+01
+1.029350000000000e+03 +2.730000000000000e+02
+4.458070000000000e+02 +3.680000000000000e+02
+7.898380000000002e+02 +1.265000000000000e+02
+1.320360000000000e+02 +3.680000000000000e+02
+1.000930000000000e+03 +2.030000000000000e+02
+1.072040000000000e+03 +1.925000000000000e+02
+3.107680000000000e+02 +3.675000000000000e+02
+1.148780000000000e+03 +2.605000000000000e+02
+3.702080000000000e+02 +3.665000000000000e+02
+7.903660000000001e+02 +1.710000000000000e+02
+6.490060000000000e+02 +1.550000000000000e+02
+3.562150000000000e+02 +3.665000000000000e+02
+9.080820000000000e+02 +3.665000000000000e+02
+9.897300000000000e+02 +7.350000000000000e+01
+1.016490000000000e+03 +1.990000000000000e+02
+2.779060000000000e+02 +8.200000000000000e+01
+7.110730000000000e+02 +2.570000000000000e+02
+2.946970000000000e+02 +3.655000000000000e+02
+7.900100000000000e+02 +1.205000000000000e+02
+4.790970000000000e+02 +3.000000000000000e+01
+2.790430000000000e+02 +9.000000000000000e+01
+7.925889999999998e+02 +1.265000000000000e+02
+1.322990000000000e+03 +1.375000000000000e+02
+1.084370000000000e+03 +1.515000000000000e+02
+8.154989999999998e+02 +1.710000000000000e+02
+2.989290000000001e+02 +3.645000000000000e+02
+2.261130000000000e+02 +3.645000000000000e+02
+1.015250000000000e+03 +2.010000000000000e+02
+1.318050000000000e+03 +3.520000000000000e+02
+7.901550000000000e+02 +1.265000000000000e+02
+8.171089999999998e+02 +1.835000000000000e+02
+3.031890000000000e+02 +3.635000000000000e+02
+7.914660000000000e+02 +1.200000000000000e+02
+5.820359999999999e+02 +1.615000000000000e+02
+6.155810000000000e+02 +1.520000000000000e+02
+4.847740000000000e+02 +3.265000000000000e+02
+6.357390000000000e+02 +1.550000000000000e+02
+3.910100000000000e+02 +3.625000000000000e+02
+6.602719999999998e+02 +8.450000000000000e+01
+7.468960000000002e+02 +1.185000000000000e+02
+1.908530000000000e+02 +3.620000000000000e+02
+1.118040000000000e+03 +1.615000000000000e+02
+6.999000000000000e+02 +3.620000000000000e+02
+2.779710000000000e+01 +3.620000000000000e+02
+1.000130000000000e+03 +2.715000000000000e+02
+8.984510000000000e+02 +3.340000000000000e+02
+1.217620000000000e+03 +1.985000000000000e+02
+6.593000000000000e+02 +1.555000000000000e+02
+7.882860000000002e+02 +1.200000000000000e+02
+1.324910000000000e+03 +3.600000000000000e+02
+1.273110000000000e+03 +3.600000000000000e+02
+0.000000000000000e+00 +3.600000000000000e+02
+1.454640000000000e+03 +3.600000000000000e+02
+3.603480000000000e+02 +3.600000000000000e+02
+1.321090000000000e+03 +3.130000000000000e+02
+1.039700000000000e+03 +1.840000000000000e+02
+1.907970000000000e+03 +3.600000000000000e+02
+3.526770000000000e+02 +3.600000000000000e+02
+1.347210000000000e+03 +3.600000000000000e+02
+2.720850000000000e+02 +3.600000000000000e+02
+6.526250000000000e+02 +1.525000000000000e+02
+1.845420000000000e+02 +3.595000000000000e+02
+7.110260000000002e+02 +2.820000000000000e+02
+7.804620000000000e+02 +1.170000000000000e+02
+1.171210000000000e+03 +3.590000000000000e+02
+1.069920000000000e+03 +1.825000000000000e+02
+7.941310000000002e+02 +1.195000000000000e+02
+7.127970000000000e+02 +7.600000000000000e+01
+6.780230000000000e+02 +1.220000000000000e+02
+7.790450000000000e+02 +1.615000000000000e+02
+5.221550000000000e+00 +3.580000000000000e+02
+1.147920000000000e+03 +3.580000000000000e+02
+1.126650000000000e+03 +3.115000000000000e+02
+1.549590000000000e+03 +2.010000000000000e+02
+9.029990000000000e+02 +3.575000000000000e+02
+1.608710000000000e+03 +2.150000000000000e+02
+1.005070000000000e+03 +1.950000000000000e+02
+6.603090000000000e+02 +1.540000000000000e+02
+3.101310000000000e+02 +3.570000000000000e+02
+3.519610000000000e+02 +1.000000000000000e+02
+7.947830000000000e+02 +1.220000000000000e+02
+1.135260000000000e+03 +3.565000000000000e+02
+4.836450000000000e+02 +5.600000000000000e+01
+3.311000000000000e+02 +3.560000000000000e+02
+7.728290000000000e+02 +1.200000000000000e+02
+9.652700000000000e+02 +1.175000000000000e+02
+1.008620000000000e+03 +1.530000000000000e+02
+3.019430000000000e+02 +3.555000000000000e+02
+1.589230000000000e+03 +3.340000000000000e+02
+4.887780000000000e+02 +3.550000000000000e+02
+3.726030000000000e+02 +3.550000000000000e+02
+9.742020000000000e+02 +1.820000000000000e+02
+1.164630000000000e+03 +3.550000000000000e+02
+4.881780000000001e+02 +3.210000000000000e+02
+6.365900000000000e+02 +1.530000000000000e+02
+4.957170000000000e+02 +3.545000000000000e+02
+7.954600000000000e+02 +1.215000000000000e+02
+5.272740000000000e+02 +3.540000000000000e+02
+3.142620000000000e+02 +3.540000000000000e+02
+8.586230000000000e+02 +3.540000000000000e+02
+6.989310000000000e+02 +3.535000000000000e+02
+1.144220000000000e+03 +3.535000000000000e+02
+7.911569999999998e+02 +1.180000000000000e+02
+1.319050000000000e+03 +2.260000000000000e+02
+1.412410000000000e+03 +1.320000000000000e+02
+9.759950000000000e+02 +1.845000000000000e+02
+9.126210000000000e+02 +9.350000000000000e+01
+8.021799999999999e+02 +1.400000000000000e+02
+7.125670000000000e+02 +2.415000000000000e+02
+7.482030000000000e+02 +1.120000000000000e+02
+2.622250000000000e+02 +3.520000000000000e+02
+4.944570000000000e+02 +1.205000000000000e+02
+1.653800000000000e+03 +2.790000000000000e+02
+7.923739999999998e+02 +1.180000000000000e+02
+9.928860000000000e+02 +1.840000000000000e+02
+1.123130000000000e+03 +3.050000000000000e+02
+6.596280000000000e+02 +7.950000000000000e+01
+1.288680000000000e+03 +3.170000000000000e+02
+6.337569999999999e+02 +1.465000000000000e+02
+1.638110000000000e+02 +3.505000000000000e+02
+2.996100000000000e+02 +3.505000000000000e+02
+9.916960000000000e+02 +1.860000000000000e+02
+1.026900000000000e+03 +1.385000000000000e+02
+8.890820000000000e+02 +3.505000000000000e+02
+1.219470000000000e+03 +1.865000000000000e+02
+6.608339999999999e+02 +1.420000000000000e+02
+7.869830000000002e+02 +1.180000000000000e+02
+1.289520000000000e+03 +2.160000000000000e+02
+8.601400000000000e+02 +8.750000000000000e+01
+3.718320000000000e+02 +3.495000000000000e+02
+9.634860000000000e+02 +1.790000000000000e+02
+6.311380000000000e+02 +6.300000000000000e+01
+2.413060000000000e+02 +3.490000000000000e+02
+3.104190000000001e+02 +3.490000000000000e+02
+6.637310000000001e+02 +7.400000000000000e+01
+1.132320000000000e+03 +3.490000000000000e+02
+6.839850000000000e+02 +2.445000000000000e+02
+6.447500000000000e+02 +1.385000000000000e+02
+7.914310000000000e+02 +1.190000000000000e+02
+9.940599999999999e+02 +1.860000000000000e+02
+1.296440000000000e+03 +3.485000000000000e+02
+1.279810000000000e+03 +3.165000000000000e+02
+5.290860000000000e+02 +3.480000000000000e+02
+3.403190000000000e+02 +3.480000000000000e+02
+8.402700000000000e+02 +8.400000000000000e+01
+4.145830000000000e+02 +3.475000000000000e+02
+1.319950000000000e+02 +3.475000000000000e+02
+9.940250000000000e+02 +1.905000000000000e+02
+1.733230000000000e+03 +2.940000000000000e+02
+2.733630000000000e+02 +3.470000000000000e+02
+1.066350000000000e+03 +1.850000000000000e+02
+6.100050000000000e+02 +1.445000000000000e+02
+3.618370000000000e+02 +3.465000000000000e+02
+3.357890000000000e+02 +3.460000000000000e+02
+7.050820000000000e+02 +2.345000000000000e+02
+1.066970000000000e+03 +1.705000000000000e+02
+7.701270000000000e+02 +1.560000000000000e+02
+2.793050000000000e+02 +3.455000000000000e+02
+3.454250000000000e+00 +3.455000000000000e+02
+1.309020000000000e+03 +2.205000000000000e+02
+1.601410000000000e+03 +2.670000000000000e+02
+4.002280000000000e+02 +3.450000000000000e+02
+7.745340000000000e+02 +1.130000000000000e+02
+1.074680000000000e+03 +3.450000000000000e+02
+1.177710000000000e+03 +3.150000000000000e+02
+7.808670000000000e+02 +1.160000000000000e+02
+9.470630000000000e+01 +3.440000000000000e+02
+1.078980000000000e+03 +1.470000000000000e+02
+1.713330000000000e+03 +3.280000000000000e+02
+6.191820000000000e+02 +7.050000000000000e+01
+9.011440000000000e+02 +1.260000000000000e+02
+6.390620000000000e+02 +6.800000000000000e+01
+7.865740000000000e+02 +1.125000000000000e+02
+1.256910000000000e+03 +1.745000000000000e+02
+2.038860000000000e+02 +3.425000000000000e+02
+3.219900000000000e+02 +3.425000000000000e+02
+3.398300000000000e+02 +8.900000000000000e+01
+1.435060000000000e+03 +3.425000000000000e+02
+7.845419999999998e+02 +1.150000000000000e+02
+9.184610000000000e+02 +8.500000000000000e+01
+2.504420000000000e+02 +3.415000000000000e+02
+2.105390000000000e+02 +3.415000000000000e+02
+8.970910000000000e+02 +3.415000000000000e+02
+2.695940000000000e+02 +3.410000000000000e+02
+2.917650000000000e+02 +3.410000000000000e+02
+7.884550000000000e+02 +1.115000000000000e+02
+1.100920000000000e+03 +3.410000000000000e+02
+1.671820000000000e+01 +3.410000000000000e+02
+6.227710000000000e+02 +1.340000000000000e+02
+4.974550000000000e+02 +3.105000000000000e+02
+7.841790000000000e+02 +1.155000000000000e+02
+1.329560000000000e+01 +3.400000000000000e+02
+1.656170000000000e+02 +3.395000000000000e+02
+2.855940000000000e+02 +3.395000000000000e+02
+7.005950000000000e+02 +2.635000000000000e+02
+6.222540000000000e+02 +1.390000000000000e+02
+3.072270000000001e+02 +3.390000000000000e+02
+6.478950000000000e+02 +1.390000000000000e+02
+7.869390000000000e+02 +1.165000000000000e+02
+1.816110000000000e+03 +2.605000000000000e+02
+2.455990000000000e+01 +3.385000000000000e+02
+1.212890000000000e+03 +2.755000000000000e+02
+7.668580000000002e+02 +1.120000000000000e+02
+1.294770000000000e+03 +2.090000000000000e+02
+8.997030000000000e+02 +1.260000000000000e+02
+2.240680000000000e+00 +3.375000000000000e+02
+7.882270000000000e+02 +1.095000000000000e+02
+1.071810000000000e+03 +1.795000000000000e+02
+2.851690000000001e+02 +3.360000000000000e+02
+3.035760000000000e+02 +3.360000000000000e+02
+9.815720000000000e+02 +1.790000000000000e+02
+1.014460000000000e+03 +2.855000000000000e+02
+7.892170000000000e+02 +1.095000000000000e+02
+2.747590000000000e+02 +3.350000000000000e+02
+1.021770000000000e+03 +2.840000000000000e+02
+6.457240000000000e+02 +1.385000000000000e+02
+1.737380000000000e+02 +3.345000000000000e+02
+6.807139999999998e+02 +3.345000000000000e+02
+1.594020000000000e+03 +2.575000000000000e+02
+1.272680000000000e+03 +2.920000000000000e+02
+4.877120000000000e+02 +3.090000000000000e+02
+2.448330000000000e+02 +3.340000000000000e+02
+7.888370000000000e+02 +1.150000000000000e+02
+9.919730000000000e+02 +1.750000000000000e+02
+7.892510000000002e+02 +1.085000000000000e+02
+1.380430000000000e+03 +2.105000000000000e+02
+9.182960000000000e+02 +3.335000000000000e+02
+8.072619999999999e+02 +1.605000000000000e+02
+6.581289999999998e+02 +6.550000000000000e+01
+1.480550000000000e+03 +3.130000000000000e+02
+2.375030000000000e+02 +3.325000000000000e+02
+3.007620000000000e+02 +3.325000000000000e+02
+7.906020000000000e+02 +1.855000000000000e+02
+1.880420000000000e+01 +3.325000000000000e+02
+6.641110000000001e+02 +1.330000000000000e+02
+1.295360000000000e+03 +2.040000000000000e+02
+2.800930000000000e+02 +3.315000000000000e+02
+6.192619999999999e+02 +5.750000000000000e+01
+1.064350000000000e+03 +1.575000000000000e+02
+6.455309999999999e+02 +1.260000000000000e+02
+3.191470000000000e+02 +3.310000000000000e+02
+1.069070000000000e+03 +1.685000000000000e+02
+6.568530000000002e+02 +1.325000000000000e+02
+7.928099999999999e+02 +1.870000000000000e+02
+6.832020000000000e+02 +3.305000000000000e+02
+1.950290000000000e+03 +2.725000000000000e+02
+6.720010000000002e+02 +1.270000000000000e+02
+7.808020000000000e+02 +1.055000000000000e+02
+2.667780000000000e+02 +3.295000000000000e+02
+7.945790000000000e+02 +1.065000000000000e+02
+1.311090000000000e+03 +2.025000000000000e+02
+1.635180000000000e+03 +3.030000000000000e+02
+2.152440000000000e+02 +3.285000000000000e+02
+9.733690000000000e+02 +1.740000000000000e+02
+6.608900000000000e+02 +1.355000000000000e+02
+9.005280000000000e+02 +1.125000000000000e+02
+2.347740000000000e+02 +3.280000000000000e+02
+2.718720000000000e+02 +3.275000000000000e+02
+4.900850000000000e+02 +3.035000000000000e+02
+2.781070000000000e+02 +5.000000000000000e+01
+3.036030000000000e+02 +3.270000000000000e+02
+1.295090000000000e+03 +2.030000000000000e+02
+1.207420000000000e+03 +2.620000000000000e+02
+7.951260000000002e+02 +1.850000000000000e+02
+1.069920000000000e+03 +1.315000000000000e+02
+6.629280000000000e+02 +1.390000000000000e+02
+8.486460000000002e+02 +1.120000000000000e+02
+2.668420000000000e+02 +3.255000000000000e+02
+1.069370000000000e+03 +1.170000000000000e+02
+1.166600000000000e+01 +3.255000000000000e+02
+6.356770000000000e+02 +1.310000000000000e+02
+1.785950000000000e+02 +3.250000000000000e+02
+6.594720000000000e+02 +5.950000000000000e+01
+2.533330000000000e+02 +3.245000000000000e+02
+7.886980000000000e+02 +1.030000000000000e+02
+1.084730000000000e+03 +1.175000000000000e+02
+3.231950000000000e+02 +3.245000000000000e+02
+2.250950000000000e+02 +3.240000000000000e+02
+2.890830000000000e+02 +3.240000000000000e+02
+2.631420000000000e+02 +3.240000000000000e+02
+6.758960000000002e+02 +3.240000000000000e+02
+1.018660000000000e+03 +3.240000000000000e+02
+3.111660000000000e+02 +3.235000000000000e+02
+5.006770000000000e+02 +3.030000000000000e+02
+9.772260000000000e+02 +1.605000000000000e+02
+2.524400000000000e+02 +3.235000000000000e+02
+1.673390000000000e+03 +3.235000000000000e+02
+8.780590000000000e+02 +3.230000000000000e+02
+1.313660000000000e+03 +2.880000000000000e+02
+3.252370000000000e+02 +7.550000000000000e+01
+1.079120000000000e+03 +1.335000000000000e+02
+9.164630000000000e+02 +3.225000000000000e+02
+3.198080000000000e+02 +3.220000000000000e+02
+1.473410000000000e+03 +3.005000000000000e+02
+2.747980000000000e+02 +3.215000000000000e+02
+1.176320000000000e+03 +3.215000000000000e+02
+9.310250000000000e+02 +2.880000000000000e+02
+9.851650000000000e+02 +2.300000000000000e+02
+2.046500000000000e+02 +3.205000000000000e+02
+6.312300000000001e+01 +3.205000000000000e+02
+1.309730000000000e+03 +1.955000000000000e+02
+1.057060000000000e+03 +1.925000000000000e+02
+1.672320000000000e+03 +2.780000000000000e+02
+8.624570000000000e+02 +1.110000000000000e+02
+1.077640000000000e+03 +1.570000000000000e+02
+3.392060000000000e+01 +3.200000000000000e+02
+6.229910000000000e+02 +5.000000000000000e+01
+1.009230000000000e+03 +1.080000000000000e+02
+7.840139999999999e+02 +1.720000000000000e+02
+1.124840000000000e+03 +1.150000000000000e+02
+1.758370000000000e+03 +2.610000000000000e+02
+1.288810000000000e+03 +2.235000000000000e+02
+1.081810000000000e+02 +3.190000000000000e+02
+2.535050000000000e+02 +3.190000000000000e+02
+9.771270000000000e+02 +1.695000000000000e+02
+6.212660000000000e+02 +4.400000000000000e+01
+2.771110000000000e+02 +3.185000000000000e+02
+7.802760000000002e+02 +1.780000000000000e+02
+8.561280000000000e+02 +3.185000000000000e+02
+6.532980000000000e+02 +1.290000000000000e+02
+2.646640000000000e+02 +3.180000000000000e+02
+6.187970000000000e+02 +4.700000000000000e+01
+7.886830000000000e+02 +1.775000000000000e+02
+3.258130000000000e+02 +6.650000000000000e+01
+1.461640000000000e+03 +3.180000000000000e+02
+1.063990000000000e+03 +1.455000000000000e+02
+2.689130000000000e+01 +3.175000000000000e+02
+8.885470000000000e+02 +2.905000000000000e+02
+1.079990000000000e+03 +1.445000000000000e+02
+6.356519999999998e+02 +1.260000000000000e+02
+3.064530000000000e+02 +3.165000000000000e+02
+4.834480000000000e+02 +2.940000000000000e+02
+1.063130000000000e+03 +1.425000000000000e+02
+6.186360000000000e+02 +1.215000000000000e+02
+5.707730000000000e+01 +3.165000000000000e+02
+2.307310000000000e+02 +3.160000000000000e+02
+9.784890000000000e+02 +2.230000000000000e+02
+2.768550000000000e+02 +3.155000000000000e+02
+6.254870000000000e+02 +4.650000000000000e+01
+1.387380000000000e+03 +2.485000000000000e+02
+2.425850000000000e+02 +3.150000000000000e+02
+7.806900000000001e+02 +1.715000000000000e+02
+3.254290000000001e+02 +7.000000000000000e+01
+8.928000000000000e+02 +1.055000000000000e+02
+6.140500000000000e+02 +4.300000000000000e+01
+1.619900000000000e+03 +2.995000000000000e+02
+5.007760000000000e+02 +7.800000000000000e+01
+1.093310000000000e+03 +1.890000000000000e+02
+2.757980000000000e+02 +3.500000000000000e+01
+9.595820000000000e+02 +1.645000000000000e+02
+1.682550000000000e+03 +3.140000000000000e+02
+2.133230000000000e+02 +3.135000000000000e+02
+2.438670000000000e+02 +3.135000000000000e+02
+2.447210000000000e+02 +3.135000000000000e+02
+7.866510000000002e+02 +1.715000000000000e+02
+9.250810000000000e+02 +3.135000000000000e+02
+1.273830000000000e+03 +3.130000000000000e+02
+9.986240000000000e+02 +2.675000000000000e+02
+2.966200000000000e+02 +3.120000000000000e+02
+3.148140000000000e+02 +3.120000000000000e+02
+2.513500000000000e+02 +3.120000000000000e+02
+3.815930000000000e+02 +3.120000000000000e+02
+1.715650000000000e+02 +3.120000000000000e+02
+2.494350000000000e+02 +3.115000000000000e+02
+1.722690000000000e+03 +2.555000000000000e+02
+2.767510000000000e+02 +3.850000000000000e+01
+1.839520000000000e+01 +3.110000000000000e+02
+5.066700000000000e+02 +2.910000000000000e+02
+1.903870000000000e+02 +3.105000000000000e+02
+2.423520000000000e+02 +3.105000000000000e+02
+3.369090000000000e+02 +6.900000000000000e+01
+4.264680000000000e+02 +3.100000000000000e+02
+2.006910000000000e+02 +3.100000000000000e+02
+2.533020000000000e+02 +3.100000000000000e+02
+7.899299999999999e+02 +1.795000000000000e+02
+6.770450000000000e+02 +3.100000000000000e+02
+1.058340000000000e+03 +3.100000000000000e+02
+6.183470000000000e+02 +1.470000000000000e+02
+6.348170000000000e+02 +3.085000000000000e+02
+1.848720000000000e+02 +3.085000000000000e+02
+2.506600000000000e+02 +3.085000000000000e+02
+9.276480000000000e+02 +1.270000000000000e+02
+7.156780000000000e+02 +3.085000000000000e+02
+9.411990000000000e+02 +1.160000000000000e+02
+1.207200000000000e+03 +2.475000000000000e+02
+1.065640000000000e+03 +2.435000000000000e+02
+2.212900000000000e+02 +3.075000000000000e+02
+9.798970000000000e+02 +1.630000000000000e+02
+2.414380000000000e+02 +3.070000000000000e+02
+2.285220000000000e+02 +3.070000000000000e+02
+7.869900000000000e+02 +1.655000000000000e+02
+9.951980000000000e+02 +1.645000000000000e+02
+4.143570000000000e+01 +3.065000000000000e+02
+6.653460000000000e+02 +1.470000000000000e+02
+1.023520000000000e+03 +1.040000000000000e+02
+4.838170000000000e+02 +3.060000000000000e+02
+6.588750000000000e+02 +1.920000000000000e+02
+1.080260000000000e+03 +1.445000000000000e+02
+2.210210000000000e+02 +3.055000000000000e+02
+9.940900000000000e+02 +1.720000000000000e+02
+8.755690000000000e+02 +1.210000000000000e+02
+1.660730000000000e+03 +2.310000000000000e+02
+6.356780000000000e+02 +1.180000000000000e+02
+7.953090000000000e+02 +1.640000000000000e+02
+1.000830000000000e+03 +1.615000000000000e+02
+7.014160000000001e+02 +9.400000000000000e+01
+7.878550000000000e+02 +1.815000000000000e+02
+1.743430000000000e+02 +3.040000000000000e+02
+1.272890000000000e+02 +3.040000000000000e+02
+7.971230000000000e+02 +1.880000000000000e+02
+2.768370000000000e+02 +2.800000000000000e+01
+6.457030000000000e+02 +1.100000000000000e+02
+1.291250000000000e+02 +3.035000000000000e+02
+1.300150000000000e+03 +1.710000000000000e+02
+7.996130000000001e+02 +2.355000000000000e+02
+1.473150000000000e+02 +3.030000000000000e+02
+1.955780000000000e+02 +3.030000000000000e+02
+9.949770000000000e+02 +1.515000000000000e+02
+8.148230000000000e+02 +2.370000000000000e+02
+1.084610000000000e+03 +1.320000000000000e+02
+1.341430000000000e+01 +3.025000000000000e+02
+1.000630000000000e+03 +1.690000000000000e+02
+1.723440000000001e+02 +3.020000000000000e+02
+2.097180000000000e+02 +3.020000000000000e+02
+8.002550000000000e+02 +1.875000000000000e+02
+1.123640000000000e+03 +3.020000000000000e+02
+3.173810000000000e+02 +3.015000000000000e+02
+9.193810000000000e+02 +1.080000000000000e+02
+1.486170000000000e+02 +3.010000000000000e+02
+9.689070000000000e+02 +1.600000000000000e+02
+7.876650000000000e+02 +1.835000000000000e+02
+6.486270000000000e+02 +9.900000000000000e+01
+6.249370000000000e+02 +1.395000000000000e+02
+8.002460000000002e+02 +1.715000000000000e+02
+9.027700000000000e+02 +1.095000000000000e+02
+1.238700000000000e+03 +3.000000000000000e+02
+3.186260000000000e+02 +3.000000000000000e+02
+1.162720000000000e+03 +3.000000000000000e+02
+6.791300000000000e+02 +1.030000000000000e+02
+7.020169999999998e+02 +9.500000000000000e+01
+1.400650000000000e+03 +3.000000000000000e+02
+3.161850000000000e+02 +3.000000000000000e+02
+1.160790000000000e+03 +2.065000000000000e+02
+2.522670000000000e+02 +2.995000000000000e+02
+6.619060000000002e+02 +1.145000000000000e+02
+1.124640000000000e+03 +2.995000000000000e+02
+2.264690000000000e+02 +2.990000000000000e+02
+9.182809999999999e+02 +1.730000000000000e+02
+2.949970000000000e+02 +2.985000000000000e+02
+7.052810000000002e+02 +2.330000000000000e+02
+2.408200000000000e+02 +2.980000000000000e+02
+2.852690000000000e+02 +1.430000000000000e+02
+6.522460000000000e+02 +1.120000000000000e+02
+1.572820000000000e+03 +2.295000000000000e+02
+1.625850000000000e+02 +2.980000000000000e+02
+1.081860000000000e+03 +2.350000000000000e+02
+6.595820000000000e+02 +1.015000000000000e+02
+1.981380000000000e+02 +2.970000000000000e+02
+9.618390000000001e+02 +1.515000000000000e+02
+8.913639999999998e+02 +1.080000000000000e+02
+1.101230000000000e+03 +2.970000000000000e+02
+9.097970000000000e+00 +2.965000000000000e+02
+1.013380000000000e+03 +2.480000000000000e+02
+8.961700000000000e+02 +1.085000000000000e+02
+2.483690000000000e+02 +2.955000000000000e+02
+6.661849999999999e+02 +1.120000000000000e+02
+2.251850000000000e+02 +2.955000000000000e+02
+2.171610000000000e+02 +2.955000000000000e+02
+1.305260000000000e+03 +1.740000000000000e+02
+7.449260000000000e+02 +2.950000000000000e+02
+1.417220000000000e+02 +2.950000000000000e+02
+1.966590000000000e+02 +2.945000000000000e+02
+1.566830000000000e+03 +2.250000000000000e+02
+9.100520000000000e+02 +9.700000000000000e+01
+1.602590000000000e+03 +2.675000000000000e+02
+3.647230000000000e+02 +2.940000000000000e+02
+1.516170000000000e+02 +2.940000000000000e+02
+6.711860000000000e+02 +2.940000000000000e+02
+1.247590000000000e+03 +2.940000000000000e+02
+1.000670000000000e+03 +2.940000000000000e+02
+2.528300000000000e+02 +2.935000000000000e+02
+1.670600000000000e+03 +2.220000000000000e+02
+9.805750000000000e+02 +1.650000000000000e+02
+1.045150000000000e+03 +1.745000000000000e+02
+8.986260000000002e+02 +9.600000000000000e+01
+1.005220000000000e+03 +2.425000000000000e+02
+7.960610000000000e+02 +1.800000000000000e+02
+3.502300000000000e+02 +2.930000000000000e+02
+8.071650000000000e+02 +1.175000000000000e+02
+1.023990000000000e+02 +2.925000000000000e+02
+5.055100000000000e+02 +2.920000000000000e+02
+2.177910000000000e+02 +2.920000000000000e+02
+1.084480000000000e+03 +1.705000000000000e+02
+9.069410000000000e+02 +1.065000000000000e+02
+1.583300000000000e+00 +2.920000000000000e+02
+1.038530000000000e+01 +2.915000000000000e+02
+1.394910000000000e+03 +1.390000000000000e+02
+2.361460000000000e+02 +2.915000000000000e+02
+1.535580000000000e+03 +2.915000000000000e+02
+2.386020000000000e+02 +2.910000000000000e+02
+1.081810000000000e+03 +2.275000000000000e+02
+2.788420000000000e+02 +2.150000000000000e+01
+6.444460000000000e+02 +1.080000000000000e+02
+6.631450000000000e+02 +2.895000000000000e+02
+5.639119999999998e+02 +2.895000000000000e+02
+6.179640000000001e+02 +9.600000000000000e+01
+6.228620000000000e+02 +1.300000000000000e+02
+7.902220000000000e+02 +1.600000000000000e+02
+9.764630000000000e+02 +2.890000000000000e+02
+1.370500000000000e+03 +2.515000000000000e+02
+6.390010000000000e+02 +9.700000000000000e+01
+2.332070000000000e+02 +2.885000000000000e+02
+1.299970000000000e+03 +1.665000000000000e+02
+4.698890000000000e+02 +2.885000000000000e+02
+2.059900000000000e+02 +2.880000000000000e+02
+7.803880000000000e+02 +1.570000000000000e+02
+2.980660000000000e+02 +2.880000000000000e+02
+6.545150000000000e+02 +1.175000000000000e+02
+6.316669999999998e+02 +1.170000000000000e+02
+1.480850000000000e+03 +2.755000000000000e+02
+1.058740000000000e+03 +2.270000000000000e+02
+9.825950000000000e+02 +1.470000000000000e+02
+1.147650000000000e+03 +2.865000000000000e+02
+8.996010000000001e+02 +1.025000000000000e+02
+8.983520000000000e+02 +1.940000000000000e+02
+2.815560000000000e+02 +2.050000000000000e+01
+6.524019999999998e+02 +1.065000000000000e+02
+1.979230000000000e+02 +2.860000000000000e+02
+3.443830000000001e+02 +5.100000000000000e+01
+4.762000000000000e+02 +2.855000000000000e+02
+4.932410000000000e+02 +4.250000000000000e+01
+6.362809999999999e+02 +9.600000000000000e+01
+1.190330000000000e+02 +2.855000000000000e+02
+7.473310000000000e+02 +1.150000000000000e+02
+1.997060000000000e+02 +2.850000000000000e+02
+1.618810000000000e+02 +2.850000000000000e+02
+6.580580000000000e+02 +9.350000000000000e+01
+9.945720000000000e+02 +1.595000000000000e+02
+8.944420000000000e+02 +1.945000000000000e+02
+1.760390000000000e+01 +2.845000000000000e+02
+7.653589999999998e+02 +2.845000000000000e+02
+8.744280000000000e+02 +1.555000000000000e+02
+6.685839999999999e+02 +9.700000000000000e+01
+8.060150000000000e+01 +2.840000000000000e+02
+7.897970000000000e+02 +1.560000000000000e+02
+9.102320000000000e+02 +1.025000000000000e+02
+1.208600000000000e+02 +2.835000000000000e+02
+1.083640000000000e+03 +1.600000000000000e+02
+3.994490000000000e+02 +2.830000000000000e+02
+1.199770000000000e+03 +2.620000000000000e+02
+9.376960000000000e+02 +1.605000000000000e+02
+2.224680000000000e+02 +2.825000000000000e+02
+7.964830000000002e+02 +1.565000000000000e+02
+8.507480000000000e+02 +1.025000000000000e+02
+1.886900000000000e-01 +2.820000000000000e+02
+1.073910000000000e+03 +2.245000000000000e+02
+6.562089999999999e+02 +1.040000000000000e+02
+5.948010000000000e+01 +2.815000000000000e+02
+7.639220000000000e+02 +1.455000000000000e+02
+9.894400000000001e+02 +1.515000000000000e+02
+8.872930000000000e+02 +1.030000000000000e+02
+6.101020000000000e+02 +1.005000000000000e+02
+1.608430000000000e+02 +2.810000000000000e+02
+7.522750000000000e+02 +1.360000000000000e+02
+1.301850000000000e+03 +2.810000000000000e+02
+2.072070000000000e+02 +2.805000000000000e+02
+1.582190000000000e+02 +2.800000000000000e+02
+9.996910000000000e+02 +2.325000000000000e+02
+1.256940000000000e+02 +2.795000000000000e+02
+4.230780000000000e+02 +2.790000000000000e+02
+1.075900000000000e+03 +2.235000000000000e+02
+7.692400000000000e+02 +1.445000000000000e+02
+1.182340000000000e+03 +2.790000000000000e+02
+1.642090000000000e+03 +2.395000000000000e+02
+1.055700000000000e+03 +2.145000000000000e+02
+1.071790000000000e+03 +1.640000000000000e+02
+8.894680000000002e+02 +8.300000000000000e+01
+8.732370000000000e+02 +2.785000000000000e+02
+2.343650000000000e+02 +2.780000000000000e+02
+6.553180000000000e+02 +9.550000000000000e+01
+1.239090000000000e+02 +2.780000000000000e+02
+9.917270000000000e+02 +1.455000000000000e+02
+7.588310000000000e+02 +1.425000000000000e+02
+1.077030000000000e+03 +2.365000000000000e+02
+6.567760000000002e+02 +8.800000000000000e+01
+1.923770000000000e+02 +2.775000000000000e+02
+2.079550000000000e+02 +2.770000000000000e+02
+1.145540000000000e+03 +2.770000000000000e+02
+1.511430000000000e+02 +2.765000000000000e+02
+2.867510000000000e+02 +2.765000000000000e+02
+1.768110000000000e+02 +2.760000000000000e+02
+1.125520000000000e+03 +1.580000000000000e+02
+2.355230000000000e+02 +2.755000000000000e+02
+5.155409999999998e+02 +2.755000000000000e+02
+1.318000000000000e+03 +1.565000000000000e+02
+8.917439999999998e+02 +9.050000000000000e+01
+3.910080000000000e+02 +2.750000000000000e+02
+8.368360000000000e+02 +2.750000000000000e+02
+8.181569999999998e+02 +2.750000000000000e+02
+9.056430000000000e+02 +1.855000000000000e+02
+1.385760000000000e+03 +2.340000000000000e+02
+1.311490000000000e+03 +1.575000000000000e+02
+9.005160000000002e+02 +8.600000000000000e+01
+5.474850000000000e+02 +2.740000000000000e+02
+1.066730000000000e+03 +2.175000000000000e+02
+2.000990000000000e+02 +2.740000000000000e+02
+8.357489999999998e+02 +2.740000000000000e+02
+1.605260000000000e+02 +2.735000000000000e+02
+1.154680000000000e+03 +2.730000000000000e+02
+9.432410000000000e+02 +2.725000000000000e+02
+1.972060000000000e+02 +2.720000000000000e+02
+2.852910000000000e+02 +2.715000000000000e+02
+1.452430000000000e+03 +2.715000000000000e+02
+1.752010000000000e+02 +2.715000000000000e+02
+1.577370000000000e+03 +2.030000000000000e+02
+1.019640000000000e+03 +1.510000000000000e+02
+2.261470000000000e+02 +2.710000000000000e+02
+9.959160000000001e+02 +2.710000000000000e+02
+2.581460000000000e+02 +2.705000000000000e+02
+9.825830000000000e+02 +2.700000000000000e+02
+5.005780000000000e+02 +2.695000000000000e+02
+7.430839999999999e+02 +1.495000000000000e+02
+6.400670000000000e+02 +9.350000000000000e+01
+9.894180000000000e+02 +1.445000000000000e+02
+2.166150000000000e+01 +2.690000000000000e+02
+6.542809999999999e+02 +2.690000000000000e+02
+1.111390000000000e+03 +2.690000000000000e+02
+1.800660000000000e+02 +2.690000000000000e+02
+7.570939999999998e+02 +1.295000000000000e+02
+1.704170000000000e+02 +2.685000000000000e+02
+7.437330000000002e+02 +1.245000000000000e+02
+7.355249999999999e+01 +2.680000000000000e+02
+9.044730000000000e+02 +1.820000000000000e+02
+2.347520000000000e+02 +2.675000000000000e+02
+4.666040000000000e+02 +2.675000000000000e+02
+1.370560000000000e+03 +2.300000000000000e+02
+2.168380000000000e+02 +2.675000000000000e+02
+1.010020000000000e+03 +2.250000000000000e+02
+1.191700000000000e+02 +2.670000000000000e+02
+1.143360000000000e+03 +2.670000000000000e+02
+8.667990000000000e+02 +2.670000000000000e+02
+1.095260000000000e+01 +2.665000000000000e+02
+1.004320000000000e+03 +1.275000000000000e+02
+8.756840000000000e+02 +1.460000000000000e+02
+1.945610000000000e+01 +2.660000000000000e+02
+1.297770000000000e+03 +1.505000000000000e+02
+9.069270000000000e+02 +8.900000000000000e+01
+7.475810000000000e+02 +1.295000000000000e+02
+7.626550000000000e+02 +1.295000000000000e+02
+1.369400000000000e+03 +2.350000000000000e+02
+6.631150000000000e+02 +1.100000000000000e+02
+1.087950000000000e+03 +2.100000000000000e+02
+7.576500000000000e+02 +1.000000000000000e+02
+8.854800000000000e+02 +1.110000000000000e+02
+7.905130000000000e+02 +2.645000000000000e+02
+1.066590000000000e+03 +2.125000000000000e+02
+7.439460000000000e+02 +1.465000000000000e+02
+6.491130000000001e+02 +7.950000000000000e+01
+1.285210000000000e+02 +2.645000000000000e+02
+1.151820000000000e+03 +2.645000000000000e+02
+6.856660000000001e+02 +2.645000000000000e+02
+4.269690000000000e+02 +2.640000000000000e+02
+6.480090000000000e+02 +7.650000000000000e+01
+1.616230000000000e+02 +2.640000000000000e+02
+7.972719999999998e+02 +2.640000000000000e+02
+1.494850000000000e+01 +2.635000000000000e+02
+1.794170000000000e+02 +2.635000000000000e+02
+6.470060000000000e+02 +2.635000000000000e+02
+1.066970000000000e+03 +1.460000000000000e+02
+1.592300000000000e+03 +2.450000000000000e+02
+5.267830000000000e+02 +2.630000000000000e+02
+6.132010000000000e+02 +7.900000000000000e+01
+7.532980000000000e+02 +2.625000000000000e+02
+4.543700000000000e+02 +6.850000000000000e+01
+7.554150000000000e+02 +1.230000000000000e+02
+2.818490000000000e+02 +9.500000000000000e+01
+8.799200000000000e+02 +8.150000000000000e+01
+2.845230000000000e+02 +2.610000000000000e+02
+3.142330000000000e+02 +2.610000000000000e+02
+7.227290000000001e+01 +2.610000000000000e+02
+8.635350000000000e+02 +2.610000000000000e+02
+5.684720000000000e-01 +2.605000000000000e+02
+5.101120000000000e+02 +2.600000000000000e+02
+2.291570000000000e+02 +2.600000000000000e+02
+1.104240000000000e+03 +1.445000000000000e+02
+7.687960000000000e+02 +1.305000000000000e+02
+7.748960000000002e+02 +1.380000000000000e+02
+3.730000000000000e+02 +2.595000000000000e+02
+1.093590000000000e+03 +2.595000000000000e+02
+8.835620000000000e+02 +1.720000000000000e+02
+1.278280000000000e+02 +2.590000000000000e+02
+5.403770000000000e+01 +2.590000000000000e+02
+2.773960000000000e+02 +2.590000000000000e+02
+1.194680000000000e+01 +2.585000000000000e+02
+1.293410000000000e+03 +1.385000000000000e+02
+8.569410000000000e+02 +1.405000000000000e+02
+1.726710000000000e+02 +2.580000000000000e+02
+6.265950000000000e+02 +1.005000000000000e+02
+1.109900000000000e+03 +1.415000000000000e+02
+3.025180000000000e+02 +2.580000000000000e+02
+7.533400000000000e+02 +1.235000000000000e+02
+2.185750000000000e+02 +2.575000000000000e+02
+1.658750000000000e+02 +2.575000000000000e+02
+1.579030000000000e+03 +1.930000000000000e+02
+6.387530000000000e+02 +7.850000000000000e+01
+9.010920000000000e+02 +8.250000000000000e+01
+7.750160000000002e+02 +2.570000000000000e+02
+9.737920000000000e+02 +1.315000000000000e+02
+8.727230000000002e+02 +2.565000000000000e+02
+2.746160000000000e+02 +2.565000000000000e+02
+1.312690000000000e+03 +1.430000000000000e+02
+9.097809999999999e+02 +8.200000000000000e+01
+1.330280000000000e+01 +2.555000000000000e+02
+3.568450000000000e+01 +2.550000000000000e+02
+6.592089999999999e+02 +9.700000000000000e+01
+8.480720000000000e+02 +7.600000000000000e+01
+7.488470000000000e+02 +1.170000000000000e+02
+3.743240000000000e+02 +2.545000000000000e+02
+7.482700000000000e+02 +1.385000000000000e+02
+1.799220000000000e+02 +2.545000000000000e+02
+1.311100000000000e+03 +1.430000000000000e+02
+1.068340000000000e+03 +2.545000000000000e+02
+4.727840000000000e+02 +2.545000000000000e+02
+2.124140000000000e+01 +2.540000000000000e+02
+3.209800000000000e+01 +2.535000000000000e+02
+6.486050000000000e+02 +1.605000000000000e+02
+7.483969999999998e+02 +1.210000000000000e+02
+7.443720000000000e+02 +1.150000000000000e+02
+6.631840000000000e+02 +1.595000000000000e+02
+5.335000000000000e+01 +2.530000000000000e+02
+1.612480000000000e+00 +2.530000000000000e+02
+2.276850000000000e+02 +2.525000000000000e+02
+2.784860000000000e+02 +8.700000000000000e+01
+7.845860000000000e+02 +2.525000000000000e+02
+7.753989999999999e+02 +9.250000000000000e+01
+7.997900000000000e+02 +1.405000000000000e+02
+1.785410000000000e+02 +2.520000000000000e+02
+1.124460000000000e+03 +2.520000000000000e+02
+1.781550000000000e+02 +2.520000000000000e+02
+1.305820000000000e+03 +1.390000000000000e+02
+1.100510000000000e+03 +1.395000000000000e+02
+6.907289999999998e+02 +1.740000000000000e+02
+1.101210000000000e+03 +2.510000000000000e+02
+6.493450000000000e+02 +1.660000000000000e+02
+7.519410000000000e+02 +2.505000000000000e+02
+8.532520000000000e+00 +2.500000000000000e+02
+7.583510000000001e+02 +1.230000000000000e+02
+1.121480000000000e+03 +2.500000000000000e+02
+2.289880000000000e+02 +2.495000000000000e+02
+1.223830000000000e+03 +1.970000000000000e+02
+2.699860000000000e+02 +2.490000000000000e+02
+8.875580000000000e+02 +2.490000000000000e+02
+1.506740000000000e+01 +2.485000000000000e+02
+7.456940000000000e+02 +1.155000000000000e+02
+4.032410000000000e+02 +2.480000000000000e+02
+1.068140000000000e+03 +1.875000000000000e+02
+1.066920000000000e+03 +1.850000000000000e+02
+9.331120000000000e+00 +2.475000000000000e+02
+9.737010000000000e+02 +1.270000000000000e+02
+7.657480000000000e+02 +1.150000000000000e+02
+6.169490000000002e+02 +2.470000000000000e+02
+1.586770000000000e+03 +2.250000000000000e+02
+2.204020000000000e+02 +2.465000000000000e+02
+2.814890000000001e+02 +2.095000000000000e+02
+5.223540000000000e+01 +2.465000000000000e+02
+7.460180000000000e+02 +1.200000000000000e+02
+1.581460000000000e+02 +2.460000000000000e+02
+1.328190000000000e+03 +2.460000000000000e+02
+6.351020000000000e+02 +2.455000000000000e+02
+6.351830000000000e+02 +1.515000000000000e+02
+1.525620000000000e+02 +2.450000000000000e+02
+3.482480000000001e+02 +2.550000000000000e+01
+1.651610000000000e+02 +2.445000000000000e+02
+7.619710000000000e+02 +1.260000000000000e+02
+1.150910000000000e+03 +2.445000000000000e+02
+1.595700000000000e+02 +2.435000000000000e+02
+1.361030000000000e+02 +2.430000000000000e+02
+9.752190000000001e+02 +1.330000000000000e+02
+9.010560000000000e+02 +7.500000000000000e+01
+2.133510000000000e+02 +2.420000000000000e+02
+3.841950000000000e+02 +2.420000000000000e+02
+6.394320000000000e+02 +1.510000000000000e+02
+1.339000000000000e+00 +2.420000000000000e+02
+1.605940000000000e+02 +2.420000000000000e+02
+2.221790000000000e+02 +2.410000000000000e+02
+1.064010000000000e+03 +1.885000000000000e+02
+3.174750000000000e+01 +2.410000000000000e+02
+1.082230000000000e+03 +1.270000000000000e+02
+1.065570000000000e+02 +2.405000000000000e+02
+3.624300000000000e+01 +2.405000000000000e+02
+1.120290000000000e+03 +1.255000000000000e+02
+1.584220000000000e+03 +2.155000000000000e+02
+7.498860000000002e+02 +1.195000000000000e+02
+9.028270000000000e+02 +1.125000000000000e+02
+2.621170000000000e+02 +2.395000000000000e+02
+1.356510000000000e+01 +2.395000000000000e+02
+2.813010000000000e+02 +1.885000000000000e+02
+5.375570000000000e-01 +2.395000000000000e+02
+7.961840000000000e+02 +1.160000000000000e+02
+1.298950000000000e+03 +1.320000000000000e+02
+8.708080000000000e+02 +6.800000000000000e+01
+8.132139999999998e+02 +2.395000000000000e+02
+2.351580000000000e+02 +2.390000000000000e+02
+6.744750000000000e+02 +1.145000000000000e+02
+1.798820000000000e+01 +2.390000000000000e+02
+7.506100000000000e+02 +1.085000000000000e+02
+9.910970000000000e+02 +1.880000000000000e+02
+4.269020000000000e+01 +2.385000000000000e+02
+5.374260000000001e-01 +2.385000000000000e+02
+9.233560000000000e+02 +2.385000000000000e+02
+3.128570000000000e+02 +2.350000000000000e+01
+9.446510000000000e+02 +2.370000000000000e+02
+9.051520000000000e+02 +1.135000000000000e+02
+1.381140000000000e+03 +2.370000000000000e+02
+1.406480000000000e+02 +2.365000000000000e+02
+7.940930000000002e+02 +1.265000000000000e+02
+6.348360000000000e+02 +2.360000000000000e+02
+6.536460000000000e+02 +2.360000000000000e+02
+7.900089999999999e+02 +1.130000000000000e+02
+9.801840000000000e+02 +1.165000000000000e+02
+8.598160000000000e+02 +7.050000000000000e+01
+1.081370000000000e+03 +1.845000000000000e+02
+6.709630000000002e+00 +2.350000000000000e+02
+1.117790000000000e+03 +1.270000000000000e+02
+2.539370000000000e+00 +2.345000000000000e+02
+1.507270000000000e+02 +2.345000000000000e+02
+9.776630000000000e+02 +1.805000000000000e+02
+2.507720000000000e+02 +2.345000000000000e+02
+2.391250000000000e+02 +2.340000000000000e+02
+2.418410000000000e+02 +2.340000000000000e+02
+1.093540000000000e+02 +2.340000000000000e+02
+9.214340000000000e+02 +2.340000000000000e+02
+4.520890000000000e+02 +4.200000000000000e+01
+1.603190000000000e+02 +2.335000000000000e+02
+1.192820000000000e+02 +2.330000000000000e+02
+6.185240000000000e+02 +7.350000000000000e+01
+6.254950000000000e+02 +7.550000000000000e+01
+8.985780000000000e+02 +1.195000000000000e+02
+2.059570000000000e+02 +2.320000000000000e+02
+1.253950000000000e+01 +2.320000000000000e+02
+8.971319999999999e+02 +2.320000000000000e+02
+4.716360000000000e+02 +6.900000000000000e+01
+1.333850000000000e+02 +2.315000000000000e+02
+7.916870000000000e+02 +1.110000000000000e+02
+1.304520000000000e+03 +2.200000000000000e+02
+1.003360000000000e+00 +2.310000000000000e+02
+6.353819999999999e+02 +1.390000000000000e+02
+2.007500000000000e+00 +2.305000000000000e+02
+1.402510000000000e+02 +2.305000000000000e+02
+7.936530000000000e+02 +1.255000000000000e+02
+9.574740000000000e+02 +2.305000000000000e+02
+2.221860000000000e+02 +2.300000000000000e+02
+7.690460000000000e+02 +2.300000000000000e+02
+9.785080000000000e+02 +1.830000000000000e+02
+6.547840000000000e+02 +2.300000000000000e+02
+1.689330000000000e+02 +2.300000000000000e+02
+1.648880000000000e+02 +2.295000000000000e+02
+4.447010000000000e+00 +2.295000000000000e+02
+9.957380000000001e+02 +1.260000000000000e+02
+1.386770000000000e+02 +2.295000000000000e+02
+1.181680000000000e+03 +2.295000000000000e+02
+2.128470000000000e+02 +2.290000000000000e+02
+6.222240000000000e+00 +2.290000000000000e+02
+4.408420000000000e+02 +2.290000000000000e+02
+8.901050000000000e+02 +1.435000000000000e+02
+1.781340000000000e+02 +2.285000000000000e+02
+2.204830000000000e+02 +2.285000000000000e+02
+5.890430000000000e+00 +2.285000000000000e+02
+2.245940000000000e+01 +2.285000000000000e+02
+9.773479999999999e+00 +2.280000000000000e+02
+1.336560000000000e+02 +2.280000000000000e+02
+8.692150000000000e+02 +1.190000000000000e+02
+1.308470000000000e+03 +2.200000000000000e+02
+1.068100000000000e+03 +1.155000000000000e+02
+8.528850000000000e+02 +2.275000000000000e+02
+1.947170000000000e+02 +2.270000000000000e+02
+5.152290000000000e+00 +2.270000000000000e+02
+3.506290000000000e+00 +2.270000000000000e+02
+1.344240000000000e+02 +2.270000000000000e+02
+8.739460000000000e+02 +9.950000000000000e+01
+7.444720000000000e+02 +1.115000000000000e+02
+1.329490000000000e+02 +2.265000000000000e+02
+6.198340000000002e+02 +7.200000000000000e+01
+7.914160000000001e+02 +1.130000000000000e+02
+9.621860000000000e+02 +1.765000000000000e+02
+4.028620000000000e-02 +2.260000000000000e+02
+1.151870000000000e+02 +2.260000000000000e+02
+7.819400000000001e+02 +1.080000000000000e+02
+5.942859999999999e+02 +2.255000000000000e+02
+7.101799999999999e+02 +2.250000000000000e+02
+6.367690000000000e+02 +1.360000000000000e+02
+1.218110000000000e+03 +1.725000000000000e+02
+1.068440000000000e+03 +1.785000000000000e+02
+3.382820000000000e+02 +7.600000000000000e+01
+1.883600000000000e+02 +2.240000000000000e+02
+2.146890000000000e+00 +2.240000000000000e+02
+6.564889999999998e+02 +1.325000000000000e+02
+2.082120000000000e+02 +2.235000000000000e+02
+1.293380000000000e+03 +2.235000000000000e+02
+1.949470000000000e+02 +2.230000000000000e+02
+1.758420000000000e+02 +2.230000000000000e+02
+1.210950000000000e+02 +2.230000000000000e+02
+7.869340000000000e+02 +1.085000000000000e+02
+9.058010000000000e+02 +1.105000000000000e+02
+7.415160000000002e+02 +1.100000000000000e+02
+1.192170000000000e+02 +2.220000000000000e+02
+8.917910000000001e+02 +1.415000000000000e+02
+2.869520000000000e+02 +6.150000000000000e+01
+6.538520000000000e+02 +1.345000000000000e+02
+9.809070000000000e+01 +2.215000000000000e+02
+1.156440000000000e+02 +2.210000000000000e+02
+1.464060000000000e+03 +2.210000000000000e+02
+1.660750000000000e+02 +2.205000000000000e+02
+2.644780000000000e+02 +5.850000000000000e+01
+6.614820000000000e+02 +1.315000000000000e+02
+3.726520000000000e-01 +2.205000000000000e+02
+1.757470000000000e+02 +2.200000000000000e+02
+3.817300000000000e+00 +2.195000000000000e+02
+1.212950000000000e+02 +2.195000000000000e+02
+4.802330000000000e-01 +2.190000000000000e+02
+6.642790000000000e+02 +1.335000000000000e+02
+7.800039999999998e+02 +1.115000000000000e+02
+1.278010000000000e+03 +2.185000000000000e+02
+8.992210000000000e+02 +1.090000000000000e+02
+5.788160000000000e+02 +2.180000000000000e+02
+4.795870000000000e+02 +6.100000000000000e+01
+7.746520000000000e-01 +2.175000000000000e+02
+8.646169999999999e+01 +2.175000000000000e+02
+2.299450000000000e-01 +2.175000000000000e+02
+8.534200000000000e+02 +2.170000000000000e+02
+7.421660000000001e+02 +1.060000000000000e+02
+7.869800000000000e+02 +1.130000000000000e+02
+4.564940000000000e+02 +2.165000000000000e+02
+1.547230000000000e+02 +2.155000000000000e+02
+1.073540000000000e+03 +1.565000000000000e+02
+7.122830000000000e-01 +2.155000000000000e+02
+1.277260000000000e+02 +2.150000000000000e+02
+1.215420000000000e+00 +2.150000000000000e+02
+1.024580000000000e+02 +2.150000000000000e+02
+6.736039999999998e+02 +1.335000000000000e+02
+8.932400000000000e+02 +1.090000000000000e+02
+8.893500000000000e+02 +1.295000000000000e+02
+8.723030000000000e+01 +2.140000000000000e+02
+4.201530000000000e+02 +6.450000000000000e+01
+8.488310000000000e+02 +1.025000000000000e+02
+6.647480000000000e+02 +1.300000000000000e+02
+7.673800000000000e+02 +1.015000000000000e+02
+1.030760000000000e+03 +2.130000000000000e+02
+6.756160000000001e+02 +2.130000000000000e+02
+5.801120000000000e+02 +2.130000000000000e+02
+1.449510000000000e+02 +2.125000000000000e+02
+7.418170000000000e+02 +1.035000000000000e+02
+1.591220000000000e+03 +1.915000000000000e+02
+6.779980000000000e+02 +2.125000000000000e+02
+7.336450000000000e+02 +2.120000000000000e+02
+2.365970000000000e+02 +2.120000000000000e+02
+9.364400000000001e+01 +2.120000000000000e+02
+9.781470000000000e+02 +1.665000000000000e+02
+8.123020000000000e+01 +2.115000000000000e+02
+7.124410000000000e+02 +2.115000000000000e+02
+1.077940000000000e+02 +2.115000000000000e+02
+9.199030000000000e+02 +1.305000000000000e+02
+6.192919999999998e+02 +5.550000000000000e+01
+7.805180000000000e+02 +1.005000000000000e+02
+1.120170000000000e+03 +1.755000000000000e+02
+1.782460000000000e-01 +2.105000000000000e+02
+8.339900000000000e+02 +1.070000000000000e+02
+9.038869999999999e+02 +1.025000000000000e+02
+1.121660000000000e+03 +1.605000000000000e+02
+4.331260000000000e+02 +2.090000000000000e+02
+7.056630000000000e+02 +2.090000000000000e+02
+1.699030000000000e-02 +2.090000000000000e+02
+3.340490000000001e+02 +6.300000000000000e+01
+9.885060000000000e+02 +1.635000000000000e+02
+1.389140000000000e+03 +1.690000000000000e+02
+9.011820000000000e+01 +2.085000000000000e+02
+7.751560000000002e+02 +1.060000000000000e+02
+1.152510000000000e+03 +2.085000000000000e+02
+4.272970000000000e+02 +2.075000000000000e+02
+8.250959999999999e-01 +2.075000000000000e+02
+7.794680000000002e+02 +9.850000000000000e+01
+1.098520000000000e+03 +1.710000000000000e+02
+2.346770000000000e+02 +2.065000000000000e+02
+7.809570000000000e+02 +1.040000000000000e+02
+8.911239999999998e+02 +1.070000000000000e+02
+6.674570000000000e+02 +2.065000000000000e+02
+4.033340000000000e+02 +2.060000000000000e+02
+9.881710000000000e+02 +1.630000000000000e+02
+1.875300000000000e+02 +2.060000000000000e+02
+1.579650000000000e+03 +1.835000000000000e+02
+9.032350000000000e+02 +1.075000000000000e+02
+1.614280000000000e+02 +2.060000000000000e+02
+1.266300000000000e+03 +2.055000000000000e+02
+6.786680000000000e+01 +2.055000000000000e+02
+6.612120000000000e+02 +5.250000000000000e+01
+2.258760000000000e+01 +2.050000000000000e+02
+7.803839999999999e+02 +9.550000000000000e+01
+1.054280000000000e+03 +2.050000000000000e+02
+2.347060000000000e+00 +2.050000000000000e+02
+1.571290000000000e+02 +2.045000000000000e+02
+6.600470000000000e+02 +1.175000000000000e+02
+1.300720000000000e+03 +1.945000000000000e+02
+7.040980000000002e+02 +2.045000000000000e+02
+1.271820000000000e+00 +2.045000000000000e+02
+1.289570000000000e+02 +2.040000000000000e+02
+3.479330000000000e+01 +2.040000000000000e+02
+9.736609999999999e+02 +2.040000000000000e+02
+7.482220000000000e+02 +9.750000000000000e+01
+1.300140000000000e+03 +1.980000000000000e+02
+4.855970000000000e+02 +2.035000000000000e+02
+9.852089999999999e+02 +1.555000000000000e+02
+6.147680000000000e+02 +1.150000000000000e+02
+1.300780000000000e+03 +1.905000000000000e+02
+7.667980000000000e+02 +2.025000000000000e+02
+6.607189999999998e+02 +1.165000000000000e+02
+3.369800000000000e+02 +5.750000000000000e+01
+8.515650000000001e+02 +1.005000000000000e+02
+4.738400000000000e+02 +2.015000000000000e+02
+1.309790000000000e+03 +1.940000000000000e+02
+1.321030000000000e+03 +1.905000000000000e+02
+1.980450000000000e+02 +2.005000000000000e+02
+7.782750000000000e+02 +9.400000000000000e+01
+8.631710000000000e+02 +2.005000000000000e+02
+5.374939999999999e-01 +2.005000000000000e+02
+2.084510000000000e+02 +2.000000000000000e+02
+7.897689999999999e+02 +1.000000000000000e+02
+1.315670000000000e+03 +1.950000000000000e+02
+9.017400000000000e+02 +9.450000000000000e+01
+8.495239999999999e+02 +1.995000000000000e+02
+8.917180000000002e+02 +9.950000000000000e+01
+6.632869999999998e+02 +1.155000000000000e+02
+1.111400000000000e+03 +1.650000000000000e+02
+1.798950000000000e+02 +1.980000000000000e+02
+7.481130000000001e+02 +9.350000000000000e+01
+6.554260000000000e+02 +1.980000000000000e+02
+1.304120000000000e+03 +1.930000000000000e+02
+8.795610000000001e+01 +1.970000000000000e+02
+7.292589999999999e+02 +1.965000000000000e+02
+6.219030000000000e+02 +1.575000000000000e+02
+1.069340000000000e+03 +1.480000000000000e+02
+5.658840000000000e+02 +1.960000000000000e+02
+7.404050000000000e+02 +1.960000000000000e+02
+9.531530000000000e+02 +7.650000000000000e+01
+8.662569999999999e+02 +1.185000000000000e+02
+8.618339999999999e+02 +1.415000000000000e+02
+5.965040000000000e+01 +1.955000000000000e+02
+6.241080000000002e+02 +1.540000000000000e+02
+8.881849999999999e+02 +1.955000000000000e+02
+9.350539999999999e+01 +1.950000000000000e+02
+1.446630000000000e+03 +1.950000000000000e+02
+2.817120000000000e+02 +3.850000000000000e+01
+6.482989999999999e+01 +1.945000000000000e+02
+8.901840000000000e+02 +9.800000000000000e+01
+3.132250000000000e+02 +1.945000000000000e+02
+6.243250000000000e+02 +1.945000000000000e+02
+5.375100000000000e-01 +1.940000000000000e+02
+2.008560000000000e+02 +1.935000000000000e+02
+2.772760000000000e+02 +1.485000000000000e+02
+6.489610000000000e+01 +1.935000000000000e+02
+9.312380000000000e+01 +1.935000000000000e+02
+9.776310000000000e+02 +1.535000000000000e+02
+1.069740000000000e+02 +1.930000000000000e+02
+7.476010000000001e+02 +1.930000000000000e+02
+6.981770000000000e+01 +1.930000000000000e+02
+9.070410000000001e+02 +1.005000000000000e+02
+7.989409999999999e+01 +1.930000000000000e+02
+6.745959999999999e+01 +1.925000000000000e+02
+1.457980000000000e+03 +1.775000000000000e+02
+1.060270000000000e+03 +1.345000000000000e+02
+6.839689999999998e+02 +1.920000000000000e+02
+6.789470000000000e+01 +1.920000000000000e+02
+1.405250000000000e+03 +1.920000000000000e+02
+4.747260000000000e+02 +1.915000000000000e+02
+7.656170000000000e+02 +8.500000000000000e+01
+8.368180000000000e+02 +9.000000000000000e+01
+6.359740000000000e+02 +1.070000000000000e+02
+7.866030000000002e+02 +8.650000000000000e+01
+8.477819999999998e+02 +1.910000000000000e+02
+9.097080000000000e+02 +9.750000000000000e+01
+1.914570000000000e+02 +1.905000000000000e+02
+1.484200000000000e+02 +1.900000000000000e+02
+8.315140000000000e+01 +1.900000000000000e+02
+7.722130000000002e+02 +1.625000000000000e+02
+8.357060000000000e+02 +1.900000000000000e+02
+8.447810000000002e+02 +1.900000000000000e+02
+2.604900000000000e+00 +1.895000000000000e+02
+6.624730000000002e+02 +1.125000000000000e+02
+1.295810000000000e+03 +1.830000000000000e+02
+9.114220000000000e+02 +9.950000000000000e+01
+9.921710000000000e+02 +1.470000000000000e+02
+8.634910000000001e+02 +9.450000000000000e+01
+6.013310000000000e+01 +1.885000000000000e+02
+7.856990000000000e+02 +1.640000000000000e+02
+9.769210000000000e+02 +1.505000000000000e+02
+8.540089999999999e+02 +1.885000000000000e+02
+4.619960000000000e+02 +1.880000000000000e+02
+6.222030000000000e+02 +1.470000000000000e+02
+1.135050000000000e+03 +1.565000000000000e+02
+6.659230000000000e+02 +1.880000000000000e+02
+8.418880000000000e+02 +1.880000000000000e+02
+8.113270000000000e+01 +1.875000000000000e+02
+6.625570000000000e+02 +1.495000000000000e+02
+7.847600000000000e+02 +8.500000000000000e+01
+1.445630000000000e+03 +1.875000000000000e+02
+7.531710000000000e+02 +7.450000000000000e+01
+8.594090000000000e+02 +1.870000000000000e+02
+5.493950000000000e+01 +1.865000000000000e+02
+2.907750000000000e-01 +1.865000000000000e+02
+7.312470000000000e+01 +1.860000000000000e+02
+1.048160000000000e+03 +1.565000000000000e+02
+2.926440000000000e+02 +1.855000000000000e+02
+8.236940000000000e+01 +1.855000000000000e+02
+7.860680000000000e+02 +8.400000000000000e+01
+7.924530000000000e+02 +1.855000000000000e+02
+1.310050000000000e+03 +1.810000000000000e+02
+1.615980000000000e+02 +1.845000000000000e+02
+6.640760000000000e+02 +6.850000000000000e+01
+1.078740000000000e+03 +1.795000000000000e+02
+1.674760000000000e-01 +1.845000000000000e+02
+1.104880000000000e+03 +1.525000000000000e+02
+1.385880000000000e+02 +1.835000000000000e+02
+8.131540000000000e+01 +1.835000000000000e+02
+3.976350000000000e+01 +1.835000000000000e+02
+7.976049999999999e+01 +1.835000000000000e+02
+7.500570000000000e+02 +8.000000000000000e+01
+8.374600000000000e+02 +1.830000000000000e+02
+9.369550000000000e+02 +1.825000000000000e+02
+1.177650000000000e+03 +1.825000000000000e+02
+5.727420000000000e+02 +1.825000000000000e+02
+9.480950000000000e+02 +1.820000000000000e+02
+7.155070000000001e+01 +1.815000000000000e+02
+5.182150000000000e+02 +1.815000000000000e+02
+5.498360000000000e-01 +1.815000000000000e+02
+7.509320000000000e+01 +1.815000000000000e+02
+1.848850000000000e+02 +1.810000000000000e+02
+6.108990000000000e+02 +1.810000000000000e+02
+2.014980000000000e+02 +1.810000000000000e+02
+1.066230000000000e+03 +1.380000000000000e+02
+4.206050000000000e+02 +9.050000000000000e+01
+1.318070000000000e+03 +1.785000000000000e+02
+7.582200000000000e+02 +1.800000000000000e+02
+2.474070000000000e+02 +1.800000000000000e+02
+7.490039999999998e+02 +1.800000000000000e+02
+1.163710000000000e+03 +1.800000000000000e+02
+2.483240000000000e+02 +1.800000000000000e+02
+9.054080000000001e+00 +1.800000000000000e+02
+1.005280000000000e+03 +1.800000000000000e+02
+7.427739999999999e+02 +1.800000000000000e+02
+6.765390000000000e+02 +9.700000000000000e+01
+1.169650000000000e+03 +1.800000000000000e+02
+2.444960000000000e+02 +1.800000000000000e+02
+4.271340000000000e+00 +1.800000000000000e+02
+6.408320000000000e+02 +9.850000000000000e+01
+1.295560000000000e+03 +1.790000000000000e+02
+9.352120000000000e+02 +1.790000000000000e+02
+9.072960000000000e+02 +9.400000000000000e+01
+3.605810000000000e+01 +1.785000000000000e+02
+7.747089999999999e+02 +1.560000000000000e+02
+1.109040000000000e+03 +1.490000000000000e+02
+1.156650000000000e+03 +1.780000000000000e+02
+8.994760000000001e+02 +9.500000000000000e+01
+3.484880000000000e-01 +1.780000000000000e+02
+1.841490000000000e+02 +1.775000000000000e+02
+5.988290000000000e+01 +1.775000000000000e+02
+1.083790000000000e+03 +1.540000000000000e+02
+8.661610000000000e+01 +1.770000000000000e+02
+1.074650000000000e+03 +1.770000000000000e+02
+7.812810000000002e+02 +1.510000000000000e+02
+9.730380000000000e+02 +1.285000000000000e+02
+6.473950000000000e+02 +8.650000000000000e+01
+5.680140000000000e+02 +1.760000000000000e+02
+6.244710000000000e+02 +9.600000000000000e+01
+5.874780000000000e+01 +1.755000000000000e+02
+6.618240000000000e+02 +1.360000000000000e+02
+8.630030000000000e+02 +1.745000000000000e+02
+4.414420000000000e+02 +1.740000000000000e+02
+8.830790000000000e+02 +9.650000000000000e+01
+5.012080000000000e+02 +1.000000000000000e+02
+7.392440000000001e+01 +1.735000000000000e+02
+3.130090000000000e+02 +1.735000000000000e+02
+7.820800000000000e+02 +1.515000000000000e+02
+1.646710000000000e+01 +1.725000000000000e+02
+8.972040000000000e+02 +1.350000000000000e+02
+8.596060000000001e+02 +1.305000000000000e+02
+4.074930000000000e+01 +1.715000000000000e+02
+1.740430000000000e+01 +1.710000000000000e+02
+1.803420000000000e+02 +1.710000000000000e+02
+9.787600000000000e+02 +1.335000000000000e+02
+8.792210000000000e+02 +9.550000000000000e+01
+2.506550000000000e+00 +1.710000000000000e+02
+6.419100000000000e+02 +1.705000000000000e+02
+1.270880000000000e+02 +1.695000000000000e+02
+5.313170000000000e+02 +1.695000000000000e+02
+3.257150000000000e+02 +1.695000000000000e+02
+3.320530000000001e+02 +1.690000000000000e+02
+7.107010000000000e+02 +1.685000000000000e+02
+1.066680000000000e+03 +1.305000000000000e+02
+1.082400000000000e+03 +1.215000000000000e+02
+5.915050000000000e+01 +1.680000000000000e+02
+7.847120000000000e+02 +1.485000000000000e+02
+9.844640000000000e-01 +1.680000000000000e+02
+5.878990000000000e+02 +1.170000000000000e+02
+8.290010000000001e+00 +1.675000000000000e+02
+1.056750000000000e+03 +1.310000000000000e+02
+6.741469999999998e+02 +1.675000000000000e+02
+7.797850000000000e+02 +1.450000000000000e+02
+6.378810000000000e+02 +8.850000000000000e+01
+7.841330000000000e+02 +1.455000000000000e+02
+3.352020000000000e+02 +3.150000000000000e+01
+1.079640000000000e+03 +1.400000000000000e+02
+8.883930000000000e+02 +1.315000000000000e+02
+1.284200000000000e+03 +1.645000000000000e+02
+1.971890000000000e+02 +1.665000000000000e+02
+6.594090000000000e+02 +9.050000000000000e+01
+8.948950000000000e+02 +1.285000000000000e+02
+1.323730000000000e+02 +1.655000000000000e+02
+1.129020000000000e+02 +1.655000000000000e+02
+7.426610000000002e+02 +6.400000000000000e+01
+3.840870000000000e+01 +1.650000000000000e+02
+1.315250000000000e-01 +1.650000000000000e+02
+5.132830000000000e+01 +1.645000000000000e+02
+2.793690000000000e+02 +2.100000000000000e+01
+4.627680000000000e+01 +1.645000000000000e+02
+9.753780000000000e+02 +1.295000000000000e+02
+9.038520000000000e+02 +1.295000000000000e+02
+3.263350000000000e+02 +1.645000000000000e+02
+1.070210000000000e+03 +1.245000000000000e+02
+3.573260000000000e+02 +1.640000000000000e+02
+6.055840000000002e+02 +1.640000000000000e+02
+3.135530000000000e+02 +1.640000000000000e+02
+3.349470000000000e+00 +1.635000000000000e+02
+1.125860000000000e+03 +1.635000000000000e+02
+8.577310000000001e+02 +1.635000000000000e+02
+5.200660000000000e+02 +1.630000000000000e+02
+7.876790000000000e+02 +1.435000000000000e+02
+7.831369999999999e+02 +1.445000000000000e+02
+1.077590000000000e+03 +1.115000000000000e+02
+1.423970000000000e+02 +1.615000000000000e+02
+5.562950000000000e+02 +1.615000000000000e+02
+6.236330000000000e+02 +1.610000000000000e+02
+4.819400000000000e+01 +1.610000000000000e+02
+7.852830000000000e+02 +1.445000000000000e+02
+1.604550000000000e+02 +1.600000000000000e+02
+2.884230000000000e+01 +1.600000000000000e+02
+1.122600000000000e+03 +1.350000000000000e+02
+5.675980000000002e+02 +1.595000000000000e+02
+1.932750000000000e+02 +1.590000000000000e+02
+7.846020000000000e+02 +1.405000000000000e+02
+6.199109999999999e+02 +1.585000000000000e+02
+2.764090000000000e+01 +1.580000000000000e+02
+8.789500000000000e+02 +1.055000000000000e+02
+1.197360000000000e-01 +1.575000000000000e+02
+7.867619999999999e+02 +1.405000000000000e+02
+5.194180000000000e+02 +1.575000000000000e+02
+4.219140000000000e+02 +1.575000000000000e+02
+9.791400000000000e+02 +1.240000000000000e+02
+8.242310000000001e+02 +1.570000000000000e+02
+1.065860000000000e+02 +1.565000000000000e+02
+1.077900000000000e+03 +1.120000000000000e+02
+9.539560000000000e+01 +1.555000000000000e+02
+6.189360000000000e+02 +1.175000000000000e+02
+3.339070000000000e+02 +2.550000000000000e+01
+6.513099999999999e+02 +1.550000000000000e+02
+1.701220000000000e+01 +1.545000000000000e+02
+7.881310000000002e+02 +1.385000000000000e+02
+6.993890000000000e+00 +1.535000000000000e+02
+2.889200000000000e+01 +1.535000000000000e+02
+9.753190000000000e+02 +1.225000000000000e+02
+8.152589999999999e+02 +1.525000000000000e+02
+3.284790000000000e+01 +1.525000000000000e+02
+8.522430000000000e+00 +1.520000000000000e+02
+1.073610000000000e+02 +1.520000000000000e+02
+3.871840000000000e+01 +1.520000000000000e+02
+1.859290000000000e+02 +1.515000000000000e+02
+9.706440000000000e+02 +1.465000000000000e+02
+7.819520000000000e+02 +1.255000000000000e+02
+2.787460000000000e+02 +1.515000000000000e+02
+1.390720000000000e+02 +1.515000000000000e+02
+1.902150000000000e+02 +1.510000000000000e+02
+4.456290000000000e+01 +1.510000000000000e+02
+3.508100000000000e+01 +1.510000000000000e+02
+3.357590000000000e+02 +2.600000000000000e+01
+8.917000000000000e+02 +1.175000000000000e+02
+1.371220000000000e+00 +1.505000000000000e+02
+6.183320000000000e+01 +1.505000000000000e+02
+7.864730000000002e+02 +1.275000000000000e+02
+3.536050000000000e+01 +1.500000000000000e+02
+8.210319999999998e+02 +1.500000000000000e+02
+1.140380000000000e+02 +1.495000000000000e+02
+5.135080000000000e+02 +1.490000000000000e+02
+8.617000000000000e+02 +1.490000000000000e+02
+7.883270000000000e+02 +1.245000000000000e+02
+1.332860000000000e+02 +1.470000000000000e+02
+9.743270000000000e+02 +1.160000000000000e+02
+5.915990000000000e+02 +1.470000000000000e+02
+3.039230000000000e+01 +1.465000000000000e+02
+8.008270000000000e+02 +1.330000000000000e+02
+9.767400000000000e+02 +1.200000000000000e+02
+2.840670000000000e+01 +1.460000000000000e+02
+4.950460000000000e+02 +3.100000000000000e+01
+9.045560000000000e+02 +1.165000000000000e+02
+8.011510000000002e+02 +1.450000000000000e+02
+1.278260000000000e+03 +1.450000000000000e+02
+6.192360000000000e+02 +1.445000000000000e+02
+7.828250000000000e+02 +1.310000000000000e+02
+8.311810000000000e+02 +1.445000000000000e+02
+8.108869999999999e+02 +1.435000000000000e+02
+3.880470000000000e+01 +1.435000000000000e+02
+9.258150000000001e+02 +1.155000000000000e+02
+1.551730000000000e+02 +1.430000000000000e+02
+6.411410000000000e+02 +1.210000000000000e+02
+5.344890000000000e+02 +1.430000000000000e+02
+3.005870000000000e+01 +1.425000000000000e+02
+7.941189999999998e+02 +1.320000000000000e+02
+7.834360000000000e+02 +1.295000000000000e+02
+9.837809999999999e+02 +1.170000000000000e+02
+1.701600000000000e+02 +1.415000000000000e+02
+9.106210000000000e+02 +1.140000000000000e+02
+6.453270000000000e+02 +1.410000000000000e+02
+1.040270000000000e+03 +1.185000000000000e+02
+5.993840000000000e+02 +9.400000000000000e+01
+3.317570000000000e+01 +1.400000000000000e+02
+8.715650000000001e+02 +1.400000000000000e+02
+5.137919999999998e+02 +1.395000000000000e+02
+7.905410000000001e+02 +1.250000000000000e+02
+3.685750000000000e+02 +1.390000000000000e+02
+1.255500000000000e+03 +1.385000000000000e+02
+9.585980000000000e+02 +1.385000000000000e+02
+8.877439999999998e+02 +1.380000000000000e+02
+7.488450000000000e+02 +1.300000000000000e+02
+1.093460000000000e+03 +1.380000000000000e+02
+8.964360000000000e+02 +1.225000000000000e+02
+8.969620000000000e+02 +1.380000000000000e+02
+5.227370000000000e+02 +1.375000000000000e+02
+2.243580000000000e+01 +1.375000000000000e+02
+7.894040000000000e+02 +1.215000000000000e+02
+7.419400000000001e+02 +1.375000000000000e+02
+6.469650000000000e+02 +1.370000000000000e+02
+9.273250000000000e+02 +7.150000000000000e+01
+7.917330000000002e+02 +1.365000000000000e+02
+7.488860000000002e+02 +1.280000000000000e+02
+6.868519999999999e+01 +1.355000000000000e+02
+8.975590000000000e+02 +1.145000000000000e+02
+7.878980000000000e+02 +1.180000000000000e+02
+3.639280000000001e+02 +1.350000000000000e+02
+1.317550000000000e+02 +1.345000000000000e+02
+2.314520000000000e+01 +1.345000000000000e+02
+8.725500000000000e+02 +1.120000000000000e+02
+9.614299999999999e+02 +1.345000000000000e+02
+9.367890000000000e+02 +1.215000000000000e+02
+6.248470000000000e+02 +9.450000000000000e+01
+6.827420000000000e+02 +1.340000000000000e+02
+4.518720000000000e+02 +6.100000000000000e+01
+1.297180000000000e+02 +1.335000000000000e+02
+6.105250000000000e+01 +1.335000000000000e+02
+1.086200000000000e+02 +1.335000000000000e+02
+9.604000000000000e+02 +1.330000000000000e+02
+3.869330000000000e+01 +1.330000000000000e+02
+7.866439999999999e+02 +1.250000000000000e+02
+2.213950000000000e+01 +1.320000000000000e+02
+7.936630000000000e+02 +1.265000000000000e+02
+6.450960000000000e+02 +1.320000000000000e+02
+9.995599999999999e+02 +7.800000000000000e+01
+2.062390000000000e+00 +1.315000000000000e+02
+1.834280000000000e+01 +1.315000000000000e+02
+5.131360000000000e+02 +1.310000000000000e+02
+9.931760000000000e+01 +1.305000000000000e+02
+8.007719999999998e+02 +1.305000000000000e+02
+6.245540000000000e+02 +9.800000000000000e+01
+1.037430000000000e+03 +1.305000000000000e+02
+8.971020000000000e+02 +1.080000000000000e+02
+1.363750000000000e+02 +1.300000000000000e+02
+1.939850000000000e+01 +1.300000000000000e+02
+1.152620000000000e+02 +1.300000000000000e+02
+1.091870000000000e+03 +1.050000000000000e+02
+6.158170000000000e+02 +1.300000000000000e+02
+7.718539999999998e+02 +1.115000000000000e+02
+2.677950000000000e+02 +1.295000000000000e+02
+1.751660000000000e+01 +1.290000000000000e+02
+2.814530000000000e+02 +8.150000000000000e+01
+5.076670000000000e+02 +1.285000000000000e+02
+1.101270000000000e+03 +1.275000000000000e+02
+2.085190000000000e+01 +1.265000000000000e+02
+8.927780000000000e+02 +1.205000000000000e+02
+4.797780000000000e+01 +1.260000000000000e+02
+3.393090000000000e+00 +1.260000000000000e+02
+7.792189999999998e+02 +1.260000000000000e+02
+6.602209999999999e+01 +1.260000000000000e+02
+6.211300000000000e+02 +9.050000000000000e+01
+5.062940000000000e+02 +1.260000000000000e+02
+1.403950000000000e+02 +1.255000000000000e+02
+8.591039999999998e+02 +1.255000000000000e+02
+4.711390000000000e+02 +1.255000000000000e+02
+7.673200000000001e+02 +1.145000000000000e+02
+1.174800000000000e+01 +1.245000000000000e+02
+7.691590000000000e+00 +1.245000000000000e+02
+7.794100000000000e+02 +1.240000000000000e+02
+1.225710000000000e+02 +1.235000000000000e+02
+4.984950000000000e+02 +1.235000000000000e+02
+7.755230000000000e+02 +1.095000000000000e+02
+1.095020000000000e+03 +1.235000000000000e+02
+2.772820000000000e+02 +7.750000000000000e+01
+1.086090000000000e+03 +1.230000000000000e+02
+8.802800000000000e+00 +1.225000000000000e+02
+1.113270000000000e+03 +1.225000000000000e+02
+1.009610000000000e+02 +1.220000000000000e+02
+7.705319999999998e+02 +1.180000000000000e+02
+3.172640000000000e+02 +1.220000000000000e+02
+1.129870000000000e+01 +1.215000000000000e+02
+5.545570000000000e+02 +1.215000000000000e+02
+7.072100000000000e+02 +1.210000000000000e+02
+6.384590000000002e+02 +1.205000000000000e+02
+1.151420000000000e+02 +1.200000000000000e+02
+4.872980000000000e+02 +6.850000000000000e+01
+9.872300000000000e+02 +1.155000000000000e+02
+8.878030000000000e+02 +1.185000000000000e+02
+6.469620000000000e+00 +1.195000000000000e+02
+4.146380000000000e+02 +1.190000000000000e+02
+8.984480000000000e+02 +7.100000000000000e+01
+7.467050000000000e+02 +1.115000000000000e+02
+4.491590000000000e+02 +1.185000000000000e+02
+5.720210000000000e+00 +1.185000000000000e+02
+8.352509999999999e+01 +1.185000000000000e+02
+4.908670000000000e+02 +1.185000000000000e+02
+8.472139999999998e+02 +9.900000000000000e+01
+8.559770000000000e+01 +1.180000000000000e+02
+9.054170000000001e+01 +1.180000000000000e+02
+5.912490000000000e+02 +1.180000000000000e+02
+6.841000000000000e+02 +1.180000000000000e+02
+6.246230000000000e+02 +1.180000000000000e+02
+4.565240000000000e+02 +4.850000000000000e+01
+1.769520000000000e+01 +1.175000000000000e+02
+7.682700000000000e+02 +1.085000000000000e+02
+4.484270000000000e+02 +1.175000000000000e+02
+9.203350000000000e+02 +1.175000000000000e+02
+6.153030000000000e+02 +1.165000000000000e+02
+9.070660000000000e+02 +9.950000000000000e+01
+2.586830000000000e+00 +1.160000000000000e+02
+7.761400000000000e+02 +1.155000000000000e+02
+3.740220000000000e+00 +1.155000000000000e+02
+7.772970000000000e+02 +1.090000000000000e+02
+1.404280000000000e+01 +1.150000000000000e+02
+3.340410000000000e+02 +7.900000000000000e+01
+8.815280000000000e+02 +1.005000000000000e+02
+2.349260000000000e+00 +1.145000000000000e+02
+5.132520000000000e+02 +1.145000000000000e+02
+9.083620000000000e+02 +9.600000000000000e+01
+2.010000000000000e+01 +1.140000000000000e+02
+7.729639999999998e+02 +1.120000000000000e+02
+9.760230000000000e+02 +1.140000000000000e+02
+2.755640000000000e+02 +6.950000000000000e+01
+4.348900000000000e+02 +1.135000000000000e+02
+3.868060000000000e+02 +1.135000000000000e+02
+9.785860000000000e+02 +1.130000000000000e+02
+8.499830000000002e+02 +1.130000000000000e+02
+2.540950000000000e+02 +1.130000000000000e+02
+6.584110000000002e+02 +7.900000000000000e+01
+7.827200000000000e+02 +1.075000000000000e+02
+1.253040000000000e+02 +1.120000000000000e+02
+4.407890000000000e+02 +1.120000000000000e+02
+6.168840000000000e-01 +1.120000000000000e+02
+3.424850000000000e+00 +1.115000000000000e+02
+1.384700000000000e+01 +1.115000000000000e+02
+4.398880000000000e+02 +1.115000000000000e+02
+6.181830000000000e+02 +1.115000000000000e+02
+9.028049999999999e+02 +1.065000000000000e+02
+4.370440000000000e+02 +1.110000000000000e+02
+6.181600000000000e+02 +7.200000000000000e+01
+7.828010000000000e+02 +1.095000000000000e+02
+4.525960000000000e+02 +1.105000000000000e+02
+5.442619999999999e+02 +1.105000000000000e+02
+7.457850000000001e+01 +1.100000000000000e+02
+6.716880000000000e+02 +1.100000000000000e+02
+3.342720000000000e+02 +1.100000000000000e+02
+9.265060000000000e+02 +1.065000000000000e+02
+1.162170000000000e+02 +1.090000000000000e+02
+7.694010000000002e+02 +1.050000000000000e+02
+3.339870000000000e+02 +7.550000000000000e+01
+6.362700000000000e+02 +1.090000000000000e+02
+1.245290000000000e+02 +1.085000000000000e+02
+5.867500000000000e+01 +1.085000000000000e+02
+2.433980000000000e+02 +1.080000000000000e+02
+2.510230000000000e+01 +1.075000000000000e+02
+4.976210000000000e+02 +1.075000000000000e+02
+8.444370000000000e+02 +9.650000000000000e+01
+6.716760000000000e+02 +1.075000000000000e+02
+2.893740000000000e+01 +1.070000000000000e+02
+6.624360000000000e+02 +1.070000000000000e+02
+8.614589999999999e+02 +1.035000000000000e+02
+6.026820000000000e+02 +1.060000000000000e+02
+2.772920000000000e+02 +6.350000000000000e+01
+6.630580000000000e+02 +7.000000000000000e+01
+7.717220000000000e+02 +1.025000000000000e+02
+4.334980000000001e+02 +1.055000000000000e+02
+8.296050000000000e+02 +1.055000000000000e+02
+1.990320000000000e+01 +1.050000000000000e+02
+6.231680000000000e+02 +7.000000000000000e+01
+4.787060000000000e+01 +1.045000000000000e+02
+7.530250000000000e+02 +1.040000000000000e+02
+5.374750000000000e-01 +1.035000000000000e+02
+7.814490000000000e+02 +1.020000000000000e+02
+4.230920000000000e+02 +1.035000000000000e+02
+7.136180000000001e+02 +1.025000000000000e+02
+9.315260000000001e+01 +1.020000000000000e+02
+4.154450000000000e+02 +1.020000000000000e+02
+6.171849999999999e+02 +1.020000000000000e+02
+4.977590000000000e+01 +1.015000000000000e+02
+5.475980000000002e+02 +1.015000000000000e+02
+1.934610000000000e+02 +1.010000000000000e+02
+4.593690000000000e+01 +1.010000000000000e+02
+7.870660000000000e+02 +1.010000000000000e+02
+5.965910000000000e+02 +1.010000000000000e+02
+6.220190000000000e+02 +6.950000000000000e+01
+5.876300000000000e+02 +1.005000000000000e+02
+1.093250000000000e+03 +1.000000000000000e+02
+6.369750000000000e+02 +6.650000000000000e+01
+3.973210000000000e+02 +1.000000000000000e+02
+5.779290000000000e+02 +1.000000000000000e+02
+7.418250000000000e+02 +9.950000000000000e+01
+7.420210000000002e+02 +9.450000000000000e+01
+8.875910000000000e+02 +9.700000000000000e+01
+8.528370000000000e+02 +9.900000000000000e+01
+3.178400000000000e+02 +9.900000000000000e+01
+7.358539999999998e+02 +9.850000000000000e+01
+3.760460000000000e+01 +9.850000000000000e+01
+7.191230000000000e+02 +9.850000000000000e+01
+5.471559999999999e+02 +9.850000000000000e+01
+1.263960000000000e+02 +9.800000000000000e+01
+6.982840000000000e+02 +9.800000000000000e+01
+3.572550000000000e+02 +9.800000000000000e+01
+9.243450000000000e+01 +9.750000000000000e+01
+6.934120000000000e+01 +9.750000000000000e+01
+5.187230000000002e+02 +9.750000000000000e+01
+8.449970000000000e+02 +8.500000000000000e+01
+5.631220000000000e+01 +9.700000000000000e+01
+7.135120000000001e+01 +9.700000000000000e+01
+3.278820000000000e+01 +9.700000000000000e+01
+7.714730000000002e+02 +9.700000000000000e+01
+4.872820000000000e+02 +7.500000000000000e+01
+1.120190000000000e+03 +9.650000000000000e+01
+5.783800000000000e+02 +9.650000000000000e+01
+1.120810000000000e+01 +9.600000000000000e+01
+1.209440000000000e+02 +9.600000000000000e+01
+6.190820000000000e+01 +9.600000000000000e+01
+7.439040000000000e+02 +9.600000000000000e+01
+4.223750000000000e+02 +9.600000000000000e+01
+5.802859999999999e+02 +9.600000000000000e+01
+9.054480000000000e+02 +9.600000000000000e+01
+3.648210000000000e+02 +9.600000000000000e+01
+4.245830000000000e+02 +9.600000000000000e+01
+4.118250000000000e+02 +9.500000000000000e+01
+5.740180000000000e+02 +9.500000000000000e+01
+3.278810000000000e+01 +9.500000000000000e+01
+1.074350000000000e+02 +9.450000000000000e+01
+7.438420000000000e+02 +9.250000000000000e+01
+9.542820000000000e+02 +9.450000000000000e+01
+4.313740000000000e+02 +9.450000000000000e+01
+1.594580000000000e+02 +9.400000000000000e+01
+8.665770000000000e+02 +9.400000000000000e+01
+6.476580000000000e+02 +4.900000000000000e+01
+2.764920000000000e+02 +9.400000000000000e+01
+6.468290000000001e+00 +9.350000000000000e+01
+3.362770000000000e+02 +6.650000000000000e+01
+2.543070000000000e+02 +9.350000000000000e+01
+9.551600000000001e+01 +9.300000000000000e+01
+6.172790000000000e+01 +9.300000000000000e+01
+5.669069999999998e+02 +9.300000000000000e+01
+1.402890000000000e+02 +9.250000000000000e+01
+6.500480000000000e+02 +9.250000000000000e+01
+2.053000000000000e+01 +9.200000000000000e+01
+6.349960000000000e+02 +9.200000000000000e+01
+6.772830000000000e+01 +9.150000000000000e+01
+1.086700000000000e+02 +9.150000000000000e+01
+1.094820000000000e+02 +9.150000000000000e+01
+1.839990000000000e+01 +9.150000000000000e+01
+9.513650000000000e+01 +9.100000000000000e+01
+1.244530000000000e+02 +9.100000000000000e+01
+7.127030000000000e+02 +9.100000000000000e+01
+5.265090000000000e+02 +9.100000000000000e+01
+5.965880000000002e+02 +9.100000000000000e+01
+8.058869999999999e+02 +9.050000000000000e+01
+2.989360000000000e+00 +9.050000000000000e+01
+6.434580000000002e+02 +9.050000000000000e+01
+8.045930000000002e+02 +9.050000000000000e+01
+3.992860000000000e+02 +9.050000000000000e+01
+7.618460000000000e+01 +9.000000000000000e+01
+1.132400000000000e+02 +9.000000000000000e+01
+4.222170000000000e+02 +9.000000000000000e+01
+5.032200000000000e+02 +9.000000000000000e+01
+5.953750000000000e+02 +9.000000000000000e+01
+1.211960000000000e+02 +8.950000000000000e+01
+7.523950000000000e+02 +5.650000000000000e+01
+4.142730000000000e+02 +8.900000000000000e+01
+6.530219999999998e+02 +4.450000000000000e+01
+2.104250000000000e+01 +8.850000000000000e+01
+6.755480000000000e+01 +8.850000000000000e+01
+3.307410000000000e+01 +8.850000000000000e+01
+9.342779999999999e+00 +8.800000000000000e+01
+4.764010000000000e+02 +8.800000000000000e+01
+9.301410000000000e+01 +8.750000000000000e+01
+2.114770000000000e+02 +8.750000000000000e+01
+5.972030000000000e+01 +8.700000000000000e+01
+6.071040000000000e+02 +8.700000000000000e+01
+7.139439999999999e-01 +8.700000000000000e+01
+6.257590000000000e+02 +5.700000000000000e+01
+6.767919999999999e+01 +8.650000000000000e+01
+7.727870000000000e+01 +8.650000000000000e+01
+1.647680000000000e+01 +8.650000000000000e+01
+2.814910000000000e+02 +4.750000000000000e+01
+4.897200000000000e+02 +8.650000000000000e+01
+4.640230000000000e+02 +8.600000000000000e+01
+7.370910000000000e+02 +8.600000000000000e+01
+1.030290000000000e+02 +8.550000000000000e+01
+3.225490000000000e+01 +8.500000000000000e+01
+2.769150000000000e+01 +8.500000000000000e+01
+1.125670000000000e+02 +8.500000000000000e+01
+5.498569999999999e-01 +8.500000000000000e+01
+6.183740000000000e+02 +8.500000000000000e+01
+7.694850000000000e+01 +8.450000000000000e+01
+9.630980000000000e+01 +8.450000000000000e+01
+8.166900000000001e+02 +8.450000000000000e+01
+5.088520000000001e-01 +8.400000000000000e+01
+2.885800000000000e+02 +8.400000000000000e+01
+6.950409999999999e+01 +8.350000000000000e+01
+6.891130000000001e+02 +8.350000000000000e+01
+3.178430000000000e+00 +8.300000000000000e+01
+4.359060000000000e+02 +8.300000000000000e+01
+6.221760000000000e+01 +8.250000000000000e+01
+6.220020000000000e+01 +8.250000000000000e+01
+7.429800000000000e+02 +8.150000000000000e+01
+5.921120000000000e+02 +8.250000000000000e+01
+6.491540000000000e+02 +8.200000000000000e+01
+6.233410000000000e+02 +5.350000000000000e+01
+7.922480000000000e+02 +8.200000000000000e+01
+7.271590000000000e+02 +8.200000000000000e+01
+2.715850000000000e+02 +8.200000000000000e+01
+4.535400000000000e+01 +8.100000000000000e+01
+6.338980000000000e+01 +8.100000000000000e+01
+4.415870000000000e+00 +8.100000000000000e+01
+6.072160000000000e+02 +8.100000000000000e+01
+8.290000000000000e+02 +8.100000000000000e+01
+4.053240000000000e+01 +8.050000000000000e+01
+5.761780000000000e+02 +8.050000000000000e+01
+3.959010000000000e+02 +8.050000000000000e+01
+9.107780000000000e+02 +7.000000000000000e+01
+4.318930000000000e+01 +8.000000000000000e+01
+3.895060000000000e+02 +8.000000000000000e+01
+5.451420000000000e+01 +7.950000000000000e+01
+7.357420000000000e+02 +7.950000000000000e+01
+9.841310000000000e+01 +7.900000000000000e+01
+2.129540000000000e+00 +7.900000000000000e+01
+2.800780000000000e+01 +7.850000000000000e+01
+7.248470000000000e+02 +7.850000000000000e+01
+6.555130000000000e+02 +7.850000000000000e+01
+7.113869999999999e+02 +7.850000000000000e+01
+1.581600000000000e+00 +7.800000000000000e+01
+6.215050000000000e+02 +7.800000000000000e+01
+2.199700000000000e+00 +7.750000000000000e+01
+5.904320000000000e+02 +7.750000000000000e+01
+6.814190000000000e+02 +7.750000000000000e+01
+5.976860000000000e+02 +7.700000000000000e+01
+5.171390000000000e+02 +7.650000000000000e+01
+3.068920000000000e+01 +7.600000000000000e+01
+6.623830000000000e+02 +4.850000000000000e+01
+5.113490000000000e+02 +7.600000000000000e+01
+6.161120000000000e+02 +4.100000000000000e+01
+3.245580000000000e+02 +7.600000000000000e+01
+2.870990000000000e+01 +7.550000000000000e+01
+1.069640000000000e+03 +7.550000000000000e+01
+3.509120000000000e-01 +7.550000000000000e+01
+4.840770000000000e+02 +7.550000000000000e+01
+5.508480000000002e+02 +7.550000000000000e+01
+1.202650000000000e-01 +7.500000000000000e+01
+5.785450000000000e+02 +7.500000000000000e+01
+7.262089999999999e+02 +7.500000000000000e+01
+7.252350000000000e+02 +7.500000000000000e+01
+2.675600000000000e-01 +7.450000000000000e+01
+5.046700000000000e+02 +7.450000000000000e+01
+4.470980000000000e+02 +7.450000000000000e+01
+4.828710000000000e+02 +7.450000000000000e+01
+7.365060000000002e+02 +7.450000000000000e+01
+5.023490000000000e+01 +7.400000000000000e+01
+4.244990000000000e+01 +7.400000000000000e+01
+8.363810000000002e+02 +7.400000000000000e+01
+4.202510000000000e+01 +7.350000000000000e+01
+5.853140000000000e+00 +7.350000000000000e+01
+2.087730000000000e+00 +7.350000000000000e+01
+3.780920000000000e+02 +7.300000000000000e+01
+4.769920000000000e+02 +7.250000000000000e+01
+6.632209999999999e-01 +7.200000000000000e+01
+4.479970000000000e+02 +7.200000000000000e+01
+9.332500000000000e+01 +7.200000000000000e+01
+4.852370000000000e+02 +3.800000000000000e+01
+5.520840000000002e+02 +7.100000000000000e+01
+7.855870000000000e+02 +7.000000000000000e+01
+4.653050000000000e+02 +7.000000000000000e+01
+4.463600000000000e+01 +6.950000000000000e+01
+4.142800000000000e+02 +6.950000000000000e+01
+3.847710000000000e+02 +6.950000000000000e+01
+4.713590000000000e+02 +6.950000000000000e+01
+1.261420000000000e+02 +6.900000000000000e+01
+2.511090000000000e+02 +6.900000000000000e+01
+2.870650000000000e+02 +6.900000000000000e+01
+9.467500000000000e+01 +6.850000000000000e+01
+4.352740000000000e+02 +6.850000000000000e+01
+1.134350000000000e+02 +6.850000000000000e+01
+1.942380000000000e+02 +6.850000000000000e+01
+1.699830000000000e+02 +6.800000000000000e+01
+2.855830000000000e+01 +6.750000000000000e+01
+4.890730000000000e+02 +6.750000000000000e+01
+3.788840000000000e+00 +6.700000000000000e+01
+4.423750000000000e+02 +6.700000000000000e+01
+7.041710000000000e+02 +6.650000000000000e+01
+2.321090000000000e+02 +6.650000000000000e+01
+1.857590000000000e+02 +6.650000000000000e+01
+7.956319999999999e+01 +6.600000000000000e+01
+2.445210000000000e+02 +6.600000000000000e+01
+4.095270000000000e+02 +6.600000000000000e+01
+3.791050000000000e+02 +6.550000000000000e+01
+6.793730000000000e+02 +6.500000000000000e+01
+4.657100000000001e+01 +6.500000000000000e+01
+5.060950000000000e+02 +6.500000000000000e+01
+4.561320000000000e+02 +6.500000000000000e+01
+7.354820000000000e+02 +6.450000000000000e+01
+4.919950000000000e+02 +6.450000000000000e+01
+6.946780000000000e+02 +6.350000000000000e+01
+2.649730000000000e+02 +6.350000000000000e+01
+3.690110000000000e+01 +6.300000000000000e+01
+4.488700000000000e+02 +6.300000000000000e+01
+1.142570000000000e+02 +6.300000000000000e+01
+8.649070000000000e+01 +6.250000000000000e+01
+5.903660000000000e+02 +6.250000000000000e+01
+3.776490000000000e+02 +6.250000000000000e+01
+9.031540000000000e+01 +6.150000000000000e+01
+1.267320000000000e+02 +6.150000000000000e+01
+4.835580000000000e+02 +6.150000000000000e+01
+7.411900000000001e+02 +6.100000000000000e+01
+1.260530000000000e+02 +6.050000000000000e+01
+3.998450000000000e+02 +6.050000000000000e+01
+2.363270000000000e+02 +6.050000000000000e+01
+2.598980000000000e+02 +6.050000000000000e+01
+8.706900000000000e+01 +6.000000000000000e+01
+7.060810000000000e+02 +6.000000000000000e+01
+4.899180000000000e+02 +6.000000000000000e+01
+1.936170000000000e+02 +6.000000000000000e+01
+4.025810000000000e+02 +6.000000000000000e+01
+3.592410000000000e+02 +6.000000000000000e+01
+2.085920000000000e+02 +6.000000000000000e+01
+6.189800000000000e+02 +5.950000000000000e+01
+4.575130000000000e+02 +5.950000000000000e+01
+4.443620000000000e+02 +5.950000000000000e+01
+3.116520000000000e+02 +5.950000000000000e+01
+2.488000000000000e+02 +5.900000000000000e+01
+2.569890000000000e+00 +5.900000000000000e+01
+1.239950000000000e+02 +5.900000000000000e+01
+2.303650000000000e+02 +5.850000000000000e+01
+6.438360000000000e+02 +5.750000000000000e+01
+3.927380000000001e+02 +5.750000000000000e+01
+3.594720000000000e+02 +5.650000000000000e+01
+4.252040000000000e+02 +5.650000000000000e+01
+1.235740000000000e+02 +5.650000000000000e+01
+6.725930000000002e+02 +5.550000000000000e+01
+4.030200000000000e+02 +5.550000000000000e+01
+1.842310000000000e+00 +5.550000000000000e+01
+8.385550000000001e+01 +5.500000000000000e+01
+2.529440000000000e+02 +5.500000000000000e+01
+3.315640000000000e+02 +5.500000000000000e+01
+6.029390000000000e+02 +5.500000000000000e+01
+3.784490000000000e+02 +5.450000000000000e+01
+9.559170000000000e+00 +5.400000000000000e+01
+3.540380000000000e+01 +5.400000000000000e+01
+5.983710000000000e+02 +5.350000000000000e+01
+2.526220000000000e+02 +5.300000000000000e+01
+3.883530000000000e+02 +5.300000000000000e+01
+1.966230000000000e+02 +5.250000000000000e+01
+2.183780000000000e+02 +5.250000000000000e+01
+4.105160000000000e+02 +5.200000000000000e+01
+3.364760000000000e+02 +5.200000000000000e+01
+2.434550000000000e+02 +5.200000000000000e+01
+2.875660000000000e+00 +5.150000000000000e+01
+7.526909999999999e+01 +5.150000000000000e+01
+6.220350000000000e+02 +5.150000000000000e+01
+3.531890000000000e+02 +5.150000000000000e+01
+2.940040000000000e+02 +5.150000000000000e+01
+5.531569999999998e+02 +5.100000000000000e+01
+2.164730000000000e+02 +5.050000000000000e+01
+3.321430000000001e+02 +5.050000000000000e+01
+2.024730000000000e+02 +5.050000000000000e+01
+3.251510000000000e+02 +5.050000000000000e+01
+3.756420000000000e+02 +5.000000000000000e+01
+8.662290000000000e+01 +4.950000000000000e+01
+2.293550000000000e+02 +4.950000000000000e+01
+6.339010000000000e+02 +4.900000000000000e+01
+1.188780000000000e+02 +4.900000000000000e+01
+7.595610000000001e+01 +4.900000000000000e+01
+2.075820000000000e+02 +4.850000000000000e+01
+3.880080000000000e+02 +4.850000000000000e+01
+2.904310000000000e-01 +4.850000000000000e+01
+1.175370000000000e+02 +4.850000000000000e+01
+6.526469999999998e+02 +4.800000000000000e+01
+3.860220000000000e+02 +4.750000000000000e+01
+1.658050000000000e+02 +4.750000000000000e+01
+7.708150000000001e+01 +4.750000000000000e+01
+3.056080000000000e+02 +4.750000000000000e+01
+3.945580000000000e+02 +4.650000000000000e+01
+3.843780000000000e+00 +4.600000000000000e+01
+1.870590000000000e+02 +4.600000000000000e+01
+3.202920000000001e+02 +4.550000000000000e+01
+7.443639999999998e+00 +4.500000000000000e+01
+5.500899999999999e+01 +4.500000000000000e+01
+7.151360000000000e+01 +4.500000000000000e+01
+1.342730000000000e+02 +4.500000000000000e+01
+2.930870000000000e+02 +4.400000000000000e+01
+2.252590000000000e+01 +4.350000000000000e+01
+8.598610000000001e+00 +4.350000000000000e+01
+2.388740000000000e+02 +4.350000000000000e+01
+1.888070000000000e+02 +4.350000000000000e+01
+3.520440000000000e+00 +4.300000000000000e+01
+2.114490000000000e+02 +4.300000000000000e+01
+2.549160000000000e+02 +4.300000000000000e+01
+7.984360000000000e+01 +4.250000000000000e+01
+1.883130000000000e+02 +4.250000000000000e+01
+2.528900000000000e+02 +4.250000000000000e+01
+3.424140000000000e+02 +4.200000000000000e+01
+2.255770000000000e+02 +4.150000000000000e+01
+3.036080000000000e+02 +4.100000000000000e+01
+5.086579999999999e-01 +4.050000000000000e+01
+2.613250000000000e+02 +4.050000000000000e+01
+2.636640000000000e+02 +3.950000000000000e+01
+3.810060000000000e+02 +3.900000000000000e+01
+2.203400000000000e+02 +3.900000000000000e+01
+2.610330000000000e+02 +3.850000000000000e+01
+2.574830000000000e+02 +3.750000000000000e+01
+2.460910000000000e+02 +3.700000000000000e+01
+1.877900000000000e+02 +3.700000000000000e+01
+1.641660000000000e+02 +3.650000000000000e+01
+3.275600000000000e+02 +3.650000000000000e+01
+5.981650000000000e+01 +3.600000000000000e+01
+5.368370000000000e+02 +3.600000000000000e+01
+1.385900000000000e+02 +3.600000000000000e+01
+3.564870000000000e+02 +3.600000000000000e+01
+6.408450000000001e+01 +3.550000000000000e+01
+2.101020000000000e+02 +3.550000000000000e+01
+1.850030000000000e+02 +3.550000000000000e+01
+1.687509999999999e+02 +3.500000000000000e+01
+2.315280000000000e+01 +3.450000000000000e+01
+1.497940000000000e+02 +3.450000000000000e+01
+1.285790000000000e+02 +3.400000000000000e+01
+2.394000000000000e+02 +3.400000000000000e+01
+1.196780000000000e+02 +3.350000000000000e+01
+3.611300000000000e+02 +3.350000000000000e+01
+1.645370000000000e+02 +3.350000000000000e+01
+3.190230000000000e+02 +3.350000000000000e+01
+4.455130000000000e+02 +3.300000000000000e+01
+1.594520000000000e+02 +3.300000000000000e+01
+2.174920000000000e+02 +3.300000000000000e+01
+1.176250000000000e+02 +3.250000000000000e+01
+3.512870000000000e+02 +3.250000000000000e+01
+2.269150000000000e+02 +3.200000000000000e+01
+2.326420000000000e+02 +3.200000000000000e+01
+3.407490000000000e+02 +2.650000000000000e+01
+1.680350000000000e+02 +3.150000000000000e+01
+1.390350000000000e+02 +3.100000000000000e+01
+2.221220000000000e+02 +3.100000000000000e+01
+3.217740000000000e+01 +3.100000000000000e+01
+1.908560000000000e+02 +3.050000000000000e+01
+1.192380000000000e+02 +3.050000000000000e+01
+2.340440000000000e+02 +3.050000000000000e+01
+1.092770000000000e+02 +3.000000000000000e+01
+2.651500000000000e+02 +3.000000000000000e+01
+2.762120000000000e+02 +2.950000000000000e+01
+1.573920000000000e+02 +2.850000000000000e+01
+5.743170000000000e+01 +2.750000000000000e+01
+3.947170000000000e+02 +2.750000000000000e+01
+1.509550000000000e+02 +2.750000000000000e+01
+5.618110000000000e+01 +2.750000000000000e+01
+2.713780000000000e+02 +2.700000000000000e+01
+1.577390000000000e+00 +2.650000000000000e+01
+9.795880000000000e+01 +2.650000000000000e+01
+2.094100000000000e+02 +2.600000000000000e+01
+3.299450000000000e+02 +2.600000000000000e+01
+3.391310000000000e+00 +2.550000000000000e+01
+1.684870000000000e+02 +2.550000000000000e+01
+1.152970000000000e+02 +2.500000000000000e+01
+1.346240000000000e+02 +2.500000000000000e+01
+5.287370000000000e+01 +2.400000000000000e+01
+3.207330000000000e+02 +2.400000000000000e+01
+1.864830000000000e+02 +2.400000000000000e+01
+1.363730000000000e+01 +2.350000000000000e+01
+1.502730000000000e+02 +2.300000000000000e+01
+2.465240000000000e+02 +2.300000000000000e+01
+3.092360000000000e+02 +2.250000000000000e+01
+9.199850000000001e+01 +2.250000000000000e+01
+1.871050000000000e+02 +2.250000000000000e+01
+1.001080000000000e+02 +2.250000000000000e+01
+8.161730000000000e+01 +2.150000000000000e+01
+5.044660000000000e+01 +2.100000000000000e+01
+1.180270000000000e+02 +2.100000000000000e+01
+2.296230000000000e+02 +2.100000000000000e+01
+2.303830000000000e+02 +2.100000000000000e+01
+1.642160000000000e+01 +2.050000000000000e+01
+2.444720000000000e+02 +2.050000000000000e+01
+8.939010000000000e+01 +2.050000000000000e+01
+6.739600000000000e+01 +2.000000000000000e+01
+7.240360000000000e+01 +2.000000000000000e+01
+1.775150000000000e+02 +2.000000000000000e+01
+1.917630000000000e+02 +1.950000000000000e+01
+2.892760000000000e+02 +1.950000000000000e+01
+2.546500000000000e+01 +1.850000000000000e+01
+1.607770000000000e+02 +1.850000000000000e+01
+6.955070000000001e+01 +1.800000000000000e+01
+1.339170000000000e+02 +1.800000000000000e+01
+1.273920000000000e+02 +1.750000000000000e+01
+4.308620000000000e+01 +1.700000000000000e+01
+2.339410000000000e+02 +1.700000000000000e+01
+4.558400000000000e+01 +1.650000000000000e+01
+2.080500000000000e+02 +1.650000000000000e+01
+2.385610000000000e+02 +1.650000000000000e+01
+7.449110000000001e+00 +1.600000000000000e+01
+2.202950000000000e+02 +1.600000000000000e+01
+1.214050000000000e+02 +1.600000000000000e+01
+1.461290000000000e+02 +1.550000000000000e+01
+1.069830000000000e+02 +1.550000000000000e+01
+2.274160000000000e+02 +1.550000000000000e+01
+2.163830000000000e+02 +1.500000000000000e+01
+1.856420000000000e+02 +1.450000000000000e+01
+1.519180000000000e+02 +1.450000000000000e+01
+9.307570000000000e+01 +1.400000000000000e+01
+1.540080000000000e+02 +1.400000000000000e+01
+1.637480000000000e+02 +1.350000000000000e+01
+1.973400000000000e+02 +1.350000000000000e+01
+3.991740000000000e+01 +1.300000000000000e+01
+5.887060000000000e+01 +1.300000000000000e+01
+9.794130000000000e+01 +1.300000000000000e+01
+1.725550000000000e+02 +1.300000000000000e+01
+3.977610000000000e+01 +1.150000000000000e+01
+1.037170000000000e+02 +1.150000000000000e+01
+7.866180000000000e+01 +1.150000000000000e+01
+1.309940000000000e+02 +1.100000000000000e+01
+1.239150000000000e+02 +1.050000000000000e+01
+4.609090000000000e+01 +1.050000000000000e+01
+1.518670000000000e+02 +1.050000000000000e+01
+8.580860000000000e+01 +1.050000000000000e+01
+2.048710000000000e+01 +1.000000000000000e+01
+3.822630000000000e+01 +1.000000000000000e+01
+1.225250000000000e+02 +1.000000000000000e+01
+1.202580000000000e+02 +1.000000000000000e+01
+1.410660000000000e+02 +9.500000000000000e+00
+4.534470000000000e+01 +9.500000000000000e+00
+7.416800000000001e+01 +8.500000000000000e+00
+8.159470000000000e+01 +8.500000000000000e+00
+9.488750000000000e+01 +8.500000000000000e+00
+5.083770000000000e+01 +8.000000000000000e+00
+3.611700000000000e+01 +8.000000000000000e+00
+7.747780000000000e+01 +8.000000000000000e+00
+2.501030000000000e+01 +7.500000000000000e+00
+6.103360000000000e+01 +7.500000000000000e+00
+6.751620000000000e+01 +7.500000000000000e+00
+4.574360000000000e+01 +7.500000000000000e+00
+9.407859999999999e+01 +7.500000000000000e+00
+2.150570000000000e+01 +7.000000000000000e+00
+6.823620000000000e+01 +7.000000000000000e+00
+8.732890000000000e+01 +6.500000000000000e+00
+1.103930000000000e+01 +6.500000000000000e+00
+1.701720000000000e+01 +6.000000000000000e+00
+1.433880000000000e+01 +6.000000000000000e+00
+5.468140000000000e+01 +6.000000000000000e+00
+3.434620000000000e+00 +5.500000000000000e+00
+1.038430000000000e+01 +5.500000000000000e+00
+1.371090000000000e+01 +5.500000000000000e+00
+1.391870000000000e+01 +5.500000000000000e+00
+6.948660000000000e+01 +5.500000000000000e+00
+6.481820000000000e+01 +5.500000000000000e+00
+3.285420000000000e+01 +5.000000000000000e+00
+8.749599999999999e+00 +5.000000000000000e+00
+8.277690000000000e+00 +4.500000000000000e+00
+4.507150000000000e+01 +4.500000000000000e+00
+3.737760000000000e+00 +4.500000000000000e+00
+6.750910000000000e+00 +4.000000000000000e+00
+4.450420000000000e+01 +4.000000000000000e+00
+4.231930000000000e+01 +4.000000000000000e+00
+3.325210000000000e+01 +4.000000000000000e+00
+1.549400000000000e+01 +3.500000000000000e+00
+2.710910000000000e+00 +2.500000000000000e+00
+1.631730000000000e+01 +2.500000000000000e+00
+2.287520000000000e+01 +2.500000000000000e+00
+1.843750000000000e+01 +2.500000000000000e+00
+1.146520000000000e+00 +2.000000000000000e+00
+5.021090000000000e-01 +2.000000000000000e+00
+1.630000000000000e+01 +2.000000000000000e+00
+1.314640000000000e-02 +1.500000000000000e+00
+4.957180000000000e+00 +1.000000000000000e+00
+8.277580000000000e-01 +5.000000000000000e-01
+3.969170000000000e-02 +5.000000000000000e-01
+1.551810000000000e+00 +5.000000000000000e-01
+1.930340000000000e+00 +5.000000000000000e-01
+3.137500000000000e+00 +5.000000000000000e-01
+0.000000000000000e+00 +0.000000000000000e+00
+0.000000000000000e+00 +0.000000000000000e+00
+0.000000000000000e+00 +0.000000000000000e+00
+0.000000000000000e+00 +0.000000000000000e+00
+0.000000000000000e+00 +0.000000000000000e+00
+0.000000000000000e+00 +0.000000000000000e+00
+0.000000000000000e+00 +0.000000000000000e+00
+0.000000000000000e+00 +0.000000000000000e+00
+0.000000000000000e+00 +0.000000000000000e+00
+0.000000000000000e+00 +0.000000000000000e+00
+0.000000000000000e+00 +0.000000000000000e+00
+0.000000000000000e+00 +0.000000000000000e+00
+0.000000000000000e+00 +0.000000000000000e+00
};
\end{axis}

\end{tikzpicture}
    \caption[Distancia vs. tiempo total]{Gráfico de dispersión, distancia total vs. tiempo total en la simulación, para escenarios con factor de demanda 100\%, 15 minutos de tiempo simulado.}
    \label{fig:distvstime}
\end{figure}

\begin{figure}[tpb]
    \centering
    % This file was created by matplotlib2tikz v0.6.10.
\begin{tikzpicture}

\definecolor{color0}{rgb}{0.2,0.8,0.133333333333333}

\begin{groupplot}[group style={group size=1 by 2}]
\nextgroupplot[
ylabel={CO2 Total [g]},
xmin=-141.460057014178, xmax=2838.93005701418,
ymin=-73.1404919642858, ymax=1350.70625925454,
width=\figurewidth,
height=\figureheight,
xtick={-500,0,500,1000,1500,2000,2500,3000},
xticklabels={},
tick align=outside,
tick pos=left,
xmajorgrids,
x grid style={lightgray!84.183006535947712!black},
ymajorgrids,
y grid style={lightgray!84.183006535947712!black},
axis line style={white},
legend entries={{PER 0.0}},
legend cell align={left},
legend style={draw=white!80.0!black, fill=white!89.803921568627459!black}
]
\addplot [only marks, draw=red, fill=red, opacity=0.75, colormap/viridis]
table{%
x                      y
+2.789250000000000e+02 +7.572980000000000e+01
+2.751700000000000e+02 +8.089990000000000e+01
+2.827610000000000e+02 +8.403310000000000e+01
+3.244580000000000e+02 +9.122929999999999e+01
+2.927260000000000e+02 +1.048630000000000e+02
+3.345040000000000e+02 +8.788060000000000e+01
+2.774470000000000e+02 +7.499209999999999e+01
+2.799990000000000e+02 +7.314510000000000e+01
+4.920510000000000e+02 +1.485950000000000e+02
+7.761550000000000e+02 +1.835240000000000e+02
+7.702950000000000e+02 +1.744430000000000e+02
+7.883260000000000e+02 +1.809040000000000e+02
+6.929910000000001e+02 +1.921190000000000e+02
+6.937250000000000e+02 +1.977470000000000e+02
+7.652639999999999e+02 +1.727640000000000e+02
+7.750030000000000e+02 +1.775380000000000e+02
+6.769030000000000e+02 +1.501700000000000e+02
+3.374630000000000e+02 +1.004820000000000e+02
+4.942410000000000e+02 +1.139950000000000e+02
+7.652170000000000e+02 +1.867300000000000e+02
+7.680560000000000e+02 +1.732610000000000e+02
+7.729119999999998e+02 +1.782310000000000e+02
+4.536680000000000e+02 +1.205450000000000e+02
+7.615050000000000e+02 +1.736690000000000e+02
+7.695260000000002e+02 +1.724830000000000e+02
+7.503439999999998e+02 +1.751140000000000e+02
+7.516039999999998e+02 +1.547780000000000e+02
+7.820750000000000e+02 +1.788590000000000e+02
+4.916920000000000e+02 +1.883340000000000e+02
+7.782580000000000e+02 +1.812540000000000e+02
+8.424169999999998e+02 +1.823740000000000e+02
+8.793720000000000e+02 +1.991970000000000e+02
+1.083850000000000e+02 +2.978880000000000e+01
+9.149010000000000e+02 +2.651070000000000e+02
+4.859460000000000e+02 +1.624270000000000e+02
+8.701080000000002e+02 +1.876110000000000e+02
+8.813600000000000e+02 +2.085740000000000e+02
+4.922770000000000e+02 +1.689610000000000e+02
+7.707869999999998e+02 +1.898820000000000e+02
+7.743650000000000e+02 +1.637170000000000e+02
+4.937380000000001e+02 +1.466710000000000e+02
+8.909920000000000e+02 +1.869340000000000e+02
+4.910120000000000e+02 +1.392520000000000e+02
+8.859299999999999e+02 +2.083930000000000e+02
+8.765510000000000e+02 +1.877650000000000e+02
+3.368600000000000e+02 +9.843290000000000e+01
+8.571039999999998e+02 +2.432620000000000e+02
+7.774040000000000e+02 +2.229680000000000e+02
+6.599019999999998e+02 +2.089550000000000e+02
+6.250269999999998e+02 +1.889840000000000e+02
+4.870110000000000e+02 +1.248150000000000e+02
+8.392420000000000e+02 +1.831940000000000e+02
+8.554490000000000e+02 +2.237780000000000e+02
+6.028950000000000e+02 +1.838360000000000e+02
+6.405459999999998e+02 +1.914770000000000e+02
+6.657650000000000e+02 +1.975780000000000e+02
+7.727520000000000e+02 +1.945290000000000e+02
+9.011180000000001e+02 +2.097420000000000e+02
+8.678980000000000e+02 +2.545750000000000e+02
+6.511390000000000e+02 +1.826740000000000e+02
+8.638570000000000e+02 +2.111490000000000e+02
+8.826810000000000e+02 +2.116740000000000e+02
+6.699600000000000e+02 +1.771780000000000e+02
+8.406660000000001e+02 +2.053060000000000e+02
+4.980300000000000e+02 +1.318860000000000e+02
+6.808989999999999e+02 +2.262920000000000e+02
+8.693700000000000e+02 +2.090600000000000e+02
+6.372060000000000e+02 +1.610880000000000e+02
+6.556920000000000e+02 +1.848440000000000e+02
+9.252900000000000e+02 +2.871800000000000e+02
+6.403869999999999e+02 +1.617240000000000e+02
+6.367410000000000e+02 +1.714570000000000e+02
+7.528860000000002e+02 +2.110180000000000e+02
+6.645080000000000e+02 +1.605900000000000e+02
+1.078750000000000e+02 +3.681970000000000e+01
+6.794160000000001e+02 +2.251070000000000e+02
+9.343110000000000e+02 +2.313120000000000e+02
+6.427090000000002e+02 +1.723590000000000e+02
+6.051230000000000e+02 +1.561720000000000e+02
+4.959750000000000e+02 +1.445010000000000e+02
+6.612380000000001e+02 +1.534970000000000e+02
+9.510910000000000e+02 +2.349960000000000e+02
+8.795350000000000e+02 +2.084550000000000e+02
+6.311300000000000e+02 +1.586450000000000e+02
+6.514410000000000e+02 +1.960750000000000e+02
+7.530640000000000e+02 +2.288480000000000e+02
+1.082450000000000e+03 +2.874450000000000e+02
+9.943860000000000e+02 +2.388950000000000e+02
+9.092920000000000e+02 +2.550590000000000e+02
+6.336230000000000e+02 +1.506800000000000e+02
+9.018980000000000e+02 +2.526520000000000e+02
+8.776870000000000e+02 +1.967550000000000e+02
+6.679220000000000e+02 +2.114380000000000e+02
+6.742680000000000e+02 +2.125790000000000e+02
+1.005390000000000e+03 +3.003390000000000e+02
+1.004130000000000e+03 +2.096680000000000e+02
+7.538180000000000e+02 +2.323300000000000e+02
+7.544160000000001e+02 +1.835790000000000e+02
+1.153810000000000e+02 +5.235870000000000e+01
+6.746410000000002e+02 +1.974830000000000e+02
+1.017310000000000e+03 +2.131460000000000e+02
+1.061130000000000e+03 +2.919460000000000e+02
+6.876990000000000e+02 +1.918580000000000e+02
+1.059400000000000e+03 +3.166650000000000e+02
+9.873360000000000e+02 +2.023240000000000e+02
+6.854330000000000e+02 +2.119070000000000e+02
+1.076600000000000e+03 +3.308470000000000e+02
+1.061180000000000e+03 +2.532270000000000e+02
+1.082910000000000e+03 +2.985600000000000e+02
+1.083180000000000e+03 +2.880190000000000e+02
+6.779220000000000e+02 +1.820910000000000e+02
+1.035650000000000e+03 +2.612740000000000e+02
+1.045950000000000e+03 +2.511100000000000e+02
+1.113520000000000e+03 +3.501210000000000e+02
+1.120120000000000e+03 +3.267010000000000e+02
+9.784960000000000e+02 +3.172330000000000e+02
+1.033020000000000e+03 +2.547700000000000e+02
+2.722100000000000e+02 +1.162870000000000e+02
+9.970080000000000e+02 +3.153840000000000e+02
+1.168360000000000e+02 +3.429110000000000e+01
+9.931680000000000e+02 +2.831490000000000e+02
+9.726490000000000e+02 +2.722900000000000e+02
+1.000600000000000e+03 +3.628780000000000e+02
+1.007110000000000e+03 +3.339180000000000e+02
+1.003110000000000e+03 +2.890500000000000e+02
+2.741560000000000e+02 +1.208880000000000e+02
+1.008090000000000e+03 +3.158560000000000e+02
+1.029620000000000e+03 +2.795290000000000e+02
+1.131450000000000e+03 +2.917190000000000e+02
+9.890240000000000e+02 +2.782130000000000e+02
+9.837650000000000e+02 +3.013760000000000e+02
+1.055960000000000e+03 +2.680840000000000e+02
+1.195210000000000e+03 +2.949110000000000e+02
+2.818490000000000e+02 +1.670380000000000e+02
+1.126080000000000e+03 +3.064590000000000e+02
+1.281020000000000e+03 +4.252830000000000e+02
+1.361980000000000e+03 +3.850200000000000e+02
+2.819530000000000e+02 +1.510340000000000e+02
+2.771970000000000e+02 +1.648880000000000e+02
+9.760050000000000e+02 +2.768950000000000e+02
+9.976070000000000e+02 +2.613220000000000e+02
+6.616760000000000e+02 +1.952980000000000e+02
+6.867030000000000e+02 +2.811980000000000e+02
+9.034400000000001e+02 +3.262220000000000e+02
+5.886840000000000e+02 +2.065020000000000e+02
+2.761120000000000e+02 +1.571200000000000e+02
+9.929140000000000e+02 +2.669550000000000e+02
+9.569390000000000e+02 +2.649640000000000e+02
+4.580520000000000e+02 +1.577200000000000e+02
+1.333760000000000e+03 +3.296260000000000e+02
+2.769890000000000e+02 +1.504650000000000e+02
+2.810300000000000e+02 +1.422040000000000e+02
+9.172310000000000e+02 +3.281200000000000e+02
+6.220290000000000e+02 +2.526530000000000e+02
+6.205850000000000e+02 +2.518610000000000e+02
+8.542030000000000e+02 +3.032920000000001e+02
+6.209950000000000e+02 +2.531670000000000e+02
+2.842980000000000e+02 +1.498850000000000e+02
+9.735540000000000e+02 +2.559680000000000e+02
+6.226500000000000e+02 +2.177620000000000e+02
+6.613760000000002e+02 +2.047840000000000e+02
+9.097670000000001e+02 +3.335100000000000e+02
+1.056970000000000e+03 +2.911860000000000e+02
+6.668960000000002e+02 +2.562690000000000e+02
+1.037340000000000e+03 +3.221860000000000e+02
+9.808030000000000e+02 +2.613660000000000e+02
+8.797869999999998e+02 +2.578890000000000e+02
+8.846790000000000e+02 +2.998610000000000e+02
+1.063840000000000e+03 +3.526730000000000e+02
+2.834900000000000e+02 +1.255700000000000e+02
+9.209600000000000e+02 +3.207430000000000e+02
+6.252730000000000e+02 +1.670770000000000e+02
+6.245790000000002e+02 +1.699110000000000e+02
+6.606139999999998e+02 +1.704090000000000e+02
+1.077230000000000e+03 +2.997320000000000e+02
+9.745210000000000e+02 +2.419170000000000e+02
+9.109400000000001e+02 +2.861890000000000e+02
+6.214370000000000e+02 +1.575750000000000e+02
+1.018190000000000e+03 +2.944140000000000e+02
+1.056790000000000e+03 +2.605130000000000e+02
+6.224390000000000e+02 +1.525680000000000e+02
+8.732840000000000e+02 +2.950000000000000e+02
+2.700270000000000e+02 +1.136900000000000e+02
+6.062370000000000e+02 +1.600150000000000e+02
+1.112550000000000e+03 +3.015810000000000e+02
+1.230000000000000e+03 +4.566070000000000e+02
+9.072340000000000e+02 +2.625790000000000e+02
+1.107910000000000e+03 +3.161820000000000e+02
+2.991810000000000e+02 +1.283670000000000e+02
+9.125350000000000e+02 +2.632280000000000e+02
+1.115000000000000e+03 +3.033950000000000e+02
+1.084100000000000e+03 +3.131540000000000e+02
+6.327730000000000e+02 +1.487140000000000e+02
+1.063320000000000e+03 +3.000790000000000e+02
+2.940270000000000e+02 +1.441260000000000e+02
+1.309740000000000e+03 +3.572710000000000e+02
+9.093339999999999e+02 +2.476640000000000e+02
+1.022410000000000e+03 +2.889270000000000e+02
+7.862030000000000e+02 +2.389960000000000e+02
+1.289490000000000e+03 +3.595590000000000e+02
+9.998480000000000e+02 +3.342380000000001e+02
+1.405180000000000e+03 +4.418950000000000e+02
+7.652769999999998e+02 +2.763730000000000e+02
+6.256669999999998e+02 +1.480200000000000e+02
+9.064010000000000e+02 +2.904500000000000e+02
+8.499360000000000e+02 +2.617600000000000e+02
+8.389560000000000e+02 +2.520790000000000e+02
+9.213240000000000e+02 +3.021950000000000e+02
+7.681710000000000e+02 +2.665830000000000e+02
+7.684710000000000e+02 +2.474170000000000e+02
+8.658339999999999e+02 +2.628850000000000e+02
+8.795910000000000e+02 +2.592100000000000e+02
+9.142070000000000e+02 +2.895840000000000e+02
+1.321400000000000e+03 +3.372820000000000e+02
+7.450100000000000e+02 +2.433160000000000e+02
+4.463610000000000e+02 +1.664790000000000e+02
+7.751060000000001e+02 +2.508850000000000e+02
+1.306300000000000e+03 +3.431510000000000e+02
+7.745960000000000e+02 +2.579570000000000e+02
+7.673850000000000e+02 +2.449350000000000e+02
+8.772430000000001e+02 +2.367200000000000e+02
+8.624530000000000e+02 +2.525350000000000e+02
+3.224100000000000e+02 +1.557540000000000e+02
+8.868789999999998e+02 +2.710780000000000e+02
+7.563610000000001e+02 +2.466370000000000e+02
+7.664480000000000e+02 +2.249090000000000e+02
+1.411260000000000e+03 +4.172600000000000e+02
+7.724630000000002e+02 +2.644860000000000e+02
+1.316050000000000e+03 +4.421870000000000e+02
+7.644370000000000e+02 +2.456600000000000e+02
+8.930740000000000e+02 +2.442000000000000e+02
+6.927239999999998e+02 +2.892900000000000e+02
+3.251410000000000e+02 +1.600060000000000e+02
+6.881130000000001e+02 +2.657680000000000e+02
+8.044800000000000e+02 +2.333180000000000e+02
+7.773270000000000e+02 +2.680380000000000e+02
+7.713680000000001e+02 +2.359850000000000e+02
+8.394560000000000e+02 +2.072260000000000e+02
+7.708620000000000e+02 +2.398200000000000e+02
+3.332760000000000e+02 +1.236700000000000e+02
+6.939380000000000e+02 +2.519460000000000e+02
+7.563689999999998e+02 +2.563570000000000e+02
+7.790549999999999e+02 +2.532000000000000e+02
+8.559760000000001e+02 +2.248750000000000e+02
+8.614190000000000e+02 +2.464210000000000e+02
+7.652080000000002e+02 +2.435000000000000e+02
+9.129770000000000e+02 +2.895810000000000e+02
+9.006480000000000e+02 +2.256860000000000e+02
+8.081730000000000e+02 +2.299900000000000e+02
+7.810250000000000e+02 +2.368160000000000e+02
+6.617710000000002e+02 +2.321480000000000e+02
+7.691210000000002e+02 +2.346710000000000e+02
+1.296930000000000e+03 +3.941490000000000e+02
+4.800760000000000e+02 +1.761690000000000e+02
+7.874839999999998e+02 +2.489550000000000e+02
+7.845210000000002e+02 +2.401360000000000e+02
+8.853380000000002e+02 +2.188330000000000e+02
+8.622980000000000e+02 +2.225480000000000e+02
+6.925230000000000e+02 +2.366300000000000e+02
+7.458620000000000e+02 +2.558730000000000e+02
+7.669240000000000e+02 +2.305650000000000e+02
+8.965960000000000e+02 +2.310100000000000e+02
+4.940560000000000e+02 +1.445830000000000e+02
+6.918860000000002e+02 +2.661380000000000e+02
+9.034990000000000e+02 +2.802500000000000e+02
+6.616050000000000e+02 +2.136490000000000e+02
+7.519169999999998e+02 +2.366230000000000e+02
+7.669720000000000e+02 +2.273440000000000e+02
+4.864470000000000e+02 +1.796090000000000e+02
+1.300960000000000e+03 +3.674950000000000e+02
+7.717700000000000e+02 +2.114270000000000e+02
+7.833930000000000e+02 +2.463940000000000e+02
+1.294660000000000e+03 +3.505950000000000e+02
+7.753550000000000e+02 +2.346680000000000e+02
+6.971100000000000e+02 +2.254450000000000e+02
+9.478620000000000e+02 +2.631870000000000e+02
+4.756930000000000e+02 +1.365540000000000e+02
+7.477020000000000e+02 +2.343330000000000e+02
+7.798380000000002e+02 +2.132510000000000e+02
+6.948639999999998e+02 +1.801070000000000e+02
+1.091750000000000e+03 +3.371630000000000e+02
+9.005039999999998e+02 +2.208150000000000e+02
+9.850130000000000e+02 +3.460270000000000e+02
+7.895139999999999e+02 +2.187630000000000e+02
+7.835410000000001e+02 +2.129500000000000e+02
+7.799080000000000e+02 +2.275530000000000e+02
+1.420550000000000e+03 +4.068480000000000e+02
+6.949750000000000e+02 +2.279090000000000e+02
+4.824170000000000e+02 +1.417910000000000e+02
+9.754500000000000e+02 +2.831890000000000e+02
+8.630560000000000e+02 +2.654820000000000e+02
+6.977500000000000e+02 +2.324040000000000e+02
+1.094000000000000e+03 +3.136570000000000e+02
+1.315300000000000e+03 +3.765070000000000e+02
+9.775520000000000e+02 +3.006060000000000e+02
+6.752310000000001e+02 +2.658160000000000e+02
+1.316370000000000e+03 +4.230170000000000e+02
+7.852170000000000e+02 +2.241760000000000e+02
+9.388960000000000e+02 +3.257480000000000e+02
+7.564780000000002e+02 +2.207390000000000e+02
+6.896230000000000e+02 +2.775350000000000e+02
+7.690700000000001e+02 +1.959710000000000e+02
+9.066620000000000e+02 +2.257000000000000e+02
+3.338320000000000e+02 +8.454989999999999e+01
+1.297830000000000e+03 +4.066770000000000e+02
+4.852120000000000e+02 +1.460370000000000e+02
+6.801230000000000e+02 +2.393190000000000e+02
+1.128730000000000e+03 +3.662910000000000e+02
+7.831380000000000e+02 +2.105790000000000e+02
+1.295870000000000e+03 +3.967320000000000e+02
+7.885830000000002e+02 +2.181250000000000e+02
+6.939789999999998e+02 +1.930990000000000e+02
+7.531920000000000e+02 +2.399810000000000e+02
+8.991519999999998e+02 +2.183740000000000e+02
+7.651389999999999e+02 +2.074480000000000e+02
+8.432230000000002e+02 +2.292850000000000e+02
+9.775950000000000e+02 +3.126520000000000e+02
+9.783480000000000e+02 +2.944840000000001e+02
+1.125690000000000e+03 +3.589450000000000e+02
+7.657520000000000e+02 +2.053770000000000e+02
+5.756210000000000e+02 +1.683070000000000e+02
+1.294110000000000e+03 +3.645640000000000e+02
+6.909390000000000e+02 +1.907070000000000e+02
+7.753450000000000e+02 +2.027390000000000e+02
+7.666170000000000e+02 +2.853060000000000e+02
+9.897880000000000e+02 +3.166280000000000e+02
+7.487020000000000e+02 +2.094220000000000e+02
+7.534100000000000e+02 +2.195350000000000e+02
+8.732050000000000e+02 +2.607170000000000e+02
+7.816569999999998e+02 +2.001030000000000e+02
+7.552239999999998e+02 +2.545900000000000e+02
+7.837439999999998e+02 +2.072530000000000e+02
+7.795350000000000e+02 +2.165200000000000e+02
+8.685360000000002e+02 +2.895490000000001e+02
+6.941239999999998e+02 +2.153460000000000e+02
+9.080270000000000e+02 +2.085090000000000e+02
+9.458560000000000e+02 +2.297620000000000e+02
+7.722239999999998e+02 +2.077150000000000e+02
+7.661880000000000e+02 +1.962030000000000e+02
+7.677070000000000e+02 +1.928240000000000e+02
+7.816020000000000e+02 +2.081390000000000e+02
+4.907540000000000e+02 +1.226750000000000e+02
+1.140290000000000e+03 +4.001400000000000e+02
+6.944530000000000e+02 +2.202100000000000e+02
+7.827160000000000e+02 +1.989370000000000e+02
+6.811880000000000e+02 +2.002220000000000e+02
+3.349040000000000e+02 +8.628730000000000e+01
+4.898910000000000e+02 +2.308010000000000e+02
+9.351220000000000e+02 +2.462870000000000e+02
+7.485770000000000e+02 +2.804590000000000e+02
+7.472550000000000e+02 +1.987240000000000e+02
+7.795490000000000e+02 +2.013700000000000e+02
+7.060740000000000e+02 +2.087510000000000e+02
+1.149660000000000e+03 +3.624940000000000e+02
+6.873520000000000e+02 +3.174160000000000e+02
+6.595980000000002e+02 +1.855740000000000e+02
+7.471100000000000e+02 +1.828900000000000e+02
+4.822420000000000e+02 +2.244850000000000e+02
+7.703670000000000e+02 +1.729790000000000e+02
+7.756389999999999e+02 +1.884490000000000e+02
+7.798680000000001e+02 +1.824090000000000e+02
+6.798819999999999e+02 +3.147050000000000e+02
+4.912730000000000e+02 +2.074540000000000e+02
+7.508360000000000e+02 +1.777570000000000e+02
+7.780069999999999e+02 +1.863060000000000e+02
+8.938839999999999e+02 +3.051470000000000e+02
+1.312460000000000e+03 +3.824400000000000e+02
+7.683090000000000e+02 +1.762530000000000e+02
+7.447130000000002e+02 +1.877440000000000e+02
+4.895660000000000e+02 +2.177890000000000e+02
+7.764430000000000e+02 +2.017340000000000e+02
+7.762719999999998e+02 +1.844220000000000e+02
+7.590330000000000e+02 +2.980550000000000e+02
+6.850150000000000e+02 +3.245530000000000e+02
+1.121130000000000e+02 +2.788220000000000e+01
+4.859700000000000e+02 +1.936450000000000e+02
+6.930480000000000e+02 +3.095670000000000e+02
+7.541100000000000e+02 +2.946020000000001e+02
+8.807900000000000e+02 +2.211210000000000e+02
+4.896330000000000e+02 +1.839460000000000e+02
+6.823560000000001e+02 +3.243320000000000e+02
+7.845030000000000e+02 +1.822170000000000e+02
+7.706220000000000e+02 +1.751810000000000e+02
+8.994750000000000e+02 +2.128370000000000e+02
+4.889190000000000e+02 +1.648890000000000e+02
+7.595950000000000e+02 +2.785340000000000e+02
+6.313009999999998e+02 +2.077530000000000e+02
+9.701990000000000e+02 +2.939640000000000e+02
+1.106650000000000e+03 +3.336650000000000e+02
+6.806880000000000e+02 +3.082950000000000e+02
+7.629580000000002e+02 +1.890990000000000e+02
+7.804639999999998e+02 +2.434500000000000e+02
+7.545500000000000e+02 +2.197750000000000e+02
+6.429530000000000e+02 +3.269170000000001e+02
+4.864290000000000e+02 +1.736980000000000e+02
+7.728770000000000e+02 +1.683120000000000e+02
+7.553610000000001e+02 +2.073720000000000e+02
+8.582970000000000e+02 +2.062990000000000e+02
+4.880160000000000e+02 +1.610560000000000e+02
+6.789600000000000e+02 +2.856470000000000e+02
+4.915810000000000e+02 +1.582900000000000e+02
+1.393260000000000e+03 +4.369710000000000e+02
+1.349740000000000e+03 +4.375370000000000e+02
+1.096110000000000e+03 +3.824490000000000e+02
+6.744980000000000e+02 +2.596740000000000e+02
+1.561150000000000e+03 +4.831950000000000e+02
+1.718540000000000e+03 +5.227080000000002e+02
+4.856900000000000e+02 +1.379270000000000e+02
+1.289110000000000e+03 +4.426280000000000e+02
+6.802900000000000e+02 +2.535990000000000e+02
+1.547940000000000e+03 +4.713860000000000e+02
+1.322320000000000e+03 +4.440510000000000e+02
+1.100150000000000e+03 +3.942730000000000e+02
+4.919410000000000e+02 +1.652770000000000e+02
+1.555860000000000e+03 +5.114280000000001e+02
+1.552270000000000e+03 +4.445150000000000e+02
+7.506139999999998e+02 +2.398730000000000e+02
+6.856400000000000e+02 +2.644710000000000e+02
+6.927280000000002e+02 +2.754630000000000e+02
+6.739550000000000e+02 +2.715060000000000e+02
+9.893690000000000e+02 +3.376760000000000e+02
+8.915089999999999e+02 +3.231930000000000e+02
+8.934250000000000e+02 +2.916260000000000e+02
+1.569290000000000e+03 +4.350260000000000e+02
+7.611139999999998e+02 +1.940600000000000e+02
+6.763070000000000e+02 +2.557310000000000e+02
+6.659150000000000e+02 +2.628750000000000e+02
+6.353040000000000e+02 +2.735650000000000e+02
+5.027070000000000e+02 +1.368500000000000e+02
+6.793869999999999e+02 +2.203490000000000e+02
+9.884400000000001e+02 +3.330650000000000e+02
+1.574100000000000e+03 +4.853120000000000e+02
+6.727919999999998e+02 +2.348480000000000e+02
+4.952050000000000e+02 +1.362720000000000e+02
+9.300359999999999e+02 +3.301730000000000e+02
+6.550770000000000e+02 +2.493120000000000e+02
+6.608510000000001e+02 +2.753010000000000e+02
+6.333490000000000e+02 +2.789860000000000e+02
+6.976860000000000e+02 +2.332880000000000e+02
+1.150560000000000e+03 +3.896630000000000e+02
+6.190930000000002e+02 +2.519550000000000e+02
+9.334200000000000e+02 +2.779250000000000e+02
+6.528960000000000e+02 +3.053890000000000e+02
+6.366450000000000e+02 +2.758170000000000e+02
+6.338330000000002e+02 +2.734340000000000e+02
+6.671830000000000e+02 +2.507540000000000e+02
+1.859400000000000e+03 +5.042290000000000e+02
+1.036370000000000e+03 +3.403570000000000e+02
+6.649370000000000e+02 +3.179490000000000e+02
+1.058910000000000e+03 +3.474080000000000e+02
+1.108120000000000e+03 +4.116160000000000e+02
+1.148060000000000e+03 +3.172950000000000e+02
+6.473910000000000e+02 +2.750300000000000e+02
+6.571550000000000e+02 +2.717670000000000e+02
+6.482530000000000e+02 +2.707040000000000e+02
+6.193060000000000e+02 +2.243810000000000e+02
+6.844290000000000e+02 +2.946920000000000e+02
+6.203460000000000e+02 +2.565970000000000e+02
+1.084310000000000e+02 +3.977650000000000e+01
+6.173540000000000e+02 +2.663350000000000e+02
+6.521060000000000e+02 +2.248490000000000e+02
+1.440040000000000e+03 +4.592660000000000e+02
+8.974910000000001e+02 +2.686740000000000e+02
+6.396130000000001e+02 +2.575850000000000e+02
+6.502170000000000e+02 +2.458930000000000e+02
+6.743570000000000e+02 +2.159360000000000e+02
+6.417900000000000e+02 +2.568330000000000e+02
+1.475220000000000e+03 +4.321400000000000e+02
+6.868360000000000e+02 +2.901470000000000e+02
+8.786260000000002e+02 +2.696250000000000e+02
+6.542210000000000e+02 +2.516960000000000e+02
+6.624560000000000e+02 +2.644120000000000e+02
+1.038500000000000e+03 +3.355850000000000e+02
+6.534620000000000e+02 +2.096650000000000e+02
+8.966810000000000e+02 +2.477210000000000e+02
+1.523430000000000e+03 +5.499019999999998e+02
+1.100330000000000e+02 +4.140660000000000e+01
+6.756139999999998e+02 +2.857780000000000e+02
+6.459910000000000e+02 +2.569190000000000e+02
+6.639230000000000e+02 +2.698980000000000e+02
+1.033870000000000e+03 +2.836410000000000e+02
+1.518630000000000e+03 +4.909540000000000e+02
+1.365120000000000e+03 +4.210920000000000e+02
+1.158710000000000e+03 +3.820050000000000e+02
+6.369340000000000e+02 +1.692580000000000e+02
+6.154080000000000e+02 +2.435850000000000e+02
+1.544140000000000e+03 +4.232650000000000e+02
+1.045370000000000e+03 +3.105280000000000e+02
+1.486080000000000e+03 +4.441220000000000e+02
+6.714340000000000e+02 +2.605530000000000e+02
+2.601390000000000e+02 +1.530040000000000e+02
+1.375450000000000e+03 +4.151470000000000e+02
+6.573819999999999e+02 +2.490850000000000e+02
+6.476970000000000e+02 +2.212240000000000e+02
+6.882130000000002e+02 +2.398250000000000e+02
+6.612880000000000e+02 +2.627490000000000e+02
+6.596780000000000e+02 +1.938980000000000e+02
+6.060920000000000e+02 +2.445500000000000e+02
+6.357480000000000e+02 +2.273370000000000e+02
+1.008110000000000e+03 +3.232340000000001e+02
+1.134130000000000e+03 +4.331120000000000e+02
+6.385090000000000e+02 +1.816320000000000e+02
+9.058440000000001e+02 +2.056730000000000e+02
+1.005690000000000e+03 +3.163430000000000e+02
+6.405400000000000e+02 +2.251660000000000e+02
+6.515710000000000e+02 +2.577670000000000e+02
+8.372339999999998e+02 +2.827050000000000e+02
+1.001170000000000e+03 +3.138270000000000e+02
+1.155440000000000e+03 +4.419400000000000e+02
+6.576519999999998e+02 +2.100310000000000e+02
+1.069470000000000e+03 +3.604630000000000e+02
+6.827530000000000e+02 +2.620490000000000e+02
+6.561770000000000e+02 +2.328590000000000e+02
+8.963389999999998e+02 +2.797340000000000e+02
+1.072620000000000e+03 +4.209160000000000e+02
+6.565730000000000e+02 +2.176010000000000e+02
+1.112390000000000e+03 +3.844560000000000e+02
+1.251470000000000e+03 +5.382840000000000e+02
+6.629850000000000e+02 +2.577030000000000e+02
+8.910110000000002e+02 +2.755400000000000e+02
+6.344240000000000e+02 +2.158010000000000e+02
+9.969480000000000e+02 +2.720290000000000e+02
+1.271830000000000e+03 +3.615170000000000e+02
+1.322910000000000e+03 +3.444450000000000e+02
+6.832600000000000e+02 +2.545600000000000e+02
+1.078230000000000e+03 +3.816910000000000e+02
+2.738260000000000e+02 +1.471150000000000e+02
+1.083360000000000e+03 +3.401450000000000e+02
+6.581930000000000e+02 +2.455110000000000e+02
+6.451080000000002e+02 +2.145530000000000e+02
+9.929980000000000e+02 +2.739380000000000e+02
+1.050970000000000e+03 +4.520430000000000e+02
+1.074230000000000e+03 +4.067760000000000e+02
+9.059829999999999e+02 +2.743650000000000e+02
+1.065400000000000e+03 +3.540300000000000e+02
+1.341560000000000e+03 +3.374369999999999e+02
+6.730880000000002e+02 +2.518360000000000e+02
+9.006189999999998e+02 +2.853580000000000e+02
+1.069040000000000e+03 +3.931780000000001e+02
+6.357940000000000e+02 +2.403290000000000e+02
+1.116720000000000e+03 +3.829910000000000e+02
+1.242070000000000e+03 +3.447260000000000e+02
+6.367150000000000e+02 +2.418270000000000e+02
+1.101900000000000e+03 +3.616100000000000e+02
+9.894230000000000e+02 +3.458280000000001e+02
+1.571790000000000e+03 +4.424170000000000e+02
+1.083660000000000e+03 +4.593220000000000e+02
+1.079160000000000e+03 +4.082680000000000e+02
+1.066100000000000e+03 +4.086260000000000e+02
+1.112200000000000e+03 +3.781220000000000e+02
+1.063510000000000e+03 +3.386650000000000e+02
+1.644840000000000e+03 +4.641140000000000e+02
+1.003520000000000e+03 +3.492660000000000e+02
+1.052090000000000e+03 +3.465320000000000e+02
+6.378640000000000e+02 +2.076170000000000e+02
+1.042870000000000e+03 +3.361350000000000e+02
+1.073070000000000e+03 +3.948980000000000e+02
+1.058260000000000e+03 +3.778610000000000e+02
+6.407130000000002e+02 +2.274730000000000e+02
+1.106910000000000e+03 +3.484480000000001e+02
+2.711850000000000e+02 +1.373460000000000e+02
+6.803190000000000e+02 +2.466650000000000e+02
+1.078600000000000e+03 +3.419900000000000e+02
+1.080370000000000e+03 +3.368550000000000e+02
+6.389150000000000e+02 +2.070480000000000e+02
+9.017670000000001e+02 +2.884200000000000e+02
+1.069270000000000e+03 +2.930900000000000e+02
+1.045150000000000e+03 +3.277490000000000e+02
+1.568560000000000e+03 +4.251000000000000e+02
+1.065480000000000e+03 +4.098650000000000e+02
+1.154520000000000e+03 +4.425270000000000e+02
+1.363140000000000e+03 +4.655970000000000e+02
+6.565230000000000e+02 +2.562190000000000e+02
+1.114770000000000e+03 +3.023150000000000e+02
+6.774040000000000e+02 +2.605100000000000e+02
+1.051410000000000e+03 +3.402970000000000e+02
+9.779750000000000e+02 +3.725190000000000e+02
+9.103060000000000e+02 +2.552520000000000e+02
+1.074870000000000e+03 +3.771880000000001e+02
+9.754450000000001e+02 +3.468420000000000e+02
+6.720069999999999e+02 +2.119330000000000e+02
+6.569730000000002e+02 +2.242700000000000e+02
+8.363240000000000e+02 +2.568270000000000e+02
+1.066510000000000e+03 +3.426050000000000e+02
+1.334640000000000e+03 +4.805330000000000e+02
+1.008140000000000e+03 +3.190210000000000e+02
+8.813420000000000e+02 +2.486390000000000e+02
+1.070020000000000e+03 +4.179780000000000e+02
+1.035320000000000e+03 +3.562020000000000e+02
+1.215990000000000e+03 +4.464310000000000e+02
+9.090460000000000e+02 +2.774410000000000e+02
+8.630369999999998e+02 +2.732090000000000e+02
+1.781850000000000e+03 +5.598630000000001e+02
+1.048800000000000e+03 +3.118210000000000e+02
+1.080970000000000e+03 +3.492619999999999e+02
+1.303840000000000e+03 +4.495240000000000e+02
+1.215670000000000e+03 +4.349800000000000e+02
+1.367530000000000e+03 +5.033510000000000e+02
+8.622930000000000e+02 +3.785570000000000e+02
+8.705910000000000e+02 +2.695570000000000e+02
+8.743049999999999e+02 +2.522220000000000e+02
+1.369960000000000e+03 +4.942600000000000e+02
+7.817389999999998e+02 +3.414469999999999e+02
+7.761680000000000e+02 +3.524110000000000e+02
+9.989560000000000e+02 +3.260400000000000e+02
+1.073050000000000e+03 +3.582170000000000e+02
+9.959030000000000e+02 +3.672270000000000e+02
+8.626820000000000e+02 +3.492990000000001e+02
+7.841319999999999e+02 +2.642160000000000e+02
+1.855480000000000e+03 +5.745119999999999e+02
+9.992940000000000e+02 +3.747000000000000e+02
+8.530930000000002e+02 +2.737430000000000e+02
+9.105990000000000e+02 +2.660050000000000e+02
+9.273869999999999e+02 +2.632240000000000e+02
+1.288470000000000e+03 +4.645570000000000e+02
+1.071580000000000e+03 +3.851910000000000e+02
+9.893940000000000e+02 +3.250210000000000e+02
+9.789890000000000e+02 +3.144040000000000e+02
+9.952020000000000e+02 +3.854600000000000e+02
+1.000240000000000e+03 +3.881180000000001e+02
+1.049500000000000e+03 +3.750980000000000e+02
+8.673830000000000e+02 +3.948470000000000e+02
+4.744770000000000e+02 +1.657650000000000e+02
+9.576130000000001e+02 +2.785110000000000e+02
+9.624520000000000e+02 +3.736940000000000e+02
+9.707790000000000e+02 +3.591090000000001e+02
+1.767740000000000e+03 +5.393630000000001e+02
+1.006500000000000e+03 +3.713110000000000e+02
+9.695410000000001e+02 +3.459850000000000e+02
+2.818990000000000e+02 +1.553430000000000e+02
+8.675650000000001e+02 +3.468920000000000e+02
+9.862000000000000e+02 +3.816920000000000e+02
+1.010930000000000e+03 +3.585900000000000e+02
+1.072290000000000e+03 +3.303330000000000e+02
+1.132600000000000e+03 +3.957790000000000e+02
+8.738630000000001e+02 +3.677100000000000e+02
+9.732569999999999e+02 +3.146210000000000e+02
+4.776820000000000e+02 +1.558170000000000e+02
+9.937260000000000e+02 +3.383420000000000e+02
+2.766260000000000e+02 +1.251850000000000e+02
+7.893950000000000e+02 +3.246920000000000e+02
+9.944160000000001e+02 +3.071400000000000e+02
+9.969510000000000e+02 +3.059570000000000e+02
+8.716720000000000e+02 +3.007980000000000e+02
+1.128970000000000e+03 +3.936160000000000e+02
+1.078040000000000e+03 +3.656660000000000e+02
+9.793140000000000e+02 +3.153970000000000e+02
+1.000680000000000e+03 +3.558840000000000e+02
+9.990839999999999e+02 +3.118150000000000e+02
+1.406640000000000e+03 +5.805870000000000e+02
+7.918130000000000e+02 +3.253670000000000e+02
+8.614180000000000e+02 +2.609220000000000e+02
+9.014410000000000e+02 +2.527650000000000e+02
+1.407000000000000e+03 +5.354019999999998e+02
+9.933530000000000e+02 +3.146330000000000e+02
+9.069570000000000e+02 +3.903010000000000e+02
+2.774420000000000e+02 +1.141580000000000e+02
+1.092470000000000e+03 +3.354730000000000e+02
+8.487940000000000e+02 +3.693060000000000e+02
+1.003640000000000e+03 +3.353550000000000e+02
+1.045520000000000e+02 +3.207460000000000e+01
+6.182840000000000e+02 +2.644770000000000e+02
+8.078560000000001e+02 +2.731760000000000e+02
+1.861630000000000e+03 +5.809220000000000e+02
+8.709660000000000e+02 +2.331240000000000e+02
+1.416610000000000e+03 +5.527430000000001e+02
+9.892740000000000e+02 +3.069900000000000e+02
+9.976410000000000e+02 +3.169120000000001e+02
+1.299640000000000e+03 +4.684490000000000e+02
+8.933450000000000e+02 +3.082800000000000e+02
+6.185010000000000e+02 +2.482670000000000e+02
+1.047620000000000e+03 +3.214300000000000e+02
+8.955870000000000e+02 +2.446430000000000e+02
+1.392550000000000e+03 +4.904150000000000e+02
+9.860470000000000e+02 +2.855900000000000e+02
+6.573560000000001e+02 +2.650130000000000e+02
+9.258730000000000e+02 +2.805680000000000e+02
+6.256630000000000e+02 +2.441310000000000e+02
+6.213270000000000e+02 +2.371150000000000e+02
+1.065530000000000e+03 +3.432480000000001e+02
+1.105600000000000e+03 +4.331750000000000e+02
+9.787480000000000e+02 +2.737350000000000e+02
+9.779580000000000e+02 +2.845510000000000e+02
+6.301110000000000e+02 +2.221000000000000e+02
+1.414820000000000e+03 +5.426640000000000e+02
+6.293930000000000e+02 +2.290020000000000e+02
+1.083240000000000e+03 +3.301560000000000e+02
+6.617180000000002e+02 +2.326460000000000e+02
+1.126530000000000e+03 +3.836150000000000e+02
+1.189700000000000e+03 +3.668980000000000e+02
+1.282820000000000e+03 +5.305540000000000e+02
+9.248040000000000e+02 +3.364400000000000e+02
+6.209760000000000e+02 +2.270710000000000e+02
+6.636419999999998e+02 +2.499890000000000e+02
+1.290160000000000e+03 +5.345500000000000e+02
+9.942520000000000e+02 +3.006850000000000e+02
+6.317940000000000e+02 +2.089860000000000e+02
+2.782850000000000e+02 +8.234310000000001e+01
+1.286990000000000e+03 +4.823120000000000e+02
+9.312170000000000e+02 +2.736280000000000e+02
+6.203140000000000e+02 +1.836470000000000e+02
+8.948670000000000e+02 +2.340220000000000e+02
+6.241400000000000e+02 +2.174230000000000e+02
+1.463780000000000e+03 +5.172090000000002e+02
+9.587390000000000e+02 +2.652310000000000e+02
+6.202950000000000e+02 +2.017400000000000e+02
+6.683660000000001e+02 +2.402490000000000e+02
+4.570590000000000e+02 +1.168020000000000e+02
+1.679330000000000e+03 +5.099170000000000e+02
+6.713650000000000e+02 +2.505120000000000e+02
+2.855090000000000e+02 +7.573630000000000e+01
+9.064000000000000e+02 +2.932700000000000e+02
+8.921980000000000e+02 +2.315260000000000e+02
+9.903170000000000e+02 +2.776030000000000e+02
+1.471300000000000e+03 +4.619440000000000e+02
+1.853750000000000e+03 +6.009050000000000e+02
+6.218070000000000e+02 +2.186030000000000e+02
+4.513120000000000e+02 +1.245630000000000e+02
+1.048940000000000e+03 +2.828620000000000e+02
+1.306340000000000e+03 +5.031060000000000e+02
+9.773390000000001e+02 +2.749450000000000e+02
+9.092460000000000e+02 +2.542480000000000e+02
+9.920250000000000e+02 +2.891030000000000e+02
+6.229109999999999e+02 +1.813810000000000e+02
+6.807880000000000e+02 +2.446020000000000e+02
+7.913819999999999e+02 +2.475620000000000e+02
+8.071160000000001e+02 +3.432110000000000e+02
+1.297890000000000e+03 +4.787500000000000e+02
+1.470230000000000e+03 +3.965680000000000e+02
+1.866930000000000e+03 +5.566930000000000e+02
+8.923500000000000e+02 +2.238690000000000e+02
+1.036800000000000e+03 +3.051460000000000e+02
+6.629200000000000e+02 +2.358470000000000e+02
+1.616920000000000e+03 +6.195490000000000e+02
+8.546600000000000e+02 +2.029740000000000e+02
+7.711860000000000e+02 +2.564180000000000e+02
+1.476300000000000e+03 +4.616640000000000e+02
+9.761200000000000e+02 +2.868840000000000e+02
+1.867420000000000e+03 +5.711790000000000e+02
+7.746720000000000e+02 +2.631630000000000e+02
+7.843370000000000e+02 +2.682720000000000e+02
+6.630790000000000e+02 +2.349540000000000e+02
+1.847880000000000e+03 +5.651310000000000e+02
+7.715770000000000e+02 +2.594130000000000e+02
+7.763930000000000e+02 +2.687330000000000e+02
+7.728850000000000e+02 +2.651730000000000e+02
+1.596540000000000e+03 +5.604520000000000e+02
+1.326600000000000e+03 +4.644450000000000e+02
+1.862580000000000e+03 +5.340459999999998e+02
+1.626420000000000e+03 +5.937270000000000e+02
+9.780170000000001e+02 +2.727790000000000e+02
+7.655939999999998e+02 +2.386110000000000e+02
+7.685119999999999e+02 +2.604470000000000e+02
+8.992970000000000e+02 +2.190550000000000e+02
+6.597739999999999e+02 +2.142520000000000e+02
+1.628500000000000e+03 +5.702560000000000e+02
+7.640610000000000e+02 +2.367900000000000e+02
+6.948070000000000e+02 +2.875160000000000e+02
+6.880690000000000e+02 +3.042080000000000e+02
+9.729930000000001e+02 +4.063720000000000e+02
+1.283130000000000e+03 +4.234730000000000e+02
+1.848620000000000e+03 +5.186540000000000e+02
+9.422560000000000e+02 +4.054480000000000e+02
+1.868570000000000e+03 +5.417930000000000e+02
+7.619220000000000e+02 +2.403440000000000e+02
+8.803080000000000e+02 +2.122580000000000e+02
+9.621390000000000e+02 +2.720340000000000e+02
+7.765860000000000e+02 +2.384300000000000e+02
+9.020960000000000e+02 +2.352430000000000e+02
+7.645230000000000e+02 +2.629980000000000e+02
+1.303490000000000e+03 +4.284290000000000e+02
+7.800830000000002e+02 +2.236440000000000e+02
+8.868580000000002e+02 +2.356380000000000e+02
+6.981580000000000e+02 +3.118510000000000e+02
+6.937189999999998e+02 +3.129900000000000e+02
+7.835930000000002e+02 +2.128930000000000e+02
+8.049750000000000e+02 +3.514630000000000e+02
+7.856960000000000e+02 +2.405500000000000e+02
+3.242440000000000e+02 +1.211750000000000e+02
+7.864220000000000e+02 +2.404170000000000e+02
+7.872210000000000e+02 +2.665550000000000e+02
+2.881940000000000e+02 +8.335190000000000e+01
+6.941280000000000e+02 +2.482960000000000e+02
+1.292100000000000e+03 +4.293560000000000e+02
+7.712110000000000e+02 +2.449200000000000e+02
+9.284299999999999e+02 +3.623400000000000e+02
+9.891130000000001e+02 +2.638130000000000e+02
+3.944520000000000e+02 +1.077460000000000e+02
+7.831039999999998e+02 +2.354410000000000e+02
+1.293640000000000e+03 +4.595770000000000e+02
+8.712680000000000e+02 +2.084600000000000e+02
+6.896480000000000e+02 +2.352080000000000e+02
+7.809680000000002e+02 +2.381170000000000e+02
+8.143750000000000e+02 +2.916470000000000e+02
+6.932639999999999e+02 +2.986540000000000e+02
+7.904889999999998e+02 +2.372360000000000e+02
+9.033560000000000e+02 +2.253910000000000e+02
+4.752800000000000e+02 +2.170700000000000e+02
+7.523230000000000e+02 +2.416290000000000e+02
+6.912950000000000e+02 +2.875770000000000e+02
+7.897760000000002e+02 +2.172280000000000e+02
+1.189600000000000e+03 +3.229620000000000e+02
+7.685670000000000e+02 +2.141150000000000e+02
+7.933339999999999e+02 +2.551310000000000e+02
+7.870700000000001e+02 +2.147570000000000e+02
+2.307360000000000e+03 +6.856220000000000e+02
+6.932810000000002e+02 +2.643800000000000e+02
+1.347140000000000e+03 +4.122480000000001e+02
+7.434900000000000e+02 +2.375880000000000e+02
+6.078520000000000e+02 +1.723470000000000e+02
+1.307610000000000e+03 +4.183850000000000e+02
+7.416180000000001e+02 +2.380830000000000e+02
+1.355970000000000e+03 +4.353650000000000e+02
+1.433300000000000e+03 +4.544770000000000e+02
+7.942320000000000e+02 +2.475980000000000e+02
+1.688000000000000e+03 +5.288300000000000e+02
+1.561180000000000e+03 +4.848630000000001e+02
+6.920780000000000e+02 +1.618490000000000e+02
+1.083040000000000e+02 +3.508610000000000e+01
+7.609610000000000e+02 +2.047020000000000e+02
+1.566360000000000e+03 +4.766550000000000e+02
+7.474069999999998e+02 +2.168480000000000e+02
+1.301220000000000e+03 +4.547150000000000e+02
+1.310550000000000e+03 +4.036640000000000e+02
+7.887210000000000e+02 +2.137980000000000e+02
+7.875560000000000e+02 +2.515650000000000e+02
+8.580080000000000e+02 +2.492210000000000e+02
+8.929030000000000e+02 +3.471540000000000e+02
+4.295350000000000e+02 +1.141710000000000e+02
+1.305180000000000e+03 +4.575990000000000e+02
+6.920889999999998e+02 +3.074170000000001e+02
+1.310780000000000e+03 +4.438790000000000e+02
+7.744910000000001e+02 +2.119810000000000e+02
+9.082310000000000e+02 +3.711770000000000e+02
+9.775800000000000e+02 +2.961650000000000e+02
+7.727980000000000e+02 +2.369290000000000e+02
+4.816240000000000e+02 +2.154020000000000e+02
+6.939860000000001e+02 +2.434150000000000e+02
+8.522990000000000e+02 +3.221180000000000e+02
+1.311360000000000e+03 +4.728160000000000e+02
+1.177910000000000e+02 +4.225090000000000e+01
+7.493670000000000e+02 +2.092950000000000e+02
+7.833589999999998e+02 +2.254670000000000e+02
+7.831080000000002e+02 +2.447500000000000e+02
+6.902560000000002e+02 +1.977320000000000e+02
+1.563110000000000e+03 +4.660020000000000e+02
+7.870030000000000e+02 +2.368400000000000e+02
+4.633770000000000e+02 +1.481220000000000e+02
+9.675490000000000e+02 +3.085420000000001e+02
+1.309010000000000e+03 +4.052500000000000e+02
+7.822110000000000e+02 +2.405640000000000e+02
+4.921460000000000e+02 +2.235790000000000e+02
+1.296350000000000e+03 +4.316650000000000e+02
+7.858589999999998e+02 +2.338490000000000e+02
+8.849800000000000e+02 +3.515860000000000e+02
+7.447510000000002e+02 +1.721330000000000e+02
+6.929560000000000e+02 +1.881070000000000e+02
+7.657630000000000e+02 +2.274340000000000e+02
+6.895030000000000e+02 +1.715270000000000e+02
+1.110340000000000e+03 +4.079130000000000e+02
+7.890160000000002e+02 +2.219970000000000e+02
+8.100430000000000e+02 +3.731460000000000e+02
+4.837160000000000e+02 +2.201440000000000e+02
+1.303720000000000e+03 +4.616430000000000e+02
+8.593739999999998e+02 +2.458250000000000e+02
+7.840760000000000e+02 +2.297460000000000e+02
+1.299410000000000e+03 +4.418710000000000e+02
+4.782460000000000e+02 +1.526390000000000e+02
+1.284610000000000e+03 +4.347870000000000e+02
+7.771860000000000e+02 +2.285130000000000e+02
+9.782569999999999e+02 +2.909570000000000e+02
+9.650069999999999e+02 +2.871770000000000e+02
+7.789870000000000e+02 +2.091280000000000e+02
+6.897320000000000e+02 +1.790690000000000e+02
+7.699280000000000e+02 +2.636760000000000e+02
+1.109260000000000e+03 +4.446320000000000e+02
+4.962450000000000e+02 +2.056410000000000e+02
+4.944750000000000e+02 +2.080380000000000e+02
+7.906250000000000e+02 +2.034370000000000e+02
+9.724580000000000e+02 +2.710790000000000e+02
+1.134440000000000e+03 +3.869790000000000e+02
+1.315130000000000e+03 +4.290970000000000e+02
+7.881319999999999e+02 +2.073500000000000e+02
+7.693889999999999e+02 +2.170410000000000e+02
+1.105790000000000e+03 +4.221130000000001e+02
+7.830380000000000e+02 +2.080450000000000e+02
+7.799349999999999e+02 +2.158570000000000e+02
+1.295770000000000e+03 +4.186460000000000e+02
+4.744870000000000e+02 +2.129270000000000e+02
+1.294050000000000e+03 +3.760820000000000e+02
+7.770430000000000e+02 +2.259030000000000e+02
+6.785060000000002e+02 +3.455250000000000e+02
+8.574140000000000e+02 +2.701490000000000e+02
+4.887740000000000e+02 +2.309130000000000e+02
+6.379250000000000e+02 +1.763950000000000e+02
+1.286280000000000e+03 +4.059850000000000e+02
+7.788160000000000e+02 +1.937040000000000e+02
+7.608539999999998e+02 +2.313150000000000e+02
+6.932890000000000e+02 +2.093150000000000e+02
+9.458200000000001e+02 +3.396969999999999e+02
+6.808339999999999e+02 +3.015620000000000e+02
+4.963380000000000e+02 +2.355360000000000e+02
+1.059500000000000e+03 +3.670520000000000e+02
+1.317410000000000e+03 +4.020460000000000e+02
+7.510580000000000e+02 +1.921830000000000e+02
+7.798280000000000e+02 +2.168280000000000e+02
+6.832880000000000e+02 +3.020680000000000e+02
+1.307120000000000e+03 +3.672170000000000e+02
+7.788819999999999e+02 +2.037520000000000e+02
+8.580200000000000e+02 +2.461070000000000e+02
+1.303220000000000e+03 +3.580350000000000e+02
+9.296330000000000e+02 +3.429980000000001e+02
+4.933930000000000e+02 +2.164680000000000e+02
+4.887770000000000e+02 +2.111300000000000e+02
+7.871070000000000e+02 +2.092280000000000e+02
+7.477210000000000e+02 +1.791440000000000e+02
+1.094750000000000e+03 +4.270630000000001e+02
+8.624130000000000e+02 +2.847820000000000e+02
+8.707719999999998e+02 +2.558230000000000e+02
+3.250210000000000e+02 +8.206420000000000e+01
+1.293860000000000e+03 +4.056780000000001e+02
+7.894299999999999e+02 +2.043000000000000e+02
+7.793560000000001e+02 +1.945370000000000e+02
+7.557680000000000e+02 +2.518340000000000e+02
+7.828539999999998e+02 +1.977060000000000e+02
+7.516270000000000e+02 +2.112310000000000e+02
+8.555280000000000e+02 +2.302440000000000e+02
+1.127450000000000e+03 +4.254180000000000e+02
+4.942210000000000e+02 +2.180210000000000e+02
+7.455470000000000e+02 +2.023180000000000e+02
+7.828170000000000e+02 +2.135930000000000e+02
+6.464190000000000e+02 +3.746070000000000e+02
+7.543969999999998e+02 +2.638910000000000e+02
+9.707030000000000e+02 +2.877560000000000e+02
+1.339850000000000e+03 +5.667170000000000e+02
+1.389660000000000e+03 +5.431480000000000e+02
+9.310340000000000e+02 +3.985760000000000e+02
+7.886890000000000e+02 +2.176650000000000e+02
+7.632410000000001e+02 +2.051050000000000e+02
+4.854800000000000e+02 +1.870910000000000e+02
+7.794900000000000e+02 +2.055880000000000e+02
+7.894140000000000e+02 +2.202510000000000e+02
+1.301740000000000e+03 +4.088230000000000e+02
+8.699299999999999e+02 +2.291180000000000e+02
+1.330150000000000e+03 +5.336640000000000e+02
+4.931880000000001e+02 +1.914360000000000e+02
+1.084620000000000e+03 +3.858450000000000e+02
+1.306590000000000e+03 +4.042130000000000e+02
+7.827560000000002e+02 +2.031110000000000e+02
+8.706700000000000e+02 +2.127560000000000e+02
+4.868200000000000e+02 +2.408410000000000e+02
+7.415250000000000e+02 +2.796520000000000e+02
+1.380860000000000e+03 +5.303690000000000e+02
+1.108860000000000e+03 +4.756690000000000e+02
+7.534580000000002e+02 +2.048640000000000e+02
+4.649990000000000e+02 +1.326940000000000e+02
+1.329440000000000e+03 +4.773940000000000e+02
+4.860720000000000e+02 +1.820500000000000e+02
+7.762580000000000e+02 +2.203560000000000e+02
+1.324990000000000e+03 +5.099540000000000e+02
+1.041750000000000e+03 +4.074970000000000e+02
+7.624950000000000e+02 +2.142710000000000e+02
+8.715440000000000e+02 +2.944530000000000e+02
+1.069780000000000e+03 +3.872590000000000e+02
+1.332970000000000e+03 +4.452570000000000e+02
+4.900820000000000e+02 +1.779420000000000e+02
+1.117740000000000e+03 +3.541140000000001e+02
+7.455900000000000e+02 +1.843420000000000e+02
+8.571319999999999e+02 +2.116350000000000e+02
+8.175700000000001e+02 +3.282270000000000e+02
+7.549160000000001e+02 +1.929180000000000e+02
+1.011560000000000e+03 +3.803680000000001e+02
+1.100260000000000e+03 +3.379320000000000e+02
+1.356210000000000e+03 +4.542800000000000e+02
+1.365020000000000e+03 +4.356650000000000e+02
+1.730490000000000e+03 +6.011290000000000e+02
+1.011340000000000e+03 +4.015810000000000e+02
+7.405889999999998e+02 +2.149330000000000e+02
+1.280900000000000e+03 +4.811640000000000e+02
+8.963320000000000e+02 +2.794360000000000e+02
+6.742439999999998e+02 +2.841390000000000e+02
+8.433140000000000e+02 +2.181090000000000e+02
+7.538110000000000e+02 +1.910470000000000e+02
+1.221440000000000e+03 +5.059630000000000e+02
+6.893400000000000e+02 +3.413880000000001e+02
+7.070330000000000e+02 +3.389180000000000e+02
+6.791660000000001e+02 +3.035700000000000e+02
+9.242510000000000e+02 +3.551650000000000e+02
+9.040570000000000e+02 +3.030820000000000e+02
+1.015920000000000e+03 +3.162300000000000e+02
+1.102980000000000e+03 +4.279610000000000e+02
+1.075640000000000e+03 +4.881660000000000e+02
+6.580419999999998e+02 +2.709820000000000e+02
+1.811120000000000e+03 +5.934200000000000e+02
+1.004430000000000e+03 +3.706350000000000e+02
+6.901200000000000e+02 +2.142560000000000e+02
+1.009750000000000e+03 +3.839940000000000e+02
+1.066830000000000e+03 +4.044640000000000e+02
+6.538300000000000e+02 +2.635270000000000e+02
+1.580330000000000e+03 +4.938170000000000e+02
+8.725490000000000e+02 +1.906730000000000e+02
+1.075460000000000e+03 +5.262780000000000e+02
+6.674780000000002e+02 +3.060770000000000e+02
+8.956270000000000e+02 +3.007660000000000e+02
+6.756820000000000e+02 +2.386900000000000e+02
+1.741130000000000e+03 +5.354620000000000e+02
+6.345290000000000e+02 +2.723250000000000e+02
+6.913250000000000e+02 +2.278920000000000e+02
+1.528950000000000e+03 +5.580340000000000e+02
+1.081580000000000e+03 +3.089980000000000e+02
+1.225660000000000e+03 +4.747460000000000e+02
+1.068350000000000e+03 +4.619970000000000e+02
+9.256849999999999e+02 +3.007980000000000e+02
+1.000740000000000e+03 +3.647040000000000e+02
+1.331260000000000e+03 +4.780640000000000e+02
+6.744789999999998e+02 +2.255150000000000e+02
+2.664430000000000e+03 +8.286260000000002e+02
+9.229520000000000e+02 +3.332120000000000e+02
+6.954330000000000e+02 +2.162260000000000e+02
+6.352650000000000e+02 +2.686600000000000e+02
+1.742680000000000e+03 +5.788230000000000e+02
+1.517340000000000e+03 +5.326559999999999e+02
+1.075780000000000e+03 +4.412880000000000e+02
+1.788050000000000e+03 +5.515549999999999e+02
+6.267330000000002e+02 +3.072360000000000e+02
+9.594180000000000e+02 +3.688670000000000e+02
+6.389380000000000e+02 +2.642340000000000e+02
+6.589090000000000e+02 +2.716230000000000e+02
+1.048300000000000e+03 +3.496869999999999e+02
+6.410599999999999e+02 +2.955510000000000e+02
+9.096770000000000e+02 +2.891900000000000e+02
+8.983330000000002e+02 +2.801840000000000e+02
+1.077510000000000e+02 +5.003200000000000e+01
+1.074910000000000e+03 +4.474020000000000e+02
+1.039350000000000e+03 +3.409540000000000e+02
+6.350700000000001e+02 +2.565920000000000e+02
+1.151150000000000e+03 +4.572590000000000e+02
+9.673440000000001e+02 +2.896450000000000e+02
+6.554600000000000e+02 +2.537330000000000e+02
+1.855720000000000e+03 +5.986630000000000e+02
+8.967600000000000e+02 +2.965700000000000e+02
+1.322290000000000e+03 +5.411970000000000e+02
+6.558260000000000e+02 +2.376550000000000e+02
+2.631300000000000e+03 +8.026039999999998e+02
+6.464930000000001e+02 +3.627340000000000e+02
+6.394109999999999e+02 +2.704600000000000e+02
+1.319680000000000e+03 +5.001280000000000e+02
+1.064620000000000e+03 +3.001200000000000e+02
+6.368410000000000e+02 +2.236910000000000e+02
+6.185110000000000e+02 +2.405750000000000e+02
+1.063070000000000e+03 +3.468550000000000e+02
+6.509770000000000e+02 +2.447410000000000e+02
+8.931039999999998e+02 +2.583480000000000e+02
+1.353730000000000e+03 +4.675930000000000e+02
+6.521970000000000e+02 +2.277250000000000e+02
+1.034360000000000e+03 +3.491540000000000e+02
+6.270119999999999e+02 +3.575310000000000e+02
+1.454410000000000e+03 +5.506669999999998e+02
+6.388270000000000e+02 +2.466200000000000e+02
+8.711080000000002e+02 +2.485990000000000e+02
+6.380210000000000e+02 +2.323630000000000e+02
+9.029500000000000e+02 +3.017630000000000e+02
+8.923830000000000e+02 +2.897740000000000e+02
+1.157470000000000e+03 +4.637490000000000e+02
+6.436510000000000e+02 +1.993470000000000e+02
+8.695880000000002e+02 +2.359830000000000e+02
+6.553530000000002e+02 +2.459320000000000e+02
+6.782430000000001e+02 +2.108120000000000e+02
+1.143580000000000e+02 +5.065890000000000e+01
+1.486170000000000e+03 +5.163460000000000e+02
+1.122000000000000e+03 +4.045560000000000e+02
+6.759589999999999e+02 +3.101410000000000e+02
+6.632539999999998e+02 +2.141060000000000e+02
+1.000580000000000e+03 +2.977760000000000e+02
+8.906160000000001e+02 +2.696910000000000e+02
+1.055390000000000e+03 +3.578250000000000e+02
+1.186760000000000e+02 +3.702490000000000e+01
+1.320070000000000e+03 +4.890230000000000e+02
+7.496430000000000e+02 +2.789740000000000e+02
+6.600939999999998e+02 +2.289500000000000e+02
+1.142470000000000e+03 +4.630570000000000e+02
+7.458980000000000e+02 +1.679950000000000e+02
+6.519710000000000e+02 +2.433130000000000e+02
+8.959310000000000e+02 +2.958310000000000e+02
+6.555570000000000e+02 +2.373610000000000e+02
+6.605860000000000e+02 +1.935040000000000e+02
+8.405130000000000e+02 +2.090180000000000e+02
+1.561370000000000e+03 +4.675010000000000e+02
+7.509939999999998e+02 +2.718730000000000e+02
+6.496860000000000e+02 +2.312720000000000e+02
+6.564130000000000e+02 +1.768290000000000e+02
+8.580260000000002e+02 +2.443610000000000e+02
+1.361740000000000e+03 +4.220350000000000e+02
+6.351730000000000e+02 +2.302750000000000e+02
+1.039340000000000e+03 +3.340140000000000e+02
+6.799490000000000e+02 +2.929190000000001e+02
+6.148910000000000e+02 +2.181250000000000e+02
+6.336849999999999e+02 +2.120090000000000e+02
+9.946559999999999e+02 +2.358420000000000e+02
+8.677769999999998e+02 +2.229630000000000e+02
+6.950069999999999e+02 +3.192200000000000e+02
+6.521590000000000e+02 +2.211570000000000e+02
+8.983300000000000e+02 +2.480210000000000e+02
+1.093930000000000e+02 +2.881830000000000e+01
+6.380430000000000e+02 +2.157340000000000e+02
+8.904510000000000e+02 +2.525870000000000e+02
+8.704169999999998e+02 +2.359080000000000e+02
+6.626830000000000e+02 +2.117340000000000e+02
+1.082120000000000e+03 +4.254520000000000e+02
+8.814019999999998e+02 +3.203750000000000e+02
+8.512910000000001e+02 +2.298990000000000e+02
+6.677360000000001e+02 +2.337540000000000e+02
+6.519540000000002e+02 +2.081400000000000e+02
+1.068640000000000e+03 +5.484830000000002e+02
+6.636100000000000e+02 +1.918530000000000e+02
+8.862890000000000e+02 +2.493850000000000e+02
+1.074890000000000e+03 +3.773170000000000e+02
+6.353740000000000e+02 +2.138260000000000e+02
+1.066780000000000e+03 +4.610000000000000e+02
+1.082490000000000e+03 +4.307700000000000e+02
+6.348720000000000e+02 +1.863220000000000e+02
+6.640360000000002e+02 +1.887820000000000e+02
+1.373290000000000e+03 +5.532470000000000e+02
+6.933750000000000e+02 +3.585650000000000e+02
+1.073580000000000e+03 +4.020700000000000e+02
+6.647150000000000e+02 +2.062210000000000e+02
+1.081620000000000e+03 +4.340380000000000e+02
+6.608860000000002e+02 +1.919740000000000e+02
+6.685450000000000e+02 +1.897410000000000e+02
+1.340690000000000e+03 +4.749790000000000e+02
+1.161210000000000e+03 +3.818210000000000e+02
+1.237930000000000e+03 +5.583680000000001e+02
+6.973049999999999e+02 +3.131180000000000e+02
+6.676430000000000e+02 +2.115290000000000e+02
+9.023830000000000e+02 +2.448950000000000e+02
+1.065430000000000e+03 +4.026190000000000e+02
+1.325290000000000e+03 +4.728490000000000e+02
+1.068890000000000e+03 +5.177690000000000e+02
+6.297940000000000e+02 +1.929820000000000e+02
+1.063560000000000e+03 +3.462830000000000e+02
+6.356100000000000e+02 +2.043670000000000e+02
+1.063240000000000e+03 +3.457260000000000e+02
+6.157840000000000e+02 +3.097490000000000e+02
+1.014130000000000e+03 +3.136910000000000e+02
+1.065290000000000e+03 +3.860810000000000e+02
+1.350040000000000e+03 +4.732860000000000e+02
+1.061650000000000e+03 +5.097930000000000e+02
+6.610160000000002e+02 +1.844280000000000e+02
+6.851330000000000e+02 +2.173000000000000e+02
+1.069510000000000e+03 +3.650060000000000e+02
+5.943890000000000e+02 +3.131210000000000e+02
+8.983270000000000e+02 +2.474210000000000e+02
+1.010260000000000e+03 +3.371730000000000e+02
+1.092240000000000e+03 +3.558540000000001e+02
+8.907850000000000e+02 +3.933300000000000e+02
+6.949140000000000e+02 +2.210520000000000e+02
+6.996039999999998e+02 +3.398840000000000e+02
+9.113160000000000e+02 +2.481830000000000e+02
+1.047300000000000e+03 +3.576700000000000e+02
+8.619400000000001e+02 +2.045730000000000e+02
+6.765889999999998e+02 +2.414290000000000e+02
+8.805710000000000e+02 +2.475390000000000e+02
+1.049270000000000e+03 +3.166670000000001e+02
+8.902470000000000e+02 +4.001660000000000e+02
+6.523060000000000e+02 +1.959460000000000e+02
+6.761039999999998e+02 +2.798440000000000e+02
+9.131470000000000e+02 +2.597500000000000e+02
+9.623800000000000e+02 +2.957940000000001e+02
+8.593210000000000e+02 +3.103750000000000e+02
+8.589630000000002e+02 +3.194620000000000e+02
+8.586950000000001e+02 +3.078460000000000e+02
+9.786120000000000e+02 +3.478370000000000e+02
+1.073770000000000e+03 +4.686470000000000e+02
+9.300260000000000e+02 +2.889800000000000e+02
+1.080960000000000e+03 +3.904330000000000e+02
+8.803410000000000e+02 +3.629340000000000e+02
+7.060239999999999e+02 +2.123210000000000e+02
+1.229980000000000e+03 +4.835010000000000e+02
+8.574920000000000e+02 +3.115200000000000e+02
+9.782390000000000e+02 +2.879840000000001e+02
+1.065530000000000e+03 +4.316340000000000e+02
+1.215190000000000e+03 +5.284310000000000e+02
+1.010890000000000e+03 +2.832310000000000e+02
+1.572440000000000e+03 +4.444900000000000e+02
+1.071360000000000e+03 +4.490890000000000e+02
+8.936050000000000e+02 +3.670580000000000e+02
+1.559630000000000e+03 +6.401120000000000e+02
+1.001700000000000e+03 +4.329980000000001e+02
+8.894380000000000e+02 +2.952800000000000e+02
+8.529970000000000e+02 +3.865440000000000e+02
+7.809110000000002e+02 +3.193460000000000e+02
+1.132570000000000e+03 +4.979520000000000e+02
+1.568390000000000e+03 +6.286980000000000e+02
+7.754220000000000e+02 +3.608360000000000e+02
+1.067240000000000e+03 +4.714200000000000e+02
+9.460069999999999e+02 +2.174420000000000e+02
+5.910660000000000e+02 +1.892950000000000e+02
+1.088780000000000e+03 +3.488530000000000e+02
+9.950460000000000e+02 +4.173370000000000e+02
+7.471790000000000e+02 +2.394000000000000e+02
+9.964790000000000e+02 +3.990320000000000e+02
+4.898100000000000e+02 +1.732120000000000e+02
+1.532860000000000e+03 +5.972380000000001e+02
+8.645230000000000e+02 +3.281350000000000e+02
+1.787660000000000e+03 +7.099770000000000e+02
+8.651330000000000e+02 +3.605400000000000e+02
+1.083840000000000e+03 +4.064530000000000e+02
+1.081080000000000e+03 +3.449830000000000e+02
+1.051820000000000e+03 +4.466410000000000e+02
+2.797730000000000e+02 +1.752640000000000e+02
+2.815610000000000e+02 +1.607880000000000e+02
+9.963260000000000e+02 +3.117290000000001e+02
+1.621000000000000e+03 +6.382340000000000e+02
+1.071050000000000e+03 +3.773570000000000e+02
+7.952430000000001e+02 +3.247470000000000e+02
+7.808180000000000e+02 +2.480240000000000e+02
+1.005040000000000e+03 +4.204530000000000e+02
+1.005560000000000e+03 +4.073020000000000e+02
+1.038010000000000e+03 +3.094650000000000e+02
+1.276110000000000e+03 +5.719620000000000e+02
+1.399120000000000e+03 +6.649530000000000e+02
+9.917329999999999e+02 +2.973400000000000e+02
+6.094109999999999e+02 +3.040130000000000e+02
+1.595160000000000e+03 +5.387030000000000e+02
+8.034400000000001e+02 +2.679420000000000e+02
+6.359580000000002e+02 +1.959500000000000e+02
+2.753600000000000e+02 +1.317840000000000e+02
+9.532600000000000e+02 +3.366500000000000e+02
+9.886060000000000e+02 +3.077700000000000e+02
+1.064730000000000e+03 +3.972470000000000e+02
+4.542900000000000e+02 +1.462950000000000e+02
+6.101180000000001e+02 +2.865970000000000e+02
+9.914390000000000e+02 +4.164400000000000e+02
+9.987840000000000e+02 +4.024430000000000e+02
+1.288290000000000e+03 +5.228040000000000e+02
+1.086570000000000e+03 +3.953940000000000e+02
+9.141960000000000e+02 +3.917720000000000e+02
+1.567790000000000e+03 +4.285540000000000e+02
+1.631730000000000e+03 +6.899580000000002e+02
+9.784430000000000e+02 +4.058380000000000e+02
+1.642310000000000e+03 +6.178890000000000e+02
+1.060780000000000e+03 +3.988290000000000e+02
+2.836060000000000e+02 +1.467730000000000e+02
+9.763090000000000e+02 +4.207550000000000e+02
+1.252950000000000e+03 +5.319510000000000e+02
+1.000350000000000e+03 +2.967340000000001e+02
+1.574440000000000e+03 +6.328320000000000e+02
+1.913790000000000e+03 +5.914860000000000e+02
+1.630860000000000e+03 +7.160419999999998e+02
+1.554070000000000e+03 +6.506510000000000e+02
+1.291620000000000e+03 +4.945910000000000e+02
+1.017500000000000e+03 +3.777360000000000e+02
+1.408410000000000e+03 +5.504470000000000e+02
+6.178120000000000e+02 +2.553020000000000e+02
+6.236619999999998e+02 +2.691850000000000e+02
+1.044120000000000e+03 +3.512390000000001e+02
+1.410560000000000e+03 +5.677420000000000e+02
+1.309950000000000e+03 +5.642300000000000e+02
+1.136090000000000e+03 +4.515770000000000e+02
+1.361400000000000e+03 +5.483690000000000e+02
+6.200040000000000e+02 +2.472990000000000e+02
+1.220480000000000e+02 +5.132890000000000e+01
+1.087160000000000e+03 +3.619180000000000e+02
+1.403540000000000e+03 +5.629890000000000e+02
+6.297869999999998e+02 +2.664260000000000e+02
+6.697230000000002e+02 +3.500910000000000e+02
+1.562640000000000e+03 +5.053460000000000e+02
+9.775130000000000e+02 +2.663630000000000e+02
+9.328090000000000e+02 +3.387680000000001e+02
+9.889140000000000e+02 +3.711650000000000e+02
+9.888190000000000e+02 +3.412790000000000e+02
+1.588280000000000e+03 +6.094470000000000e+02
+1.112240000000000e+03 +3.718340000000000e+02
+1.561360000000000e+03 +5.264000000000000e+02
+1.584210000000000e+03 +5.821240000000000e+02
+2.324560000000000e+03 +7.733950000000000e+02
+1.410980000000000e+03 +5.585450000000000e+02
+9.762880000000000e+02 +3.000000000000000e+02
+9.138320000000000e+02 +2.791690000000000e+02
+6.245530000000000e+02 +2.078880000000000e+02
+6.190180000000000e+02 +2.466120000000000e+02
+6.642380000000001e+02 +3.473730000000001e+02
+6.584910000000001e+02 +2.276150000000000e+02
+1.571690000000000e+03 +4.676520000000000e+02
+1.367620000000000e+03 +4.415910000000000e+02
+1.371060000000000e+03 +5.788590000000000e+02
+1.424250000000000e+03 +6.250990000000000e+02
+9.314030000000000e+02 +3.243910000000000e+02
+1.570920000000000e+03 +5.271290000000000e+02
+1.550710000000000e+03 +5.450670000000000e+02
+1.585200000000000e+03 +4.740450000000000e+02
+2.817700000000000e+02 +8.263209999999999e+01
+1.057900000000000e+02 +3.188020000000000e+01
+1.288540000000000e+03 +5.081100000000000e+02
+9.778180000000000e+02 +2.824130000000000e+02
+6.248210000000000e+02 +2.387180000000000e+02
+6.250509999999998e+02 +1.918230000000000e+02
+1.282440000000000e+03 +5.009930000000001e+02
+9.117960000000000e+02 +3.156290000000000e+02
+2.687510000000000e+03 +8.336590000000000e+02
+9.920560000000000e+02 +3.467760000000000e+02
+1.570030000000000e+03 +5.017390000000000e+02
+6.216659999999998e+02 +2.176300000000000e+02
+1.003530000000000e+03 +3.199010000000000e+02
+1.549580000000000e+03 +4.709460000000000e+02
+9.771520000000000e+02 +2.670290000000000e+02
+1.539390000000000e+03 +4.895110000000000e+02
+6.590230000000000e+02 +2.499120000000000e+02
+1.574800000000000e+03 +4.870340000000000e+02
+1.401580000000000e+03 +5.758600000000000e+02
+1.868550000000000e+03 +7.280300000000000e+02
+6.223110000000000e+02 +1.761690000000000e+02
+1.292450000000000e+03 +4.931240000000000e+02
+1.571220000000000e+03 +4.834640000000000e+02
+1.053700000000000e+03 +3.780420000000000e+02
+6.220190000000000e+02 +1.818330000000000e+02
+6.576530000000000e+02 +1.968260000000000e+02
+9.856650000000000e+02 +2.572390000000000e+02
+6.364250000000000e+02 +2.180730000000000e+02
+2.679850000000000e+03 +8.249870000000000e+02
+1.848540000000000e+03 +6.907800000000000e+02
+6.189400000000001e+02 +1.940390000000000e+02
+1.569480000000000e+03 +4.950380000000000e+02
+1.323950000000000e+03 +4.534440000000000e+02
+2.765570000000000e+02 +1.032900000000000e+02
+6.188600000000000e+02 +1.675570000000000e+02
+9.786980000000000e+02 +2.590220000000000e+02
+1.061060000000000e+03 +3.723480000000000e+02
+6.646120000000000e+02 +1.842190000000000e+02
+1.247330000000000e+03 +4.609350000000000e+02
+1.426450000000000e+03 +5.196580000000000e+02
+6.236690000000000e+02 +1.620300000000000e+02
+1.417780000000000e+03 +4.661480000000000e+02
+7.774700000000000e+02 +2.618760000000000e+02
+6.224560000000000e+02 +1.618230000000000e+02
+7.476350000000000e+02 +2.801160000000000e+02
+1.305390000000000e+03 +4.764100000000000e+02
+7.867510000000002e+02 +3.029600000000000e+02
+1.298800000000000e+03 +4.756900000000000e+02
+1.296560000000000e+03 +4.945540000000000e+02
+1.231470000000000e+03 +4.483020000000000e+02
+1.295110000000000e+03 +4.615040000000000e+02
+7.696760000000000e+02 +2.701580000000000e+02
+7.781330000000000e+02 +2.971350000000000e+02
+7.748049999999999e+02 +3.017750000000000e+02
+1.496370000000000e+03 +5.884360000000000e+02
+1.613760000000000e+03 +6.427190000000001e+02
+1.615560000000000e+03 +6.224010000000000e+02
+2.816110000000000e+02 +1.033180000000000e+02
+6.873580000000002e+02 +2.920350000000000e+02
+1.434350000000000e+03 +4.674300000000000e+02
+6.892660000000002e+02 +2.827890000000000e+02
+1.483140000000000e+03 +5.839380000000000e+02
+7.847070000000000e+02 +2.769030000000000e+02
+7.802239999999998e+02 +2.753880000000000e+02
+1.602540000000000e+03 +6.448550000000000e+02
+1.609440000000000e+03 +5.844250000000000e+02
+7.800030000000000e+02 +2.829370000000000e+02
+6.956319999999999e+02 +2.549880000000000e+02
+1.470270000000000e+03 +5.756000000000000e+02
+1.869350000000000e+03 +7.229870000000000e+02
+1.840090000000000e+03 +6.893880000000000e+02
+7.830100000000000e+02 +2.846590000000000e+02
+1.631610000000000e+03 +6.222060000000000e+02
+1.588410000000000e+03 +6.550480000000000e+02
+7.796360000000002e+02 +2.720880000000000e+02
+7.837710000000002e+02 +2.741940000000000e+02
+2.722370000000000e+02 +2.665640000000000e+02
+1.490860000000000e+03 +6.070180000000000e+02
+1.853040000000000e+03 +6.696270000000000e+02
+6.079520000000000e+02 +2.070270000000000e+02
+1.614570000000000e+03 +6.419650000000000e+02
+6.928560000000001e+02 +2.857580000000000e+02
+7.802210000000000e+02 +2.704700000000000e+02
+9.590750000000000e+02 +4.243480000000000e+02
+7.702630000000000e+02 +2.441740000000000e+02
+1.433810000000000e+03 +4.484780000000000e+02
+1.867350000000000e+03 +6.742060000000000e+02
+6.920670000000000e+02 +2.474200000000000e+02
+1.060980000000000e+02 +2.786780000000000e+01
+1.302890000000000e+03 +5.013710000000000e+02
+7.901680000000000e+02 +2.855540000000001e+02
+7.631950000000001e+02 +2.584380000000000e+02
+2.823050000000000e+02 +8.352490000000000e+01
+1.456170000000000e+03 +4.311410000000000e+02
+7.547040000000000e+02 +2.426090000000000e+02
+7.909930000000001e+02 +2.887890000000000e+02
+1.297800000000000e+03 +4.917030000000000e+02
+6.933710000000002e+02 +2.725160000000000e+02
+9.249500000000000e+02 +3.839410000000000e+02
+7.801940000000000e+02 +2.729440000000000e+02
+8.834130000000000e+02 +3.828890000000000e+02
+1.298730000000000e+03 +4.649690000000000e+02
+7.759910000000001e+02 +2.564820000000000e+02
+8.806519999999998e+02 +3.481150000000000e+02
+6.968750000000000e+02 +2.377600000000000e+02
+7.759240000000000e+02 +2.429100000000000e+02
+1.302440000000000e+03 +4.476800000000000e+02
+1.452760000000000e+03 +3.722310000000000e+02
+7.515490000000000e+02 +2.524820000000000e+02
+6.913510000000001e+02 +2.174360000000000e+02
+1.117180000000000e+03 +4.888190000000000e+02
+5.028980000000000e+02 +2.674730000000000e+02
+9.806870000000000e+02 +3.238900000000000e+02
+7.840060000000002e+02 +2.365070000000000e+02
+1.300960000000000e+03 +4.678840000000000e+02
+8.795139999999999e+02 +3.206250000000000e+02
+6.945290000000000e+02 +2.615330000000000e+02
+8.587420000000000e+02 +3.846170000000000e+02
+4.934060000000000e+02 +2.465080000000000e+02
+7.803110000000000e+02 +2.604250000000000e+02
+6.911710000000000e+02 +1.942270000000000e+02
+1.127300000000000e+03 +4.552360000000000e+02
+7.687719999999998e+02 +2.459010000000000e+02
+1.320000000000000e+03 +4.348400000000000e+02
+7.873800000000000e+02 +2.316250000000000e+02
+1.305480000000000e+03 +4.778030000000001e+02
+7.809030000000000e+02 +2.522230000000000e+02
+6.966150000000000e+02 +1.949530000000000e+02
+1.740760000000000e+03 +5.946759999999998e+02
+5.019030000000000e+02 +2.522320000000000e+02
+1.319260000000000e+03 +4.923850000000000e+02
+7.525670000000000e+02 +2.353690000000000e+02
+8.883290000000000e+02 +3.230620000000000e+02
+6.910830000000002e+02 +2.216100000000000e+02
+3.260670000000000e+02 +1.601030000000000e+02
+1.145340000000000e+03 +4.905550000000000e+02
+7.732830000000000e+02 +2.532120000000000e+02
+8.810700000000001e+02 +3.243320000000000e+02
+1.066040000000000e+03 +3.697380000000001e+02
+8.709019999999998e+02 +3.047060000000000e+02
+1.311690000000000e+03 +4.745280000000000e+02
+7.828339999999999e+02 +2.390210000000000e+02
+1.310600000000000e+03 +4.826490000000000e+02
+6.892270000000000e+02 +2.130970000000000e+02
+1.123510000000000e+03 +4.821150000000000e+02
+1.865110000000000e+03 +6.684119999999998e+02
+4.862930000000000e+02 +2.419440000000000e+02
+7.838969999999998e+02 +2.388140000000000e+02
+1.089610000000000e+03 +3.769420000000000e+02
+1.058130000000000e+03 +3.558960000000000e+02
+6.942960000000000e+02 +2.169750000000000e+02
+7.863689999999998e+02 +2.588560000000000e+02
+9.086070000000000e+02 +3.357619999999999e+02
+1.087410000000000e+02 +3.121650000000000e+01
+6.573580000000002e+02 +2.476140000000000e+02
+6.961310000000002e+02 +2.098890000000000e+02
+9.736760000000000e+02 +3.034740000000000e+02
+1.314740000000000e+03 +4.748360000000000e+02
+7.895410000000001e+02 +2.496460000000000e+02
+1.043750000000000e+03 +3.240390000000000e+02
+1.298640000000000e+03 +4.718310000000000e+02
+9.086830000000000e+02 +3.228120000000000e+02
+1.117310000000000e+03 +3.799630000000000e+02
+1.869350000000000e+03 +6.487950000000000e+02
+7.766100000000000e+02 +2.204750000000000e+02
+8.692689999999999e+02 +3.722850000000000e+02
+4.772770000000000e+02 +2.272810000000000e+02
+2.813530000000000e+02 +1.801460000000000e+02
+9.768620000000000e+02 +2.707090000000000e+02
+1.849970000000000e+03 +6.744720000000000e+02
+7.788150000000001e+02 +2.287980000000000e+02
+1.023080000000000e+03 +3.675130000000000e+02
+1.046790000000000e+03 +3.822670000000000e+02
+6.882669999999998e+02 +2.040830000000000e+02
+9.527990000000000e+02 +4.247590000000000e+02
+1.738420000000000e+03 +6.599050000000000e+02
+1.143130000000000e+03 +4.575230000000000e+02
+4.930510000000000e+02 +2.164140000000000e+02
+9.935050000000000e+02 +3.636420000000000e+02
+7.687100000000000e+02 +2.291430000000000e+02
+8.810460000000000e+02 +3.265620000000000e+02
+1.479690000000000e+03 +6.719989999999998e+02
+1.296690000000000e+03 +4.723620000000000e+02
+8.635300000000000e+02 +2.764840000000000e+02
+6.507680000000000e+02 +2.422750000000000e+02
+1.741680000000000e+03 +6.321490000000000e+02
+1.474000000000000e+03 +5.499690000000001e+02
+7.784440000000000e+02 +2.337260000000000e+02
+6.935110000000002e+02 +1.628300000000000e+02
+1.772890000000000e+03 +6.120450000000000e+02
+1.592950000000000e+03 +6.829630000000002e+02
+7.003230000000000e+02 +2.367150000000000e+02
+9.775040000000000e+02 +2.944120000000001e+02
+4.910370000000000e+02 +2.298220000000000e+02
+1.299320000000000e+03 +4.366720000000000e+02
+1.317820000000000e+03 +4.532740000000000e+02
+1.088910000000000e+03 +3.464910000000000e+02
+1.093870000000000e+03 +3.142670000000000e+02
+6.749720000000000e+02 +2.319080000000000e+02
+1.860850000000000e+03 +7.316560000000002e+02
+1.477290000000000e+03 +4.942240000000000e+02
+7.511200000000000e+02 +2.122420000000000e+02
+9.526070000000000e+02 +3.336130000000001e+02
+7.932650000000000e+02 +2.385140000000000e+02
+1.299200000000000e+03 +4.290490000000000e+02
+1.299810000000000e+03 +4.206350000000000e+02
+1.107530000000000e+03 +3.677580000000000e+02
+1.876380000000000e+03 +7.535760000000000e+02
+7.855810000000000e+02 +2.411340000000000e+02
+9.455790000000000e+02 +2.900800000000000e+02
+6.288020000000000e+02 +4.166770000000000e+02
+1.128610000000000e+03 +4.502650000000000e+02
+1.082660000000000e+03 +3.463550000000000e+02
+1.588120000000000e+03 +6.508670000000000e+02
+6.943980000000000e+02 +1.782670000000000e+02
+6.678670000000000e+02 +2.524940000000000e+02
+4.995450000000000e+02 +2.054250000000000e+02
+7.740030000000000e+02 +2.218950000000000e+02
+9.075590000000000e+02 +3.031440000000000e+02
+9.531369999999999e+02 +4.502480000000001e+02
+6.895290000000000e+02 +1.720200000000000e+02
+1.108310000000000e+03 +4.344830000000000e+02
+6.930069999999999e+02 +3.464350000000000e+02
+7.446799999999999e+02 +2.044140000000000e+02
+7.469330000000000e+02 +2.432890000000000e+02
+4.837150000000000e+02 +2.227030000000000e+02
+7.911280000000000e+02 +2.269450000000000e+02
+1.081920000000000e+03 +3.034520000000000e+02
+1.605010000000000e+03 +6.284090000000000e+02
+1.357230000000000e+03 +5.718690000000000e+02
+1.310430000000000e+03 +4.419080000000000e+02
+1.324830000000000e+03 +4.053820000000000e+02
+1.926020000000000e+03 +6.567589999999999e+02
+9.281880000000000e+02 +4.304650000000000e+02
+4.676780000000001e+02 +1.965340000000000e+02
+9.858960000000000e+02 +3.329340000000000e+02
+7.409370000000000e+02 +2.296060000000000e+02
+7.734470000000000e+02 +2.329100000000000e+02
+1.268240000000000e+03 +5.233290000000002e+02
+1.155820000000000e+03 +4.512960000000000e+02
+7.706740000000000e+02 +2.244570000000000e+02
+1.307420000000000e+03 +3.820430000000000e+02
+1.064450000000000e+03 +3.180720000000000e+02
+2.664320000000000e+03 +9.151440000000000e+02
+1.302430000000000e+03 +4.035020000000000e+02
+4.783400000000000e+02 +1.752160000000000e+02
+7.800850000000000e+02 +2.454820000000000e+02
+7.796500000000000e+02 +2.296790000000000e+02
+9.678890000000000e+02 +3.408960000000000e+02
+1.335700000000000e+03 +5.713380000000002e+02
+1.327980000000000e+03 +5.542380000000001e+02
+7.461160000000001e+02 +2.177320000000000e+02
+5.053580000000000e+02 +2.077530000000000e+02
+9.253600000000000e+02 +2.887640000000000e+02
+7.834850000000000e+02 +2.381700000000000e+02
+1.180290000000000e+03 +4.497540000000000e+02
+7.047420000000000e+02 +3.335719999999999e+02
+4.965020000000000e+02 +1.970660000000000e+02
+9.878590000000000e+02 +3.149710000000000e+02
+1.160730000000000e+03 +4.841970000000000e+02
+7.896220000000000e+02 +2.315300000000000e+02
+8.934160000000001e+02 +2.725710000000000e+02
+8.781270000000000e+02 +2.839680000000000e+02
+1.302380000000000e+03 +5.684550000000000e+02
+7.815069999999999e+02 +2.190010000000000e+02
+1.295760000000000e+03 +3.901720000000000e+02
+9.010490000000000e+02 +3.018110000000000e+02
+4.838840000000000e+02 +1.529500000000000e+02
+7.540970000000000e+02 +2.167610000000000e+02
+7.009450000000001e+02 +3.664780000000000e+02
+9.685060000000000e+02 +3.297910000000000e+02
+1.328300000000000e+03 +6.295480000000000e+02
+7.840280000000000e+02 +2.273770000000000e+02
+7.873190000000000e+02 +2.225830000000000e+02
+4.849410000000000e+02 +2.145460000000000e+02
+4.966870000000000e+02 +2.096130000000000e+02
+1.326630000000000e+03 +6.261900000000001e+02
+7.412410000000001e+02 +1.983320000000000e+02
+8.776339999999999e+02 +2.586840000000000e+02
+9.429680000000000e+02 +4.442260000000000e+02
+7.806750000000000e+02 +2.268300000000000e+02
+7.973110000000000e+02 +2.268730000000000e+02
+9.818720000000000e+02 +3.313180000000001e+02
+1.394480000000000e+03 +5.361400000000000e+02
+6.747070000000000e+02 +3.800190000000000e+02
+1.300440000000000e+03 +3.937840000000000e+02
+9.281079999999999e+02 +2.683590000000000e+02
+1.130400000000000e+03 +4.256870000000000e+02
+8.331880000000000e+02 +2.402100000000000e+02
+7.461260000000002e+02 +1.943020000000000e+02
+7.790700000000001e+02 +2.330310000000000e+02
+1.156140000000000e+03 +5.291070000000000e+02
+1.113450000000000e+03 +5.084820000000000e+02
+7.718720000000000e+02 +2.068930000000000e+02
+7.744080000000000e+02 +2.151910000000000e+02
+9.917950000000000e+02 +4.302010000000000e+02
+1.151070000000000e+03 +4.561000000000000e+02
+6.762050000000000e+02 +3.940610000000000e+02
+1.320380000000000e+03 +4.586250000000000e+02
+9.095440000000000e+02 +2.671470000000000e+02
+9.057220000000000e+02 +2.533930000000000e+02
+9.547410000000000e+02 +4.524110000000000e+02
+1.363900000000000e+03 +5.216690000000000e+02
+9.207040000000000e+02 +3.920250000000000e+02
+7.726139999999998e+02 +2.135980000000000e+02
+7.781080000000002e+02 +2.146430000000000e+02
+6.249190000000000e+02 +4.280980000000000e+02
+7.854440000000000e+02 +2.191560000000000e+02
+9.124960000000000e+02 +3.005280000000000e+02
+9.716130000000001e+02 +3.302090000000000e+02
+1.297200000000000e+03 +6.489660000000000e+02
+1.339670000000000e+03 +5.199460000000000e+02
+1.630170000000000e+03 +7.272760000000002e+02
+7.841500000000000e+02 +2.190500000000000e+02
+4.696520000000000e+02 +1.526880000000000e+02
+6.930750000000000e+02 +3.806780000000001e+02
+8.421970000000000e+02 +2.321840000000000e+02
+7.546339999999999e+02 +1.676830000000000e+02
+1.077720000000000e+03 +3.914540000000000e+02
+8.604510000000000e+02 +3.083090000000000e+02
+1.315430000000000e+03 +6.065350000000000e+02
+9.363180000000000e+02 +4.334920000000000e+02
+1.037380000000000e+03 +3.899400000000000e+02
+1.304180000000000e+03 +5.812880000000000e+02
+1.063050000000000e+03 +3.646780000000001e+02
+1.017230000000000e+03 +3.236410000000000e+02
+9.074660000000000e+02 +2.539580000000000e+02
+8.673550000000000e+02 +3.037890000000000e+02
+6.913060000000000e+02 +3.253630000000000e+02
+1.877730000000000e+03 +6.875260000000002e+02
+1.854860000000000e+03 +7.004360000000000e+02
+1.004200000000000e+03 +4.283430000000000e+02
+1.613220000000000e+03 +7.370710000000000e+02
+1.849990000000000e+03 +6.927100000000000e+02
+6.979730000000002e+02 +2.738210000000000e+02
+6.424360000000000e+02 +2.829650000000000e+02
+9.990690000000000e+02 +4.383200000000000e+02
+8.419820000000000e+02 +2.806950000000000e+02
+9.140510000000000e+02 +2.599830000000000e+02
+6.546350000000000e+02 +3.950050000000000e+02
+1.483600000000000e+03 +4.872590000000000e+02
+6.640089999999999e+02 +2.523810000000000e+02
+8.803150000000001e+02 +2.587350000000000e+02
+9.153869999999999e+02 +3.255770000000000e+02
+1.854800000000000e+03 +6.505680000000000e+02
+6.855800000000000e+02 +3.005640000000000e+02
+6.405970000000000e+02 +2.643750000000000e+02
+6.731000000000000e+02 +2.734260000000000e+02
+1.835600000000000e+03 +7.083680000000001e+02
+9.319990000000000e+02 +3.823910000000000e+02
+1.007510000000000e+03 +3.803600000000000e+02
+6.472160000000000e+02 +3.919710000000000e+02
+6.456200000000000e+02 +2.892850000000000e+02
+1.005830000000000e+03 +3.693000000000000e+02
+1.272240000000000e+03 +5.309550000000000e+02
+1.106860000000000e+03 +5.699900000000000e+02
+6.572170000000000e+02 +2.481590000000000e+02
+1.888230000000000e+03 +6.396609999999999e+02
+1.529770000000000e+03 +6.296619999999998e+02
+8.913919999999998e+02 +2.484690000000000e+02
+6.379410000000000e+02 +2.838960000000000e+02
+8.876950000000001e+02 +4.008100000000000e+02
+8.630169999999998e+02 +2.888460000000000e+02
+6.609989999999998e+02 +4.031860000000000e+02
+6.925280000000000e+02 +3.095710000000000e+02
+6.620889999999998e+02 +2.899280000000000e+02
+2.769680000000000e+02 +8.288549999999999e+01
+6.381920000000000e+02 +2.650530000000000e+02
+9.287180000000000e+02 +3.910460000000000e+02
+8.857719999999998e+02 +2.343820000000000e+02
+8.989019999999998e+02 +4.175260000000000e+02
+1.338400000000000e+03 +5.273660000000000e+02
+1.232720000000000e+03 +4.137370000000000e+02
+6.369040000000000e+02 +2.703460000000000e+02
+6.536910000000000e+02 +2.609760000000000e+02
+1.137900000000000e+03 +5.716519999999998e+02
+7.908589999999998e+02 +2.876300000000000e+02
+6.853570000000000e+02 +2.688300000000000e+02
+6.321090000000000e+02 +2.467440000000000e+02
+1.105400000000000e+03 +5.157480000000000e+02
+9.393350000000000e+02 +3.952510000000000e+02
+2.650270000000000e+02 +1.571510000000000e+02
+1.163130000000000e+03 +5.415880000000002e+02
+1.154490000000000e+03 +4.143680000000001e+02
+8.933670000000000e+02 +2.236330000000000e+02
+6.405190000000000e+02 +2.585070000000000e+02
+6.349510000000000e+02 +2.376380000000000e+02
+1.138260000000000e+03 +5.001690000000000e+02
+7.080630000000000e+02 +3.903700000000000e+02
+6.177800000000000e+02 +2.553260000000000e+02
+1.140440000000000e+03 +4.914800000000000e+02
+1.084160000000000e+03 +5.139660000000000e+02
+1.134130000000000e+03 +4.399840000000000e+02
+1.036450000000000e+03 +3.707470000000000e+02
+6.348450000000000e+02 +2.318160000000000e+02
+8.653869999999999e+02 +3.667860000000000e+02
+1.073740000000000e+03 +5.266740000000000e+02
+6.344550000000000e+02 +2.313230000000000e+02
+9.493579999999999e+02 +3.988870000000000e+02
+1.134500000000000e+03 +4.829200000000000e+02
+6.165790000000002e+02 +2.100900000000000e+02
+2.027940000000000e+03 +7.777470000000000e+02
+1.101620000000000e+03 +4.274440000000000e+02
+9.007420000000000e+02 +2.403890000000000e+02
+8.113600000000000e+02 +2.767460000000000e+02
+1.469470000000000e+03 +5.007090000000000e+02
+1.101530000000000e+03 +3.884090000000000e+02
+6.427660000000000e+02 +2.592690000000000e+02
+6.494310000000000e+02 +2.303880000000000e+02
+6.511920000000000e+02 +2.001990000000000e+02
+1.939750000000000e+03 +7.689150000000000e+02
+1.173610000000000e+03 +4.743700000000000e+02
+6.726469999999998e+02 +2.185950000000000e+02
+9.042060000000000e+02 +2.257480000000000e+02
+9.020580000000000e+02 +2.288200000000000e+02
+1.070140000000000e+03 +4.508510000000000e+02
+6.354510000000000e+02 +2.457990000000000e+02
+6.346669999999998e+02 +2.229420000000000e+02
+6.180820000000000e+02 +1.723590000000000e+02
+9.131930000000000e+02 +2.904700000000000e+02
+7.522189999999998e+02 +2.796510000000000e+02
+6.463730000000000e+02 +2.238910000000000e+02
+6.137600000000000e+02 +1.890270000000000e+02
+6.371580000000000e+02 +1.845730000000000e+02
+6.871239999999998e+02 +3.633630000000001e+02
+1.378100000000000e+03 +6.029900000000000e+02
+1.117270000000000e+03 +4.115190000000000e+02
+6.356730000000000e+02 +2.422380000000000e+02
+6.524770000000000e+02 +1.738350000000000e+02
+1.887100000000000e+03 +6.894260000000000e+02
+9.141620000000000e+02 +2.381520000000000e+02
+6.551000000000000e+02 +1.952950000000000e+02
+1.103560000000000e+03 +4.055790000000000e+02
+1.081330000000000e+03 +4.740350000000000e+02
+1.563610000000000e+03 +5.938009999999998e+02
+6.801289999999998e+02 +2.975640000000000e+02
+6.382400000000000e+02 +2.206430000000000e+02
+6.352769999999998e+02 +1.817970000000000e+02
+1.157350000000000e+03 +5.306680000000000e+02
+6.406880000000000e+02 +1.700640000000000e+02
+1.862040000000000e+03 +6.780060000000002e+02
+1.572440000000000e+03 +6.164130000000000e+02
+1.698550000000000e+03 +5.294720000000000e+02
+7.046950000000001e+02 +3.725020000000000e+02
+6.455459999999998e+02 +1.683010000000000e+02
+6.274760000000000e+02 +2.128170000000000e+02
+6.687200000000000e+02 +1.734870000000000e+02
+1.564350000000000e+03 +5.387360000000000e+02
+1.542500000000000e+03 +4.890840000000000e+02
+1.656430000000000e+03 +5.640570000000000e+02
+1.046760000000000e+03 +4.008270000000000e+02
+1.322300000000000e+03 +4.923810000000000e+02
+6.926210000000002e+02 +3.978380000000000e+02
+6.397250000000000e+02 +2.049590000000000e+02
+6.634190000000000e+02 +1.664350000000000e+02
+9.050839999999999e+02 +2.150680000000000e+02
+1.183240000000000e+02 +5.612080000000000e+01
+1.567630000000000e+03 +5.520280000000000e+02
+1.047870000000000e+03 +3.977810000000000e+02
+1.086600000000000e+02 +4.115340000000000e+01
+1.276790000000000e+03 +5.446820000000000e+02
+6.498480000000002e+02 +1.689280000000000e+02
+1.622100000000000e+03 +7.019540000000000e+02
+1.692850000000000e+03 +5.122970000000000e+02
+6.730020000000000e+02 +2.289740000000000e+02
+1.874510000000000e+03 +7.409839999999998e+02
+1.250430000000000e+03 +5.119430000000000e+02
+1.077740000000000e+02 +3.918090000000000e+01
+1.561130000000000e+03 +5.384570000000000e+02
+1.531120000000000e+03 +5.176390000000000e+02
+1.236450000000000e+03 +5.431319999999999e+02
+1.566060000000000e+03 +5.086710000000000e+02
+1.568280000000000e+03 +4.568580000000000e+02
+1.092960000000000e+03 +4.836620000000000e+02
+6.781330000000000e+02 +3.346800000000000e+02
+7.075760000000000e+02 +3.278900000000000e+02
+8.951050000000000e+02 +4.469210000000000e+02
+6.704960000000002e+02 +2.229570000000000e+02
+8.946519999999998e+02 +2.202490000000000e+02
+7.911100000000000e+02 +3.726260000000000e+02
+1.035960000000000e+03 +4.114090000000000e+02
+1.089900000000000e+03 +4.671080000000000e+02
+1.563580000000000e+03 +4.834700000000000e+02
+6.337050000000000e+02 +1.827020000000000e+02
+7.919190000000000e+02 +3.499110000000000e+02
+1.677170000000000e+03 +6.081640000000000e+02
+1.095660000000000e+03 +4.757270000000000e+02
+8.927050000000000e+02 +4.564210000000000e+02
+6.365020000000000e+02 +1.932830000000000e+02
+7.956080000000002e+02 +3.441440000000000e+02
+1.054980000000000e+03 +3.973350000000000e+02
+1.583980000000000e+03 +5.411290000000000e+02
+1.011400000000000e+03 +3.950100000000000e+02
+1.083730000000000e+03 +4.289690000000000e+02
+7.953960000000002e+02 +3.221760000000000e+02
+1.083580000000000e+03 +4.520010000000000e+02
+1.070730000000000e+03 +3.569760000000000e+02
+7.019980000000000e+02 +2.546450000000000e+02
+6.938900000000000e+02 +2.640850000000000e+02
+1.083220000000000e+03 +3.237410000000000e+02
+4.209520000000000e+02 +1.208030000000000e+02
+1.024020000000000e+03 +3.945670000000000e+02
+6.952189999999998e+02 +2.818100000000000e+02
+4.568890000000000e+02 +1.255240000000000e+02
+7.001799999999999e+02 +2.654560000000000e+02
+1.545710000000000e+03 +6.662960000000000e+02
+1.012590000000000e+03 +3.599860000000000e+02
+4.808260000000000e+02 +1.353520000000000e+02
+8.491849999999999e+02 +5.051970000000000e+02
+1.050930000000000e+03 +3.780620000000000e+02
+1.068210000000000e+03 +3.696520000000000e+02
+1.069460000000000e+03 +3.823900000000000e+02
+1.566090000000000e+03 +6.222460000000000e+02
+1.575840000000000e+03 +5.742150000000000e+02
+1.038970000000000e+03 +3.608290000000000e+02
+1.065480000000000e+03 +3.482170000000000e+02
+7.022530000000000e+02 +2.479830000000000e+02
+1.030780000000000e+03 +3.599150000000000e+02
+1.090310000000000e+03 +3.834420000000000e+02
+1.572950000000000e+03 +5.451380000000000e+02
+1.878480000000000e+03 +7.962070000000000e+02
+8.798460000000000e+02 +3.748830000000000e+02
+8.585660000000000e+02 +3.472450000000000e+02
+9.756100000000000e+02 +3.692550000000000e+02
+9.726830000000000e+02 +4.030780000000000e+02
+1.616440000000000e+03 +6.763099999999999e+02
+9.682200000000000e+02 +2.198020000000000e+02
+8.606870000000000e+02 +3.398600000000000e+02
+1.065900000000000e+03 +3.403590000000001e+02
+6.920790000000000e+02 +3.815480000000000e+02
+7.088780000000000e+02 +3.603890000000000e+02
+1.127380000000000e+03 +4.766730000000000e+02
+1.490320000000000e+03 +7.056060000000001e+02
+1.016800000000000e+03 +2.948960000000000e+02
+1.476330000000000e+03 +5.952750000000000e+02
+1.499540000000000e+03 +6.715290000000000e+02
+1.062450000000000e+03 +3.764960000000000e+02
+9.053700000000000e+02 +4.170240000000000e+02
+1.815440000000000e+03 +8.275269999999998e+02
+8.689960000000002e+02 +3.703840000000000e+02
+7.806760000000000e+02 +3.343780000000001e+02
+2.047940000000000e+03 +8.385310000000002e+02
+9.674220000000000e+02 +2.693490000000000e+02
+9.941210000000000e+02 +3.629120000000000e+02
+7.961500000000000e+02 +2.932540000000000e+02
+9.962809999999999e+02 +4.861160000000000e+02
+4.567290000000000e+02 +1.175040000000000e+02
+7.961380000000000e+02 +2.983010000000000e+02
+2.031180000000000e+03 +6.721330000000000e+02
+1.017330000000000e+03 +4.421420000000000e+02
+1.298220000000000e+03 +6.122230000000002e+02
+9.901849999999999e+02 +4.962050000000000e+02
+8.506430000000000e+02 +2.917010000000000e+02
+2.818490000000000e+02 +1.777390000000000e+02
+2.794370000000000e+02 +1.729500000000000e+02
+9.799000000000000e+02 +3.691880000000001e+02
+9.782530000000000e+02 +3.523650000000000e+02
+6.855760000000000e+02 +2.120770000000000e+02
+1.030710000000000e+03 +4.049600000000000e+02
+1.481770000000000e+03 +7.594230000000000e+02
+1.311080000000000e+03 +6.092120000000000e+02
+8.084169999999998e+02 +3.096810000000000e+02
+1.027680000000000e+03 +4.947400000000000e+02
+1.011740000000000e+03 +4.265390000000000e+02
+1.040660000000000e+03 +3.303540000000001e+02
+9.983869999999999e+02 +4.203380000000000e+02
+2.840520000000000e+02 +1.687630000000000e+02
+2.812750000000000e+02 +1.645970000000000e+02
+1.285250000000000e+03 +6.128170000000000e+02
+9.906240000000000e+02 +3.787750000000000e+02
+1.007930000000000e+03 +4.291670000000000e+02
+1.990490000000000e+03 +7.801189999999998e+02
+1.369720000000000e+03 +5.750020000000000e+02
+1.045370000000000e+03 +3.762960000000000e+02
+9.881390000000000e+02 +3.105750000000000e+02
+9.153080000000000e+02 +4.104830000000000e+02
+1.010830000000000e+03 +4.597040000000000e+02
+1.139550000000000e+03 +3.641330000000000e+02
+1.011540000000000e+03 +3.067460000000000e+02
+2.817420000000000e+02 +1.489050000000000e+02
+1.375090000000000e+03 +6.083570000000000e+02
+9.830230000000000e+02 +3.204200000000000e+02
+9.535510000000000e+02 +3.146880000000000e+02
+1.403960000000000e+03 +6.830440000000000e+02
+1.406010000000000e+03 +6.471530000000000e+02
+6.232959999999998e+02 +2.673260000000000e+02
+1.279970000000000e+03 +4.987340000000000e+02
+6.678780000000000e+02 +1.575570000000000e+02
+1.371980000000000e+03 +5.571950000000001e+02
+9.162610000000000e+02 +4.087980000000000e+02
+6.207010000000000e+02 +2.515470000000000e+02
+1.307410000000000e+03 +5.342450000000000e+02
+6.268740000000000e+02 +2.569520000000000e+02
+6.347530000000000e+02 +2.647680000000000e+02
+6.622960000000000e+02 +2.526470000000000e+02
+9.949180000000000e+02 +3.673400000000000e+02
+1.059760000000000e+03 +3.775520000000000e+02
+2.818570000000000e+02 +1.254840000000000e+02
+1.423080000000000e+03 +6.156280000000000e+02
+1.398540000000000e+03 +6.410800000000000e+02
+9.928080000000000e+02 +4.111040000000000e+02
+6.196380000000000e+02 +2.218470000000000e+02
+6.591369999999999e+02 +2.380810000000000e+02
+8.805710000000000e+02 +2.895000000000000e+02
+8.953339999999999e+02 +2.986100000000000e+02
+1.288980000000000e+03 +4.836340000000000e+02
+9.854000000000000e+02 +3.390010000000000e+02
+9.256750000000000e+02 +3.722370000000000e+02
+1.006670000000000e+03 +3.653580000000000e+02
+2.818420000000000e+02 +1.150390000000000e+02
+1.414440000000000e+03 +6.783350000000000e+02
+9.866470000000000e+02 +3.402560000000000e+02
+1.402850000000000e+03 +5.788300000000000e+02
+6.251160000000000e+02 +2.497120000000000e+02
+1.063560000000000e+03 +3.791940000000000e+02
+1.021990000000000e+03 +3.639110000000000e+02
+9.065110000000000e+02 +3.860850000000000e+02
+6.190060000000000e+02 +2.394800000000000e+02
+8.989870000000000e+02 +3.033590000000001e+02
+1.094400000000000e+03 +3.670520000000000e+02
+6.620239999999999e+02 +2.308160000000000e+02
+7.895980000000002e+02 +2.591470000000000e+02
+9.897250000000000e+02 +3.286010000000000e+02
+9.003020000000000e+02 +2.846890000000000e+02
+9.824800000000000e+02 +2.932480000000000e+02
+2.813320000000000e+02 +1.312080000000000e+02
+1.032050000000000e+03 +3.684380000000001e+02
+6.396390000000000e+02 +2.263500000000000e+02
+8.882760000000002e+02 +2.855320000000000e+02
+1.845520000000000e+03 +7.487110000000000e+02
+9.130590000000000e+02 +3.635290000000000e+02
+1.014450000000000e+03 +3.338860000000000e+02
+6.346799999999999e+02 +2.350330000000000e+02
+6.580650000000001e+02 +1.691880000000000e+02
+9.955490000000000e+02 +3.220240000000000e+02
+1.848280000000000e+03 +7.213930000000000e+02
+8.966980000000000e+02 +2.655410000000000e+02
+2.842320000000000e+02 +1.474350000000000e+02
+6.253780000000000e+02 +1.990020000000000e+02
+9.056240000000000e+02 +3.048970000000000e+02
+1.038670000000000e+03 +3.090230000000000e+02
+1.849460000000000e+03 +7.105740000000000e+02
+6.226990000000002e+02 +2.141070000000000e+02
+1.858830000000000e+03 +6.584280000000000e+02
+1.075460000000000e+03 +3.390270000000000e+02
+1.493320000000000e+03 +6.740920000000000e+02
+6.609000000000000e+02 +1.986670000000000e+02
+2.788510000000000e+02 +1.162850000000000e+02
+9.766050000000000e+02 +2.890020000000000e+02
+1.768110000000000e+03 +7.626400000000000e+02
+1.110570000000000e+03 +3.236370000000000e+02
+1.060060000000000e+03 +3.829550000000000e+02
+1.745590000000000e+03 +6.954950000000000e+02
+6.249910000000000e+02 +1.782200000000000e+02
+6.182780000000000e+02 +2.123520000000000e+02
+1.063090000000000e+02 +2.923890000000000e+01
+1.837250000000000e+03 +7.660920000000000e+02
+6.605470000000000e+02 +1.633090000000000e+02
+8.953120000000000e+02 +2.712570000000000e+02
+9.745119999999999e+02 +2.948010000000000e+02
+8.196690000000000e+02 +2.308980000000000e+02
+9.985380000000000e+02 +3.321360000000000e+02
+9.718220000000000e+02 +3.335840000000000e+02
+6.239670000000000e+02 +1.751000000000000e+02
+6.209680000000002e+02 +1.909790000000000e+02
+1.201700000000000e+03 +5.056390000000000e+02
+2.885880000000000e+02 +1.362550000000000e+02
+1.876810000000000e+03 +7.576139999999998e+02
+1.313360000000000e+03 +5.387780000000000e+02
+7.817270000000000e+02 +3.023720000000000e+02
+1.107760000000000e+03 +3.635690000000000e+02
+1.865340000000000e+03 +7.494230000000000e+02
+9.415140000000000e+02 +3.475360000000000e+02
+6.281010000000000e+02 +1.969650000000000e+02
+7.752370000000000e+02 +3.028570000000000e+02
+9.087350000000000e+02 +2.623310000000000e+02
+1.067460000000000e+03 +3.424490000000000e+02
+1.440450000000000e+03 +6.063900000000000e+02
+6.244900000000000e+02 +1.627690000000000e+02
+1.851990000000000e+03 +7.415030000000000e+02
+1.867790000000000e+03 +7.530580000000000e+02
+9.890940000000001e+02 +4.000190000000000e+02
+9.594310000000000e+02 +4.180430000000000e+02
+1.446790000000000e+03 +6.158710000000000e+02
+1.418460000000000e+03 +5.213440000000001e+02
+7.886990000000000e+02 +2.911840000000000e+02
+1.871100000000000e+03 +7.131510000000002e+02
+9.087809999999999e+02 +3.372850000000000e+02
+7.723360000000000e+02 +2.888370000000000e+02
+1.130740000000000e+03 +3.355910000000000e+02
+9.771070000000000e+02 +2.949110000000000e+02
+6.757260000000001e+02 +3.158030000000000e+02
+1.846280000000000e+03 +7.356350000000000e+02
+1.409370000000000e+03 +4.893220000000000e+02
+1.063760000000000e+03 +3.752240000000000e+02
+7.685620000000000e+02 +2.859360000000000e+02
+9.093180000000000e+02 +2.820210000000000e+02
+1.086500000000000e+02 +3.453150000000000e+01
+7.802130000000002e+02 +2.695050000000000e+02
+8.734310000000000e+02 +2.766170000000000e+02
+1.123880000000000e+03 +3.615000000000000e+02
+2.672500000000000e+02 +1.115860000000000e+02
+6.631010000000001e+02 +3.129650000000000e+02
+1.614740000000000e+03 +6.751519999999998e+02
+9.760960000000000e+02 +2.888010000000000e+02
+7.854800000000000e+02 +2.842300000000000e+02
+1.011070000000000e+03 +2.643760000000000e+02
+8.978099999999999e+02 +4.855860000000000e+02
+7.911319999999999e+02 +2.653560000000000e+02
+3.454450000000000e+02 +1.737670000000000e+02
+4.130720000000000e+02 +1.958320000000000e+02
+6.800920000000000e+02 +3.059040000000000e+02
+8.592500000000000e+02 +2.368980000000000e+02
+6.922689999999999e+02 +2.935460000000000e+02
+6.892189999999998e+02 +2.662090000000000e+02
+9.836150000000000e+02 +2.837370000000000e+02
+4.922920000000000e+02 +2.890500000000000e+02
+1.590920000000000e+03 +5.657850000000000e+02
+9.025230000000000e+02 +4.449930000000001e+02
+7.853850000000000e+02 +2.829740000000000e+02
+3.280780000000001e+02 +1.585800000000000e+02
+7.826870000000000e+02 +2.741810000000000e+02
+8.750549999999999e+02 +3.325080000000001e+02
+6.925730000000000e+02 +2.550020000000000e+02
+6.904040000000000e+02 +2.112950000000000e+02
+9.834170000000000e+02 +2.936410000000000e+02
+7.773240000000000e+02 +2.589390000000000e+02
+4.884930000000001e+02 +2.895040000000000e+02
+7.874760000000001e+02 +2.604370000000000e+02
+7.708270000000000e+02 +2.650710000000000e+02
+8.936419999999998e+02 +4.616010000000000e+02
+8.842919999999998e+02 +4.694040000000000e+02
+7.843830000000000e+02 +2.469940000000000e+02
+7.880350000000000e+02 +2.721320000000000e+02
+1.526630000000000e+03 +4.849570000000000e+02
+1.299230000000000e+03 +4.593200000000000e+02
+7.693389999999998e+02 +2.659570000000000e+02
+1.119600000000000e+03 +3.656120000000000e+02
+1.750860000000000e+03 +7.723160000000000e+02
+9.907050000000000e+02 +3.008890000000000e+02
+7.811730000000000e+02 +2.464890000000000e+02
+7.773969999999998e+02 +2.435440000000000e+02
+1.060610000000000e+03 +3.615380000000000e+02
+7.788290000000000e+02 +3.356200000000000e+02
+9.897470000000000e+02 +3.004790000000001e+02
+8.976760000000000e+02 +4.610280000000000e+02
+8.901669999999998e+02 +2.792350000000000e+02
+9.913660000000000e+02 +2.926380000000000e+02
+1.085680000000000e+03 +5.039750000000000e+02
+5.073800000000000e+02 +2.930520000000000e+02
+7.755820000000000e+02 +2.480100000000000e+02
+1.607480000000000e+03 +6.663760000000002e+02
+7.663700000000000e+02 +2.516190000000000e+02
+1.072270000000000e+02 +3.129950000000000e+01
+7.571760000000000e+02 +2.249550000000000e+02
+1.037240000000000e+03 +3.590100000000000e+02
+1.505810000000000e+03 +3.888090000000000e+02
+7.710510000000000e+02 +2.406960000000000e+02
+9.066390000000000e+02 +2.597000000000000e+02
+7.509670000000000e+02 +3.698890000000000e+02
+9.732340000000000e+02 +2.800710000000000e+02
+9.980839999999999e+02 +2.861050000000000e+02
+4.945080000000000e+02 +2.933390000000000e+02
+7.826350000000000e+02 +2.576930000000000e+02
+1.072110000000000e+03 +3.548950000000000e+02
+8.803539999999998e+02 +3.479770000000000e+02
+1.609130000000000e+03 +6.306150000000000e+02
+7.640720000000000e+02 +2.348070000000000e+02
+1.080420000000000e+03 +3.544980000000001e+02
+1.911300000000000e+03 +5.915309999999999e+02
+9.127450000000000e+02 +2.309370000000000e+02
+1.151240000000000e+03 +4.826570000000000e+02
+6.397310000000000e+02 +2.143210000000000e+02
+1.754020000000000e+03 +7.421750000000000e+02
+1.558910000000000e+03 +4.995360000000000e+02
+7.831460000000002e+02 +2.504030000000000e+02
+7.692810000000002e+02 +2.359770000000000e+02
+5.057190000000000e+02 +2.939440000000000e+02
+7.917700000000000e+02 +2.439990000000000e+02
+1.336500000000000e+03 +4.396610000000000e+02
+7.933660000000001e+02 +2.502220000000000e+02
+1.425960000000000e+03 +5.010510000000000e+02
+7.657060000000000e+02 +2.231340000000000e+02
+8.656640000000000e+02 +3.202470000000000e+02
+1.385240000000000e+03 +4.088510000000000e+02
+1.109990000000000e+03 +4.903260000000000e+02
+8.949180000000000e+02 +2.321880000000000e+02
+8.993090000000000e+02 +2.314500000000000e+02
+4.881460000000000e+02 +2.933340000000000e+02
+1.435320000000000e+03 +5.033540000000000e+02
+1.132710000000000e+03 +4.024820000000000e+02
+7.681310000000002e+02 +2.089480000000000e+02
+1.287190000000000e+03 +4.362530000000000e+02
+5.132320000000000e+02 +1.460200000000000e+02
+9.998890000000000e+02 +2.997320000000000e+02
+7.750610000000000e+02 +2.355060000000000e+02
+1.280690000000000e+03 +4.297820000000000e+02
+7.489180000000000e+02 +2.184070000000000e+02
+8.639780000000002e+02 +2.148250000000000e+02
+8.884480000000000e+02 +2.329730000000000e+02
+5.014090000000000e+02 +2.867170000000000e+02
+1.286150000000000e+03 +4.379640000000000e+02
+1.938060000000000e+03 +6.642890000000000e+02
+8.692719999999998e+02 +3.553720000000000e+02
+6.662600000000000e+02 +2.712000000000000e+02
+8.588730000000000e+02 +2.760870000000000e+02
+8.397200000000000e+02 +3.148130000000000e+02
+9.751090000000000e+02 +2.655490000000000e+02
+7.747550000000000e+02 +2.290480000000000e+02
+1.314690000000000e+03 +4.299020000000000e+02
+4.981690000000000e+02 +3.009070000000000e+02
+1.319540000000000e+03 +4.342430000000001e+02
+7.754030000000000e+02 +2.351240000000000e+02
+8.891000000000000e+02 +2.345920000000000e+02
+1.387980000000000e+03 +6.342330000000002e+02
+1.327840000000000e+03 +6.296680000000000e+02
+1.301530000000000e+03 +4.173140000000000e+02
+7.894050000000000e+02 +2.116550000000000e+02
+3.358610000000000e+02 +9.777350000000000e+01
+9.884450000000001e+02 +2.736160000000000e+02
+7.412530000000000e+02 +2.762200000000000e+02
+4.870380000000000e+02 +2.458600000000000e+02
+1.311890000000000e+03 +4.435870000000000e+02
+7.743980000000000e+02 +2.358090000000000e+02
+1.295980000000000e+03 +3.781290000000000e+02
+7.697050000000000e+02 +2.082590000000000e+02
+8.468630000000001e+02 +2.107010000000000e+02
+1.340240000000000e+03 +6.515830000000002e+02
+1.040700000000000e+03 +3.483500000000000e+02
+1.335790000000000e+03 +6.373060000000000e+02
+4.895490000000000e+02 +2.801290000000000e+02
+1.301820000000000e+03 +4.004030000000000e+02
+7.518400000000000e+02 +2.116850000000000e+02
+4.825490000000000e+02 +1.108810000000000e+02
+8.610250000000000e+02 +4.031110000000000e+02
+1.300750000000000e+03 +4.121970000000000e+02
+7.783570000000000e+02 +2.254660000000000e+02
+8.797980000000000e+02 +3.353670000000000e+02
+4.896410000000000e+02 +2.469170000000000e+02
+1.336120000000000e+03 +6.072850000000000e+02
+1.356340000000000e+03 +5.449550000000000e+02
+7.469119999999998e+02 +2.464070000000000e+02
+7.742910000000001e+02 +2.194350000000000e+02
+8.980660000000000e+02 +2.239670000000000e+02
+1.305760000000000e+03 +3.966040000000000e+02
+1.294020000000000e+03 +3.536820000000000e+02
+1.953250000000000e+03 +6.533150000000001e+02
+8.734470000000000e+02 +3.071310000000000e+02
+4.911120000000000e+02 +2.845290000000000e+02
+7.774160000000001e+02 +2.048270000000000e+02
+7.792130000000002e+02 +2.241200000000000e+02
+8.784700000000000e+02 +2.306170000000000e+02
+1.326910000000000e+03 +5.611340000000000e+02
+1.382900000000000e+03 +6.022350000000000e+02
+4.888930000000000e+02 +2.331210000000000e+02
+7.439390000000000e+02 +2.288070000000000e+02
+7.025910000000000e+02 +3.586500000000000e+02
+7.912710000000002e+02 +2.376460000000000e+02
+7.770680000000000e+02 +1.950130000000000e+02
+1.298160000000000e+03 +3.936440000000000e+02
+7.665970000000000e+02 +2.298220000000000e+02
+1.074150000000000e+03 +6.255509999999998e+02
+6.851060000000001e+02 +3.815500000000000e+02
+1.301660000000000e+03 +3.550350000000000e+02
+1.166040000000000e+03 +5.064520000000000e+02
+9.030180000000000e+02 +2.281020000000000e+02
+1.127710000000000e+03 +4.216850000000000e+02
+1.294960000000000e+03 +3.806100000000000e+02
+7.810280000000000e+02 +2.371700000000000e+02
+7.801650000000000e+02 +1.911640000000000e+02
+1.067790000000000e+03 +5.873170000000000e+02
+7.610419999999998e+02 +2.177810000000000e+02
+1.284280000000000e+03 +3.751020000000000e+02
+9.016650000000000e+02 +2.318140000000000e+02
+1.112970000000000e+02 +3.106550000000000e+01
+3.435480000000000e+02 +9.845550000000000e+01
+1.321040000000000e+03 +5.960180000000000e+02
+9.266190000000000e+02 +3.556680000000000e+02
+8.796920000000000e+02 +3.344900000000000e+02
+7.837700000000000e+02 +2.014760000000000e+02
+7.939720000000000e+02 +2.378270000000000e+02
+1.306270000000000e+03 +6.426810000000000e+02
+7.611180000000001e+02 +2.210330000000000e+02
+1.073210000000000e+03 +5.827000000000000e+02
+1.102880000000000e+03 +3.372540000000000e+02
+7.812270000000000e+02 +2.141240000000000e+02
+9.751900000000001e+02 +3.515250000000000e+02
+6.913610000000001e+02 +3.403950000000000e+02
+6.639430000000000e+02 +4.654380000000001e+02
+1.072530000000000e+03 +5.409330000000000e+02
+1.306800000000000e+03 +5.605060000000000e+02
+7.840369999999998e+02 +2.155910000000000e+02
+1.114430000000000e+03 +6.073819999999999e+02
+8.748739999999998e+02 +2.233230000000000e+02
+7.448630000000001e+02 +1.712330000000000e+02
+6.703260000000000e+02 +3.275630000000000e+02
+1.308300000000000e+03 +4.128070000000000e+02
+7.764920000000000e+02 +2.187860000000000e+02
+1.307640000000000e+03 +5.692460000000000e+02
+7.757040000000000e+02 +2.067080000000000e+02
+7.742030000000000e+02 +2.259260000000000e+02
+1.122050000000000e+03 +6.674320000000000e+02
+7.696460000000002e+02 +1.850470000000000e+02
+6.799000000000000e+02 +3.548930000000001e+02
+1.637890000000000e+03 +5.479990000000000e+02
+9.630230000000000e+02 +3.399410000000000e+02
+7.822970000000000e+02 +2.202250000000000e+02
+2.308980000000000e+03 +8.848040000000000e+02
+1.591350000000000e+03 +5.282400000000000e+02
+7.658630000000001e+02 +2.711830000000000e+02
+7.796550000000000e+02 +1.991170000000000e+02
+7.873260000000000e+02 +2.324690000000000e+02
+2.315500000000000e+03 +8.651460000000002e+02
+6.524670000000000e+02 +4.584670000000000e+02
+6.910010000000002e+02 +3.738830000000000e+02
+2.911920000000000e+02 +8.932989999999999e+01
+9.383850000000000e+02 +4.939360000000000e+02
+7.511770000000000e+02 +1.918720000000000e+02
+8.146810000000000e+02 +4.043840000000000e+02
+1.063950000000000e+03 +4.944400000000000e+02
+8.456910000000000e+02 +2.901520000000000e+02
+3.618580000000000e+02 +1.843120000000000e+02
+1.072540000000000e+03 +4.788650000000000e+02
+6.623300000000000e+02 +2.532210000000000e+02
+1.010150000000000e+03 +4.068460000000000e+02
+8.261250000000000e+02 +4.053610000000000e+02
+6.826100000000000e+02 +3.060900000000000e+02
+9.854660000000000e+02 +4.613420000000000e+02
+1.583060000000000e+03 +6.119140000000000e+02
+6.162710000000000e+02 +4.819480000000000e+02
+1.107940000000000e+03 +5.748360000000000e+02
+1.084470000000000e+03 +4.650470000000000e+02
+7.050650000000001e+02 +3.441850000000000e+02
+7.152760000000002e+02 +2.832620000000000e+02
+7.004770000000000e+02 +3.375250000000000e+02
+8.308689999999998e+02 +3.245130000000000e+02
+1.536290000000000e+03 +5.243770000000000e+02
+1.163520000000000e+03 +4.851710000000000e+02
+1.050090000000000e+03 +4.165750000000000e+02
+1.664660000000000e+03 +5.462300000000000e+02
+1.148160000000000e+03 +5.354710000000000e+02
+6.338180000000000e+02 +2.694640000000000e+02
+4.858720000000000e+02 +1.424660000000000e+02
+6.359870000000000e+02 +2.700680000000000e+02
+2.039270000000000e+03 +8.607000000000000e+02
+6.963160000000000e+02 +4.334700000000000e+02
+6.479190000000000e+02 +2.748100000000000e+02
+9.190309999999999e+02 +3.980240000000000e+02
+8.488430000000002e+02 +3.159920000000000e+02
+9.391710000000000e+02 +4.594360000000000e+02
+1.144970000000000e+03 +5.216530000000000e+02
+6.728150000000001e+02 +4.903320000000000e+02
+1.064680000000000e+03 +4.243440000000000e+02
+6.340190000000000e+02 +2.705940000000000e+02
+6.376770000000000e+02 +2.462770000000000e+02
+7.324739999999998e+02 +2.122860000000000e+02
+6.624100000000000e+02 +2.853730000000000e+02
+9.289190000000000e+02 +3.921100000000000e+02
+6.807750000000000e+02 +4.217980000000000e+02
+6.588030000000000e+02 +2.826960000000000e+02
+1.143340000000000e+03 +5.216430000000000e+02
+9.941460000000000e+02 +3.918860000000000e+02
+1.078020000000000e+03 +4.642500000000000e+02
+2.864180000000000e+02 +2.305660000000000e+02
+6.179380000000000e+02 +2.551790000000000e+02
+6.159420000000000e+02 +2.488280000000000e+02
+2.093700000000000e+03 +8.500560000000000e+02
+6.182260000000000e+02 +2.172160000000000e+02
+1.066560000000000e+03 +4.029810000000000e+02
+6.847719999999998e+02 +4.387350000000000e+02
+6.577940000000000e+02 +2.808760000000000e+02
+9.036670000000000e+02 +4.045480000000000e+02
+6.486030000000002e+02 +2.667940000000000e+02
+6.081120000000000e+02 +3.014940000000000e+02
+6.541210000000000e+02 +2.357850000000000e+02
+2.093050000000000e+03 +9.429000000000000e+02
+1.047860000000000e+03 +3.788480000000000e+02
+1.140060000000000e+03 +4.108130000000001e+02
+7.031930000000000e+02 +3.887410000000000e+02
+6.378510000000000e+02 +2.512720000000000e+02
+1.287160000000000e+03 +5.631930000000000e+02
+6.406440000000000e+02 +2.376780000000000e+02
+9.411000000000000e+02 +4.233590000000000e+02
+6.419950000000000e+02 +2.185910000000000e+02
+1.397010000000000e+03 +5.744090000000000e+02
+1.056290000000000e+03 +5.762370000000000e+02
+6.387520000000000e+02 +2.443300000000000e+02
+6.361120000000000e+02 +2.324970000000000e+02
+2.067940000000000e+03 +8.748060000000000e+02
+6.200660000000000e+02 +2.061110000000000e+02
+6.773480000000002e+02 +3.290050000000000e+02
+8.868240000000000e+02 +4.014790000000000e+02
+1.033020000000000e+03 +4.029810000000000e+02
+6.785910000000000e+02 +2.084000000000000e+02
+6.644540000000000e+02 +2.577710000000000e+02
+8.897880000000000e+02 +3.187710000000000e+02
+7.387410000000001e+02 +4.264440000000000e+02
+6.369450000000001e+02 +2.091390000000000e+02
+6.363930000000000e+02 +2.002980000000000e+02
+1.585550000000000e+03 +5.755130000000000e+02
+1.159480000000000e+03 +4.635620000000000e+02
+1.168430000000000e+03 +5.164860000000000e+02
+6.667160000000000e+02 +2.153970000000000e+02
+6.604889999999998e+02 +1.800540000000000e+02
+1.332940000000000e+03 +5.331820000000000e+02
+6.386369999999999e+02 +2.337080000000000e+02
+1.141600000000000e+03 +6.455620000000000e+02
+1.076360000000000e+03 +6.066340000000000e+02
+9.375930000000000e+02 +3.431540000000000e+02
+6.590620000000000e+02 +2.133680000000000e+02
+6.357809999999999e+02 +2.004390000000000e+02
+1.236860000000000e+03 +5.898040000000000e+02
+1.164240000000000e+03 +5.392270000000000e+02
+9.315460000000000e+02 +3.710330000000000e+02
+8.969360000000000e+02 +2.550470000000000e+02
+7.016220000000000e+02 +2.607320000000000e+02
+1.849060000000000e+03 +8.400980000000002e+02
+6.701160000000001e+02 +2.245390000000000e+02
+6.568780000000000e+02 +2.333200000000000e+02
+2.127850000000000e+03 +7.482930000000000e+02
+6.408780000000000e+02 +2.535650000000000e+02
+1.061360000000000e+03 +3.712790000000000e+02
+6.352640000000000e+02 +2.501240000000000e+02
+6.498130000000000e+02 +2.040720000000000e+02
+6.356190000000000e+02 +1.565700000000000e+02
+7.916690000000000e+02 +3.892570000000000e+02
+1.201330000000000e+03 +5.431600000000000e+02
+6.362930000000000e+02 +2.318070000000000e+02
+6.516780000000000e+02 +2.370760000000000e+02
+8.048800000000000e+02 +3.773210000000000e+02
+7.065520000000000e+02 +2.929120000000001e+02
+9.994000000000000e+02 +4.357690000000000e+02
+6.689210000000000e+02 +2.748700000000000e+02
+6.373000000000000e+02 +2.358060000000000e+02
+7.920230000000000e+02 +3.666410000000000e+02
+6.600450000000000e+02 +2.358840000000000e+02
+8.696230000000000e+02 +2.823500000000000e+02
+2.699940000000000e+03 +1.082340000000000e+03
+8.969550000000000e+02 +2.928340000000000e+02
+7.729710000000000e+02 +2.936630000000000e+02
+1.076230000000000e+03 +6.018450000000000e+02
+6.541369999999999e+02 +2.096290000000000e+02
+1.601440000000000e+03 +6.370419999999998e+02
+1.260850000000000e+03 +4.796060000000000e+02
+6.545180000000000e+02 +2.278920000000000e+02
+6.356210000000000e+02 +2.090970000000000e+02
+8.076160000000001e+02 +3.612230000000000e+02
+1.600380000000000e+03 +6.280620000000000e+02
+8.365410000000001e+02 +2.575300000000000e+02
+1.236160000000000e+03 +5.090300000000000e+02
+4.866730000000000e+02 +1.283450000000000e+02
+8.976189999999998e+02 +5.636040000000000e+02
+6.349090000000000e+02 +2.327260000000000e+02
+1.540490000000000e+03 +5.905650000000001e+02
+1.007290000000000e+03 +3.492950000000000e+02
+6.164299999999999e+02 +2.001700000000000e+02
+1.863700000000000e+03 +8.328589999999998e+02
+1.126350000000000e+03 +4.288010000000000e+02
+1.019640000000000e+03 +3.260740000000000e+02
+1.078720000000000e+03 +4.558530000000000e+02
+6.854670000000000e+02 +3.212870000000001e+02
+8.547389999999998e+02 +2.657280000000000e+02
+1.598230000000000e+03 +7.574550000000000e+02
+1.070750000000000e+03 +4.889770000000000e+02
+6.599980000000000e+02 +3.244220000000000e+02
+1.636130000000000e+03 +8.294330000000000e+02
+1.040930000000000e+03 +3.961370000000000e+02
+1.625760000000000e+03 +7.988450000000000e+02
+7.039960000000002e+02 +2.870450000000000e+02
+1.069300000000000e+03 +3.835600000000000e+02
+9.918300000000000e+02 +3.397569999999999e+02
+1.077000000000000e+03 +4.310430000000000e+02
+6.885210000000002e+02 +3.369880000000001e+02
+8.940110000000002e+02 +2.647820000000000e+02
+1.268930000000000e+03 +5.006100000000000e+02
+1.628280000000000e+03 +7.611960000000000e+02
+1.046930000000000e+03 +4.115220000000000e+02
+1.555710000000000e+03 +4.982080000000000e+02
+1.325530000000000e+03 +4.456850000000000e+02
+6.968930000000000e+02 +3.145780000000000e+02
+6.794850000000000e+02 +3.275340000000000e+02
+1.095030000000000e+03 +4.238630000000001e+02
+1.086460000000000e+03 +3.906510000000000e+02
+1.783560000000000e+03 +6.967150000000000e+02
+1.075570000000000e+03 +3.634510000000000e+02
+6.972650000000000e+02 +2.890910000000000e+02
+1.931700000000000e+03 +7.594370000000000e+02
+1.220200000000000e+03 +4.123660000000000e+02
+1.921010000000000e+03 +7.871010000000001e+02
+6.833539999999998e+02 +2.776010000000000e+02
+7.099440000000000e+02 +2.977430000000000e+02
+1.573440000000000e+03 +5.318140000000000e+02
+8.911700000000000e+02 +2.636450000000000e+02
+1.132360000000000e+03 +3.980080000000000e+02
+1.079370000000000e+03 +4.385310000000000e+02
+1.072110000000000e+03 +3.851870000000000e+02
+6.593250000000000e+02 +2.772590000000000e+02
+1.671870000000000e+03 +6.337540000000000e+02
+1.617620000000000e+03 +6.333919999999998e+02
+1.508630000000000e+03 +7.807270000000000e+02
+1.077040000000000e+03 +4.389450000000000e+02
+8.773550000000000e+02 +3.974100000000000e+02
+8.757289999999998e+02 +3.944090000000000e+02
+8.704010000000002e+02 +4.033980000000000e+02
+1.073050000000000e+03 +4.280670000000000e+02
+1.079090000000000e+03 +3.926690000000000e+02
+1.671450000000000e+03 +6.335570000000000e+02
+1.122310000000000e+03 +5.709510000000000e+02
+9.743240000000000e+02 +3.706890000000000e+02
+8.396220000000000e+02 +2.457890000000000e+02
+1.572960000000000e+03 +5.344710000000000e+02
+9.004190000000000e+02 +2.759970000000000e+02
+8.075350000000000e+02 +3.593880000000001e+02
+1.063800000000000e+03 +4.380300000000000e+02
+1.585410000000000e+03 +5.762260000000000e+02
+8.045139999999999e+02 +3.633720000000000e+02
+1.086450000000000e+03 +3.999010000000000e+02
+1.585490000000000e+03 +4.942390000000000e+02
+1.107180000000000e+03 +3.360830000000000e+02
+1.039640000000000e+03 +3.473620000000000e+02
+9.926140000000000e+02 +2.984400000000000e+02
+1.290030000000000e+03 +7.573520000000000e+02
+9.878460000000000e+02 +3.347870000000000e+02
+1.383040000000000e+03 +4.997800000000000e+02
+7.924420000000000e+02 +3.162310000000000e+02
+1.008710000000000e+03 +4.666230000000001e+02
+1.394770000000000e+03 +5.173700000000000e+02
+1.053060000000000e+03 +3.419570000000000e+02
+1.569380000000000e+03 +6.918410000000000e+02
+1.389640000000000e+03 +5.720750000000000e+02
+8.668650000000000e+02 +2.830150000000000e+02
+1.048030000000000e+03 +3.917490000000000e+02
+6.985870000000000e+02 +3.408200000000000e+02
+9.933440000000001e+02 +2.272510000000000e+02
+1.086490000000000e+03 +3.074160000000000e+02
+8.860930000000002e+02 +2.168910000000000e+02
+1.911830000000000e+03 +7.961430000000000e+02
+1.130100000000000e+03 +3.886620000000000e+02
+1.079450000000000e+03 +4.487130000000000e+02
+9.891140000000000e+02 +3.521490000000000e+02
+1.012410000000000e+03 +4.760610000000000e+02
+1.379240000000000e+03 +5.820400000000000e+02
+1.575970000000000e+03 +4.555880000000000e+02
+1.006150000000000e+03 +4.491300000000000e+02
+1.064940000000000e+03 +3.738310000000000e+02
+7.027539999999998e+02 +3.225150000000000e+02
+1.384680000000000e+03 +4.766950000000000e+02
+8.033900000000000e+02 +3.391020000000000e+02
+1.107800000000000e+03 +3.101390000000000e+02
+1.096760000000000e+03 +3.668430000000000e+02
+9.764670000000000e+02 +2.466970000000000e+02
+1.047550000000000e+03 +4.417660000000000e+02
+1.014880000000000e+03 +4.915400000000000e+02
+2.818490000000000e+02 +1.707020000000000e+02
+2.794470000000000e+02 +1.653930000000000e+02
+1.338140000000000e+03 +4.845100000000000e+02
+1.065510000000000e+03 +4.206650000000000e+02
+1.387010000000000e+03 +5.348470000000000e+02
+8.601260000000002e+02 +2.838640000000001e+02
+9.961200000000000e+02 +3.492050000000000e+02
+2.747480000000000e+02 +1.509820000000000e+02
+1.000690000000000e+03 +3.893720000000000e+02
+9.835700000000001e+02 +3.171460000000000e+02
+9.930510000000000e+02 +4.398130000000001e+02
+1.114660000000000e+03 +4.476630000000000e+02
+1.007980000000000e+03 +2.707560000000000e+02
+1.325970000000000e+03 +4.732380000000001e+02
+1.074460000000000e+03 +4.360420000000000e+02
+1.423690000000000e+03 +7.185230000000000e+02
+1.402510000000000e+03 +6.869950000000000e+02
+1.289320000000000e+03 +5.303869999999999e+02
+9.023150000000001e+02 +2.465430000000000e+02
+1.144130000000000e+03 +3.733730000000001e+02
+1.146110000000000e+03 +4.458160000000000e+02
+9.934950000000000e+02 +3.428840000000000e+02
+6.818120000000000e+02 +2.063820000000000e+02
+9.956720000000000e+02 +2.527320000000000e+02
+1.394380000000000e+03 +7.406500000000000e+02
+7.096519999999998e+02 +2.114340000000000e+02
+1.064460000000000e+03 +4.121250000000000e+02
+9.871559999999999e+02 +3.308280000000001e+02
+1.002060000000000e+03 +4.193190000000000e+02
+8.864939999999998e+02 +4.425120000000000e+02
+1.001300000000000e+03 +4.427350000000000e+02
+9.967600000000000e+02 +2.656290000000000e+02
+1.073290000000000e+03 +4.389010000000000e+02
+8.974470000000000e+02 +2.619350000000000e+02
+8.275419999999998e+02 +2.306560000000000e+02
+1.227070000000000e+03 +5.122900000000000e+02
+9.875850000000000e+02 +3.182350000000000e+02
+6.198300000000000e+02 +2.787130000000000e+02
+1.162960000000000e+03 +5.522430000000001e+02
+1.122320000000000e+03 +4.942380000000001e+02
+1.148620000000000e+03 +3.591650000000000e+02
+1.428820000000000e+03 +7.676640000000000e+02
+6.205850000000000e+02 +2.665520000000000e+02
+1.619850000000000e+03 +7.303180000000000e+02
+9.892340000000000e+02 +4.226490000000000e+02
+1.595400000000000e+03 +7.785470000000000e+02
+2.817890000000000e+02 +1.098470000000000e+02
+1.082490000000000e+03 +5.123400000000000e+02
+1.448210000000000e+03 +5.935340000000000e+02
+9.300119999999999e+02 +4.485140000000000e+02
+9.767340000000000e+02 +3.629520000000000e+02
+8.930939999999998e+02 +2.780780000000000e+02
+6.614100000000000e+02 +2.768720000000000e+02
+1.380870000000000e+03 +6.642819999999998e+02
+1.053000000000000e+03 +3.529990000000000e+02
+6.247840000000000e+02 +2.848690000000000e+02
+6.247110000000000e+02 +2.697640000000000e+02
+1.362360000000000e+03 +6.909380000000000e+02
+1.598730000000000e+03 +7.774380000000000e+02
+1.289700000000000e+03 +6.981660000000001e+02
+1.072770000000000e+03 +4.579270000000000e+02
+1.407040000000000e+03 +7.219380000000000e+02
+1.006260000000000e+03 +5.301050000000000e+02
+1.626190000000000e+03 +7.805219999999998e+02
+1.010740000000000e+03 +4.417120000000000e+02
+8.982160000000000e+02 +2.359760000000000e+02
+1.296840000000000e+03 +6.853240000000000e+02
+1.914850000000000e+03 +8.803389999999998e+02
+9.908800000000000e+02 +2.875420000000001e+02
+9.621609999999999e+02 +3.056000000000000e+02
+6.225810000000000e+02 +2.495810000000000e+02
+2.816630000000000e+02 +1.466460000000000e+02
+2.792490000000000e+02 +1.577410000000000e+02
+1.410610000000000e+03 +7.763240000000000e+02
+1.025900000000000e+03 +3.903980000000000e+02
+1.461200000000000e+03 +6.029540000000002e+02
+9.337880000000000e+02 +4.324920000000000e+02
+1.865770000000000e+03 +8.379500000000000e+02
+8.783260000000000e+02 +2.119200000000000e+02
+1.052040000000000e+03 +4.046690000000000e+02
+6.244400000000001e+02 +2.263770000000000e+02
+6.643110000000000e+02 +2.349510000000000e+02
+6.551750000000000e+02 +1.655760000000000e+02
+2.827820000000000e+02 +1.236930000000000e+02
+1.313730000000000e+03 +5.749650000000000e+02
+9.886120000000000e+02 +2.971040000000001e+02
+9.002510000000002e+02 +2.367390000000000e+02
+2.962260000000000e+02 +1.587080000000000e+02
+6.252130000000002e+02 +2.388780000000000e+02
+6.228600000000000e+02 +2.296630000000000e+02
+1.642170000000000e+03 +7.765169999999998e+02
+8.744000000000000e+02 +2.279990000000000e+02
+1.334910000000000e+03 +5.889700000000000e+02
+6.615089999999999e+02 +2.107440000000000e+02
+1.859530000000000e+03 +7.901780000000000e+02
+6.598910000000002e+02 +2.135670000000000e+02
+1.870670000000000e+03 +7.706450000000000e+02
+2.819980000000000e+02 +1.047390000000000e+02
+6.205860000000000e+02 +1.758700000000000e+02
+8.878400000000000e+02 +2.227480000000000e+02
+1.071710000000000e+03 +4.655900000000000e+02
+8.738950000000000e+02 +5.468710000000000e+02
+9.145510000000000e+02 +4.208870000000000e+02
+1.047100000000000e+03 +4.317950000000000e+02
+2.805150000000000e+02 +1.051160000000000e+02
+1.738790000000000e+03 +7.760490000000000e+02
+6.356430000000000e+02 +1.815290000000000e+02
+6.307200000000000e+02 +1.647140000000000e+02
+6.607550000000000e+02 +2.059910000000000e+02
+1.881820000000000e+03 +7.873869999999999e+02
+8.986120000000000e+02 +2.254410000000000e+02
+1.066010000000000e+03 +3.793230000000000e+02
+9.945710000000000e+02 +3.087720000000000e+02
+9.884780000000000e+02 +3.261720000000000e+02
+6.235700000000001e+02 +1.673860000000000e+02
+1.077980000000000e+03 +3.824450000000000e+02
+1.309790000000000e+03 +5.363190000000000e+02
+9.780620000000000e+02 +2.877120000000000e+02
+9.284070000000000e+02 +4.384430000000000e+02
+6.594320000000000e+02 +2.058230000000000e+02
+1.289330000000000e+03 +6.415790000000002e+02
+6.242470000000000e+02 +1.446510000000000e+02
+1.004090000000000e+03 +4.336160000000000e+02
+8.945530000000000e+02 +2.179610000000000e+02
+6.257909999999998e+02 +1.558030000000000e+02
+2.832220000000000e+02 +7.618170000000001e+01
+4.898500000000000e+02 +3.233890000000000e+02
+1.011820000000000e+03 +3.072480000000000e+02
+1.317590000000000e+03 +5.932520000000000e+02
+6.585050000000000e+02 +1.711510000000000e+02
+1.589260000000000e+03 +6.029370000000000e+02
+9.143970000000000e+02 +4.117920000000000e+02
+9.920140000000000e+02 +3.159520000000000e+02
+9.796300000000000e+02 +3.782660000000000e+02
+7.814820000000000e+02 +2.929740000000000e+02
+6.258150000000001e+02 +1.623080000000000e+02
+4.920060000000000e+02 +3.164140000000000e+02
+7.841680000000000e+02 +2.800580000000000e+02
+1.609210000000000e+03 +7.121920000000000e+02
+9.990300000000000e+02 +3.221290000000000e+02
+6.604220000000000e+02 +1.662020000000000e+02
+2.828090000000000e+02 +8.273779999999999e+01
+1.067050000000000e+03 +4.144660000000000e+02
+7.767830000000000e+02 +3.100760000000000e+02
+1.060760000000000e+03 +3.521790000000001e+02
+9.777130000000000e+02 +2.652620000000000e+02
+9.329850000000000e+02 +4.310820000000000e+02
+4.861300000000000e+02 +3.043570000000000e+02
+1.866440000000000e+03 +8.572669999999998e+02
+7.907530000000000e+02 +2.735370000000000e+02
+1.303570000000000e+03 +5.503740000000000e+02
+9.879520000000000e+02 +2.902350000000000e+02
+6.780419999999998e+02 +3.209260000000000e+02
+7.850860000000000e+02 +2.754950000000000e+02
+9.300970000000000e+02 +4.278980000000000e+02
+4.960720000000000e+02 +3.113980000000000e+02
+7.844839999999998e+02 +2.730580000000000e+02
+8.720020000000000e+02 +2.156740000000000e+02
+8.639860000000001e+02 +3.353210000000000e+02
+6.925260000000002e+02 +2.960830000000000e+02
+1.293520000000000e+03 +5.439760000000000e+02
+6.461770000000000e+02 +3.160940000000000e+02
+7.684299999999999e+02 +2.923400000000000e+02
+1.607340000000000e+03 +6.252150000000000e+02
+1.852070000000000e+03 +8.310770000000000e+02
+7.857450000000000e+02 +2.787360000000000e+02
+1.290040000000000e+03 +6.192350000000000e+02
+9.738380000000000e+02 +2.820240000000000e+02
+7.832050000000000e+02 +2.660940000000000e+02
+7.700850000000000e+02 +2.964090000000000e+02
+4.881460000000000e+02 +2.899490000000000e+02
+7.689010000000002e+02 +2.941880000000000e+02
+7.821590000000000e+02 +2.827940000000001e+02
+6.974180000000000e+02 +3.546650000000000e+02
+1.288340000000000e+03 +4.981790000000000e+02
+4.931890000000000e+02 +2.992810000000000e+02
+7.740130000000000e+02 +3.001680000000000e+02
+8.949210000000000e+02 +2.092790000000000e+02
+8.750419999999998e+02 +5.740380000000000e+02
+1.862830000000000e+03 +8.660400000000000e+02
+1.613670000000000e+03 +6.785950000000000e+02
+1.911140000000000e+03 +6.795239999999999e+02
+7.809280000000000e+02 +2.903850000000000e+02
+6.953720000000000e+02 +3.145430000000000e+02
+1.219460000000000e+03 +4.501920000000000e+02
+9.777880000000000e+02 +2.765890000000000e+02
+4.910380000000000e+02 +2.910200000000000e+02
+7.878960000000002e+02 +2.583440000000000e+02
+6.952569999999999e+02 +3.302970000000000e+02
+1.853700000000000e+03 +8.029060000000002e+02
+7.864900000000000e+02 +2.915380000000000e+02
+6.908450000000000e+02 +3.465480000000000e+02
+7.563770000000000e+02 +2.794020000000000e+02
+7.874200000000000e+02 +2.728270000000000e+02
+6.895219999999998e+02 +2.704180000000000e+02
+4.871660000000000e+02 +2.849470000000000e+02
+7.878040000000000e+02 +2.961190000000000e+02
+7.743190000000000e+02 +2.574470000000000e+02
+8.718420000000000e+02 +3.570980000000000e+02
+1.324350000000000e+03 +5.947850000000000e+02
+1.855070000000000e+03 +9.322320000000000e+02
+1.609240000000000e+03 +5.726790000000000e+02
+7.911970000000000e+02 +2.725280000000000e+02
+8.777220000000000e+02 +3.928120000000000e+02
+1.850420000000000e+03 +8.399069999999998e+02
+1.506450000000000e+03 +6.489080000000000e+02
+1.606570000000000e+03 +5.662220000000000e+02
+1.539360000000000e+03 +6.444500000000000e+02
+1.344250000000000e+03 +6.190390000000000e+02
+4.942910000000000e+02 +2.916920000000000e+02
+7.767150000000000e+02 +2.675890000000000e+02
+7.542080000000002e+02 +2.554920000000000e+02
+1.608990000000000e+03 +5.210660000000000e+02
+7.833939999999999e+02 +2.674740000000000e+02
+7.846110000000001e+02 +2.824370000000000e+02
+1.450240000000000e+03 +6.928310000000000e+02
+1.355970000000000e+03 +4.584750000000000e+02
+6.909050000000000e+02 +3.024580000000000e+02
+9.951150000000000e+02 +2.932130000000000e+02
+1.847890000000000e+03 +8.431750000000000e+02
+7.852139999999998e+02 +2.749560000000000e+02
+1.573630000000000e+03 +6.876450000000000e+02
+8.660690000000000e+02 +3.639130000000000e+02
+8.970000000000000e+02 +5.521709999999998e+02
+1.318380000000000e+03 +5.789960000000000e+02
+1.690220000000000e+03 +6.693689999999998e+02
+1.670890000000000e+03 +5.817869999999998e+02
+1.126930000000000e+03 +4.474420000000000e+02
+4.864550000000000e+02 +2.847690000000000e+02
+7.752050000000000e+02 +2.493810000000000e+02
+7.946189999999998e+02 +2.527530000000000e+02
+1.295470000000000e+03 +4.897930000000000e+02
+1.561840000000000e+03 +5.169940000000000e+02
+1.603770000000000e+03 +6.228090000000000e+02
+1.297120000000000e+03 +5.660800000000000e+02
+1.271640000000000e+03 +5.506050000000000e+02
+1.376030000000000e+03 +6.349119999999998e+02
+1.107930000000000e+03 +4.099230000000000e+02
+7.852719999999998e+02 +2.538540000000000e+02
+1.573170000000000e+03 +5.116240000000000e+02
+3.407530000000001e+02 +1.943310000000000e+02
+7.847780000000000e+02 +2.652350000000000e+02
+1.273610000000000e+03 +4.349840000000000e+02
+1.572160000000000e+03 +5.559680000000002e+02
+5.024430000000000e+02 +2.933120000000000e+02
+1.708390000000000e+03 +6.362780000000000e+02
+7.807320000000000e+02 +2.503180000000000e+02
+1.362210000000000e+03 +6.210660000000000e+02
+1.873030000000000e+03 +8.615740000000000e+02
+1.861680000000000e+03 +8.284530000000000e+02
+1.301520000000000e+03 +5.103360000000000e+02
+7.886640000000000e+02 +2.622590000000000e+02
+1.304600000000000e+03 +4.703500000000000e+02
+1.661180000000000e+03 +5.671210000000000e+02
+1.372250000000000e+03 +6.377500000000000e+02
+1.569910000000000e+03 +5.149610000000000e+02
+1.823480000000000e+03 +8.544090000000000e+02
+2.320400000000000e+03 +9.466450000000000e+02
+1.302160000000000e+03 +4.945090000000000e+02
+7.452400000000000e+02 +2.728010000000000e+02
+5.002260000000000e+02 +2.959460000000000e+02
+1.304800000000000e+03 +4.618820000000000e+02
+1.369180000000000e+03 +6.256590000000000e+02
+7.498630000000001e+02 +2.547680000000000e+02
+2.318940000000000e+03 +9.717340000000000e+02
+8.074250000000000e+02 +4.602580000000000e+02
+7.830430000000000e+02 +2.367830000000000e+02
+1.304300000000000e+03 +4.851180000000001e+02
+8.635169999999998e+02 +3.537470000000000e+02
+1.317250000000000e+03 +5.527200000000000e+02
+9.737150000000000e+02 +4.007380000000001e+02
+1.318210000000000e+03 +4.339190000000000e+02
+7.868900000000000e+02 +2.755760000000000e+02
+6.580820000000000e+02 +5.554130000000000e+02
+1.348920000000000e+03 +6.361360000000000e+02
+1.307480000000000e+03 +4.037240000000000e+02
+7.836189999999998e+02 +2.457520000000000e+02
+1.597200000000000e+03 +6.278840000000000e+02
+7.580930000000002e+02 +2.603820000000000e+02
+7.861790000000000e+02 +2.656120000000000e+02
+2.318510000000000e+03 +8.869450000000001e+02
+1.372930000000000e+03 +6.469069999999998e+02
+1.310850000000000e+03 +4.366810000000000e+02
+1.284550000000000e+03 +4.167200000000000e+02
+3.414110000000000e+02 +1.692290000000000e+02
+1.328240000000000e+03 +6.254780000000002e+02
+7.817310000000001e+02 +2.601580000000000e+02
+7.665119999999999e+02 +2.370470000000000e+02
+8.584939999999998e+02 +3.413020000000000e+02
+1.299590000000000e+03 +4.103670000000000e+02
+1.291530000000000e+03 +5.485269999999998e+02
+9.633810000000000e+02 +2.750980000000000e+02
+6.740889999999998e+02 +2.603670000000000e+02
+7.492650000000000e+02 +2.271180000000000e+02
+1.309440000000000e+03 +4.442860000000000e+02
+7.745030000000000e+02 +2.393460000000000e+02
+1.293380000000000e+03 +3.974520000000000e+02
+7.927310000000001e+02 +2.532790000000000e+02
+1.306090000000000e+03 +4.063640000000000e+02
+1.919280000000000e+03 +6.735500000000000e+02
+8.817200000000000e+02 +3.668490000000000e+02
+1.289260000000000e+03 +6.213830000000000e+02
+7.879310000000000e+02 +2.506570000000000e+02
+1.361980000000000e+03 +5.966910000000000e+02
+3.378219999999999e+02 +1.561740000000000e+02
+7.873680000000001e+02 +2.573020000000000e+02
+7.872660000000002e+02 +2.368550000000000e+02
+8.753989999999999e+02 +3.358970000000000e+02
+1.356130000000000e+03 +5.577330000000002e+02
+1.317250000000000e+03 +4.165150000000000e+02
+3.274480000000000e+02 +1.602100000000000e+02
+1.310060000000000e+03 +4.158200000000000e+02
+7.506250000000000e+02 +2.436440000000000e+02
+8.555740000000000e+02 +3.285050000000000e+02
+1.404920000000000e+03 +6.968869999999999e+02
+4.934730000000000e+02 +2.754560000000000e+02
+1.286320000000000e+03 +4.031680000000000e+02
+7.879320000000000e+02 +2.317490000000000e+02
+1.311040000000000e+03 +3.775840000000000e+02
+7.779710000000000e+02 +2.620640000000000e+02
+3.250290000000000e+02 +1.510720000000000e+02
+7.872040000000000e+02 +2.393620000000000e+02
+1.084260000000000e+03 +6.484760000000000e+02
+7.792730000000000e+02 +2.497660000000000e+02
+1.068570000000000e+03 +6.223020000000000e+02
+1.298800000000000e+03 +3.695570000000000e+02
+1.319590000000000e+03 +4.147380000000001e+02
+7.840939999999998e+02 +2.395490000000000e+02
+5.039160000000000e+02 +1.524450000000000e+02
+7.915390000000000e+02 +2.555240000000000e+02
+1.319870000000000e+03 +6.121280000000000e+02
+1.299540000000000e+03 +3.590200000000000e+02
+1.300670000000000e+03 +3.716950000000000e+02
+7.594470000000000e+02 +2.370770000000000e+02
+8.598080000000000e+02 +3.437590000000000e+02
+1.576970000000000e+03 +7.131230000000000e+02
+1.918290000000000e+03 +7.857420000000000e+02
+7.822439999999998e+02 +2.496700000000000e+02
+1.282450000000000e+03 +3.639270000000000e+02
+7.916270000000000e+02 +2.558940000000000e+02
+1.071260000000000e+03 +5.642640000000000e+02
+6.669989999999998e+02 +3.750220000000000e+02
+7.928869999999999e+02 +2.340330000000000e+02
+7.875580000000000e+02 +2.638600000000000e+02
+3.473150000000000e+02 +1.440230000000000e+02
+1.080260000000000e+03 +6.332690000000000e+02
+7.887080000000002e+02 +2.414620000000000e+02
+6.669900000000000e+02 +5.706799999999999e+02
+4.884410000000000e+02 +2.193250000000000e+02
+1.064600000000000e+03 +5.676210000000000e+02
+7.847120000000000e+02 +2.446020000000000e+02
+7.764349999999999e+02 +2.313650000000000e+02
+7.866770000000000e+02 +2.384780000000000e+02
+1.084390000000000e+03 +3.960030000000000e+02
+7.761150000000000e+02 +2.413090000000000e+02
+7.936669999999998e+02 +2.517030000000000e+02
+7.939670000000000e+02 +2.378890000000000e+02
+3.463510000000000e+02 +1.258220000000000e+02
+1.097430000000000e+03 +6.430650000000001e+02
+7.897970000000000e+02 +2.362240000000000e+02
+1.129110000000000e+03 +5.677000000000000e+02
+1.064900000000000e+03 +5.533130000000000e+02
+9.861890000000000e+02 +4.743610000000000e+02
+1.110160000000000e+03 +4.148140000000000e+02
+6.537800000000000e+02 +5.533650000000000e+02
+7.062910000000001e+02 +3.995640000000000e+02
+8.462890000000000e+02 +2.514840000000000e+02
+1.677950000000000e+03 +6.938939999999999e+02
+1.035100000000000e+03 +5.361669999999998e+02
+8.660470000000000e+02 +3.960570000000000e+02
+8.583210000000000e+02 +2.589290000000000e+02
+1.087520000000000e+03 +4.083720000000000e+02
+8.151300000000000e+02 +4.933750000000000e+02
+1.866120000000000e+03 +8.769689999999998e+02
+1.029600000000000e+03 +3.821930000000000e+02
+1.030360000000000e+03 +5.527210000000000e+02
+6.931580000000000e+02 +3.510760000000000e+02
+7.114910000000001e+02 +3.911030000000000e+02
+7.106480000000000e+02 +3.729300000000000e+02
+6.688620000000000e+02 +2.190560000000000e+02
+1.690660000000000e+03 +6.937560000000002e+02
+1.671390000000000e+03 +6.926469999999998e+02
+1.269340000000000e+03 +5.890660000000000e+02
+1.479730000000000e+03 +7.109550000000000e+02
+1.675590000000000e+03 +6.390459999999998e+02
+1.035150000000000e+03 +4.885530000000001e+02
+6.357460000000000e+02 +2.675710000000000e+02
+1.872240000000000e+03 +9.377420000000000e+02
+1.670470000000000e+03 +6.593810000000002e+02
+6.343850000000000e+02 +2.753530000000000e+02
+9.629120000000000e+02 +4.872620000000000e+02
+1.020090000000000e+03 +3.629510000000000e+02
+6.968330000000002e+02 +3.581410000000000e+02
+7.048400000000000e+02 +3.995030000000000e+02
+7.922680000000000e+02 +3.003400000000000e+02
+1.378350000000000e+03 +5.369230000000000e+02
+6.377700000000000e+02 +2.532960000000000e+02
+2.449090000000000e+03 +1.143330000000000e+03
+2.026650000000000e+03 +9.464700000000000e+02
+1.110080000000000e+03 +4.009290000000000e+02
+1.081470000000000e+03 +3.593500000000000e+02
+1.027480000000000e+03 +5.473680000000001e+02
+6.974520000000000e+02 +3.553119999999999e+02
+6.244430000000000e+02 +2.745130000000000e+02
+6.570219999999998e+02 +2.745770000000000e+02
+1.390480000000000e+03 +5.363640000000000e+02
+6.367040000000002e+02 +2.510140000000000e+02
+8.950910000000000e+02 +2.668250000000000e+02
+6.345680000000000e+02 +2.726910000000000e+02
+1.308820000000000e+03 +7.544550000000000e+02
+1.384590000000000e+03 +6.220269999999998e+02
+6.181980000000000e+02 +2.379080000000000e+02
+1.124280000000000e+03 +2.907670000000000e+02
+1.011440000000000e+03 +5.148760000000000e+02
+1.154010000000000e+03 +5.408660000000000e+02
+1.679280000000000e+03 +5.961360000000000e+02
+2.440740000000000e+03 +1.221500000000000e+03
+9.112220000000000e+02 +2.822770000000000e+02
+6.513250000000000e+02 +2.428080000000000e+02
+6.622530000000000e+02 +2.762990000000000e+02
+6.508170000000000e+02 +2.842990000000001e+02
+2.690030000000000e+03 +9.444470000000000e+02
+1.381170000000000e+03 +6.086920000000000e+02
+9.001450000000000e+02 +2.505480000000000e+02
+6.881560000000002e+02 +2.216010000000000e+02
+1.857100000000000e+03 +9.089290000000000e+02
+9.042190000000001e+02 +2.646300000000000e+02
+6.342660000000000e+02 +2.703660000000000e+02
+6.723290000000000e+02 +2.199900000000000e+02
+1.373920000000000e+03 +5.923670000000000e+02
+6.618919999999998e+02 +2.773330000000000e+02
+2.107570000000000e+03 +9.594750000000000e+02
+1.738860000000000e+03 +8.012439999999998e+02
+1.099050000000000e+03 +3.484080000000000e+02
+1.069460000000000e+03 +3.100290000000000e+02
+6.915139999999999e+02 +3.681050000000000e+02
+6.567669999999998e+02 +2.580050000000000e+02
+9.092760000000000e+02 +2.688420000000000e+02
+1.433970000000000e+03 +4.863380000000000e+02
+6.638700000000000e+02 +2.742610000000000e+02
+6.057430000000001e+02 +1.970550000000000e+02
+2.115860000000000e+03 +9.847470000000000e+02
+1.856630000000000e+03 +8.835630000000000e+02
+8.386489999999999e+02 +2.508810000000000e+02
+8.949839999999998e+02 +2.668110000000000e+02
+6.331990000000002e+02 +2.496810000000000e+02
+6.553989999999999e+02 +2.120630000000000e+02
+1.872480000000000e+03 +7.953460000000000e+02
+6.639280000000000e+02 +3.797670000000000e+02
+1.548790000000000e+03 +7.007070000000000e+02
+8.869169999999998e+02 +2.390420000000000e+02
+9.107270000000000e+02 +2.385240000000000e+02
+1.862420000000000e+03 +7.835010000000002e+02
+8.989910000000001e+02 +5.889050000000000e+02
+7.359739999999998e+02 +3.000600000000000e+02
+6.883570000000000e+02 +3.325290000000000e+02
+6.575910000000000e+02 +2.607190000000000e+02
+6.243940000000000e+02 +1.797140000000000e+02
+1.471030000000000e+03 +5.754240000000000e+02
+6.537569999999999e+02 +2.522430000000000e+02
+8.976250000000000e+02 +2.573720000000000e+02
+1.757980000000000e+03 +7.737520000000000e+02
+6.088560000000000e+02 +2.933490000000000e+02
+8.801230000000000e+02 +2.296610000000000e+02
+6.791750000000000e+02 +3.571890000000000e+02
+6.339030000000000e+02 +2.066230000000000e+02
+6.782110000000000e+02 +2.304150000000000e+02
+1.303790000000000e+03 +5.348240000000002e+02
+6.710850000000000e+02 +2.660190000000000e+02
+1.859000000000000e+03 +8.735030000000000e+02
+8.789460000000000e+02 +3.606580000000000e+02
+6.401930000000000e+02 +2.560300000000000e+02
+8.447200000000000e+02 +2.574250000000000e+02
+8.843420000000000e+02 +5.874270000000000e+02
+1.219210000000000e+03 +6.485350000000000e+02
+8.734670000000000e+02 +3.726310000000000e+02
+6.587320000000000e+02 +2.122870000000000e+02
+1.728840000000000e+03 +7.519630000000002e+02
+1.351880000000000e+03 +5.699470000000000e+02
+1.167350000000000e+03 +3.938320000000000e+02
+6.951289999999998e+02 +4.481390000000000e+02
+6.636450000000000e+02 +2.542840000000000e+02
+1.594750000000000e+03 +6.885239999999999e+02
+6.201469999999998e+02 +2.324450000000000e+02
+1.853130000000000e+03 +8.647250000000000e+02
+8.887210000000000e+02 +2.595080000000000e+02
+6.337809999999999e+02 +2.624270000000000e+02
+6.378800000000000e+02 +2.087690000000000e+02
+1.535320000000000e+03 +6.669610000000000e+02
+6.088150000000001e+02 +2.094800000000000e+02
+1.608200000000000e+03 +7.505030000000000e+02
+8.317430000000001e+02 +2.145200000000000e+02
+1.223100000000000e+03 +6.211519999999998e+02
+7.754380000000000e+02 +3.109310000000000e+02
+1.856060000000000e+03 +7.685310000000002e+02
+1.590520000000000e+03 +7.227160000000000e+02
+6.534019999999998e+02 +2.639370000000000e+02
+8.804920000000000e+02 +2.113840000000000e+02
+6.964050000000000e+02 +3.807450000000000e+02
+6.419820000000000e+02 +2.050090000000000e+02
+9.039970000000000e+02 +2.355490000000000e+02
+6.461950000000001e+02 +2.127820000000000e+02
+1.588710000000000e+03 +5.394230000000000e+02
+1.599520000000000e+03 +6.562489999999998e+02
+8.665970000000000e+02 +3.790160000000000e+02
+1.303150000000000e+03 +5.789470000000000e+02
+1.615150000000000e+03 +8.371950000000001e+02
+6.526410000000000e+02 +2.567650000000000e+02
+1.855410000000000e+03 +7.756550000000000e+02
+1.055940000000000e+03 +7.308520000000000e+02
+6.340780000000000e+02 +2.024540000000000e+02
+1.728250000000000e+03 +7.119290000000000e+02
+1.065360000000000e+03 +7.068520000000000e+02
+6.738439999999998e+02 +2.247230000000000e+02
+1.852860000000000e+03 +7.819500000000000e+02
+1.868840000000000e+03 +8.789110000000002e+02
+6.394400000000001e+02 +2.416530000000000e+02
+1.729530000000000e+03 +6.191600000000000e+02
+6.615760000000000e+02 +2.292950000000000e+02
+1.568970000000000e+03 +5.577850000000000e+02
+8.824110000000002e+02 +2.995040000000000e+02
+9.070680000000000e+02 +2.322540000000000e+02
+8.756710000000000e+02 +3.432260000000000e+02
+1.068600000000000e+03 +5.163340000000002e+02
+8.194989999999998e+02 +4.343160000000000e+02
+1.854070000000000e+03 +7.771500000000000e+02
+1.070520000000000e+03 +4.860210000000000e+02
+6.506870000000000e+02 +2.427960000000000e+02
+4.627030000000000e+02 +1.557190000000000e+02
+1.914900000000000e+03 +5.410440000000000e+02
+6.477959999999998e+02 +2.313320000000000e+02
+1.605860000000000e+03 +5.383049999999999e+02
+8.292900000000000e+02 +2.208770000000000e+02
+1.057020000000000e+03 +4.598460000000000e+02
+1.063150000000000e+03 +5.057710000000000e+02
+6.922430000000001e+02 +3.770060000000000e+02
+6.489870000000000e+02 +2.351100000000000e+02
+7.905280000000000e+02 +3.330750000000000e+02
+8.742170000000000e+02 +3.484620000000000e+02
+4.577800000000000e+02 +1.288910000000000e+02
+6.411270000000000e+02 +1.917540000000000e+02
+1.638590000000000e+03 +8.343650000000000e+02
+8.598170000000000e+02 +2.237500000000000e+02
+1.583570000000000e+03 +8.388380000000002e+02
+8.700419999999998e+02 +3.934680000000000e+02
+1.046300000000000e+03 +4.575300000000000e+02
+6.621489999999999e+02 +2.209130000000000e+02
+8.079630000000002e+02 +3.180140000000000e+02
+1.863420000000000e+03 +7.656980000000000e+02
+8.312689999999999e+02 +2.532060000000000e+02
+1.069380000000000e+03 +3.983120000000000e+02
+6.631210000000002e+02 +2.199240000000000e+02
+1.626280000000000e+03 +8.336100000000000e+02
+1.065420000000000e+03 +3.772280000000000e+02
+8.514349999999999e+02 +2.061370000000000e+02
+7.020510000000000e+02 +3.683750000000000e+02
+1.060520000000000e+03 +3.437619999999999e+02
+6.339019999999998e+02 +2.115870000000000e+02
+1.579120000000000e+03 +5.450100000000000e+02
+1.063150000000000e+03 +4.430830000000000e+02
+1.062520000000000e+03 +3.509910000000000e+02
+7.110610000000000e+02 +4.181330000000000e+02
+1.944210000000000e+03 +6.806000000000000e+02
+1.061760000000000e+03 +3.612190000000000e+02
+8.446070000000000e+02 +2.000580000000000e+02
+1.063500000000000e+03 +3.360670000000000e+02
+1.057820000000000e+03 +3.849290000000000e+02
+7.028670000000000e+02 +4.274270000000000e+02
+7.103450000000000e+02 +3.359120000000000e+02
+1.305510000000000e+03 +7.847890000000000e+02
+1.306120000000000e+03 +7.794430000000000e+02
+1.083880000000000e+03 +3.134560000000000e+02
+8.979340000000000e+02 +2.244930000000000e+02
+1.062300000000000e+03 +3.614040000000000e+02
+7.068280000000000e+02 +4.195500000000000e+02
+1.569000000000000e+03 +5.233060000000000e+02
+1.592710000000000e+03 +8.671619999999998e+02
+9.770480000000000e+02 +3.857660000000000e+02
+1.456140000000000e+03 +8.554130000000000e+02
+1.291970000000000e+03 +7.146810000000000e+02
+1.075790000000000e+03 +3.683990000000000e+02
+1.039530000000000e+03 +3.543219999999999e+02
+9.876630000000000e+02 +3.630600000000000e+02
+1.074390000000000e+03 +3.689030000000000e+02
+8.954520000000000e+02 +1.974000000000000e+02
+8.955219999999998e+02 +2.162880000000000e+02
+8.132030000000000e+02 +4.865140000000000e+02
+1.290200000000000e+03 +7.651339999999999e+02
+1.077670000000000e+03 +3.375230000000000e+02
+6.519299999999999e+02 +2.648080000000000e+02
+6.514950000000000e+02 +2.669300000000000e+02
+1.404950000000000e+03 +4.911490000000000e+02
+8.722730000000000e+02 +3.247470000000000e+02
+1.909780000000000e+03 +6.908939999999999e+02
+1.062870000000000e+03 +3.473670000000000e+02
+1.142840000000000e+03 +6.542730000000000e+02
+1.056180000000000e+03 +3.438160000000000e+02
+1.007290000000000e+03 +5.192370000000000e+02
+9.673880000000000e+02 +3.715050000000000e+02
+1.216280000000000e+03 +5.093540000000000e+02
+1.004980000000000e+03 +5.466080000000002e+02
+1.019070000000000e+03 +5.165169999999998e+02
+8.935269999999998e+02 +1.967500000000000e+02
+1.268880000000000e+03 +4.629410000000000e+02
+7.527189999999998e+02 +2.239840000000000e+02
+1.009930000000000e+03 +4.492130000000000e+02
+2.818490000000000e+02 +1.726420000000000e+02
+1.785840000000000e+03 +6.759410000000000e+02
+1.015000000000000e+03 +4.993950000000000e+02
+1.007430000000000e+03 +4.586700000000000e+02
+1.337510000000000e+03 +5.976430000000000e+02
+9.794990000000000e+02 +3.866340000000000e+02
+7.018110000000000e+02 +3.664100000000000e+02
+1.021880000000000e+03 +5.060920000000000e+02
+1.013800000000000e+03 +4.566050000000000e+02
+1.603720000000000e+03 +7.865139999999999e+02
+2.784860000000000e+02 +1.577010000000000e+02
+9.906020000000000e+02 +3.464720000000000e+02
+1.269140000000000e+03 +6.103770000000000e+02
+8.843950000000000e+02 +2.139220000000000e+02
+1.629690000000000e+03 +7.814190000000000e+02
+1.073230000000000e+03 +5.239790000000000e+02
+1.290910000000000e+03 +7.205570000000000e+02
+1.267880000000000e+03 +5.556220000000000e+02
+1.459060000000000e+03 +7.233380000000002e+02
+1.197680000000000e+03 +4.754280000000001e+02
+1.007760000000000e+03 +4.108720000000000e+02
+1.003720000000000e+03 +4.147280000000000e+02
+1.660070000000000e+03 +6.250350000000000e+02
+1.621650000000000e+03 +5.525570000000000e+02
+1.101930000000000e+03 +5.391240000000000e+02
+1.071180000000000e+03 +4.872990000000000e+02
+1.282500000000000e+03 +5.551830000000000e+02
+1.009530000000000e+03 +4.745060000000000e+02
+1.038030000000000e+03 +4.176320000000000e+02
+1.926340000000000e+03 +6.743270000000000e+02
+2.746240000000000e+02 +1.297020000000000e+02
+1.423480000000000e+03 +8.053570000000000e+02
+9.780210000000000e+02 +3.375200000000000e+02
+6.183470000000000e+02 +2.800470000000000e+02
+1.068470000000000e+03 +4.342860000000000e+02
+9.903010000000000e+02 +3.678950000000000e+02
+1.384280000000000e+03 +7.232630000000000e+02
+9.148120000000000e+02 +5.202080000000002e+02
+6.210250000000000e+02 +2.488880000000000e+02
+1.648380000000000e+03 +4.459740000000000e+02
+7.053320000000000e+02 +2.024090000000000e+02
+1.016440000000000e+03 +5.005910000000000e+02
+1.145010000000000e+03 +5.454950000000000e+02
+1.680920000000000e+03 +6.693989999999999e+02
+2.876790000000001e+02 +1.294540000000000e+02
+1.083960000000000e+03 +4.493430000000000e+02
+1.421300000000000e+03 +8.685620000000000e+02
+6.248110000000000e+02 +1.972320000000000e+02
+6.647370000000000e+02 +2.808170000000000e+02
+1.006260000000000e+03 +3.935630000000001e+02
+6.233910000000000e+02 +2.484550000000000e+02
+1.505170000000000e+03 +7.279750000000000e+02
+7.873639999999998e+02 +2.818310000000000e+02
+1.589310000000000e+03 +4.962160000000000e+02
+1.067250000000000e+03 +3.998210000000000e+02
+9.204990000000000e+02 +4.994640000000000e+02
+1.545670000000000e+03 +5.563700000000000e+02
+1.270650000000000e+03 +5.951290000000000e+02
+6.195760000000000e+02 +2.109830000000000e+02
+6.183940000000000e+02 +1.879020000000000e+02
+6.592130000000002e+02 +2.197940000000000e+02
+9.934010000000000e+02 +3.398650000000000e+02
+1.063150000000000e+03 +4.199250000000000e+02
+9.962660000000000e+02 +3.529170000000000e+02
+2.747700000000000e+02 +1.070440000000000e+02
+8.909160000000001e+02 +4.930010000000000e+02
+1.289560000000000e+03 +3.838480000000000e+02
+1.088080000000000e+03 +4.861100000000000e+02
+8.990880000000002e+02 +6.200850000000000e+02
+6.199150000000000e+02 +1.544130000000000e+02
+6.631010000000001e+02 +2.349970000000000e+02
+5.138310000000000e+02 +3.902670000000000e+02
+1.299770000000000e+03 +5.795730000000000e+02
+6.208280000000000e+02 +1.757570000000000e+02
+7.786120000000000e+02 +2.327140000000000e+02
+1.358380000000000e+03 +5.194480000000000e+02
+9.684750000000000e+02 +3.147400000000000e+02
+9.749070000000000e+02 +3.552580000000001e+02
+9.535770000000000e+02 +2.573270000000000e+02
+3.176000000000000e+02 +1.656810000000000e+02
+4.991090000000000e+02 +3.843190000000000e+02
+1.391890000000000e+03 +4.916960000000000e+02
+4.807600000000000e+02 +1.539000000000000e+02
+1.858670000000000e+03 +9.131160000000000e+02
+6.581669999999998e+02 +1.559400000000000e+02
+1.264860000000000e+03 +5.823969999999998e+02
+1.828720000000000e+03 +9.290359999999999e+02
+1.396440000000000e+03 +4.721680000000000e+02
+1.073180000000000e+03 +5.132750000000000e+02
+1.480670000000000e+03 +7.811510000000002e+02
+5.001600000000000e+02 +3.911400000000000e+02
+9.934170000000000e+02 +3.300300000000000e+02
+1.384000000000000e+03 +4.254920000000000e+02
+4.184730000000000e+02 +1.971840000000000e+02
+1.309220000000000e+03 +5.813530000000002e+02
+9.595990000000000e+02 +3.411650000000000e+02
+1.465390000000000e+03 +7.638350000000000e+02
+4.989940000000000e+02 +3.697530000000000e+02
+3.394240000000001e+02 +1.456710000000000e+02
+1.758530000000000e+03 +9.103170000000000e+02
+2.327120000000000e+03 +8.691900000000001e+02
+9.046110000000000e+02 +6.021740000000000e+02
+7.836239999999998e+02 +3.322850000000000e+02
+1.878320000000000e+03 +8.878450000000000e+02
+4.919940000000000e+02 +3.371210000000000e+02
+1.617150000000000e+03 +6.502320000000000e+02
+7.788939999999999e+02 +3.245350000000000e+02
+9.630000000000000e+02 +3.065960000000000e+02
+9.949840000000000e+02 +3.193990000000000e+02
+9.135300000000000e+02 +4.764050000000000e+02
+7.809610000000000e+02 +3.309850000000000e+02
+7.762370000000000e+02 +3.011600000000000e+02
+7.804320000000000e+02 +3.134640000000000e+02
+1.627050000000000e+03 +5.964059999999999e+02
+7.860910000000000e+02 +4.479050000000000e+02
+4.875800000000000e+02 +3.781210000000000e+02
+2.822160000000000e+02 +7.599670000000000e+01
+7.731619999999998e+02 +3.159720000000000e+02
+7.770549999999999e+02 +2.966440000000000e+02
+7.657320000000000e+02 +3.043030000000000e+02
+7.793210000000000e+02 +3.693450000000000e+02
+3.593769999999999e+02 +1.566910000000000e+02
+6.878789999999998e+02 +4.165720000000000e+02
+4.887830000000000e+02 +3.408470000000000e+02
+1.627590000000000e+03 +6.027569999999999e+02
+7.874169999999998e+02 +3.005810000000000e+02
+7.765200000000000e+02 +3.064490000000000e+02
+1.355810000000000e+03 +4.003060000000000e+02
+8.788020000000000e+02 +6.742940000000000e+02
+1.590510000000000e+03 +5.416519999999998e+02
+3.325250000000000e+02 +1.329270000000000e+02
+9.788410000000000e+02 +3.230320000000000e+02
+7.812500000000000e+02 +3.001300000000000e+02
+6.901810000000000e+02 +4.106720000000000e+02
+7.474119999999998e+02 +2.677130000000000e+02
+7.847310000000001e+02 +3.155060000000000e+02
+1.091890000000000e+03 +4.871200000000000e+02
+5.043640000000000e+02 +3.808260000000000e+02
+7.960160000000002e+02 +3.146920000000000e+02
+1.604150000000000e+03 +5.118640000000000e+02
+2.291900000000000e+03 +1.196930000000000e+03
+7.514620000000000e+02 +2.790380000000000e+02
+9.908780000000000e+02 +3.250870000000000e+02
+7.627170000000000e+02 +2.549870000000000e+02
+7.897239999999998e+02 +3.027780000000000e+02
+1.874020000000000e+03 +9.367640000000000e+02
+4.976250000000000e+02 +3.511900000000000e+02
+3.365760000000000e+02 +1.168090000000000e+02
+9.792340000000000e+02 +3.145440000000001e+02
+1.614700000000000e+03 +5.894330000000000e+02
+7.901469999999998e+02 +2.900650000000000e+02
+1.386760000000000e+03 +6.736419999999998e+02
+1.339700000000000e+03 +7.865650000000001e+02
+7.934270000000000e+02 +3.134840000000001e+02
+1.624350000000000e+03 +7.924069999999998e+02
+9.925700000000001e+02 +3.320290000000000e+02
+1.865050000000000e+03 +8.821039999999998e+02
+9.873000000000000e+02 +3.243560000000000e+02
+7.788430000000002e+02 +2.707320000000000e+02
+7.885989999999998e+02 +3.173670000000000e+02
+1.660270000000000e+03 +5.066900000000000e+02
+7.938330000000002e+02 +2.835750000000000e+02
+1.003010000000000e+03 +3.496040000000001e+02
+1.097770000000000e+03 +5.412440000000000e+02
+1.378320000000000e+03 +6.760380000000000e+02
+1.876120000000000e+03 +8.925400000000000e+02
+4.888150000000000e+02 +3.479650000000000e+02
+7.720700000000001e+02 +2.657050000000000e+02
+9.950200000000000e+02 +3.034880000000000e+02
+7.959780000000002e+02 +2.713520000000000e+02
+7.707280000000002e+02 +2.744900000000000e+02
+4.886710000000000e+02 +1.897740000000000e+02
+6.940599999999999e+02 +4.070080000000000e+02
+1.259590000000000e+03 +5.117990000000000e+02
+1.872320000000000e+03 +9.056750000000000e+02
+1.306720000000000e+03 +4.455820000000000e+02
+7.896960000000000e+02 +3.144410000000000e+02
+8.560219999999998e+02 +3.284180000000000e+02
+1.355530000000000e+03 +6.790870000000000e+02
+1.844070000000000e+03 +8.902639999999999e+02
+5.023130000000001e+02 +3.442410000000000e+02
+7.840039999999998e+02 +2.939090000000000e+02
+1.170460000000000e+03 +5.587120000000000e+02
+7.981350000000000e+02 +2.817130000000000e+02
+6.689310000000000e+02 +2.396240000000000e+02
+7.836039999999998e+02 +2.514150000000000e+02
+1.335480000000000e+03 +6.397809999999999e+02
+1.375960000000000e+03 +6.450780000000000e+02
+1.487860000000000e+03 +7.101750000000000e+02
+4.845500000000000e+02 +3.293090000000000e+02
+1.314730000000000e+03 +4.311020000000000e+02
+1.319770000000000e+03 +4.907350000000000e+02
+7.437410000000001e+02 +2.476240000000000e+02
+3.383819999999999e+02 +7.626470000000000e+01
+1.391420000000000e+03 +6.226410000000000e+02
+7.949710000000000e+02 +2.869650000000000e+02
+9.993430000000000e+02 +3.007690000000000e+02
+7.975410000000001e+02 +2.710140000000000e+02
+1.101300000000000e+03 +4.445580000000000e+02
+9.882850000000000e+02 +3.289080000000000e+02
+1.299180000000000e+03 +4.785290000000000e+02
+1.083380000000000e+03 +4.156340000000000e+02
+8.892489999999998e+02 +2.671590000000000e+02
+1.786740000000000e+03 +8.480599999999999e+02
+6.730720000000000e+02 +6.524670000000000e+02
+1.294630000000000e+03 +4.146520000000000e+02
+1.300890000000000e+03 +4.581140000000000e+02
+7.544390000000000e+02 +2.471920000000000e+02
+1.371440000000000e+03 +6.653939999999999e+02
+8.937400000000000e+02 +3.179360000000000e+02
+1.299700000000000e+03 +3.722860000000000e+02
+1.299160000000000e+03 +4.609700000000000e+02
+7.795580000000000e+02 +2.424590000000000e+02
+1.090690000000000e+03 +4.605230000000000e+02
+1.092100000000000e+03 +3.561090000000001e+02
+3.388640000000001e+02 +1.048340000000000e+02
+1.080530000000000e+03 +6.962750000000000e+02
+8.017320000000000e+02 +2.982930000000000e+02
+1.067960000000000e+03 +4.065850000000000e+02
+8.699710000000000e+02 +2.534500000000000e+02
+9.869790000000000e+02 +2.838050000000000e+02
+7.810770000000000e+02 +2.584610000000000e+02
+1.307680000000000e+03 +4.172860000000000e+02
+1.009020000000000e+03 +4.181960000000000e+02
+8.619080000000000e+02 +2.931920000000000e+02
+1.587560000000000e+03 +7.424260000000000e+02
+9.792140000000001e+02 +2.858090000000000e+02
+1.312590000000000e+03 +3.837340000000000e+02
+7.870599999999999e+02 +2.709040000000000e+02
+1.665810000000000e+03 +6.975530000000000e+02
+6.728930000000000e+02 +6.275630000000000e+02
+1.104540000000000e+03 +4.378570000000000e+02
+1.169710000000000e+03 +5.175750000000000e+02
+7.698120000000000e+02 +2.370900000000000e+02
+3.465760000000000e+02 +1.158970000000000e+02
+1.069490000000000e+03 +5.780020000000000e+02
+9.597290000000000e+02 +3.346220000000000e+02
+1.312630000000000e+03 +4.356280000000000e+02
+1.310820000000000e+03 +3.564250000000000e+02
+1.584250000000000e+03 +7.852560000000002e+02
+7.483380000000002e+02 +2.902310000000000e+02
+7.920560000000000e+02 +2.640660000000000e+02
+1.937140000000000e+03 +7.728389999999998e+02
+6.645269999999998e+02 +6.510540000000000e+02
+6.601360000000002e+02 +1.840390000000000e+02
+1.022300000000000e+03 +3.156430000000000e+02
+1.687150000000000e+03 +7.233160000000000e+02
+1.296940000000000e+03 +3.352380000000001e+02
+7.820400000000000e+02 +2.532730000000000e+02
+9.942180000000000e+02 +3.189560000000000e+02
+9.072870000000000e+02 +3.119080000000000e+02
+1.032410000000000e+03 +3.729210000000000e+02
+7.423910000000002e+02 +2.691520000000000e+02
+7.658880000000000e+02 +2.242380000000000e+02
+9.743150000000001e+02 +3.031380000000000e+02
+8.609570000000000e+02 +2.339380000000000e+02
+7.802980000000000e+02 +2.544920000000000e+02
+1.098510000000000e+03 +3.134220000000000e+02
+1.065110000000000e+03 +4.457130000000000e+02
+1.658610000000000e+03 +7.638550000000000e+02
+6.677430000000001e+02 +5.930260000000000e+02
+1.069370000000000e+03 +3.178800000000000e+02
+1.044720000000000e+03 +3.830880000000000e+02
+7.451660000000001e+02 +2.541950000000000e+02
+7.819889999999998e+02 +2.235710000000000e+02
+7.452020000000000e+02 +2.458610000000000e+02
+1.317470000000000e+03 +3.732810000000000e+02
+1.319750000000000e+03 +3.899470000000000e+02
+1.068790000000000e+03 +4.177350000000000e+02
+7.736310000000002e+02 +2.263730000000000e+02
+1.128200000000000e+03 +3.689720000000000e+02
+9.736330000000000e+02 +2.949840000000001e+02
+1.680080000000000e+03 +7.470570000000000e+02
+7.784660000000000e+02 +2.302420000000000e+02
+7.747669999999998e+02 +2.519570000000000e+02
+7.742130000000002e+02 +2.167510000000000e+02
+1.081280000000000e+03 +3.717510000000000e+02
+1.054730000000000e+03 +4.265230000000000e+02
+8.783290000000000e+02 +3.027210000000000e+02
+7.456860000000000e+02 +2.161720000000000e+02
+7.789100000000000e+02 +2.427570000000000e+02
+8.792970000000000e+02 +2.725430000000000e+02
+7.759060000000002e+02 +2.131290000000000e+02
+1.672250000000000e+03 +7.684400000000001e+02
+1.310080000000000e+03 +3.651490000000000e+02
+7.773630000000001e+02 +2.341530000000000e+02
+1.089600000000000e+03 +3.517060000000000e+02
+4.949320000000000e+02 +1.477450000000000e+02
+7.455770000000000e+02 +2.116950000000000e+02
+1.316740000000000e+03 +3.910940000000000e+02
+1.078380000000000e+03 +4.107050000000000e+02
+1.956160000000000e+03 +7.387960000000000e+02
+7.686180000000001e+02 +2.140490000000000e+02
+1.678030000000000e+03 +7.171360000000002e+02
+9.024870000000000e+02 +2.591460000000000e+02
+8.613099999999999e+02 +2.465960000000000e+02
+7.443110000000000e+02 +2.120170000000000e+02
+7.727950000000000e+02 +2.291270000000000e+02
+7.820350000000000e+02 +2.114760000000000e+02
+2.976640000000000e+02 +2.620500000000000e+02
+7.692739999999999e+02 +2.342550000000000e+02
+8.789220000000000e+02 +2.841050000000000e+02
+9.946600000000000e+02 +3.180810000000000e+02
+7.448439999999998e+02 +2.164940000000000e+02
+7.871410000000002e+02 +2.074750000000000e+02
+7.819480000000000e+02 +2.346660000000000e+02
+1.061000000000000e+03 +3.990810000000000e+02
+1.305290000000000e+03 +3.672600000000000e+02
+8.344010000000002e+02 +2.548420000000000e+02
+1.099860000000000e+03 +4.185250000000000e+02
+9.031670000000000e+02 +2.948540000000001e+02
+7.463270000000000e+02 +1.987240000000000e+02
+1.312290000000000e+03 +3.790450000000000e+02
+7.838880000000000e+02 +2.245920000000000e+02
+7.826210000000002e+02 +2.392220000000000e+02
+7.464260000000000e+02 +1.964650000000000e+02
+7.817780000000000e+02 +2.089160000000000e+02
+9.087770000000000e+02 +2.574840000000000e+02
+7.595630000000000e+02 +2.074460000000000e+02
+9.836480000000000e+02 +2.887270000000001e+02
+1.139230000000000e+03 +6.529860000000000e+02
+7.442139999999998e+02 +1.885360000000000e+02
+7.829050000000000e+02 +2.033650000000000e+02
+7.897370000000000e+02 +2.343620000000000e+02
+7.051330000000000e+02 +5.187420000000000e+02
+7.698860000000002e+02 +2.254760000000000e+02
+8.852360000000001e+02 +2.664740000000000e+02
+8.716089999999998e+02 +2.571400000000000e+02
+1.986940000000000e+03 +7.427439999999998e+02
+6.992010000000000e+02 +4.106200000000000e+02
+7.756569999999998e+02 +1.880220000000000e+02
+7.758489999999998e+02 +1.922550000000000e+02
+7.793489999999998e+02 +2.258140000000000e+02
+8.584900000000000e+02 +2.515600000000000e+02
+7.064240000000000e+02 +3.533320000000000e+02
+8.966080000000002e+02 +2.574920000000000e+02
+7.699930000000001e+02 +2.181330000000000e+02
+9.048300000000000e+02 +2.743770000000000e+02
+7.721950000000001e+02 +4.291870000000000e+02
+7.035440000000000e+02 +3.475140000000000e+02
+8.294900000000000e+02 +2.280460000000000e+02
+9.662089999999999e+02 +3.612320000000000e+02
+8.926700000000000e+02 +2.508090000000000e+02
+7.596160000000001e+02 +4.118930000000000e+02
+6.995410000000001e+02 +4.533480000000000e+02
+1.014400000000000e+03 +6.201640000000000e+02
+8.499950000000000e+02 +2.391100000000000e+02
+1.325740000000000e+03 +6.180770000000000e+02
+1.378190000000000e+03 +5.289570000000000e+02
+1.080970000000000e+03 +6.468090000000000e+02
+1.061800000000000e+03 +7.496239999999998e+02
+8.970380000000000e+02 +2.273320000000000e+02
+7.195750000000000e+02 +4.710670000000000e+02
+6.597339999999998e+02 +3.641240000000000e+02
+1.408150000000000e+03 +4.790120000000000e+02
+8.723070000000000e+02 +4.651270000000000e+02
+1.464490000000000e+03 +7.070089999999999e+02
+7.214069999999998e+02 +4.804500000000000e+02
+1.231020000000000e+03 +5.477360000000000e+02
+1.014160000000000e+03 +6.115780000000000e+02
+8.976530000000000e+02 +2.987650000000000e+02
+9.169340000000000e+02 +5.017160000000000e+02
+6.884299999999999e+02 +4.146820000000000e+02
+6.562250000000000e+02 +2.800720000000000e+02
+6.922110000000000e+02 +3.314010000000000e+02
+1.809030000000000e+03 +9.399900000000000e+02
+8.730430000000000e+02 +2.109760000000000e+02
+7.705050000000000e+02 +4.437060000000000e+02
+1.235600000000000e+03 +4.963440000000000e+02
+6.689560000000000e+02 +2.959750000000000e+02
+6.448110000000000e+02 +3.340450000000000e+02
+1.076760000000000e+03 +5.757669999999998e+02
+8.833099999999999e+02 +5.363270000000000e+02
+1.845720000000000e+03 +9.623940000000000e+02
+8.271669999999998e+02 +2.058140000000000e+02
+9.001020000000000e+02 +2.257030000000000e+02
+9.920240000000000e+02 +5.500450000000000e+02
+1.487300000000000e+03 +6.893800000000000e+02
+1.014580000000000e+03 +5.844320000000000e+02
+1.820640000000000e+03 +9.781830000000000e+02
+1.747480000000000e+03 +7.770060000000002e+02
+6.415630000000000e+02 +3.330140000000000e+02
+6.787060000000000e+02 +2.703810000000000e+02
+6.612389999999998e+02 +3.070280000000000e+02
+6.497690000000000e+02 +2.973210000000000e+02
+8.884510000000000e+02 +2.278740000000000e+02
+8.918960000000002e+02 +7.235350000000000e+02
+9.076930000000000e+02 +5.147719999999998e+02
+1.078290000000000e+03 +6.389630000000002e+02
+7.077150000000000e+02 +3.942750000000000e+02
+4.998880000000000e+02 +1.836150000000000e+02
+9.808420000000000e+02 +6.462930000000000e+02
+6.397100000000000e+02 +2.883420000000000e+02
+6.571650000000000e+02 +2.400760000000000e+02
+9.415990000000000e+02 +5.270990000000000e+02
+1.051630000000000e+03 +6.190450000000000e+02
+8.968330000000002e+02 +2.334070000000000e+02
+1.221580000000000e+03 +4.940830000000000e+02
+6.822950000000000e+02 +4.462910000000000e+02
+6.352980000000000e+02 +2.311030000000000e+02
+1.640340000000000e+03 +1.001640000000000e+03
+9.216110000000000e+02 +5.012950000000000e+02
+5.123430000000002e+02 +1.783890000000000e+02
+8.946750000000000e+02 +2.169990000000000e+02
+6.699800000000000e+02 +3.089690000000000e+02
+9.274700000000000e+02 +4.685790000000000e+02
+9.197190000000001e+02 +5.143720000000000e+02
+1.620320000000000e+03 +7.159140000000000e+02
+6.369030000000000e+02 +2.044170000000000e+02
+6.794989999999998e+02 +2.209180000000000e+02
+6.358640000000000e+02 +2.731960000000000e+02
+6.379400000000001e+02 +2.743410000000000e+02
+4.552430000000001e+02 +1.320740000000000e+02
+1.294930000000000e+03 +6.221100000000000e+02
+8.820050000000000e+02 +2.265660000000000e+02
+1.071700000000000e+03 +7.747120000000000e+02
+6.338000000000000e+02 +1.981410000000000e+02
+8.531500000000000e+02 +5.622360000000000e+02
+1.589060000000000e+03 +7.342070000000000e+02
+6.579970000000000e+02 +2.484390000000000e+02
+4.566190000000000e+02 +1.235640000000000e+02
+8.606110000000001e+02 +4.859350000000000e+02
+1.053370000000000e+03 +7.394610000000000e+02
+6.398910000000000e+02 +2.570420000000000e+02
+6.572350000000000e+02 +1.990030000000000e+02
+9.063830000000000e+02 +4.646840000000000e+02
+1.586980000000000e+03 +5.759360000000000e+02
+8.914360000000000e+02 +2.184030000000000e+02
+1.791190000000000e+03 +8.313880000000000e+02
+6.195440000000000e+02 +2.317590000000000e+02
+6.479600000000000e+02 +2.488540000000000e+02
+6.581230000000000e+02 +1.810760000000000e+02
+8.933560000000001e+02 +4.567620000000000e+02
+1.562510000000000e+03 +5.518290000000002e+02
+1.779060000000000e+03 +5.943220000000000e+02
+8.702200000000000e+02 +2.166630000000000e+02
+6.594820000000000e+02 +2.498200000000000e+02
+6.490169999999998e+02 +2.509580000000000e+02
+6.558819999999999e+02 +1.877710000000000e+02
+8.054910000000001e+02 +4.195300000000000e+02
+1.567930000000000e+03 +5.580319999999998e+02
+6.905610000000000e+02 +4.608330000000000e+02
+8.672660000000002e+02 +2.735040000000000e+02
+1.577380000000000e+03 +5.286390000000000e+02
+6.659630000000002e+02 +2.412910000000000e+02
+6.624140000000000e+02 +2.462700000000000e+02
+8.876120000000000e+02 +1.992700000000000e+02
+1.073790000000000e+03 +5.370520000000000e+02
+6.512580000000000e+02 +2.071900000000000e+02
+1.626330000000000e+03 +8.942189999999998e+02
+8.970230000000000e+02 +2.135820000000000e+02
+1.665500000000000e+03 +5.419420000000000e+02
+1.374150000000000e+03 +8.099520000000000e+02
+1.619480000000000e+03 +8.167550000000000e+02
+6.370780000000000e+02 +1.995890000000000e+02
+1.632690000000000e+03 +8.834440000000000e+02
+1.069730000000000e+03 +5.415660000000000e+02
+6.402370000000000e+02 +2.391450000000000e+02
+6.361860000000000e+02 +1.790890000000000e+02
+1.623080000000000e+03 +8.896180000000001e+02
+1.611610000000000e+03 +8.669850000000000e+02
+1.083000000000000e+03 +5.714450000000001e+02
+6.502859999999999e+02 +2.421600000000000e+02
+1.568760000000000e+03 +5.199840000000000e+02
+6.371690000000000e+02 +2.194870000000000e+02
+8.072460000000002e+02 +3.824720000000000e+02
+6.367700000000000e+02 +1.954460000000000e+02
+1.600310000000000e+03 +4.863320000000000e+02
+1.562020000000000e+03 +6.912869999999998e+02
+1.081370000000000e+03 +4.823460000000000e+02
+6.449200000000000e+02 +2.160180000000000e+02
+1.586250000000000e+03 +7.850280000000000e+02
+6.800089999999999e+02 +3.566450000000000e+02
+6.502040000000002e+02 +1.681870000000000e+02
+1.561590000000000e+03 +4.839110000000000e+02
+6.572230000000002e+02 +2.046190000000000e+02
+8.963310000000000e+02 +2.227410000000000e+02
+1.072710000000000e+03 +5.989700000000000e+02
+6.385069999999999e+02 +2.058200000000000e+02
+6.524780000000002e+02 +3.345040000000000e+02
+1.557180000000000e+03 +4.461770000000000e+02
+2.318550000000000e+03 +9.706310000000000e+02
+6.343030000000000e+02 +1.738550000000000e+02
+1.005140000000000e+03 +5.692569999999999e+02
+1.664100000000000e+03 +4.763440000000000e+02
+1.058060000000000e+03 +5.005640000000000e+02
+8.026160000000001e+02 +3.690720000000000e+02
+7.089789999999998e+02 +4.584820000000000e+02
+7.184900000000000e+02 +5.174520000000000e+02
+8.201130000000001e+02 +4.658160000000000e+02
+1.389770000000000e+03 +7.461080000000002e+02
+6.471200000000000e+02 +1.625440000000000e+02
+7.080860000000000e+02 +3.308260000000000e+02
+1.082550000000000e+03 +4.633700000000000e+02
+8.180160000000002e+02 +3.805590000000000e+02
+1.068790000000000e+03 +4.654660000000000e+02
+1.080050000000000e+03 +4.556810000000000e+02
+1.334000000000000e+03 +5.859890000000000e+02
+6.642539999999998e+02 +3.658810000000000e+02
+6.709839999999998e+02 +3.178270000000000e+02
+1.060480000000000e+03 +3.754760000000000e+02
+2.322630000000000e+03 +8.329870000000000e+02
+7.216360000000002e+02 +3.536160000000000e+02
+1.574680000000000e+03 +5.356460000000000e+02
+1.062380000000000e+03 +3.748060000000000e+02
+7.097110000000000e+02 +4.001080000000000e+02
+1.875830000000000e+03 +8.521610000000002e+02
+1.063610000000000e+03 +4.127030000000000e+02
+1.847210000000000e+03 +9.721319999999999e+02
+1.635190000000000e+03 +7.846860000000000e+02
+1.064990000000000e+03 +3.026850000000000e+02
+2.155180000000000e+03 +9.179320000000000e+02
+1.081460000000000e+03 +4.406730000000000e+02
+9.743840000000000e+02 +3.401410000000000e+02
+9.924220000000000e+02 +2.167820000000000e+02
+8.513980000000000e+02 +2.591400000000000e+02
+1.855520000000000e+03 +8.273280000000000e+02
+1.611140000000000e+03 +8.429280000000000e+02
+1.064850000000000e+03 +3.125510000000000e+02
+1.085130000000000e+03 +4.237060000000000e+02
+1.563060000000000e+03 +5.292440000000000e+02
+5.919200000000000e+02 +2.162000000000000e+02
+1.494010000000000e+03 +9.372040000000000e+02
+9.811910000000000e+02 +3.609650000000000e+02
+1.088460000000000e+03 +3.964450000000000e+02
+8.975700000000001e+02 +2.120070000000000e+02
+1.270440000000000e+03 +6.038620000000000e+02
+1.333040000000000e+03 +7.361460000000002e+02
+1.225610000000000e+03 +4.259560000000000e+02
+1.732080000000000e+03 +6.205169999999998e+02
+1.004130000000000e+03 +5.408480000000002e+02
+1.013640000000000e+03 +5.094220000000000e+02
+1.006060000000000e+03 +5.079460000000000e+02
+1.005820000000000e+03 +5.172120000000000e+02
+1.064250000000000e+03 +3.053200000000000e+02
+9.746570000000000e+02 +3.324090000000000e+02
+1.289020000000000e+03 +5.352310000000000e+02
+5.700990000000000e+02 +1.833970000000000e+02
+9.759130000000000e+02 +3.368410000000000e+02
+1.589410000000000e+03 +8.494450000000001e+02
+1.296200000000000e+03 +7.732040000000000e+02
+1.044920000000000e+03 +2.705910000000000e+02
+1.094820000000000e+03 +5.702590000000000e+02
+1.010620000000000e+03 +5.353540000000000e+02
+1.011580000000000e+03 +5.057540000000000e+02
+1.024030000000000e+03 +6.265970000000000e+02
+9.809670000000000e+02 +3.191020000000001e+02
+6.551820000000000e+02 +2.152790000000000e+02
+1.017820000000000e+03 +5.316400000000000e+02
+1.559960000000000e+03 +7.759950000000000e+02
+1.217010000000000e+03 +6.445520000000000e+02
+2.822470000000000e+02 +1.600600000000000e+02
+1.080270000000000e+03 +3.324080000000000e+02
+1.071180000000000e+03 +4.018610000000000e+02
+1.406790000000000e+03 +9.476250000000000e+02
+1.007330000000000e+03 +6.195230000000000e+02
+1.329890000000000e+03 +5.504109999999999e+02
+1.011760000000000e+03 +4.826220000000000e+02
+1.012270000000000e+03 +5.445530000000000e+02
+1.448820000000000e+03 +8.049030000000000e+02
+1.085590000000000e+03 +5.352100000000000e+02
+2.783490000000000e+02 +1.611550000000000e+02
+1.065970000000000e+03 +3.786080000000000e+02
+6.550369999999998e+02 +1.824620000000000e+02
+1.005790000000000e+03 +5.415910000000000e+02
+9.806260000000000e+02 +3.404260000000000e+02
+9.755450000000000e+02 +4.544900000000000e+02
+1.007830000000000e+03 +4.920980000000000e+02
+2.753950000000000e+02 +1.267000000000000e+02
+1.078410000000000e+03 +3.094160000000000e+02
+1.618640000000000e+03 +8.656030000000002e+02
+1.414760000000000e+03 +8.886100000000000e+02
+5.046700000000000e+02 +4.242070000000000e+02
+8.991020000000000e+02 +6.708370000000000e+02
+1.116810000000000e+03 +5.329780000000002e+02
+1.066730000000000e+03 +3.537740000000000e+02
+1.442600000000000e+03 +8.174950000000000e+02
+9.830010000000000e+02 +3.496120000000000e+02
+8.823200000000001e+02 +3.149880000000000e+02
+1.845440000000000e+03 +8.695400000000000e+02
+2.778890000000000e+02 +1.198510000000000e+02
+8.704290000000000e+02 +6.604210000000000e+02
+1.023050000000000e+03 +4.694790000000000e+02
+6.241070000000000e+02 +2.657540000000000e+02
+4.867180000000000e+02 +4.094340000000000e+02
+1.045770000000000e+03 +2.805160000000000e+02
+6.220150000000000e+02 +2.593750000000000e+02
+6.679839999999998e+02 +1.798710000000000e+02
+1.012410000000000e+03 +4.622570000000000e+02
+9.779820000000000e+02 +3.245460000000000e+02
+9.905380000000000e+02 +4.563240000000000e+02
+1.467120000000000e+03 +7.747580000000000e+02
+1.206740000000000e+03 +6.233940000000000e+02
+1.292160000000000e+03 +6.682239999999998e+02
+6.648150000000001e+02 +2.676830000000000e+02
+2.486760000000000e+03 +1.191760000000000e+03
+1.420100000000000e+03 +6.584370000000000e+02
+9.804870000000000e+02 +3.011070000000000e+02
+6.221849999999999e+02 +2.601850000000000e+02
+1.004280000000000e+03 +4.728230000000000e+02
+1.612420000000000e+03 +8.251020000000000e+02
+2.820590000000000e+02 +1.419080000000000e+02
+6.263790000000000e+02 +2.059190000000000e+02
+5.119470000000000e+02 +4.308290000000000e+02
+1.399710000000000e+03 +8.604150000000000e+02
+6.222380000000001e+02 +2.349270000000000e+02
+8.685020000000000e+02 +2.787520000000000e+02
+2.683440000000000e+03 +1.244070000000000e+03
+6.643950000000000e+02 +2.570630000000000e+02
+6.678330000000002e+02 +2.611300000000000e+02
+8.778900000000000e+02 +2.651820000000000e+02
+9.926340000000000e+02 +3.308810000000000e+02
+6.268480000000002e+02 +1.967370000000000e+02
+3.343980000000000e+02 +1.686010000000000e+02
+6.230680000000000e+02 +2.071100000000000e+02
+6.616050000000000e+02 +1.954190000000000e+02
+8.800530000000000e+02 +2.627650000000000e+02
+7.600210000000002e+02 +2.589410000000000e+02
+8.786080000000002e+02 +2.463480000000000e+02
+1.134930000000000e+03 +4.481600000000000e+02
+1.081890000000000e+03 +3.858050000000000e+02
+1.362780000000000e+03 +4.618550000000000e+02
+9.298880000000000e+02 +5.852520000000000e+02
+9.885220000000000e+02 +3.208510000000000e+02
+4.877800000000000e+02 +3.917480000000001e+02
+1.590520000000000e+03 +5.697890000000000e+02
+6.234270000000000e+02 +1.711080000000000e+02
+6.208480000000002e+02 +2.000400000000000e+02
+1.376980000000000e+03 +4.815960000000000e+02
+1.664880000000000e+03 +8.071220000000000e+02
+2.664660000000000e+03 +1.219950000000000e+03
+3.374390000000000e+02 +1.595000000000000e+02
+6.626870000000000e+02 +1.807220000000000e+02
+4.920720000000000e+02 +4.036390000000000e+02
+1.078330000000000e+03 +4.081630000000000e+02
+1.076430000000000e+03 +4.029140000000000e+02
+1.750400000000000e+03 +1.066250000000000e+03
+1.455560000000000e+03 +7.628839999999999e+02
+7.480520000000000e+02 +1.943590000000000e+02
+1.580630000000000e+03 +8.302539999999998e+02
+1.008620000000000e+03 +3.526620000000000e+02
+9.815690000000000e+02 +2.589030000000000e+02
+9.945480000000000e+02 +2.804900000000000e+02
+6.399970000000000e+02 +1.809050000000000e+02
+6.357710000000000e+02 +1.738540000000000e+02
+4.841650000000000e+02 +3.795810000000000e+02
+1.278520000000000e+03 +6.193099999999999e+02
+1.069670000000000e+03 +4.225460000000000e+02
+8.930119999999999e+02 +2.772010000000000e+02
+6.190670000000000e+02 +1.585960000000000e+02
+4.891080000000000e+02 +3.945910000000000e+02
+2.669520000000000e+03 +1.188340000000000e+03
+1.209940000000000e+03 +4.412820000000000e+02
+1.759230000000000e+03 +9.485549999999999e+02
+1.857380000000000e+03 +9.810980000000000e+02
+1.410570000000000e+03 +3.947940000000000e+02
+1.054600000000000e+03 +3.631020000000000e+02
+1.089760000000000e+03 +4.420790000000000e+02
+9.792150000000000e+02 +2.738610000000000e+02
+8.915820000000000e+02 +5.437210000000000e+02
+1.091970000000000e+03 +4.539200000000000e+02
+3.333490000000000e+02 +1.257570000000000e+02
+6.204209999999998e+02 +1.628430000000000e+02
+8.933720000000000e+02 +2.515270000000000e+02
+4.855590000000000e+02 +3.755180000000000e+02
+1.108280000000000e+03 +3.774810000000000e+02
+1.865840000000000e+03 +9.700200000000000e+02
+7.728600000000000e+02 +3.055930000000000e+02
+1.061110000000000e+03 +4.152820000000000e+02
+7.871500000000000e+02 +5.203620000000000e+02
+6.574430000000000e+02 +1.381170000000000e+02
+1.634420000000000e+03 +6.349940000000000e+02
+7.885290000000000e+02 +3.164480000000000e+02
+7.726489999999999e+02 +2.698990000000000e+02
+8.927589999999999e+02 +2.599220000000000e+02
+1.082510000000000e+03 +3.702010000000000e+02
+7.752530000000000e+02 +2.782660000000000e+02
+1.113670000000000e+03 +3.738130000000001e+02
+6.356170000000000e+02 +2.309340000000000e+02
+1.588440000000000e+03 +5.437740000000000e+02
+7.856260000000002e+02 +5.305570000000000e+02
+1.071270000000000e+03 +5.059510000000000e+02
+1.227160000000000e+03 +5.206840000000000e+02
+9.820700000000001e+02 +2.793070000000000e+02
+7.789370000000000e+02 +2.771110000000000e+02
+1.778940000000000e+03 +8.565820000000000e+02
+4.831750000000000e+02 +1.993750000000000e+02
+1.377250000000000e+03 +7.479380000000000e+02
+1.591620000000000e+03 +5.561830000000000e+02
+9.237230000000000e+02 +5.910400000000000e+02
+4.978180000000000e+02 +3.956160000000000e+02
+7.842300000000000e+02 +2.728390000000000e+02
+9.776540000000000e+02 +2.797920000000000e+02
+7.857669999999998e+02 +2.920950000000000e+02
+1.350620000000000e+03 +6.869400000000001e+02
+9.797340000000000e+02 +2.820760000000000e+02
+1.476300000000000e+03 +8.117170000000000e+02
+7.823839999999999e+02 +2.907050000000000e+02
+1.798820000000000e+03 +8.432360000000001e+02
+8.932289999999998e+02 +2.567890000000000e+02
+6.898470000000000e+02 +5.040870000000000e+02
+1.869740000000000e+03 +1.032950000000000e+03
+4.958330000000000e+02 +3.929690000000000e+02
+7.856849999999999e+02 +2.628590000000000e+02
+7.767719999999998e+02 +2.870780000000000e+02
+7.714349999999999e+02 +2.691390000000000e+02
+8.681810000000000e+02 +4.848300000000000e+02
+4.814680000000000e+02 +1.921730000000000e+02
+1.067490000000000e+03 +4.261840000000000e+02
+1.330640000000000e+03 +6.808410000000000e+02
+7.550440000000000e+02 +2.621540000000000e+02
+8.846480000000000e+02 +2.707370000000000e+02
+1.479970000000000e+03 +8.250889999999998e+02
+7.887600000000000e+02 +2.744350000000000e+02
+7.921000000000000e+02 +2.724990000000000e+02
+1.680780000000000e+03 +7.703730000000000e+02
+8.494860000000001e+02 +2.273750000000000e+02
+1.083580000000000e+03 +3.906950000000000e+02
+1.262920000000000e+03 +5.088540000000000e+02
+1.486280000000000e+03 +7.708330000000002e+02
+1.331130000000000e+03 +6.640580000000000e+02
+9.293240000000000e+02 +5.034070000000000e+02
+4.898690000000000e+02 +3.809840000000000e+02
+8.978480000000002e+02 +2.823650000000000e+02
+7.506510000000002e+02 +2.343180000000000e+02
+1.122080000000000e+03 +4.119730000000000e+02
+1.490830000000000e+03 +7.559839999999998e+02
+8.990810000000000e+02 +2.443100000000000e+02
+8.882350000000000e+02 +5.435590000000000e+02
+4.851450000000000e+02 +3.791970000000000e+02
+7.883810000000002e+02 +2.578550000000000e+02
+7.897669999999998e+02 +2.741000000000000e+02
+1.447860000000000e+03 +4.448030000000001e+02
+1.387410000000000e+03 +6.877910000000001e+02
+7.674110000000002e+02 +2.268100000000000e+02
+1.574970000000000e+03 +7.909220000000000e+02
+1.323760000000000e+03 +7.028250000000000e+02
+2.303800000000000e+03 +1.050460000000000e+03
+7.688270000000000e+02 +2.530620000000000e+02
+1.106890000000000e+03 +3.379500000000000e+02
+6.645069999999999e+02 +7.323919999999998e+02
+7.523550000000000e+02 +2.280450000000000e+02
+7.819510000000000e+02 +2.689700000000000e+02
+7.490310000000002e+02 +5.372950000000000e+02
+1.570120000000000e+03 +6.258670000000000e+02
+2.285540000000000e+03 +1.112560000000000e+03
+9.091660000000001e+02 +2.899000000000000e+02
+8.961500000000000e+02 +2.591280000000000e+02
+9.351220000000000e+02 +6.030970000000000e+02
+2.288310000000000e+03 +1.086660000000000e+03
+7.734989999999998e+02 +2.646710000000000e+02
+1.331520000000000e+03 +6.562950000000000e+02
+1.303500000000000e+03 +7.020590000000000e+02
+7.708600000000000e+02 +2.221050000000000e+02
+8.711430000000000e+02 +2.654980000000000e+02
+7.612539999999998e+02 +5.080530000000001e+02
+7.847160000000000e+02 +2.275210000000000e+02
+4.840740000000000e+02 +1.546550000000000e+02
+1.848500000000000e+03 +9.301070000000000e+02
+9.162560000000000e+02 +2.908850000000000e+02
+1.306430000000000e+03 +4.736850000000000e+02
+8.760230000000000e+02 +2.455370000000000e+02
+1.324270000000000e+03 +6.501950000000001e+02
+7.858980000000000e+02 +2.231530000000000e+02
+7.905730000000000e+02 +2.606980000000000e+02
+1.066090000000000e+03 +6.726500000000000e+02
+1.294980000000000e+03 +4.523160000000000e+02
+8.295260000000002e+02 +2.093490000000000e+02
+1.490780000000000e+03 +6.981080000000002e+02
+9.014610000000000e+02 +2.280480000000000e+02
+7.802000000000000e+02 +2.164930000000000e+02
+1.594770000000000e+03 +8.408060000000000e+02
+1.858860000000000e+03 +1.025150000000000e+03
+9.581900000000001e+02 +3.070550000000000e+02
+7.817189999999998e+02 +2.206330000000000e+02
+7.744240000000000e+02 +2.592610000000000e+02
+1.305670000000000e+03 +4.183050000000000e+02
+1.462350000000000e+03 +6.966380000000000e+02
+1.315860000000000e+03 +4.444790000000000e+02
+7.817780000000000e+02 +2.483390000000000e+02
+9.116750000000000e+02 +2.345870000000000e+02
+7.860480000000000e+02 +2.202360000000000e+02
+1.294830000000000e+03 +4.210720000000000e+02
+8.862960000000000e+02 +2.168840000000000e+02
+1.298210000000000e+03 +4.593170000000000e+02
+9.696079999999999e+02 +3.149760000000000e+02
+7.572710000000002e+02 +2.094840000000000e+02
+6.451870000000000e+02 +6.485950000000000e+02
+1.320640000000000e+03 +4.564990000000000e+02
+8.875950000000000e+02 +2.157760000000000e+02
+7.806450000000000e+02 +2.337400000000000e+02
+7.766790000000000e+02 +2.099900000000000e+02
+8.641189999999998e+02 +2.531560000000000e+02
+8.695549999999999e+02 +2.212780000000000e+02
+3.250050000000000e+02 +7.603730000000000e+01
+7.675219999999998e+02 +2.015220000000000e+02
+7.810360000000002e+02 +2.256430000000000e+02
+1.309880000000000e+03 +4.899030000000000e+02
+8.779540000000000e+02 +2.092100000000000e+02
+7.816180000000001e+02 +2.088570000000000e+02
+7.910930000000002e+02 +2.392970000000000e+02
+9.779950000000000e+02 +2.923390000000000e+02
+1.299220000000000e+03 +3.794000000000000e+02
+9.750280000000000e+02 +2.971590000000000e+02
+1.306830000000000e+03 +4.331500000000000e+02
+7.471630000000000e+02 +2.037330000000000e+02
+7.688489999999998e+02 +2.148840000000000e+02
+1.296910000000000e+03 +3.937500000000000e+02
+1.020450000000000e+03 +9.267740000000000e+02
+2.174620000000000e+03 +1.264970000000000e+03
+1.394640000000000e+03 +1.076410000000000e+03
+1.415470000000000e+03 +1.085240000000000e+03
+1.699310000000000e+03 +1.202780000000000e+03
+9.874700000000000e+02 +9.116559999999999e+02
+9.637840000000000e+02 +9.266600000000000e+02
+2.358990000000000e+03 +1.190150000000000e+03
+9.903240000000000e+02 +9.106840000000000e+02
+1.360800000000000e+03 +1.026300000000000e+03
+1.655260000000000e+03 +1.105250000000000e+03
+1.367260000000000e+03 +1.048380000000000e+03
+1.809830000000000e+03 +1.126570000000000e+03
+1.362030000000000e+03 +1.038580000000000e+03
+1.789210000000000e+03 +1.080630000000000e+03
+1.829130000000000e+02 +7.418360000000000e+02
+1.350870000000000e+03 +1.013790000000000e+03
+8.403130000000000e+02 +8.636530000000000e+02
+2.621520000000000e+03 +1.173080000000000e+03
+3.283630000000001e+02 +7.526189999999998e+02
+9.705810000000000e+02 +8.682370000000000e+02
+1.268130000000000e+03 +9.511460000000000e+02
+1.797950000000000e+03 +1.075540000000000e+03
+1.714840000000000e+02 +7.308240000000000e+02
+1.358360000000000e+03 +1.033460000000000e+03
+1.235910000000000e+03 +9.633750000000000e+02
+1.446990000000000e+03 +1.043410000000000e+03
+1.727790000000000e+03 +1.042910000000000e+03
+9.973840000000000e+02 +8.966139999999998e+02
+1.325460000000000e+03 +9.641660000000001e+02
+1.476700000000000e+03 +9.891420000000001e+02
+1.181600000000000e+03 +9.415330000000000e+02
+8.411560000000002e+02 +8.274540000000000e+02
+1.538320000000000e+03 +9.965160000000000e+02
+1.825580000000000e+03 +9.825100000000000e+02
+8.380599999999999e+02 +8.451810000000000e+02
+1.234830000000000e+03 +9.436060000000000e+02
+1.804490000000000e+03 +9.710880000000000e+02
+1.230570000000000e+03 +9.247270000000000e+02
+1.035990000000000e+03 +8.725490000000000e+02
+3.029180000000000e+02 +6.718310000000000e+02
+1.915630000000000e+03 +9.874750000000000e+02
+1.582980000000000e+03 +9.711340000000000e+02
+1.341550000000000e+03 +9.640330000000000e+02
+1.846260000000000e+03 +9.682660000000000e+02
+1.835020000000000e+03 +1.014550000000000e+03
+8.382160000000000e+02 +7.776350000000000e+02
+1.075590000000000e+03 +8.035020000000000e+02
+9.359710000000000e+02 +7.937960000000000e+02
+9.836330000000000e+02 +8.057270000000000e+02
+2.722150000000000e+02 +6.739780000000002e+02
+9.878070000000000e+02 +8.430430000000000e+02
+1.197420000000000e+03 +8.628969999999998e+02
+9.662440000000000e+02 +7.821560000000002e+02
+9.511480000000000e+02 +7.979900000000000e+02
+9.206860000000000e+02 +7.787950000000000e+02
+1.041490000000000e+03 +8.446720000000000e+02
+9.164100000000000e+02 +7.660020000000000e+02
+1.025440000000000e+03 +8.789460000000000e+02
+1.485660000000000e+03 +9.102569999999999e+02
+1.419000000000000e+03 +9.044960000000000e+02
+1.208660000000000e+03 +8.649400000000001e+02
+9.509860000000000e+02 +7.704330000000000e+02
+9.277120000000000e+02 +7.981260000000002e+02
+1.331510000000000e+03 +8.638589999999998e+02
+8.611910000000000e+02 +7.387420000000000e+02
+2.901320000000000e+02 +6.671880000000000e+02
+1.865600000000000e+03 +9.623900000000000e+02
+9.188060000000000e+02 +7.697180000000002e+02
+8.358180000000000e+02 +7.289989999999998e+02
+9.010800000000000e+02 +7.839610000000000e+02
+8.972930000000000e+02 +7.612370000000000e+02
+1.247560000000000e+03 +8.436300000000000e+02
+7.816780000000000e+02 +7.446860000000000e+02
+9.221470000000000e+02 +7.637380000000001e+02
+1.166420000000000e+03 +8.297500000000000e+02
+6.741630000000000e+02 +7.427150000000000e+02
+1.310110000000000e+03 +8.553520000000000e+02
+1.538580000000000e+03 +8.818980000000000e+02
+8.640530000000000e+02 +7.367689999999999e+02
+1.896150000000000e+03 +9.817130000000000e+02
+2.841880000000000e+02 +6.337859999999999e+02
+2.585660000000000e+02 +6.299209999999998e+02
+1.431180000000000e+03 +8.533300000000000e+02
+6.522430000000001e+02 +7.307030000000000e+02
+1.849430000000000e+03 +9.326390000000000e+02
+2.722010000000000e+02 +6.222230000000002e+02
+1.370630000000000e+03 +8.675820000000000e+02
+1.864220000000000e+03 +9.250660000000000e+02
+1.102730000000000e+03 +7.999340000000000e+02
+6.696350000000000e+02 +7.241500000000000e+02
+1.279040000000000e+03 +8.217530000000000e+02
+1.300090000000000e+03 +8.580680000000000e+02
+1.894520000000000e+03 +9.551500000000000e+02
+7.828420000000000e+02 +7.479250000000000e+02
+1.285720000000000e+03 +8.194910000000001e+02
+9.953500000000000e+02 +8.188830000000000e+02
+9.021920000000000e+02 +7.054450000000001e+02
+1.481420000000000e+03 +8.651310000000002e+02
+1.091760000000000e+03 +7.713810000000002e+02
+1.880570000000000e+03 +9.861960000000000e+02
+8.520169999999998e+02 +7.018300000000000e+02
+7.703370000000000e+02 +7.330369999999998e+02
+8.607830000000000e+02 +7.387010000000000e+02
+1.273030000000000e+03 +8.329440000000000e+02
+8.046730000000000e+02 +7.096569999999998e+02
+2.654310000000000e+02 +6.050130000000000e+02
+8.128700000000000e+02 +7.120850000000000e+02
+1.075850000000000e+03 +7.686540000000000e+02
+1.267240000000000e+03 +7.945290000000000e+02
+8.083489999999998e+02 +7.110189999999999e+02
+1.196520000000000e+03 +8.011170000000000e+02
+7.733339999999999e+02 +7.349530000000000e+02
+1.688430000000000e+03 +8.778930000000000e+02
+1.259170000000000e+03 +8.127940000000000e+02
+8.001790000000000e+02 +6.914240000000000e+02
+1.292170000000000e+03 +7.838190000000000e+02
+7.711389999999999e+02 +7.232180000000002e+02
+1.300890000000000e+03 +8.050780000000000e+02
+1.285210000000000e+03 +8.085599999999999e+02
+7.847370000000000e+02 +7.184480000000000e+02
+7.891310000000002e+02 +7.040319999999998e+02
+1.151220000000000e+03 +7.456389999999999e+02
+1.047530000000000e+03 +7.979950000000000e+02
+1.265880000000000e+03 +7.881590000000000e+02
+7.735790000000000e+02 +6.972650000000000e+02
+7.357410000000001e+02 +6.998520000000000e+02
+7.878620000000000e+02 +7.096940000000000e+02
+2.603160000000000e+02 +5.880800000000000e+02
+1.849670000000000e+03 +9.144550000000000e+02
+2.525920000000000e+02 +5.836060000000000e+02
+7.952470000000000e+02 +6.825860000000000e+02
+1.870290000000000e+03 +9.083660000000000e+02
+1.070460000000000e+03 +7.523480000000002e+02
+7.793850000000000e+02 +6.994480000000000e+02
+1.482070000000000e+03 +8.239860000000001e+02
+5.035680000000000e+02 +6.714520000000000e+02
+7.925760000000000e+01 +5.528160000000000e+02
+1.254900000000000e+03 +7.832270000000000e+02
+7.563320000000000e+02 +7.029480000000000e+02
+1.090080000000000e+03 +7.823720000000000e+02
+4.932260000000000e+02 +6.733869999999999e+02
+7.860080000000000e+02 +6.803210000000000e+02
+8.185110000000002e+02 +7.398960000000002e+02
+8.004970000000000e+02 +6.948789999999998e+02
+8.464480000000000e+02 +6.985369999999998e+02
+1.294390000000000e+03 +8.017470000000000e+02
+6.935139999999999e+02 +6.703140000000000e+02
+5.514840000000000e+02 +6.852800000000000e+02
+1.187400000000000e+03 +7.586580000000000e+02
+1.142480000000000e+03 +7.456000000000000e+02
+6.812360000000001e+02 +6.896930000000000e+02
+7.910910000000000e+02 +7.076110000000001e+02
+9.241350000000000e+02 +6.722110000000000e+02
+6.907589999999999e+02 +6.927030000000000e+02
+1.804930000000000e+03 +8.741419999999998e+02
+2.410080000000000e+02 +5.550210000000000e+02
+1.931210000000000e+03 +8.552769999999998e+02
+9.220740000000000e+02 +6.733980000000000e+02
+7.027760000000002e+02 +6.607230000000002e+02
+7.209130000000000e+02 +6.513420000000000e+02
+6.976720000000000e+02 +6.651400000000000e+02
+7.072539999999998e+02 +6.731120000000000e+02
+9.783440000000001e+02 +7.333960000000002e+02
+2.583200000000000e+02 +5.578500000000000e+02
+2.168880000000000e+02 +5.556160000000000e+02
+7.699090000000000e+02 +6.595020000000000e+02
+7.291089999999998e+02 +6.704000000000000e+02
+6.778480000000002e+02 +6.451369999999999e+02
+6.958889999999999e+02 +6.509810000000000e+02
+6.737010000000000e+02 +6.596890000000000e+02
+7.567869999999998e+02 +6.507090000000002e+02
+2.328130000000000e+02 +5.653210000000000e+02
+9.319349999999999e+02 +7.082250000000000e+02
+1.114290000000000e+03 +7.333150000000001e+02
+7.142739999999999e+02 +6.554870000000000e+02
+7.518170000000000e+02 +6.669530000000000e+02
+7.122350000000000e+02 +6.613230000000000e+02
+2.519970000000000e+02 +5.553250000000000e+02
+1.115800000000000e+03 +7.020710000000000e+02
+1.586350000000000e+03 +8.186640000000000e+02
+1.163740000000000e+03 +7.190760000000000e+02
+7.679630000000002e+02 +6.685150000000000e+02
+7.521669999999998e+02 +6.526210000000000e+02
+4.613470000000000e+02 +6.290000000000000e+02
+6.469119999999998e+02 +6.439540000000002e+02
+2.543990000000000e+02 +5.546160000000000e+02
+7.480419999999998e+02 +6.679349999999999e+02
+7.378680000000001e+02 +6.400590000000000e+02
+1.231270000000000e+03 +7.876799999999999e+02
+4.845190000000000e+02 +6.189169999999998e+02
+1.181750000000000e+03 +7.355390000000000e+02
+6.664980000000000e+02 +6.366380000000000e+02
+7.565760000000000e+02 +6.765210000000002e+02
+4.665940000000000e+02 +6.183819999999999e+02
+4.602380000000001e+02 +6.253980000000000e+02
+9.667180000000000e+02 +6.819650000000000e+02
+8.507930000000000e+02 +6.674980000000000e+02
+9.988400000000000e+02 +7.096940000000000e+02
+6.438910000000000e+02 +6.323000000000000e+02
+9.008360000000000e+02 +6.656600000000000e+02
+7.670750000000000e+02 +6.633200000000001e+02
+7.652160000000000e+02 +6.602089999999999e+02
+1.342270000000000e+03 +7.298860000000002e+02
+6.899430000000000e+02 +6.279820000000000e+02
+8.138160000000000e+02 +6.600100000000000e+02
+1.210190000000000e+03 +7.477220000000000e+02
+7.592619999999999e+02 +6.548360000000000e+02
+4.475740000000000e+02 +6.064180000000000e+02
+8.907470000000000e+02 +6.797080000000002e+02
+1.156600000000000e+03 +7.276260000000002e+02
+2.185480000000000e+02 +5.216830000000000e+02
+7.434340000000000e+02 +6.398460000000000e+02
+2.394020000000000e+02 +5.325150000000000e+02
+1.340860000000000e+03 +7.799370000000000e+02
+7.358580000000002e+02 +6.412909999999998e+02
+4.336640000000000e+02 +5.914169999999998e+02
+9.475560000000000e+02 +6.734019999999998e+02
+1.098050000000000e+03 +7.002830000000000e+02
+2.321540000000000e+02 +5.328869999999999e+02
+6.851750000000000e+02 +6.144019999999998e+02
+4.119740000000000e+02 +5.894440000000000e+02
+3.994740000000000e+02 +5.807980000000000e+02
+9.489860000000000e+02 +6.473690000000000e+02
+6.044930000000001e+02 +6.062030000000000e+02
+6.973020000000000e+02 +6.458330000000002e+02
+8.556020000000000e+02 +6.646280000000000e+02
+6.035860000000000e+02 +6.185080000000000e+02
+7.167850000000000e+02 +6.321060000000000e+02
+6.053750000000000e+02 +6.214310000000000e+02
+7.100119999999999e+02 +6.341220000000000e+02
+3.987690000000000e+02 +5.784850000000000e+02
+3.370390000000000e+02 +5.648270000000000e+02
+3.724370000000000e+02 +5.812719999999998e+02
+9.894290000000000e+02 +6.868860000000002e+02
+2.110120000000000e+02 +5.064770000000000e+02
+7.106239999999998e+02 +6.330820000000000e+02
+8.669270000000000e+02 +6.513099999999999e+02
+1.066710000000000e+03 +6.796289999999998e+02
+5.830050000000000e+02 +6.028290000000002e+02
+6.346040000000000e+02 +6.104700000000000e+02
+1.091920000000000e+03 +7.152200000000000e+02
+8.570269999999998e+02 +6.320040000000000e+02
+4.731080000000000e+02 +6.009159999999998e+02
+7.749810000000001e+02 +6.293640000000000e+02
+1.579520000000000e+03 +8.053320000000000e+02
+3.576110000000000e+02 +5.692760000000000e+02
+7.734530000000000e+02 +6.211750000000000e+02
+3.603360000000000e+02 +5.677060000000000e+02
+5.722070000000000e+02 +5.932569999999999e+02
+6.225190000000000e+02 +5.818400000000000e+02
+6.452790000000000e+02 +6.127250000000000e+02
+9.972770000000000e+02 +6.601750000000000e+02
+8.170530000000000e+02 +6.452680000000000e+02
+9.320430000000000e+02 +6.725050000000000e+02
+1.030080000000000e+03 +6.980350000000000e+02
+5.555710000000000e+02 +5.945480000000000e+02
+9.969720000000000e+02 +6.559060000000002e+02
+1.418140000000000e+03 +7.012650000000000e+02
+9.425100000000000e+02 +6.148230000000000e+02
+6.418090000000000e+02 +6.094030000000000e+02
+3.772340000000000e+02 +5.623099999999999e+02
+5.485419999999998e+02 +5.818400000000000e+02
+2.028690000000000e+02 +4.942240000000000e+02
+3.849690000000000e+02 +5.674520000000000e+02
+5.132840000000000e+02 +5.780880000000002e+02
+9.255250000000000e+02 +6.682010000000000e+02
+7.108410000000000e+02 +6.095800000000000e+02
+7.731369999999999e+02 +6.129109999999999e+02
+3.532680000000001e+02 +5.581430000000000e+02
+9.856470000000000e+02 +6.526870000000000e+02
+8.444750000000000e+02 +6.402840000000000e+02
+9.655820000000000e+02 +6.493900000000000e+02
+3.692870000000000e+02 +5.554820000000000e+02
+3.689780000000000e+02 +5.617880000000000e+02
+6.474500000000000e+02 +5.722590000000000e+02
+6.537180000000002e+02 +5.914230000000000e+02
+9.652540000000000e+02 +6.808010000000000e+02
+7.053210000000000e+02 +6.016500000000000e+02
+1.004270000000000e+03 +6.738210000000000e+02
+7.639330000000000e+02 +5.716110000000000e+02
+5.166940000000000e+02 +5.880369999999998e+02
+5.078040000000000e+02 +5.806020000000000e+02
+3.974870000000000e+02 +5.675990000000000e+02
+5.200040000000000e+02 +5.798590000000000e+02
+8.615210000000002e+02 +6.236160000000000e+02
+3.310740000000000e+02 +5.285830000000002e+02
+5.209150000000000e+02 +5.827150000000000e+02
+3.556260000000000e+02 +5.557950000000000e+02
+1.309510000000000e+03 +7.555450000000000e+02
+6.598720000000000e+02 +5.698410000000000e+02
+3.432680000000001e+02 +5.454080000000000e+02
+8.677660000000002e+02 +6.278869999999999e+02
+9.425270000000000e+02 +6.510280000000000e+02
+3.579010000000000e+02 +5.507730000000000e+02
+6.515180000000000e+02 +5.734430000000000e+02
+6.444780000000002e+02 +5.955990000000000e+02
+9.013900000000000e+02 +6.227140000000001e+02
+7.943620000000000e+02 +6.001469999999998e+02
+6.895239999999999e+02 +5.680630000000000e+02
+3.260870000000000e+02 +5.438080000000000e+02
+5.797909999999998e+02 +5.793580000000002e+02
+6.943310000000000e+02 +6.017040000000002e+02
+9.348760000000000e+02 +6.589410000000000e+02
+9.427120000000000e+02 +6.465800000000000e+02
+5.224790000000000e+02 +5.663220000000000e+02
+5.015290000000000e+02 +5.528680000000001e+02
+4.720980000000000e+02 +5.500640000000000e+02
+7.803300000000000e+02 +6.248940000000000e+02
+1.004510000000000e+03 +6.799069999999998e+02
+9.297240000000000e+02 +6.268150000000001e+02
+6.818090000000000e+02 +5.814340000000000e+02
+1.053190000000000e+03 +6.484320000000000e+02
+3.409060000000000e+02 +5.363300000000000e+02
+7.831840000000000e+02 +6.289460000000000e+02
+5.461600000000000e+02 +5.695590000000000e+02
+7.547639999999999e+02 +6.068270000000000e+02
+6.812830000000000e+02 +5.806419999999998e+02
+3.340600000000000e+02 +5.338580000000002e+02
+6.735839999999999e+02 +5.875690000000000e+02
+5.204670000000000e+02 +5.481799999999999e+02
+7.552730000000000e+02 +5.790260000000000e+02
+6.565830000000002e+02 +5.828280000000000e+02
+9.033560000000000e+02 +6.148640000000000e+02
+7.439950000000000e+02 +5.539910000000000e+02
+4.722760000000000e+02 +5.488230000000000e+02
+1.053110000000000e+03 +6.928960000000002e+02
+5.216990000000002e+02 +5.612560000000000e+02
+7.373700000000000e+02 +5.536480000000000e+02
+6.691920000000000e+02 +5.630700000000001e+02
+7.965369999999998e+02 +6.222290000000000e+02
+3.470790000000000e+02 +5.344540000000002e+02
+1.250350000000000e+03 +7.486610000000002e+02
+7.257250000000000e+02 +5.909640000000001e+02
+1.016720000000000e+03 +6.183270000000000e+02
+7.078700000000000e+02 +5.799910000000000e+02
+9.279640000000001e+02 +6.107710000000000e+02
+5.084970000000000e+02 +5.446750000000000e+02
+1.363510000000000e+03 +6.794530000000000e+02
+9.104240000000000e+02 +6.075790000000002e+02
+6.512250000000000e+02 +5.637640000000000e+02
+5.142270000000000e+02 +5.508550000000000e+02
+6.213460000000000e+02 +5.538220000000000e+02
+4.676730000000000e+02 +5.330640000000000e+02
+4.411210000000000e+02 +5.401290000000000e+02
+7.083520000000000e+02 +5.396270000000000e+02
+7.236860000000000e+02 +5.530980000000002e+02
+8.719739999999998e+02 +5.959410000000000e+02
+4.845780000000000e+02 +5.580870000000000e+02
+1.210740000000000e+03 +7.180570000000000e+02
+7.044639999999998e+02 +5.387060000000000e+02
+9.676470000000000e+02 +6.189880000000001e+02
+3.293630000000001e+02 +5.120480000000000e+02
+7.683220000000000e+02 +5.480800000000000e+02
+6.218610000000000e+02 +5.598700000000000e+02
+7.377639999999999e+02 +5.853770000000000e+02
+5.625549999999999e+02 +5.518910000000000e+02
+9.015560000000000e+02 +5.769840000000000e+02
+1.219510000000000e+03 +6.924900000000000e+02
+3.690850000000000e+02 +5.192010000000000e+02
+1.067090000000000e+03 +6.579639999999998e+02
+6.840860000000000e+02 +5.662980000000000e+02
+3.253750000000000e+02 +5.132300000000000e+02
+8.203500000000000e+02 +5.845640000000000e+02
+9.004580000000002e+02 +6.105549999999999e+02
+9.335660000000000e+02 +6.115960000000000e+02
+5.295090000000000e+02 +5.327760000000000e+02
+7.318049999999999e+02 +5.503780000000000e+02
+7.874260000000000e+02 +5.782430000000001e+02
+9.140800000000000e+02 +5.954220000000000e+02
+6.007240000000000e+02 +5.532790000000000e+02
+1.302880000000000e+03 +6.533660000000000e+02
+6.172890000000000e+02 +5.443530000000002e+02
+6.397480000000000e+02 +5.394270000000000e+02
+7.485780000000000e+02 +5.408650000000000e+02
+1.241700000000000e+02 +4.304470000000000e+02
+3.074270000000000e+02 +4.986420000000000e+02
+9.469650000000000e+02 +5.885140000000000e+02
+8.945690000000000e+02 +5.632120000000000e+02
+3.056180000000000e+02 +4.964610000000000e+02
+6.013400000000000e+02 +5.390990000000000e+02
+5.471080000000002e+02 +5.262710000000000e+02
+1.164380000000000e+03 +6.671369999999999e+02
+6.386709999999998e+02 +5.410780000000000e+02
+2.919350000000000e+02 +4.523150000000000e+02
+1.421390000000000e+03 +6.704710000000000e+02
+1.810920000000000e+03 +7.431220000000000e+02
+4.926250000000000e+02 +5.111800000000000e+02
+7.700330000000000e+02 +5.737340000000000e+02
+8.888720000000000e+02 +5.805920000000000e+02
+6.543520000000000e+02 +5.368840000000000e+02
+9.182700000000000e+02 +5.720780000000000e+02
+7.525939999999998e+02 +5.280219999999998e+02
+1.278490000000000e+03 +6.571450000000000e+02
+8.642189999999998e+02 +5.784420000000000e+02
+1.031740000000000e+03 +6.218610000000000e+02
+7.645230000000000e+02 +5.424720000000000e+02
+7.070180000000000e+02 +5.644870000000000e+02
+1.053600000000000e+03 +6.389910000000000e+02
+5.247390000000000e+02 +4.994680000000000e+02
+6.242480000000000e+02 +5.276280000000000e+02
+7.612730000000000e+02 +5.726920000000000e+02
+5.644410000000000e+02 +5.177610000000000e+02
+2.855340000000000e+02 +4.831300000000000e+02
+6.011910000000000e+02 +5.209109999999999e+02
+3.267320000000000e+02 +4.871150000000000e+02
+2.848990000000000e+02 +4.821060000000000e+02
+9.680990000000000e+02 +5.837430000000001e+02
+1.926670000000000e+02 +4.295360000000000e+02
+2.705740000000000e+02 +4.733570000000000e+02
+7.264080000000000e+02 +5.035850000000000e+02
+8.040490000000000e+01 +4.083900000000000e+02
+1.874710000000000e+02 +4.222180000000000e+02
+1.178580000000000e+02 +4.111970000000000e+02
+2.320450000000000e+02 +4.613550000000000e+02
+9.398430000000000e+02 +6.145300000000000e+02
+1.340610000000000e+02 +4.415630000000001e+02
+7.230920000000000e+02 +5.436840000000000e+02
+5.771210000000000e+02 +5.117470000000000e+02
+2.026470000000000e+02 +4.560190000000000e+02
+6.790150000000000e+02 +5.233580000000002e+02
+7.737239999999998e+02 +5.524670000000000e+02
+8.110419999999998e+02 +5.440880000000002e+02
+4.747360000000000e+02 +5.013990000000000e+02
+9.228690000000000e+02 +5.747360000000000e+02
+7.636920000000000e+02 +5.324990000000000e+02
+1.075200000000000e+02 +4.253090000000000e+02
+7.695150000000000e+02 +5.393060000000000e+02
+5.663500000000000e+01 +3.984120000000000e+02
+6.860280000000000e+02 +5.251930000000000e+02
+2.256130000000000e+02 +4.312640000000000e+02
+1.611750000000000e+02 +4.449600000000000e+02
+7.501980000000000e+02 +5.584960000000000e+02
+1.004220000000000e+02 +4.256420000000000e+02
+9.225970000000000e+02 +6.038250000000000e+02
+1.043190000000000e+03 +5.772090000000002e+02
+6.026100000000000e+02 +5.118810000000000e+02
+3.038180000000000e+01 +4.123190000000000e+02
+9.430839999999999e+02 +5.664550000000000e+02
+6.310610000000000e+02 +4.888290000000000e+02
+2.004960000000000e+01 +4.006070000000000e+02
+1.672030000000000e+02 +4.230600000000000e+02
+8.678860000000002e+02 +5.922380000000001e+02
+1.111410000000000e+03 +6.078420000000000e+02
+1.305700000000000e+02 +4.335060000000000e+02
+1.752030000000000e+02 +4.193000000000000e+02
+3.583270000000000e+02 +4.589130000000000e+02
+2.184000000000000e+02 +4.179690000000000e+02
+5.371080000000002e+02 +4.912910000000000e+02
+1.089940000000000e+03 +5.781369999999999e+02
+6.587680000000000e+02 +5.004860000000000e+02
+8.964789999999998e+02 +5.544520000000000e+02
+7.736900000000001e+02 +5.473190000000000e+02
+1.420170000000000e+01 +3.978200000000000e+02
+2.162840000000000e+02 +4.453530000000000e+02
+8.273080000000000e+02 +5.471990000000002e+02
+6.580380000000000e+02 +5.010300000000000e+02
+8.317960000000000e+02 +5.476400000000000e+02
+7.465660000000000e+02 +5.422020000000000e+02
+9.086290000000000e+02 +5.832669999999998e+02
+6.566830000000000e+02 +5.051660000000000e+02
+4.105240000000000e+01 +3.808060000000000e+02
+5.459280000000000e+01 +4.156660000000000e+02
+8.342230000000002e+02 +5.442370000000000e+02
+8.205330000000000e+02 +5.278260000000000e+02
+3.765330000000000e+01 +4.034630000000000e+02
+5.740069999999999e+02 +4.965220000000000e+02
+7.072320000000000e+02 +5.067280000000000e+02
+6.250520000000000e+02 +4.847060000000000e+02
+1.139760000000000e+03 +6.187590000000000e+02
+9.149680000000000e+02 +5.587050000000000e+02
+1.209950000000000e+01 +3.898970000000000e+02
+6.849130000000000e+02 +4.947250000000000e+02
+2.398020000000000e+01 +3.976360000000000e+02
+9.137590000000000e+02 +5.508869999999999e+02
+8.631150000000000e+02 +5.561669999999998e+02
+1.867090000000000e+01 +3.911560000000000e+02
+8.544589999999999e+02 +5.227980000000000e+02
+1.318860000000000e+03 +6.306980000000000e+02
+8.727450000000000e+02 +5.476070000000000e+02
+8.613500000000000e+02 +5.226230000000000e+02
+1.112030000000000e+03 +6.019630000000002e+02
+6.270040000000000e+02 +5.095420000000000e+02
+8.519430000000000e+02 +4.884160000000000e+02
+6.635030000000000e+02 +4.967890000000000e+02
+2.328670000000000e+02 +4.385520000000000e+02
+2.239760000000000e+02 +4.209140000000000e+02
+1.172010000000000e+03 +5.687380000000001e+02
+5.475180000000000e+02 +4.730260000000000e+02
+6.968230000000000e+02 +5.047730000000000e+02
+2.552480000000000e+01 +3.689540000000000e+02
+6.121280000000000e+02 +4.791880000000001e+02
+5.977890000000000e+02 +4.756200000000000e+02
+1.754970000000000e+02 +3.630200000000000e+02
+6.978969999999998e+02 +5.146540000000000e+02
+7.907760000000002e+02 +5.289140000000000e+02
+1.662710000000000e+02 +3.395160000000000e+02
+1.545500000000000e+02 +3.860350000000000e+02
+8.847940000000000e+02 +5.413440000000001e+02
+1.515170000000000e+02 +3.895400000000000e+02
+8.398639999999998e+02 +5.441920000000000e+02
+7.331619999999998e+02 +5.057710000000000e+02
+1.068030000000000e+02 +2.195550000000000e+02
+6.333560000000000e+02 +4.801690000000000e+02
+7.344250000000000e+02 +5.137350000000000e+02
+1.479710000000000e+02 +3.864150000000000e+02
+8.538660000000001e+02 +5.350030000000000e+02
+1.663920000000000e+02 +3.923230000000000e+02
+8.322389999999998e+02 +5.328560000000000e+02
+6.555350000000000e+02 +4.876880000000001e+02
+1.155210000000000e+01 +2.344760000000000e+02
+1.747460000000000e+02 +3.756360000000000e+02
+6.505440000000000e+02 +4.796560000000000e+02
+5.952580000000000e+02 +4.567690000000000e+02
+7.130419999999998e+02 +4.917670000000000e+02
+7.170760000000000e+02 +5.170530000000000e+02
+2.577220000000000e+02 +4.221010000000000e+02
+5.578500000000000e+02 +4.767080000000000e+02
+1.493040000000000e+02 +3.902370000000000e+02
+3.019980000000000e+01 +3.617870000000000e+02
+6.827120000000000e+02 +4.779280000000001e+02
+1.949910000000000e+01 +3.560900000000000e+02
+2.386600000000000e+02 +3.972670000000000e+02
+7.998989999999999e+02 +5.033170000000000e+02
+1.387760000000000e+03 +5.983240000000002e+02
+1.382070000000000e+02 +3.772280000000000e+02
+1.293960000000000e+03 +5.676380000000000e+02
+1.602460000000000e+03 +6.422750000000000e+02
+1.905020000000000e+03 +6.424840000000000e+02
+1.309860000000000e+03 +5.795650000000001e+02
+1.345880000000000e+03 +5.704050000000000e+02
+3.283090000000000e+02 +4.135700000000000e+02
+2.266570000000000e+01 +3.585550000000000e+02
+6.085549999999999e+02 +4.639150000000000e+02
+3.117710000000000e+01 +3.608480000000000e+02
+1.121370000000000e+03 +5.707270000000000e+02
+6.517540000000000e+02 +4.848840000000000e+02
+1.136940000000000e+03 +5.466000000000000e+02
+4.531090000000000e+02 +4.326360000000000e+02
+1.437320000000000e+01 +2.034310000000000e+02
+5.657260000000000e+02 +4.895230000000000e+02
+5.952950000000000e+02 +4.460000000000000e+02
+6.990030000000000e+02 +4.861740000000000e+02
+8.369950000000000e+02 +5.058580000000000e+02
+1.084310000000000e+03 +5.673540000000000e+02
+6.459970000000000e+02 +4.867890000000000e+02
+1.151980000000000e+03 +5.630830000000002e+02
+2.051950000000000e+02 +3.732080000000000e+02
+7.572630000000000e+02 +4.855960000000000e+02
+1.766300000000000e+02 +3.419650000000000e+02
+7.241130000000001e+02 +5.164019999999998e+02
+1.615090000000000e+01 +3.422540000000000e+02
+8.567800000000000e+02 +5.057320000000000e+02
+8.952539999999998e+02 +5.007660000000000e+02
+9.691220000000000e+02 +5.584370000000000e+02
+6.050300000000000e+02 +4.506920000000000e+02
+7.067050000000000e+02 +4.855910000000000e+02
+5.246200000000000e+02 +4.525100000000000e+02
+6.199460000000000e+02 +4.411630000000000e+02
+1.681160000000000e+02 +3.808220000000000e+02
+7.517650000000000e+02 +4.918200000000000e+02
+5.869230000000000e+02 +4.521310000000000e+02
+6.108490000000000e+02 +4.609860000000000e+02
+1.170730000000000e+01 +3.378450000000000e+02
+5.481810000000000e+02 +4.511940000000000e+02
+5.161310000000000e+02 +4.397390000000000e+02
+5.344200000000000e+02 +4.761990000000000e+02
+6.453510000000000e+02 +4.564330000000000e+02
+2.493730000000000e+02 +4.002430000000001e+02
+1.088070000000000e+03 +5.680480000000000e+02
+1.111620000000000e+02 +3.541020000000000e+02
+1.536320000000000e+01 +3.184120000000001e+02
+8.978429999999999e+00 +2.418750000000000e+02
+2.392920000000000e+01 +3.437050000000000e+02
+1.093490000000000e+03 +5.617410000000000e+02
+1.245310000000000e+02 +3.553710000000000e+02
+1.255120000000000e+02 +2.424740000000000e+02
+4.708770000000000e+02 +4.472650000000000e+02
+5.149209999999998e+02 +4.396660000000000e+02
+8.595180000000000e+02 +5.090880000000000e+02
+6.323020000000000e+02 +4.494650000000000e+02
+5.972170000000000e+02 +4.507810000000000e+02
+1.068380000000000e+03 +5.193620000000000e+02
+6.865630000000000e+01 +3.502840000000000e+02
+1.181130000000000e+02 +3.559280000000001e+02
+1.413720000000000e+01 +3.322060000000000e+02
+4.719500000000000e+02 +4.283390000000000e+02
+5.541740000000000e+02 +4.913720000000000e+02
+4.645580000000000e+02 +4.343750000000000e+02
+1.544940000000000e+01 +3.306890000000000e+02
+1.088470000000000e+02 +3.524890000000001e+02
+1.264580000000000e+01 +3.299680000000000e+02
+5.334410000000000e+02 +4.455950000000000e+02
+5.146540000000000e+02 +4.238340000000000e+02
+7.192930000000000e+02 +4.926240000000000e+02
+1.015690000000000e+02 +3.438150000000000e+02
+1.410020000000000e+02 +2.680610000000000e+02
+6.388490000000000e+02 +4.627160000000000e+02
+1.452300000000000e+03 +6.012780000000000e+02
+6.144500000000000e+02 +4.329410000000000e+02
+5.926540000000000e+02 +4.314850000000000e+02
+3.456100000000000e+02 +4.180180000000000e+02
+1.618440000000000e+02 +3.519220000000000e+02
+8.944060000000002e+02 +4.754340000000000e+02
+5.943670000000000e+02 +4.422400000000000e+02
+5.735610000000000e+02 +4.452550000000000e+02
+7.207110000000000e+02 +5.155670000000000e+02
+4.931610000000000e+02 +4.345430000000000e+02
+6.590640000000000e+01 +2.151510000000000e+02
+6.416110000000000e+02 +4.564130000000000e+02
+9.284930000000001e+02 +5.117910000000000e+02
+1.348620000000000e+02 +3.411250000000000e+02
+8.808930000000000e+02 +4.993470000000000e+02
+1.057310000000000e+02 +3.360860000000000e+02
+5.666830000000000e+02 +4.243190000000000e+02
+2.112340000000000e+02 +3.803580000000000e+02
+8.008380000000002e+02 +4.754620000000000e+02
+1.287990000000000e+02 +3.385410000000000e+02
+5.877100000000000e+02 +4.241200000000000e+02
+1.046200000000000e+02 +3.296310000000000e+02
+2.564270000000000e+02 +3.714480000000000e+02
+1.523280000000000e+02 +3.017680000000000e+02
+5.829330000000000e+02 +4.271080000000000e+02
+6.598030000000000e+02 +4.455280000000000e+02
+2.012010000000000e+01 +3.202530000000000e+02
+7.848099999999999e+02 +4.630440000000000e+02
+8.557180000000002e+02 +4.724450000000000e+02
+4.984740000000000e+02 +4.170640000000000e+02
+9.332100000000000e+01 +3.261880000000000e+02
+6.225100000000000e+02 +4.406120000000000e+02
+8.000490000000000e+02 +4.893530000000000e+02
+4.164150000000000e+02 +4.159820000000000e+02
+3.928830000000000e+02 +4.034070000000000e+02
+9.353420000000000e+01 +3.375300000000000e+02
+6.078260000000000e+02 +4.471460000000000e+02
+7.996339999999999e+02 +4.556940000000000e+02
+8.105690000000000e+02 +4.932480000000001e+02
+3.700970000000000e+02 +4.109580000000000e+02
+8.784980000000000e+02 +4.906200000000000e+02
+3.872470000000000e+02 +3.930930000000000e+02
+7.531380000000000e+02 +4.684810000000000e+02
+2.109980000000000e+02 +3.688510000000000e+02
+1.188590000000000e+02 +2.266610000000000e+02
+5.310630000000000e+02 +4.018070000000000e+02
+5.764990000000000e+02 +4.353700000000000e+02
+5.875380000000000e+02 +4.300240000000000e+02
+7.285200000000000e+02 +4.641910000000000e+02
+2.806230000000000e+02 +3.617950000000000e+02
+6.194290000000000e+02 +4.410420000000000e+02
+1.509010000000000e+02 +2.755760000000000e+02
+4.740890000000000e+02 +4.126630000000000e+02
+8.601030000000000e+01 +3.210150000000000e+02
+8.756260000000002e+02 +4.884170000000000e+02
+9.085359999999999e+02 +5.154130000000000e+02
+5.032530000000000e+02 +4.191720000000000e+02
+6.728620000000000e+02 +4.187610000000000e+02
+8.710630000000000e+01 +3.158170000000000e+02
+2.208770000000000e+02 +3.565130000000001e+02
+8.803220000000000e+01 +3.178860000000000e+02
+1.398430000000000e+01 +3.085960000000000e+02
+3.834820000000000e+02 +4.013200000000000e+02
+5.150720000000000e+02 +4.068390000000000e+02
+3.067360000000000e+02 +3.831050000000000e+02
+1.890140000000000e+02 +3.193440000000000e+02
+3.808890000000000e+02 +3.933290000000000e+02
+6.287120000000000e+02 +4.569680000000000e+02
+3.336900000000000e+01 +3.149190000000001e+02
+7.174440000000000e+02 +4.706550000000000e+02
+8.778789999999999e+01 +3.192290000000001e+02
+3.029580000000000e+02 +3.807760000000000e+02
+1.419900000000000e+03 +6.037619999999999e+02
+9.101880000000000e+01 +3.153510000000000e+02
+3.547020000000000e+02 +3.948960000000000e+02
+4.187490000000000e+02 +3.820280000000000e+02
+9.807089999999999e+01 +2.304990000000000e+02
+7.010770000000000e+02 +4.430940000000000e+02
+2.277820000000000e+02 +2.975170000000000e+02
+7.509450000000001e+02 +4.504220000000000e+02
+1.033480000000000e+03 +4.867620000000000e+02
+3.838650000000000e+02 +3.882410000000000e+02
+1.105660000000000e+03 +5.071550000000000e+02
+4.894600000000000e+02 +3.979660000000000e+02
+6.196569999999998e+02 +4.242630000000000e+02
+1.087830000000000e+03 +5.092580000000000e+02
+5.129280000000000e+02 +4.113860000000000e+02
+6.206669999999998e+02 +4.124370000000000e+02
+6.345050000000000e+02 +4.299440000000000e+02
+1.115830000000000e+02 +3.264860000000000e+02
+9.669040000000000e+01 +3.138940000000000e+02
+6.597970000000000e+02 +4.077370000000000e+02
+2.517190000000000e+02 +3.499100000000000e+02
+4.448560000000000e+02 +3.774670000000000e+02
+1.731530000000000e+02 +2.763970000000000e+02
+6.152960000000000e+00 +2.896690000000001e+02
+1.452320000000000e+03 +6.050630000000000e+02
+3.369510000000000e+02 +3.695940000000000e+02
+5.346170000000000e+02 +3.938230000000000e+02
+3.188010000000000e+02 +3.620000000000000e+02
+5.729610000000000e+02 +4.106270000000000e+02
+6.413049999999999e+02 +4.058070000000000e+02
+4.875960000000000e+02 +3.888000000000000e+02
+7.122070000000000e+02 +4.338840000000000e+02
+5.366319999999999e+02 +3.918340000000000e+02
+7.944720000000000e+01 +3.049640000000000e+02
+1.165390000000000e+03 +4.865470000000000e+02
+1.603220000000000e+03 +5.472669999999998e+02
+1.157820000000000e+03 +5.017930000000000e+02
+1.291380000000000e+03 +5.063060000000000e+02
+1.309970000000000e+03 +4.881180000000001e+02
+2.470190000000000e+02 +3.421800000000000e+02
+7.409100000000002e+01 +3.035230000000000e+02
+1.039750000000000e+03 +5.458930000000000e+02
+1.367300000000000e+02 +2.395090000000000e+02
+3.373400000000000e+02 +3.723570000000000e+02
+8.969020000000000e+01 +2.271290000000000e+02
+4.109930000000001e+02 +3.930520000000000e+02
+1.866630000000000e+01 +7.443980000000001e+01
+1.134160000000000e+01 +2.957390000000001e+02
+7.298950000000001e+01 +2.970070000000000e+02
+3.030800000000000e+02 +3.588280000000001e+02
+7.697660000000002e+02 +4.803950000000000e+02
+3.229320000000000e+02 +3.646810000000000e+02
+3.170850000000000e+02 +3.693600000000000e+02
+2.473100000000000e+02 +3.356540000000000e+02
+5.681350000000000e+02 +4.024140000000000e+02
+4.412540000000000e+02 +3.726380000000000e+02
+7.514130000000000e+02 +4.352230000000000e+02
+9.261870000000000e+00 +2.943370000000000e+02
+4.422970000000000e+02 +3.659250000000000e+02
+6.196830000000000e+02 +4.128340000000000e+02
+4.432640000000000e+01 +3.053030000000000e+02
+1.059030000000000e+03 +4.879990000000000e+02
+5.920240000000000e+02 +4.152410000000000e+02
+1.071620000000000e+03 +4.883520000000000e+02
+2.676140000000000e+02 +3.444310000000000e+02
+6.763920000000000e+01 +2.918150000000000e+02
+3.439420000000000e+02 +4.110820000000000e+02
+4.066970000000000e+02 +3.695330000000000e+02
+7.524050000000000e+01 +2.960700000000000e+02
+4.629170000000000e+02 +4.069750000000000e+02
+5.278969999999998e+02 +3.923260000000000e+02
+3.375260000000000e+00 +2.861230000000000e+02
+1.611810000000000e+02 +2.734820000000000e+02
+5.389059999999999e+02 +3.976110000000000e+02
+6.675600000000000e+01 +2.892600000000000e+02
+1.232590000000000e+03 +5.452660000000000e+02
+6.729789999999998e+02 +4.092050000000000e+02
+7.255329999999999e+01 +2.922870000000001e+02
+6.948560000000001e+01 +2.525220000000000e+02
+6.051530000000000e+02 +3.990940000000000e+02
+9.130050000000000e+02 +4.446200000000000e+02
+8.518320000000000e+02 +4.389590000000000e+02
+6.602139999999998e+02 +4.327710000000000e+02
+6.724450000000000e+01 +2.894230000000000e+02
+2.673230000000000e+02 +3.203610000000000e+02
+5.000650000000000e+02 +3.842080000000000e+02
+5.045800000000000e+02 +3.848750000000000e+02
+2.125030000000000e+02 +3.395200000000000e+02
+7.821930000000000e+02 +4.254480000000000e+02
+6.877460000000001e+01 +2.856050000000000e+02
+1.128690000000000e+03 +4.855140000000000e+02
+4.482290000000000e+02 +3.899730000000000e+02
+5.117870000000000e+02 +3.793240000000000e+02
+4.287780000000000e+02 +3.705260000000000e+02
+4.429820000000000e+02 +3.917750000000000e+02
+8.766170000000000e+02 +4.281170000000000e+02
+4.598470000000000e+02 +3.772690000000000e+02
+5.472930000000000e+01 +2.820570000000000e+02
+1.035620000000000e+03 +4.669760000000000e+02
+2.796840000000000e+01 +7.431080000000000e+01
+2.396510000000000e+02 +3.337010000000000e+02
+4.410980000000000e+02 +3.671100000000000e+02
+8.383560000000000e+01 +2.511430000000000e+02
+6.018120000000000e+01 +2.807230000000000e+02
+1.258190000000000e+01 +3.045540000000001e+02
+2.733700000000000e+02 +3.257640000000000e+02
+4.545200000000000e+02 +3.668710000000000e+02
+1.276910000000000e+03 +5.133750000000000e+02
+4.696420000000000e+02 +3.795690000000000e+02
+1.163870000000000e+02 +2.506530000000000e+02
+7.884540000000000e+02 +4.218050000000000e+02
+4.546450000000000e+02 +3.575710000000000e+02
+2.168330000000000e+02 +3.276840000000000e+02
+8.209180000000000e+02 +4.021280000000000e+02
+1.466410000000000e+03 +5.490640000000000e+02
+3.465820000000000e+02 +3.615930000000000e+02
+2.513680000000000e+02 +3.210370000000001e+02
+5.468090000000000e+02 +3.874640000000000e+02
+2.942960000000000e+02 +3.341930000000001e+02
+3.155970000000000e+02 +3.364040000000000e+02
+5.576490000000000e+01 +2.755380000000000e+02
+7.880060000000002e+02 +4.165260000000000e+02
+4.895070000000000e+02 +3.736120000000000e+02
+4.461190000000000e+01 +2.882700000000000e+02
+5.611830000000000e+02 +3.795560000000000e+02
+5.371880000000000e+01 +2.747370000000000e+02
+2.454670000000000e+02 +3.209770000000001e+02
+5.043630000000001e+02 +3.783780000000000e+02
+4.897560000000000e+02 +3.823120000000000e+02
+8.640300000000000e+02 +4.309760000000000e+02
+9.087750000000000e+02 +4.777310000000000e+02
+4.218160000000000e+02 +3.552860000000000e+02
+9.638410000000000e+02 +4.805630000000001e+02
+4.912970000000000e+01 +2.760040000000000e+02
+4.564580000000000e+02 +3.622080000000000e+02
+6.962810000000002e+02 +3.810420000000000e+02
+9.981140000000000e+02 +4.312120000000000e+02
+2.451530000000000e+02 +3.092780000000000e+02
+4.763080000000000e+02 +3.562380000000001e+02
+7.682719999999998e+02 +4.090040000000000e+02
+4.385240000000000e+02 +3.560880000000000e+02
+3.950660000000000e+02 +3.545590000000000e+02
+1.009270000000000e+03 +4.330420000000000e+02
+1.529340000000000e+02 +2.671900000000000e+02
+1.355050000000000e+02 +2.650080000000000e+02
+7.782510000000002e+02 +4.245870000000000e+02
+2.135040000000000e+02 +3.218870000000000e+02
+2.331730000000000e+01 +2.666120000000000e+02
+1.038300000000000e+03 +4.489540000000000e+02
+3.216110000000000e+02 +3.395590000000000e+02
+4.577730000000000e+02 +3.606370000000000e+02
+1.845050000000000e+01 +2.667180000000000e+02
+3.917380000000001e+02 +3.522720000000000e+02
+4.697700000000000e+02 +3.636160000000000e+02
+5.994560000000000e+02 +3.735380000000000e+02
+3.926770000000000e+02 +3.543100000000000e+02
+5.210720000000000e+02 +3.559880000000001e+02
+4.717590000000000e+01 +2.631350000000000e+02
+7.045280000000000e+02 +3.783820000000000e+02
+8.563180000000000e+02 +4.349030000000000e+02
+2.608480000000000e+02 +3.063310000000000e+02
+4.827570000000000e+01 +2.641680000000000e+02
+2.579320000000000e+02 +3.072950000000000e+02
+4.296170000000000e+01 +2.641600000000000e+02
+7.868140000000000e+02 +4.239570000000000e+02
+7.596880000000000e+02 +4.046830000000000e+02
+8.385530000000000e+02 +4.122320000000000e+02
+1.690440000000000e+02 +3.120880000000000e+02
+5.579660000000000e+02 +3.691050000000000e+02
+7.578070000000000e+02 +3.931170000000000e+02
+3.573200000000000e+01 +7.515760000000000e+01
+2.157550000000000e+02 +3.007940000000001e+02
+4.185630000000001e+02 +3.523530000000000e+02
+5.667809999999999e+02 +3.664850000000000e+02
+4.094030000000000e+00 +2.593400000000000e+02
+5.113080000000000e+02 +3.678360000000000e+02
+2.708910000000000e+02 +3.212620000000000e+02
+3.650540000000000e+01 +2.616690000000000e+02
+8.317669999999998e+02 +3.842620000000000e+02
+1.157330000000000e+02 +2.693270000000000e+02
+5.065670000000000e+01 +2.608860000000000e+02
+5.564450000000001e+02 +3.642580000000000e+02
+8.205430000000000e+02 +3.895920000000000e+02
+2.239790000000000e+01 +2.593920000000000e+02
+4.191700000000000e+01 +2.573060000000000e+02
+5.436910000000000e+02 +3.665200000000000e+02
+1.295800000000000e+01 +2.527720000000000e+02
+4.497930000000000e+01 +2.580370000000000e+02
+3.508350000000000e+02 +3.306070000000000e+02
+1.683690000000000e+03 +5.218750000000000e+02
+3.497800000000000e+02 +3.291660000000000e+02
+4.780890000000000e+02 +3.511360000000000e+02
+7.967750000000000e+00 +2.587150000000000e+02
+2.425410000000000e+02 +3.155640000000000e+02
+5.370950000000000e+02 +3.636460000000000e+02
+6.699950000000000e+01 +2.102330000000000e+02
+4.086200000000000e+02 +3.424469999999999e+02
+3.813720000000000e+01 +2.562410000000000e+02
+3.080720000000000e+02 +3.297340000000001e+02
+4.274890000000000e+02 +3.484250000000000e+02
+3.167650000000000e+02 +3.186030000000000e+02
+3.864240000000000e+01 +2.571390000000000e+02
+1.093010000000000e+03 +4.653620000000000e+02
+7.878610000000001e+02 +4.094860000000000e+02
+2.750230000000000e+02 +3.121430000000000e+02
+2.556500000000000e+01 +2.543140000000000e+02
+3.887400000000000e+01 +2.547450000000000e+02
+1.825100000000000e+02 +2.970550000000000e+02
+3.926270000000000e+02 +3.454020000000000e+02
+3.795200000000000e+02 +3.242870000000001e+02
+7.919670000000000e+02 +3.957840000000000e+02
+4.439010000000000e+02 +3.433970000000000e+02
+2.569690000000000e+02 +2.975510000000000e+02
+1.235690000000000e+02 +2.872750000000000e+02
+2.915940000000000e+02 +3.122540000000000e+02
+4.425990000000001e+00 +2.504820000000000e+02
+2.384540000000000e+02 +2.995550000000000e+02
+8.029620000000000e+02 +3.716350000000000e+02
+3.668200000000000e+02 +3.237360000000000e+02
+6.477809999999999e+02 +3.510780000000001e+02
+8.668140000000000e+01 +2.581880000000000e+02
+2.403600000000000e+01 +2.476220000000000e+02
+5.939680000000002e+02 +3.415980000000000e+02
+7.859780000000002e+02 +3.728710000000000e+02
+4.565540000000000e+00 +2.495870000000000e+02
+7.776270000000000e+02 +4.022160000000000e+02
+2.211860000000000e+02 +2.808910000000000e+02
+2.792160000000000e+01 +2.481060000000000e+02
+9.444760000000000e+02 +4.284900000000000e+02
+7.492060000000000e+02 +3.914080000000000e+02
+9.929200000000000e+00 +2.454140000000000e+02
+7.889500000000000e+01 +2.688990000000000e+02
+2.021900000000000e+00 +2.435250000000000e+02
+5.347290000000000e+02 +3.713920000000000e+02
+2.321930000000000e+02 +2.886790000000001e+02
+1.175130000000000e+02 +2.838390000000000e+02
+4.417880000000000e+02 +3.282180000000000e+02
+7.114570000000000e+02 +4.163410000000000e+02
+9.240670000000000e+02 +4.101970000000000e+02
+5.343580000000002e+02 +3.619980000000001e+02
+2.615660000000000e+02 +2.929810000000000e+02
+3.372050000000000e+02 +3.094830000000000e+02
+1.244660000000000e+03 +4.463250000000000e+02
+2.555290000000000e+02 +2.914740000000000e+02
+2.406180000000000e+02 +2.878420000000000e+02
+1.978170000000000e+01 +2.383550000000000e+02
+2.352910000000000e+02 +2.845000000000000e+02
+2.614870000000000e+01 +2.540360000000000e+02
+2.307510000000000e+02 +2.829800000000000e+02
+3.209900000000000e+02 +3.056280000000000e+02
+1.486080000000000e+01 +2.383920000000000e+02
+2.146990000000000e+02 +2.737150000000000e+02
+2.340420000000000e+02 +2.850250000000000e+02
+4.506640000000000e+02 +3.154830000000000e+02
+4.136320000000000e+02 +3.242010000000000e+02
+1.895930000000000e+01 +2.451110000000000e+02
+2.844300000000000e+01 +2.336390000000000e+02
+8.387130000000002e+02 +3.881530000000000e+02
+5.065430000000000e+01 +2.139460000000000e+02
+2.638670000000000e+01 +2.406650000000000e+02
+9.148510000000000e+01 +2.587430000000000e+02
+7.806730000000000e+02 +3.640570000000000e+02
+2.421750000000000e+01 +2.370790000000000e+02
+8.942439999999998e+02 +4.328300000000000e+02
+1.367510000000000e+03 +4.822200000000000e+02
+2.954400000000000e+02 +2.967760000000000e+02
+1.740760000000000e+02 +2.769900000000000e+02
+6.675450000000000e+02 +3.411120000000000e+02
+1.025440000000000e+03 +4.234890000000000e+02
+4.107920000000000e+02 +3.162380000000000e+02
+5.338030000000000e+01 +2.081360000000000e+02
+3.199400000000000e+00 +2.296440000000000e+02
+3.457560000000000e+02 +3.081960000000000e+02
+2.511920000000000e+02 +2.844660000000000e+02
+4.440820000000000e+02 +3.089180000000000e+02
+1.038170000000000e+03 +4.515970000000000e+02
+4.657770000000000e+02 +3.174770000000001e+02
+9.500610000000000e+01 +2.130830000000000e+02
+2.295120000000000e+02 +2.772720000000000e+02
+9.670280000000000e-01 +2.254170000000000e+02
+8.842910000000001e+02 +4.233100000000000e+02
+1.525730000000000e+01 +2.258970000000000e+02
+2.255050000000000e+02 +2.749140000000000e+02
+3.954590000000000e+02 +3.103730000000000e+02
+1.323940000000000e+02 +2.567700000000000e+02
+2.716720000000000e+02 +2.873180000000000e+02
+2.134110000000000e+02 +2.830090000000000e+02
+1.154070000000000e+01 +2.343190000000000e+02
+6.624589999999999e+02 +3.328030000000001e+02
+7.974020000000000e+01 +2.316200000000000e+02
+1.126380000000000e+02 +2.381950000000000e+02
+1.376640000000000e+01 +2.286880000000000e+02
+2.314830000000000e+02 +2.713820000000000e+02
+5.827640000000000e+02 +3.441290000000000e+02
+3.419560000000000e+02 +3.022860000000000e+02
+4.277630000000000e+02 +3.291100000000000e+02
+2.139950000000000e+02 +2.668150000000000e+02
+6.667280000000000e+00 +2.264030000000000e+02
+2.538410000000000e+02 +2.679250000000000e+02
+1.063060000000000e+02 +2.509280000000000e+02
+3.785750000000000e+02 +3.128320000000000e+02
+1.158220000000000e+01 +2.228500000000000e+02
+6.109620000000000e+02 +3.170190000000000e+02
+6.629160000000001e+00 +2.240520000000000e+02
+4.238300000000000e+02 +3.191070000000000e+02
+2.140130000000000e+02 +2.682610000000000e+02
+3.325470000000000e+02 +3.006440000000000e+02
+2.621260000000000e+02 +2.787290000000000e+02
+5.359349999999999e+02 +3.626240000000000e+02
+1.187550000000000e+01 +2.211980000000000e+02
+1.147320000000000e+01 +2.222790000000000e+02
+5.248650000000000e+02 +3.299950000000000e+02
+6.628880000000000e+02 +3.566330000000001e+02
+6.550580000000000e+02 +3.446700000000000e+02
+1.998510000000000e+01 +1.893440000000000e+02
+4.781020000000000e+02 +3.059530000000000e+02
+2.270760000000000e+02 +2.591820000000000e+02
+1.033040000000000e+02 +2.519240000000000e+02
+8.859889999999998e+02 +3.826350000000000e+02
+8.634620000000001e-01 +2.164930000000000e+02
+2.005710000000000e+02 +2.626830000000000e+02
+9.239530000000000e+00 +2.170560000000000e+02
+2.403050000000000e+01 +2.181490000000000e+02
+3.201690000000001e+02 +2.885500000000000e+02
+6.169000000000000e+02 +3.405800000000000e+02
+1.032930000000000e+02 +2.445370000000000e+02
+4.424840000000000e+02 +3.158160000000000e+02
+2.927480000000000e+02 +2.784750000000000e+02
+9.402780000000000e+02 +3.984620000000000e+02
+2.030770000000000e+02 +2.677550000000000e+02
+3.931260000000000e+02 +2.987380000000000e+02
+2.901210000000000e+02 +2.749480000000000e+02
+5.052350000000000e+00 +2.149040000000000e+02
+7.896510000000002e+02 +3.635980000000000e+02
+5.110720000000000e+02 +3.041320000000000e+02
+1.529940000000000e+02 +2.539790000000000e+02
+4.328200000000000e+02 +3.078850000000000e+02
+2.754400000000000e+02 +2.659620000000000e+02
+3.249340000000000e+00 +2.111410000000000e+02
+7.477510000000002e+02 +3.505550000000000e+02
+9.210950000000000e+02 +4.005940000000000e+02
+4.426970000000000e+02 +2.980160000000000e+02
+5.450630000000000e+01 +2.263970000000000e+02
+8.032150000000000e+00 +2.112200000000000e+02
+3.480220000000000e+02 +2.957250000000000e+02
+2.522900000000000e+02 +2.564340000000000e+02
+1.912530000000000e+02 +2.502230000000000e+02
+2.378880000000000e+01 +2.128070000000000e+02
+3.440340000000000e+02 +3.032280000000000e+02
+4.567950000000000e+02 +2.931890000000000e+02
+8.082830000000000e+02 +3.513710000000000e+02
+6.832370000000000e+02 +3.751230000000001e+02
+9.262820000000000e+02 +3.702550000000000e+02
+1.242500000000000e+02 +2.498200000000000e+02
+7.768980000000000e+02 +3.628530000000000e+02
+7.077930000000000e+02 +3.722070000000000e+02
+4.049480000000000e+02 +2.929410000000000e+02
+2.656550000000000e+02 +2.700560000000000e+02
+4.724570000000000e+02 +2.828240000000000e+02
+7.487510000000002e+02 +3.945220000000000e+02
+1.567140000000000e+02 +2.538760000000000e+02
+2.763200000000000e+02 +2.727880000000000e+02
+2.360500000000000e+00 +2.070450000000000e+02
+3.185760000000000e+02 +2.869400000000000e+02
+2.272640000000000e+02 +2.564570000000000e+02
+4.416730000000000e+02 +2.883310000000000e+02
+6.752910000000001e+02 +3.347820000000000e+02
+3.723540000000000e+02 +2.569960000000000e+02
+9.720880000000000e+02 +4.321300000000000e+02
+4.578430000000000e+02 +2.891960000000000e+02
+4.911420000000000e+02 +3.012040000000000e+02
+4.129640000000000e+02 +2.600260000000000e+02
+8.172180000000000e+01 +2.066310000000000e+02
+5.075470000000000e+02 +2.638040000000000e+02
+1.566840000000000e+01 +2.027610000000000e+02
+6.748260000000000e+02 +3.308060000000000e+02
+3.532710000000000e+02 +2.885810000000000e+02
+3.974510000000000e+02 +2.660490000000000e+02
+7.178639999999998e+02 +3.274950000000000e+02
+7.352240000000000e+01 +2.099240000000000e+02
+9.317569999999999e+02 +3.815710000000000e+02
+4.374830000000000e+02 +2.806740000000000e+02
+8.461559999999999e-01 +1.989780000000000e+02
+3.814150000000000e+02 +2.610990000000000e+02
+1.061090000000000e+03 +4.751640000000000e+02
+8.292500000000000e+02 +3.810990000000000e+02
+9.724840000000000e+02 +3.902100000000000e+02
+2.095560000000000e+02 +2.510260000000000e+02
+4.256290000000000e+02 +2.722620000000000e+02
+6.705269999999998e+02 +3.154830000000000e+02
+5.128099999999999e+02 +2.791780000000000e+02
+4.899730000000000e+02 +3.027690000000000e+02
+8.332350000000000e+02 +3.852820000000000e+02
+1.056200000000000e+01 +2.044450000000000e+02
+3.570810000000000e+02 +2.559190000000000e+02
+7.066240000000000e+00 +1.971600000000000e+02
+2.187930000000000e+02 +2.448270000000000e+02
+7.268320000000000e+02 +3.386340000000000e+02
+7.300080000000000e+02 +3.428710000000000e+02
+4.272130000000000e+02 +2.732770000000000e+02
+1.320490000000000e+03 +4.606850000000000e+02
+7.846000000000000e+02 +3.540210000000000e+02
+3.865430000000000e+00 +1.942990000000000e+02
+2.480550000000000e+02 +2.295950000000000e+02
+7.428240000000000e+02 +3.456269999999999e+02
+2.518240000000000e+02 +2.454090000000000e+02
+9.459510000000000e+02 +3.903890000000000e+02
+7.425780000000000e+02 +3.342330000000000e+02
+4.939650000000000e+02 +2.878820000000000e+02
+8.031790000000000e+02 +3.503310000000000e+02
+4.305760000000000e+02 +2.677030000000000e+02
+2.538680000000000e+02 +2.258380000000000e+02
+8.865730000000000e+02 +3.832970000000000e+02
+1.776559999999999e+02 +2.325690000000000e+02
+1.450720000000000e+03 +4.836250000000000e+02
+3.077580000000000e+01 +1.152990000000000e+02
+4.452270000000000e+02 +2.917780000000000e+02
+4.421140000000000e+02 +2.906820000000000e+02
+2.851740000000001e+02 +2.548800000000000e+02
+5.074850000000000e+00 +1.916260000000000e+02
+2.009170000000000e+02 +2.323320000000000e+02
+7.255140000000000e+01 +2.137910000000000e+02
+4.472200000000000e+02 +2.788810000000000e+02
+2.471790000000000e+02 +2.400510000000000e+02
+1.944580000000000e+00 +1.882110000000000e+02
+1.541750000000000e+02 +2.259340000000000e+02
+2.268620000000000e+02 +2.514060000000000e+02
+3.658960000000000e+02 +2.574530000000000e+02
+6.103640000000000e+02 +3.330260000000000e+02
+2.275130000000000e+02 +2.358030000000000e+02
+4.009380000000000e+00 +1.876810000000000e+02
+1.629370000000000e+03 +4.731140000000000e+02
+6.132400000000000e+00 +1.916420000000000e+02
+8.140400000000000e+02 +3.372100000000000e+02
+2.119600000000000e+02 +2.299700000000000e+02
+2.330070000000000e+02 +2.236590000000000e+02
+4.244630000000000e+02 +2.823990000000000e+02
+2.289420000000000e+02 +2.329690000000000e+02
+7.934580000000000e-01 +1.845660000000000e+02
+7.882040000000000e+02 +3.249950000000000e+02
+7.668770000000000e+02 +3.510980000000000e+02
+8.727780000000001e+00 +1.922240000000000e+02
+1.114970000000000e+00 +1.853720000000000e+02
+4.316070000000000e+02 +2.756700000000000e+02
+9.461540000000000e+02 +3.692190000000000e+02
+4.060190000000000e+02 +2.779750000000000e+02
+3.932010000000000e+02 +2.707660000000000e+02
+6.409940000000000e+02 +3.154700000000000e+02
+7.095450000000000e+02 +3.057120000000000e+02
+8.612680000000000e+02 +3.063750000000000e+02
+1.619860000000000e+02 +2.203750000000000e+02
+7.155100000000000e+02 +3.292440000000000e+02
+7.534349999999999e+02 +3.074860000000000e+02
+8.213420000000000e+02 +3.151400000000000e+02
+2.198490000000000e+02 +2.236080000000000e+02
+1.491820000000000e-01 +1.804920000000000e+02
+6.075740000000001e+00 +1.307770000000000e+02
+3.652880000000000e+02 +2.604210000000000e+02
+1.604000000000000e+02 +2.182880000000000e+02
+7.901020000000000e+02 +3.378690000000000e+02
+1.250370000000000e+03 +4.304610000000000e+02
+3.254940000000000e+01 +2.016520000000000e+02
+2.394200000000000e+02 +2.221260000000000e+02
+4.575920000000000e+02 +2.714490000000000e+02
+2.240350000000000e+02 +2.252660000000000e+02
+2.294780000000000e+02 +2.412320000000000e+02
+9.685790000000000e+02 +3.609460000000000e+02
+1.082980000000000e+03 +3.895410000000000e+02
+6.398210000000000e+02 +2.988560000000000e+02
+1.056390000000000e+00 +1.761130000000000e+02
+4.030600000000000e+02 +2.619290000000000e+02
+1.589540000000000e+02 +2.188520000000000e+02
+9.142370000000000e-01 +1.761130000000000e+02
+2.417470000000000e+02 +2.440010000000000e+02
+3.637690000000000e+02 +2.476100000000000e+02
+5.015130000000000e+02 +2.816530000000000e+02
+8.638190000000000e+02 +3.862590000000000e+02
+1.263010000000000e+03 +4.098760000000000e+02
+7.582200000000000e+02 +3.289170000000000e+02
+9.696020000000000e+02 +3.511310000000000e+02
+7.459900000000000e+02 +3.025740000000000e+02
+6.568880000000000e-01 +1.746540000000000e+02
+1.154980000000000e+03 +3.663310000000000e+02
+9.486050000000000e+02 +3.297480000000001e+02
+1.028400000000000e+01 +1.778780000000000e+02
+5.355910000000000e+02 +2.762030000000000e+02
+7.391980000000000e+02 +3.038960000000000e+02
+1.161480000000000e+03 +3.539070000000000e+02
+9.248860000000000e+02 +3.290530000000001e+02
+1.262830000000000e+02 +2.063660000000000e+02
+1.245510000000000e+03 +3.992530000000000e+02
+4.080070000000000e+02 +2.598800000000000e+02
+1.200890000000000e+03 +4.403000000000000e+02
+1.921390000000000e+02 +2.204070000000000e+02
+1.935260000000000e+02 +2.178220000000000e+02
+1.296040000000000e+02 +2.055370000000000e+02
+5.801460000000000e+01 +1.249950000000000e+02
+3.513860000000000e+02 +2.482930000000000e+02
+6.684230000000000e+02 +3.084470000000000e+02
+2.338960000000000e+02 +2.182230000000000e+02
+1.158710000000000e+03 +3.915070000000000e+02
+2.102990000000000e+02 +2.094320000000000e+02
+5.125630000000000e+02 +2.832440000000000e+02
+9.274540000000000e+02 +3.903980000000000e+02
+2.596470000000000e+01 +1.819180000000000e+02
+1.292380000000000e+02 +2.005620000000000e+02
+3.286780000000001e+02 +2.672690000000000e+02
+4.593460000000000e+02 +2.486290000000000e+02
+2.049300000000000e+02 +2.029600000000000e+02
+2.418600000000000e+01 +1.778490000000000e+02
+1.701150000000000e+02 +2.021150000000000e+02
+3.471319999999999e+02 +2.505460000000000e+02
+4.692790000000000e+02 +2.389090000000000e+02
+3.258540000000001e+02 +2.311330000000000e+02
+3.324940000000000e+02 +2.281000000000000e+02
+7.413960000000002e+02 +2.974570000000000e+02
+4.402780000000000e+02 +2.646230000000000e+02
+5.640040000000000e+00 +7.241060000000001e+00
+3.671520000000000e+02 +2.430660000000000e+02
+1.868360000000000e+02 +2.005210000000000e+02
+2.603090000000000e+00 +1.659650000000000e+02
+6.372450000000000e+02 +2.646740000000000e+02
+5.603350000000000e+02 +2.731230000000000e+02
+1.254470000000000e+03 +3.927600000000000e+02
+7.903300000000000e+02 +3.427140000000000e+02
+1.639500000000000e+02 +2.035810000000000e+02
+1.652490000000000e+01 +1.575470000000000e+02
+1.870300000000000e+02 +1.978020000000000e+02
+1.163660000000000e+02 +1.469080000000000e+02
+3.052050000000000e+02 +2.269390000000000e+02
+1.612300000000000e+00 +1.599610000000000e+02
+5.431580000000000e+02 +2.649400000000000e+02
+8.337040000000000e+02 +3.146860000000000e+02
+9.605440000000000e+01 +1.164650000000000e+02
+1.179170000000000e+03 +3.635970000000000e+02
+6.982619999999999e+02 +2.970820000000000e+02
+9.409080000000000e+01 +1.473530000000000e+02
+2.033710000000000e+02 +1.954950000000000e+02
+1.450900000000000e+02 +1.625460000000000e+02
+8.898860000000002e+02 +2.904660000000000e+02
+6.051830000000000e+02 +2.610440000000000e+02
+1.133390000000000e+02 +1.840070000000000e+02
+1.833420000000000e+01 +1.599900000000000e+02
+2.102660000000000e+02 +2.173100000000000e+02
+1.787310000000000e+02 +1.902170000000000e+02
+3.678500000000000e+02 +2.274740000000000e+02
+8.756480000000000e+02 +3.519790000000001e+02
+1.862290000000000e+02 +1.965270000000000e+02
+3.330340000000000e+02 +2.261620000000000e+02
+3.861120000000000e+01 +9.529100000000000e+01
+6.087840000000000e+02 +2.622930000000000e+02
+1.712550000000000e+02 +1.900140000000000e+02
+1.579120000000000e+02 +1.677450000000000e+02
+1.685050000000000e+02 +1.366090000000000e+02
+6.811260000000000e+01 +1.605200000000000e+02
+4.617120000000000e+02 +2.380210000000000e+02
+8.003960000000002e+02 +2.892920000000001e+02
+1.671640000000000e+02 +1.893210000000000e+02
+1.500950000000000e+00 +1.485770000000000e+02
+1.140470000000000e+02 +1.469630000000000e+02
+1.119150000000000e+03 +3.591980000000001e+02
+1.189660000000000e+02 +1.793530000000000e+02
+2.408570000000000e+02 +2.003740000000000e+02
+9.306900000000000e+01 +1.496700000000000e+02
+1.972720000000000e+02 +1.854870000000000e+02
+1.082300000000000e+02 +1.765840000000000e+02
+7.647950000000000e+02 +2.883620000000000e+02
+8.066070000000000e+02 +2.767410000000000e+02
+1.631330000000000e+02 +1.837710000000000e+02
+1.629340000000000e+01 +8.762500000000000e+01
+3.195250000000000e+02 +2.049960000000000e+02
+6.942410000000000e+01 +1.313230000000000e+02
+1.793720000000000e+02 +1.806950000000000e+02
+2.390460000000000e+02 +1.487340000000000e+02
+2.618260000000000e+02 +2.013100000000000e+02
+2.801250000000000e+02 +2.097050000000000e+02
+6.523450000000000e+01 +1.537710000000000e+02
+1.468790000000000e+02 +1.785250000000000e+02
+4.007440000000000e+02 +2.065240000000000e+02
+1.112740000000000e+02 +1.748280000000000e+02
+2.130900000000000e+02 +1.916100000000000e+02
+2.944330000000000e+02 +1.973100000000000e+02
+5.646580000000000e+02 +2.337250000000000e+02
+6.809390000000000e+02 +2.779440000000000e+02
+1.729550000000000e+02 +1.756040000000000e+02
+1.672350000000000e+02 +1.730810000000000e+02
+1.371270000000000e+02 +1.697450000000000e+02
+1.961620000000000e+02 +1.843690000000000e+02
+5.585160000000000e+02 +2.256080000000000e+02
+9.923050000000001e+01 +1.690130000000000e+02
+3.258180000000000e+02 +1.914590000000000e+02
+5.085400000000000e+01 +8.868310000000000e+01
+8.110730000000000e+02 +2.986680000000000e+02
+5.887200000000000e+02 +2.390500000000000e+02
+8.245139999999999e+02 +3.264250000000000e+02
+2.368580000000000e+02 +1.824710000000000e+02
+2.215060000000000e+00 +1.853140000000000e+01
+7.120360000000002e+02 +2.034290000000000e+02
+8.012869999999998e+02 +2.669710000000000e+02
+9.762500000000000e+01 +1.638640000000000e+02
+1.148610000000000e+02 +1.277250000000000e+02
+1.876710000000000e+02 +1.694240000000000e+02
+8.263920000000000e+01 +1.544990000000000e+02
+1.792340000000000e+02 +1.686330000000000e+02
+6.371290000000000e+02 +2.535390000000000e+02
+8.957769999999998e+02 +3.608280000000000e+02
+2.031250000000000e+02 +1.487950000000000e+02
+5.412280000000002e+02 +2.541750000000000e+02
+3.472510000000000e+02 +2.190220000000000e+02
+1.668410000000000e+02 +1.714410000000000e+02
+8.213470000000000e+02 +3.148190000000000e+02
+1.609550000000000e+02 +1.657340000000000e+02
+1.090690000000000e+02 +1.558390000000000e+02
+7.021289999999998e+02 +2.552340000000000e+02
+1.149030000000000e+03 +3.413990000000000e+02
+8.863860000000002e+02 +3.097650000000000e+02
+8.946650000000000e+01 +1.597980000000000e+02
+7.957230000000002e+02 +2.670140000000000e+02
+6.406800000000000e+00 +1.341470000000000e+02
+6.000910000000000e+02 +2.715450000000000e+02
+8.256880000000000e+02 +2.673370000000000e+02
+1.715180000000000e+02 +1.706710000000000e+02
+5.673870000000000e+00 +7.193490000000000e+00
+6.800960000000000e+02 +2.593270000000000e+02
+1.099520000000000e+01 +7.113220000000000e+00
+1.301630000000000e+02 +1.724080000000000e+02
+1.973540000000000e+02 +1.724290000000000e+02
+1.542180000000000e+02 +1.625720000000000e+02
+9.852910000000001e+02 +3.703290000000000e+02
+4.411010000000000e+01 +1.015670000000000e+02
+6.304400000000001e+02 +2.430850000000000e+02
+1.910340000000000e+02 +1.600720000000000e+02
+6.748980000000000e+01 +1.377350000000000e+02
+3.940520000000000e+02 +1.867150000000000e+02
+1.501160000000000e+02 +1.595940000000000e+02
+4.338470000000000e+01 +1.170740000000000e+02
+1.474470000000000e+02 +1.687230000000000e+02
+9.025920000000001e+01 +1.554470000000000e+02
+8.403100000000001e+01 +1.511130000000000e+02
+3.492300000000000e+00 +1.279790000000000e+02
+1.193920000000000e+01 +1.306470000000000e+02
+4.617080000000000e+02 +2.249530000000000e+02
+8.094789999999998e+02 +2.669780000000000e+02
+4.609610000000000e+02 +2.092550000000000e+02
+1.441070000000000e+02 +1.578080000000000e+02
+7.716450000000000e+01 +1.394760000000000e+02
+5.092550000000000e+02 +2.278180000000000e+02
+2.190580000000000e+02 +1.813820000000000e+02
+5.670000000000000e+01 +1.055460000000000e+02
+2.906680000000000e+02 +1.866950000000000e+02
+4.621150000000000e+00 +6.996850000000001e+01
+3.213710000000000e+01 +1.322350000000000e+02
+8.127400000000000e+01 +1.472210000000000e+02
+4.915660000000000e+02 +2.092140000000000e+02
+3.440340000000000e-01 +1.237130000000000e+02
+4.400990000000000e+02 +2.084390000000000e+02
+5.694040000000000e+01 +1.141250000000000e+02
+4.917340000000000e+02 +2.054560000000000e+02
+7.714700000000000e+02 +3.003870000000000e+02
+2.361090000000000e+00 +5.562810000000000e+01
+7.632210000000000e+02 +2.415090000000000e+02
+4.984370000000000e+02 +2.091300000000000e+02
+4.362400000000000e+02 +2.091340000000000e+02
+1.279660000000000e+02 +1.244230000000000e+02
+4.896850000000000e+02 +1.988310000000000e+02
+6.331850000000000e+01 +9.927970000000001e+01
+2.466880000000000e+01 +1.284340000000000e+02
+1.878290000000000e+02 +1.581280000000000e+02
+2.174600000000000e+02 +1.728200000000000e+02
+1.270430000000000e+02 +1.495780000000000e+02
+3.744950000000000e+02 +2.078290000000000e+02
+7.589560000000000e+02 +2.213070000000000e+02
+8.517340000000000e+01 +1.445630000000000e+02
+1.206350000000000e+02 +1.436160000000000e+02
+4.847480000000001e+02 +2.167660000000000e+02
+2.245070000000000e+01 +1.286060000000000e+02
+1.171640000000000e+03 +3.552960000000000e+02
+1.120940000000000e+03 +3.134370000000000e+02
+1.054190000000000e+02 +1.477170000000000e+02
+1.142690000000000e+03 +3.483890000000000e+02
+1.827120000000000e+01 +1.250210000000000e+02
+3.381869999999999e+02 +1.930940000000000e+02
+7.182140000000000e+01 +1.374340000000000e+02
+4.346340000000000e+02 +2.140220000000000e+02
+6.403810000000000e+01 +8.707980000000001e+01
+6.885260000000002e+02 +2.513620000000000e+02
+6.336559999999999e+02 +2.625750000000000e+02
+3.848240000000000e+01 +8.250250000000000e+01
+1.312620000000000e+01 +1.211000000000000e+02
+6.920340000000000e+01 +1.351020000000000e+02
+4.656430000000000e+02 +1.933450000000000e+02
+2.172070000000000e+02 +1.651270000000000e+02
+6.621530000000000e+02 +2.445930000000000e+02
+4.664290000000000e+02 +1.915920000000000e+02
+9.560010000000000e+01 +1.400000000000000e+02
+4.265850000000000e+01 +1.190590000000000e+02
+5.095990000000000e+02 +2.351590000000000e+02
+6.143250000000000e+02 +2.214230000000000e+02
+5.289320000000000e+02 +2.209890000000000e+02
+6.073240000000002e+02 +2.623830000000000e+02
+7.250590000000000e+01 +1.017720000000000e+02
+4.558580000000000e+02 +1.962440000000000e+02
+1.102120000000000e+03 +3.349500000000000e+02
+8.878810000000002e+02 +2.640190000000000e+02
+8.281480000000001e+01 +1.085710000000000e+02
+5.357850000000000e+02 +2.191490000000000e+02
+4.718820000000000e+01 +1.075390000000000e+02
+8.347380000000000e+01 +1.333030000000000e+02
+7.737030000000000e+02 +2.439870000000000e+02
+5.978190000000000e+01 +1.308410000000000e+02
+4.631470000000000e+02 +1.834250000000000e+02
+7.883220000000000e+01 +1.308200000000000e+02
+6.413470000000000e+02 +2.598020000000000e+02
+2.664310000000000e+01 +7.323500000000000e+01
+4.595050000000000e+02 +2.116960000000000e+02
+1.992530000000000e+01 +1.179700000000000e+02
+1.151150000000000e+02 +1.261810000000000e+02
+4.399230000000000e+02 +2.103940000000000e+02
+3.789140000000000e+02 +2.038340000000000e+02
+3.999920000000000e+01 +1.264650000000000e+02
+4.393540000000000e+02 +2.128850000000000e+02
+7.207910000000001e+02 +2.861020000000001e+02
+2.321870000000000e+02 +1.540720000000000e+02
+9.635240000000000e+01 +1.171340000000000e+02
+7.311870000000000e+01 +1.303900000000000e+02
+1.826420000000000e+02 +1.435450000000000e+02
+4.068300000000000e+02 +2.091040000000000e+02
+1.627580000000000e+02 +1.503110000000000e+02
+7.596120000000001e+01 +1.199130000000000e+02
+4.779340000000000e+02 +1.825190000000000e+02
+4.422850000000000e+02 +2.090200000000000e+02
+6.254270000000000e+02 +2.391530000000000e+02
+4.205700000000000e+02 +2.009720000000000e+02
+1.498230000000000e+02 +1.321040000000000e+02
+4.285520000000000e+02 +1.972650000000000e+02
+5.426369999999999e+02 +2.091230000000000e+02
+1.155380000000000e+00 +1.050840000000000e+02
+5.680330000000000e+01 +1.113020000000000e+02
+4.328290000000000e+02 +1.930500000000000e+02
+3.031070000000000e+02 +1.769420000000000e+02
+1.326520000000000e+02 +1.207010000000000e+02
+6.366310000000000e+02 +2.116290000000000e+02
+1.801170000000000e+02 +1.315450000000000e+02
+4.577820000000000e+02 +1.747740000000000e+02
+1.120080000000000e+01 +9.011630000000000e+01
+4.135260000000000e+01 +1.272910000000000e+02
+3.174890000000000e+01 +1.184400000000000e+02
+2.568060000000000e+02 +1.725800000000000e+02
+3.702010000000000e+01 +1.069280000000000e+02
+3.293530000000000e+02 +1.553400000000000e+02
+4.338340000000000e+02 +2.074140000000000e+02
+6.696650000000000e+02 +2.289560000000000e+02
+1.795650000000000e+01 +6.588260000000000e+01
+4.627280000000000e+02 +1.819280000000000e+02
+3.607190000000000e+02 +1.698210000000000e+02
+1.882210000000000e+02 +9.852209999999999e+01
+6.226600000000000e+02 +2.352160000000000e+02
+4.349070000000000e+02 +1.872140000000000e+02
+4.066470000000000e+00 +1.073560000000000e+02
+4.121090000000000e+01 +6.915430000000001e+01
+4.328390000000000e+02 +1.832220000000000e+02
+1.939510000000000e+01 +1.083790000000000e+02
+4.523950000000000e+02 +1.901660000000000e+02
+1.952110000000000e+02 +1.056960000000000e+02
+3.502870000000000e+02 +1.572000000000000e+02
+4.110740000000000e+02 +1.926990000000000e+02
+7.786810000000000e+02 +2.226960000000000e+02
+7.723430000000000e+01 +1.092510000000000e+02
+4.356040000000000e+02 +1.783759999999999e+02
+6.253960000000000e+02 +2.256260000000000e+02
+3.137420000000000e+02 +1.541680000000000e+02
+4.427580000000000e+02 +1.786220000000000e+02
+4.137220000000000e+02 +1.706270000000000e+02
+1.229970000000000e+02 +8.303540000000000e+01
+6.204720000000000e+02 +2.397100000000000e+02
+3.569250000000000e+01 +9.247570000000000e+01
+5.663650000000000e+02 +2.184930000000000e+02
+5.884950000000000e+02 +1.894890000000000e+02
+4.013890000000000e+02 +1.849350000000000e+02
+5.408270000000000e+02 +1.961840000000000e+02
+1.391680000000000e+02 +1.338790000000000e+02
+7.398730000000000e+02 +2.041620000000000e+02
+6.231220000000000e+02 +2.187830000000000e+02
+5.264780000000000e+00 +9.802360000000000e+01
+3.153850000000000e+02 +1.560460000000000e+02
+7.483989999999999e+02 +2.174010000000000e+02
+4.357240000000000e+02 +1.662920000000000e+02
+7.410069999999999e+02 +2.096140000000000e+02
+4.601070000000000e+02 +1.739330000000000e+02
+8.432180000000000e+01 +1.176360000000000e+02
+7.237669999999998e+02 +2.078440000000000e+02
+1.831840000000000e+02 +1.414600000000000e+02
+3.592120000000000e+02 +1.600380000000000e+02
+1.523400000000000e+02 +8.539260000000000e+01
+7.493360000000000e+01 +1.240610000000000e+02
+4.738200000000000e+02 +1.956790000000000e+02
+4.394450000000000e+02 +1.590050000000000e+02
+5.872430000000001e+02 +1.990590000000000e+02
+5.580080000000000e+02 +2.037160000000000e+02
+1.057510000000000e+02 +9.978860000000000e+01
+3.838490000000000e+01 +9.650620000000001e+01
+4.217870000000000e+02 +1.938730000000000e+02
+5.296830000000000e+02 +1.667000000000000e+02
+9.197190000000001e+02 +3.008420000000000e+02
+2.849620000000000e+02 +1.518950000000000e+02
+9.083910000000000e+01 +1.142210000000000e+02
+3.509700000000000e+01 +7.682160000000000e+01
+2.125900000000000e+02 +1.343110000000000e+02
+4.202920000000000e+02 +1.724490000000000e+02
+5.283030000000000e+02 +2.187060000000000e+02
+6.513890000000000e+00 +9.325060000000001e+01
+5.394240000000000e+00 +5.332320000000000e+01
+5.619460000000000e+02 +1.913810000000000e+02
+6.928260000000000e+01 +1.128870000000000e+02
+5.944770000000000e+02 +1.901010000000000e+02
+1.156160000000000e+02 +1.119310000000000e+02
+5.898180000000000e+01 +8.117000000000000e+01
+8.790219999999998e+02 +2.514550000000000e+02
+6.386600000000001e+01 +1.092370000000000e+02
+2.432660000000000e+02 +1.335110000000000e+02
+2.829140000000000e+02 +1.481980000000000e+02
+2.010460000000000e+02 +1.095880000000000e+02
+9.856479999999999e+00 +9.419770000000000e+01
+1.706920000000000e+02 +1.293220000000000e+02
+5.240900000000000e+02 +1.903600000000000e+02
+2.373120000000000e+01 +1.024620000000000e+02
+3.415719999999999e+02 +9.317020000000000e+02
+7.039140000000000e+02 +2.061640000000000e+02
+2.751380000000000e+02 +1.507940000000000e+02
+3.624250000000000e+01 +1.021810000000000e+02
+6.281700000000000e+02 +1.833150000000000e+02
+1.138990000000000e+02 +1.099800000000000e+02
+4.196700000000000e+02 +1.557070000000000e+02
+1.630350000000000e+02 +8.275470000000000e+01
+4.776970000000000e+02 +2.073610000000000e+02
+5.467490000000000e+02 +1.708170000000000e+02
+8.575150000000000e+02 +2.433080000000000e+02
+4.137080000000000e+02 +1.648880000000000e+02
+2.052510000000000e+01 +9.509470000000000e+01
+1.010030000000000e+02 +1.044220000000000e+02
+4.018310000000000e+02 +1.691490000000000e+02
+4.622510000000000e+02 +1.783930000000000e+02
+5.752160000000000e+02 +1.859900000000000e+02
+1.165240000000000e+02 +1.056980000000000e+02
+1.002400000000000e+02 +7.642160000000000e+01
+2.685070000000000e+01 +5.345370000000000e+01
+1.364490000000000e+02 +8.285830000000000e+01
+4.273450000000000e+02 +1.629470000000000e+02
+1.904080000000000e+01 +5.535920000000000e+01
+7.244290000000000e+02 +2.324560000000000e+02
+4.435900000000000e+02 +1.663940000000000e+02
+9.674450000000000e+01 +6.940810000000000e+01
+3.739770000000000e+02 +1.667270000000000e+02
+8.969360000000000e+01 +1.065070000000000e+02
+5.623440000000001e+02 +1.899460000000000e+02
+3.968960000000000e+02 +1.617170000000000e+02
+6.522810000000000e+00 +8.744320000000000e+01
+4.261230000000001e+02 +1.917890000000000e+02
+3.940350000000000e+02 +1.583120000000000e+02
+4.761710000000000e-01 +8.319150000000000e+01
+1.257160000000000e+02 +1.079880000000000e+02
+1.413820000000000e+02 +8.009260000000000e+01
+1.064520000000000e+02 +9.684150000000000e+01
+2.872290000000001e+02 +1.310700000000000e+02
+2.181800000000000e+01 +5.092090000000000e+01
+6.651000000000000e+02 +1.986330000000000e+02
+1.111390000000000e+02 +1.038190000000000e+02
+7.693300000000001e+01 +6.771510000000001e+01
+3.049470000000000e+02 +1.386000000000000e+02
+2.747060000000000e+02 +1.540250000000000e+02
+2.595270000000000e+01 +6.379940000000000e+01
+4.247720000000000e+02 +1.825540000000000e+02
+5.392110000000000e+02 +1.739430000000000e+02
+6.428770000000000e+02 +1.972890000000000e+02
+1.112600000000000e+03 +2.882190000000000e+02
+2.714890000000000e+01 +9.139200000000000e+01
+1.941680000000000e-02 +7.929949999999999e+01
+2.248490000000000e+02 +1.306620000000000e+02
+2.062050000000000e+01 +7.700350000000000e+01
+3.971940000000000e+02 +1.531540000000000e+02
+1.213240000000000e+02 +8.792290000000000e+01
+3.775990000000000e+02 +1.852210000000000e+02
+7.213970000000000e+01 +9.353579999999999e+01
+6.146770000000000e+02 +1.787390000000000e+02
+4.142220000000000e+02 +1.598390000000000e+02
+4.203710000000000e+02 +1.671450000000000e+02
+4.946020000000000e+01 +7.382830000000000e+01
+1.125310000000000e+02 +9.074850000000001e+01
+3.290380000000000e+02 +1.730090000000000e+02
+7.223530000000002e+02 +1.826270000000000e+02
+5.943800000000000e+02 +1.764480000000000e+02
+3.524469999999999e+02 +1.579140000000000e+02
+2.596380000000000e+01 +5.393040000000000e+01
+5.954730000000002e+02 +1.622230000000000e+02
+3.779660000000000e+00 +5.415730000000000e+01
+4.120160000000000e+02 +1.486230000000000e+02
+1.414940000000000e+02 +1.000680000000000e+02
+7.724600000000000e+02 +2.124930000000000e+02
+1.148350000000000e+02 +7.025920000000001e+01
+2.766880000000000e+01 +8.456690000000000e+01
+7.157520000000000e+02 +2.008010000000000e+02
+5.661659999999998e+02 +1.752330000000000e+02
+1.130270000000000e+00 +7.425360000000001e+01
+1.180560000000000e+02 +7.293899999999999e+01
+1.225830000000000e+01 +8.299180000000000e+01
+4.357800000000000e+02 +1.583640000000000e+02
+2.518330000000000e+02 +1.623520000000000e+02
+3.245500000000000e+02 +1.623750000000000e+02
+5.848840000000000e+02 +1.620520000000000e+02
+2.765920000000000e+02 +1.230220000000000e+02
+1.549680000000000e+02 +9.474270000000000e+01
+8.730490000000000e+02 +2.097840000000000e+02
+4.188200000000000e+02 +1.482920000000000e+02
+4.922050000000000e+01 +7.694060000000000e+01
+4.197470000000000e+02 +1.499200000000000e+02
+9.499290000000000e+00 +8.019589999999999e+01
+6.442480000000000e+02 +1.902430000000000e+02
+3.703000000000000e+02 +1.421530000000000e+02
+3.993480000000000e+02 +1.519670000000000e+02
+1.808930000000000e+02 +9.558260000000000e+01
+8.981560000000002e+02 +2.299980000000000e+02
+3.208030000000000e+02 +1.415660000000000e+02
+1.405230000000000e+00 +5.501890000000000e+01
+3.659610000000000e+02 +1.381780000000000e+02
+4.253110000000000e+02 +1.457950000000000e+02
+3.299110000000000e+02 +1.312780000000000e+02
+1.740040000000000e+02 +9.225550000000000e+01
+5.423680000000001e+02 +1.711120000000000e+02
+1.885270000000000e+01 +9.840870000000001e+00
+4.003800000000000e+02 +1.360250000000000e+02
+6.871310000000002e+02 +2.028280000000000e+02
+8.269570000000000e+02 +2.215550000000000e+02
+2.981560000000000e+02 +1.268130000000000e+02
+2.986460000000000e+02 +1.095540000000000e+02
+4.144340000000000e+02 +1.565530000000000e+02
+3.604090000000000e+02 +1.618660000000000e+02
+8.351500000000000e+02 +2.167500000000000e+02
+1.077170000000000e+02 +8.047990000000000e+01
+1.266580000000000e+02 +8.848770000000000e+01
+6.663589999999998e+02 +1.959370000000000e+02
+5.774650000000000e+01 +7.210710000000000e+01
+1.059450000000000e+02 +8.628160000000000e+01
+4.761190000000000e+02 +1.820770000000000e+02
+3.669970000000000e+01 +2.672930000000000e+01
+1.798740000000001e+02 +1.215940000000000e+02
+3.886120000000000e+02 +1.245930000000000e+02
+1.587510000000000e+02 +9.105610000000000e+01
+5.603800000000000e-01 +6.551730000000001e+01
+5.299300000000000e+00 +6.672069999999999e+01
+8.223830000000000e+02 +2.237850000000000e+02
+2.909480000000000e+02 +1.227450000000000e+02
+1.024840000000000e+00 +5.838000000000000e+01
+8.016870000000000e+02 +1.961010000000000e+02
+7.016580000000000e+02 +1.859300000000000e+02
+4.353190000000000e+02 +1.572140000000000e+02
+1.529620000000000e+02 +8.752040000000000e+01
+4.821530000000000e+02 +1.655510000000000e+02
+4.279060000000000e+02 +1.485410000000000e+02
+1.467300000000000e+01 +5.998190000000000e+01
+3.732300000000000e+02 +1.231030000000000e+02
+1.264990000000000e+02 +7.696960000000000e+01
+3.977930000000000e+02 +1.324290000000000e+02
+7.442850000000000e+02 +1.832380000000000e+02
+4.462720000000000e-01 +6.275850000000000e+01
+3.369000000000000e+01 +6.594060000000000e+01
+6.076220000000000e+02 +1.584320000000000e+02
+3.854590000000000e+02 +1.407170000000000e+02
+5.225570000000000e+02 +1.489270000000000e+02
+3.447360000000000e+02 +1.488810000000000e+02
+2.844020000000000e+02 +1.229400000000000e+02
+1.342240000000000e+01 +6.800560000000000e+01
+7.115369999999998e+02 +1.678930000000000e+02
+3.136690000000001e+02 +1.224790000000000e+02
+6.275200000000000e+02 +1.623890000000000e+02
+4.094200000000000e+00 +7.598700000000000e+00
+2.703030000000000e+02 +1.151480000000000e+02
+1.201570000000000e+02 +7.735660000000000e+01
+9.948699999999999e+01 +7.487770000000000e+01
+5.370010000000000e+02 +1.335580000000000e+02
+3.787280000000000e+02 +1.342080000000000e+02
+9.463190000000000e+01 +7.622980000000000e+01
+8.031680000000000e-01 +5.964150000000000e+01
+1.266420000000000e+02 +8.790670000000000e+01
+5.909600000000000e+02 +1.557130000000000e+02
+7.224140000000000e+02 +2.087180000000000e+02
+2.381030000000000e+02 +1.082940000000000e+02
+7.556180000000000e+00 +6.344810000000000e+01
+1.179010000000000e+00 +4.329850000000000e+01
+1.258710000000000e+02 +8.650100000000000e+01
+3.868640000000000e+02 +1.250180000000000e+02
+6.989440000000000e+02 +1.923970000000000e+02
+2.578260000000000e+02 +9.874650000000000e+01
+1.006420000000000e+02 +7.473240000000000e+01
+7.051260000000002e+02 +1.590080000000000e+02
+1.536300000000000e+02 +7.783440000000000e+01
+5.011560000000000e+02 +1.455250000000000e+02
+4.763490000000000e+02 +1.379670000000000e+02
+4.040420000000000e+02 +1.471380000000000e+02
+3.593980000000000e+02 +1.273850000000000e+02
+2.081790000000000e+02 +9.605159999999999e+01
+3.990120000000000e+00 +1.228190000000000e+01
+6.911980000000000e+02 +1.555910000000000e+02
+2.613880000000000e+02 +1.047360000000000e+02
+5.743480000000002e+02 +1.626860000000000e+02
+4.073880000000000e+02 +1.283420000000000e+02
+2.367210000000000e+02 +9.095480000000001e+01
+7.231950000000001e+02 +1.699000000000000e+02
+7.062869999999999e+01 +3.502740000000000e+01
+5.875040000000000e+01 +6.310190000000000e+01
+2.264630000000000e+02 +1.070490000000000e+02
+1.297700000000000e+02 +7.877640000000000e+01
+2.425720000000000e+02 +1.147600000000000e+02
+6.956750000000000e+02 +1.498900000000000e+02
+2.424870000000000e+02 +1.011780000000000e+02
+4.056000000000000e+02 +1.289490000000000e+02
+2.120400000000000e+01 +6.195390000000000e+01
+3.600780000000000e+00 +4.778820000000000e+01
+4.014510000000000e+02 +1.351910000000000e+02
+3.690230000000000e+02 +1.204890000000000e+02
+1.790280000000000e+02 +8.687640000000000e+01
+1.612500000000000e+02 +1.054830000000000e+02
+5.995470000000000e+02 +1.752910000000000e+02
+1.486710000000000e+02 +7.546610000000000e+01
+3.684590000000000e+02 +1.253270000000000e+02
+5.437370000000000e+01 +1.988670000000000e+01
+8.688799999999998e+01 +6.759120000000000e+01
+2.044490000000000e+01 +5.227760000000000e+01
+5.438380000000002e+02 +1.543500000000000e+02
+3.221180000000000e+02 +1.322940000000000e+02
+6.167480000000000e+02 +1.539570000000000e+02
+1.063460000000000e+01 +4.652870000000000e+01
+3.858000000000000e+02 +1.180080000000000e+02
+1.219810000000000e+02 +7.291390000000000e+01
+2.772370000000000e+01 +5.212720000000000e+01
+6.514720000000000e+02 +2.007920000000000e+02
+2.429960000000000e+02 +9.653460000000000e+01
+6.246420000000000e+00 +3.545910000000000e+01
+5.317490000000000e+02 +1.509320000000000e+02
+4.010040000000000e+02 +1.089410000000000e+02
+2.347100000000000e+02 +9.266679999999999e+01
+2.669950000000000e+02 +1.067220000000000e+02
+5.234450000000001e+02 +1.381310000000000e+02
+3.841580000000000e+02 +1.316910000000000e+02
+4.236710000000000e+02 +1.384330000000000e+02
+6.184460000000000e+02 +1.496530000000000e+02
+8.786399999999998e+01 +6.170490000000000e+01
+6.354109999999999e+02 +1.411990000000000e+02
+2.418760000000000e+02 +1.085820000000000e+02
+3.818800000000000e+02 +1.186000000000000e+02
+2.936290000000000e+02 +1.225930000000000e+02
+2.598250000000000e+02 +1.091810000000000e+02
+1.075010000000000e+02 +6.155270000000000e+01
+5.872340000000000e+02 +1.375930000000000e+02
+2.121450000000000e+02 +1.043180000000000e+02
+2.328620000000000e+02 +1.046420000000000e+02
+3.237240000000000e+02 +1.152020000000000e+02
+4.813380000000000e+02 +1.333950000000000e+02
+3.324610000000000e+02 +1.169430000000000e+02
+7.027450000000000e+01 +6.587479999999999e+01
+8.869799999999999e+01 +6.284680000000000e+01
+2.305580000000000e+02 +8.898000000000000e+01
+2.641090000000000e+00 +4.962960000000000e+00
+6.487950000000000e+02 +1.427080000000000e+02
+1.188860000000000e+02 +7.187170000000000e+01
+8.000850000000000e+01 +6.221310000000000e+01
+4.926110000000000e+02 +1.336440000000000e+02
+3.976070000000000e+02 +1.118000000000000e+02
+2.393200000000000e+01 +6.393550000000000e+00
+5.488470000000000e+01 +5.292600000000000e+01
+5.827380000000001e+02 +1.494300000000000e+02
+1.824710000000000e+02 +8.888090000000000e+01
+3.519280000000001e+02 +1.211730000000000e+02
+4.697630000000000e+02 +1.431180000000000e+02
+3.691800000000000e+01 +5.232680000000000e+01
+2.535420000000000e+02 +1.029300000000000e+02
+3.222990000000001e+02 +1.025250000000000e+02
+1.226620000000000e+02 +6.833230000000000e+01
+2.483640000000000e+01 +4.532310000000000e+01
+1.086220000000000e+02 +7.032389999999999e+01
+3.751290000000000e+02 +1.139020000000000e+02
+2.303440000000000e+02 +9.479700000000000e+01
+1.882150000000000e+02 +9.588050000000000e+01
+3.153500000000000e+02 +9.606829999999999e+01
+8.098720000000000e+01 +5.430890000000000e+01
+6.461510000000000e+01 +5.060110000000000e+01
+3.405000000000000e+02 +9.062410000000000e+01
+2.228250000000000e+02 +9.337350000000001e+01
+4.708140000000000e+02 +1.435720000000000e+02
+2.188260000000000e+02 +7.747560000000000e+01
+6.149330000000000e+01 +5.579500000000000e+01
+2.261420000000000e+02 +8.752549999999999e+01
+7.569810000000000e+01 +5.397380000000000e+01
+9.631870000000001e+01 +6.470850000000000e+01
+5.724720000000000e+02 +1.432160000000000e+02
+2.115610000000000e+02 +8.110509999999999e+01
+3.765900000000000e+01 +4.531500000000000e+01
+4.018780000000000e+02 +1.229420000000000e+02
+7.664480000000000e+00 +2.323170000000000e+01
+4.420570000000000e+02 +1.281990000000000e+02
+1.352200000000000e+00 +1.133760000000000e+01
+3.676800000000000e+02 +1.193680000000000e+02
+2.267530000000000e+01 +1.474180000000000e+01
+1.453820000000000e+02 +6.669680000000000e+01
+2.633210000000000e+02 +1.063350000000000e+02
+2.843950000000000e+02 +9.123060000000000e+01
+2.336900000000000e+02 +9.781970000000000e+01
+3.292980000000000e+02 +8.123210000000000e+01
+2.507630000000000e+02 +9.676519999999999e+01
+8.671570000000000e+01 +5.585650000000000e+01
+2.378510000000000e+02 +8.701990000000001e+01
+9.106670000000000e+01 +5.640670000000000e+01
+2.902530000000000e+02 +1.032570000000000e+02
+3.143940000000000e+02 +1.033070000000000e+02
+4.339780000000000e+02 +9.880380000000000e+01
+1.369030000000000e+02 +7.702549999999999e+01
+9.613310000000000e+01 +5.359420000000000e+01
+2.038470000000000e+02 +7.252690000000000e+01
+3.626900000000000e+02 +1.028370000000000e+02
+7.269050000000000e+01 +4.430010000000000e+01
+5.285050000000000e+02 +1.116950000000000e+02
+1.024570000000000e+02 +6.970059999999999e+01
+1.936150000000000e+02 +7.289980000000000e+01
+3.668730000000001e+02 +8.768519999999999e+01
+6.703570000000001e+01 +4.424180000000000e+01
+9.595530000000000e+00 +1.243810000000000e+01
+2.616100000000000e+02 +1.025710000000000e+02
+1.731500000000000e+02 +9.109229999999999e+01
+1.023420000000000e+02 +5.201800000000000e+01
+4.223830000000000e+02 +1.131040000000000e+02
+1.152360000000000e+01 +3.808030000000000e+01
+8.230549999999999e+00 +5.709140000000001e+00
+1.411380000000000e+02 +4.696060000000000e+01
+1.916220000000000e+02 +7.624530000000000e+01
+2.692610000000000e+02 +1.002310000000000e+02
+1.196780000000000e+02 +5.656830000000000e+01
+3.562790000000000e+02 +8.766200000000001e+01
+1.946400000000000e+02 +8.255589999999999e+01
+1.035420000000000e+02 +4.914360000000000e+01
+6.545460000000000e+00 +5.678270000000000e+00
+4.490750000000000e+02 +9.516070000000001e+01
+6.034870000000000e+00 +2.347100000000000e+01
+2.475200000000000e+02 +7.625140000000000e+01
+2.430310000000000e+02 +8.937700000000002e+01
+6.359580000000000e+00 +3.079100000000000e+00
+1.800740000000000e+02 +7.234860000000000e+01
+3.417940000000001e+02 +1.070790000000000e+02
+1.158350000000000e+02 +5.342430000000000e+01
+8.516450000000000e+01 +4.742810000000000e+01
+2.640840000000000e+02 +9.718110000000000e+01
+3.774280000000001e+02 +9.485630000000000e+01
+3.750860000000000e+02 +9.928640000000000e+01
+8.922250000000000e-01 +1.459500000000000e+00
+2.991950000000000e+01 +3.475380000000000e+01
+2.523480000000000e+02 +7.795550000000000e+01
+2.111350000000000e+02 +6.938190000000000e+01
+3.978660000000000e+00 +3.295250000000000e+01
+1.913380000000000e+02 +6.926110000000000e+01
+1.074550000000000e+02 +4.634390000000000e+01
+1.192380000000000e+02 +4.974560000000000e+01
+1.681120000000000e+02 +6.157480000000000e+01
+2.373340000000000e+02 +7.898050000000001e+01
+1.579440000000000e+02 +8.500570000000000e+01
+2.264740000000000e+02 +7.852340000000000e+01
+3.154800000000000e+02 +7.479530000000000e+01
+1.719430000000000e+01 +3.158610000000000e+01
+3.096590000000000e+01 +3.400750000000000e+01
+3.458360000000000e+02 +8.752799999999998e+01
+2.014100000000000e+02 +7.453189999999999e+01
+7.627090000000000e+01 +3.556170000000000e+01
+5.962080000000000e+01 +3.185840000000000e+01
+4.036100000000000e+02 +9.546050000000000e+01
+7.901260000000001e+01 +4.226200000000000e+01
+3.387569999999999e+02 +8.813530000000000e+01
+3.333850000000000e+01 +2.944380000000000e+01
+1.255960000000000e+02 +5.651710000000000e+01
+2.850270000000000e+02 +1.036990000000000e+02
+1.605700000000000e+02 +5.966360000000000e+01
+2.094100000000000e+02 +6.096890000000000e+01
+3.202870000000001e+02 +8.083530000000000e+01
+2.176250000000000e+02 +6.721820000000000e+01
+3.963080000000000e+01 +1.660640000000000e+01
+2.142030000000000e+02 +6.700860000000000e+01
+2.564970000000000e+02 +6.583940000000000e+01
+1.043340000000000e+02 +4.830510000000000e+01
+6.539630000000000e+01 +2.757680000000000e+01
+3.210530000000000e+02 +8.617990000000000e+01
+1.540690000000000e+02 +6.705110000000001e+01
+2.450250000000000e+01 +2.414380000000000e+01
+2.930170000000000e+00 +2.362610000000000e+01
+2.098080000000000e+02 +6.765860000000001e+01
+2.225990000000000e+02 +7.089990000000000e+01
+3.084060000000000e+02 +8.096830000000000e+01
+1.684160000000000e+02 +6.832080000000001e+01
+3.013230000000000e+00 +2.280970000000000e+01
+1.804090000000000e+02 +5.864390000000000e+01
+1.912900000000000e+02 +7.448040000000000e+01
+6.327500000000000e+00 +8.012030000000001e+00
+5.144750000000000e-01 +2.183390000000000e+01
+4.398600000000000e+01 +2.671430000000000e+01
+2.027290000000000e+02 +5.964050000000000e+01
+5.197480000000000e+01 +2.443170000000000e+01
+8.888790000000000e+01 +3.510300000000000e+01
+2.208140000000000e+02 +6.093620000000000e+01
+2.314420000000000e+02 +9.180570000000000e+01
+3.012490000000000e+02 +9.008560000000000e+01
+2.421250000000000e+02 +5.859250000000000e+01
+8.258499999999999e+01 +3.807120000000000e+01
+1.389370000000000e+02 +4.724380000000000e+01
+1.496960000000000e+02 +4.538250000000000e+01
+2.762730000000000e+02 +7.434670000000000e+01
+2.310180000000000e+00 +1.973930000000000e+01
+2.659820000000000e+02 +7.423480000000001e+01
+6.236010000000000e+01 +3.346070000000000e+01
+1.770580000000000e+02 +6.980020000000000e+01
+1.635550000000000e+02 +5.620630000000000e+01
+1.242290000000000e+02 +3.002990000000000e+01
+1.470630000000000e+02 +4.534450000000000e+01
+4.472060000000000e+01 +1.961940000000000e+01
+2.585100000000000e+02 +7.879250000000000e+01
+5.806350000000000e+01 +2.021670000000000e+01
+2.080500000000000e+02 +5.157100000000001e+01
+2.404970000000000e+02 +6.865370000000000e+01
+2.166710000000000e+02 +6.536230000000000e+01
+1.291000000000000e+02 +4.039490000000000e+01
+3.346520000000000e-01 +1.459500000000000e+00
+1.974340000000000e+02 +6.870350000000001e+01
+1.069830000000000e+02 +3.256550000000000e+01
+2.236380000000000e+02 +6.651270000000000e+01
+2.127860000000000e+02 +6.237460000000000e+01
+2.089320000000000e+02 +6.358770000000000e+01
+1.377000000000000e+02 +4.318490000000000e+01
+1.035810000000000e+02 +3.155580000000000e+01
+1.544480000000000e+02 +4.477010000000000e+01
+1.637480000000000e+02 +4.402140000000000e+01
+5.647210000000000e+01 +2.762960000000000e+01
+1.971780000000000e+02 +6.568040000000001e+01
+4.224290000000000e+01 +1.524730000000000e+01
+4.534940000000000e+01 +1.916780000000000e+01
+2.401010000000000e+01 +1.430790000000000e+01
+1.243900000000000e+02 +4.473100000000000e+01
+9.850650000000000e+01 +4.413360000000000e+01
+7.094459999999999e+01 +1.965690000000000e+01
+4.762130000000000e+01 +2.986750000000000e+01
+2.133360000000000e+01 +1.391310000000000e+01
+5.515400000000000e+01 +1.720420000000000e+01
+8.368770000000001e+01 +3.723320000000000e+01
+1.414360000000000e+02 +4.833050000000000e+01
+7.140200000000000e+01 +2.472870000000000e+01
+9.516990000000000e+01 +3.356010000000000e+01
+1.125180000000000e+02 +4.901150000000000e+01
+8.174079999999999e+01 +2.558380000000000e+01
+1.501470000000000e+02 +5.230960000000000e+01
+8.192540000000000e+01 +3.582850000000000e+01
+9.256900000000000e+01 +6.721700000000000e+01
+3.849120000000000e+01 +1.121710000000000e+01
+1.033690000000000e+02 +4.048070000000000e+01
+1.255790000000000e+02 +4.622130000000000e+01
+1.405960000000000e+01 +9.885160000000001e+00
+1.391200000000000e+02 +5.504050000000000e+01
+4.220250000000000e+01 +1.332200000000000e+01
+6.852900000000000e+01 +1.889220000000000e+01
+6.890630000000000e+01 +2.051710000000000e+01
+9.432920000000000e+01 +2.946110000000000e+01
+9.147230000000000e+01 +3.428420000000000e+01
+6.357150000000000e+01 +2.081650000000000e+01
+7.075109999999999e+01 +1.929510000000000e+01
+8.456170000000000e+01 +2.761380000000000e+01
+2.581800000000000e+01 +9.842380000000000e+00
+6.587350000000001e+01 +3.085920000000000e+01
+6.751620000000000e+01 +2.032930000000000e+01
+9.074240000000000e+01 +3.351230000000000e+01
+1.006960000000000e+02 +4.267510000000000e+01
+1.553180000000000e+01 +7.263350000000000e+00
+5.666550000000000e+01 +2.636110000000000e+01
+1.018180000000000e+01 +6.933560000000001e+00
+8.089210000000000e+01 +4.146680000000000e+01
+7.364319999999998e+00 +5.733820000000000e+00
+5.427690000000000e+01 +2.104760000000000e+01
+2.825420000000000e+01 +5.871200000000000e+00
+4.044720000000000e-01 +4.865000000000000e+00
+6.650620000000001e+01 +3.819400000000000e+01
+5.966140000000000e+01 +3.070790000000000e+01
+3.285420000000000e+01 +1.437810000000000e+01
+2.477780000000000e+01 +5.478730000000001e+00
+3.570510000000000e+01 +1.708600000000000e+01
+3.884440000000000e+00 +3.988470000000000e+00
+1.589760000000000e+01 +7.781300000000000e+00
+4.394050000000000e+01 +2.856660000000000e+01
+4.148580000000000e+01 +2.322860000000000e+01
+3.787390000000000e+01 +2.871620000000000e+01
+2.865460000000000e+01 +1.205590000000000e+01
+1.366360000000000e+01 +4.280830000000000e+00
+6.474750000000000e-01 +1.946000000000000e+00
+1.492360000000000e+01 +3.573130000000000e+00
+2.287520000000000e+01 +1.546430000000000e+01
+1.843750000000000e+01 +1.357310000000000e+01
+1.338260000000000e+01 +5.906320000000000e+00
+4.971600000000000e+00 +6.712650000000000e-01
+4.197890000000000e+00 +4.865000000000000e-01
+2.574990000000000e+00 +0.000000000000000e+00
+0.000000000000000e+00 +0.000000000000000e+00
+0.000000000000000e+00 +0.000000000000000e+00
+0.000000000000000e+00 +0.000000000000000e+00
+0.000000000000000e+00 +0.000000000000000e+00
+0.000000000000000e+00 +0.000000000000000e+00
+0.000000000000000e+00 +0.000000000000000e+00
+0.000000000000000e+00 +0.000000000000000e+00
+0.000000000000000e+00 +0.000000000000000e+00
+0.000000000000000e+00 +0.000000000000000e+00
+0.000000000000000e+00 +0.000000000000000e+00
+0.000000000000000e+00 +0.000000000000000e+00
+0.000000000000000e+00 +0.000000000000000e+00
+0.000000000000000e+00 +0.000000000000000e+00
+2.156010000000000e+03 +1.285980000000000e+03
+1.058870000000000e+03 +9.783850000000000e+02
+1.710230000000000e+03 +1.208580000000000e+03
};
\nextgroupplot[
xlabel={Distancia Total},
ylabel={CO2 Total [g]},
xmin=-141.460057014178, xmax=2838.93005701418,
ymin=-73.1404919642858, ymax=1350.70625925454,
width=\figurewidth,
height=\figureheight,
tick align=outside,
tick pos=left,
xmajorgrids,
x grid style={lightgray!84.183006535947712!black},
ymajorgrids,
y grid style={lightgray!84.183006535947712!black},
axis line style={white},
legend entries={{PER 1.0}},
legend cell align={left},
legend style={draw=white!80.0!black, fill=white!89.803921568627459!black}
]
\addplot [only marks, draw=color0, fill=color0, opacity=0.75, colormap/viridis]
table{%
x                      y
+1.855480000000000e+03 +5.762230000000002e+02
+6.929910000000001e+02 +1.921190000000000e+02
+1.358890000000000e+03 +3.848280000000000e+02
+1.082450000000000e+03 +2.874450000000000e+02
+1.000600000000000e+03 +3.628780000000000e+02
+2.789250000000000e+02 +7.572980000000000e+01
+1.005390000000000e+03 +3.003390000000000e+02
+9.784960000000000e+02 +3.172330000000000e+02
+6.599019999999998e+02 +2.089550000000000e+02
+1.035650000000000e+03 +2.612740000000000e+02
+9.931680000000000e+02 +2.831490000000000e+02
+1.063840000000000e+03 +3.526730000000000e+02
+8.571039999999998e+02 +2.432620000000000e+02
+4.916920000000000e+02 +1.883340000000000e+02
+9.213240000000000e+02 +3.021950000000000e+02
+6.937250000000000e+02 +1.977470000000000e+02
+8.678980000000000e+02 +2.545750000000000e+02
+2.927260000000000e+02 +1.048630000000000e+02
+7.761550000000000e+02 +1.835240000000000e+02
+1.009020000000000e+03 +3.099800000000000e+02
+9.732370000000000e+02 +1.025780000000000e+03
+7.702950000000000e+02 +1.744430000000000e+02
+2.751700000000000e+02 +8.089990000000000e+01
+7.883260000000000e+02 +1.809040000000000e+02
+1.405180000000000e+03 +4.418950000000000e+02
+1.314050000000000e+03 +4.258160000000000e+02
+1.128820000000000e+03 +3.667280000000000e+02
+6.205850000000000e+02 +2.518610000000000e+02
+1.113520000000000e+03 +3.501210000000000e+02
+9.923200000000001e+02 +2.927710000000000e+02
+2.688120000000000e+03 +8.025640000000000e+02
+1.861630000000000e+03 +5.807640000000000e+02
+1.787660000000000e+03 +7.099770000000000e+02
+1.419050000000000e+03 +4.060640000000000e+02
+2.827610000000000e+02 +8.403310000000000e+01
+2.673470000000000e+03 +8.425549999999999e+02
+3.244580000000000e+02 +9.122929999999999e+01
+9.149010000000000e+02 +2.651070000000000e+02
+6.220290000000000e+02 +2.526530000000000e+02
+7.652639999999999e+02 +1.727640000000000e+02
+9.018980000000000e+02 +2.526520000000000e+02
+1.007110000000000e+03 +3.339180000000000e+02
+1.316050000000000e+03 +4.421870000000000e+02
+1.333760000000000e+03 +3.296260000000000e+02
+2.332510000000000e+03 +9.122460000000000e+02
+9.943860000000000e+02 +2.388950000000000e+02
+1.555860000000000e+03 +5.114280000000001e+02
+7.528860000000002e+02 +2.110180000000000e+02
+7.750030000000000e+02 +1.775380000000000e+02
+9.961030000000000e+02 +3.409640000000000e+02
+1.076600000000000e+03 +3.308470000000000e+02
+9.726490000000000e+02 +2.722900000000000e+02
+1.407650000000000e+03 +5.898240000000002e+02
+6.405459999999998e+02 +1.914770000000000e+02
+2.311750000000000e+03 +6.813310000000000e+02
+1.037340000000000e+03 +3.221860000000000e+02
+1.544140000000000e+03 +4.232650000000000e+02
+8.793720000000000e+02 +1.991970000000000e+02
+6.769030000000000e+02 +1.501700000000000e+02
+1.090250000000000e+03 +3.401010000000000e+02
+1.077230000000000e+03 +2.997320000000000e+02
+1.615110000000000e+03 +6.186630000000000e+02
+1.853750000000000e+03 +6.009050000000000e+02
+1.281020000000000e+03 +4.252830000000000e+02
+7.615050000000000e+02 +1.736690000000000e+02
+1.561150000000000e+03 +4.831950000000000e+02
+1.718540000000000e+03 +5.227080000000002e+02
+1.521870000000000e+03 +5.467400000000000e+02
+6.814839999999998e+02 +2.248440000000000e+02
+1.367550000000000e+03 +5.862950000000000e+02
+8.424169999999998e+02 +1.823740000000000e+02
+9.092920000000000e+02 +2.550590000000000e+02
+9.142070000000000e+02 +2.895840000000000e+02
+1.411260000000000e+03 +4.172600000000000e+02
+1.783500000000000e+03 +5.521100000000000e+02
+8.813600000000000e+02 +2.085740000000000e+02
+6.657650000000000e+02 +1.975780000000000e+02
+9.034400000000001e+02 +3.262220000000000e+02
+4.922770000000000e+02 +1.689610000000000e+02
+6.927030000000000e+02 +2.880380000000000e+02
+7.652170000000000e+02 +1.867300000000000e+02
+1.120120000000000e+03 +3.267010000000000e+02
+6.450870000000000e+02 +3.140700000000000e+02
+6.250269999999998e+02 +1.889840000000000e+02
+1.033020000000000e+03 +2.547700000000000e+02
+8.701080000000002e+02 +1.876110000000000e+02
+4.859460000000000e+02 +1.624270000000000e+02
+6.028950000000000e+02 +1.838360000000000e+02
+6.742680000000000e+02 +2.125790000000000e+02
+3.369630000000000e+02 +9.436620000000000e+02
+7.680560000000000e+02 +1.732610000000000e+02
+1.349740000000000e+03 +4.375370000000000e+02
+6.668960000000002e+02 +2.562690000000000e+02
+1.574100000000000e+03 +4.853120000000000e+02
+1.303820000000000e+03 +4.427500000000000e+02
+1.864100000000000e+03 +5.682560000000000e+02
+1.446080000000000e+03 +4.585970000000000e+02
+7.516039999999998e+02 +1.547780000000000e+02
+1.297880000000000e+03 +4.109230000000000e+02
+1.059400000000000e+03 +3.166650000000000e+02
+1.029620000000000e+03 +2.795290000000000e+02
+8.765510000000000e+02 +1.877650000000000e+02
+6.209950000000000e+02 +2.531670000000000e+02
+7.729119999999998e+02 +1.782310000000000e+02
+1.392970000000000e+03 +4.361450000000000e+02
+8.909920000000000e+02 +1.869340000000000e+02
+8.542030000000000e+02 +3.032920000000001e+02
+6.794160000000001e+02 +2.251070000000000e+02
+8.859299999999999e+02 +2.083930000000000e+02
+9.172310000000000e+02 +3.281200000000000e+02
+1.004570000000000e+03 +2.894760000000000e+02
+1.309740000000000e+03 +3.572710000000000e+02
+9.760050000000000e+02 +2.768950000000000e+02
+7.774040000000000e+02 +2.229680000000000e+02
+1.082910000000000e+03 +2.985600000000000e+02
+7.743650000000000e+02 +1.637170000000000e+02
+9.081860000000000e+02 +3.330750000000000e+02
+1.045950000000000e+03 +2.511100000000000e+02
+1.289490000000000e+03 +3.595590000000000e+02
+3.345040000000000e+02 +8.788060000000000e+01
+1.230000000000000e+03 +4.566070000000000e+02
+1.061130000000000e+03 +2.919460000000000e+02
+1.127560000000000e+03 +3.010920000000000e+02
+6.511390000000000e+02 +1.826740000000000e+02
+9.662790000000000e+02 +1.022440000000000e+03
+4.920510000000000e+02 +1.485950000000000e+02
+1.412390000000000e+03 +5.474119999999998e+02
+8.554490000000000e+02 +2.237780000000000e+02
+6.699600000000000e+02 +1.771780000000000e+02
+1.322320000000000e+03 +4.440510000000000e+02
+7.695260000000002e+02 +1.724830000000000e+02
+1.302550000000000e+03 +4.084160000000000e+02
+1.475220000000000e+03 +4.321400000000000e+02
+1.364020000000000e+03 +4.284630000000000e+02
+4.942410000000000e+02 +1.139950000000000e+02
+9.976070000000000e+02 +2.613220000000000e+02
+1.755750000000000e+03 +4.493670000000000e+02
+1.335690000000000e+03 +4.828810000000000e+02
+6.873520000000000e+02 +3.174160000000000e+02
+1.055960000000000e+03 +2.680840000000000e+02
+1.679330000000000e+03 +5.110550000000000e+02
+6.679220000000000e+02 +2.114380000000000e+02
+1.125690000000000e+03 +3.589450000000000e+02
+1.321400000000000e+03 +3.372820000000000e+02
+1.151700000000000e+03 +4.413490000000000e+02
+6.867030000000000e+02 +2.811980000000000e+02
+1.868770000000000e+03 +5.459750000000000e+02
+1.061180000000000e+03 +2.532270000000000e+02
+3.276340000000000e+02 +9.365230000000000e+02
+7.820750000000000e+02 +1.788590000000000e+02
+1.593120000000000e+03 +5.592900000000000e+02
+4.937380000000001e+02 +1.466710000000000e+02
+7.530640000000000e+02 +2.288480000000000e+02
+9.970080000000000e+02 +3.153840000000000e+02
+1.083920000000000e+03 +2.811040000000000e+02
+1.138390000000000e+03 +3.900800000000000e+02
+1.306300000000000e+03 +3.431510000000000e+02
+7.503439999999998e+02 +1.751140000000000e+02
+1.112550000000000e+03 +3.015810000000000e+02
+1.409210000000000e+03 +5.645470000000000e+02
+9.030790000000000e+02 +2.807370000000000e+02
+9.929140000000000e+02 +2.669550000000000e+02
+1.624530000000000e+03 +5.875369999999998e+02
+1.548120000000000e+03 +4.723120000000000e+02
+1.330140000000000e+03 +5.307950000000000e+02
+2.774470000000000e+02 +7.499209999999999e+01
+9.401740000000000e+02 +3.102770000000000e+02
+1.289110000000000e+03 +4.426280000000000e+02
+7.782580000000000e+02 +1.812540000000000e+02
+1.296930000000000e+03 +3.941490000000000e+02
+8.826810000000000e+02 +2.116740000000000e+02
+9.219180000000000e+02 +3.178420000000000e+02
+1.004130000000000e+03 +2.096680000000000e+02
+8.392420000000000e+02 +1.831940000000000e+02
+9.895470000000000e+02 +2.723180000000000e+02
+1.056690000000000e+03 +2.815290000000000e+02
+6.888700000000000e+02 +2.659720000000000e+02
+9.011180000000001e+02 +2.097420000000000e+02
+1.251460000000000e+03 +5.405610000000000e+02
+1.854570000000000e+03 +5.562070000000000e+02
+9.510910000000000e+02 +2.349960000000000e+02
+7.621510000000002e+02 +2.811230000000000e+02
+1.687820000000000e+03 +5.278480000000002e+02
+9.252900000000000e+02 +2.871800000000000e+02
+1.068550000000000e+03 +4.446400000000000e+02
+1.285780000000000e+03 +5.088640000000000e+02
+2.799990000000000e+02 +7.314510000000000e+01
+8.868789999999998e+02 +2.710780000000000e+02
+1.488760000000000e+03 +4.412310000000000e+02
+1.506670000000000e+03 +4.848570000000000e+02
+7.533420000000000e+02 +2.854090000000000e+02
+8.638570000000000e+02 +2.111490000000000e+02
+1.050970000000000e+03 +4.520430000000000e+02
+6.372060000000000e+02 +1.610880000000000e+02
+8.693700000000000e+02 +2.090600000000000e+02
+1.463790000000000e+03 +5.189920000000000e+02
+6.746410000000002e+02 +1.974830000000000e+02
+8.406660000000001e+02 +2.053060000000000e+02
+1.306350000000000e+03 +4.341650000000000e+02
+6.915980000000002e+02 +2.653670000000000e+02
+1.555840000000000e+03 +4.350350000000000e+02
+8.665139999999999e+02 +3.941210000000000e+02
+1.079740000000000e+03 +4.366230000000001e+02
+1.107910000000000e+03 +3.161820000000000e+02
+9.343110000000000e+02 +2.313120000000000e+02
+4.536680000000000e+02 +1.205450000000000e+02
+1.375300000000000e+03 +4.120170000000000e+02
+1.371760000000000e+03 +5.142590000000000e+02
+9.590110000000000e+02 +1.007070000000000e+03
+8.587040000000000e+02 +2.491470000000000e+02
+3.163020000000000e+02 +9.133520000000000e+02
+3.415719999999999e+02 +9.317020000000000e+02
+6.427090000000002e+02 +1.723590000000000e+02
+8.410200000000000e+02 +2.434580000000000e+02
+6.929989999999998e+02 +2.680460000000000e+02
+7.707869999999998e+02 +1.898820000000000e+02
+1.074230000000000e+03 +4.067760000000000e+02
+6.367410000000000e+02 +1.714570000000000e+02
+3.374630000000000e+02 +1.004820000000000e+02
+7.727520000000000e+02 +1.945290000000000e+02
+7.538180000000000e+02 +2.323300000000000e+02
+1.063320000000000e+03 +3.000790000000000e+02
+1.289720000000000e+03 +5.114040000000000e+02
+6.403869999999999e+02 +1.617240000000000e+02
+1.728760000000000e+03 +5.976080000000002e+02
+6.798819999999999e+02 +3.147050000000000e+02
+6.556920000000000e+02 +1.848440000000000e+02
+1.871470000000000e+03 +7.239939999999998e+02
+9.569390000000000e+02 +2.649640000000000e+02
+9.852000000000000e+02 +3.444550000000000e+02
+1.143960000000000e+03 +3.799840000000000e+02
+1.095920000000000e+03 +3.843100000000000e+02
+9.154320000000000e+02 +2.974920000000000e+02
+1.840350000000000e+03 +6.983670000000000e+02
+1.018190000000000e+03 +2.944140000000000e+02
+1.717270000000000e+03 +4.704550000000000e+02
+9.112089999999999e+02 +2.918340000000000e+02
+1.860700000000000e+03 +7.186730000000000e+02
+4.910120000000000e+02 +1.392520000000000e+02
+6.522100000000000e+02 +3.032040000000000e+02
+6.853520000000000e+02 +3.264960000000000e+02
+1.101700000000000e+03 +3.129270000000000e+02
+9.594070000000000e+02 +9.899020000000000e+02
+7.663120000000000e+02 +2.770310000000000e+02
+1.134520000000000e+03 +4.315100000000000e+02
+6.645080000000000e+02 +1.605900000000000e+02
+7.520870000000000e+02 +2.711720000000000e+02
+1.363140000000000e+03 +4.655970000000000e+02
+1.867270000000000e+03 +7.279380000000000e+02
+1.146150000000000e+03 +3.931890000000000e+02
+7.458620000000000e+02 +2.558730000000000e+02
+7.681710000000000e+02 +2.665830000000000e+02
+1.300960000000000e+03 +3.674950000000000e+02
+8.795350000000000e+02 +2.084550000000000e+02
+8.732840000000000e+02 +2.950000000000000e+02
+2.326670000000000e+03 +7.962250000000000e+02
+1.320140000000000e+03 +5.584720000000000e+02
+7.746770000000000e+02 +2.580000000000000e+02
+1.865620000000000e+03 +5.432330000000002e+02
+1.060150000000000e+03 +3.605420000000000e+02
+1.295920000000000e+03 +3.740690000000000e+02
+1.932970000000000e+03 +6.201440000000000e+02
+1.753470000000000e+03 +4.467900000000000e+02
+1.388780000000000e+03 +4.815980000000000e+02
+1.094170000000000e+03 +4.653680000000001e+02
+6.664250000000000e+02 +3.109330000000000e+02
+2.818490000000000e+02 +1.670380000000000e+02
+1.017310000000000e+03 +2.131460000000000e+02
+3.224120000000001e+02 +8.982610000000002e+02
+9.808030000000000e+02 +2.613660000000000e+02
+1.215670000000000e+03 +4.348280000000000e+02
+1.064130000000000e+03 +3.993420000000000e+02
+6.876990000000000e+02 +1.918580000000000e+02
+1.340300000000000e+03 +5.511860000000000e+02
+8.938839999999999e+02 +3.051470000000000e+02
+1.108810000000000e+03 +3.683110000000000e+02
+6.311300000000000e+02 +1.586450000000000e+02
+1.156770000000000e+03 +4.585940000000000e+02
+1.874760000000000e+03 +7.157910000000001e+02
+6.782950000000000e+02 +3.060350000000000e+02
+1.017980000000000e+03 +3.279250000000000e+02
+4.870110000000000e+02 +1.248150000000000e+02
+2.771970000000000e+02 +1.648880000000000e+02
+7.673850000000000e+02 +2.449350000000000e+02
+1.868780000000000e+03 +5.396559999999999e+02
+6.868489999999998e+02 +3.171990000000000e+02
+6.226500000000000e+02 +2.177620000000000e+02
+9.735540000000000e+02 +2.559680000000000e+02
+8.776870000000000e+02 +1.967550000000000e+02
+1.074710000000000e+03 +4.024310000000000e+02
+6.854330000000000e+02 +2.119070000000000e+02
+9.873360000000000e+02 +2.023240000000000e+02
+1.084100000000000e+03 +3.131540000000000e+02
+1.066170000000000e+03 +4.192660000000000e+02
+6.612380000000001e+02 +1.534970000000000e+02
+1.853290000000000e+03 +6.566380000000000e+02
+7.751060000000001e+02 +2.508850000000000e+02
+1.885520000000000e+03 +7.262480000000000e+02
+7.634880000000001e+02 +2.849280000000000e+02
+5.886840000000000e+02 +2.065020000000000e+02
+6.051230000000000e+02 +1.561720000000000e+02
+1.301590000000000e+03 +5.386690000000000e+02
+1.740150000000000e+03 +5.671880000000000e+02
+1.072860000000000e+03 +4.225580000000000e+02
+3.017960000000000e+02 +9.017950000000000e+02
+9.745210000000000e+02 +2.419170000000000e+02
+1.326760000000000e+03 +5.079220000000000e+02
+1.131450000000000e+03 +2.917190000000000e+02
+1.079120000000000e+03 +3.952870000000000e+02
+7.544160000000001e+02 +1.835790000000000e+02
+2.667070000000000e+03 +7.540460000000000e+02
+1.115000000000000e+03 +3.033950000000000e+02
+1.412080000000000e+03 +5.306840000000000e+02
+7.684710000000000e+02 +2.474170000000000e+02
+1.385550000000000e+03 +4.512970000000000e+02
+2.340510000000000e+03 +7.665750000000000e+02
+8.846790000000000e+02 +2.998610000000000e+02
+1.478670000000000e+03 +6.628400000000000e+02
+1.371670000000000e+03 +4.498480000000000e+02
+7.862030000000000e+02 +2.389960000000000e+02
+9.374980000000000e+02 +3.135000000000000e+02
+1.294660000000000e+03 +3.505950000000000e+02
+6.514410000000000e+02 +1.960750000000000e+02
+1.314960000000000e+03 +3.849430000000000e+02
+1.736100000000000e+03 +4.253870000000000e+02
+2.819530000000000e+02 +1.510340000000000e+02
+1.868040000000000e+03 +6.801239999999998e+02
+1.195210000000000e+03 +2.949110000000000e+02
+6.896230000000000e+02 +2.775350000000000e+02
+1.767740000000000e+03 +5.393080000000000e+02
+9.855599999999999e+02 +3.366300000000000e+02
+9.644340000000000e+02 +9.740020000000000e+02
+3.122490000000000e+02 +8.784730000000002e+02
+7.724630000000002e+02 +2.644860000000000e+02
+1.432610000000000e+03 +4.273100000000000e+02
+1.485030000000000e+03 +6.070459999999998e+02
+7.563610000000001e+02 +2.466370000000000e+02
+8.797869999999998e+02 +2.578890000000000e+02
+1.070180000000000e+03 +4.188120000000000e+02
+4.980300000000000e+02 +1.318860000000000e+02
+9.409660000000000e+02 +9.930490000000000e+02
+2.910620000000000e+02 +8.730820000000000e+02
+1.553080000000000e+03 +4.713300000000000e+02
+9.848819999999999e+02 +9.847290000000000e+02
+7.644370000000000e+02 +2.456600000000000e+02
+1.154520000000000e+03 +4.425270000000000e+02
+2.761120000000000e+02 +1.571200000000000e+02
+6.336230000000000e+02 +1.506800000000000e+02
+1.642100000000000e+03 +6.176310000000000e+02
+9.473900000000000e+02 +2.716230000000000e+02
+1.099940000000000e+03 +3.555550000000000e+02
+1.137910000000000e+03 +4.124660000000000e+02
+1.040900000000000e+03 +3.262770000000000e+02
+7.773270000000000e+02 +2.680380000000000e+02
+1.095900000000000e+03 +3.651480000000000e+02
+9.908400000000000e+02 +4.014650000000000e+02
+1.383490000000000e+03 +5.461300000000000e+02
+1.621390000000000e+03 +5.185160000000000e+02
+8.854490000000000e+02 +3.205190000000000e+02
+5.109180000000000e+02 +9.280920000000000e+02
+7.477020000000000e+02 +2.343330000000000e+02
+7.006310000000002e+02 +2.407020000000000e+02
+1.994490000000000e+03 +6.490620000000000e+02
+6.752819999999998e+02 +2.658080000000000e+02
+1.853440000000000e+03 +4.991810000000000e+02
+8.630560000000000e+02 +2.654820000000000e+02
+1.100030000000000e+03 +3.251230000000000e+02
+7.790549999999999e+02 +2.532000000000000e+02
+9.754500000000000e+02 +2.831890000000000e+02
+9.395630000000000e+02 +3.487740000000000e+02
+1.062900000000000e+03 +2.644960000000000e+02
+7.450100000000000e+02 +2.433160000000000e+02
+2.346410000000000e+03 +8.250410000000001e+02
+8.733410000000000e+02 +3.728510000000000e+02
+9.422270000000000e+02 +9.602569999999999e+02
+2.793860000000000e+02 +8.759950000000000e+02
+9.775520000000000e+02 +3.006060000000000e+02
+6.353480000000002e+02 +2.733630000000000e+02
+8.616540000000000e+02 +2.467180000000000e+02
+7.708620000000000e+02 +2.398200000000000e+02
+1.286160000000000e+03 +4.928690000000000e+02
+2.769890000000000e+02 +1.504650000000000e+02
+9.895940000000001e+02 +3.592630000000001e+02
+7.536330000000000e+02 +2.978340000000000e+02
+9.842980000000000e+02 +3.281330000000001e+02
+6.613760000000002e+02 +2.047840000000000e+02
+6.880050000000000e+02 +2.342990000000000e+02
+9.079470000000000e+02 +2.842730000000000e+02
+4.959750000000000e+02 +1.445010000000000e+02
+8.044800000000000e+02 +2.333180000000000e+02
+6.333490000000000e+02 +2.789830000000000e+02
+3.053900000000000e+02 +8.758839999999999e+02
+1.957680000000000e+03 +6.112230000000002e+02
+1.061520000000000e+03 +5.043020000000000e+02
+6.787100000000000e+02 +2.854980000000000e+02
+9.757900000000000e+02 +9.703430000000000e+02
+7.810250000000000e+02 +2.368160000000000e+02
+1.059000000000000e+03 +3.784780000000000e+02
+1.008060000000000e+03 +3.169470000000000e+02
+9.093170000000000e+02 +2.720610000000000e+02
+9.908130000000000e+02 +3.809480000000000e+02
+1.071600000000000e+03 +4.131790000000000e+02
+7.005340000000000e+02 +2.532910000000000e+02
+7.563689999999998e+02 +2.563570000000000e+02
+9.069530000000000e+02 +2.585240000000000e+02
+8.522619999999999e+02 +2.618030000000000e+02
+6.421030000000002e+02 +2.888660000000000e+02
+3.368600000000000e+02 +9.843290000000000e+01
+1.559630000000000e+03 +6.401120000000000e+02
+1.071320000000000e+03 +4.208300000000000e+02
+1.217790000000000e+03 +5.132560000000000e+02
+1.083850000000000e+02 +2.978880000000000e+01
+1.312460000000000e+03 +3.824400000000000e+02
+8.718589999999998e+02 +2.666170000000000e+02
+6.895599999999999e+02 +3.190360000000000e+02
+7.652080000000002e+02 +2.435000000000000e+02
+1.995420000000000e+03 +6.095580000000000e+02
+1.857680000000000e+03 +6.712040000000000e+02
+6.366369999999999e+02 +2.755780000000000e+02
+7.874839999999998e+02 +2.489550000000000e+02
+1.834420000000000e+03 +7.105680000000000e+02
+6.779220000000000e+02 +1.820910000000000e+02
+9.775950000000000e+02 +3.126520000000000e+02
+1.118680000000000e+03 +4.052200000000000e+02
+9.545810000000000e+02 +2.990700000000000e+02
+8.599119999999998e+02 +2.321800000000000e+02
+1.269720000000000e+03 +3.679870000000000e+02
+6.837869999999998e+02 +2.824300000000000e+02
+6.739550000000000e+02 +2.715060000000000e+02
+1.568390000000000e+03 +6.262869999999998e+02
+6.617710000000002e+02 +2.321480000000000e+02
+1.001510000000000e+03 +3.421560000000000e+02
+1.571630000000000e+03 +4.422320000000000e+02
+7.621860000000000e+02 +2.280240000000000e+02
+1.074730000000000e+03 +3.862230000000000e+02
+8.081730000000000e+02 +2.299900000000000e+02
+2.841130000000000e+02 +1.499130000000000e+02
+1.107510000000000e+03 +4.176090000000000e+02
+9.897880000000000e+02 +3.166280000000000e+02
+7.647830000000000e+02 +2.822030000000000e+02
+6.474290000000000e+02 +2.720350000000000e+02
+7.767300000000000e+02 +2.400480000000000e+02
+1.109260000000000e+03 +3.697680000000000e+02
+1.284240000000000e+03 +4.431350000000000e+02
+6.338350000000000e+02 +2.731000000000000e+02
+1.372080000000000e+03 +5.496700000000000e+02
+8.695520000000000e+02 +3.696150000000000e+02
+6.570790000000000e+02 +2.718250000000000e+02
+6.935280000000000e+02 +2.390330000000000e+02
+9.783480000000000e+02 +2.944840000000001e+02
+7.813930000000000e+02 +2.404740000000000e+02
+1.068750000000000e+03 +3.649880000000001e+02
+1.215980000000000e+03 +4.473560000000000e+02
+1.065360000000000e+03 +4.021610000000000e+02
+1.010290000000000e+03 +3.326250000000000e+02
+2.316700000000000e+03 +8.217869999999998e+02
+7.519169999999998e+02 +2.366230000000000e+02
+8.780500000000000e+02 +2.318600000000000e+02
+1.849450000000000e+03 +6.719060000000002e+02
+2.741560000000000e+02 +1.208880000000000e+02
+9.541310000000000e+02 +9.793160000000000e+02
+8.836020000000000e+02 +2.392410000000000e+02
+2.894710000000000e+02 +8.814080000000000e+02
+2.722100000000000e+02 +1.162870000000000e+02
+7.520630000000000e+02 +2.490120000000000e+02
+9.085930000000000e+02 +2.456540000000000e+02
+1.447420000000000e+03 +5.559109999999999e+02
+1.571700000000000e+03 +4.848700000000000e+02
+1.300200000000000e+03 +4.456150000000000e+02
+8.685360000000002e+02 +2.895490000000001e+02
+6.173750000000000e+02 +2.626950000000000e+02
+1.568770000000000e+03 +4.305540000000000e+02
+7.833930000000000e+02 +2.463940000000000e+02
+8.927630000000000e+02 +2.390160000000000e+02
+1.124050000000000e+03 +3.986900000000000e+02
+4.898910000000000e+02 +2.308010000000000e+02
+6.616050000000000e+02 +2.136490000000000e+02
+6.763070000000000e+02 +2.557310000000000e+02
+4.463600000000000e+02 +1.664770000000000e+02
+7.750110000000002e+02 +2.222630000000000e+02
+1.537790000000000e+03 +5.978049999999999e+02
+4.580520000000000e+02 +1.577200000000000e+02
+2.655820000000000e+02 +8.498889999999999e+02
+3.224100000000000e+02 +1.557540000000000e+02
+1.052960000000000e+03 +3.439120000000000e+02
+2.810300000000000e+02 +1.422040000000000e+02
+6.608700000000000e+02 +2.816660000000000e+02
+6.856400000000000e+02 +2.644710000000000e+02
+1.001390000000000e+03 +3.851060000000000e+02
+1.038310000000000e+03 +3.041710000000000e+02
+1.119310000000000e+03 +3.692400000000000e+02
+7.711050000000000e+02 +3.350069999999999e+02
+8.625419999999998e+02 +3.423890000000000e+02
+1.074490000000000e+03 +3.789170000000000e+02
+1.000060000000000e+03 +3.202940000000001e+02
+6.918090000000000e+02 +2.047370000000000e+02
+1.929100000000000e+03 +5.774370000000000e+02
+1.302610000000000e+03 +4.624720000000000e+02
+1.401780000000000e+03 +6.155720000000000e+02
+1.057420000000000e+03 +3.091350000000000e+02
+3.251410000000000e+02 +1.600060000000000e+02
+1.109320000000000e+03 +3.819290000000000e+02
+1.491390000000000e+03 +7.269440000000000e+02
+1.004920000000000e+03 +3.694420000000000e+02
+1.043590000000000e+03 +3.414270000000000e+02
+1.737300000000000e+03 +7.535089999999999e+02
+9.064880000000001e+02 +3.899070000000000e+02
+6.397780000000000e+02 +2.578530000000000e+02
+1.306460000000000e+03 +4.114300000000000e+02
+1.078750000000000e+02 +3.681970000000000e+01
+6.927280000000002e+02 +2.754630000000000e+02
+6.243390000000001e+02 +1.699160000000000e+02
+1.554790000000000e+03 +6.556700000000000e+02
+9.001080000000002e+02 +2.289900000000000e+02
+1.643010000000000e+03 +4.486020000000000e+02
+6.570520000000000e+02 +2.662630000000000e+02
+6.659150000000000e+02 +2.628750000000000e+02
+1.147470000000000e+03 +3.179760000000000e+02
+6.801230000000000e+02 +2.393190000000000e+02
+2.289960000000000e+03 +9.620390000000000e+02
+9.885520000000000e+02 +3.615860000000000e+02
+1.048400000000000e+03 +3.701770000000000e+02
+6.509550000000000e+02 +2.416600000000000e+02
+7.665280000000000e+02 +2.249370000000000e+02
+8.564400000000001e+02 +2.305110000000000e+02
+1.473170000000000e+03 +4.591890000000000e+02
+1.325310000000000e+03 +4.763920000000000e+02
+9.779730000000000e+02 +3.725190000000000e+02
+1.076090000000000e+03 +3.786740000000000e+02
+2.697470000000000e+03 +1.060860000000000e+03
+2.353080000000000e+03 +8.115000000000000e+02
+8.828780000000000e+02 +2.247130000000000e+02
+1.100910000000000e+03 +4.170400000000000e+02
+7.645510000000000e+02 +2.380240000000000e+02
+1.747960000000000e+03 +7.198720000000000e+02
+1.302030000000000e+03 +4.612610000000000e+02
+2.013590000000000e+03 +5.682710000000000e+02
+8.437950000000000e+02 +2.071430000000000e+02
+8.679060000000002e+02 +3.367860000000000e+02
+7.894340000000000e+02 +3.245800000000000e+02
+6.202510000000000e+02 +2.591620000000000e+02
+7.513789999999998e+02 +2.337540000000000e+02
+7.745980000000002e+02 +2.119140000000000e+02
+9.000630000000000e+02 +2.379660000000000e+02
+8.934250000000000e+02 +2.916260000000000e+02
+4.822860000000000e+02 +2.226210000000000e+02
+9.755839999999999e+02 +3.468520000000000e+02
+4.800760000000000e+02 +1.761690000000000e+02
+6.452130000000002e+02 +2.530100000000000e+02
+6.550530000000000e+02 +2.490910000000000e+02
+9.047310000000000e+02 +2.214620000000000e+02
+1.281020000000000e+03 +4.777620000000000e+02
+1.426250000000000e+02 +8.245980000000002e+02
+1.294290000000000e+03 +4.569880000000001e+02
+1.317090000000000e+03 +4.927710000000000e+02
+7.918639999999998e+02 +3.263230000000000e+02
+8.051940000000000e+02 +3.657800000000000e+02
+1.305950000000000e+03 +4.589680000000000e+02
+1.362340000000000e+03 +5.751880000000000e+02
+9.026230000000000e+02 +3.794160000000000e+02
+1.745480000000000e+03 +5.970930000000002e+02
+1.006470000000000e+03 +3.725650000000000e+02
+7.564780000000002e+02 +2.201250000000000e+02
+9.337050000000000e+02 +3.460410000000000e+02
+6.186330000000000e+02 +1.658780000000000e+02
+2.008270000000000e+03 +6.719939999999998e+02
+1.010440000000000e+03 +3.280119999999999e+02
+6.715030000000000e+02 +2.564150000000000e+02
+1.321370000000000e+03 +4.848240000000000e+02
+6.829069999999998e+02 +2.905890000000000e+02
+6.200520000000000e+02 +2.520550000000000e+02
+1.308850000000000e+03 +4.720430000000000e+02
+9.701990000000000e+02 +2.939640000000000e+02
+6.625000000000000e+02 +2.446890000000000e+02
+6.452040000000002e+02 +3.584060000000000e+02
+1.562040000000000e+03 +4.058720000000000e+02
+7.685910000000000e+02 +2.124540000000000e+02
+1.111010000000000e+03 +3.594760000000000e+02
+1.074890000000000e+03 +4.486490000000000e+02
+6.816640000000000e+02 +2.923720000000000e+02
+1.065210000000000e+03 +3.734380000000001e+02
+6.612930000000000e+02 +2.614000000000000e+02
+2.326120000000000e+03 +8.731239999999998e+02
+7.869800000000000e+02 +2.225040000000000e+02
+1.615310000000000e+03 +6.251360000000000e+02
+8.762460000000002e+02 +2.315470000000000e+02
+1.319820000000000e+03 +4.634770000000000e+02
+8.487940000000000e+02 +3.693060000000000e+02
+1.081300000000000e+03 +3.352790000000000e+02
+8.049220000000000e+02 +3.433850000000000e+02
+1.032840000000000e+03 +2.724200000000000e+02
+1.325730000000000e+03 +4.928530000000000e+02
+7.848960000000002e+02 +2.264960000000000e+02
+1.768030000000000e+03 +5.294480000000000e+02
+1.471360000000000e+03 +4.612600000000000e+02
+1.037910000000000e+03 +3.510880000000000e+02
+7.474190000000000e+02 +2.007840000000000e+02
+1.776370000000000e+03 +5.295830000000002e+02
+1.010250000000000e+03 +3.618670000000000e+02
+2.833470000000000e+02 +1.254680000000000e+02
+8.427150000000000e+02 +2.426030000000000e+02
+6.666430000000000e+02 +2.020190000000000e+02
+1.853830000000000e+03 +6.477990000000000e+02
+1.095390000000000e+03 +4.108250000000000e+02
+1.487440000000000e+03 +5.362040000000002e+02
+6.606139999999998e+02 +1.704090000000000e+02
+7.552110000000000e+02 +2.279440000000000e+02
+1.081190000000000e+03 +3.466300000000000e+02
+4.895660000000000e+02 +2.177890000000000e+02
+1.362440000000000e+03 +4.585490000000000e+02
+1.010850000000000e+03 +3.219510000000000e+02
+6.727919999999998e+02 +2.346930000000000e+02
+1.877950000000000e+03 +6.541770000000000e+02
+2.830630000000000e+02 +8.417220000000000e+02
+1.321620000000000e+03 +3.685110000000000e+02
+1.433640000000000e+03 +4.607570000000000e+02
+1.206440000000000e+03 +4.889770000000000e+02
+6.973700000000000e+02 +2.332020000000000e+02
+1.618460000000000e+03 +5.878250000000000e+02
+1.742300000000000e+03 +5.207750000000000e+02
+6.657360000000001e+02 +2.626130000000000e+02
+7.857339999999998e+02 +2.294820000000000e+02
+1.297900000000000e+03 +4.422740000000000e+02
+1.117060000000000e+03 +3.552460000000000e+02
+9.117630000000000e+02 +2.214270000000000e+02
+6.891310000000002e+02 +1.840240000000000e+02
+7.747790000000000e+02 +2.086980000000000e+02
+9.789890000000000e+02 +3.144040000000000e+02
+1.942310000000000e+03 +7.455900000000000e+02
+6.545140000000000e+02 +2.462750000000000e+02
+1.241810000000000e+03 +3.394820000000000e+02
+1.763760000000000e+03 +7.400460000000000e+02
+7.488099999999999e+02 +2.077640000000000e+02
+2.319060000000000e+03 +7.703939999999999e+02
+6.902739999999999e+02 +2.188890000000000e+02
+3.332760000000000e+02 +1.236700000000000e+02
+1.782100000000000e+03 +5.348030000000000e+02
+1.064850000000000e+03 +3.513980000000000e+02
+9.627830000000000e+02 +9.333210000000000e+02
+1.573340000000000e+03 +4.102370000000000e+02
+7.031540000000000e+02 +2.229460000000000e+02
+1.569280000000000e+03 +6.241430000000000e+02
+1.967380000000000e+03 +5.691519999999998e+02
+1.426260000000000e+03 +5.803660000000000e+02
+7.562210000000000e+02 +2.309380000000000e+02
+6.400960000000000e+02 +2.432730000000000e+02
+6.230760000000000e+02 +1.589190000000000e+02
+9.876860000000000e+02 +3.218570000000000e+02
+1.061240000000000e+03 +4.716770000000000e+02
+2.887470000000000e+02 +1.184770000000000e+02
+9.199310000000000e+02 +9.156630000000000e+02
+1.860820000000000e+03 +7.422350000000000e+02
+4.912730000000000e+02 +2.074540000000000e+02
+2.654390000000000e+02 +1.140280000000000e+02
+1.321260000000000e+03 +4.600180000000000e+02
+2.476750000000000e+02 +8.249200000000000e+02
+1.782300000000000e+03 +6.740160000000002e+02
+9.179360000000000e+02 +2.287770000000000e+02
+1.081360000000000e+03 +3.289410000000000e+02
+6.314010000000000e+02 +1.613220000000000e+02
+1.148280000000000e+02 +4.950730000000000e+01
+1.297950000000000e+03 +4.414870000000000e+02
+1.331810000000000e+03 +5.995210000000000e+02
+1.855760000000000e+03 +6.348650000000000e+02
+6.523640000000000e+02 +2.465390000000000e+02
+7.710790000000000e+02 +2.375970000000000e+02
+6.229310000000000e+02 +1.544500000000000e+02
+9.762480000000000e+02 +3.761960000000000e+02
+6.662869999999998e+02 +2.573830000000000e+02
+8.259540000000000e+02 +3.430640000000000e+02
+6.785820000000000e+02 +2.625900000000000e+02
+6.062000000000000e+02 +1.639320000000000e+02
+2.946740000000001e+02 +1.429460000000000e+02
+1.875040000000000e+03 +6.386920000000000e+02
+1.126730000000000e+03 +3.983660000000000e+02
+8.970450000000000e+02 +2.147350000000000e+02
+1.331490000000000e+03 +6.057630000000000e+02
+6.761139999999998e+02 +2.536810000000000e+02
+9.514600000000000e+02 +9.208230000000000e+02
+7.684510000000000e+02 +2.198970000000000e+02
+9.783760000000000e+02 +3.157110000000000e+02
+9.476210000000000e+02 +4.314240000000000e+02
+1.321000000000000e+03 +4.320560000000000e+02
+1.308710000000000e+03 +4.518240000000000e+02
+9.690110000000000e+02 +3.212370000000000e+02
+1.862610000000000e+03 +7.226400000000000e+02
+7.690850000000000e+02 +1.907240000000000e+02
+1.089790000000000e+03 +3.393700000000000e+02
+1.846800000000000e+03 +6.371759999999998e+02
+9.983230000000000e+02 +3.533720000000000e+02
+6.490640000000000e+02 +2.404870000000000e+02
+6.726389999999999e+02 +2.740690000000000e+02
+1.283460000000000e+03 +4.281090000000000e+02
+9.796280000000000e+02 +3.182750000000000e+02
+1.340630000000000e+03 +3.813820000000000e+02
+7.791419999999998e+02 +2.072340000000000e+02
+9.294420000000000e+02 +2.700070000000000e+02
+2.335980000000000e+03 +8.191050000000000e+02
+7.763720000000000e+02 +1.969510000000000e+02
+1.590060000000000e+03 +5.477790000000000e+02
+1.408370000000000e+03 +5.633520000000000e+02
+9.707110000000000e+02 +3.597480000000001e+02
+1.054800000000000e+03 +3.018110000000000e+02
+2.605300000000000e+02 +8.218489999999998e+02
+4.864470000000000e+02 +1.796090000000000e+02
+9.096380000000000e+02 +2.055330000000000e+02
+1.846660000000000e+03 +6.213049999999999e+02
+1.065540000000000e+03 +3.435580000000000e+02
+1.051100000000000e+03 +3.188880000000000e+02
+1.311820000000000e+03 +4.270870000000000e+02
+1.003340000000000e+03 +3.446030000000000e+02
+7.821189999999998e+02 +2.169390000000000e+02
+6.536900000000001e+02 +2.425820000000000e+02
+2.359510000000000e+03 +9.370309999999999e+02
+1.057460000000000e+03 +4.457300000000000e+02
+6.513410000000000e+02 +2.589080000000000e+02
+1.852630000000000e+03 +7.141600000000000e+02
+2.722880000000000e+02 +8.089540000000000e+02
+1.861550000000000e+03 +6.161120000000000e+02
+6.614330000000000e+02 +3.461570000000000e+02
+1.308010000000000e+03 +4.779130000000000e+02
+6.251120000000000e+02 +1.485540000000000e+02
+1.311600000000000e+03 +4.308130000000001e+02
+1.004200000000000e+03 +4.449160000000000e+02
+7.458800000000000e+02 +1.872590000000000e+02
+1.130830000000000e+03 +5.161270000000000e+02
+1.308450000000000e+03 +5.634930000000001e+02
+1.006730000000000e+03 +3.546610000000000e+02
+1.083930000000000e+03 +3.895140000000000e+02
+6.880939999999998e+02 +1.800730000000000e+02
+1.037360000000000e+03 +3.278940000000000e+02
+1.104100000000000e+03 +4.904980000000001e+02
+6.782530000000000e+02 +2.602670000000000e+02
+2.513040000000000e+02 +8.220410000000001e+02
+2.366290000000000e+03 +9.743900000000000e+02
+6.313009999999998e+02 +2.077530000000000e+02
+6.882170000000000e+02 +1.844790000000000e+02
+1.337650000000000e+03 +4.747100000000000e+02
+1.159280000000000e+03 +4.824400000000000e+02
+5.727900000000000e+02 +1.757890000000000e+02
+1.868090000000000e+03 +6.148090000000000e+02
+6.536830000000000e+02 +2.372510000000000e+02
+1.484340000000000e+03 +5.335369999999998e+02
+1.346140000000000e+03 +4.545020000000000e+02
+1.751490000000000e+03 +5.471920000000000e+02
+1.277630000000000e+03 +5.277450000000000e+02
+4.896330000000000e+02 +1.839460000000000e+02
+6.875330000000000e+02 +2.597350000000000e+02
+2.312500000000000e+03 +8.293230000000000e+02
+2.523460000000000e+02 +8.127220000000000e+02
+7.623500000000000e+02 +1.964000000000000e+02
+1.063970000000000e+03 +3.437030000000001e+02
+1.469460000000000e+03 +5.718670000000000e+02
+1.308310000000000e+03 +4.144850000000000e+02
+1.857400000000000e+03 +6.100210000000000e+02
+1.068510000000000e+03 +3.423910000000000e+02
+1.298660000000000e+03 +4.109030000000000e+02
+4.859700000000000e+02 +1.936450000000000e+02
+7.008680000000001e+02 +1.800560000000000e+02
+9.885900000000000e+02 +3.166230000000000e+02
+1.219640000000000e+03 +5.252859999999999e+02
+1.313000000000000e+03 +4.231200000000000e+02
+7.762100000000000e+02 +2.102100000000000e+02
+1.557650000000000e+03 +5.452669999999998e+02
+2.778350000000000e+02 +8.327289999999998e+02
+9.492140000000001e+02 +3.431880000000001e+02
+1.570870000000000e+03 +5.806230000000000e+02
+2.353210000000000e+03 +9.261940000000000e+02
+7.508240000000000e+02 +2.036590000000000e+02
+4.913500000000000e+02 +8.758969999999998e+02
+1.069030000000000e+03 +3.567410000000000e+02
+1.010190000000000e+03 +2.911080000000000e+02
+7.631610000000002e+02 +2.022630000000000e+02
+1.774670000000000e+03 +5.023860000000000e+02
+4.940560000000000e+02 +1.445830000000000e+02
+6.295010000000000e+02 +2.326770000000000e+02
+7.527300000000000e+02 +2.153270000000000e+02
+6.340219999999998e+02 +2.325430000000000e+02
+7.840300000000000e+02 +2.133490000000000e+02
+1.300860000000000e+03 +4.165710000000000e+02
+1.020390000000000e+03 +2.836920000000000e+02
+1.737210000000000e+03 +5.713610000000000e+02
+1.158220000000000e+03 +4.975230000000000e+02
+1.309160000000000e+03 +4.163250000000000e+02
+1.088470000000000e+03 +3.299590000000000e+02
+6.372060000000000e+02 +2.332710000000000e+02
+2.631070000000000e+03 +9.681450000000000e+02
+1.319580000000000e+03 +4.615460000000000e+02
+9.557850000000000e+02 +9.280359999999999e+02
+1.309850000000000e+03 +4.196250000000000e+02
+1.294470000000000e+03 +4.148020000000000e+02
+9.941609999999999e+02 +3.189220000000000e+02
+9.360630000000000e+02 +3.923830000000000e+02
+1.227580000000000e+03 +4.663430000000000e+02
+9.948330000000000e+02 +3.416230000000001e+02
+1.570060000000000e+03 +5.189740000000000e+02
+7.658639999999998e+02 +1.904970000000000e+02
+6.900039999999998e+02 +1.697960000000000e+02
+1.037520000000000e+03 +3.092090000000000e+02
+1.847270000000000e+03 +6.012950000000000e+02
+1.470490000000000e+03 +3.978470000000000e+02
+9.008620000000000e+02 +2.877440000000000e+02
+1.612110000000000e+03 +6.084580000000002e+02
+8.671430000000000e+02 +3.152810000000000e+02
+6.454320000000000e+02 +2.067370000000000e+02
+6.182809999999999e+02 +2.649630000000000e+02
+1.135130000000000e+02 +4.346400000000000e+01
+8.936519999999998e+02 +3.520350000000000e+02
+6.804019999999998e+02 +2.740530000000000e+02
+7.864060000000002e+02 +2.139770000000000e+02
+8.824570000000000e+02 +2.963310000000000e+02
+1.402120000000000e+03 +5.841720000000000e+02
+6.817170000000000e+02 +2.413270000000000e+02
+1.864830000000000e+03 +6.261440000000000e+02
+6.885540000000000e+02 +1.701730000000000e+02
+1.064660000000000e+03 +2.990960000000000e+02
+1.924940000000000e+03 +6.507980000000000e+02
+1.913800000000000e+03 +5.900470000000000e+02
+2.066380000000000e+03 +1.129810000000000e+03
+6.788910000000002e+02 +2.193510000000000e+02
+8.093889999999999e+02 +2.825320000000000e+02
+6.365750000000000e+02 +2.305700000000000e+02
+1.132360000000000e+03 +4.006520000000000e+02
+6.185010000000000e+02 +2.482670000000000e+02
+1.364020000000000e+03 +4.313730000000001e+02
+1.502090000000000e+03 +6.943330000000002e+02
+4.889190000000000e+02 +1.648890000000000e+02
+1.862410000000000e+03 +7.217860000000002e+02
+7.476840000000000e+02 +1.869660000000000e+02
+6.667850000000000e+02 +3.374960000000000e+02
+4.864290000000000e+02 +1.736980000000000e+02
+6.406510000000000e+02 +2.259190000000000e+02
+1.478140000000000e+03 +5.797190000000001e+02
+9.808390000000001e+02 +2.963290000000000e+02
+8.970670000000000e+02 +2.770640000000000e+02
+2.651520000000000e+03 +9.775950000000000e+02
+6.867660000000002e+02 +2.950070000000000e+02
+6.793040000000000e+02 +2.556900000000000e+02
+2.336300000000000e+03 +9.064140000000000e+02
+6.788439999999998e+02 +2.358830000000000e+02
+7.629620000000000e+02 +1.880990000000000e+02
+1.421190000000000e+03 +3.831150000000000e+02
+7.447760000000002e+02 +1.937210000000000e+02
+8.980400000000000e+02 +2.882910000000000e+02
+1.080780000000000e+03 +3.184110000000000e+02
+7.073489999999998e+02 +2.374440000000000e+02
+1.401420000000000e+03 +5.628740000000000e+02
+1.001490000000000e+03 +3.104650000000000e+02
+7.533819999999999e+02 +2.014610000000000e+02
+2.726200000000000e+02 +1.738840000000000e+02
+1.146710000000000e+03 +4.033190000000000e+02
+6.640269999999998e+02 +2.653810000000000e+02
+7.714620000000000e+02 +1.835290000000000e+02
+1.352740000000000e+03 +4.376830000000000e+02
+1.589940000000000e+03 +6.548140000000000e+02
+8.841540000000000e+02 +3.323510000000000e+02
+1.303180000000000e+03 +5.753130000000000e+02
+8.713450000000000e+02 +2.587720000000000e+02
+8.778889999999999e+02 +3.265020000000000e+02
+7.650580000000000e+02 +1.773130000000000e+02
+1.370060000000000e+03 +4.258760000000000e+02
+6.725180000000000e+02 +2.486400000000000e+02
+1.402330000000000e+03 +5.628650000000000e+02
+8.729510000000000e+02 +2.676770000000000e+02
+9.443240000000000e+02 +8.953439999999998e+02
+7.527800000000000e+02 +2.120720000000000e+02
+1.429070000000000e+03 +6.748989999999999e+02
+1.867390000000000e+03 +5.996250000000000e+02
+9.995280000000000e+02 +3.047410000000000e+02
+6.409059999999999e+02 +2.121450000000000e+02
+7.789989999999998e+02 +1.949730000000000e+02
+9.407150000000000e+02 +3.066100000000000e+02
+9.775260000000000e+02 +2.979000000000000e+02
+6.370630000000000e+02 +2.218870000000000e+02
+7.589420000000000e+02 +2.091330000000000e+02
+1.291860000000000e+03 +5.762100000000000e+02
+1.865570000000000e+03 +5.913350000000000e+02
+4.880160000000000e+02 +1.610560000000000e+02
+9.922450000000000e+02 +3.253000000000000e+02
+1.079690000000000e+03 +3.793460000000000e+02
+8.994639999999998e+02 +2.738590000000000e+02
+1.010900000000000e+03 +3.791330000000000e+02
+9.031060000000000e+02 +2.728180000000000e+02
+6.947030000000000e+02 +2.981120000000000e+02
+1.747900000000000e+03 +6.014770000000000e+02
+8.523600000000000e+02 +2.785530000000000e+02
+2.359710000000000e+03 +8.885050000000000e+02
+1.054600000000000e+03 +2.829160000000000e+02
+9.056500000000000e+02 +2.709200000000000e+02
+1.178740000000000e+03 +3.408590000000001e+02
+1.864550000000000e+03 +5.860290000000000e+02
+6.634250000000000e+02 +1.687050000000000e+02
+9.259770000000000e+02 +3.436740000000001e+02
+8.585670000000000e+02 +2.040920000000000e+02
+9.746860000000000e+02 +2.923650000000000e+02
+1.333550000000000e+03 +4.306280000000000e+02
+9.093210000000000e+02 +2.844540000000000e+02
+8.900020000000000e+02 +3.311670000000000e+02
+1.012040000000000e+03 +3.966510000000000e+02
+6.233500000000000e+02 +2.441160000000000e+02
+7.712010000000000e+02 +1.888020000000000e+02
+1.498560000000000e+03 +4.594950000000000e+02
+8.518250000000000e+02 +2.558450000000000e+02
+1.073500000000000e+03 +4.528370000000000e+02
+1.567260000000000e+03 +4.322460000000000e+02
+9.758620000000000e+02 +2.848190000000000e+02
+9.199880000000001e+02 +2.545430000000000e+02
+1.372590000000000e+03 +4.885470000000000e+02
+7.826060000000001e+02 +1.930220000000000e+02
+1.023740000000000e+03 +3.017730000000000e+02
+9.213690000000000e+02 +8.949970000000000e+02
+9.104800000000000e+02 +2.755940000000000e+02
+1.416280000000000e+03 +5.214059999999999e+02
+1.334160000000000e+03 +5.221090000000000e+02
+6.345920000000000e+02 +2.125520000000000e+02
+1.118250000000000e+03 +3.813960000000000e+02
+7.646480000000000e+02 +1.906670000000000e+02
+9.010200000000000e+02 +2.706300000000000e+02
+6.846619999999998e+02 +3.349990000000000e+02
+6.580400000000000e+02 +2.406710000000000e+02
+1.579590000000000e+03 +5.085980000000000e+02
+1.325520000000000e+03 +4.729860000000000e+02
+1.328920000000000e+03 +4.514370000000000e+02
+1.005230000000000e+03 +3.843100000000000e+02
+1.035940000000000e+03 +2.862950000000000e+02
+2.308840000000000e+03 +8.080610000000000e+02
+1.075150000000000e+03 +4.460930000000000e+02
+4.885090000000000e+02 +1.643750000000000e+02
+7.768130000000000e+02 +1.748440000000001e+02
+1.395640000000000e+03 +5.409770000000000e+02
+1.010970000000000e+03 +3.898960000000000e+02
+1.152590000000000e+03 +3.940440000000000e+02
+7.866480000000000e+02 +1.902080000000000e+02
+1.415770000000000e+03 +5.361559999999999e+02
+4.915810000000000e+02 +1.582900000000000e+02
+6.404740000000000e+02 +2.151280000000000e+02
+7.603680000000001e+02 +2.075690000000000e+02
+7.778380000000002e+02 +1.941210000000000e+02
+6.718480000000002e+02 +3.166530000000000e+02
+6.693720000000000e+02 +2.414310000000000e+02
+1.083920000000000e+03 +3.508570000000000e+02
+1.047120000000000e+03 +2.846440000000000e+02
+4.759490000000000e+02 +1.378180000000000e+02
+9.579780000000000e+02 +3.373430000000000e+02
+7.876750000000000e+02 +4.089920000000000e+02
+1.850980000000000e+03 +5.971440000000000e+02
+8.123710000000002e+02 +2.817790000000000e+02
+6.633489999999998e+02 +2.568720000000000e+02
+1.301070000000000e+03 +4.015620000000000e+02
+6.701810000000000e+02 +2.449100000000000e+02
+1.159650000000000e+03 +4.934160000000000e+02
+8.557760000000002e+02 +3.266530000000000e+02
+6.794019999999998e+02 +2.495980000000000e+02
+4.712450000000000e+02 +1.416840000000000e+02
+1.240080000000000e+03 +5.689920000000000e+02
+1.502600000000000e+03 +6.719530000000000e+02
+2.331090000000000e+03 +7.878290000000000e+02
+1.285760000000000e+03 +5.063130000000001e+02
+8.177990000000000e+02 +2.813620000000000e+02
+1.856020000000000e+03 +6.555910000000000e+02
+6.213270000000000e+02 +2.371150000000000e+02
+4.783400000000000e+02 +1.385120000000000e+02
+8.894210000000000e+02 +2.515080000000000e+02
+1.489490000000000e+03 +5.267619999999999e+02
+6.418790000000000e+02 +1.932220000000000e+02
+1.240520000000000e+03 +5.629590000000002e+02
+4.904700000000000e+02 +1.258770000000000e+02
+7.502430000000001e+02 +1.927710000000000e+02
+1.128850000000000e+03 +4.431200000000000e+02
+1.300690000000000e+03 +6.528630000000001e+02
+1.100230000000000e+03 +4.765310000000000e+02
+2.487260000000000e+02 +7.783200000000001e+02
+1.044020000000000e+03 +3.981140000000000e+02
+1.857940000000000e+03 +7.229850000000000e+02
+6.350230000000000e+02 +2.110290000000000e+02
+7.633099999999999e+02 +1.881950000000000e+02
+6.249880000000001e+02 +2.357900000000000e+02
+8.931239999999998e+02 +4.011730000000000e+02
+1.189650000000000e+03 +3.241290000000000e+02
+6.926669999999998e+02 +2.696980000000000e+02
+1.578920000000000e+03 +4.955010000000000e+02
+1.290190000000000e+03 +5.222040000000002e+02
+1.115010000000000e+03 +3.921220000000000e+02
+1.114960000000000e+03 +5.420440000000000e+02
+8.539900000000000e+02 +2.784870000000000e+02
+1.069660000000000e+03 +4.416090000000000e+02
+1.563440000000000e+03 +4.914130000000000e+02
+1.750210000000000e+03 +5.557450000000000e+02
+1.078640000000000e+03 +4.347860000000000e+02
+2.557130000000000e+02 +7.853290000000000e+02
+1.300870000000000e+03 +3.769460000000000e+02
+8.997160000000000e+02 +2.671880000000000e+02
+8.966810000000000e+02 +2.477210000000000e+02
+1.836470000000000e+03 +7.230060000000002e+02
+2.737550000000000e+02 +1.520790000000000e+02
+1.082480000000000e+03 +4.303730000000001e+02
+6.236730000000000e+02 +2.267210000000000e+02
+7.711289999999998e+02 +1.721660000000000e+02
+6.743170000000000e+02 +3.078690000000000e+02
+7.573520000000000e+02 +2.908820000000000e+02
+9.889320000000000e+02 +2.569080000000000e+02
+1.436610000000000e+03 +5.213790000000000e+02
+1.848410000000000e+03 +6.911740000000000e+02
+6.194620000000000e+02 +2.227540000000000e+02
+1.751760000000000e+03 +7.341640000000000e+02
+1.423420000000000e+03 +6.783580000000002e+02
+1.003970000000000e+03 +4.317030000000000e+02
+1.127060000000000e+03 +4.729310000000000e+02
+6.962020000000000e+02 +2.604350000000000e+02
+3.314580000000000e+02 +8.863700000000000e+01
+1.348450000000000e+03 +5.536090000000000e+02
+6.595630000000000e+02 +2.161600000000000e+02
+1.585940000000000e+03 +6.183610000000000e+02
+6.620810000000000e+02 +3.083700000000000e+02
+6.541960000000000e+02 +2.066750000000000e+02
+9.765309999999999e+02 +2.859760000000000e+02
+1.542300000000000e+03 +4.946160000000000e+02
+1.045600000000000e+03 +3.608640000000000e+02
+9.941750000000000e+02 +2.657130000000000e+02
+1.140840000000000e+03 +4.218650000000000e+02
+1.428310000000000e+03 +4.289320000000000e+02
+4.926440000000000e+02 +1.299840000000000e+02
+6.404150000000000e+02 +2.071210000000000e+02
+1.315690000000000e+03 +5.180610000000000e+02
+1.119370000000000e+03 +5.390650000000001e+02
+9.111380000000000e+02 +3.490530000000001e+02
+1.468690000000000e+03 +5.180690000000000e+02
+7.686120000000000e+02 +1.742340000000000e+02
+9.856920000000000e+02 +2.680790000000000e+02
+7.816270000000000e+02 +2.723280000000000e+02
+1.329390000000000e+03 +5.361569999999998e+02
+7.624750000000000e+02 +1.864730000000000e+02
+6.791560000000002e+02 +2.908470000000000e+02
+9.167790000000000e+02 +3.327150000000000e+02
+4.857490000000000e+02 +1.339470000000000e+02
+1.008460000000000e+03 +4.340380000000000e+02
+6.537020000000000e+02 +1.968320000000000e+02
+1.057420000000000e+03 +3.400740000000000e+02
+1.331670000000000e+03 +5.584460000000000e+02
+9.319020000000000e+02 +4.794780000000000e+02
+5.067460000000000e+02 +1.278150000000000e+02
+6.605839999999999e+02 +2.343590000000000e+02
+2.010580000000000e+03 +6.591710000000000e+02
+1.617780000000000e+03 +6.744540000000000e+02
+7.746720000000000e+02 +2.631600000000000e+02
+1.301610000000000e+03 +5.166860000000000e+02
+6.810210000000002e+02 +2.938180000000000e+02
+8.990500000000000e+02 +2.598140000000000e+02
+2.642860000000000e+02 +7.786369999999999e+02
+9.767030000000000e+02 +3.144750000000000e+02
+1.558400000000000e+03 +4.815050000000000e+02
+8.857310000000001e+02 +2.542310000000000e+02
+1.847040000000000e+03 +7.295939999999998e+02
+1.374910000000000e+03 +5.788070000000000e+02
+7.783819999999999e+02 +2.678240000000000e+02
+1.292770000000000e+03 +3.667480000000001e+02
+8.768800000000000e+02 +2.379800000000000e+02
+1.306070000000000e+03 +4.073200000000000e+02
+1.168290000000000e+03 +4.820990000000000e+02
+9.909610000000000e+02 +4.159260000000000e+02
+9.060830000000000e+02 +2.617780000000000e+02
+6.718310000000000e+02 +4.145750000000000e+02
+1.032230000000000e+03 +3.477750000000000e+02
+6.574720000000000e+02 +2.302840000000000e+02
+1.575010000000000e+03 +4.961260000000000e+02
+1.066830000000000e+03 +4.451020000000000e+02
+3.454010000000000e+02 +9.281860000000000e+01
+2.758740000000000e+02 +1.504060000000000e+02
+6.342180000000002e+02 +2.025970000000000e+02
+1.078810000000000e+03 +4.036680000000000e+02
+7.871480000000000e+02 +2.417090000000000e+02
+6.565239999999999e+02 +2.042250000000000e+02
+1.548140000000000e+03 +4.615190000000000e+02
+1.870990000000000e+03 +7.628030000000000e+02
+2.383640000000000e+02 +7.593720000000000e+02
+9.072130000000000e+02 +2.603210000000000e+02
+1.400770000000000e+03 +6.485169999999998e+02
+8.460219999999998e+02 +3.712520000000000e+02
+6.873610000000001e+02 +3.062100000000000e+02
+1.079840000000000e+03 +4.004890000000000e+02
+7.687790000000000e+02 +2.582670000000000e+02
+9.855930000000000e+02 +3.089490000000000e+02
+1.869570000000000e+03 +8.247110000000000e+02
+9.863740000000000e+02 +4.008160000000000e+02
+1.142240000000000e+03 +4.691130000000001e+02
+2.343850000000000e+03 +1.127530000000000e+03
+6.592880000000000e+02 +2.259620000000000e+02
+9.718690000000000e+02 +2.786310000000000e+02
+1.739660000000000e+03 +6.642100000000000e+02
+9.058440000000001e+02 +2.056730000000000e+02
+1.307190000000000e+03 +3.632280000000000e+02
+4.921650000000000e+02 +2.147110000000000e+02
+1.129160000000000e+03 +4.613930000000000e+02
+1.295810000000000e+03 +5.137840000000000e+02
+7.801660000000001e+02 +2.398970000000000e+02
+1.985380000000000e+03 +6.237420000000000e+02
+9.126310000000000e+02 +2.730250000000000e+02
+1.872910000000000e+03 +7.215470000000000e+02
+6.894650000000000e+02 +2.672840000000000e+02
+7.704960000000002e+02 +2.621310000000000e+02
+1.567530000000000e+03 +4.957920000000000e+02
+9.501730000000000e+02 +4.195760000000000e+02
+1.304780000000000e+03 +5.551830000000000e+02
+6.198830000000000e+02 +2.182700000000000e+02
+7.865480000000000e+02 +2.439750000000000e+02
+9.948630000000001e+02 +2.704510000000000e+02
+1.327350000000000e+03 +5.304680000000002e+02
+1.302190000000000e+03 +4.095880000000000e+02
+1.331270000000000e+03 +5.520940000000001e+02
+9.298600000000000e+02 +2.822830000000000e+02
+7.910039999999998e+02 +2.539420000000000e+02
+9.558530000000000e+02 +4.099210000000000e+02
+7.787930000000000e+02 +2.397520000000000e+02
+1.720230000000000e+03 +5.926469999999998e+02
+6.911080000000002e+02 +3.324190000000001e+02
+2.678710000000000e+03 +9.870430000000000e+02
+7.869180000000000e+02 +2.519250000000000e+02
+1.581940000000000e+03 +4.710950000000000e+02
+1.002240000000000e+03 +4.060520000000000e+02
+1.858720000000000e+03 +6.880650000000001e+02
+6.308350000000000e+02 +2.085880000000000e+02
+1.002480000000000e+03 +2.664520000000000e+02
+9.197619999999999e+02 +3.152540000000000e+02
+1.609050000000000e+03 +5.497970000000000e+02
+4.836950000000000e+02 +2.119520000000000e+02
+6.374430000000000e+02 +1.747180000000000e+02
+1.158370000000000e+03 +4.587480000000001e+02
+1.572840000000000e+03 +6.160990000000000e+02
+1.109500000000000e+03 +3.642600000000000e+02
+1.295450000000000e+03 +6.223160000000000e+02
+8.714050000000000e+02 +8.440430000000000e+02
+1.754730000000000e+03 +8.004610000000000e+02
+1.223680000000000e+03 +4.239220000000000e+02
+1.064160000000000e+03 +3.894670000000000e+02
+7.428020000000000e+02 +2.377640000000000e+02
+1.145970000000000e+03 +4.535080000000000e+02
+6.899930000000001e+02 +2.863220000000000e+02
+7.739610000000000e+02 +2.529020000000000e+02
+8.920740000000000e+02 +2.476710000000000e+02
+1.247170000000000e+03 +4.866680000000000e+02
+8.589240000000000e+02 +3.800640000000000e+02
+1.546190000000000e+03 +4.888090000000000e+02
+6.691230000000000e+02 +3.494920000000000e+02
+1.308370000000000e+03 +4.880380000000000e+02
+1.055770000000000e+03 +3.370550000000000e+02
+1.550790000000000e+03 +5.077180000000000e+02
+1.855040000000000e+03 +7.380500000000000e+02
+6.848190000000000e+02 +3.015490000000001e+02
+6.490590000000000e+02 +3.344970000000000e+02
+1.394060000000000e+03 +5.371040000000000e+02
+6.342780000000000e+02 +1.767820000000000e+02
+1.853670000000000e+03 +7.083020000000000e+02
+9.962300000000000e+02 +4.065700000000000e+02
+1.282220000000000e+03 +5.166090000000000e+02
+6.250790000000002e+02 +2.072170000000000e+02
+1.863020000000000e+03 +8.238290000000000e+02
+1.156610000000000e+03 +4.523740000000000e+02
+1.767210000000000e+03 +5.841750000000000e+02
+8.882030000000000e+02 +2.548270000000000e+02
+6.733140000000000e+02 +3.888070000000000e+02
+4.821910000000000e+02 +2.138330000000000e+02
+7.727970000000000e+02 +3.437020000000000e+02
+8.755119999999999e+02 +8.745690000000000e+02
+7.662819999999998e+02 +2.580450000000000e+02
+1.302110000000000e+03 +3.668060000000000e+02
+1.140040000000000e+03 +5.230210000000000e+02
+4.744540000000000e+02 +1.652370000000000e+02
+1.460170000000000e+03 +8.649310000000000e+02
+1.304000000000000e+03 +5.153420000000000e+02
+4.856240000000000e+02 +2.105100000000000e+02
+1.072860000000000e+03 +3.868550000000000e+02
+9.473460000000000e+02 +4.463110000000000e+02
+8.580560000000000e+02 +2.538910000000000e+02
+6.417430000000001e+02 +2.787420000000000e+02
+7.641880000000000e+02 +2.774600000000000e+02
+1.529990000000000e+03 +6.285260000000000e+02
+7.834580000000002e+02 +2.508230000000000e+02
+7.870570000000000e+02 +2.486160000000000e+02
+1.054010000000000e+03 +4.828870000000000e+02
+7.823919999999998e+02 +8.503789999999998e+02
+2.343770000000000e+03 +8.756990000000000e+02
+7.698839999999999e+02 +2.406300000000000e+02
+1.337430000000000e+03 +6.330850000000000e+02
+1.341110000000000e+03 +5.648750000000000e+02
+5.044190000000000e+02 +2.289310000000000e+02
+9.784970000000000e+02 +2.901850000000000e+02
+6.378680000000001e+02 +2.958500000000000e+02
+7.351760000000000e+02 +2.753370000000000e+02
+1.294770000000000e+03 +4.726940000000000e+02
+1.290030000000000e+03 +6.109299999999999e+02
+6.540520000000000e+02 +3.956350000000000e+02
+4.921630000000000e+02 +2.348950000000000e+02
+1.089440000000000e+03 +4.973630000000001e+02
+9.118110000000000e+02 +2.630150000000000e+02
+9.904180000000000e+02 +3.997400000000000e+02
+6.740210000000002e+02 +2.893660000000000e+02
+2.300310000000000e+02 +7.354349999999999e+02
+7.488110000000000e+02 +2.375390000000000e+02
+1.756150000000000e+03 +6.498250000000000e+02
+4.776780000000001e+02 +1.552500000000000e+02
+1.298900000000000e+03 +6.129270000000000e+02
+1.066960000000000e+02 +3.277230000000000e+01
+7.743090000000000e+02 +2.378340000000000e+02
+9.836130000000001e+02 +2.938610000000000e+02
+9.430870000000000e+02 +2.652160000000000e+02
+6.588739999999998e+02 +2.993500000000000e+02
+8.845410000000001e+02 +8.681960000000000e+02
+1.871110000000000e+03 +6.517140000000001e+02
+6.200419999999998e+02 +1.893420000000000e+02
+1.316380000000000e+03 +4.776220000000000e+02
+9.492590000000000e+02 +4.362450000000000e+02
+9.046210000000000e+02 +4.002850000000000e+02
+7.843730000000000e+02 +2.357600000000000e+02
+1.829190000000000e+03 +6.950169999999998e+02
+1.591560000000000e+03 +7.594080000000000e+02
+7.926870000000000e+02 +2.365070000000000e+02
+1.810230000000000e+03 +1.044170000000000e+03
+2.176260000000000e+02 +7.321369999999999e+02
+1.939680000000000e+03 +6.755670000000000e+02
+1.911930000000000e+03 +6.568580000000002e+02
+1.322270000000000e+03 +4.370260000000000e+02
+1.042180000000000e+03 +3.570080000000001e+02
+1.951580000000000e+03 +6.690350000000000e+02
+1.924320000000000e+03 +6.670540000000000e+02
+2.129750000000000e+03 +7.765610000000000e+02
+9.344230000000000e+02 +8.505219999999998e+02
+2.535750000000000e+02 +7.433980000000000e+02
+1.322590000000000e+03 +6.840910000000000e+02
+4.913140000000000e+02 +2.239760000000000e+02
+1.673960000000000e+03 +6.162900000000000e+02
+6.962769999999998e+02 +2.882200000000000e+02
+1.853480000000000e+03 +6.611849999999999e+02
+1.148560000000000e+03 +4.924650000000000e+02
+7.423310000000000e+02 +2.469460000000000e+02
+8.913860000000002e+02 +2.210780000000000e+02
+6.418690000000000e+02 +2.688820000000000e+02
+2.278390000000000e+02 +7.354870000000000e+02
+1.314490000000000e+03 +4.785020000000000e+02
+1.380490000000000e+03 +7.381080000000002e+02
+1.951830000000000e+03 +8.718750000000000e+02
+8.568800000000000e+02 +3.614070000000000e+02
+1.858200000000000e+03 +7.304160000000001e+02
+1.607310000000000e+03 +6.528620000000000e+02
+1.166740000000000e+03 +5.080890000000000e+02
+1.205220000000000e+03 +4.574880000000001e+02
+6.537600000000000e+02 +2.513410000000000e+02
+1.952100000000000e+03 +6.303020000000000e+02
+1.060180000000000e+03 +3.812360000000000e+02
+6.403840000000000e+02 +2.813770000000000e+02
+1.794320000000000e+03 +7.597170000000000e+02
+2.312590000000000e+03 +8.037880000000000e+02
+9.737940000000000e+02 +2.995560000000000e+02
+1.578740000000000e+03 +5.069970000000000e+02
+1.768270000000000e+03 +5.912959999999998e+02
+1.284530000000000e+03 +5.816060000000000e+02
+1.428140000000000e+03 +4.870890000000000e+02
+6.249600000000000e+02 +1.932140000000000e+02
+1.307050000000000e+03 +3.959970000000000e+02
+1.256600000000000e+03 +5.421540000000000e+02
+2.853860000000000e+02 +7.474520000000000e+02
+1.422650000000000e+03 +3.706400000000000e+02
+8.590520000000000e+02 +8.526480000000000e+02
+1.856990000000000e+03 +7.749000000000000e+02
+9.643740000000000e+02 +2.685250000000000e+02
+8.985210000000002e+02 +2.336960000000000e+02
+9.892100000000000e+02 +3.947330000000000e+02
+1.380590000000000e+03 +4.619040000000000e+02
+7.409420000000000e+02 +2.381390000000000e+02
+7.764500000000000e+02 +8.407250000000000e+02
+1.368330000000000e+03 +5.506550000000000e+02
+8.969950000000000e+02 +3.056820000000000e+02
+1.065000000000000e+03 +3.976350000000000e+02
+9.325380000000000e+02 +8.420530000000000e+02
+2.348740000000000e+03 +9.469260000000000e+02
+6.956280000000000e+02 +2.688280000000000e+02
+1.114120000000000e+03 +5.993880000000000e+02
+6.473020000000000e+02 +3.641620000000000e+02
+4.950630000000001e+02 +2.128400000000000e+02
+1.433040000000000e+03 +6.292640000000000e+02
+4.887560000000000e+02 +2.127340000000000e+02
+1.323240000000000e+03 +5.012220000000000e+02
+9.000920000000000e+02 +3.103500000000000e+02
+9.897859999999999e+02 +3.953430000000000e+02
+6.411130000000001e+02 +2.635350000000000e+02
+8.636780000000000e+02 +2.780940000000000e+02
+1.004580000000000e+03 +3.688700000000000e+02
+1.570040000000000e+03 +4.462370000000000e+02
+9.840050000000000e+02 +3.996340000000000e+02
+2.818940000000000e+02 +1.528720000000000e+02
+1.428800000000000e+03 +7.569270000000000e+02
+1.003040000000000e+03 +3.821070000000000e+02
+6.350160000000000e+02 +2.759730000000000e+02
+9.492809999999999e+02 +3.956630000000000e+02
+1.552820000000000e+03 +4.295430000000000e+02
+1.116810000000000e+03 +3.558850000000000e+02
+9.256910000000000e+02 +3.127370000000000e+02
+4.831970000000000e+02 +2.230230000000000e+02
+8.258900000000000e+02 +8.522270000000000e+02
+1.878480000000000e+03 +7.880780000000000e+02
+7.711820000000000e+02 +2.323760000000000e+02
+7.806500000000000e+02 +2.378420000000000e+02
+2.027390000000000e+02 +7.172630000000000e+02
+6.250409999999998e+02 +1.846210000000000e+02
+7.643830000000000e+02 +2.093350000000000e+02
+1.068750000000000e+03 +3.642150000000000e+02
+2.691470000000000e+03 +1.074080000000000e+03
+7.890219999999998e+02 +2.244440000000000e+02
+1.296830000000000e+03 +4.787630000000000e+02
+1.390050000000000e+03 +6.119240000000000e+02
+8.981940000000000e+02 +2.966160000000000e+02
+1.417380000000000e+03 +6.368770000000000e+02
+1.472380000000000e+03 +7.439150000000000e+02
+9.911720000000000e+02 +3.509770000000000e+02
+1.057360000000000e+03 +3.844700000000000e+02
+1.869260000000000e+03 +7.498439999999998e+02
+2.206570000000000e+02 +7.117160000000000e+02
+8.625380000000000e+02 +2.644910000000000e+02
+6.566200000000000e+02 +2.703940000000000e+02
+1.292250000000000e+03 +4.755770000000000e+02
+1.052980000000000e+03 +4.618770000000000e+02
+6.827280000000002e+02 +3.856100000000000e+02
+9.887809999999999e+02 +3.459570000000000e+02
+1.659120000000000e+03 +5.515520000000000e+02
+7.571039999999998e+02 +2.855260000000000e+02
+7.891080000000002e+02 +2.272390000000000e+02
+8.486860000000000e+02 +5.018490000000000e+02
+1.859430000000000e+03 +7.404400000000001e+02
+7.663350000000000e+02 +8.169069999999998e+02
+8.696790000000000e+02 +8.262320000000000e+02
+1.876780000000000e+03 +6.911480000000000e+02
+4.824150000000000e+02 +2.125900000000000e+02
+1.861310000000000e+03 +7.952180000000002e+02
+9.144220000000000e+02 +2.909350000000000e+02
+1.961400000000000e+03 +7.368939999999999e+02
+9.310480000000000e+02 +3.332040000000000e+02
+7.670730000000000e+02 +2.643130000000000e+02
+7.666900000000001e+02 +2.227330000000000e+02
+1.364010000000000e+03 +6.099290000000000e+02
+1.946930000000000e+03 +8.728270000000000e+02
+2.031840000000000e+03 +7.741740000000000e+02
+4.892110000000000e+02 +2.131190000000000e+02
+6.751780000000000e+02 +2.764220000000000e+02
+1.081430000000000e+02 +5.171640000000000e+01
+6.892160000000000e+02 +3.015370000000001e+02
+7.437689999999999e+02 +2.189950000000000e+02
+1.058410000000000e+03 +3.555630000000001e+02
+1.097680000000000e+02 +3.978400000000000e+01
+1.289860000000000e+03 +7.078980000000000e+02
+5.872760000000000e+02 +1.927770000000000e+02
+6.704030000000000e+02 +2.899600000000000e+02
+1.154970000000000e+03 +4.790230000000000e+02
+8.823630000000001e+02 +2.722960000000000e+02
+1.388970000000000e+03 +6.935350000000000e+02
+8.461439999999999e+02 +8.562250000000000e+02
+7.762270000000000e+02 +2.311020000000000e+02
+8.930590000000000e+02 +5.808110000000000e+02
+1.573580000000000e+03 +5.438450000000000e+02
+1.382260000000000e+03 +5.900720000000000e+02
+9.190430000000000e+02 +3.199500000000000e+02
+1.129110000000000e+03 +3.883240000000000e+02
+1.065020000000000e+03 +3.411369999999999e+02
+9.315470000000000e+02 +3.553470000000000e+02
+1.024910000000000e+03 +4.786070000000000e+02
+2.765450000000000e+02 +1.249070000000000e+02
+7.883420000000000e+02 +2.303650000000000e+02
+1.299680000000000e+03 +4.627880000000000e+02
+1.044610000000000e+03 +3.282090000000000e+02
+1.873970000000000e+03 +6.327080000000002e+02
+8.547719999999998e+02 +8.313140000000000e+02
+1.893110000000000e+02 +7.033160000000000e+02
+7.726110000000001e+02 +2.312100000000000e+02
+9.787160000000000e+02 +3.450510000000000e+02
+9.155860000000000e+02 +3.833090000000000e+02
+8.673130000000000e+02 +3.825550000000000e+02
+6.142250000000000e+02 +3.589120000000000e+02
+1.079810000000000e+03 +3.845980000000000e+02
+2.229350000000000e+02 +7.121000000000000e+02
+1.188650000000000e+03 +5.234590000000002e+02
+1.342160000000000e+03 +9.760940000000001e+02
+8.913220000000000e+02 +3.135730000000000e+02
+7.495210000000002e+02 +2.087500000000000e+02
+7.918680000000001e+02 +2.452320000000000e+02
+7.883739999999998e+02 +3.255460000000000e+02
+7.538910000000002e+02 +2.823450000000000e+02
+8.763950000000000e+02 +8.183780000000000e+02
+1.948270000000000e+02 +7.074460000000000e+02
+7.469810000000001e+02 +2.054960000000000e+02
+1.352520000000000e+03 +5.757410000000000e+02
+1.077270000000000e+03 +5.866270000000000e+02
+1.137360000000000e+03 +5.793940000000000e+02
+1.846960000000000e+03 +7.511300000000000e+02
+1.010650000000000e+03 +3.557920000000000e+02
+1.744390000000000e+03 +7.136160000000001e+02
+1.426280000000000e+03 +4.533070000000000e+02
+9.915160000000000e+02 +3.015940000000000e+02
+1.380020000000000e+03 +5.767550000000000e+02
+1.594960000000000e+03 +5.740030000000000e+02
+1.421890000000000e+03 +7.186750000000000e+02
+2.729040000000000e+02 +2.726530000000000e+02
+6.532170000000000e+02 +2.632350000000000e+02
+1.298420000000000e+03 +4.554720000000000e+02
+9.071810000000000e+02 +2.832510000000000e+02
+1.872110000000000e+03 +8.639090000000000e+02
+6.908910000000002e+02 +2.336170000000000e+02
+6.969939999999998e+02 +2.803740000000000e+02
+1.071380000000000e+03 +3.612690000000000e+02
+7.812890000000000e+02 +2.417370000000000e+02
+1.556920000000000e+03 +5.350130000000000e+02
+2.512230000000000e+02 +7.349460000000000e+02
+6.424590000000002e+02 +2.465210000000000e+02
+7.899480000000000e+02 +2.341170000000000e+02
+9.963650000000000e+02 +3.292160000000000e+02
+1.554410000000000e+03 +5.681080000000002e+02
+6.560269999999998e+02 +2.653370000000000e+02
+2.666070000000000e+03 +1.210960000000000e+03
+1.376510000000000e+03 +5.370770000000000e+02
+7.756039999999998e+02 +7.876030000000002e+02
+7.581680000000000e+02 +2.280670000000000e+02
+1.298950000000000e+03 +4.474000000000000e+02
+1.064640000000000e+03 +3.320690000000000e+02
+6.920139999999999e+02 +4.066900000000000e+02
+1.983900000000000e+03 +5.953840000000000e+02
+1.010940000000000e+03 +4.854130000000000e+02
+6.530840000000002e+02 +2.432770000000000e+02
+1.866310000000000e+03 +6.432850000000000e+02
+1.564710000000000e+03 +5.453750000000000e+02
+2.287250000000000e+03 +8.665310000000002e+02
+1.868530000000000e+03 +6.339970000000000e+02
+6.881770000000000e+02 +2.449830000000000e+02
+1.130730000000000e+03 +5.006410000000000e+02
+1.489400000000000e+03 +9.293360000000000e+02
+6.935360000000002e+02 +1.849490000000000e+02
+8.976020000000000e+02 +2.794120000000000e+02
+6.724889999999998e+02 +4.657480000000001e+02
+1.479120000000000e+03 +4.475940000000000e+02
+8.606489999999999e+02 +2.701880000000000e+02
+6.542080000000002e+02 +2.564550000000000e+02
+7.899720000000000e+02 +3.236140000000001e+02
+8.975570000000000e+02 +3.496290000000000e+02
+1.463360000000000e+03 +6.257260000000000e+02
+7.790950000000000e+02 +2.340810000000000e+02
+9.883260000000000e+02 +3.247250000000000e+02
+6.078450000000000e+02 +1.723460000000000e+02
+7.879860000000001e+02 +2.459900000000000e+02
+8.990930000000002e+02 +2.950360000000000e+02
+4.902260000000000e+02 +2.080960000000000e+02
+8.252089999999999e+02 +8.147819999999998e+02
+7.823660000000001e+02 +2.185980000000000e+02
+1.483570000000000e+03 +4.254990000000000e+02
+9.753140000000000e+02 +2.844840000000001e+02
+1.755180000000000e+03 +6.156330000000000e+02
+1.075070000000000e+03 +5.895730000000000e+02
+6.602139999999998e+02 +2.497970000000000e+02
+6.755839999999999e+02 +2.346020000000000e+02
+1.141150000000000e+03 +5.079760000000000e+02
+2.778890000000000e+02 +1.144250000000000e+02
+1.780310000000000e+03 +5.651640000000000e+02
+1.405830000000000e+03 +7.285210000000002e+02
+8.838439999999998e+02 +5.536690000000000e+02
+1.000090000000000e+03 +4.816980000000000e+02
+7.574180000000000e+02 +2.436940000000000e+02
+3.261000000000000e+02 +1.194720000000000e+02
+8.963900000000000e+02 +2.880720000000000e+02
+1.127560000000000e+03 +4.096090000000000e+02
+1.942590000000000e+02 +6.875470000000000e+02
+7.832210000000000e+02 +2.422460000000000e+02
+9.002040000000000e+02 +2.769700000000000e+02
+1.410900000000000e+03 +8.501419999999998e+02
+6.175730000000000e+02 +1.753130000000000e+02
+1.006620000000000e+03 +3.412760000000000e+02
+8.599650000000000e+02 +2.450820000000000e+02
+7.773660000000001e+02 +2.173740000000000e+02
+6.056180000000001e+02 +3.086790000000001e+02
+1.311920000000000e+03 +4.460870000000000e+02
+1.334140000000000e+03 +5.967750000000000e+02
+6.911810000000000e+02 +1.841650000000000e+02
+7.879900000000000e+02 +2.306450000000000e+02
+1.213960000000000e+03 +4.895290000000000e+02
+6.618450000000000e+02 +4.444160000000000e+02
+1.072660000000000e+03 +3.543130000000001e+02
+1.002330000000000e+03 +3.151940000000000e+02
+1.913290000000000e+02 +6.873680000000001e+02
+1.450500000000000e+03 +4.456320000000000e+02
+1.078610000000000e+03 +3.756260000000000e+02
+1.291910000000000e+03 +4.630580000000000e+02
+1.001780000000000e+03 +3.406500000000000e+02
+1.093440000000000e+03 +3.531100000000000e+02
+6.844349999999999e+02 +1.683830000000000e+02
+7.841410000000002e+02 +2.259860000000000e+02
+8.965889999999998e+02 +2.624850000000000e+02
+1.872180000000000e+03 +6.159360000000000e+02
+1.307770000000000e+03 +4.521540000000000e+02
+7.850630000000000e+02 +2.141070000000000e+02
+1.313300000000000e+03 +4.584140000000000e+02
+9.467650000000000e+02 +3.971860000000000e+02
+1.007320000000000e+03 +4.756810000000000e+02
+8.772089999999999e+02 +2.533820000000000e+02
+7.442189999999998e+02 +2.145760000000000e+02
+9.921350000000000e+02 +3.014170000000001e+02
+8.962300000000000e+02 +2.769990000000000e+02
+1.859540000000000e+03 +8.822100000000000e+02
+1.001310000000000e+03 +4.358300000000000e+02
+9.018740000000000e+02 +2.542910000000000e+02
+1.048480000000000e+03 +3.637920000000000e+02
+6.746720000000000e+02 +2.803470000000000e+02
+9.371600000000000e+02 +3.848470000000000e+02
+4.662390000000000e+02 +1.772780000000000e+02
+7.925200000000000e+02 +2.370160000000000e+02
+6.611170000000000e+02 +4.590640000000000e+02
+4.853040000000000e+02 +1.951100000000000e+02
+6.373770000000000e+02 +2.299510000000000e+02
+1.885600000000000e+02 +6.839360000000000e+02
+7.608350000000000e+02 +2.207650000000000e+02
+1.952480000000000e+03 +5.862040000000002e+02
+9.156340000000000e+02 +2.485580000000000e+02
+6.853190000000000e+02 +2.970840000000000e+02
+1.086950000000000e+03 +3.636280000000000e+02
+2.281470000000000e+03 +8.641150000000000e+02
+4.934820000000000e+02 +1.947640000000000e+02
+7.832919999999998e+02 +2.285500000000000e+02
+9.545400000000000e+02 +7.903330000000002e+02
+9.893950000000000e+02 +7.967900000000000e+02
+1.066960000000000e+03 +2.797680000000000e+02
+6.544180000000000e+02 +4.398780000000000e+02
+6.539480000000000e+02 +2.391220000000000e+02
+1.306100000000000e+03 +4.435770000000000e+02
+1.294790000000000e+03 +7.996030000000002e+02
+9.045300000000000e+02 +2.512880000000000e+02
+7.505480000000000e+02 +7.766410000000002e+02
+1.308810000000000e+03 +4.621490000000000e+02
+7.805260000000002e+02 +2.038160000000000e+02
+1.300990000000000e+03 +4.429650000000000e+02
+1.580690000000000e+03 +5.619950000000000e+02
+1.252120000000000e+03 +6.321690000000000e+02
+9.254770000000000e+02 +4.045170000000000e+02
+8.672769999999998e+02 +2.315660000000000e+02
+6.894530000000000e+02 +1.806300000000000e+02
+9.621150000000000e+02 +3.030520000000000e+02
+6.394670000000000e+02 +2.364450000000000e+02
+7.882840000000000e+02 +2.222210000000000e+02
+6.921740000000000e+02 +3.000380000000000e+02
+1.013510000000000e+03 +3.080370000000001e+02
+6.731550000000000e+02 +2.250310000000000e+02
+1.304060000000000e+03 +4.536430000000000e+02
+1.301640000000000e+03 +7.814100000000000e+02
+8.614320000000000e+02 +3.208160000000000e+02
+1.580840000000000e+03 +5.044120000000000e+02
+1.143150000000000e+03 +5.763740000000000e+02
+1.061780000000000e+03 +6.907130000000002e+02
+6.940010000000002e+02 +2.154890000000000e+02
+6.908420000000000e+02 +1.850780000000000e+02
+1.319990000000000e+03 +4.509660000000000e+02
+2.410700000000000e+03 +1.068640000000000e+03
+1.287430000000000e+03 +5.370570000000000e+02
+6.527390000000000e+02 +2.332720000000000e+02
+9.250010000000000e+02 +7.832500000000000e+02
+7.891890000000000e+02 +2.296190000000000e+02
+4.506030000000000e+02 +1.176510000000000e+02
+6.955450000000000e+02 +3.975970000000000e+02
+6.363670000000000e+02 +2.237600000000000e+02
+7.461790000000000e+02 +2.162700000000000e+02
+1.586420000000000e+03 +6.397980000000000e+02
+9.386340000000000e+02 +3.860030000000000e+02
+1.304170000000000e+03 +5.315260000000000e+02
+6.950300000000000e+02 +2.190950000000000e+02
+6.245580000000000e+02 +2.621230000000000e+02
+4.635030000000000e+02 +1.291950000000000e+02
+1.051470000000000e+03 +5.539670000000000e+02
+7.658160000000000e+02 +7.868869999999999e+02
+8.996139999999998e+02 +2.539980000000000e+02
+1.277450000000000e+03 +9.410890000000001e+02
+8.586239999999998e+02 +3.150630000000000e+02
+1.107190000000000e+03 +7.385189999999999e+02
+8.700640000000000e+02 +3.210110000000000e+02
+8.590989999999998e+02 +3.045750000000000e+02
+6.498650000000000e+02 +2.258670000000000e+02
+8.576950000000001e+02 +2.429730000000000e+02
+7.769540000000000e+02 +2.102030000000000e+02
+9.328770000000000e+02 +3.266450000000000e+02
+8.937970000000000e+02 +4.004490000000000e+02
+1.000580000000000e+03 +2.977760000000000e+02
+1.368860000000000e+03 +6.452700000000000e+02
+7.757460000000002e+02 +7.839610000000000e+02
+9.832859999999999e+02 +3.457050000000000e+02
+1.013910000000000e+03 +4.836570000000000e+02
+1.077250000000000e+03 +3.556230000000001e+02
+7.447589999999999e+02 +1.721550000000000e+02
+1.037960000000000e+03 +3.660420000000000e+02
+7.810250000000000e+02 +2.197170000000000e+02
+1.311070000000000e+03 +4.214450000000000e+02
+1.208980000000000e+03 +3.223410000000000e+02
+7.530700000000001e+02 +2.047550000000000e+02
+8.835650000000001e+02 +3.795940000000000e+02
+8.316070000000000e+02 +7.961030000000002e+02
+1.331370000000000e+03 +5.691100000000000e+02
+4.932870000000000e+02 +1.721160000000000e+02
+6.576750000000000e+02 +2.194470000000000e+02
+1.805470000000000e+02 +6.695169999999998e+02
+7.772919999999998e+02 +2.199420000000000e+02
+1.522610000000000e+03 +5.019370000000000e+02
+1.155910000000000e+03 +5.211490000000000e+02
+4.855390000000000e+02 +2.723880000000000e+02
+8.116210000000002e+02 +7.744150000000000e+02
+1.563710000000000e+03 +5.024850000000000e+02
+9.072220000000000e+02 +2.527260000000000e+02
+4.960260000000000e+02 +2.741750000000000e+02
+1.699170000000000e+03 +5.342190000000001e+02
+1.455740000000000e+03 +6.289150000000000e+02
+8.761360000000002e+02 +2.359620000000000e+02
+9.342500000000000e+02 +3.391720000000000e+02
+1.918550000000000e+03 +7.929900000000000e+02
+1.910920000000000e+03 +6.041110000000000e+02
+1.809360000000000e+03 +9.730069999999999e+02
+6.767330000000002e+02 +2.137530000000000e+02
+1.004730000000000e+03 +4.410980000000000e+02
+1.078030000000000e+03 +5.009100000000000e+02
+1.556710000000000e+03 +4.789930000000001e+02
+6.555160000000002e+02 +4.400370000000000e+02
+1.016710000000000e+03 +4.086500000000000e+02
+1.140280000000000e+03 +5.136630000000000e+02
+7.868730000000000e+02 +2.238040000000000e+02
+1.316830000000000e+03 +4.115270000000000e+02
+1.778060000000000e+03 +5.035670000000000e+02
+4.800520000000000e+02 +1.494430000000000e+02
+6.237869999999998e+02 +2.664760000000000e+02
+8.702700000000000e+02 +2.325610000000000e+02
+2.409430000000000e+03 +1.017710000000000e+03
+7.749280000000000e+02 +7.762530000000000e+02
+1.157790000000000e+03 +4.227450000000000e+02
+7.895239999999999e+02 +3.113100000000000e+02
+6.128580000000002e+02 +1.606290000000000e+02
+1.748120000000000e+02 +6.578080000000000e+02
+7.977339999999998e+02 +3.165650000000000e+02
+7.819019999999998e+02 +2.159220000000000e+02
+9.380780000000000e+02 +7.781730000000000e+02
+7.949130000000000e+02 +2.041790000000000e+02
+9.936510000000000e+02 +2.863520000000001e+02
+1.448840000000000e+03 +3.811070000000000e+02
+1.687940000000000e+03 +7.288589999999998e+02
+6.455440000000000e+02 +2.320190000000000e+02
+1.073310000000000e+03 +3.740450000000000e+02
+8.583170000000000e+02 +2.454350000000000e+02
+1.555700000000000e+03 +4.978240000000000e+02
+7.061920000000000e+02 +3.391240000000000e+02
+6.672660000000002e+02 +2.208800000000000e+02
+1.083160000000000e+03 +3.444560000000000e+02
+2.782850000000000e+02 +8.233880000000001e+01
+9.237490000000000e+02 +7.986139999999998e+02
+7.602089999999999e+02 +2.041940000000000e+02
+8.780860000000000e+02 +3.134510000000000e+02
+6.598989999999999e+02 +2.225040000000000e+02
+7.911750000000000e+02 +7.747489999999998e+02
+6.243000000000000e+02 +2.647400000000000e+02
+1.102400000000000e+03 +3.778250000000000e+02
+7.903700000000000e+02 +7.587270000000000e+02
+4.945090000000000e+02 +2.538280000000000e+02
+1.314010000000000e+03 +7.006799999999999e+02
+6.350040000000000e+02 +2.266120000000000e+02
+7.910770000000000e+02 +3.056200000000000e+02
+1.046220000000000e+03 +3.779840000000000e+02
+1.851130000000000e+03 +7.091319999999999e+02
+7.589169999999998e+02 +2.038030000000000e+02
+9.067430000000001e+02 +3.198460000000000e+02
+8.738510000000001e+02 +5.211940000000000e+02
+6.999980000000000e+02 +3.626170000000000e+02
+9.000450000000000e+02 +2.808880000000000e+02
+2.853980000000000e+02 +7.548630000000000e+01
+7.369260000000000e+02 +2.163040000000000e+02
+8.939689999999998e+02 +4.297020000000000e+02
+6.899240000000000e+02 +3.386110000000000e+02
+6.861330000000000e+02 +2.092140000000000e+02
+7.690780000000000e+02 +7.881160000000001e+02
+1.864200000000000e+03 +7.143090000000000e+02
+9.808400000000000e+02 +4.016060000000000e+02
+6.680100000000000e+02 +2.105620000000000e+02
+7.919720000000000e+02 +3.182860000000000e+02
+1.215620000000000e+03 +4.875680000000000e+02
+5.018070000000000e+02 +2.682360000000000e+02
+8.833810000000002e+02 +3.412690000000000e+02
+1.041330000000000e+02 +2.607630000000000e+01
+8.825889999999998e+02 +2.295710000000000e+02
+9.195330000000000e+02 +7.773589999999998e+02
+1.316490000000000e+03 +4.529280000000001e+02
+1.308690000000000e+03 +7.790360000000002e+02
+9.969400000000001e+02 +3.797040000000000e+02
+6.586550000000000e+02 +2.481470000000000e+02
+9.755660000000000e+02 +2.803890000000000e+02
+8.939760000000001e+02 +3.348680000000001e+02
+8.996740000000000e+02 +4.418860000000000e+02
+9.547770000000000e+02 +3.296510000000000e+02
+1.740960000000000e+02 +6.568220000000000e+02
+9.056750000000000e+02 +3.766360000000000e+02
+6.688839999999999e+02 +2.243020000000000e+02
+8.490989999999998e+02 +2.193320000000000e+02
+1.055850000000000e+03 +4.803210000000000e+02
+1.863350000000000e+03 +6.679639999999998e+02
+6.190180000000000e+02 +2.468070000000000e+02
+2.340170000000000e+02 +7.003339999999999e+02
+7.756310000000002e+02 +1.975070000000000e+02
+9.390560000000000e+02 +2.800310000000000e+02
+1.405000000000000e+03 +8.034560000000000e+02
+6.515130000000000e+02 +2.188890000000000e+02
+7.607230000000002e+02 +7.784580000000002e+02
+7.931519999999998e+02 +3.023910000000000e+02
+1.066370000000000e+03 +3.733390000000000e+02
+8.859800000000000e+02 +2.437720000000000e+02
+6.746230000000000e+02 +2.396720000000000e+02
+8.931200000000000e+02 +2.601780000000000e+02
+1.007290000000000e+03 +3.857400000000000e+02
+1.092690000000000e+03 +4.896060000000000e+02
+7.862840000000000e+02 +3.014130000000000e+02
+1.054750000000000e+03 +7.731110000000001e+02
+1.080940000000000e+03 +3.190160000000000e+02
+9.093390000000001e+02 +7.817660000000002e+02
+3.936290000000000e+02 +1.098210000000000e+02
+6.907700000000000e+02 +2.833320000000000e+02
+1.574350000000000e+03 +5.996500000000000e+02
+9.166910000000000e+02 +3.427980000000000e+02
+6.295130000000000e+02 +5.578150000000001e+02
+8.824040000000000e+02 +3.312800000000000e+02
+1.055050000000000e+03 +3.358340000000000e+02
+1.416780000000000e+03 +7.668190000000000e+02
+7.781310000000002e+02 +2.850470000000000e+02
+1.697710000000000e+02 +6.705770000000000e+02
+7.899390000000000e+02 +3.008070000000000e+02
+9.775170000000001e+02 +2.596400000000000e+02
+1.108700000000000e+03 +3.783350000000000e+02
+1.773650000000000e+02 +6.548819999999999e+02
+1.006810000000000e+03 +3.614300000000000e+02
+9.007610000000000e+02 +2.576000000000000e+02
+1.058510000000000e+03 +3.387600000000000e+02
+9.186780000000000e+02 +7.643049999999999e+02
+1.874650000000000e+03 +6.610169999999998e+02
+6.197650000000000e+02 +2.472130000000000e+02
+7.610830000000002e+02 +3.028190000000000e+02
+6.478750000000000e+02 +2.181970000000000e+02
+9.377550000000000e+02 +7.583939999999999e+02
+9.765760000000000e+02 +2.642340000000000e+02
+9.109430000000000e+02 +4.000150000000000e+02
+6.538960000000000e+02 +2.132710000000000e+02
+1.295600000000000e+03 +4.106690000000000e+02
+9.257380000000001e+02 +3.420930000000000e+02
+6.708869999999999e+02 +2.031430000000000e+02
+1.006160000000000e+03 +4.201410000000000e+02
+2.297830000000000e+03 +9.387730000000000e+02
+8.697669999999998e+02 +2.363860000000000e+02
+6.550880000000002e+02 +4.181260000000000e+02
+9.988070000000000e+02 +3.655260000000000e+02
+7.882510000000002e+02 +2.821610000000000e+02
+1.463860000000000e+03 +5.651550000000000e+02
+7.477210000000000e+02 +1.791440000000000e+02
+7.525820000000000e+02 +1.873570000000000e+02
+1.404380000000000e+03 +7.945650000000001e+02
+6.488850000000000e+02 +5.272060000000000e+02
+1.584330000000000e+03 +5.598260000000000e+02
+6.598730000000000e+02 +2.187990000000000e+02
+6.844750000000000e+02 +2.197920000000000e+02
+4.821550000000000e+02 +2.152020000000000e+02
+1.081750000000000e+03 +4.506050000000000e+02
+1.083050000000000e+03 +3.432340000000001e+02
+6.248099999999999e+02 +2.387930000000000e+02
+1.660940000000001e+02 +6.582869999999998e+02
+1.311140000000000e+03 +4.162180000000000e+02
+1.404320000000000e+03 +9.126240000000000e+02
+4.941960000000000e+02 +2.415650000000000e+02
+1.396000000000000e+03 +3.743820000000000e+02
+6.949380000000000e+02 +2.307890000000000e+02
+1.531050000000000e+02 +6.501950000000001e+02
+8.022530000000000e+02 +2.508830000000000e+02
+2.824030000000000e+02 +8.452430000000000e+01
+1.066500000000000e+03 +4.022330000000000e+02
+8.717810000000002e+02 +2.135640000000000e+02
+1.281710000000000e+03 +5.645450000000000e+02
+1.847380000000000e+03 +5.576290000000000e+02
+9.130370000000000e+02 +3.351790000000001e+02
+4.595590000000000e+02 +7.365730000000000e+02
+1.284930000000000e+03 +6.668989999999999e+02
+6.628120000000000e+02 +2.200770000000000e+02
+1.960140000000000e+03 +6.611849999999999e+02
+1.063990000000000e+03 +7.637580000000000e+02
+1.217760000000000e+03 +4.473460000000000e+02
+1.045450000000000e+03 +4.044640000000000e+02
+7.547790000000000e+02 +1.843170000000000e+02
+1.588790000000000e+03 +6.562530000000000e+02
+1.247180000000000e+03 +5.597390000000000e+02
+1.485350000000000e+03 +6.663860000000002e+02
+1.115710000000000e+03 +4.652970000000000e+02
+1.338120000000000e+03 +5.962270000000000e+02
+1.184070000000000e+03 +5.113120000000000e+02
+7.862460000000002e+02 +2.751940000000000e+02
+7.921890000000000e+02 +2.643880000000000e+02
+6.893450000000000e+02 +3.827170000000000e+02
+8.418270000000000e+02 +7.545670000000000e+02
+6.808510000000001e+02 +3.506240000000000e+02
+1.081100000000000e+03 +4.465690000000000e+02
+8.566039999999998e+02 +7.755710000000000e+02
+1.552740000000000e+02 +6.500820000000000e+02
+7.816799999999999e+02 +2.780780000000000e+02
+9.761110000000000e+02 +2.737360000000000e+02
+8.778900000000000e+02 +3.409320000000000e+02
+8.206350000000000e+02 +7.576910000000000e+02
+7.965230000000000e+02 +2.842580000000000e+02
+1.103780000000000e+03 +3.441360000000000e+02
+9.483920000000001e+02 +4.894830000000000e+02
+8.722260000000001e+02 +4.892100000000000e+02
+6.355459999999998e+02 +2.011250000000000e+02
+6.901130000000001e+02 +2.773430000000000e+02
+7.848800000000000e+02 +2.859290000000001e+02
+1.323290000000000e+03 +5.458810000000000e+02
+1.025060000000000e+03 +4.857200000000000e+02
+7.651030000000002e+02 +2.791940000000000e+02
+7.902360000000001e+02 +7.484360000000000e+02
+8.711080000000002e+02 +2.485990000000000e+02
+1.695870000000000e+03 +6.626150000000000e+02
+7.924989999999998e+02 +2.757160000000000e+02
+8.822569999999999e+02 +3.824760000000000e+02
+1.644070000000000e+03 +6.602669999999998e+02
+1.863390000000000e+03 +6.472950000000000e+02
+1.298440000000000e+03 +4.975670000000000e+02
+1.607190000000000e+03 +6.329180000000000e+02
+6.982439999999998e+02 +2.115160000000000e+02
+1.013390000000000e+03 +4.279280000000001e+02
+1.040350000000000e+03 +4.346610000000000e+02
+6.478390000000001e+02 +2.109150000000000e+02
+8.796750000000000e+02 +4.921460000000000e+02
+1.122000000000000e+03 +4.735160000000000e+02
+7.875150000000000e+02 +2.873410000000000e+02
+1.292180000000000e+03 +8.834460000000000e+02
+1.082230000000000e+03 +4.311410000000000e+02
+9.901330000000000e+02 +3.605240000000000e+02
+7.955110000000002e+02 +7.722460000000002e+02
+2.797730000000000e+02 +1.752640000000000e+02
+1.364190000000000e+03 +6.203690000000000e+02
+8.731489999999999e+02 +3.302070000000000e+02
+1.082940000000000e+03 +3.564710000000000e+02
+7.770520000000000e+02 +7.499340000000000e+02
+8.613389999999998e+02 +2.807270000000000e+02
+7.016960000000000e+02 +3.181240000000000e+02
+1.211050000000000e+03 +3.737810000000000e+02
+9.899600000000000e+02 +4.380490000000000e+02
+8.792410000000001e+02 +7.662970000000000e+02
+7.455889999999998e+02 +1.843410000000000e+02
+6.806799999999999e+02 +3.423360000000000e+02
+1.089550000000000e+03 +4.052600000000000e+02
+1.613290000000000e+03 +7.159989999999998e+02
+7.805960000000000e+02 +2.466500000000000e+02
+1.473860000000000e+03 +6.433190000000000e+02
+1.019910000000000e+03 +3.798720000000000e+02
+1.171510000000000e+03 +6.120650000000001e+02
+1.344600000000000e+03 +7.901970000000000e+02
+7.555350000000000e+02 +2.455750000000000e+02
+9.956319999999999e+02 +2.392540000000000e+02
+6.413810000000000e+02 +1.912520000000000e+02
+1.541690000000000e+02 +6.248150000000001e+02
+9.030549999999999e+02 +3.240630000000000e+02
+6.632689999999999e+02 +2.056180000000000e+02
+1.856410000000000e+03 +6.444620000000000e+02
+9.851420000000001e+02 +2.712100000000000e+02
+5.915369999999998e+02 +1.968890000000000e+02
+6.883439999999998e+02 +2.507580000000000e+02
+7.861319999999999e+02 +2.841500000000000e+02
+9.104610000000000e+02 +2.852600000000000e+02
+8.794180000000000e+02 +3.145350000000000e+02
+7.118539999999998e+02 +7.170930000000002e+02
+8.043739999999998e+02 +7.451280000000000e+02
+1.855000000000000e+03 +7.102200000000000e+02
+1.598670000000000e+03 +7.160080000000000e+02
+6.549470000000000e+02 +3.399860000000000e+02
+7.477339999999998e+02 +7.360750000000000e+02
+7.812960000000000e+02 +2.637840000000000e+02
+1.592860000000000e+03 +6.168590000000000e+02
+1.288520000000000e+03 +6.146890000000000e+02
+4.722520000000000e+02 +1.353730000000000e+02
+9.079520000000000e+02 +3.252690000000000e+02
+7.947739999999999e+02 +3.783240000000000e+02
+9.493440000000001e+02 +7.099650000000000e+02
+8.086820000000000e+02 +7.408260000000000e+02
+1.478000000000000e+02 +6.243280000000000e+02
+1.160050000000000e+03 +2.933420000000000e+02
+9.907960000000000e+02 +2.572790000000000e+02
+1.078240000000000e+03 +3.322290000000001e+02
+1.847870000000000e+03 +8.107200000000000e+02
+6.210740000000002e+02 +1.925770000000000e+02
+7.836860000000000e+02 +2.712140000000000e+02
+1.995830000000000e+03 +6.614580000000002e+02
+9.099950000000000e+02 +3.151440000000000e+02
+6.201870000000000e+02 +1.807090000000000e+02
+2.314230000000000e+03 +8.885490000000000e+02
+1.846820000000000e+03 +8.457810000000002e+02
+6.575080000000000e+02 +2.051030000000000e+02
+7.587980000000000e+02 +7.328040000000000e+02
+6.933980000000000e+02 +2.290750000000000e+02
+7.944370000000000e+02 +2.870400000000000e+02
+9.169310000000000e+02 +3.109880000000000e+02
+1.099920000000000e+03 +6.820989999999998e+02
+1.386840000000000e+03 +8.724349999999999e+02
+1.953390000000000e+03 +6.422100000000000e+02
+1.421030000000000e+03 +9.088740000000000e+02
+8.333730000000000e+02 +3.297700000000000e+02
+4.875220000000000e+02 +2.224330000000000e+02
+6.672280000000002e+02 +1.972300000000000e+02
+8.137160000000000e+02 +7.332680000000000e+02
+1.080200000000000e+02 +2.806250000000000e+01
+6.701189999999998e+02 +5.026520000000000e+02
+1.411320000000000e+03 +5.028860000000000e+02
+8.599680000000002e+02 +2.957380000000000e+02
+6.224420000000000e+02 +2.272300000000000e+02
+1.173250000000000e+02 +3.213760000000000e+01
+1.143810000000000e+03 +5.207880000000000e+02
+8.405150000000000e+02 +2.090280000000000e+02
+9.822140000000001e+02 +3.417360000000000e+02
+1.057450000000000e+03 +4.017190000000000e+02
+7.783510000000001e+02 +2.563160000000000e+02
+6.302950000000000e+02 +2.092680000000000e+02
+1.551530000000000e+03 +4.457950000000000e+02
+6.446930000000000e+02 +2.271260000000000e+02
+1.753160000000000e+03 +7.518989999999999e+02
+4.452200000000000e+02 +1.637810000000000e+02
+1.430010000000000e+03 +4.850100000000000e+02
+1.862790000000000e+03 +8.123860000000002e+02
+1.008580000000000e+03 +4.132550000000000e+02
+6.600880000000002e+02 +1.909270000000000e+02
+1.055500000000000e+03 +4.118810000000000e+02
+7.283760000000002e+02 +7.068660000000001e+02
+2.333310000000000e+02 +6.556060000000001e+02
+9.587940000000000e+02 +3.462310000000000e+02
+1.009370000000000e+03 +2.927840000000000e+02
+1.752730000000000e+03 +6.361469999999998e+02
+6.604490000000000e+02 +1.730330000000000e+02
+1.858040000000000e+03 +6.521900000000001e+02
+7.807400000000000e+02 +2.714080000000000e+02
+1.339080000000000e+03 +6.185710000000000e+02
+9.524850000000000e+02 +4.486590000000000e+02
+7.581460000000002e+02 +7.190960000000000e+02
+7.807339999999998e+02 +2.585820000000000e+02
+1.536240000000000e+03 +5.441799999999999e+02
+1.290190000000000e+03 +8.569660000000000e+02
+6.610300000000000e+02 +1.736690000000000e+02
+7.296630000000000e+02 +7.122589999999999e+02
+1.347220000000000e+03 +5.520549999999999e+02
+2.815090000000000e+02 +1.674610000000000e+02
+6.867420000000000e+02 +3.302380000000001e+02
+7.893480000000002e+02 +2.582000000000000e+02
+1.141340000000000e+03 +5.370500000000000e+02
+7.939939999999998e+02 +7.305020000000000e+02
+7.530660000000000e+02 +2.569500000000000e+02
+1.296850000000000e+03 +3.846320000000000e+02
+9.783780000000000e+02 +3.551470000000000e+02
+6.223230000000000e+02 +1.976360000000000e+02
+7.739920000000000e+02 +2.570250000000000e+02
+1.296930000000000e+03 +4.343760000000000e+02
+9.691090000000000e+02 +3.352860000000000e+02
+1.363580000000000e+02 +6.134240000000000e+02
+3.250260000000000e+02 +8.204219999999999e+01
+1.047330000000000e+03 +7.477710000000002e+02
+1.423610000000000e+03 +4.936480000000000e+02
+1.751280000000000e+03 +6.766640000000000e+02
+6.456180000000001e+02 +2.887420000000000e+02
+1.142340000000000e+03 +5.152630000000000e+02
+8.920970000000000e+02 +3.217470000000000e+02
+6.138090000000000e+02 +1.680830000000000e+02
+7.810230000000000e+02 +2.513070000000000e+02
+1.097180000000000e+03 +3.416880000000001e+02
+1.137120000000000e+03 +5.328680000000001e+02
+6.620880000000002e+02 +6.166910000000000e+02
+7.025810000000000e+02 +6.934860000000001e+02
+6.629450000000001e+02 +2.050010000000000e+02
+1.759090000000000e+03 +7.636790000000000e+02
+9.992820000000000e+02 +4.259540000000000e+02
+6.608070000000000e+02 +2.885410000000000e+02
+7.951540000000000e+02 +7.384349999999999e+02
+7.798080000000000e+02 +2.591680000000000e+02
+1.834840000000000e+03 +7.988170000000000e+02
+7.877360000000001e+02 +2.585880000000000e+02
+1.722570000000000e+03 +6.102440000000000e+02
+1.385640000000000e+03 +6.622960000000000e+02
+4.939470000000000e+02 +2.208030000000000e+02
+7.003860000000002e+02 +3.247330000000000e+02
+1.641360000000000e+03 +6.511720000000000e+02
+8.234520000000000e+02 +7.249839999999998e+02
+1.057430000000000e+01 +5.955240000000000e+02
+7.458980000000000e+02 +1.679940000000000e+02
+8.827339999999998e+02 +5.070410000000000e+02
+9.836200000000000e+02 +3.311460000000000e+02
+1.014500000000000e+03 +3.999980000000001e+02
+1.309480000000000e+03 +5.204159999999998e+02
+1.324520000000000e+03 +6.502300000000000e+02
+9.130030000000000e+02 +3.147940000000001e+02
+1.344440000000000e+03 +5.398250000000000e+02
+6.316170000000000e+02 +2.104170000000000e+02
+7.912170000000000e+02 +2.601970000000000e+02
+1.307020000000000e+03 +5.918730000000000e+02
+1.856920000000000e+03 +6.540459999999998e+02
+9.450910000000000e+02 +2.224370000000000e+02
+1.869440000000000e+03 +8.187000000000000e+02
+6.300890000000001e+02 +6.094150000000000e+02
+8.970829999999999e+01 +6.046590000000000e+02
+7.938410000000000e+02 +3.565660000000000e+02
+7.467739999999999e+02 +2.378610000000000e+02
+9.951799999999999e+02 +3.723490000000000e+02
+7.870599999999999e+02 +2.630960000000000e+02
+1.863910000000000e+03 +7.811540000000000e+02
+1.070630000000000e+03 +5.387150000000000e+02
+7.142580000000000e+02 +3.621560000000000e+02
+6.911590000000000e+02 +3.159720000000000e+02
+1.906070000000000e+02 +6.152970000000000e+02
+1.296660000000000e+03 +7.324160000000001e+02
+8.455050000000000e+02 +2.063210000000000e+02
+6.950580000000000e+02 +4.332370000000000e+02
+1.002760000000000e+03 +4.280240000000000e+02
+1.855460000000000e+03 +6.443969999999998e+02
+7.552810000000002e+02 +2.468200000000000e+02
+7.446469999999998e+02 +7.264080000000000e+02
+1.834930000000000e+02 +6.111980000000000e+02
+9.833380000000000e+02 +3.308400000000000e+02
+1.189730000000000e+03 +4.471760000000000e+02
+8.966220000000000e+02 +5.158720000000000e+02
+7.808760000000002e+02 +2.422050000000000e+02
+1.525420000000000e+03 +5.511590000000000e+02
+8.387230000000002e+02 +2.095060000000000e+02
+1.407330000000000e+03 +8.988910000000002e+02
+1.322060000000000e+03 +4.704220000000000e+02
+1.877420000000000e+03 +6.277800000000000e+02
+1.532320000000000e+03 +6.504090000000000e+02
+8.673939999999999e+02 +2.057070000000000e+02
+6.796940000000000e+02 +2.694230000000000e+02
+1.921080000000000e+03 +6.622880000000000e+02
+1.907810000000000e+03 +6.756920000000000e+02
+6.370119999999999e+02 +2.804420000000000e+02
+7.920430000000000e+02 +2.944940000000000e+02
+6.188660000000000e+02 +2.021200000000000e+02
+7.892930000000000e+02 +2.455150000000000e+02
+8.805360000000002e+02 +3.680430000000000e+02
+6.969019999999998e+02 +2.824140000000000e+02
+1.522840000000000e+03 +4.020480000000000e+02
+8.118510000000001e+02 +3.401470000000000e+02
+6.370840000000002e+02 +2.805090000000000e+02
+1.327540000000000e+03 +5.188840000000000e+02
+9.775860000000000e+02 +2.854640000000000e+02
+1.031970000000000e+03 +3.388680000000001e+02
+5.825850000000000e+02 +2.121830000000000e+02
+9.726830000000000e+02 +4.057260000000000e+02
+8.492030000000000e+02 +1.995250000000000e+02
+1.204480000000000e+03 +3.834260000000000e+02
+6.207619999999999e+02 +6.793960000000002e+02
+1.486230000000000e+02 +6.011790000000000e+02
+1.322980000000000e+03 +5.516030000000002e+02
+4.874680000000000e+02 +2.278700000000000e+02
+2.921790000000001e+02 +1.813170000000000e+02
+1.496970000000000e+03 +6.399130000000000e+02
+9.757130000000000e+02 +3.041490000000000e+02
+1.376310000000000e+03 +6.494640000000001e+02
+7.007010000000000e+02 +2.775260000000000e+02
+1.364930000000000e+02 +5.996030000000002e+02
+2.384680000000000e+03 +8.154240000000000e+02
+6.775219999999998e+02 +2.383480000000000e+02
+1.365580000000000e+03 +7.996840000000000e+02
+1.364730000000000e+03 +4.932370000000000e+02
+6.356010000000000e+02 +2.643660000000000e+02
+1.040850000000000e+03 +4.045230000000000e+02
+6.839520000000000e+02 +3.085330000000000e+02
+6.770440000000000e+02 +2.571630000000000e+02
+1.942340000000000e+02 +6.011550000000000e+02
+6.393400000000000e+02 +2.797180000000000e+02
+1.852740000000000e+03 +7.502669999999998e+02
+1.668470000000000e+03 +6.674290000000000e+02
+1.285150000000000e+03 +5.628980000000000e+02
+1.008840000000000e+03 +4.043720000000000e+02
+7.137589999999999e+02 +6.910430000000000e+02
+1.467750000000000e+02 +6.000710000000000e+02
+1.922450000000000e+03 +6.381270000000000e+02
+8.633160000000000e+02 +2.779030000000000e+02
+9.132360000000000e+02 +4.049340000000000e+02
+9.849290000000000e+02 +3.873000000000000e+02
+2.753600000000000e+02 +1.317840000000000e+02
+1.125590000000000e+03 +4.803430000000000e+02
+1.816460000000000e+03 +6.235459999999998e+02
+7.848789999999998e+02 +6.963620000000000e+02
+1.092240000000000e+02 +4.528350000000000e+01
+8.761760000000000e+02 +6.879520000000000e+02
+6.706660000000001e+02 +2.531920000000000e+02
+6.733670000000000e+02 +2.519130000000000e+02
+1.861270000000000e+03 +7.387930000000000e+02
+1.705490000000000e+02 +6.102900000000000e+02
+9.756220000000000e+02 +3.728900000000000e+02
+4.542820000000000e+02 +1.466740000000000e+02
+1.364280000000000e+03 +7.898300000000000e+02
+7.801419999999998e+02 +6.854800000000000e+02
+1.375910000000000e+02 +5.958310000000000e+02
+7.718770000000000e+02 +2.313130000000000e+02
+1.602360000000000e+03 +5.642120000000000e+02
+6.651980000000000e+02 +6.194920000000000e+02
+1.087220000000000e+03 +3.834490000000000e+02
+6.405319999999998e+02 +2.584510000000000e+02
+8.819270000000000e+02 +2.812570000000000e+02
+1.333890000000000e+03 +8.628130000000000e+02
+1.834990000000000e+03 +7.919169999999998e+02
+9.985880000000000e+02 +3.895180000000000e+02
+7.742990000000000e+02 +7.152539999999998e+02
+9.784480000000000e+02 +2.891420000000000e+02
+6.122010000000000e+02 +1.791650000000000e+02
+2.816580000000000e+02 +1.387830000000000e+02
+1.286620000000000e+03 +6.148620000000000e+02
+1.616280000000000e+03 +6.769260000000000e+02
+7.767089999999999e+02 +2.378660000000000e+02
+4.699100000000000e+02 +2.356650000000000e+02
+8.417400000000000e+02 +2.840790000000000e+02
+1.460740000000000e+03 +7.349280000000000e+02
+8.115039999999998e+02 +2.891130000000000e+02
+1.066400000000000e+03 +3.105830000000000e+02
+1.628200000000000e+03 +6.339600000000000e+02
+1.873670000000000e+03 +6.412430000000001e+02
+1.158880000000000e+02 +7.550060000000001e+01
+1.372450000000000e+03 +6.194010000000000e+02
+6.815989999999998e+02 +3.030610000000000e+02
+1.406780000000000e+03 +5.109780000000000e+02
+6.485800000000000e+02 +2.614430000000000e+02
+7.397900000000000e+02 +7.058330000000002e+02
+1.163340000000000e+02 +5.887320000000000e+02
+7.738000000000000e+02 +2.479110000000000e+02
+8.090160000000002e+02 +3.419170000000000e+02
+8.384310000000000e+02 +3.878830000000000e+02
+1.075360000000000e+03 +3.603120000000000e+02
+6.645630000000000e+02 +2.037450000000000e+02
+7.874180000000000e+02 +2.400640000000000e+02
+1.057910000000000e+03 +3.088780000000000e+02
+1.797020000000000e+03 +8.415960000000000e+02
+7.854119999999998e+02 +2.423780000000000e+02
+1.864040000000000e+03 +7.762930000000000e+02
+1.070270000000000e+03 +3.794580000000000e+02
+1.235290000000000e+02 +5.912700000000000e+02
+7.458589999999998e+02 +2.235790000000000e+02
+9.247340000000000e+02 +4.057090000000000e+02
+6.233900000000000e+02 +1.704400000000000e+02
+1.164400000000000e+03 +3.594400000000000e+02
+7.253339999999999e+02 +6.779360000000000e+02
+7.391619999999998e+02 +7.176770000000000e+02
+3.258570000000000e+02 +1.643320000000000e+02
+1.394630000000000e+03 +4.269210000000000e+02
+6.441910000000000e+02 +2.384700000000000e+02
+1.159600000000000e+02 +5.860740000000002e+02
+1.159860000000000e+03 +4.793570000000000e+02
+1.010890000000000e+03 +3.416090000000001e+02
+7.943830000000000e+02 +2.505150000000000e+02
+6.899260000000000e+02 +3.903450000000000e+02
+7.457339999999998e+02 +2.253440000000000e+02
+6.127640000000000e+02 +2.621480000000000e+02
+1.073200000000000e+03 +3.637680000000000e+02
+7.811619999999998e+02 +6.766070000000000e+02
+7.094490000000000e+02 +6.651369999999999e+02
+6.248360000000000e+02 +1.746630000000000e+02
+1.269050000000000e+02 +5.872490000000000e+02
+1.111200000000000e+03 +3.835090000000000e+02
+7.520110000000002e+02 +2.796080000000000e+02
+9.025390000000000e+02 +2.645370000000000e+02
+1.154900000000000e+03 +7.950520000000000e+02
+6.358700000000000e+02 +2.527550000000000e+02
+1.553850000000000e+03 +5.590620000000000e+02
+1.857540000000000e+03 +6.582030000000000e+02
+4.878150000000000e+02 +2.071550000000000e+02
+7.434440000000000e+02 +6.853910000000002e+02
+1.315850000000000e+03 +6.112970000000000e+02
+7.753150000000001e+02 +2.292670000000000e+02
+7.153750000000000e+02 +6.840810000000000e+02
+7.175310000000002e+02 +6.995650000000001e+02
+1.092400000000000e+02 +5.787850000000000e+02
+1.497790000000000e+03 +4.957850000000000e+02
+1.171590000000000e+03 +8.058800000000000e+02
+4.818390000000000e+02 +2.059560000000000e+02
+1.069500000000000e+03 +3.803790000000000e+02
+9.059230000000000e+02 +5.777890000000000e+02
+9.110970000000000e+02 +3.865400000000000e+02
+1.073160000000000e+03 +3.781050000000000e+02
+7.070700000000001e+02 +6.887220000000000e+02
+7.913480000000002e+02 +6.916050000000000e+02
+7.885380000000000e+02 +2.338510000000000e+02
+1.673200000000000e+03 +6.317040000000002e+02
+8.969150000000000e+02 +2.695520000000000e+02
+6.582330000000002e+02 +5.867809999999999e+02
+4.987620000000000e+02 +2.033400000000000e+02
+1.373820000000000e+03 +6.066160000000000e+02
+1.713750000000000e+03 +6.250540000000000e+02
+2.386740000000000e+03 +9.670540000000000e+02
+1.302920000000000e+03 +4.883070000000000e+02
+1.445480000000000e+03 +6.214930000000001e+02
+1.119840000000000e+03 +5.104130000000000e+02
+9.006590000000000e+02 +6.611250000000000e+02
+9.817080000000000e+02 +3.842240000000000e+02
+7.456230000000000e+02 +2.174430000000000e+02
+7.222489999999998e+02 +6.787289999999998e+02
+1.722860000000000e+03 +7.138810000000002e+02
+6.944100000000000e+02 +3.192650000000000e+02
+8.451189999999998e+02 +2.584110000000000e+02
+9.211100000000000e+02 +2.679770000000000e+02
+1.459300000000000e+03 +6.059220000000000e+02
+1.603440000000000e+03 +6.582550000000000e+02
+7.196220000000000e+02 +6.835570000000000e+02
+6.249180000000000e+02 +1.714750000000000e+02
+9.955870000000000e+01 +5.742340000000000e+02
+7.816389999999999e+02 +2.355290000000000e+02
+9.856590000000000e+02 +3.468190000000000e+02
+7.039750000000000e+02 +3.133850000000000e+02
+7.466519999999998e+02 +2.101820000000000e+02
+7.313439999999998e+02 +6.640050000000000e+02
+1.306310000000000e+03 +4.213330000000000e+02
+1.991750000000000e+03 +6.124690000000001e+02
+1.557610000000000e+03 +5.325380000000000e+02
+8.612270000000000e+02 +3.880530000000001e+02
+6.512450000000000e+02 +2.473060000000000e+02
+7.093220000000000e+02 +6.796380000000000e+02
+1.869610000000000e+03 +8.937339999999998e+02
+9.811920000000000e+02 +3.529040000000000e+02
+9.891040000000000e+01 +5.861740000000000e+02
+9.874250000000000e+02 +3.742890000000000e+02
+1.070350000000000e+03 +3.750190000000000e+02
+6.916189999999998e+02 +3.886170000000000e+02
+1.524980000000000e+03 +5.662880000000000e+02
+1.316230000000000e+03 +4.624580000000000e+02
+7.832880000000000e+02 +2.268940000000000e+02
+1.393470000000000e+03 +6.253220000000000e+02
+6.990230000000000e+02 +6.709939999999998e+02
+1.188830000000000e+02 +5.789000000000000e+02
+7.459860000000001e+02 +2.146390000000000e+02
+1.701660000000000e+03 +5.317880000000000e+02
+1.168140000000000e+03 +8.184860000000001e+02
+8.587410000000001e+02 +3.740330000000000e+02
+8.899770000000000e+02 +7.188570000000000e+02
+1.789040000000000e+03 +8.456400000000000e+02
+4.829530000000000e+02 +1.993400000000000e+02
+6.213530000000002e+02 +1.597650000000000e+02
+9.156910000000000e+02 +3.602010000000000e+02
+4.949270000000000e+02 +2.796290000000000e+02
+8.587030000000000e+02 +3.890520000000000e+02
+8.075150000000000e+02 +3.533790000000000e+02
+5.006640000000000e+02 +1.649980000000000e+02
+1.034200000000000e+03 +3.345230000000000e+02
+9.068160000000000e+02 +2.665190000000000e+02
+7.145920000000000e+02 +6.804030000000000e+02
+6.800770000000000e+02 +6.673240000000000e+02
+7.708580000000002e+02 +2.244440000000000e+02
+1.658100000000000e+03 +6.109040000000000e+02
+1.224100000000000e+03 +5.222859999999999e+02
+1.761340000000000e+03 +9.119640000000001e+02
+5.005240000000000e+02 +2.828940000000000e+02
+1.675300000000000e+03 +5.991140000000000e+02
+9.394650000000000e+02 +3.898370000000000e+02
+6.599290000000000e+02 +6.608739999999998e+02
+7.741920000000000e+02 +2.375440000000000e+02
+1.302580000000000e+03 +4.476410000000000e+02
+1.097670000000000e+03 +3.697210000000000e+02
+1.869230000000000e+03 +8.711339999999999e+02
+5.036340000000000e+02 +2.944200000000000e+02
+9.197960000000000e+01 +5.670950000000000e+02
+1.466680000000000e+03 +5.039380000000001e+02
+1.130210000000000e+03 +4.739300000000000e+02
+1.743070000000000e+03 +6.904010000000002e+02
+6.383670000000000e+02 +2.335200000000000e+02
+9.956460000000000e+02 +3.214520000000000e+02
+1.148010000000000e+03 +7.919860000000001e+02
+6.986450000000000e+02 +4.129980000000001e+02
+7.830510000000000e+02 +2.294030000000000e+02
+9.943660000000000e+02 +3.468190000000000e+02
+1.973340000000000e+03 +7.088860000000002e+02
+5.056770000000000e+02 +2.952820000000000e+02
+7.998560000000001e+02 +3.089480000000000e+02
+7.807200000000000e+02 +2.370380000000000e+02
+1.292280000000000e+03 +4.470560000000000e+02
+1.869790000000000e+03 +9.018630000000001e+02
+6.503240000000002e+02 +6.603539999999998e+02
+7.592600000000000e+02 +2.274030000000000e+02
+9.152320000000000e+02 +2.561150000000000e+02
+9.095890000000001e+02 +3.612860000000000e+02
+1.306710000000000e+03 +5.743450000000000e+02
+1.073940000000000e+03 +5.399390000000000e+02
+8.529920000000000e+02 +6.934660000000000e+02
+1.856280000000000e+03 +7.154030000000000e+02
+9.864780000000000e+02 +3.141330000000000e+02
+1.313720000000000e+03 +4.152120000000000e+02
+1.085030000000000e+03 +4.629080000000000e+02
+6.374230000000000e+02 +2.483880000000000e+02
+6.228500000000000e+02 +1.677480000000000e+02
+1.390050000000000e+03 +5.558540000000000e+02
+1.745180000000000e+03 +5.718650000000000e+02
+1.874400000000000e+03 +8.983280000000000e+02
+7.843530000000002e+02 +2.380810000000000e+02
+1.381660000000000e+03 +5.912850000000000e+02
+8.785590000000000e+02 +2.410080000000000e+02
+5.011920000000000e+02 +2.781680000000000e+02
+1.035190000000000e+03 +4.003260000000000e+02
+9.349930000000001e+01 +5.751180000000001e+02
+7.950250000000000e+02 +2.469050000000000e+02
+6.508040000000000e+02 +6.545319999999998e+02
+1.872860000000000e+03 +7.379190000000000e+02
+1.308570000000000e+03 +4.387300000000000e+02
+1.002620000000000e+03 +5.245580000000000e+02
+4.753100000000000e+02 +2.755610000000000e+02
+6.310620000000000e+02 +6.600790000000000e+02
+8.140009999999999e+01 +5.619299999999999e+02
+7.784530000000000e+02 +2.217910000000000e+02
+1.183730000000000e+03 +7.914550000000000e+02
+6.165500000000000e+02 +6.751540000000000e+02
+6.032310000000000e+02 +6.467669999999998e+02
+1.074110000000000e+03 +5.260169999999998e+02
+7.951330000000000e+02 +2.324360000000000e+02
+1.609390000000000e+03 +6.277190000000001e+02
+1.493480000000000e+03 +5.714370000000000e+02
+1.206430000000000e+03 +4.587910000000000e+02
+4.960080000000000e+02 +2.931460000000000e+02
+1.013790000000000e+03 +3.457420000000000e+02
+6.709639999999998e+02 +3.164630000000000e+02
+1.864860000000000e+03 +7.135219999999998e+02
+7.436510000000002e+02 +1.928210000000000e+02
+9.991840000000001e+01 +5.612730000000000e+02
+9.891900000000001e+02 +3.150690000000000e+02
+7.964650000000000e+02 +2.327010000000000e+02
+8.943200000000001e+02 +2.864890000000001e+02
+8.318439999999998e+02 +6.952739999999999e+02
+7.040710000000000e+02 +6.760630000000000e+02
+1.632420000000000e+03 +6.070850000000000e+02
+1.704760000000000e+03 +6.427420000000000e+02
+1.839300000000000e+03 +7.061300000000000e+02
+7.798400000000000e+02 +2.294460000000000e+02
+9.923370000000000e+02 +3.149750000000000e+02
+9.021180000000001e+02 +2.567110000000000e+02
+1.277850000000000e+03 +4.845230000000000e+02
+8.895360000000002e+02 +4.021040000000000e+02
+1.042460000000000e+03 +3.594750000000000e+02
+2.003500000000000e+03 +7.251389999999999e+02
+6.644000000000000e+02 +3.102250000000000e+02
+7.025000000000000e+02 +2.689770000000000e+02
+4.678090000000000e+02 +2.087010000000000e+02
+1.297370000000000e+03 +5.273160000000000e+02
+6.363220000000000e+02 +2.332280000000000e+02
+1.850560000000000e+03 +9.353900000000000e+02
+1.872500000000000e+03 +7.294830000000002e+02
+1.491480000000000e+03 +6.023640000000000e+02
+9.900340000000000e+02 +3.466170000000000e+02
+1.563970000000000e+03 +4.639470000000000e+02
+1.316870000000000e+03 +4.438810000000000e+02
+7.029230000000000e+02 +2.602460000000000e+02
+1.366050000000000e+03 +6.562970000000000e+02
+1.068020000000000e+03 +4.652330000000000e+02
+8.706110000000000e+01 +5.591240000000000e+02
+1.576650000000000e+03 +5.873130000000000e+02
+6.789050000000000e+02 +6.983040000000000e+02
+6.781880000000000e+02 +2.953960000000000e+02
+6.365690000000000e+02 +2.372230000000000e+02
+7.901369999999999e+02 +3.091810000000000e+02
+1.073690000000000e+03 +4.857590000000000e+02
+8.116380000000000e+02 +2.978730000000000e+02
+6.996260000000002e+02 +6.699610000000000e+02
+1.309990000000000e+03 +5.588180000000000e+02
+1.294270000000000e+03 +4.320620000000000e+02
+1.353700000000000e+03 +6.791600000000000e+02
+9.027320000000000e+02 +2.634500000000000e+02
+1.296870000000000e+03 +5.156080000000002e+02
+6.496110000000000e+02 +2.419870000000000e+02
+1.463310000000000e+00 +5.384320000000000e+02
+7.787589999999999e+02 +2.299450000000000e+02
+1.725930000000000e+03 +8.680300000000000e+02
+1.874920000000000e+03 +8.996960000000000e+02
+8.813610000000001e+02 +3.694440000000000e+02
+1.315410000000000e+03 +4.352490000000000e+02
+1.102070000000000e+03 +3.520330000000000e+02
+1.738030000000000e+03 +5.623730000000000e+02
+6.390180000000000e+02 +6.817930000000000e+02
+4.772200000000000e+02 +1.618420000000000e+02
+7.092060000000000e+02 +6.501590000000000e+02
+9.545510000000000e+02 +3.066450000000000e+02
+1.058420000000000e+03 +6.242970000000000e+02
+1.444150000000000e+03 +8.121039999999998e+02
+1.600620000000000e+03 +7.620130000000000e+02
+6.070970000000000e+02 +6.495160000000000e+02
+8.744210000000000e+02 +2.454440000000000e+02
+1.382600000000000e+03 +7.368460000000000e+02
+1.348450000000000e+03 +4.613730000000001e+02
+2.319180000000000e+03 +9.291140000000000e+02
+7.794299999999999e+02 +2.089890000000000e+02
+1.755460000000000e+03 +6.823339999999999e+02
+8.740419999999998e+02 +2.362260000000000e+02
+9.006780000000000e+02 +6.814710000000000e+02
+1.570940000000000e+03 +5.700319999999998e+02
+1.998620000000000e+03 +7.155430000000000e+02
+8.412320000000000e+02 +2.328770000000000e+02
+1.383570000000000e+03 +5.378270000000000e+02
+7.417580000000000e+02 +1.881430000000000e+02
+6.774480000000000e+02 +6.336530000000000e+02
+4.520880000000000e+01 +5.430409999999998e+02
+1.136890000000000e+03 +5.260610000000000e+02
+6.835130000000000e+02 +3.723810000000000e+02
+6.921280000000000e+02 +2.700350000000000e+02
+1.394660000000000e+03 +7.078739999999998e+02
+1.299980000000000e+03 +5.243390000000001e+02
+1.234610000000000e+03 +6.176810000000000e+02
+1.872360000000000e+03 +8.578980000000000e+02
+4.823500000000000e+02 +2.845230000000000e+02
+1.082600000000000e+03 +4.565950000000000e+02
+6.923530000000002e+02 +6.458450000000000e+02
+1.303590000000000e+03 +4.499690000000000e+02
+1.137690000000000e+03 +4.701960000000000e+02
+7.762410000000001e+02 +3.154690000000000e+02
+4.886330000000000e+02 +1.969510000000000e+02
+8.910050000000000e+02 +4.068540000000000e+02
+1.581020000000000e+03 +4.792230000000000e+02
+2.011490000000000e+03 +6.954980000000000e+02
+9.032140000000001e+02 +2.572250000000000e+02
+9.795020000000000e+02 +3.061460000000000e+02
+1.288210000000000e+03 +5.813550000000000e+02
+7.418589999999998e+02 +6.702030000000000e+02
+1.863360000000000e+03 +8.601400000000000e+02
+6.230169999999998e+02 +2.547710000000000e+02
+1.073790000000000e+03 +6.903099999999999e+02
+9.663500000000000e+02 +3.553290000000000e+02
+7.906350000000000e+02 +3.031000000000000e+02
+5.945300000000000e+02 +6.717000000000000e+02
+7.720570000000000e+02 +2.225300000000000e+02
+9.235580000000000e+02 +3.905010000000000e+02
+1.665360000000000e+03 +6.695239999999999e+02
+1.085660000000000e+03 +4.743460000000000e+02
+4.923280000000000e+02 +2.825710000000000e+02
+2.986780000000000e+02 +1.670620000000000e+02
+1.037030000000000e+03 +4.587190000000000e+02
+6.208830000000000e+02 +2.451020000000000e+02
+1.382990000000000e+03 +6.502250000000000e+02
+9.127630000000000e+02 +3.404600000000000e+02
+1.593830000000000e+03 +5.996849999999999e+02
+1.079060000000000e+03 +4.851210000000000e+02
+8.288480000000002e+02 +7.389689999999998e+02
+1.080590000000000e+03 +2.894870000000000e+02
+6.251350000000000e+02 +2.628840000000000e+02
+1.308050000000000e+03 +3.893940000000000e+02
+1.015460000000000e+03 +4.866840000000000e+02
+6.399610000000000e+02 +2.391670000000000e+02
+6.873180000000000e+02 +6.478320000000000e+02
+7.485180000000000e+01 +5.394299999999999e+02
+7.627320000000000e+02 +3.214040000000000e+02
+6.888339999999999e+02 +3.421910000000000e+02
+6.381410000000000e+02 +2.324340000000000e+02
+6.713260000000000e+02 +6.412420000000000e+02
+7.785100000000000e+02 +2.121440000000000e+02
+1.375540000000000e+03 +5.953819999999999e+02
+1.139690000000000e+03 +6.413550000000000e+02
+1.299480000000000e+03 +4.363100000000000e+02
+1.339630000000000e+03 +4.434910000000000e+02
+1.379760000000000e+03 +4.441730000000000e+02
+1.817900000000000e+03 +8.439939999999998e+02
+7.620780000000001e+01 +5.424040000000000e+02
+1.936860000000000e+03 +7.691849999999999e+02
+1.911930000000000e+03 +6.757600000000000e+02
+1.957040000000000e+03 +7.398410000000000e+02
+1.928000000000000e+03 +6.694810000000001e+02
+1.322320000000000e+03 +4.759810000000000e+02
+1.039620000000000e+03 +3.542610000000000e+02
+1.863730000000000e+03 +8.549310000000000e+02
+6.747180000000002e+02 +1.959560000000000e+02
+1.988100000000000e+03 +7.335740000000000e+02
+1.946130000000000e+03 +6.849550000000000e+02
+8.642919999999998e+02 +7.102610000000002e+02
+8.144000000000000e+02 +2.547950000000000e+02
+6.942610000000001e+01 +5.455850000000000e+02
+7.799349999999999e+02 +3.136620000000001e+02
+8.971750000000000e+02 +2.536790000000000e+02
+9.059410000000000e+02 +4.027790000000000e+02
+1.015400000000000e+03 +3.127560000000000e+02
+7.076130000000001e+02 +3.496130000000001e+02
+1.681040000000000e+03 +6.648240000000000e+02
+1.606720000000000e+03 +7.362170000000000e+02
+6.556089999999998e+02 +3.712320000000000e+02
+1.657790000000000e+03 +5.807480000000000e+02
+6.520650000000001e+02 +2.259390000000000e+02
+1.852910000000000e+03 +6.947739999999999e+02
+9.609299999999999e+02 +2.828820000000000e+02
+9.083990000000000e+02 +2.453730000000000e+02
+1.343500000000000e+03 +4.636180000000001e+02
+1.693170000000000e+03 +6.857520000000000e+02
+1.295980000000000e+03 +6.545450000000000e+02
+1.088280000000000e+03 +3.493270000000000e+02
+2.745690000000000e+02 +8.428789999999999e+01
+6.185010000000000e+02 +2.135060000000000e+02
+5.430800000000000e+02 +6.170150000000000e+02
+6.774589999999999e+02 +6.214180000000000e+02
+1.269500000000000e+03 +5.395530000000000e+02
+8.253370000000000e+02 +6.828889999999999e+02
+6.674960000000002e+02 +6.201390000000000e+02
+1.410890000000000e+03 +7.778600000000000e+02
+1.236880000000000e+03 +4.858600000000000e+02
+1.065740000000000e+03 +3.468620000000000e+02
+1.300630000000000e+03 +3.833500000000000e+02
+7.946840000000000e+02 +3.281280000000001e+02
+8.036810000000000e+02 +2.661750000000000e+02
+1.088860000000000e+03 +4.420710000000000e+02
+6.839780000000000e+01 +5.298220000000000e+02
+1.584410000000000e+03 +5.876330000000000e+02
+8.918610000000000e+01 +5.330490000000000e+02
+8.787739999999999e+02 +6.375280000000000e+02
+9.378790000000000e+02 +3.338070000000000e+02
+9.799550000000000e+02 +2.932600000000000e+02
+9.231910000000000e+02 +3.169890000000001e+02
+1.395340000000000e+03 +5.188000000000000e+02
+6.249980000000000e+02 +2.469490000000000e+02
+5.396290000000000e+01 +5.354270000000000e+02
+7.652150000000000e+02 +3.068080000000000e+02
+9.199550000000000e+02 +4.560880000000000e+02
+7.816610000000002e+02 +3.092370000000000e+02
+1.134170000000000e+03 +4.692190000000000e+02
+5.877800000000000e+02 +6.284980000000000e+02
+6.031630000000000e+02 +6.158600000000000e+02
+9.190490000000000e+02 +3.132620000000000e+02
+1.029250000000000e+03 +3.294900000000000e+02
+1.869130000000000e+03 +8.356389999999999e+02
+2.817010000000000e+02 +9.953460000000000e+01
+7.015390000000000e+02 +3.379550000000000e+02
+6.577869999999998e+02 +6.190240000000000e+02
+9.259990000000000e+02 +4.593670000000000e+02
+1.478630000000000e+03 +6.784210000000000e+02
+1.844360000000000e+03 +6.434660000000000e+02
+1.673650000000000e+03 +7.350069999999999e+02
+6.827869999999998e+02 +6.436180000000001e+02
+5.581150000000000e+02 +6.950180000000000e+02
+1.079690000000000e+03 +4.262310000000000e+02
+4.951210000000000e+02 +3.469540000000000e+02
+4.958500000000000e+02 +1.371900000000000e+02
+8.385880000000002e+02 +2.123670000000000e+02
+9.750700000000001e+02 +2.896660000000000e+02
+1.017230000000000e+03 +2.805940000000000e+02
+5.710910000000000e+02 +6.472640000000000e+02
+1.331260000000000e+03 +5.327130000000002e+02
+6.227220000000000e+02 +2.550310000000000e+02
+1.012970000000000e+03 +4.704710000000000e+02
+2.127360000000000e+02 +5.515319999999998e+02
+6.796890000000000e+02 +6.255630000000000e+02
+8.886920000000000e+02 +3.149940000000000e+02
+1.050850000000000e+03 +4.458700000000000e+02
+2.812460000000000e+02 +9.804600000000001e+01
+7.734360000000000e+02 +2.849830000000000e+02
+1.032840000000000e+03 +3.740400000000000e+02
+9.757280000000000e+02 +5.524970000000000e+02
+7.958750000000000e+02 +3.663720000000000e+02
+6.157390000000000e+02 +1.973690000000000e+02
+1.090860000000000e+03 +3.343620000000000e+02
+7.158110000000000e+02 +6.377970000000000e+02
+6.311970000000000e+02 +6.636500000000000e+02
+1.081010000000000e+03 +4.579090000000000e+02
+7.793510000000001e+02 +6.572760000000002e+02
+6.609989999999998e+02 +2.540030000000000e+02
+7.852780000000000e+02 +3.004690000000000e+02
+6.212640000000000e+02 +2.133610000000000e+02
+1.295100000000000e+03 +3.859590000000000e+02
+1.140840000000000e+03 +4.299090000000000e+02
+9.305170000000001e+02 +3.793440000000000e+02
+1.475450000000000e+03 +5.467869999999998e+02
+1.262600000000000e+03 +5.920560000000000e+02
+1.326640000000000e+03 +5.291280000000000e+02
+5.633070000000000e+01 +5.183550000000000e+02
+1.367060000000000e+03 +6.620089999999999e+02
+9.105860000000000e+02 +3.004870000000000e+02
+1.070000000000000e+03 +4.659810000000000e+02
+6.591870000000000e+02 +1.941120000000000e+02
+9.466020000000000e+02 +3.704120000000000e+02
+6.702950000000000e+02 +2.004010000000000e+02
+8.253700000000000e+02 +6.848560000000001e+02
+1.863510000000000e+03 +8.088730000000000e+02
+1.292930000000000e+03 +3.898160000000000e+02
+1.024290000000000e+03 +4.209650000000000e+02
+1.354500000000000e+03 +7.043020000000000e+02
+6.916439999999999e+02 +3.396750000000000e+02
+1.632540000000000e+03 +6.949639999999998e+02
+8.915720000000000e+02 +3.073920000000000e+02
+6.430830000000002e+02 +5.400540000000000e+02
+1.315840000000000e+03 +5.125280000000000e+02
+8.273630000000000e+01 +5.245580000000000e+02
+1.873890000000000e+03 +8.420010000000002e+02
+8.565720000000000e+01 +5.259040000000000e+02
+1.952480000000000e+03 +6.655920000000000e+02
+9.069890000000000e+02 +3.101950000000000e+02
+8.678260000000000e+02 +2.923910000000000e+02
+4.868270000000000e+02 +2.591120000000000e+02
+1.307620000000000e+03 +5.005970000000000e+02
+8.199750000000000e+02 +2.685020000000000e+02
+1.571050000000000e+03 +4.774860000000000e+02
+1.671220000000000e+03 +7.271330000000000e+02
+9.061150000000000e+02 +3.144120000000001e+02
+1.012540000000000e+03 +7.246469999999998e+02
+1.120600000000000e+03 +7.240100000000000e+02
+1.404070000000000e+03 +4.929800000000000e+02
+6.404720000000000e+02 +2.280540000000000e+02
+5.368960000000000e+02 +6.252220000000000e+02
+1.166440000000000e+03 +3.983620000000000e+02
+7.947200000000000e+02 +3.617110000000000e+02
+5.473230000000000e+02 +6.119980000000000e+02
+1.371240000000000e+03 +7.761510000000002e+02
+9.082850000000000e+02 +3.060010000000000e+02
+6.695939999999998e+02 +2.055450000000000e+02
+8.803670000000000e+02 +2.819670000000000e+02
+6.204000000000000e+01 +5.193170000000000e+02
+1.140600000000000e+03 +4.428450000000000e+02
+9.263720000000000e+02 +4.104720000000000e+02
+1.177090000000000e+03 +7.316300000000000e+02
+5.360050000000000e+02 +6.114059999999999e+02
+7.928040000000000e+01 +5.237700000000000e+02
+2.391990000000000e+01 +5.071270000000000e+02
+9.745760000000000e+02 +2.879610000000000e+02
+1.006750000000000e+03 +4.423070000000000e+02
+6.498860000000000e+02 +1.993570000000000e+02
+9.382060000000000e+02 +4.378880000000000e+02
+7.719390000000000e+02 +6.745139999999999e+02
+1.000620000000000e+03 +5.627919999999998e+02
+8.707930000000000e+02 +3.478119999999999e+02
+1.016310000000000e+03 +2.887010000000000e+02
+1.882760000000000e+03 +7.843099999999999e+02
+6.591139999999998e+02 +2.384130000000000e+02
+4.924690000000000e+01 +5.164240000000000e+02
+9.943500000000000e+02 +2.941860000000000e+02
+6.595790000000000e+02 +3.333560000000000e+02
+6.584870000000000e+02 +3.184610000000000e+02
+1.590200000000000e+03 +5.015380000000000e+02
+1.598830000000000e+03 +6.059280000000000e+02
+2.781780000000000e+02 +7.550020000000001e+01
+5.196490000000000e+02 +6.004520000000000e+02
+7.693550000000000e+02 +2.852930000000000e+02
+1.165260000000000e+03 +7.493270000000000e+02
+1.492050000000000e+03 +6.504820000000000e+02
+1.061870000000000e+03 +4.523400000000000e+02
+7.684760000000001e+02 +2.848330000000000e+02
+6.714240000000000e+02 +6.565660000000000e+02
+1.278810000000000e+03 +5.076840000000000e+02
+9.789660000000000e+02 +2.815870000000000e+02
+1.603920000000000e+03 +7.382030000000000e+02
+7.039780000000002e+02 +6.206290000000000e+02
+1.873010000000000e+03 +7.433389999999998e+02
+7.672930000000000e+02 +6.705690000000000e+02
+7.508819999999999e+02 +1.668040000000000e+02
+8.558049999999999e+02 +6.644530000000000e+02
+6.619880000000001e+02 +2.361310000000000e+02
+7.818480000000002e+02 +2.738390000000000e+02
+1.133490000000000e+03 +4.349910000000000e+02
+1.467760000000000e+03 +7.635770000000000e+02
+1.306570000000000e+03 +4.556340000000000e+02
+6.398940000000000e+02 +6.067900000000000e+02
+1.314020000000000e+03 +4.992030000000000e+02
+6.367560000000000e+02 +2.380980000000000e+02
+1.119100000000000e+03 +7.014870000000000e+02
+1.075940000000000e+03 +4.496310000000000e+02
+5.804660000000000e+02 +6.012240000000000e+02
+6.152390000000000e+02 +2.376820000000000e+02
+1.303020000000000e+03 +3.789860000000000e+02
+1.372340000000000e+03 +4.450040000000000e+02
+9.905290000000000e+02 +2.961140000000001e+02
+1.724410000000000e+03 +6.581770000000000e+02
+1.064890000000000e+03 +4.625550000000000e+02
+6.336910000000000e+01 +5.133200000000001e+02
+7.903560000000001e+02 +2.799360000000000e+02
+1.948250000000000e+03 +7.580549999999999e+02
+1.144840000000000e+03 +7.306870000000000e+02
+6.424540000000002e+02 +2.349520000000000e+02
+8.902020000000000e+02 +5.241740000000000e+02
+8.069390000000000e+02 +3.690620000000000e+02
+6.336120000000000e+02 +2.017710000000000e+02
+1.861130000000000e+03 +8.256610000000002e+02
+2.331010000000000e+03 +8.455660000000000e+02
+2.794370000000000e+02 +1.729500000000000e+02
+1.862890000000000e+03 +7.679989999999998e+02
+8.877220000000000e+02 +2.777380000000000e+02
+1.608140000000000e+03 +5.666750000000000e+02
+1.481410000000000e+03 +5.449119999999998e+02
+6.396380000000000e+02 +6.192840000000000e+02
+1.867870000000000e+03 +7.572040000000000e+02
+2.818490000000000e+02 +1.777390000000000e+02
+6.761560000000002e+02 +3.088200000000000e+02
+7.713280000000000e+02 +2.707170000000000e+02
+7.913650000000000e+02 +3.593210000000000e+02
+1.301330000000000e+03 +3.711250000000000e+02
+9.004430000000000e+02 +2.987580000000000e+02
+1.163610000000000e+03 +4.467770000000000e+02
+4.946500000000000e+02 +3.351520000000000e+02
+1.485300000000000e+03 +5.348530000000002e+02
+6.554000000000000e+02 +2.093440000000000e+02
+1.053620000000000e+03 +4.191750000000000e+02
+1.551610000000000e+03 +5.499320000000000e+02
+8.891350000000000e+02 +6.072470000000000e+02
+1.065430000000000e+03 +4.363050000000000e+02
+6.317930000000000e+02 +2.408830000000000e+02
+5.819730000000000e+01 +5.062420000000000e+02
+1.303600000000000e+03 +3.756190000000000e+02
+1.173270000000000e+03 +7.293450000000000e+02
+6.147010000000000e+02 +1.841610000000000e+02
+7.042580000000000e+02 +2.985320000000000e+02
+7.783049999999999e+02 +2.717980000000000e+02
+4.858830000000000e+02 +1.542380000000000e+02
+2.005210000000000e+03 +6.340640000000000e+02
+1.007330000000000e+03 +4.221060000000000e+02
+9.844960000000000e+02 +2.674500000000000e+02
+1.085030000000000e+03 +3.078090000000000e+02
+1.494680000000000e+03 +5.134390000000000e+02
+2.790550000000000e+02 +1.673690000000000e+02
+6.343650000000000e+02 +1.899170000000000e+02
+6.796530000000000e+02 +6.201469999999998e+02
+1.035970000000000e+03 +3.125460000000000e+02
+1.306620000000000e+03 +5.180450000000000e+02
+4.284390000000000e+01 +5.024950000000000e+02
+6.688180000000000e+02 +5.900050000000000e+02
+7.899310000000000e+02 +2.747720000000000e+02
+9.956190000000000e+02 +3.094650000000000e+02
+8.936480000000000e+02 +6.428470000000000e+02
+4.902590000000000e+01 +4.977480000000001e+02
+1.549410000000000e+03 +5.318250000000000e+02
+8.997689999999999e+02 +2.781420000000000e+02
+6.371110000000000e+02 +1.856390000000000e+02
+7.753610000000001e+02 +2.699530000000000e+02
+1.651680000000000e+03 +6.269520000000000e+02
+6.776980000000000e+02 +3.561160000000000e+02
+5.047880000000000e+01 +5.025470000000000e+02
+1.992970000000000e+03 +6.567610000000002e+02
+1.158310000000000e+03 +4.222980000000000e+02
+1.673260000000000e+03 +7.038180000000000e+02
+1.368470000000000e+03 +7.628989999999999e+02
+9.142640000000000e+02 +3.382950000000000e+02
+1.011690000000000e+03 +4.369860000000000e+02
+9.710160000000000e+02 +4.490350000000000e+02
+1.043830000000000e+03 +4.243270000000000e+02
+7.807730000000000e+02 +2.746180000000000e+02
+1.131210000000000e+03 +7.087070000000000e+02
+1.210530000000000e+03 +3.937060000000000e+02
+4.846010000000000e+02 +3.263620000000000e+02
+6.735189999999999e+02 +5.918040000000000e+02
+8.023090000000000e+02 +6.490700000000001e+02
+7.884980000000000e+02 +2.632670000000000e+02
+1.763900000000000e+03 +6.193410000000000e+02
+1.021330000000000e+03 +2.569830000000000e+02
+4.257450000000000e+01 +4.960640000000000e+02
+7.514130000000000e+02 +2.713590000000000e+02
+1.682510000000000e+03 +5.940830000000002e+02
+1.162190000000000e+03 +6.703889999999999e+02
+1.082540000000000e+03 +4.539870000000000e+02
+1.097680000000000e+03 +4.240440000000000e+02
+9.939850000000000e+02 +3.090700000000000e+02
+6.143290000000002e+02 +1.550350000000000e+02
+1.526170000000000e+03 +5.446780000000000e+02
+6.371240000000000e+02 +1.730320000000000e+02
+1.855040000000000e+03 +7.813660000000001e+02
+1.315710000000000e+03 +3.672210000000000e+02
+9.900630000000000e+02 +2.864660000000000e+02
+1.344140000000000e+03 +7.547260000000001e+02
+8.784030000000000e+02 +3.426860000000000e+02
+1.189840000000000e+03 +3.559420000000000e+02
+1.646820000000000e+03 +5.870200000000000e+02
+1.221590000000000e+03 +4.173630000000001e+02
+1.110700000000000e+03 +4.123480000000000e+02
+9.218370000000000e+02 +2.790990000000000e+02
+8.688240000000000e+02 +4.283440000000000e+02
+2.868760000000000e+02 +1.764460000000000e+02
+6.405910000000000e+02 +5.754830000000002e+02
+7.810460000000000e+02 +2.585610000000000e+02
+6.514740000000000e+02 +1.764160000000000e+02
+6.769190000000000e+02 +2.954050000000000e+02
+3.640710000000000e+01 +4.913940000000000e+02
+1.099480000000000e+03 +4.293820000000000e+02
+6.712430000000001e+02 +1.747460000000000e+02
+1.852100000000000e+03 +7.677760000000002e+02
+1.337090000000000e+03 +7.496389999999999e+02
+8.839370000000000e+02 +5.941100000000000e+02
+1.050750000000000e+03 +4.254470000000000e+02
+7.503150000000001e+02 +2.685080000000000e+02
+6.742160000000000e+02 +3.025240000000000e+02
+9.056300000000000e+02 +2.194410000000000e+02
+4.954180000000000e+02 +3.382670000000000e+02
+6.841760000000000e+02 +5.931540000000000e+02
+4.134860000000000e+02 +5.690390000000000e+02
+1.308430000000000e+03 +4.261680000000000e+02
+1.605990000000000e+03 +5.359600000000000e+02
+8.741960000000000e+02 +6.290190000000000e+02
+1.025400000000000e+03 +2.465260000000000e+02
+1.085850000000000e+03 +4.054660000000000e+02
+6.333819999999999e+02 +1.736520000000000e+02
+7.765110000000002e+02 +2.651130000000000e+02
+1.318440000000000e+03 +3.700440000000000e+02
+1.329770000000000e+03 +6.550549999999999e+02
+1.234770000000000e+03 +6.902430000000001e+02
+4.164240000000000e+02 +5.644960000000000e+02
+9.074840000000000e+02 +3.335550000000000e+02
+1.119910000000000e+03 +6.914520000000000e+02
+1.074050000000000e+03 +4.378600000000000e+02
+6.637040000000000e+02 +1.697630000000000e+02
+4.098810000000000e+02 +5.568230000000000e+02
+8.941760000000000e+02 +2.216260000000000e+02
+1.855810000000000e+03 +7.889620000000000e+02
+7.798339999999999e+02 +2.520010000000000e+02
+9.213460000000000e+02 +4.288150000000000e+02
+6.668789999999998e+02 +2.855510000000000e+02
+9.660510000000000e+02 +2.932160000000000e+02
+6.195190000000000e+02 +2.121360000000000e+02
+6.559470000000000e+02 +6.257330000000002e+02
+9.325520000000000e+02 +3.033970000000000e+02
+1.424770000000000e+03 +4.798680000000001e+02
+7.672370000000000e+02 +6.491950000000001e+02
+6.468370000000000e+02 +1.857040000000000e+02
+9.997960000000000e+02 +3.975060000000000e+02
+8.742850000000000e+02 +3.940600000000000e+02
+1.787650000000000e+03 +6.810180000000000e+02
+7.775740000000000e+02 +2.606230000000000e+02
+9.846470000000000e+02 +2.960070000000000e+02
+1.302120000000000e+03 +7.397110000000000e+02
+3.611970000000000e+01 +4.800910000000000e+02
+7.932470000000000e+02 +2.649830000000000e+02
+6.785520000000000e+02 +3.150270000000000e+02
+7.728000000000000e+02 +6.135580000000000e+02
+1.552700000000000e+03 +5.099550000000000e+02
+1.140800000000000e+03 +7.005900000000000e+02
+1.673490000000000e+03 +6.481720000000000e+02
+6.491130000000001e+02 +3.008400000000000e+02
+6.827089999999999e+02 +6.209430000000000e+02
+6.228310000000000e+02 +2.119090000000000e+02
+4.392180000000000e+01 +4.834940000000000e+02
+7.778739999999998e+02 +2.398300000000000e+02
+9.718440000000001e+02 +4.187690000000000e+02
+3.966480000000000e+02 +5.370630000000000e+02
+1.575230000000000e+03 +6.447590000000000e+02
+6.213020000000000e+02 +2.025420000000000e+02
+6.452950000000000e+02 +2.760050000000000e+02
+5.158880000000000e+02 +5.513810000000000e+02
+1.338780000000000e+03 +7.538320000000000e+02
+9.019890000000000e+02 +2.808590000000000e+02
+1.237290000000000e+03 +5.195960000000000e+02
+2.814160000000000e+02 +1.956260000000000e+02
+8.992610000000002e+02 +2.744100000000000e+02
+6.466720000000000e+02 +6.010610000000000e+02
+1.669650000000000e+03 +8.367689999999999e+02
+5.054780000000000e+02 +5.935590000000000e+02
+1.004700000000000e+03 +4.151730000000000e+02
+9.137350000000000e+02 +3.975390000000000e+02
+1.072440000000000e+03 +4.084800000000000e+02
+6.367010000000000e+02 +2.714550000000000e+02
+4.259630000000000e+02 +5.514890000000000e+02
+1.865520000000000e+03 +7.657530000000000e+02
+7.740939999999998e+02 +2.373810000000000e+02
+1.071550000000000e+03 +3.952120000000000e+02
+6.488060000000000e+02 +6.167710000000000e+02
+1.488930000000000e+03 +5.464240000000000e+02
+8.687960000000000e+02 +4.191630000000000e+02
+6.004940000000000e+02 +5.972869999999998e+02
+8.654000000000000e+02 +2.601590000000000e+02
+9.736900000000001e+02 +3.540580000000000e+02
+9.019810000000000e+02 +2.645740000000000e+02
+5.018080000000000e+02 +5.883110000000000e+02
+4.935080000000000e+02 +3.260770000000000e+02
+1.092800000000000e+03 +4.077570000000000e+02
+8.770980000000002e+02 +2.736510000000000e+02
+1.088270000000000e+03 +4.092590000000000e+02
+7.001220000000000e+02 +3.978990000000000e+02
+3.485000000000000e+02 +5.300119999999999e+02
+5.240850000000000e+02 +5.524059999999999e+02
+1.056330000000000e+03 +4.408600000000000e+02
+1.707620000000000e+03 +8.480480000000000e+02
+6.636910000000000e+02 +1.971650000000000e+02
+7.727550000000000e+02 +2.377060000000000e+02
+7.814639999999998e+02 +2.530200000000000e+02
+1.375460000000000e+03 +7.367410000000001e+02
+1.453520000000000e+03 +7.162980000000000e+02
+3.345260000000000e+02 +1.785690000000000e+02
+1.771020000000000e+03 +7.157960000000000e+02
+1.098330000000000e+03 +6.402780000000000e+02
+8.926469999999998e+02 +3.223470000000000e+02
+9.048140000000000e+02 +2.730110000000000e+02
+1.023370000000000e+03 +6.511890000000000e+02
+6.951089999999998e+02 +4.116720000000000e+02
+2.778570000000000e+02 +1.576610000000000e+02
+5.207700000000000e+02 +5.519800000000000e+02
+7.878320000000000e+02 +2.458380000000000e+02
+1.012180000000000e+03 +6.681730000000000e+02
+6.361260000000000e+02 +2.951460000000000e+02
+9.023490000000000e+02 +2.997870000000001e+02
+1.610970000000000e+03 +5.179090000000000e+02
+1.759900000000000e+03 +7.418539999999998e+02
+6.354820000000000e+02 +5.672300000000000e+02
+5.606190000000000e+02 +5.789620000000000e+02
+9.911630000000000e+02 +3.441690000000001e+02
+9.094030000000000e+02 +3.007640000000000e+02
+9.367530000000000e+02 +4.003830000000000e+02
+7.580020000000000e+02 +6.311110000000000e+02
+3.454110000000000e+02 +5.394010000000000e+02
+2.994060000000000e+01 +4.714860000000000e+02
+4.862930000000000e+02 +3.172580000000000e+02
+6.406890000000000e+02 +2.856380000000000e+02
+7.402180000000002e+02 +6.198120000000000e+02
+5.000160000000000e+02 +5.569370000000000e+02
+8.874800000000000e+02 +6.466840000000000e+02
+1.077100000000000e+03 +4.670900000000000e+02
+4.816670000000000e+02 +5.776270000000000e+02
+5.584159999999998e+02 +5.865350000000000e+02
+1.288130000000000e+03 +4.822320000000000e+02
+7.726130000000001e+02 +2.684000000000000e+02
+1.292420000000000e+03 +4.754420000000000e+02
+9.286700000000000e+02 +3.833700000000000e+02
+4.905370000000000e+02 +5.796250000000000e+02
+8.058330000000002e+02 +6.240590000000000e+02
+1.043900000000000e+03 +3.782930000000000e+02
+1.227890000000000e+03 +7.141650000000000e+02
+1.008490000000000e+03 +5.150690000000000e+02
+7.117189999999998e+02 +2.911600000000000e+02
+2.743620000000000e+01 +4.672600000000000e+02
+9.600450000000000e+02 +6.622840000000000e+02
+1.174400000000000e+03 +5.495330000000000e+02
+8.957689999999999e+02 +6.230380000000000e+02
+1.018640000000000e+03 +4.801460000000000e+02
+1.304460000000000e+03 +4.646780000000001e+02
+5.291430000000000e+02 +5.900200000000000e+02
+4.283100000000000e+02 +5.479169999999998e+02
+5.297400000000000e+02 +5.511970000000000e+02
+1.002670000000000e+03 +5.183280000000000e+02
+8.110230000000000e+02 +3.585540000000001e+02
+1.572390000000000e+03 +4.527080000000000e+02
+1.556190000000000e+03 +6.942769999999998e+02
+1.411110000000000e+03 +7.652980000000000e+02
+6.437880000000000e+02 +2.201590000000000e+02
+4.814730000000000e+02 +5.608170000000000e+02
+7.690080000000000e+02 +2.362150000000000e+02
+1.304420000000000e+03 +4.748860000000000e+02
+9.748150000000001e+02 +6.426540000000000e+02
+7.708350000000000e+02 +6.029040000000000e+02
+9.933330000000000e+02 +4.084480000000000e+02
+3.538970000000000e+02 +5.366400000000000e+02
+4.825150000000000e+02 +5.438240000000002e+02
+3.468260000000000e+00 +4.602290000000000e+02
+9.859690000000001e+02 +3.514080000000000e+02
+5.507750000000000e+02 +5.982840000000000e+02
+2.820110000000000e+02 +1.433180000000000e+02
+5.235269999999998e+02 +5.509790000000000e+02
+7.737239999999998e+02 +2.357660000000000e+02
+1.778190000000000e+03 +5.718969999999998e+02
+1.382410000000000e+03 +5.511530000000000e+02
+4.257780000000000e+02 +5.556450000000000e+02
+6.439430000000000e+02 +2.730530000000000e+02
+7.035010000000002e+02 +3.942110000000000e+02
+1.716700000000000e+03 +8.345710000000000e+02
+5.176090000000000e+02 +5.803280000000000e+02
+6.579030000000000e+02 +2.748110000000000e+02
+7.899530000000000e+02 +2.574430000000000e+02
+1.066290000000000e+03 +3.752220000000000e+02
+1.880930000000000e+03 +7.409870000000000e+02
+7.455520000000000e+02 +2.305450000000000e+02
+4.872180000000000e+02 +3.025560000000000e+02
+1.079760000000000e+03 +4.290080000000000e+02
+4.373230000000000e+02 +5.427780000000000e+02
+4.230570000000000e+01 +4.638040000000000e+02
+7.777260000000001e+02 +2.507970000000000e+02
+9.133860000000000e+02 +3.701580000000000e+02
+1.116460000000000e+03 +6.615700000000001e+02
+9.941380000000000e+02 +3.959930000000001e+02
+7.467320000000000e+02 +2.467950000000000e+02
+6.570680000000000e+02 +2.695350000000000e+02
+3.154460000000000e+02 +5.244080000000000e+02
+4.961670000000000e+02 +3.235370000000001e+02
+7.454410000000000e+02 +6.010200000000000e+02
+6.967210000000000e+02 +2.820700000000000e+02
+7.805169999999998e+02 +2.394890000000000e+02
+3.358990000000000e+02 +1.537080000000000e+02
+1.145270000000000e+03 +3.625330000000000e+02
+7.006770000000000e+02 +3.876270000000000e+02
+6.669420000000000e+02 +2.567430000000000e+02
+5.840910000000000e+02 +5.482730000000000e+02
+6.190490000000000e+02 +1.882780000000000e+02
+6.259990000000000e+02 +5.772690000000000e+02
+1.775880000000000e+03 +6.748460000000000e+02
+9.891980000000000e+02 +6.715319999999998e+02
+2.607350000000000e+02 +7.693729999999999e+01
+9.033530000000000e+02 +6.177530000000000e+02
+1.054890000000000e+03 +4.018870000000000e+02
+6.247650000000000e+02 +2.735850000000000e+02
+6.591230000000000e+02 +5.865090000000000e+02
+6.305830000000002e+02 +2.012640000000000e+02
+4.693670000000000e+02 +5.356659999999998e+02
+1.863250000000000e+03 +7.477840000000000e+02
+1.563950000000000e+03 +5.931390000000000e+02
+4.899990000000000e+02 +3.154780000000000e+02
+7.041400000000000e+02 +3.843400000000000e+02
+1.300500000000000e+03 +4.771320000000000e+02
+9.181400000000000e+02 +2.508730000000000e+02
+9.975130000000000e+02 +3.406500000000000e+02
+1.985750000000000e+03 +7.340989999999998e+02
+6.088200000000001e+02 +5.993740000000000e+02
+4.566630000000000e+02 +5.409069999999998e+02
+1.072240000000000e+03 +4.145340000000000e+02
+1.502220000000000e+01 +4.546150000000000e+02
+7.728989999999999e+02 +2.430190000000000e+02
+9.110850000000000e+02 +5.151190000000000e+02
+7.447060000000000e+02 +6.162610000000000e+02
+8.740250000000000e+02 +3.894050000000000e+02
+3.424550000000000e+02 +5.128120000000000e+02
+9.894380000000000e+02 +3.385240000000000e+02
+1.228890000000000e+03 +4.881120000000000e+02
+7.471310000000002e+02 +6.152640000000000e+02
+7.032500000000000e+02 +3.934860000000000e+02
+6.636610000000002e+02 +1.824290000000000e+02
+2.978260000000000e+01 +4.549700000000000e+02
+9.947820000000000e+02 +3.323060000000000e+02
+1.015540000000000e+03 +6.832680000000000e+02
+4.566750000000000e+02 +1.261970000000000e+02
+1.009900000000000e+03 +4.195580000000000e+02
+1.376930000000000e+03 +5.225470000000000e+02
+1.028250000000000e+03 +6.676150000000000e+02
+8.819040000000000e+02 +3.887050000000000e+02
+4.936950000000000e+02 +5.490880000000002e+02
+1.683470000000000e+03 +6.898589999999998e+02
+1.230340000000000e+03 +5.245520000000000e+02
+7.087050000000000e+02 +5.196040000000000e+02
+1.754350000000000e+03 +8.034180000000000e+02
+1.019970000000000e+03 +5.101810000000000e+02
+2.905990000000000e+02 +5.133730000000000e+02
+8.927160000000000e+02 +2.516010000000000e+02
+9.367130000000000e+02 +5.204400000000001e+02
+2.730020000000000e+02 +5.081520000000000e+02
+2.126800000000000e+01 +4.505980000000000e+02
+7.711910000000000e+02 +2.333700000000000e+02
+1.594110000000000e+03 +5.343340000000002e+02
+6.208550000000000e+02 +5.611020000000000e+02
+7.683600000000000e+02 +2.657100000000000e+02
+1.015310000000000e+03 +6.683750000000000e+02
+7.045060000000002e+02 +2.967100000000000e+02
+2.577980000000000e+01 +4.499320000000000e+02
+3.101790000000000e+01 +4.536130000000001e+02
+4.703480000000000e+02 +5.670880000000002e+02
+2.822410000000000e+02 +1.430070000000000e+02
+6.443060000000000e+02 +2.562020000000000e+02
+3.233170000000000e+01 +4.518330000000000e+02
+1.184370000000000e+03 +5.809250000000000e+02
+4.794230000000000e+02 +1.325600000000000e+02
+7.145780000000000e+02 +3.961780000000001e+02
+8.068830000000000e+02 +3.402390000000001e+02
+4.237350000000000e+02 +5.506050000000000e+02
+1.859740000000000e+03 +7.600660000000000e+02
+7.781239999999998e+02 +2.281410000000000e+02
+6.388009999999998e+02 +2.490490000000000e+02
+8.754930000000001e+02 +2.747730000000000e+02
+6.249760000000000e+02 +1.880850000000000e+02
+7.460790000000000e+02 +2.407640000000000e+02
+3.803940000000000e+02 +5.192980000000000e+02
+1.300880000000000e+03 +4.410210000000000e+02
+5.461430000000000e+02 +5.598590000000000e+02
+6.232470000000000e+02 +1.825870000000000e+02
+1.223850000000000e+03 +5.334710000000000e+02
+4.034470000000000e+02 +5.388150000000001e+02
+4.899090000000000e+02 +3.921190000000000e+02
+7.660560000000000e+02 +2.126990000000000e+02
+9.859700000000000e+02 +3.347170000000000e+02
+1.022600000000000e+03 +5.145340000000000e+02
+7.118780000000000e+02 +6.050700000000001e+02
+8.233720000000000e+02 +3.098910000000000e+02
+5.648180000000000e+02 +5.481790000000000e+02
+7.817250000000000e+02 +2.225720000000000e+02
+4.228530000000000e+02 +1.162960000000000e+02
+6.959700000000000e+02 +5.988710000000000e+02
+1.009730000000000e+03 +4.733940000000000e+02
+6.571439999999999e+02 +3.779480000000000e+02
+6.284290000000000e+02 +2.044290000000000e+02
+1.337130000000000e+03 +6.598380000000002e+02
+5.035820000000000e+02 +3.949650000000000e+02
+6.380369999999998e+02 +2.626940000000000e+02
+6.951330000000000e+02 +2.936390000000000e+02
+1.321730000000000e+03 +6.766289999999998e+02
+5.106490000000000e+02 +4.083240000000000e+02
+7.823610000000001e+02 +2.963580000000000e+02
+4.964130000000000e+02 +5.263370000000000e+02
+6.595020000000000e+02 +1.701440000000000e+02
+8.015670000000000e+02 +5.898270000000000e+02
+5.508250000000000e+02 +5.749650000000000e+02
+1.499680000000000e+03 +5.758000000000000e+02
+2.783400000000000e+02 +1.229370000000000e+02
+4.763000000000000e+02 +5.294270000000000e+02
+7.849130000000000e+02 +2.348850000000000e+02
+8.998930000000000e+02 +2.492940000000000e+02
+1.004290000000000e+03 +6.458350000000000e+02
+4.824550000000000e+02 +2.982980000000000e+02
+1.353140000000000e+03 +5.844259999999998e+02
+1.338920000000000e+03 +6.870030000000000e+02
+8.294930000000001e+02 +5.781659999999998e+02
+6.189660000000000e+02 +5.385490000000000e+02
+4.718240000000000e+02 +5.217220000000000e+02
+1.574490000000000e+03 +5.890260000000000e+02
+1.150730000000000e+03 +4.770690000000000e+02
+1.014340000000000e+03 +5.255690000000000e+02
+8.621960000000000e+02 +3.717970000000000e+02
+6.043860000000000e+02 +2.415260000000000e+02
+3.817430000000001e+02 +5.073020000000000e+02
+1.106350000000000e+03 +3.465180000000001e+02
+9.891570000000000e+02 +3.833520000000000e+02
+6.730119999999999e+02 +3.228120000000000e+02
+7.838339999999999e+02 +2.062490000000000e+02
+7.020300000000000e+02 +3.080250000000000e+02
+1.535310000000000e+01 +4.411110000000000e+02
+5.051720000000000e+02 +4.125010000000000e+02
+1.577790000000000e+01 +4.406690000000000e+02
+1.299790000000000e+03 +4.367880000000000e+02
+1.779290000000000e+03 +6.892890000000000e+02
+4.187500000000000e+02 +5.164620000000000e+02
+5.232730000000000e+02 +5.361369999999999e+02
+8.586900000000001e+02 +2.556520000000000e+02
+1.093370000000000e+03 +3.645910000000000e+02
+8.476990000000000e+02 +2.245290000000000e+02
+5.435330000000000e+02 +5.674069999999998e+02
+4.861990000000000e+02 +2.939790000000001e+02
+1.270080000000000e+03 +7.551569999999998e+02
+1.310110000000000e+03 +6.691410000000002e+02
+2.818770000000000e+02 +1.118050000000000e+02
+4.527910000000000e+02 +5.205780000000000e+02
+6.186369999999999e+02 +1.669400000000000e+02
+7.782990000000000e+02 +2.143780000000000e+02
+1.382050000000000e+03 +6.786310000000002e+02
+8.893360000000000e+02 +2.364320000000000e+02
+1.611540000000000e+03 +5.952619999999999e+02
+2.948230000000000e+02 +1.305710000000000e+02
+1.309030000000000e+03 +5.686419999999998e+02
+1.015790000000000e+03 +6.313400000000000e+02
+6.516319999999999e+02 +2.469030000000000e+02
+4.777130000000000e+02 +5.213220000000000e+02
+1.177320000000000e+01 +4.375250000000000e+02
+9.460090000000000e+02 +6.221940000000000e+02
+8.965690000000000e+02 +4.261490000000000e+02
+4.992450000000000e+02 +3.792790000000000e+02
+1.332600000000000e+03 +5.861780000000000e+02
+9.792690000000000e+02 +3.351860000000000e+02
+1.081120000000000e+03 +3.380840000000000e+02
+9.768030000000000e+02 +4.152810000000000e+02
+3.740970000000000e+02 +5.089250000000000e+02
+8.948919999999998e+02 +2.354830000000000e+02
+9.985640000000000e+02 +3.690590000000000e+02
+7.573550000000000e+02 +2.275620000000000e+02
+7.705730000000000e+02 +2.269090000000000e+02
+6.430540000000000e+02 +1.757270000000000e+02
+4.997500000000000e+02 +3.865450000000000e+02
+4.229730000000000e+02 +5.199180000000000e+02
+7.806089999999998e+02 +2.389090000000000e+02
+1.109140000000000e+03 +3.366690000000001e+02
+7.095730000000000e+02 +4.994000000000000e+02
+7.639620000000000e+02 +2.250300000000000e+02
+2.336120000000000e+02 +4.837520000000000e+02
+3.036640000000000e+02 +1.368060000000000e+02
+4.386270000000000e+02 +5.478610000000000e+02
+1.563550000000000e+03 +5.075300000000000e+02
+2.443910000000000e+01 +4.327620000000000e+02
+7.581260000000002e+02 +2.292280000000000e+02
+1.748250000000000e+03 +6.995820000000000e+02
+4.234020000000000e+02 +1.052240000000000e+02
+1.068210000000000e+03 +5.175990000000000e+02
+2.662660000000000e+02 +4.933090000000000e+02
+1.328530000000000e+03 +5.950440000000000e+02
+8.635920000000000e+02 +2.336490000000000e+02
+1.752070000000000e+01 +4.366670000000000e+02
+7.842130000000002e+02 +2.289580000000000e+02
+1.957050000000000e+03 +7.228880000000000e+02
+9.040870000000000e+02 +3.996940000000000e+02
+1.063500000000000e+03 +4.915010000000000e+02
+6.495830000000002e+02 +2.345150000000000e+02
+9.187790000000000e+02 +4.793190000000000e+02
+3.887540000000000e+02 +5.249280000000000e+02
+6.237040000000002e+02 +1.590850000000000e+02
+8.698850000000000e+02 +4.018360000000000e+02
+1.626570000000000e+03 +6.149730000000002e+02
+1.661640000000000e+03 +8.031270000000000e+02
+1.773740000000000e+03 +6.895900000000000e+02
+1.066460000000000e+03 +3.603380000000000e+02
+6.132520000000000e+02 +5.452710000000000e+02
+7.007639999999999e+02 +3.273090000000000e+02
+7.067800000000000e+02 +3.826680000000000e+02
+7.580540000000000e+02 +2.156460000000000e+02
+6.414150000000000e+02 +2.491220000000000e+02
+7.092960000000000e+02 +2.680910000000000e+02
+8.999100000000000e+02 +2.197800000000000e+02
+1.069040000000000e+03 +4.948670000000000e+02
+4.175550000000000e+02 +1.714450000000000e+02
+7.744400000000001e+02 +2.337850000000000e+02
+9.885020000000000e+02 +3.155630000000000e+02
+4.437130000000000e+02 +5.369119999999998e+02
+7.817500000000000e+02 +2.254800000000000e+02
+6.205850000000000e+02 +2.702030000000000e+02
+7.741660000000001e+02 +2.155780000000000e+02
+9.627990000000000e+02 +3.096980000000000e+02
+1.064200000000000e+03 +3.669140000000000e+02
+1.781960000000000e+03 +5.447530000000000e+02
+8.239290000000000e+02 +3.166100000000000e+02
+6.541540000000000e+02 +2.324340000000000e+02
+5.873440000000001e+02 +5.507270000000000e+02
+1.866720000000000e+03 +7.068650000000000e+02
+1.317010000000000e+03 +4.169790000000000e+02
+4.858900000000000e+02 +5.232520000000000e+02
+6.226870000000000e+02 +2.763230000000000e+02
+7.451369999999999e+02 +2.142650000000000e+02
+1.563690000000000e+03 +5.668250000000000e+02
+1.097460000000000e+03 +3.607280000000000e+02
+8.508049999999999e+02 +5.624960000000000e+02
+6.364540000000002e+02 +2.426360000000000e+02
+3.351170000000000e+02 +4.784950000000000e+02
+4.117570000000000e+02 +5.049990000000000e+02
+1.316270000000000e+03 +4.365790000000000e+02
+9.233170000000000e+02 +5.973060000000000e+02
+6.628170000000000e+02 +3.188930000000000e+02
+1.177190000000000e+03 +6.783969999999998e+02
+9.096180000000001e+02 +2.270900000000000e+02
+9.806060000000000e+02 +4.909330000000000e+02
+4.746810000000000e+02 +5.279610000000000e+02
+1.064200000000000e+03 +4.655870000000000e+02
+6.397550000000000e+02 +2.344620000000000e+02
+1.172060000000000e+03 +5.800250000000000e+02
+6.071060000000000e+02 +5.356000000000000e+02
+8.501760000000000e+02 +2.408330000000000e+02
+2.957360000000000e+02 +1.271050000000000e+02
+6.623950000000000e+02 +2.950780000000000e+02
+6.552040000000002e+02 +2.443970000000000e+02
+4.915100000000000e+02 +5.309069999999998e+02
+1.076570000000000e+03 +4.661970000000000e+02
+1.313740000000000e+03 +6.000150000000000e+02
+8.775920000000000e+02 +2.333420000000000e+02
+6.687239999999998e+02 +3.560180000000001e+02
+4.504030000000000e+02 +5.308950000000000e+02
+1.271780000000000e+03 +5.758840000000000e+02
+7.773150000000001e+02 +2.276490000000000e+02
+8.983220000000000e+02 +2.295170000000000e+02
+1.026810000000000e+03 +4.770600000000000e+02
+5.733099999999999e+02 +5.211140000000000e+02
+1.558810000000000e+03 +5.450520000000000e+02
+6.622780000000000e+02 +2.789180000000000e+02
+9.294330000000000e+02 +5.881880000000000e+02
+1.066980000000000e+03 +4.752120000000000e+02
+2.777000000000000e+02 +1.089560000000000e+02
+2.209510000000000e+02 +4.644300000000000e+02
+4.339080000000000e+02 +5.060760000000000e+02
+7.741289999999998e+02 +2.090810000000000e+02
+1.300390000000000e+03 +4.212670000000000e+02
+1.610970000000000e+03 +6.635910000000000e+02
+2.927940000000001e+02 +4.786980000000000e+02
+1.858940000000000e+03 +6.952719999999998e+02
+7.605060000000001e+00 +4.191210000000000e+02
+7.789360000000000e+02 +2.165960000000000e+02
+1.133460000000000e+03 +6.317590000000000e+02
+6.407950000000000e+02 +2.326170000000000e+02
+6.224259999999998e+02 +2.676320000000000e+02
+1.137240000000000e+03 +6.373770000000000e+02
+1.037040000000000e+03 +5.121300000000000e+02
+1.086540000000000e+03 +4.982680000000000e+02
+7.183530000000002e+02 +3.497470000000000e+02
+7.809860000000001e+02 +2.039770000000000e+02
+1.301830000000000e+03 +4.268660000000000e+02
+9.892320000000000e+02 +5.961490000000000e+02
+8.656750000000000e+02 +3.361840000000000e+02
+9.053610000000000e+02 +3.834460000000000e+02
+5.141990000000002e+02 +5.410740000000002e+02
+9.942210000000000e+02 +3.176740000000001e+02
+1.623050000000000e+03 +6.638360000000000e+02
+7.834380000000000e+02 +2.272290000000000e+02
+1.610040000000000e+03 +7.251780000000000e+02
+6.882610000000002e+02 +2.579820000000000e+02
+4.832080000000000e+02 +3.858490000000000e+02
+1.063520000000000e+03 +4.784180000000000e+02
+5.288510000000000e+02 +5.298520000000000e+02
+1.055760000000000e+03 +3.695360000000000e+02
+9.261310000000000e+02 +5.878410000000000e+02
+1.073250000000000e+03 +4.730090000000000e+02
+6.346250000000000e+02 +2.350950000000000e+02
+5.009630000000000e+02 +3.935810000000000e+02
+2.660650000000000e+02 +4.795800000000000e+02
+1.310120000000000e+03 +4.249150000000000e+02
+7.778750000000000e+02 +5.679590000000002e+02
+1.324910000000000e+03 +5.379209999999998e+02
+7.789010000000002e+02 +2.065330000000000e+02
+9.980650000000001e+02 +3.244760000000000e+02
+7.802530000000000e+02 +5.699100000000000e+02
+2.006610000000000e+02 +4.443160000000000e+02
+3.933390000000000e+02 +5.033550000000000e+02
+1.306350000000000e+01 +4.171200000000000e+02
+7.747970000000000e+02 +3.069610000000000e+02
+1.297350000000000e+03 +4.050880000000000e+02
+1.027900000000000e+03 +3.216470000000000e+02
+6.973070000000000e+02 +3.868970000000000e+02
+8.636760000000000e+02 +3.644190000000000e+02
+7.859620000000000e+02 +2.152740000000000e+02
+9.747850000000000e+02 +6.275990000000000e+02
+1.778470000000000e+03 +5.211759999999998e+02
+4.976240000000000e+02 +3.794410000000000e+02
+4.926270000000000e+02 +1.214020000000000e+02
+6.002180000000002e+02 +5.167900000000000e+02
+9.124990000000000e+02 +2.897300000000000e+02
+1.628190000000000e+03 +6.019760000000000e+02
+5.489970000000000e+02 +5.330850000000000e+02
+7.839830000000002e+02 +3.192740000000000e+02
+1.113670000000000e+03 +6.191900000000001e+02
+4.185930000000000e+02 +5.088900000000000e+02
+1.273540000000000e+03 +5.401700000000000e+02
+1.860420000000000e+03 +6.523430000000002e+02
+2.676790000000000e+02 +4.565720000000000e+02
+1.684160000000000e+03 +7.280540000000000e+02
+9.038720000000000e+02 +2.910470000000000e+02
+9.101680000000000e+02 +3.719980000000001e+02
+9.944970000000000e+02 +3.056500000000000e+02
+1.135040000000000e+03 +3.465970000000000e+02
+1.486250000000000e+03 +5.675710000000000e+02
+4.208060000000000e+02 +4.855880000000000e+02
+8.345219999999999e+00 +4.095480000000000e+02
+1.402710000000000e+03 +6.638300000000000e+02
+1.911740000000000e+03 +5.799810000000000e+02
+1.818440000000000e+03 +7.311400000000000e+02
+6.794730000000002e+02 +2.225680000000000e+02
+1.626990000000000e+03 +7.094340000000000e+02
+2.577850000000000e+02 +4.726590000000000e+02
+6.401250000000000e+02 +2.618970000000000e+02
+1.072430000000000e+03 +4.453350000000000e+02
+5.942530000000000e+02 +5.140390000000000e+02
+8.793260000000000e+02 +5.863570000000000e+02
+7.510230000000000e+02 +2.906510000000000e+02
+3.917540000000000e+02 +4.834800000000000e+02
+1.303900000000000e+03 +3.954340000000000e+02
+1.114160000000000e+03 +3.325750000000000e+02
+9.349380000000000e+02 +5.817630000000000e+02
+6.453880000000000e+02 +2.270610000000000e+02
+8.651790000000000e+02 +2.340180000000000e+02
+1.750980000000000e+03 +6.580500000000000e+02
+1.363620000000000e+03 +5.074300000000000e+02
+6.629030000000000e+02 +2.201870000000000e+02
+1.754900000000000e+01 +4.059120000000000e+02
+9.828470000000000e+02 +3.251890000000000e+02
+1.089090000000000e+03 +3.300230000000000e+02
+6.351690000000000e+02 +2.325540000000000e+02
+1.680800000000000e+03 +6.802200000000000e+02
+5.057490000000000e+02 +5.012280000000000e+02
+1.135580000000000e+03 +6.414150000000000e+02
+8.077769999999998e+02 +5.304690000000001e+02
+6.108830000000000e+02 +2.193190000000000e+02
+5.663030000000000e+02 +5.196810000000000e+02
+1.189910000000000e+03 +5.208600000000000e+02
+9.384070000000000e+02 +5.859450000000001e+02
+8.817890000000000e+02 +4.087480000000001e+02
+6.556820000000000e+02 +2.184700000000000e+02
+3.024490000000000e+02 +4.614160000000000e+02
+1.666610000000000e+03 +6.949110000000002e+02
+1.302520000000000e+03 +5.317350000000000e+02
+7.671180000000001e+02 +2.888420000000000e+02
+9.943930000000000e+02 +3.366810000000000e+02
+6.510150000000000e+02 +2.388520000000000e+02
+2.456280000000000e+02 +4.457460000000000e+02
+1.606370000000000e+03 +7.148939999999999e+02
+7.688049999999999e+02 +2.842970000000000e+02
+1.116700000000000e+03 +6.586640000000000e+02
+9.066369999999999e+02 +2.839540000000000e+02
+5.687750000000000e+02 +5.215459999999998e+02
+2.326400000000000e+02 +4.519350000000000e+02
+6.210010000000000e+02 +2.553170000000000e+02
+9.352230000000000e+02 +5.617660000000000e+02
+4.255320000000000e+02 +4.945710000000000e+02
+1.091530000000000e+03 +4.805550000000000e+02
+8.628889999999999e+02 +2.393530000000000e+02
+9.898160000000000e+02 +2.804280000000000e+02
+9.256079999999999e+02 +3.774930000000001e+02
+1.611860000000000e+03 +7.120290000000000e+02
+7.775610000000000e+02 +3.038840000000000e+02
+5.298190000000000e+02 +4.923920000000000e+02
+1.300420000000000e+03 +4.140600000000000e+02
+3.725490000000000e+02 +4.926080000000000e+02
+6.562730000000000e+02 +3.527460000000000e+02
+2.128540000000000e+02 +4.474710000000000e+02
+1.095940000000000e+03 +3.502930000000000e+02
+1.370250000000000e+03 +4.457790000000000e+02
+1.320660000000000e+03 +4.084880000000001e+02
+9.001600000000000e+02 +2.723540000000000e+02
+1.263360000000000e+03 +4.486850000000000e+02
+6.528220000000000e+02 +2.093910000000000e+02
+9.909760000000000e+02 +3.204530000000000e+02
+6.507260000000000e+02 +2.268340000000000e+02
+7.886360000000002e+02 +2.940390000000000e+02
+1.540190000000000e+03 +4.883240000000000e+02
+9.056690000000000e+02 +2.827520000000000e+02
+3.979520000000000e+02 +4.911470000000000e+02
+6.176460000000000e+02 +2.088590000000000e+02
+4.053740000000000e+02 +4.745230000000000e+02
+1.006600000000000e+03 +5.801530000000000e+02
+8.021770000000000e+00 +3.990550000000000e+02
+7.675139999999999e+02 +2.926650000000000e+02
+1.255750000000000e+03 +4.970790000000000e+02
+1.288190000000000e+03 +5.164290000000000e+02
+4.025250000000000e+02 +4.487460000000000e+02
+1.320460000000000e+01 +3.970050000000000e+02
+6.636010000000001e+02 +2.332690000000000e+02
+3.707200000000000e+02 +4.844660000000000e+02
+1.733810000000000e+03 +6.788430000000002e+02
+5.698120000000000e+02 +4.933840000000000e+02
+6.235150000000000e+02 +2.528140000000000e+02
+5.849400000000001e+02 +4.900540000000000e+02
+7.785060000000002e+02 +2.979410000000000e+02
+1.313310000000000e+03 +4.151290000000000e+02
+2.776000000000000e+02 +4.628960000000000e+02
+4.103670000000000e+02 +5.055910000000000e+02
+1.270850000000000e+01 +3.926060000000000e+02
+7.790590000000000e+02 +2.838780000000000e+02
+9.191770000000000e+02 +3.683680000000001e+02
+1.016280000000000e+03 +4.600100000000000e+02
+2.789650000000000e+00 +3.948020000000000e+02
+7.844889999999998e+02 +2.851630000000000e+02
+8.588220000000000e+02 +2.661340000000000e+02
+1.208340000000000e+03 +4.285190000000000e+02
+7.576480000000000e+02 +2.922060000000000e+02
+3.892970000000000e+02 +4.875470000000000e+02
+1.235950000000000e+03 +6.575750000000000e+02
+6.803670000000000e+02 +3.161270000000000e+02
+1.219320000000000e+03 +4.297860000000000e+02
+3.330470000000000e+02 +4.773310000000000e+02
+1.008970000000000e+03 +4.346470000000000e+02
+6.381200000000000e+02 +2.313380000000000e+02
+1.767390000000000e+03 +5.198430000000002e+02
+5.915940000000001e+02 +4.835720000000000e+02
+3.488570000000000e+02 +4.414230000000000e+02
+7.874660000000000e+02 +2.952440000000000e+02
+8.617800000000000e+02 +3.430040000000000e+02
+7.498650000000000e+02 +1.747000000000000e+02
+4.884960000000000e+02 +5.016350000000000e+02
+7.796630000000000e+02 +2.975090000000000e+02
+9.845710000000000e+02 +3.056340000000000e+02
+3.165290000000000e+02 +4.714290000000000e+02
+1.304850000000000e+03 +3.845050000000000e+02
+1.775780000000000e+03 +5.354340000000000e+02
+4.844880000000001e+02 +1.148130000000000e+02
+5.489830000000002e+02 +4.926410000000000e+02
+4.124000000000000e+02 +4.523200000000000e+02
+1.108370000000000e+03 +6.336040000000000e+02
+7.035889999999998e+02 +4.402450000000000e+02
+5.229590000000002e+02 +4.906560000000000e+02
+1.282230000000000e+03 +3.753460000000000e+02
+1.232630000000000e+03 +5.483520000000000e+02
+5.295419999999998e+02 +4.903220000000000e+02
+3.812870000000000e+02 +4.428030000000001e+02
+1.868980000000000e+03 +6.290700000000001e+02
+1.308340000000000e+01 +3.867670000000000e+02
+3.161210000000000e+02 +4.678790000000000e+02
+7.034260000000000e+02 +3.469590000000000e+02
+8.720060000000002e+02 +3.403880000000001e+02
+5.057890000000000e+02 +4.814030000000000e+02
+1.630090000000000e+03 +6.980599999999999e+02
+3.663840000000000e+02 +4.851770000000000e+02
+6.955490000000000e+02 +5.362560000000000e+02
+4.202620000000000e+02 +4.594960000000000e+02
+3.640370000000000e-01 +3.857950000000000e+02
+7.864639999999998e+02 +2.821690000000001e+02
+8.466400000000000e+02 +2.586000000000000e+02
+3.001880000000000e+02 +4.685440000000000e+02
+4.867590000000000e+02 +3.698100000000000e+02
+5.614310000000000e+02 +4.852220000000000e+02
+6.581070000000000e+02 +2.340340000000000e+02
+7.852700000000000e+02 +2.768580000000000e+02
+9.042150000000000e+02 +2.730890000000000e+02
+8.712650000000000e+02 +5.511020000000000e+02
+9.260570000000000e+02 +3.390240000000000e+02
+5.049020000000000e+00 +3.859920000000000e+02
+8.998400000000000e+02 +2.501700000000000e+02
+2.496320000000000e+02 +4.587740000000000e+02
+7.011510000000002e+02 +3.276550000000000e+02
+1.000640000000000e+03 +2.780330000000000e+02
+7.898260000000000e+02 +2.796060000000000e+02
+9.228280000000000e+02 +6.164780000000002e+02
+6.355430000000000e+02 +2.333810000000000e+02
+2.753880000000000e+00 +3.848660000000000e+02
+3.535860000000000e+02 +4.398370000000000e+02
+9.380150000000000e-01 +3.823890000000000e+02
+1.937250000000000e+02 +4.442480000000001e+02
+6.661750000000000e+02 +3.070440000000001e+02
+1.063280000000000e+03 +4.263100000000000e+02
+6.345309999999999e+02 +2.056610000000000e+02
+7.924450000000001e+02 +2.863060000000000e+02
+1.369320000000000e+03 +6.473480000000002e+02
+1.011270000000000e+03 +4.367440000000000e+02
+1.067540000000000e+03 +4.496850000000000e+02
+4.138400000000000e+02 +4.687710000000000e+02
+3.845800000000000e+02 +4.596070000000000e+02
+1.039800000000000e+03 +3.818990000000000e+02
+1.429310000000000e+03 +4.225480000000000e+02
+1.403950000000000e+03 +6.869190000000000e+02
+4.410880000000000e+01 +4.124410000000000e+02
+1.005690000000000e+03 +5.659150000000000e+02
+6.248070000000000e+02 +2.213470000000000e+02
+1.239560000000000e+03 +6.534780000000002e+02
+1.093630000000000e+03 +6.002690000000000e+02
+1.101090000000000e+02 +4.239330000000000e+02
+5.400530000000000e+02 +4.665590000000000e+02
+1.294780000000000e+03 +3.867490000000000e+02
+1.164340000000000e+03 +5.651700000000000e+02
+8.638020000000000e+02 +4.252520000000000e+02
+3.336980000000001e+02 +4.521590000000000e+02
+1.108270000000000e+03 +3.976120000000000e+02
+6.405160000000000e+02 +2.183220000000000e+02
+1.862580000000000e+03 +6.529970000000000e+02
+7.852569999999999e+02 +2.738870000000000e+02
+1.027950000000000e+03 +5.674540000000002e+02
+1.727630000000000e+03 +6.603020000000000e+02
+9.191260000000000e+02 +5.925860000000000e+02
+5.526530000000000e+02 +4.659660000000000e+02
+5.339360000000000e+02 +4.842860000000000e+02
+7.625740000000000e+02 +2.635450000000000e+02
+9.796520000000000e+02 +3.129600000000000e+02
+2.844360000000000e+01 +4.083260000000000e+02
+1.001110000000000e+03 +3.972140000000000e+02
+3.360290000000000e+02 +9.286770000000000e+01
+2.153660000000000e+01 +4.021910000000000e+02
+1.299840000000000e+03 +5.096310000000000e+02
+5.231440000000000e+02 +4.806560000000000e+02
+6.213590000000000e+02 +2.247180000000000e+02
+1.289620000000000e+03 +3.661120000000000e+02
+1.766750000000000e+03 +6.289100000000000e+02
+8.626120000000000e+02 +3.349060000000000e+02
+5.412120000000000e+02 +4.800830000000000e+02
+7.934580000000002e+02 +2.866060000000000e+02
+1.700420000000000e+03 +6.416180000000001e+02
+1.155850000000000e+03 +3.967840000000000e+02
+1.086710000000000e+03 +4.657810000000000e+02
+1.002850000000000e+03 +5.972380000000001e+02
+6.501700000000000e+02 +1.823820000000000e+02
+7.911319999999999e+02 +2.912340000000001e+02
+2.794470000000000e+02 +1.653930000000000e+02
+6.645710000000000e+02 +2.024610000000000e+02
+1.318250000000000e+03 +6.425710000000000e+02
+5.005700000000000e+01 +3.878540000000000e+02
+3.963490000000000e+01 +4.058190000000000e+02
+4.833970000000000e+02 +3.554040000000000e+02
+6.890960000000000e+02 +3.171620000000001e+02
+2.818490000000000e+02 +1.707020000000000e+02
+5.551890000000000e+02 +4.673370000000000e+02
+3.893450000000000e+02 +4.311190000000000e+02
+3.461070000000000e+02 +8.294040000000000e+01
+9.096319999999999e+02 +5.439900000000000e+02
+8.725630000000000e+02 +2.460160000000000e+02
+2.130320000000000e+01 +3.959510000000000e+02
+1.395060000000000e+03 +4.769180000000000e+02
+6.228990000000000e+02 +2.057820000000000e+02
+5.307290000000000e+02 +4.635900000000000e+02
+1.865250000000000e+03 +6.762230000000002e+02
+6.988520000000000e+02 +4.087020000000000e+02
+3.540060000000000e+02 +4.337380000000001e+02
+7.965470000000000e+02 +2.758200000000000e+02
+8.919510000000000e+02 +2.463820000000000e+02
+6.580080000000000e+02 +1.786780000000000e+02
+2.650280000000000e+02 +4.240140000000000e+02
+8.897850000000000e+02 +2.546510000000000e+02
+4.927900000000000e+02 +3.537880000000000e+02
+7.998020000000000e+02 +2.625300000000000e+02
+1.374610000000000e+03 +6.361310000000000e+02
+1.340370000000000e+03 +6.844670000000000e+02
+1.350740000000000e+01 +3.886590000000000e+02
+7.440720000000000e+02 +2.797700000000000e+02
+1.319370000000000e+03 +4.932100000000000e+02
+1.595470000000000e+01 +3.917040000000000e+02
+9.990180000000000e+02 +5.723919999999998e+02
+9.009800000000000e+02 +2.479580000000000e+02
+3.761790000000000e+02 +4.600330000000000e+02
+7.526890000000000e+02 +2.576340000000000e+02
+7.923000000000000e+02 +2.675710000000000e+02
+1.563600000000000e+01 +3.907730000000000e+02
+1.029560000000000e+03 +4.040410000000000e+02
+1.296780000000000e+03 +3.723150000000000e+02
+8.791000000000000e+02 +5.550980000000002e+02
+4.988460000000000e+02 +4.331960000000000e+02
+4.861030000000000e+02 +4.349970000000000e+02
+5.450200000000000e+02 +4.366200000000000e+02
+1.880120000000000e+03 +7.006239999999998e+02
+1.302930000000000e+03 +3.828100000000000e+02
+8.911600000000000e+02 +3.188580000000000e+02
+7.645510000000000e+02 +2.716190000000000e+02
+8.766950000000001e+02 +5.551980000000000e+02
+8.909390000000000e+02 +5.526340000000000e+02
+4.234310000000000e+01 +3.812330000000000e+02
+1.374570000000000e+03 +4.594640000000000e+02
+3.773160000000000e+02 +4.343420000000000e+02
+1.362230000000000e+03 +5.898840000000000e+02
+8.948150000000001e+02 +2.517830000000000e+02
+8.907780000000000e+02 +3.305640000000000e+02
+5.233330000000002e+02 +4.714840000000000e+02
+3.361230000000001e+02 +4.245900000000000e+02
+7.931720000000000e+02 +2.675340000000000e+02
+3.778450000000000e+02 +4.636300000000000e+02
+1.879880000000000e+02 +4.034310000000000e+02
+1.027740000000000e+03 +5.562610000000000e+02
+4.975350000000000e+02 +4.386670000000000e+02
+7.967680000000000e+02 +2.707380000000000e+02
+4.792600000000000e+02 +4.348990000000000e+02
+1.808570000000000e+02 +4.101200000000000e+02
+1.321850000000000e+03 +6.730710000000000e+02
+6.351050000000000e+02 +1.651050000000000e+02
+9.756900000000001e+02 +2.737330000000000e+02
+1.438430000000000e+03 +6.472340000000000e+02
+2.774880000000000e+02 +1.573430000000000e+02
+3.315250000000000e+02 +4.257420000000000e+02
+8.511050000000000e+02 +5.410560000000000e+02
+2.939050000000000e+02 +1.525920000000000e+02
+1.184230000000000e+03 +5.716799999999999e+02
+3.907430000000001e+02 +4.282190000000000e+02
+3.984790000000000e+02 +4.287050000000000e+02
+7.882919999999998e+02 +2.587610000000000e+02
+1.888530000000000e+01 +3.829770000000000e+02
+4.892000000000000e+02 +4.544480000000000e+02
+6.619220000000000e+02 +2.151420000000000e+02
+1.664540000000000e+03 +4.991690000000000e+02
+3.167150000000000e+02 +9.307230000000000e+01
+5.387630000000000e+02 +4.734610000000000e+02
+1.032060000000000e+03 +3.676590000000000e+02
+7.057650000000000e+02 +4.361690000000000e+02
+9.783380000000000e+02 +2.803140000000000e+02
+1.555800000000000e+03 +4.552910000000000e+02
+8.429160000000001e+02 +2.400840000000000e+02
+1.063110000000000e+03 +4.097790000000000e+02
+7.925520000000000e+02 +2.778690000000000e+02
+9.940010000000000e+02 +2.866250000000000e+02
+8.992460000000002e+02 +2.492520000000000e+02
+8.624530000000000e+02 +5.259660000000000e+02
+5.410660000000000e+02 +4.625320000000000e+02
+7.907550000000000e+02 +2.684040000000000e+02
+5.297360000000000e+02 +4.366880000000001e+02
+7.666419999999998e+02 +2.550270000000000e+02
+8.603570000000000e+02 +2.522290000000000e+02
+6.400260000000000e+02 +2.727500000000000e+02
+9.844730000000000e+02 +2.546430000000000e+02
+9.077800000000000e+02 +2.640180000000000e+02
+1.029350000000000e+03 +4.888990000000000e+02
+4.458070000000000e+02 +4.331890000000000e+02
+7.898380000000002e+02 +2.757340000000000e+02
+1.320360000000000e+02 +4.013170000000000e+02
+1.000930000000000e+03 +3.832230000000000e+02
+1.072040000000000e+03 +3.927060000000000e+02
+3.107680000000000e+02 +4.216410000000000e+02
+1.148780000000000e+03 +4.849840000000000e+02
+3.702080000000000e+02 +4.416940000000000e+02
+7.903660000000001e+02 +3.170410000000000e+02
+6.490060000000000e+02 +2.792760000000000e+02
+3.562150000000000e+02 +4.059680000000000e+02
+9.080820000000000e+02 +5.679019999999998e+02
+9.897300000000000e+02 +2.236520000000000e+02
+1.016490000000000e+03 +4.239950000000000e+02
+2.779060000000000e+02 +1.514710000000000e+02
+7.110730000000000e+02 +4.210730000000000e+02
+2.946970000000000e+02 +4.158790000000000e+02
+7.900100000000000e+02 +2.684690000000000e+02
+4.790970000000000e+02 +1.452880000000000e+02
+2.790430000000000e+02 +1.528850000000000e+02
+7.925889999999998e+02 +2.556330000000000e+02
+1.322990000000000e+03 +3.645120000000000e+02
+1.084370000000000e+03 +3.518320000000000e+02
+8.154989999999998e+02 +3.149850000000000e+02
+2.989290000000001e+02 +4.134370000000000e+02
+2.261130000000000e+02 +4.074120000000000e+02
+1.015250000000000e+03 +3.921810000000000e+02
+1.318050000000000e+03 +6.214690000000001e+02
+7.901550000000000e+02 +2.669850000000000e+02
+8.171089999999998e+02 +3.295480000000000e+02
+3.031890000000000e+02 +4.155370000000000e+02
+7.914660000000000e+02 +2.598210000000000e+02
+5.820359999999999e+02 +2.939020000000000e+02
+6.155810000000000e+02 +2.720940000000000e+02
+4.847740000000000e+02 +4.256360000000000e+02
+6.357390000000000e+02 +2.778110000000000e+02
+3.910100000000000e+02 +4.385360000000000e+02
+6.602719999999998e+02 +1.998250000000000e+02
+7.468960000000002e+02 +2.520150000000000e+02
+1.908530000000000e+02 +3.993030000000001e+02
+1.118040000000000e+03 +3.453040000000001e+02
+6.999000000000000e+02 +4.870260000000000e+02
+2.779710000000000e+01 +3.574880000000001e+02
+1.000130000000000e+03 +4.644060000000000e+02
+8.984510000000000e+02 +5.263130000000000e+02
+1.217620000000000e+03 +4.405200000000000e+02
+6.593000000000000e+02 +2.758090000000000e+02
+7.882860000000002e+02 +2.609060000000000e+02
+1.324910000000000e+03 +5.877750000000000e+02
+1.273110000000000e+03 +5.682180000000002e+02
+0.000000000000000e+00 +3.497930000000000e+02
+1.454640000000000e+03 +6.153980000000000e+02
+3.603480000000000e+02 +4.119210000000000e+02
+1.321090000000000e+03 +5.396369999999999e+02
+1.039700000000000e+03 +3.395810000000000e+02
+1.907970000000000e+03 +6.686389999999999e+02
+3.526770000000000e+02 +4.108250000000000e+02
+1.347210000000000e+03 +5.765050000000000e+02
+2.720850000000000e+02 +3.966160000000000e+02
+6.526250000000000e+02 +2.727810000000000e+02
+1.845420000000000e+02 +3.910190000000000e+02
+7.110260000000002e+02 +4.496640000000000e+02
+7.804620000000000e+02 +2.591520000000000e+02
+1.171210000000000e+03 +5.449310000000000e+02
+1.069920000000000e+03 +3.861760000000000e+02
+7.941310000000002e+02 +2.608450000000000e+02
+7.127970000000000e+02 +2.202140000000000e+02
+6.780230000000000e+02 +2.754530000000000e+02
+7.790450000000000e+02 +3.046910000000000e+02
+5.221550000000000e+00 +3.489180000000000e+02
+1.147920000000000e+03 +5.682719999999998e+02
+1.126650000000000e+03 +5.260090000000000e+02
+1.549590000000000e+03 +4.800750000000000e+02
+9.029990000000000e+02 +5.236810000000000e+02
+1.608710000000000e+03 +5.067140000000000e+02
+1.005070000000000e+03 +3.793160000000000e+02
+6.603090000000000e+02 +2.751700000000000e+02
+3.101310000000000e+02 +4.045760000000000e+02
+3.519610000000000e+02 +2.013620000000000e+02
+7.947830000000000e+02 +2.639470000000000e+02
+1.135260000000000e+03 +5.613310000000000e+02
+4.836450000000000e+02 +1.349640000000000e+02
+3.311000000000000e+02 +4.127680000000000e+02
+7.728290000000000e+02 +2.658570000000000e+02
+9.652700000000000e+02 +2.916370000000000e+02
+1.008620000000000e+03 +3.478510000000000e+02
+3.019430000000000e+02 +3.962160000000000e+02
+1.589230000000000e+03 +6.502780000000000e+02
+4.887780000000000e+02 +4.360600000000000e+02
+3.726030000000000e+02 +4.234530000000000e+02
+9.742020000000000e+02 +3.892020000000000e+02
+1.164630000000000e+03 +5.713710000000000e+02
+4.881780000000001e+02 +4.187400000000000e+02
+6.365900000000000e+02 +2.786210000000000e+02
+4.957170000000000e+02 +4.437970000000000e+02
+7.954600000000000e+02 +2.644870000000000e+02
+5.272740000000000e+02 +4.539550000000000e+02
+3.142620000000000e+02 +3.990160000000000e+02
+8.586230000000000e+02 +5.206440000000000e+02
+6.989310000000000e+02 +5.527840000000000e+02
+1.144220000000000e+03 +5.933600000000000e+02
+7.911569999999998e+02 +2.409620000000000e+02
+1.319050000000000e+03 +5.002010000000000e+02
+1.412410000000000e+03 +3.627030000000000e+02
+9.759950000000000e+02 +3.661300000000000e+02
+9.126210000000000e+02 +2.223530000000000e+02
+8.021799999999999e+02 +2.982340000000001e+02
+7.125670000000000e+02 +4.136930000000000e+02
+7.482030000000000e+02 +2.452490000000000e+02
+2.622250000000000e+02 +3.971320000000000e+02
+4.944570000000000e+02 +2.382760000000000e+02
+1.653800000000000e+03 +5.791900000000001e+02
+7.923739999999998e+02 +2.496150000000000e+02
+9.928860000000000e+02 +3.555440000000001e+02
+1.123130000000000e+03 +5.209390000000000e+02
+6.596280000000000e+02 +2.103230000000000e+02
+1.288680000000000e+03 +5.952190000000001e+02
+6.337569999999999e+02 +2.661420000000000e+02
+1.638110000000000e+02 +3.809750000000000e+02
+2.996100000000000e+02 +3.931360000000000e+02
+9.916960000000000e+02 +3.897370000000000e+02
+1.026900000000000e+03 +3.313980000000000e+02
+8.890820000000000e+02 +4.979800000000000e+02
+1.219470000000000e+03 +4.118560000000000e+02
+6.608339999999999e+02 +2.488070000000000e+02
+7.869830000000002e+02 +2.607670000000000e+02
+1.289520000000000e+03 +4.552720000000000e+02
+8.601400000000000e+02 +2.186250000000000e+02
+3.718320000000000e+02 +4.035360000000000e+02
+9.634860000000000e+02 +3.506700000000000e+02
+6.311380000000000e+02 +1.995890000000000e+02
+2.413060000000000e+02 +3.913050000000000e+02
+3.104190000000001e+02 +3.914460000000000e+02
+6.637310000000001e+02 +1.893980000000000e+02
+1.132320000000000e+03 +5.373430000000002e+02
+6.839850000000000e+02 +4.005260000000000e+02
+6.447500000000000e+02 +2.738810000000000e+02
+7.914310000000000e+02 +2.558380000000000e+02
+9.940599999999999e+02 +3.677820000000000e+02
+1.296440000000000e+03 +5.928200000000001e+02
+1.279810000000000e+03 +6.052600000000000e+02
+5.290860000000000e+02 +4.328420000000000e+02
+3.403190000000000e+02 +4.116710000000000e+02
+8.402700000000000e+02 +2.124230000000000e+02
+4.145830000000000e+02 +4.627660000000000e+02
+1.319950000000000e+02 +3.719350000000000e+02
+9.940250000000000e+02 +3.446180000000001e+02
+1.733230000000000e+03 +5.963690000000000e+02
+2.733630000000000e+02 +3.874840000000000e+02
+1.066350000000000e+03 +3.706080000000000e+02
+6.100050000000000e+02 +2.541660000000000e+02
+3.618370000000000e+02 +4.063090000000000e+02
+3.357890000000000e+02 +4.037590000000000e+02
+7.050820000000000e+02 +4.259020000000000e+02
+1.066970000000000e+03 +3.762680000000000e+02
+7.701270000000000e+02 +2.969900000000000e+02
+2.793050000000000e+02 +3.882750000000000e+02
+3.454250000000000e+00 +3.369800000000000e+02
+1.309020000000000e+03 +4.879890000000000e+02
+1.601410000000000e+03 +5.414490000000002e+02
+4.002280000000000e+02 +4.514420000000000e+02
+7.745340000000000e+02 +2.386920000000000e+02
+1.074680000000000e+03 +5.663290000000002e+02
+1.177710000000000e+03 +5.223210000000000e+02
+7.808670000000000e+02 +2.539040000000000e+02
+9.470630000000000e+01 +3.649020000000000e+02
+1.078980000000000e+03 +3.183770000000000e+02
+1.713330000000000e+03 +6.534780000000002e+02
+6.191820000000000e+02 +1.792220000000000e+02
+9.011440000000000e+02 +2.918610000000000e+02
+6.390620000000000e+02 +1.836100000000000e+02
+7.865740000000000e+02 +2.492330000000000e+02
+1.256910000000000e+03 +4.733300000000000e+02
+2.038860000000000e+02 +3.762830000000000e+02
+3.219900000000000e+02 +3.966190000000000e+02
+3.398300000000000e+02 +1.540250000000000e+02
+1.435060000000000e+03 +6.061300000000000e+02
+7.845419999999998e+02 +2.541600000000000e+02
+9.184610000000000e+02 +2.426270000000000e+02
+2.504420000000000e+02 +3.906730000000000e+02
+2.105390000000000e+02 +3.786170000000000e+02
+8.970910000000000e+02 +5.149450000000001e+02
+2.695940000000000e+02 +4.021670000000000e+02
+2.917650000000000e+02 +3.876250000000000e+02
+7.884550000000000e+02 +2.572660000000000e+02
+1.100920000000000e+03 +5.203090000000000e+02
+1.671820000000000e+01 +3.331720000000000e+02
+6.227710000000000e+02 +2.407690000000000e+02
+4.974550000000000e+02 +4.080790000000000e+02
+7.841790000000000e+02 +2.666630000000000e+02
+1.329560000000000e+01 +3.304790000000001e+02
+1.656170000000000e+02 +3.861190000000000e+02
+2.855940000000000e+02 +3.818750000000000e+02
+7.005950000000000e+02 +4.149260000000000e+02
+6.222540000000000e+02 +2.450730000000000e+02
+3.072270000000001e+02 +4.017260000000000e+02
+6.478950000000000e+02 +2.389770000000000e+02
+7.869390000000000e+02 +2.625710000000000e+02
+1.816110000000000e+03 +6.186600000000000e+02
+2.455990000000000e+01 +3.399140000000000e+02
+1.212890000000000e+03 +5.145180000000000e+02
+7.668580000000002e+02 +2.706920000000000e+02
+1.294770000000000e+03 +4.265160000000000e+02
+8.997030000000000e+02 +2.964930000000000e+02
+2.240680000000000e+00 +3.284630000000000e+02
+7.882270000000000e+02 +2.478350000000000e+02
+1.071810000000000e+03 +3.760110000000000e+02
+2.851690000000001e+02 +3.815280000000000e+02
+3.035760000000000e+02 +3.929120000000000e+02
+9.815720000000000e+02 +3.347020000000000e+02
+1.014460000000000e+03 +4.818970000000000e+02
+7.892170000000000e+02 +2.443210000000000e+02
+2.747590000000000e+02 +3.776990000000000e+02
+1.021770000000000e+03 +5.201970000000000e+02
+6.457240000000000e+02 +2.611580000000000e+02
+1.737380000000000e+02 +3.610100000000000e+02
+6.807139999999998e+02 +5.234290000000000e+02
+1.594020000000000e+03 +5.702919999999998e+02
+1.272680000000000e+03 +5.049270000000000e+02
+4.877120000000000e+02 +4.039780000000000e+02
+2.448330000000000e+02 +3.736250000000000e+02
+7.888370000000000e+02 +2.519500000000000e+02
+9.919730000000000e+02 +3.389070000000000e+02
+7.892510000000002e+02 +2.520490000000000e+02
+1.380430000000000e+03 +4.233090000000000e+02
+9.182960000000000e+02 +4.933660000000000e+02
+8.072619999999999e+02 +3.177830000000000e+02
+6.581289999999998e+02 +1.677350000000000e+02
+1.480550000000000e+03 +5.721140000000000e+02
+2.375030000000000e+02 +3.789970000000000e+02
+3.007620000000000e+02 +3.906330000000000e+02
+7.906020000000000e+02 +3.510140000000000e+02
+1.880420000000000e+01 +3.271860000000000e+02
+6.641110000000001e+02 +2.670180000000000e+02
+1.295360000000000e+03 +4.593720000000000e+02
+2.800930000000000e+02 +3.756120000000000e+02
+6.192619999999999e+02 +1.622090000000000e+02
+1.064350000000000e+03 +3.534430000000001e+02
+6.455309999999999e+02 +2.379150000000000e+02
+3.191470000000000e+02 +3.744270000000000e+02
+1.069070000000000e+03 +3.391260000000000e+02
+6.568530000000002e+02 +2.496770000000000e+02
+7.928099999999999e+02 +3.385660000000000e+02
+6.832020000000000e+02 +4.957150000000000e+02
+1.950290000000000e+03 +6.249540000000002e+02
+6.720010000000002e+02 +2.513180000000000e+02
+7.808020000000000e+02 +2.539520000000000e+02
+2.667780000000000e+02 +3.868350000000000e+02
+7.945790000000000e+02 +2.687610000000000e+02
+1.311090000000000e+03 +4.490510000000000e+02
+1.635180000000000e+03 +6.809660000000000e+02
+2.152440000000000e+02 +3.637040000000000e+02
+9.733690000000000e+02 +3.403340000000000e+02
+6.608900000000000e+02 +2.613180000000000e+02
+9.005280000000000e+02 +2.918280000000000e+02
+2.347740000000000e+02 +3.735760000000000e+02
+2.718720000000000e+02 +3.690770000000000e+02
+4.900850000000000e+02 +3.962600000000000e+02
+2.781070000000000e+02 +1.218460000000000e+02
+3.036030000000000e+02 +3.759440000000000e+02
+1.295090000000000e+03 +4.585230000000000e+02
+1.207420000000000e+03 +4.720520000000000e+02
+7.951260000000002e+02 +3.399770000000000e+02
+1.069920000000000e+03 +3.020920000000000e+02
+6.629280000000000e+02 +2.965620000000000e+02
+8.486460000000002e+02 +2.719850000000000e+02
+2.668420000000000e+02 +3.652820000000000e+02
+1.069370000000000e+03 +2.920150000000000e+02
+1.166600000000000e+01 +3.162250000000000e+02
+6.356770000000000e+02 +2.632080000000000e+02
+1.785950000000000e+02 +3.516540000000000e+02
+6.594720000000000e+02 +1.543110000000000e+02
+2.533330000000000e+02 +3.586420000000000e+02
+7.886980000000000e+02 +2.573300000000000e+02
+1.084730000000000e+03 +3.290390000000000e+02
+3.231950000000000e+02 +3.939220000000000e+02
+2.250950000000000e+02 +3.493780000000001e+02
+2.890830000000000e+02 +3.655060000000000e+02
+2.631420000000000e+02 +3.646470000000000e+02
+6.758960000000002e+02 +4.877020000000000e+02
+1.018660000000000e+03 +5.119370000000000e+02
+3.111660000000000e+02 +3.839990000000000e+02
+5.006770000000000e+02 +3.922160000000000e+02
+9.772260000000000e+02 +3.582090000000000e+02
+2.524400000000000e+02 +3.700270000000000e+02
+1.673390000000000e+03 +6.251720000000000e+02
+8.780590000000000e+02 +4.549680000000000e+02
+1.313660000000000e+03 +5.451070000000000e+02
+3.252370000000000e+02 +1.481330000000000e+02
+1.079120000000000e+03 +3.147660000000000e+02
+9.164630000000000e+02 +4.814190000000000e+02
+3.198080000000000e+02 +3.905180000000000e+02
+1.473410000000000e+03 +5.707210000000000e+02
+2.747980000000000e+02 +3.668620000000000e+02
+1.176320000000000e+03 +5.484420000000000e+02
+9.310250000000000e+02 +4.822730000000000e+02
+9.851650000000000e+02 +4.136630000000000e+02
+2.046500000000000e+02 +3.537490000000000e+02
+6.312300000000001e+01 +3.306840000000000e+02
+1.309730000000000e+03 +4.386690000000000e+02
+1.057060000000000e+03 +3.834460000000000e+02
+1.672320000000000e+03 +5.986930000000000e+02
+8.624570000000000e+02 +2.650310000000000e+02
+1.077640000000000e+03 +3.589250000000000e+02
+3.392060000000000e+01 +3.142960000000000e+02
+6.229910000000000e+02 +1.453980000000000e+02
+1.009230000000000e+03 +3.419120000000000e+02
+7.840139999999999e+02 +3.474390000000000e+02
+1.124840000000000e+03 +3.144120000000001e+02
+1.758370000000000e+03 +5.727440000000000e+02
+1.288810000000000e+03 +4.587080000000000e+02
+1.081810000000000e+02 +3.407800000000000e+02
+2.535050000000000e+02 +3.509710000000000e+02
+9.771270000000000e+02 +3.218730000000000e+02
+6.212660000000000e+02 +1.799300000000000e+02
+2.771110000000000e+02 +3.594590000000000e+02
+7.802760000000002e+02 +3.391740000000001e+02
+8.561280000000000e+02 +4.979520000000000e+02
+6.532980000000000e+02 +2.543360000000000e+02
+2.646640000000000e+02 +3.571720000000000e+02
+6.187970000000000e+02 +1.516150000000000e+02
+7.886830000000000e+02 +3.324300000000000e+02
+3.258130000000000e+02 +1.387720000000000e+02
+1.461640000000000e+03 +5.999259999999998e+02
+1.063990000000000e+03 +3.467880000000000e+02
+2.689130000000000e+01 +3.122850000000000e+02
+8.885470000000000e+02 +4.758490000000000e+02
+1.079990000000000e+03 +3.587640000000000e+02
+6.356519999999998e+02 +2.475340000000000e+02
+3.064530000000000e+02 +3.831010000000000e+02
+4.834480000000000e+02 +3.896190000000000e+02
+1.063130000000000e+03 +3.440560000000000e+02
+6.186360000000000e+02 +2.350340000000000e+02
+5.707730000000000e+01 +3.273950000000000e+02
+2.307310000000000e+02 +3.664070000000000e+02
+9.784890000000000e+02 +4.136860000000000e+02
+2.768550000000000e+02 +3.533230000000001e+02
+6.254870000000000e+02 +1.494310000000000e+02
+1.387380000000000e+03 +5.190509999999998e+02
+2.425850000000000e+02 +3.432370000000000e+02
+7.806900000000001e+02 +3.197490000000000e+02
+3.254290000000001e+02 +1.284340000000000e+02
+8.928000000000000e+02 +2.469020000000000e+02
+6.140500000000000e+02 +1.589590000000000e+02
+1.619900000000000e+03 +6.602730000000000e+02
+5.007760000000000e+02 +1.872360000000000e+02
+1.093310000000000e+03 +3.866720000000000e+02
+2.757980000000000e+02 +9.994310000000000e+01
+9.595820000000000e+02 +3.122600000000000e+02
+1.682550000000000e+03 +6.088830000000000e+02
+2.133230000000000e+02 +3.440760000000000e+02
+2.438670000000000e+02 +3.563750000000000e+02
+2.447210000000000e+02 +3.427569999999999e+02
+7.866510000000002e+02 +3.295619999999999e+02
+9.250810000000000e+02 +5.285490000000000e+02
+1.273830000000000e+03 +5.216220000000000e+02
+9.986240000000000e+02 +4.778090000000000e+02
+2.966200000000000e+02 +3.787640000000000e+02
+3.148140000000000e+02 +3.782730000000000e+02
+2.513500000000000e+02 +3.555670000000000e+02
+3.815930000000000e+02 +4.082330000000000e+02
+1.715650000000000e+02 +3.410910000000000e+02
+2.494350000000000e+02 +3.504950000000000e+02
+1.722690000000000e+03 +5.472040000000002e+02
+2.767510000000000e+02 +1.077000000000000e+02
+1.839520000000000e+01 +3.064080000000000e+02
+5.066700000000000e+02 +3.880200000000000e+02
+1.903870000000000e+02 +3.410570000000000e+02
+2.423520000000000e+02 +3.455670000000000e+02
+3.369090000000000e+02 +1.288240000000000e+02
+4.264680000000000e+02 +4.123090000000000e+02
+2.006910000000000e+02 +3.467960000000000e+02
+2.533020000000000e+02 +3.520540000000001e+02
+7.899299999999999e+02 +3.264130000000000e+02
+6.770450000000000e+02 +4.658890000000000e+02
+1.058340000000000e+03 +4.902430000000001e+02
+6.183470000000000e+02 +2.795890000000000e+02
+6.348170000000000e+02 +4.735060000000000e+02
+1.848720000000000e+02 +3.332820000000000e+02
+2.506600000000000e+02 +3.553450000000000e+02
+9.276480000000000e+02 +2.930780000000000e+02
+7.156780000000000e+02 +4.641260000000000e+02
+9.411990000000000e+02 +2.950080000000000e+02
+1.207200000000000e+03 +4.552900000000000e+02
+1.065640000000000e+03 +4.851810000000000e+02
+2.212900000000000e+02 +3.360150000000000e+02
+9.798970000000000e+02 +3.099170000000001e+02
+2.414380000000000e+02 +3.442350000000000e+02
+2.285220000000000e+02 +3.459950000000000e+02
+7.869900000000000e+02 +3.237870000000001e+02
+9.951980000000000e+02 +3.324900000000000e+02
+4.143570000000000e+01 +3.032730000000000e+02
+6.653460000000000e+02 +2.781670000000000e+02
+1.023520000000000e+03 +3.180520000000000e+02
+4.838170000000000e+02 +4.000320000000000e+02
+6.588750000000000e+02 +3.652780000000000e+02
+1.080260000000000e+03 +3.207660000000000e+02
+2.210210000000000e+02 +3.367770000000000e+02
+9.940900000000000e+02 +3.269750000000000e+02
+8.755690000000000e+02 +2.967110000000000e+02
+1.660730000000000e+03 +5.248180000000000e+02
+6.356780000000000e+02 +2.342730000000000e+02
+7.953090000000000e+02 +3.220250000000000e+02
+1.000830000000000e+03 +3.309870000000000e+02
+7.014160000000001e+02 +2.268220000000000e+02
+7.878550000000000e+02 +3.169800000000000e+02
+1.743430000000000e+02 +3.419430000000001e+02
+1.272890000000000e+02 +3.343090000000000e+02
+7.971230000000000e+02 +3.442310000000000e+02
+2.768370000000000e+02 +1.023570000000000e+02
+6.457030000000000e+02 +2.375060000000000e+02
+1.291250000000000e+02 +3.310610000000000e+02
+1.300150000000000e+03 +4.475870000000000e+02
+7.996130000000001e+02 +3.714160000000000e+02
+1.473150000000000e+02 +3.335510000000000e+02
+1.955780000000000e+02 +3.294880000000001e+02
+9.949770000000000e+02 +3.139740000000000e+02
+8.148230000000000e+02 +3.753030000000001e+02
+1.084610000000000e+03 +3.305320000000000e+02
+1.341430000000000e+01 +2.984360000000000e+02
+1.000630000000000e+03 +3.368850000000000e+02
+1.723440000000001e+02 +3.350960000000000e+02
+2.097180000000000e+02 +3.434620000000000e+02
+8.002550000000000e+02 +3.398050000000000e+02
+1.123640000000000e+03 +4.934050000000000e+02
+3.173810000000000e+02 +3.665470000000000e+02
+9.193810000000000e+02 +2.653510000000000e+02
+1.486170000000000e+02 +3.269360000000000e+02
+9.689070000000000e+02 +3.269810000000000e+02
+7.876650000000000e+02 +3.293210000000000e+02
+6.486270000000000e+02 +2.199840000000000e+02
+6.249370000000000e+02 +2.588800000000000e+02
+8.002460000000002e+02 +3.139400000000000e+02
+9.027700000000000e+02 +2.716160000000000e+02
+1.238700000000000e+03 +5.005590000000000e+02
+3.186260000000000e+02 +3.300480000000000e+02
+1.162720000000000e+03 +5.086090000000000e+02
+6.791300000000000e+02 +2.289170000000000e+02
+7.020169999999998e+02 +2.236850000000000e+02
+1.400650000000000e+03 +5.447390000000000e+02
+3.161850000000000e+02 +3.306770000000000e+02
+1.160790000000000e+03 +4.306850000000000e+02
+2.522670000000000e+02 +3.368730000000001e+02
+6.619060000000002e+02 +2.443810000000000e+02
+1.124640000000000e+03 +5.390180000000000e+02
+2.264690000000000e+02 +3.244020000000000e+02
+9.182809999999999e+02 +3.299100000000000e+02
+2.949970000000000e+02 +3.627100000000000e+02
+7.052810000000002e+02 +4.098840000000000e+02
+2.408200000000000e+02 +3.445600000000000e+02
+2.852690000000000e+02 +2.120150000000000e+02
+6.522460000000000e+02 +2.368850000000000e+02
+1.572820000000000e+03 +5.343880000000000e+02
+1.625850000000000e+02 +3.273900000000000e+02
+1.081860000000000e+03 +4.667150000000000e+02
+6.595820000000000e+02 +2.232810000000000e+02
+1.981380000000000e+02 +3.144360000000000e+02
+9.618390000000001e+02 +3.240730000000000e+02
+8.913639999999998e+02 +2.829060000000000e+02
+1.101230000000000e+03 +5.691540000000000e+02
+9.097970000000000e+00 +2.899460000000000e+02
+1.013380000000000e+03 +4.672420000000000e+02
+8.961700000000000e+02 +2.798150000000000e+02
+2.483690000000000e+02 +3.328960000000000e+02
+6.661849999999999e+02 +2.321300000000000e+02
+2.251850000000000e+02 +3.286480000000000e+02
+2.171610000000000e+02 +3.261370000000000e+02
+1.305260000000000e+03 +3.872540000000000e+02
+7.449260000000000e+02 +4.616750000000000e+02
+1.417220000000000e+02 +3.194430000000000e+02
+1.966590000000000e+02 +3.146530000000000e+02
+1.566830000000000e+03 +4.938320000000000e+02
+9.100520000000000e+02 +2.582170000000000e+02
+1.602590000000000e+03 +6.106469999999998e+02
+3.647230000000000e+02 +3.709140000000000e+02
+1.516170000000000e+02 +3.176710000000000e+02
+6.711860000000000e+02 +4.433620000000000e+02
+1.247590000000000e+03 +5.180850000000000e+02
+1.000670000000000e+03 +4.804810000000000e+02
+2.528300000000000e+02 +3.317730000000000e+02
+1.670600000000000e+03 +5.105930000000000e+02
+9.805750000000000e+02 +3.189510000000000e+02
+1.045150000000000e+03 +3.800120000000000e+02
+8.986260000000002e+02 +2.568030000000000e+02
+1.005220000000000e+03 +4.554190000000000e+02
+7.960610000000000e+02 +3.229040000000000e+02
+3.502300000000000e+02 +3.917030000000000e+02
+8.071650000000000e+02 +2.692690000000000e+02
+1.023990000000000e+02 +3.120240000000000e+02
+5.055100000000000e+02 +4.301230000000001e+02
+2.177910000000000e+02 +3.278300000000000e+02
+1.084480000000000e+03 +3.696140000000000e+02
+9.069410000000000e+02 +2.791560000000000e+02
+1.583300000000000e+00 +2.853990000000000e+02
+1.038530000000000e+01 +2.966800000000000e+02
+1.394910000000000e+03 +3.675150000000000e+02
+2.361460000000000e+02 +3.231700000000000e+02
+1.535580000000000e+03 +5.942050000000000e+02
+2.386020000000000e+02 +3.285170000000000e+02
+1.081810000000000e+03 +4.746460000000000e+02
+2.788420000000000e+02 +7.984020000000000e+01
+6.444460000000000e+02 +2.260450000000000e+02
+6.631450000000000e+02 +4.304290000000000e+02
+5.639119999999998e+02 +3.995560000000000e+02
+6.179640000000001e+02 +2.129390000000000e+02
+6.228620000000000e+02 +2.450440000000000e+02
+7.902220000000000e+02 +3.013160000000000e+02
+9.764630000000000e+02 +4.461400000000000e+02
+1.370500000000000e+03 +5.759380000000000e+02
+6.390010000000000e+02 +2.224150000000000e+02
+2.332070000000000e+02 +3.242720000000000e+02
+1.299970000000000e+03 +4.007120000000000e+02
+4.698890000000000e+02 +3.711890000000000e+02
+2.059900000000000e+02 +3.104500000000000e+02
+7.803880000000000e+02 +3.238110000000000e+02
+2.980660000000000e+02 +3.646570000000000e+02
+6.545150000000000e+02 +2.601770000000000e+02
+6.316669999999998e+02 +2.609870000000000e+02
+1.480850000000000e+03 +5.190670000000000e+02
+1.058740000000000e+03 +4.408420000000000e+02
+9.825950000000000e+02 +3.109960000000000e+02
+1.147650000000000e+03 +4.891610000000000e+02
+8.996010000000001e+02 +2.650860000000000e+02
+8.983520000000000e+02 +3.740060000000000e+02
+2.815560000000000e+02 +8.369560000000000e+01
+6.524019999999998e+02 +2.398120000000000e+02
+1.979230000000000e+02 +3.117860000000000e+02
+3.443830000000001e+02 +1.070850000000000e+02
+4.762000000000000e+02 +3.898470000000000e+02
+4.932410000000000e+02 +1.217540000000000e+02
+6.362809999999999e+02 +2.310770000000000e+02
+1.190330000000000e+02 +3.041920000000000e+02
+7.473310000000000e+02 +2.699930000000000e+02
+1.997060000000000e+02 +3.345580000000000e+02
+1.618810000000000e+02 +3.257600000000000e+02
+6.580580000000000e+02 +2.234360000000000e+02
+9.945720000000000e+02 +3.404790000000001e+02
+8.944420000000000e+02 +3.736430000000000e+02
+1.760390000000000e+01 +2.931280000000000e+02
+7.653589999999998e+02 +4.514870000000000e+02
+8.744280000000000e+02 +3.321560000000000e+02
+6.685839999999999e+02 +2.216260000000000e+02
+8.060150000000000e+01 +3.035310000000000e+02
+7.897970000000000e+02 +3.028810000000000e+02
+9.102320000000000e+02 +2.545760000000000e+02
+1.208600000000000e+02 +3.034880000000000e+02
+1.083640000000000e+03 +3.819780000000000e+02
+3.994490000000000e+02 +3.607780000000000e+02
+1.199770000000000e+03 +5.068180000000000e+02
+9.376960000000000e+02 +3.289940000000000e+02
+2.224680000000000e+02 +3.106500000000000e+02
+7.964830000000002e+02 +3.067590000000000e+02
+8.507480000000000e+02 +2.563490000000000e+02
+1.886900000000000e-01 +2.738990000000000e+02
+1.073910000000000e+03 +4.342070000000000e+02
+6.562089999999999e+02 +2.285330000000000e+02
+5.948010000000000e+01 +2.885700000000000e+02
+7.639220000000000e+02 +3.025220000000000e+02
+9.894400000000001e+02 +3.383440000000000e+02
+8.872930000000000e+02 +2.563590000000000e+02
+6.101020000000000e+02 +2.123910000000000e+02
+1.608430000000000e+02 +3.083330000000000e+02
+7.522750000000000e+02 +2.904960000000000e+02
+1.301850000000000e+03 +5.409710000000000e+02
+2.072070000000000e+02 +3.128730000000000e+02
+1.582190000000000e+02 +3.130610000000000e+02
+9.996910000000000e+02 +4.188410000000000e+02
+1.256940000000000e+02 +3.037670000000000e+02
+4.230780000000000e+02 +3.791710000000000e+02
+1.075900000000000e+03 +4.679530000000000e+02
+7.692400000000000e+02 +2.932490000000000e+02
+1.182340000000000e+03 +4.632500000000000e+02
+1.642090000000000e+03 +5.390700000000001e+02
+1.055700000000000e+03 +4.645550000000000e+02
+1.071790000000000e+03 +3.553160000000000e+02
+8.894680000000002e+02 +2.642650000000000e+02
+8.732370000000000e+02 +4.354680000000000e+02
+2.343650000000000e+02 +3.220470000000000e+02
+6.553180000000000e+02 +2.221150000000000e+02
+1.239090000000000e+02 +3.077640000000000e+02
+9.917270000000000e+02 +3.107340000000001e+02
+7.588310000000000e+02 +2.883250000000000e+02
+1.077030000000000e+03 +4.818310000000000e+02
+6.567760000000002e+02 +2.129530000000000e+02
+1.923770000000000e+02 +3.088090000000000e+02
+2.079550000000000e+02 +3.138380000000000e+02
+1.145540000000000e+03 +4.502020000000000e+02
+1.511430000000000e+02 +3.074520000000000e+02
+2.867510000000000e+02 +3.416000000000000e+02
+1.768110000000000e+02 +3.086280000000000e+02
+1.125520000000000e+03 +3.671660000000000e+02
+2.355230000000000e+02 +3.204260000000000e+02
+5.155409999999998e+02 +3.953350000000000e+02
+1.318000000000000e+03 +3.936440000000000e+02
+8.917439999999998e+02 +2.579430000000000e+02
+3.910080000000000e+02 +3.520140000000000e+02
+8.368360000000000e+02 +4.398150000000000e+02
+8.181569999999998e+02 +4.558590000000000e+02
+9.056430000000000e+02 +3.429290000000001e+02
+1.385760000000000e+03 +4.708980000000000e+02
+1.311490000000000e+03 +4.219780000000000e+02
+9.005160000000002e+02 +2.828200000000000e+02
+5.474850000000000e+02 +4.014380000000001e+02
+1.066730000000000e+03 +4.279140000000000e+02
+2.000990000000000e+02 +3.081070000000000e+02
+8.357489999999998e+02 +4.372060000000000e+02
+1.605260000000000e+02 +3.041340000000000e+02
+1.154680000000000e+03 +4.797370000000000e+02
+9.432410000000000e+02 +4.611020000000000e+02
+1.972060000000000e+02 +3.091150000000000e+02
+2.852910000000000e+02 +3.347580000000001e+02
+1.452430000000000e+03 +5.288520000000000e+02
+1.752010000000000e+02 +2.996920000000000e+02
+1.577370000000000e+03 +4.872860000000000e+02
+1.019640000000000e+03 +3.343750000000000e+02
+2.261470000000000e+02 +3.024150000000000e+02
+9.959160000000001e+02 +4.554430000000000e+02
+2.581460000000000e+02 +3.318500000000000e+02
+9.825830000000000e+02 +4.234680000000000e+02
+5.005780000000000e+02 +3.750950000000000e+02
+7.430839999999999e+02 +2.872240000000000e+02
+6.400670000000000e+02 +2.223830000000000e+02
+9.894180000000000e+02 +2.974420000000000e+02
+2.166150000000000e+01 +2.722730000000000e+02
+6.542809999999999e+02 +4.104520000000000e+02
+1.111390000000000e+03 +4.518200000000000e+02
+1.800660000000000e+02 +3.164100000000000e+02
+7.570939999999998e+02 +2.758220000000000e+02
+1.704170000000000e+02 +2.958790000000000e+02
+7.437330000000002e+02 +2.704060000000000e+02
+7.355249999999999e+01 +2.767730000000000e+02
+9.044730000000000e+02 +3.368790000000000e+02
+2.347520000000000e+02 +3.325030000000001e+02
+4.666040000000000e+02 +3.768950000000000e+02
+1.370560000000000e+03 +4.666980000000000e+02
+2.168380000000000e+02 +3.023830000000000e+02
+1.010020000000000e+03 +4.335270000000000e+02
+1.191700000000000e+02 +2.911980000000000e+02
+1.143360000000000e+03 +4.498480000000000e+02
+8.667990000000000e+02 +4.222600000000000e+02
+1.095260000000000e+01 +2.608400000000000e+02
+1.004320000000000e+03 +2.907690000000000e+02
+8.756840000000000e+02 +3.083020000000000e+02
+1.945610000000000e+01 +2.660860000000000e+02
+1.297770000000000e+03 +3.703640000000000e+02
+9.069270000000000e+02 +2.469100000000000e+02
+7.475810000000000e+02 +2.598730000000000e+02
+7.626550000000000e+02 +2.849090000000000e+02
+1.369400000000000e+03 +5.087010000000000e+02
+6.631150000000000e+02 +2.392430000000000e+02
+1.087950000000000e+03 +4.277890000000000e+02
+7.576500000000000e+02 +2.404810000000000e+02
+8.854800000000000e+02 +2.487450000000000e+02
+7.905130000000000e+02 +4.312250000000000e+02
+1.066590000000000e+03 +4.078060000000000e+02
+7.439460000000000e+02 +2.735520000000000e+02
+6.491130000000001e+02 +2.019310000000000e+02
+1.285210000000000e+02 +2.841900000000000e+02
+1.151820000000000e+03 +4.566270000000000e+02
+6.856660000000001e+02 +4.036020000000000e+02
+4.269690000000000e+02 +3.656130000000001e+02
+6.480090000000000e+02 +1.932940000000000e+02
+1.616230000000000e+02 +2.909260000000000e+02
+7.972719999999998e+02 +3.918190000000000e+02
+1.494850000000000e+01 +2.562870000000000e+02
+1.794170000000000e+02 +2.995580000000000e+02
+6.470060000000000e+02 +3.838550000000000e+02
+1.066970000000000e+03 +3.544850000000000e+02
+1.592300000000000e+03 +5.491970000000000e+02
+5.267830000000000e+02 +3.758300000000000e+02
+6.132010000000000e+02 +1.895060000000000e+02
+7.532980000000000e+02 +3.821310000000000e+02
+4.543700000000000e+02 +1.459400000000000e+02
+7.554150000000000e+02 +2.910880000000000e+02
+2.818490000000000e+02 +1.726420000000000e+02
+8.799200000000000e+02 +2.362210000000000e+02
+2.845230000000000e+02 +3.308190000000000e+02
+3.142330000000000e+02 +3.332480000000001e+02
+7.227290000000001e+01 +2.652360000000000e+02
+8.635350000000000e+02 +4.235190000000000e+02
+5.684720000000000e-01 +2.530490000000000e+02
+5.101120000000000e+02 +3.689720000000000e+02
+2.291570000000000e+02 +2.939390000000000e+02
+1.104240000000000e+03 +3.397750000000000e+02
+7.687960000000000e+02 +2.649780000000000e+02
+7.748960000000002e+02 +2.715060000000000e+02
+3.730000000000000e+02 +3.390090000000000e+02
+1.093590000000000e+03 +4.273050000000000e+02
+8.835620000000000e+02 +3.113540000000001e+02
+1.278280000000000e+02 +2.857890000000000e+02
+5.403770000000000e+01 +2.641880000000000e+02
+2.773960000000000e+02 +3.188870000000000e+02
+1.194680000000000e+01 +2.519390000000000e+02
+1.293410000000000e+03 +3.580370000000000e+02
+8.569410000000000e+02 +2.973300000000000e+02
+1.726710000000000e+02 +2.940080000000000e+02
+6.265950000000000e+02 +2.103940000000000e+02
+1.109900000000000e+03 +3.717590000000000e+02
+3.025180000000000e+02 +3.267620000000000e+02
+7.533400000000000e+02 +2.696710000000000e+02
+2.185750000000000e+02 +2.900850000000000e+02
+1.658750000000000e+02 +2.823310000000000e+02
+1.579030000000000e+03 +4.383640000000000e+02
+6.387530000000000e+02 +2.012040000000000e+02
+9.010920000000000e+02 +2.223090000000000e+02
+7.750160000000002e+02 +4.010190000000000e+02
+9.737920000000000e+02 +2.767820000000000e+02
+8.727230000000002e+02 +4.137500000000000e+02
+2.746160000000000e+02 +3.199940000000000e+02
+1.312690000000000e+03 +3.538869999999999e+02
+9.097809999999999e+02 +2.418380000000000e+02
+1.330280000000000e+01 +2.489320000000000e+02
+3.568450000000000e+01 +2.537610000000000e+02
+6.592089999999999e+02 +2.199100000000000e+02
+8.480720000000000e+02 +2.150760000000000e+02
+7.488470000000000e+02 +2.604670000000000e+02
+3.743240000000000e+02 +3.350730000000001e+02
+7.482700000000000e+02 +2.577080000000000e+02
+1.799220000000000e+02 +2.840210000000000e+02
+1.311100000000000e+03 +3.631230000000001e+02
+1.068340000000000e+03 +4.695830000000000e+02
+4.727840000000000e+02 +3.471930000000001e+02
+2.124140000000000e+01 +2.526210000000000e+02
+3.209800000000000e+01 +2.632160000000000e+02
+6.486050000000000e+02 +3.008770000000000e+02
+7.483969999999998e+02 +2.628820000000000e+02
+7.443720000000000e+02 +2.629770000000000e+02
+6.631840000000000e+02 +3.075410000000000e+02
+5.335000000000000e+01 +2.567150000000000e+02
+1.612480000000000e+00 +2.463720000000000e+02
+2.276850000000000e+02 +2.970400000000000e+02
+2.784860000000000e+02 +1.577010000000000e+02
+7.845860000000000e+02 +3.742900000000000e+02
+7.753989999999999e+02 +2.255210000000000e+02
+7.997900000000000e+02 +2.872480000000000e+02
+1.785410000000000e+02 +2.675750000000000e+02
+1.124460000000000e+03 +4.345650000000000e+02
+1.781550000000000e+02 +2.773600000000000e+02
+1.305820000000000e+03 +3.520280000000000e+02
+1.100510000000000e+03 +3.607720000000000e+02
+6.907289999999998e+02 +2.960400000000000e+02
+1.101210000000000e+03 +4.095190000000000e+02
+6.493450000000000e+02 +3.060880000000000e+02
+7.519410000000000e+02 +3.712470000000000e+02
+8.532520000000000e+00 +2.448490000000000e+02
+7.583510000000001e+02 +2.555590000000000e+02
+1.121480000000000e+03 +4.318970000000000e+02
+2.289880000000000e+02 +2.876140000000001e+02
+1.223830000000000e+03 +4.257500000000000e+02
+2.699860000000000e+02 +3.168740000000000e+02
+8.875580000000000e+02 +3.863490000000000e+02
+1.506740000000000e+01 +2.487740000000000e+02
+7.456940000000000e+02 +2.537070000000000e+02
+4.032410000000000e+02 +3.404260000000000e+02
+1.068140000000000e+03 +3.910750000000000e+02
+1.066920000000000e+03 +3.783600000000000e+02
+9.331120000000000e+00 +2.442720000000000e+02
+9.737010000000000e+02 +2.929040000000000e+02
+7.657480000000000e+02 +2.632210000000000e+02
+6.169490000000002e+02 +3.551010000000000e+02
+1.586770000000000e+03 +6.081990000000002e+02
+2.204020000000000e+02 +2.838750000000000e+02
+2.814890000000001e+02 +2.811950000000000e+02
+5.223540000000000e+01 +2.516060000000000e+02
+7.460180000000000e+02 +2.682610000000000e+02
+1.581460000000000e+02 +2.712900000000000e+02
+1.328190000000000e+03 +4.935140000000000e+02
+6.351020000000000e+02 +3.611670000000000e+02
+6.351830000000000e+02 +2.767320000000000e+02
+1.525620000000000e+02 +2.674370000000000e+02
+3.482480000000001e+02 +8.835410000000000e+01
+1.651610000000000e+02 +2.672990000000000e+02
+7.619710000000000e+02 +2.540910000000000e+02
+1.150910000000000e+03 +4.365590000000000e+02
+1.595700000000000e+02 +2.633150000000000e+02
+1.361030000000000e+02 +2.572740000000000e+02
+9.752190000000001e+02 +3.146050000000000e+02
+9.010560000000000e+02 +2.337980000000000e+02
+2.133510000000000e+02 +2.823910000000000e+02
+3.841950000000000e+02 +3.333480000000000e+02
+6.394320000000000e+02 +2.642890000000000e+02
+1.339000000000000e+00 +2.362480000000000e+02
+1.605940000000000e+02 +2.610100000000000e+02
+2.221790000000000e+02 +2.724090000000000e+02
+1.064010000000000e+03 +3.845790000000000e+02
+3.174750000000000e+01 +2.397770000000000e+02
+1.082230000000000e+03 +3.373060000000000e+02
+1.065570000000000e+02 +2.546750000000000e+02
+3.624300000000000e+01 +2.375720000000000e+02
+1.120290000000000e+03 +3.299290000000001e+02
+1.584220000000000e+03 +5.514830000000002e+02
+7.498860000000002e+02 +2.638260000000000e+02
+9.028270000000000e+02 +3.109550000000000e+02
+2.621170000000000e+02 +2.991860000000000e+02
+1.356510000000000e+01 +2.343510000000000e+02
+2.813010000000000e+02 +2.686870000000000e+02
+5.375570000000000e-01 +2.325470000000000e+02
+7.961840000000000e+02 +2.513060000000000e+02
+1.298950000000000e+03 +3.773220000000000e+02
+8.708080000000000e+02 +2.204050000000000e+02
+8.132139999999998e+02 +4.089190000000000e+02
+2.351580000000000e+02 +2.909030000000000e+02
+6.744750000000000e+02 +2.592650000000000e+02
+1.798820000000000e+01 +2.384670000000000e+02
+7.506100000000000e+02 +2.412060000000000e+02
+9.910970000000000e+02 +3.860770000000000e+02
+4.269020000000000e+01 +2.494900000000000e+02
+5.374260000000001e-01 +2.315740000000000e+02
+9.233560000000000e+02 +4.187580000000000e+02
+3.128570000000000e+02 +8.088460000000001e+01
+9.446510000000000e+02 +4.447210000000000e+02
+9.051520000000000e+02 +2.859950000000000e+02
+1.381140000000000e+03 +4.969050000000000e+02
+1.406480000000000e+02 +2.440880000000000e+02
+7.940930000000002e+02 +2.685250000000000e+02
+6.348360000000000e+02 +3.506130000000001e+02
+6.536460000000000e+02 +3.578210000000000e+02
+7.900089999999999e+02 +2.605560000000000e+02
+9.801840000000000e+02 +2.982000000000000e+02
+8.598160000000000e+02 +2.267620000000000e+02
+1.081370000000000e+03 +3.877960000000000e+02
+6.709630000000002e+00 +2.319740000000000e+02
+1.117790000000000e+03 +3.536090000000001e+02
+2.539370000000000e+00 +2.282340000000000e+02
+1.507270000000000e+02 +2.479880000000000e+02
+9.776630000000000e+02 +3.755960000000000e+02
+2.507720000000000e+02 +2.915770000000000e+02
+2.391250000000000e+02 +2.860160000000000e+02
+2.418410000000000e+02 +2.856750000000000e+02
+1.093540000000000e+02 +2.507260000000000e+02
+9.214340000000000e+02 +4.028850000000000e+02
+4.520890000000000e+02 +1.260980000000000e+02
+1.603190000000000e+02 +2.708430000000000e+02
+1.192820000000000e+02 +2.535700000000000e+02
+6.185240000000000e+02 +2.000100000000000e+02
+6.254950000000000e+02 +1.875040000000000e+02
+8.985780000000000e+02 +2.794970000000000e+02
+2.059570000000000e+02 +2.644710000000000e+02
+1.253950000000000e+01 +2.291930000000000e+02
+8.971319999999999e+02 +4.290700000000000e+02
+4.716360000000000e+02 +1.576740000000000e+02
+1.333850000000000e+02 +2.413760000000000e+02
+7.916870000000000e+02 +2.500390000000000e+02
+1.304520000000000e+03 +4.656120000000000e+02
+1.003360000000000e+00 +2.242760000000000e+02
+6.353819999999999e+02 +2.589190000000000e+02
+2.007500000000000e+00 +2.258450000000000e+02
+1.402510000000000e+02 +2.506770000000000e+02
+7.936530000000000e+02 +2.492230000000000e+02
+9.574740000000000e+02 +4.480630000000001e+02
+2.221860000000000e+02 +2.804390000000000e+02
+7.690460000000000e+02 +3.695350000000000e+02
+9.785080000000000e+02 +3.608650000000000e+02
+6.547840000000000e+02 +3.399790000000001e+02
+1.689330000000000e+02 +2.420470000000000e+02
+1.648880000000000e+02 +2.684300000000000e+02
+4.447010000000000e+00 +2.254130000000000e+02
+9.957380000000001e+02 +3.179480000000000e+02
+1.386770000000000e+02 +2.426140000000000e+02
+1.181680000000000e+03 +4.308210000000000e+02
+2.128470000000000e+02 +2.806970000000000e+02
+6.222240000000000e+00 +2.288490000000000e+02
+4.408420000000000e+02 +3.203790000000000e+02
+8.901050000000000e+02 +3.065150000000000e+02
+1.781340000000000e+02 +2.713770000000000e+02
+2.204830000000000e+02 +2.605140000000000e+02
+5.890430000000000e+00 +2.235790000000000e+02
+2.245940000000000e+01 +2.340880000000000e+02
+9.773479999999999e+00 +2.254390000000000e+02
+1.336560000000000e+02 +2.515830000000000e+02
+8.692150000000000e+02 +3.116350000000000e+02
+1.308470000000000e+03 +4.812110000000000e+02
+1.068100000000000e+03 +3.348910000000000e+02
+8.528850000000000e+02 +3.768520000000000e+02
+1.947170000000000e+02 +2.750710000000000e+02
+5.152290000000000e+00 +2.232380000000000e+02
+3.506290000000000e+00 +2.233250000000000e+02
+1.344240000000000e+02 +2.454580000000000e+02
+8.739460000000000e+02 +2.717190000000000e+02
+7.444720000000000e+02 +2.449800000000000e+02
+1.329490000000000e+02 +2.471680000000000e+02
+6.198340000000002e+02 +1.753810000000000e+02
+7.914160000000001e+02 +2.326830000000000e+02
+9.621860000000000e+02 +3.348430000000000e+02
+4.028620000000000e-02 +2.194110000000000e+02
+1.151870000000000e+02 +2.434700000000000e+02
+7.819400000000001e+02 +2.294410000000000e+02
+5.942859999999999e+02 +3.676630000000000e+02
+7.101799999999999e+02 +3.813860000000000e+02
+6.367690000000000e+02 +2.431130000000000e+02
+1.218110000000000e+03 +3.892750000000000e+02
+1.068440000000000e+03 +3.867000000000000e+02
+3.382820000000000e+02 +1.619060000000000e+02
+1.883600000000000e+02 +2.671560000000000e+02
+2.146890000000000e+00 +2.175540000000000e+02
+6.564889999999998e+02 +2.628270000000000e+02
+2.082120000000000e+02 +2.565120000000000e+02
+1.293380000000000e+03 +4.313120000000000e+02
+1.949470000000000e+02 +2.689500000000000e+02
+1.758420000000000e+02 +2.655390000000000e+02
+1.210950000000000e+02 +2.435940000000000e+02
+7.869340000000000e+02 +2.284160000000000e+02
+9.058010000000000e+02 +2.659010000000000e+02
+7.415160000000002e+02 +2.219550000000000e+02
+1.192170000000000e+02 +2.325010000000000e+02
+8.917910000000001e+02 +3.405440000000001e+02
+2.869520000000000e+02 +1.427710000000000e+02
+6.538520000000000e+02 +2.451460000000000e+02
+9.809070000000000e+01 +2.518900000000000e+02
+1.156440000000000e+02 +2.398900000000000e+02
+1.464060000000000e+03 +4.604190000000000e+02
+1.660750000000000e+02 +2.583930000000000e+02
+2.644780000000000e+02 +1.301000000000000e+02
+6.614820000000000e+02 +2.433040000000000e+02
+3.726520000000000e-01 +2.140600000000000e+02
+1.757470000000000e+02 +2.623350000000000e+02
+3.817300000000000e+00 +2.170350000000000e+02
+1.212950000000000e+02 +2.290880000000000e+02
+4.802330000000000e-01 +2.126010000000000e+02
+6.642790000000000e+02 +2.499520000000000e+02
+7.800039999999998e+02 +2.618830000000000e+02
+1.278010000000000e+03 +4.642680000000000e+02
+8.992210000000000e+02 +2.901910000000000e+02
+5.788160000000000e+02 +3.642800000000000e+02
+4.795870000000000e+02 +1.350230000000000e+02
+7.746520000000000e-01 +2.111410000000000e+02
+8.646169999999999e+01 +2.304090000000000e+02
+2.299450000000000e-01 +2.111410000000000e+02
+8.534200000000000e+02 +3.859450000000000e+02
+7.421660000000001e+02 +2.195440000000000e+02
+7.869800000000000e+02 +2.455830000000000e+02
+4.564940000000000e+02 +2.941650000000000e+02
+1.547230000000000e+02 +2.517070000000000e+02
+1.073540000000000e+03 +3.653800000000000e+02
+7.122830000000000e-01 +2.094170000000000e+02
+1.277260000000000e+02 +2.474020000000000e+02
+1.215420000000000e+00 +2.104750000000000e+02
+1.024580000000000e+02 +2.293660000000000e+02
+6.736039999999998e+02 +2.618250000000000e+02
+8.932400000000000e+02 +2.671700000000000e+02
+8.893500000000000e+02 +3.250610000000000e+02
+8.723030000000000e+01 +2.287550000000000e+02
+4.201530000000000e+02 +1.976850000000000e+02
+8.488310000000000e+02 +2.669690000000000e+02
+6.647480000000000e+02 +2.752240000000000e+02
+7.673800000000000e+02 +2.204680000000000e+02
+1.030760000000000e+03 +4.066360000000000e+02
+6.756160000000001e+02 +3.562930000000000e+02
+5.801120000000000e+02 +3.266910000000000e+02
+1.449510000000000e+02 +2.472530000000000e+02
+7.418170000000000e+02 +2.230180000000000e+02
+1.591220000000000e+03 +4.456670000000000e+02
+6.779980000000000e+02 +3.520440000000001e+02
+7.336450000000000e+02 +3.487270000000000e+02
+2.365970000000000e+02 +2.615550000000000e+02
+9.364400000000001e+01 +2.211740000000000e+02
+9.781470000000000e+02 +3.365530000000001e+02
+8.123020000000000e+01 +2.290640000000000e+02
+7.124410000000000e+02 +3.849550000000000e+02
+1.077940000000000e+02 +2.325830000000000e+02
+9.199030000000000e+02 +2.834270000000000e+02
+6.192919999999998e+02 +1.526690000000000e+02
+7.805180000000000e+02 +2.267960000000000e+02
+1.120170000000000e+03 +4.125650000000000e+02
+1.782460000000000e-01 +2.043300000000000e+02
+8.339900000000000e+02 +2.383600000000000e+02
+9.038869999999999e+02 +2.745080000000000e+02
+1.121660000000000e+03 +3.796920000000000e+02
+4.331260000000000e+02 +2.875900000000000e+02
+7.056630000000000e+02 +3.623440000000000e+02
+1.699030000000000e-02 +2.028700000000000e+02
+3.340490000000001e+02 +1.315070000000000e+02
+9.885060000000000e+02 +3.297050000000000e+02
+1.389140000000000e+03 +4.440510000000000e+02
+9.011820000000000e+01 +2.275570000000000e+02
+7.751560000000002e+02 +2.475740000000000e+02
+1.152510000000000e+03 +4.393060000000000e+02
+4.272970000000000e+02 +3.052380000000000e+02
+8.250959999999999e-01 +2.014110000000000e+02
+7.794680000000002e+02 +2.272520000000000e+02
+1.098520000000000e+03 +3.851450000000000e+02
+2.346770000000000e+02 +2.579860000000000e+02
+7.809570000000000e+02 +2.477010000000000e+02
+8.911239999999998e+02 +2.432590000000000e+02
+6.674570000000000e+02 +3.667600000000000e+02
+4.033340000000000e+02 +2.985870000000000e+02
+9.881710000000000e+02 +3.493280000000001e+02
+1.875300000000000e+02 +2.601800000000000e+02
+1.579650000000000e+03 +4.392260000000000e+02
+9.032350000000000e+02 +2.601360000000000e+02
+1.614280000000000e+02 +2.228570000000000e+02
+1.266300000000000e+03 +4.336340000000000e+02
+6.786680000000000e+01 +2.210750000000000e+02
+6.612120000000000e+02 +1.608920000000000e+02
+2.258760000000000e+01 +2.170630000000000e+02
+7.803839999999999e+02 +2.259240000000000e+02
+1.054280000000000e+03 +4.628290000000000e+02
+2.347060000000000e+00 +1.989780000000000e+02
+1.571290000000000e+02 +2.406600000000000e+02
+6.600470000000000e+02 +2.475890000000000e+02
+1.300720000000000e+03 +4.496330000000000e+02
+7.040980000000002e+02 +3.425130000000001e+02
+1.271820000000000e+00 +1.984920000000000e+02
+1.289570000000000e+02 +2.440500000000000e+02
+3.479330000000000e+01 +2.162690000000000e+02
+9.736609999999999e+02 +3.822410000000000e+02
+7.482220000000000e+02 +2.148330000000000e+02
+1.300140000000000e+03 +4.332440000000000e+02
+4.855970000000000e+02 +3.178970000000000e+02
+9.852089999999999e+02 +3.791910000000000e+02
+6.147680000000000e+02 +2.268570000000000e+02
+1.300780000000000e+03 +4.466020000000000e+02
+7.667980000000000e+02 +3.483470000000000e+02
+6.607189999999998e+02 +2.401540000000000e+02
+3.369800000000000e+02 +1.302500000000000e+02
+8.515650000000001e+02 +2.334020000000000e+02
+4.738400000000000e+02 +2.989420000000000e+02
+1.309790000000000e+03 +4.281740000000000e+02
+1.321030000000000e+03 +4.505410000000000e+02
+1.980450000000000e+02 +2.396710000000000e+02
+7.782750000000000e+02 +2.159000000000000e+02
+8.631710000000000e+02 +3.622370000000000e+02
+5.374939999999999e-01 +1.946000000000000e+02
+2.084510000000000e+02 +2.262700000000000e+02
+7.897689999999999e+02 +2.411020000000000e+02
+1.315670000000000e+03 +4.593970000000000e+02
+9.017400000000000e+02 +2.748070000000000e+02
+8.495239999999999e+02 +3.654840000000000e+02
+8.917180000000002e+02 +2.482350000000000e+02
+6.632869999999998e+02 +2.320280000000000e+02
+1.111400000000000e+03 +3.681000000000000e+02
+1.798950000000000e+02 +2.371730000000000e+02
+7.481130000000001e+02 +2.092120000000000e+02
+6.554260000000000e+02 +3.231950000000000e+02
+1.304120000000000e+03 +4.565310000000000e+02
+8.795610000000001e+01 +2.065930000000000e+02
+7.292589999999999e+02 +3.409360000000000e+02
+6.219030000000000e+02 +2.603810000000000e+02
+1.069340000000000e+03 +3.793710000000000e+02
+5.658840000000000e+02 +3.022630000000000e+02
+7.404050000000000e+02 +3.340480000000000e+02
+9.531530000000000e+02 +2.679610000000000e+02
+8.662569999999999e+02 +2.731350000000000e+02
+8.618339999999999e+02 +3.374530000000001e+02
+5.965040000000000e+01 +2.071670000000000e+02
+6.241080000000002e+02 +2.701300000000000e+02
+8.881849999999999e+02 +4.126130000000001e+02
+9.350539999999999e+01 +2.051070000000000e+02
+1.446630000000000e+03 +4.765620000000000e+02
+2.817120000000000e+02 +1.066900000000000e+02
+6.482989999999999e+01 +2.017770000000000e+02
+8.901840000000000e+02 +2.434870000000000e+02
+3.132250000000000e+02 +2.756340000000000e+02
+6.243250000000000e+02 +3.490290000000000e+02
+5.375100000000000e-01 +1.882750000000000e+02
+2.008560000000000e+02 +2.293300000000000e+02
+2.772760000000000e+02 +2.189350000000000e+02
+6.489610000000000e+01 +2.050550000000000e+02
+9.312380000000000e+01 +2.017540000000000e+02
+9.776310000000000e+02 +3.159300000000000e+02
+1.069740000000000e+02 +2.236390000000000e+02
+7.476010000000001e+02 +3.215460000000000e+02
+6.981770000000000e+01 +1.982100000000000e+02
+9.070410000000001e+02 +2.599830000000000e+02
+7.989409999999999e+01 +2.105480000000000e+02
+6.745959999999999e+01 +1.968450000000000e+02
+1.457980000000000e+03 +4.323170000000000e+02
+1.060270000000000e+03 +3.361830000000000e+02
+6.839689999999998e+02 +3.604220000000000e+02
+6.789470000000000e+01 +1.969100000000000e+02
+1.405250000000000e+03 +4.587480000000001e+02
+4.747260000000000e+02 +2.956420000000000e+02
+7.656170000000000e+02 +2.154770000000000e+02
+8.368180000000000e+02 +2.374940000000000e+02
+6.359740000000000e+02 +2.291000000000000e+02
+7.866030000000002e+02 +2.189740000000000e+02
+8.477819999999998e+02 +3.665630000000001e+02
+9.097080000000000e+02 +2.728980000000000e+02
+1.914570000000000e+02 +2.224070000000000e+02
+1.484200000000000e+02 +2.202420000000000e+02
+8.315140000000000e+01 +1.988360000000000e+02
+7.722130000000002e+02 +3.335230000000000e+02
+8.357060000000000e+02 +3.362720000000000e+02
+8.447810000000002e+02 +3.350830000000000e+02
+2.604900000000000e+00 +1.858480000000000e+02
+6.624730000000002e+02 +2.343150000000000e+02
+1.295810000000000e+03 +4.391930000000000e+02
+9.114220000000000e+02 +2.477450000000000e+02
+9.921710000000000e+02 +3.055660000000000e+02
+8.634910000000001e+02 +2.345280000000000e+02
+6.013310000000000e+01 +1.991060000000000e+02
+7.856990000000000e+02 +3.223480000000000e+02
+9.769210000000000e+02 +3.103940000000000e+02
+8.540089999999999e+02 +3.359530000000001e+02
+4.619960000000000e+02 +2.750760000000000e+02
+6.222030000000000e+02 +2.578960000000000e+02
+1.135050000000000e+03 +3.748450000000000e+02
+6.659230000000000e+02 +3.142410000000000e+02
+8.418880000000000e+02 +3.179160000000000e+02
+8.113270000000000e+01 +1.908530000000000e+02
+6.625570000000000e+02 +2.772080000000000e+02
+7.847600000000000e+02 +2.155580000000000e+02
+1.445630000000000e+03 +4.514980000000001e+02
+7.531710000000000e+02 +1.952010000000000e+02
+8.594090000000000e+02 +3.358190000000000e+02
+5.493950000000000e+01 +1.945170000000000e+02
+2.907750000000000e-01 +1.809780000000000e+02
+7.312470000000000e+01 +1.929540000000000e+02
+1.048160000000000e+03 +3.651390000000000e+02
+2.926440000000000e+02 +2.558700000000000e+02
+8.236940000000000e+01 +1.925630000000000e+02
+7.860680000000000e+02 +2.213170000000000e+02
+7.924530000000000e+02 +3.198300000000000e+02
+1.310050000000000e+03 +4.266190000000000e+02
+1.615980000000000e+02 +2.187350000000000e+02
+6.640760000000000e+02 +1.925960000000000e+02
+1.078740000000000e+03 +3.985500000000000e+02
+1.674760000000000e-01 +1.790320000000000e+02
+1.104880000000000e+03 +3.616290000000000e+02
+1.385880000000000e+02 +2.151420000000000e+02
+8.131540000000000e+01 +2.072790000000000e+02
+3.976350000000000e+01 +1.833470000000000e+02
+7.976049999999999e+01 +1.940710000000000e+02
+7.500570000000000e+02 +2.093630000000000e+02
+8.374600000000000e+02 +3.059350000000000e+02
+9.369550000000000e+02 +3.582280000000000e+02
+1.177650000000000e+03 +3.946550000000000e+02
+5.727420000000000e+02 +2.791430000000000e+02
+9.480950000000000e+02 +3.644600000000000e+02
+7.155070000000001e+01 +2.029360000000000e+02
+5.182150000000000e+02 +3.176700000000000e+02
+5.498360000000000e-01 +1.761130000000000e+02
+7.509320000000000e+01 +1.922860000000000e+02
+1.848850000000000e+02 +2.253510000000000e+02
+6.108990000000000e+02 +2.945040000000000e+02
+2.014980000000000e+02 +2.310510000000000e+02
+1.066230000000000e+03 +3.271670000000001e+02
+4.206050000000000e+02 +1.846130000000000e+02
+1.318070000000000e+03 +4.117770000000000e+02
+7.582200000000000e+02 +3.274910000000000e+02
+2.474070000000000e+02 +2.158560000000000e+02
+7.490039999999998e+02 +2.961130000000000e+02
+1.163710000000000e+03 +3.563080000000000e+02
+2.483240000000000e+02 +2.215650000000000e+02
+9.054080000000001e+00 +1.836650000000000e+02
+1.005280000000000e+03 +3.426180000000001e+02
+7.427739999999999e+02 +3.009910000000000e+02
+6.765390000000000e+02 +2.240830000000000e+02
+1.169650000000000e+03 +3.536319999999999e+02
+2.444960000000000e+02 +2.206150000000000e+02
+4.271340000000000e+00 +1.746540000000000e+02
+6.408320000000000e+02 +2.297800000000000e+02
+1.295560000000000e+03 +4.164300000000000e+02
+9.352120000000000e+02 +3.867150000000000e+02
+9.072960000000000e+02 +2.405020000000000e+02
+3.605810000000000e+01 +1.748650000000000e+02
+7.747089999999999e+02 +3.035170000000000e+02
+1.109040000000000e+03 +3.580180000000001e+02
+1.156650000000000e+03 +3.869420000000000e+02
+8.994760000000001e+02 +2.545410000000000e+02
+3.484880000000000e-01 +1.727080000000000e+02
+1.841490000000000e+02 +2.090610000000000e+02
+5.988290000000000e+01 +1.836560000000000e+02
+1.083790000000000e+03 +3.608610000000000e+02
+8.661610000000000e+01 +1.875920000000000e+02
+1.074650000000000e+03 +3.682570000000000e+02
+7.812810000000002e+02 +3.016770000000000e+02
+9.730380000000000e+02 +3.339610000000000e+02
+6.473950000000000e+02 +2.289740000000000e+02
+5.680140000000000e+02 +2.644560000000000e+02
+6.244710000000000e+02 +2.104940000000000e+02
+5.874780000000000e+01 +1.837820000000000e+02
+6.618240000000000e+02 +2.587570000000000e+02
+8.630030000000000e+02 +3.710440000000000e+02
+4.414420000000000e+02 +2.686800000000000e+02
+8.830790000000000e+02 +2.505160000000000e+02
+5.012080000000000e+02 +1.829680000000000e+02
+7.392440000000001e+01 +1.798140000000000e+02
+3.130090000000000e+02 +2.580460000000000e+02
+7.820800000000000e+02 +2.979010000000000e+02
+1.646710000000000e+01 +1.714380000000000e+02
+8.972040000000000e+02 +3.216550000000000e+02
+8.596060000000001e+02 +3.128170000000000e+02
+4.074930000000000e+01 +1.702830000000000e+02
+1.740430000000000e+01 +1.716920000000000e+02
+1.803420000000000e+02 +2.002920000000000e+02
+9.787600000000000e+02 +2.715370000000000e+02
+8.792210000000000e+02 +2.444620000000000e+02
+2.506550000000000e+00 +1.667480000000000e+02
+6.419100000000000e+02 +2.983750000000000e+02
+1.270880000000000e+02 +1.980810000000000e+02
+5.313170000000000e+02 +2.535230000000000e+02
+3.257150000000000e+02 +2.216250000000000e+02
+3.320530000000001e+02 +2.336410000000000e+02
+7.107010000000000e+02 +2.760550000000000e+02
+1.066680000000000e+03 +3.431580000000000e+02
+1.082400000000000e+03 +3.538440000000000e+02
+5.915050000000000e+01 +1.702480000000000e+02
+7.847120000000000e+02 +2.932720000000000e+02
+9.844640000000000e-01 +1.633550000000000e+02
+5.878990000000000e+02 +2.126270000000000e+02
+8.290010000000001e+00 +1.646620000000000e+02
+1.056750000000000e+03 +3.297690000000000e+02
+6.741469999999998e+02 +2.912440000000000e+02
+7.797850000000000e+02 +2.874680000000000e+02
+6.378810000000000e+02 +2.091920000000000e+02
+7.841330000000000e+02 +2.802400000000000e+02
+3.352020000000000e+02 +8.601120000000000e+01
+1.079640000000000e+03 +3.303800000000000e+02
+8.883930000000000e+02 +3.057440000000000e+02
+1.284200000000000e+03 +3.909020000000000e+02
+1.971890000000000e+02 +2.159880000000000e+02
+6.594090000000000e+02 +2.191230000000000e+02
+8.948950000000000e+02 +3.145520000000000e+02
+1.323730000000000e+02 +1.980380000000000e+02
+1.129020000000000e+02 +1.989150000000000e+02
+7.426610000000002e+02 +1.898050000000000e+02
+3.840870000000000e+01 +1.639110000000000e+02
+1.315250000000000e-01 +1.600590000000000e+02
+5.132830000000000e+01 +1.794370000000000e+02
+2.793690000000000e+02 +7.237990000000001e+01
+4.627680000000000e+01 +1.647360000000000e+02
+9.753780000000000e+02 +2.863040000000001e+02
+9.038520000000000e+02 +3.078490000000000e+02
+3.263350000000000e+02 +2.222560000000000e+02
+1.070210000000000e+03 +3.073190000000000e+02
+3.573260000000000e+02 +2.439800000000000e+02
+6.055840000000002e+02 +2.699600000000000e+02
+3.135530000000000e+02 +2.240410000000000e+02
+3.349470000000000e+00 +1.602340000000000e+02
+1.125860000000000e+03 +3.631320000000000e+02
+8.577310000000001e+02 +2.994000000000000e+02
+5.200660000000000e+02 +2.534720000000000e+02
+7.876790000000000e+02 +2.899750000000000e+02
+7.831369999999999e+02 +2.941620000000001e+02
+1.077590000000000e+03 +3.064010000000000e+02
+1.423970000000000e+02 +2.026790000000000e+02
+5.562950000000000e+02 +2.541250000000000e+02
+6.236330000000000e+02 +2.707560000000000e+02
+4.819400000000000e+01 +1.611740000000000e+02
+7.852830000000000e+02 +2.960240000000000e+02
+1.604550000000000e+02 +1.972010000000000e+02
+2.884230000000000e+01 +1.608550000000000e+02
+1.122600000000000e+03 +3.248230000000000e+02
+5.675980000000002e+02 +2.781360000000000e+02
+1.932750000000000e+02 +1.917070000000000e+02
+7.846020000000000e+02 +2.875420000000001e+02
+6.199109999999999e+02 +2.720840000000000e+02
+2.764090000000000e+01 +1.662170000000000e+02
+8.789500000000000e+02 +2.714700000000000e+02
+1.197360000000000e-01 +1.527610000000000e+02
+7.867619999999999e+02 +2.762050000000000e+02
+5.194180000000000e+02 +2.419410000000000e+02
+4.219140000000000e+02 +2.426350000000000e+02
+9.791400000000000e+02 +2.582690000000000e+02
+8.242310000000001e+02 +2.926970000000000e+02
+1.065860000000000e+02 +1.844540000000000e+02
+1.077900000000000e+03 +3.357569999999999e+02
+9.539560000000000e+01 +1.812600000000000e+02
+6.189360000000000e+02 +2.311460000000000e+02
+3.339070000000000e+02 +7.781840000000000e+01
+6.513099999999999e+02 +2.964850000000000e+02
+1.701220000000000e+01 +1.548550000000000e+02
+7.881310000000002e+02 +2.883400000000000e+02
+6.993890000000000e+00 +1.584330000000000e+02
+2.889200000000000e+01 +1.541510000000000e+02
+9.753190000000000e+02 +2.795000000000000e+02
+8.152589999999999e+02 +3.023290000000000e+02
+3.284790000000000e+01 +1.609140000000000e+02
+8.522430000000000e+00 +1.577750000000000e+02
+1.073610000000000e+02 +1.777270000000000e+02
+3.871840000000000e+01 +1.494650000000000e+02
+1.859290000000000e+02 +1.840930000000000e+02
+9.706440000000000e+02 +3.139610000000000e+02
+7.819520000000000e+02 +2.795030000000000e+02
+2.787460000000000e+02 +2.013510000000000e+02
+1.390720000000000e+02 +1.754900000000000e+02
+1.902150000000000e+02 +1.818060000000000e+02
+4.456290000000000e+01 +1.566860000000000e+02
+3.508100000000000e+01 +1.512990000000000e+02
+3.357590000000000e+02 +7.992210000000000e+01
+8.917000000000000e+02 +2.963460000000000e+02
+1.371220000000000e+00 +1.469870000000000e+02
+6.183320000000000e+01 +1.562350000000000e+02
+7.864730000000002e+02 +2.764970000000000e+02
+3.536050000000000e+01 +1.533100000000000e+02
+8.210319999999998e+02 +2.763850000000000e+02
+1.140380000000000e+02 +1.773020000000000e+02
+5.135080000000000e+02 +2.362290000000000e+02
+8.617000000000000e+02 +2.960650000000000e+02
+7.883270000000000e+02 +2.933180000000000e+02
+1.332860000000000e+02 +1.672190000000001e+02
+9.743270000000000e+02 +2.728170000000000e+02
+5.915990000000000e+02 +2.392920000000000e+02
+3.039230000000000e+01 +1.503580000000000e+02
+8.008270000000000e+02 +2.956310000000000e+02
+9.767400000000000e+02 +2.717690000000000e+02
+2.840670000000000e+01 +1.461600000000000e+02
+4.950460000000000e+02 +1.475230000000000e+02
+9.045560000000000e+02 +2.899120000000001e+02
+8.011510000000002e+02 +2.932720000000000e+02
+1.278260000000000e+03 +3.680120000000000e+02
+6.192360000000000e+02 +2.623320000000000e+02
+7.828250000000000e+02 +2.739370000000000e+02
+8.311810000000000e+02 +3.239100000000000e+02
+8.108869999999999e+02 +2.767900000000000e+02
+3.880470000000000e+01 +1.430600000000000e+02
+9.258150000000001e+02 +3.059800000000000e+02
+1.551730000000000e+02 +1.870450000000000e+02
+6.411410000000000e+02 +2.322230000000000e+02
+5.344890000000000e+02 +2.327170000000000e+02
+3.005870000000000e+01 +1.428110000000000e+02
+7.941189999999998e+02 +2.792420000000000e+02
+7.834360000000000e+02 +2.870140000000000e+02
+9.837809999999999e+02 +2.859580000000000e+02
+1.701600000000000e+02 +1.861570000000000e+02
+9.106210000000000e+02 +2.772480000000000e+02
+6.453270000000000e+02 +2.553090000000000e+02
+1.040270000000000e+03 +3.563800000000000e+02
+5.993840000000000e+02 +2.034970000000000e+02
+3.317570000000000e+01 +1.438270000000000e+02
+8.715650000000001e+02 +3.309590000000000e+02
+5.137919999999998e+02 +2.371580000000000e+02
+7.905410000000001e+02 +2.742580000000000e+02
+3.685750000000000e+02 +2.704610000000000e+02
+1.255500000000000e+03 +3.475490000000001e+02
+9.585980000000000e+02 +3.611960000000000e+02
+8.877439999999998e+02 +3.572410000000000e+02
+7.488450000000000e+02 +2.588880000000000e+02
+1.093460000000000e+03 +3.380010000000000e+02
+8.964360000000000e+02 +2.960960000000000e+02
+8.969620000000000e+02 +2.995700000000000e+02
+5.227370000000000e+02 +2.229210000000000e+02
+2.243580000000000e+01 +1.377670000000000e+02
+7.894040000000000e+02 +2.688500000000000e+02
+7.419400000000001e+02 +2.882580000000000e+02
+6.469650000000000e+02 +2.890030000000000e+02
+9.273250000000000e+02 +2.248730000000000e+02
+7.917330000000002e+02 +2.603170000000000e+02
+7.488860000000002e+02 +2.705920000000000e+02
+6.868519999999999e+01 +1.478160000000000e+02
+8.975590000000000e+02 +2.914300000000000e+02
+7.878980000000000e+02 +2.652380000000000e+02
+3.639280000000001e+02 +2.176210000000000e+02
+1.317550000000000e+02 +1.561540000000000e+02
+2.314520000000000e+01 +1.384200000000000e+02
+8.725500000000000e+02 +2.803000000000000e+02
+9.614299999999999e+02 +3.747660000000000e+02
+9.367890000000000e+02 +2.876560000000000e+02
+6.248470000000000e+02 +2.150180000000000e+02
+6.827420000000000e+02 +2.638580000000000e+02
+4.518720000000000e+02 +1.337750000000000e+02
+1.297180000000000e+02 +1.556080000000000e+02
+6.105250000000000e+01 +1.436280000000000e+02
+1.086200000000000e+02 +1.615810000000000e+02
+9.604000000000000e+02 +3.820080000000000e+02
+3.869330000000000e+01 +1.371190000000000e+02
+7.866439999999999e+02 +2.605410000000000e+02
+2.213950000000000e+01 +1.361380000000000e+02
+7.936630000000000e+02 +2.599190000000000e+02
+6.450960000000000e+02 +2.320450000000000e+02
+9.995599999999999e+02 +2.411830000000000e+02
+2.062390000000000e+00 +1.295200000000000e+02
+1.834280000000000e+01 +1.320540000000000e+02
+5.131360000000000e+02 +2.297020000000000e+02
+9.931760000000000e+01 +1.499390000000000e+02
+8.007719999999998e+02 +2.672920000000000e+02
+6.245540000000000e+02 +2.108640000000000e+02
+1.037430000000000e+03 +3.077370000000000e+02
+8.971020000000000e+02 +2.822930000000000e+02
+1.363750000000000e+02 +1.703930000000000e+02
+1.939850000000000e+01 +1.350770000000000e+02
+1.152620000000000e+02 +1.507550000000000e+02
+1.091870000000000e+03 +2.927470000000000e+02
+6.158170000000000e+02 +2.308870000000000e+02
+7.718539999999998e+02 +2.538640000000000e+02
+2.677950000000000e+02 +1.794530000000000e+02
+1.751660000000000e+01 +1.298710000000000e+02
+2.814530000000000e+02 +1.494340000000000e+02
+5.076670000000000e+02 +2.256500000000000e+02
+1.101270000000000e+03 +3.497360000000000e+02
+2.085190000000000e+01 +1.259890000000000e+02
+8.927780000000000e+02 +2.972840000000000e+02
+4.797780000000000e+01 +1.370490000000000e+02
+3.393090000000000e+00 +1.259150000000000e+02
+7.792189999999998e+02 +2.429030000000000e+02
+6.602209999999999e+01 +1.362150000000000e+02
+6.211300000000000e+02 +2.105010000000000e+02
+5.062940000000000e+02 +2.112300000000000e+02
+1.403950000000000e+02 +1.534560000000000e+02
+8.591039999999998e+02 +3.304560000000000e+02
+4.711390000000000e+02 +2.240900000000000e+02
+7.673200000000001e+02 +2.411250000000000e+02
+1.174800000000000e+01 +1.285820000000000e+02
+7.691590000000000e+00 +1.230280000000000e+02
+7.794100000000000e+02 +2.362690000000000e+02
+1.225710000000000e+02 +1.410270000000000e+02
+4.984950000000000e+02 +2.112070000000000e+02
+7.755230000000000e+02 +2.526170000000000e+02
+1.095020000000000e+03 +3.155870000000000e+02
+2.772820000000000e+02 +1.501040000000000e+02
+1.086090000000000e+03 +3.160400000000000e+02
+8.802800000000000e+00 +1.269060000000000e+02
+1.113270000000000e+03 +3.084280000000000e+02
+1.009610000000000e+02 +1.459870000000000e+02
+7.705319999999998e+02 +2.615550000000000e+02
+3.172640000000000e+02 +2.051470000000000e+02
+1.129870000000000e+01 +1.225070000000000e+02
+5.545570000000000e+02 +2.449800000000000e+02
+7.072100000000000e+02 +2.625060000000000e+02
+6.384590000000002e+02 +2.442830000000000e+02
+1.151420000000000e+02 +1.440960000000000e+02
+4.872980000000000e+02 +1.637360000000000e+02
+9.872300000000000e+02 +2.803060000000000e+02
+8.878030000000000e+02 +2.914700000000000e+02
+6.469620000000000e+00 +1.225930000000000e+02
+4.146380000000000e+02 +2.050950000000000e+02
+8.984480000000000e+02 +2.069960000000000e+02
+7.467050000000000e+02 +2.369990000000000e+02
+4.491590000000000e+02 +2.073930000000000e+02
+5.720210000000000e+00 +1.220320000000000e+02
+8.352509999999999e+01 +1.473650000000000e+02
+4.908670000000000e+02 +2.161240000000000e+02
+8.472139999999998e+02 +2.549630000000000e+02
+8.559770000000000e+01 +1.367670000000000e+02
+9.054170000000001e+01 +1.404820000000000e+02
+5.912490000000000e+02 +2.227290000000000e+02
+6.841000000000000e+02 +2.533370000000000e+02
+6.246230000000000e+02 +2.455630000000000e+02
+4.565240000000000e+02 +1.232760000000000e+02
+1.769520000000000e+01 +1.178810000000000e+02
+7.682700000000000e+02 +2.429970000000000e+02
+4.484270000000000e+02 +2.221520000000000e+02
+9.203350000000000e+02 +3.022650000000000e+02
+6.153030000000000e+02 +2.419590000000000e+02
+9.070660000000000e+02 +2.632770000000000e+02
+2.586830000000000e+00 +1.161740000000000e+02
+7.761400000000000e+02 +2.329400000000000e+02
+3.740220000000000e+00 +1.129280000000000e+02
+7.772970000000000e+02 +2.288440000000000e+02
+1.404280000000000e+01 +1.159880000000000e+02
+3.340410000000000e+02 +1.754410000000000e+02
+8.815280000000000e+02 +2.729440000000000e+02
+2.349260000000000e+00 +1.128420000000000e+02
+5.132520000000000e+02 +2.219300000000000e+02
+9.083620000000000e+02 +2.701030000000000e+02
+2.010000000000000e+01 +1.140530000000000e+02
+7.729639999999998e+02 +2.396620000000000e+02
+9.760230000000000e+02 +2.823530000000000e+02
+2.755640000000000e+02 +1.451340000000000e+02
+4.348900000000000e+02 +2.000190000000000e+02
+3.868060000000000e+02 +1.942210000000000e+02
+9.785860000000000e+02 +2.843780000000000e+02
+8.499830000000002e+02 +2.988410000000000e+02
+2.540950000000000e+02 +1.807900000000000e+02
+6.584110000000002e+02 +1.945850000000000e+02
+7.827200000000000e+02 +2.370240000000000e+02
+1.253040000000000e+02 +1.310610000000000e+02
+4.407890000000000e+02 +2.018250000000000e+02
+6.168840000000000e-01 +1.085660000000000e+02
+3.424850000000000e+00 +1.109390000000000e+02
+1.384700000000000e+01 +1.128040000000000e+02
+4.398880000000000e+02 +1.958300000000000e+02
+6.181830000000000e+02 +2.375880000000000e+02
+9.028049999999999e+02 +3.027130000000000e+02
+4.370440000000000e+02 +1.699560000000000e+02
+6.181600000000000e+02 +1.903980000000000e+02
+7.828010000000000e+02 +2.423000000000000e+02
+4.525960000000000e+02 +2.062530000000000e+02
+5.442619999999999e+02 +2.156930000000000e+02
+7.457850000000001e+01 +1.138790000000000e+02
+6.716880000000000e+02 +2.327130000000000e+02
+3.342720000000000e+02 +1.973210000000000e+02
+9.265060000000000e+02 +2.613180000000000e+02
+1.162170000000000e+02 +1.256470000000000e+02
+7.694010000000002e+02 +2.362870000000000e+02
+3.339870000000000e+02 +1.607430000000000e+02
+6.362700000000000e+02 +2.268900000000000e+02
+1.245290000000000e+02 +1.216090000000000e+02
+5.867500000000000e+01 +1.148030000000000e+02
+2.433980000000000e+02 +1.625310000000000e+02
+2.510230000000000e+01 +1.084270000000000e+02
+4.976210000000000e+02 +1.957020000000000e+02
+8.444370000000000e+02 +2.551730000000000e+02
+6.716760000000000e+02 +2.343140000000000e+02
+2.893740000000000e+01 +1.107520000000000e+02
+6.624360000000000e+02 +2.434160000000000e+02
+8.614589999999999e+02 +2.497730000000000e+02
+6.026820000000000e+02 +2.161850000000000e+02
+2.772920000000000e+02 +1.404760000000000e+02
+6.630580000000000e+02 +1.839500000000000e+02
+7.717220000000000e+02 +2.304090000000000e+02
+4.334980000000001e+02 +1.824680000000000e+02
+8.296050000000000e+02 +3.032450000000000e+02
+1.990320000000000e+01 +1.062870000000000e+02
+6.231680000000000e+02 +1.715870000000000e+02
+4.787060000000000e+01 +1.075900000000000e+02
+7.530250000000000e+02 +2.341110000000000e+02
+5.374750000000000e-01 +1.002190000000000e+02
+7.814490000000000e+02 +2.420750000000000e+02
+4.230920000000000e+02 +1.686000000000000e+02
+7.136180000000001e+02 +2.029510000000000e+02
+9.315260000000001e+01 +1.235510000000000e+02
+4.154450000000000e+02 +1.844890000000000e+02
+6.171849999999999e+02 +2.068000000000000e+02
+4.977590000000000e+01 +1.053320000000000e+02
+5.475980000000002e+02 +2.027210000000000e+02
+1.934610000000000e+02 +1.532210000000000e+02
+4.593690000000000e+01 +1.044940000000000e+02
+7.870660000000000e+02 +2.419250000000000e+02
+5.965910000000000e+02 +2.067870000000000e+02
+6.220190000000000e+02 +1.650420000000000e+02
+5.876300000000000e+02 +2.166390000000000e+02
+1.093250000000000e+03 +3.090290000000000e+02
+6.369750000000000e+02 +1.768090000000000e+02
+3.973210000000000e+02 +1.712940000000000e+02
+5.779290000000000e+02 +2.111020000000000e+02
+7.418250000000000e+02 +2.222580000000000e+02
+7.420210000000002e+02 +2.095660000000000e+02
+8.875910000000000e+02 +2.655270000000000e+02
+8.528370000000000e+02 +2.711520000000000e+02
+3.178400000000000e+02 +1.600780000000000e+02
+7.358539999999998e+02 +2.056850000000000e+02
+3.760460000000000e+01 +1.012330000000000e+02
+7.191230000000000e+02 +2.212060000000000e+02
+5.471559999999999e+02 +1.857610000000000e+02
+1.263960000000000e+02 +1.262210000000000e+02
+6.982840000000000e+02 +2.188310000000000e+02
+3.572550000000000e+02 +1.820120000000000e+02
+9.243450000000000e+01 +1.218730000000000e+02
+6.934120000000000e+01 +1.177220000000000e+02
+5.187230000000002e+02 +2.100590000000000e+02
+8.449970000000000e+02 +2.433040000000000e+02
+5.631220000000000e+01 +1.118690000000000e+02
+7.135120000000001e+01 +1.154850000000000e+02
+3.278820000000000e+01 +1.001640000000000e+02
+7.714730000000002e+02 +2.401740000000000e+02
+4.872820000000000e+02 +1.853670000000000e+02
+1.120190000000000e+03 +3.059650000000000e+02
+5.783800000000000e+02 +1.931230000000000e+02
+1.120810000000000e+01 +9.844970000000001e+01
+1.209440000000000e+02 +1.176900000000000e+02
+6.190820000000000e+01 +9.942659999999999e+01
+7.439040000000000e+02 +2.331560000000000e+02
+4.223750000000000e+02 +1.822760000000000e+02
+5.802859999999999e+02 +2.023570000000000e+02
+9.054480000000000e+02 +2.669590000000000e+02
+3.648210000000000e+02 +1.679530000000000e+02
+4.245830000000000e+02 +2.143300000000000e+02
+4.118250000000000e+02 +1.855780000000000e+02
+5.740180000000000e+02 +2.082300000000000e+02
+3.278810000000000e+01 +9.963690000000000e+01
+1.074350000000000e+02 +1.148420000000000e+02
+7.438420000000000e+02 +2.081920000000000e+02
+9.542820000000000e+02 +2.749810000000000e+02
+4.313740000000000e+02 +2.193750000000000e+02
+1.594580000000000e+02 +1.398120000000000e+02
+8.665770000000000e+02 +2.645670000000000e+02
+6.476580000000000e+02 +1.955010000000000e+02
+2.764920000000000e+02 +1.358380000000000e+02
+6.468290000000001e+00 +9.325910000000000e+01
+3.362770000000000e+02 +1.294430000000000e+02
+2.543070000000000e+02 +1.394940000000000e+02
+9.551600000000001e+01 +1.136480000000000e+02
+6.172790000000000e+01 +1.094040000000000e+02
+5.669069999999998e+02 +2.090360000000000e+02
+1.402890000000000e+02 +1.340260000000000e+02
+6.500480000000000e+02 +2.151250000000000e+02
+2.053000000000000e+01 +9.500830000000001e+01
+6.349960000000000e+02 +1.891060000000000e+02
+6.772830000000000e+01 +1.078120000000000e+02
+1.086700000000000e+02 +1.157770000000000e+02
+1.094820000000000e+02 +1.038520000000000e+02
+1.839990000000000e+01 +9.567980000000000e+01
+9.513650000000000e+01 +1.128640000000000e+02
+1.244530000000000e+02 +1.199350000000000e+02
+7.127030000000000e+02 +2.256830000000000e+02
+5.265090000000000e+02 +2.006030000000000e+02
+5.965880000000002e+02 +1.931220000000000e+02
+8.058869999999999e+02 +2.302120000000000e+02
+2.989360000000000e+00 +8.867410000000000e+01
+6.434580000000002e+02 +2.147070000000000e+02
+8.045930000000002e+02 +2.379030000000000e+02
+3.992860000000000e+02 +1.916800000000000e+02
+7.618460000000000e+01 +1.084710000000000e+02
+1.132400000000000e+02 +1.245380000000000e+02
+4.222170000000000e+02 +1.648410000000000e+02
+5.032200000000000e+02 +2.038130000000000e+02
+5.953750000000000e+02 +1.879800000000000e+02
+1.211960000000000e+02 +1.053140000000000e+02
+7.523950000000000e+02 +1.805730000000000e+02
+4.142730000000000e+02 +1.753850000000000e+02
+6.530219999999998e+02 +1.883720000000000e+02
+2.104250000000000e+01 +9.835150000000000e+01
+6.755480000000000e+01 +1.045420000000000e+02
+3.307410000000000e+01 +9.365980000000000e+01
+9.342779999999999e+00 +9.015850000000000e+01
+4.764010000000000e+02 +1.623660000000000e+02
+9.301410000000000e+01 +1.101460000000000e+02
+2.114770000000000e+02 +1.390880000000000e+02
+5.972030000000000e+01 +1.006010000000000e+02
+6.071040000000000e+02 +1.969610000000000e+02
+7.139439999999999e-01 +8.416450000000000e+01
+6.257590000000000e+02 +1.590750000000000e+02
+6.767919999999999e+01 +1.016710000000000e+02
+7.727870000000000e+01 +1.070790000000000e+02
+1.647680000000000e+01 +9.003540000000000e+01
+2.814910000000000e+02 +1.122090000000000e+02
+4.897200000000000e+02 +2.006620000000000e+02
+4.640230000000000e+02 +1.612340000000000e+02
+7.370910000000000e+02 +2.378000000000000e+02
+1.030290000000000e+02 +9.512460000000000e+01
+3.225490000000000e+01 +9.571410000000000e+01
+2.769150000000000e+01 +9.160020000000000e+01
+1.125670000000000e+02 +1.031730000000000e+02
+5.498569999999999e-01 +8.221850000000001e+01
+6.183740000000000e+02 +1.984400000000000e+02
+7.694850000000000e+01 +9.780610000000000e+01
+9.630980000000000e+01 +9.889570000000001e+01
+8.166900000000001e+02 +2.290720000000000e+02
+5.088520000000001e-01 +8.124550000000001e+01
+2.885800000000000e+02 +1.492580000000000e+02
+6.950409999999999e+01 +9.625539999999999e+01
+6.891130000000001e+02 +2.104380000000000e+02
+3.178430000000000e+00 +8.480990000000000e+01
+4.359060000000000e+02 +1.837610000000000e+02
+6.221760000000000e+01 +9.675810000000000e+01
+6.220020000000000e+01 +9.480180000000000e+01
+7.429800000000000e+02 +2.071080000000000e+02
+5.921120000000000e+02 +1.901040000000000e+02
+6.491540000000000e+02 +1.970920000000000e+02
+6.233410000000000e+02 +1.556820000000000e+02
+7.922480000000000e+02 +2.001400000000000e+02
+7.271590000000000e+02 +2.322930000000000e+02
+2.715850000000000e+02 +1.514140000000000e+02
+4.535400000000000e+01 +9.383369999999999e+01
+6.338980000000000e+01 +9.866390000000000e+01
+4.415870000000000e+00 +7.982980000000001e+01
+6.072160000000000e+02 +1.808890000000000e+02
+8.290000000000000e+02 +2.537010000000000e+02
+4.053240000000000e+01 +8.981420000000000e+01
+5.761780000000000e+02 +1.860550000000000e+02
+3.959010000000000e+02 +1.534160000000000e+02
+9.107780000000000e+02 +2.150240000000000e+02
+4.318930000000000e+01 +8.898240000000000e+01
+3.895060000000000e+02 +1.812830000000000e+02
+5.451420000000000e+01 +9.232150000000000e+01
+7.357420000000000e+02 +2.035410000000000e+02
+9.841310000000000e+01 +8.806390000000000e+01
+2.129540000000000e+00 +7.638050000000000e+01
+2.800780000000000e+01 +8.425910000000000e+01
+7.248470000000000e+02 +2.067320000000000e+02
+6.555130000000000e+02 +2.087670000000000e+02
+7.113869999999999e+02 +2.146470000000000e+02
+1.581600000000000e+00 +7.718650000000000e+01
+6.215050000000000e+02 +1.846200000000000e+02
+2.199700000000000e+00 +7.575500000000000e+01
+5.904320000000000e+02 +1.775100000000000e+02
+6.814190000000000e+02 +1.984820000000000e+02
+5.976860000000000e+02 +1.922300000000000e+02
+5.171390000000000e+02 +1.774440000000000e+02
+3.068920000000000e+01 +8.410390000000000e+01
+6.623830000000000e+02 +1.453200000000000e+02
+5.113490000000000e+02 +1.782110000000000e+02
+6.161120000000000e+02 +1.557290000000000e+02
+3.245580000000000e+02 +1.625460000000000e+02
+2.870990000000000e+01 +7.962850000000000e+01
+1.069640000000000e+03 +2.775730000000000e+02
+3.509120000000000e-01 +7.297499999999999e+01
+4.840770000000000e+02 +1.578360000000000e+02
+5.508480000000002e+02 +1.874400000000000e+02
+1.202650000000000e-01 +7.248850000000000e+01
+5.785450000000000e+02 +1.838730000000000e+02
+7.262089999999999e+02 +1.818100000000000e+02
+7.252350000000000e+02 +2.001350000000000e+02
+2.675600000000000e-01 +7.200200000000000e+01
+5.046700000000000e+02 +1.866650000000000e+02
+4.470980000000000e+02 +1.528060000000000e+02
+4.828710000000000e+02 +1.776700000000000e+02
+7.365060000000002e+02 +2.009150000000000e+02
+5.023490000000000e+01 +8.553480000000000e+01
+4.244990000000000e+01 +8.344840000000001e+01
+8.363810000000002e+02 +2.231860000000000e+02
+4.202510000000000e+01 +8.491119999999999e+01
+5.853140000000000e+00 +7.530560000000000e+01
+2.087730000000000e+00 +7.351000000000001e+01
+3.780920000000000e+02 +1.538060000000000e+02
+4.769920000000000e+02 +1.597250000000000e+02
+6.632209999999999e-01 +6.956950000000001e+01
+4.479970000000000e+02 +1.743240000000000e+02
+9.332500000000000e+01 +9.264420000000000e+01
+4.852370000000000e+02 +1.155080000000000e+02
+5.520840000000002e+02 +1.735760000000000e+02
+7.855870000000000e+02 +2.079360000000000e+02
+4.653050000000000e+02 +1.997920000000000e+02
+4.463600000000000e+01 +8.165580000000000e+01
+4.142800000000000e+02 +1.436460000000000e+02
+3.847710000000000e+02 +1.512330000000000e+02
+4.713590000000000e+02 +1.533000000000000e+02
+1.261420000000000e+02 +9.925190000000001e+01
+2.511090000000000e+02 +1.312460000000000e+02
+2.870650000000000e+02 +1.242710000000000e+02
+9.467500000000000e+01 +7.646150000000000e+01
+4.352740000000000e+02 +1.669790000000000e+02
+1.134350000000000e+02 +8.882580000000000e+01
+1.942380000000000e+02 +1.064580000000000e+02
+1.699830000000000e+02 +9.527570000000000e+01
+2.855830000000000e+01 +7.520710000000000e+01
+4.890730000000000e+02 +1.556310000000000e+02
+3.788840000000000e+00 +6.786520000000000e+01
+4.423750000000000e+02 +1.392770000000000e+02
+7.041710000000000e+02 +1.554900000000000e+02
+2.321090000000000e+02 +1.238350000000000e+02
+1.857590000000000e+02 +9.020010000000001e+01
+7.956319999999999e+01 +9.341119999999999e+01
+2.445210000000000e+02 +1.179050000000000e+02
+4.095270000000000e+02 +1.321840000000000e+02
+3.791050000000000e+02 +1.351280000000000e+02
+6.793730000000000e+02 +1.792760000000000e+02
+4.657100000000001e+01 +6.864140000000000e+01
+5.060950000000000e+02 +1.604360000000000e+02
+4.561320000000000e+02 +1.484380000000000e+02
+7.354820000000000e+02 +1.845070000000000e+02
+4.919950000000000e+02 +1.790320000000000e+02
+6.946780000000000e+02 +1.723060000000000e+02
+2.649730000000000e+02 +1.077890000000000e+02
+3.690110000000000e+01 +6.884780000000001e+01
+4.488700000000000e+02 +1.874640000000000e+02
+1.142570000000000e+02 +8.287179999999999e+01
+8.649070000000000e+01 +7.887820000000001e+01
+5.903660000000000e+02 +1.537590000000000e+02
+3.776490000000000e+02 +1.390400000000000e+02
+9.031540000000000e+01 +6.926000000000001e+01
+1.267320000000000e+02 +8.242149999999999e+01
+4.835580000000000e+02 +1.557460000000000e+02
+7.411900000000001e+02 +1.972160000000000e+02
+1.260530000000000e+02 +8.862150000000000e+01
+3.998450000000000e+02 +1.354370000000000e+02
+2.363270000000000e+02 +1.140260000000000e+02
+2.598980000000000e+02 +1.058160000000000e+02
+8.706900000000000e+01 +7.366589999999999e+01
+7.060810000000000e+02 +1.445190000000000e+02
+4.899180000000000e+02 +1.352220000000000e+02
+1.936170000000000e+02 +9.179250000000000e+01
+4.025810000000000e+02 +1.408250000000000e+02
+3.592410000000000e+02 +1.254510000000000e+02
+2.085920000000000e+02 +1.006800000000000e+02
+6.189800000000000e+02 +1.868530000000000e+02
+4.575130000000000e+02 +1.543440000000000e+02
+4.443620000000000e+02 +1.237350000000000e+02
+3.116520000000000e+02 +1.277020000000000e+02
+2.488000000000000e+02 +1.006150000000000e+02
+2.569890000000000e+00 +5.891900000000000e+01
+1.239950000000000e+02 +7.393290000000000e+01
+2.303650000000000e+02 +1.036850000000000e+02
+6.438360000000000e+02 +1.490210000000000e+02
+3.927380000000001e+02 +1.309120000000000e+02
+3.594720000000000e+02 +1.222070000000000e+02
+4.252040000000000e+02 +1.338340000000000e+02
+1.235740000000000e+02 +9.236690000000000e+01
+6.725930000000002e+02 +1.446510000000000e+02
+4.030200000000000e+02 +1.326030000000000e+02
+1.842310000000000e+00 +5.432050000000000e+01
+8.385550000000001e+01 +6.302070000000000e+01
+2.529440000000000e+02 +1.173560000000000e+02
+3.315640000000000e+02 +1.319890000000000e+02
+6.029390000000000e+02 +1.516570000000000e+02
+3.784490000000000e+02 +1.297660000000000e+02
+9.559170000000000e+00 +5.710450000000000e+01
+3.540380000000000e+01 +5.549570000000000e+01
+5.983710000000000e+02 +1.660730000000000e+02
+2.526220000000000e+02 +1.179640000000000e+02
+3.883530000000000e+02 +1.169260000000000e+02
+1.966230000000000e+02 +8.969660000000000e+01
+2.183780000000000e+02 +7.977589999999999e+01
+4.105160000000000e+02 +1.240650000000000e+02
+3.364760000000000e+02 +1.156810000000000e+02
+2.434550000000000e+02 +8.333669999999999e+01
+2.875660000000000e+00 +5.079320000000000e+01
+7.526909999999999e+01 +6.308100000000000e+01
+6.220350000000000e+02 +1.400680000000000e+02
+3.531890000000000e+02 +1.257240000000000e+02
+2.940040000000000e+02 +1.231080000000000e+02
+5.531569999999998e+02 +1.279140000000000e+02
+2.164730000000000e+02 +9.443680000000001e+01
+3.321430000000001e+02 +1.188260000000000e+02
+2.024730000000000e+02 +8.610860000000000e+01
+3.251510000000000e+02 +1.134720000000000e+02
+3.756420000000000e+02 +1.347930000000000e+02
+8.662290000000000e+01 +5.980610000000000e+01
+2.293550000000000e+02 +9.420800000000000e+01
+6.339010000000000e+02 +1.361570000000000e+02
+1.188780000000000e+02 +7.172480000000000e+01
+7.595610000000001e+01 +7.687020000000000e+01
+2.075820000000000e+02 +8.873690000000001e+01
+3.880080000000000e+02 +1.123430000000000e+02
+2.904310000000000e-01 +4.670399999999999e+01
+1.175370000000000e+02 +6.517220000000000e+01
+6.526469999999998e+02 +1.728140000000000e+02
+3.860220000000000e+02 +1.146860000000000e+02
+1.658050000000000e+02 +7.657280000000000e+01
+7.708150000000001e+01 +6.440479999999999e+01
+3.056080000000000e+02 +1.199320000000000e+02
+3.945580000000000e+02 +1.083530000000000e+02
+3.843780000000000e+00 +4.846260000000000e+01
+1.870590000000000e+02 +8.857080000000001e+01
+3.202920000000001e+02 +1.073210000000000e+02
+7.443639999999998e+00 +4.662880000000000e+01
+5.500899999999999e+01 +4.871150000000000e+01
+7.151360000000000e+01 +5.647700000000000e+01
+1.342730000000000e+02 +7.252540000000000e+01
+2.930870000000000e+02 +1.102040000000000e+02
+2.252590000000000e+01 +5.146910000000000e+01
+8.598610000000001e+00 +4.430810000000000e+01
+2.388740000000000e+02 +1.004200000000000e+02
+1.888070000000000e+02 +7.655929999999999e+01
+3.520440000000000e+00 +4.240440000000000e+01
+2.114490000000000e+02 +7.817040000000000e+01
+2.549160000000000e+02 +1.059020000000000e+02
+7.984360000000000e+01 +4.939170000000000e+01
+1.883130000000000e+02 +9.090430000000001e+01
+2.528900000000000e+02 +8.419750000000001e+01
+3.424140000000000e+02 +1.254540000000000e+02
+2.255770000000000e+02 +9.321899999999999e+01
+3.036080000000000e+02 +1.014800000000000e+02
+5.086579999999999e-01 +3.892000000000000e+01
+2.613250000000000e+02 +1.197520000000000e+02
+2.636640000000000e+02 +9.925320000000001e+01
+3.810060000000000e+02 +1.072000000000000e+02
+2.203400000000000e+02 +8.459940000000000e+01
+2.610330000000000e+02 +8.807769999999999e+01
+2.574830000000000e+02 +9.313120000000001e+01
+2.460910000000000e+02 +9.609560000000000e+01
+1.877900000000000e+02 +8.163979999999999e+01
+1.641660000000000e+02 +7.316790000000000e+01
+3.275600000000000e+02 +1.090230000000000e+02
+5.981650000000000e+01 +4.272790000000000e+01
+5.368370000000000e+02 +1.054930000000000e+02
+1.385900000000000e+02 +6.531950000000001e+01
+3.564870000000000e+02 +1.017630000000000e+02
+6.408450000000001e+01 +4.328910000000000e+01
+2.101020000000000e+02 +9.688670000000000e+01
+1.850030000000000e+02 +7.899299999999999e+01
+1.687509999999999e+02 +6.434810000000000e+01
+2.315280000000000e+01 +4.003520000000000e+01
+1.497940000000000e+02 +5.571510000000000e+01
+1.285790000000000e+02 +5.993080000000000e+01
+2.394000000000000e+02 +9.939610000000000e+01
+1.196780000000000e+02 +5.656830000000000e+01
+3.611300000000000e+02 +1.104740000000000e+02
+1.645370000000000e+02 +5.931090000000000e+01
+3.190230000000000e+02 +9.338530000000000e+01
+4.455130000000000e+02 +9.616180000000000e+01
+1.594520000000000e+02 +5.480250000000000e+01
+2.174920000000000e+02 +8.672820000000000e+01
+1.176250000000000e+02 +4.947130000000000e+01
+3.512870000000000e+02 +1.095720000000000e+02
+2.269150000000000e+02 +9.341260000000000e+01
+2.326420000000000e+02 +6.752360000000000e+01
+3.407490000000000e+02 +1.058820000000000e+02
+1.680350000000000e+02 +5.150840000000000e+01
+1.390350000000000e+02 +5.446600000000000e+01
+2.221220000000000e+02 +8.077700000000000e+01
+3.217740000000000e+01 +3.311700000000000e+01
+1.908560000000000e+02 +6.377920000000000e+01
+1.192380000000000e+02 +4.974560000000000e+01
+2.340440000000000e+02 +7.229370000000000e+01
+1.092770000000000e+02 +5.831480000000000e+01
+2.651500000000000e+02 +1.122540000000000e+02
+2.762120000000000e+02 +9.413039999999999e+01
+1.573920000000000e+02 +5.501980000000000e+01
+5.743170000000000e+01 +3.926850000000000e+01
+3.947170000000000e+02 +9.318940000000001e+01
+1.509550000000000e+02 +7.009139999999999e+01
+5.618110000000000e+01 +3.423380000000000e+01
+2.713780000000000e+02 +1.031950000000000e+02
+1.577390000000000e+00 +2.649710000000000e+01
+9.795880000000000e+01 +4.187680000000000e+01
+2.094100000000000e+02 +6.182390000000000e+01
+3.299450000000000e+02 +1.092110000000000e+02
+3.391310000000000e+00 +2.674380000000000e+01
+1.684870000000000e+02 +6.641419999999999e+01
+1.152970000000000e+02 +4.076380000000000e+01
+1.346240000000000e+02 +5.050210000000000e+01
+5.287370000000000e+01 +3.088660000000000e+01
+3.207330000000000e+02 +8.563560000000000e+01
+1.864830000000000e+02 +7.294790000000000e+01
+1.363730000000000e+01 +2.507620000000000e+01
+1.502730000000000e+02 +5.629210000000000e+01
+2.465240000000000e+02 +8.752740000000000e+01
+3.092360000000000e+02 +7.871469999999999e+01
+9.199850000000001e+01 +4.562780000000000e+01
+1.871050000000000e+02 +6.768670000000000e+01
+1.001080000000000e+02 +3.737460000000000e+01
+8.161730000000000e+01 +3.219900000000000e+01
+5.044660000000000e+01 +2.593290000000000e+01
+1.180270000000000e+02 +4.056090000000000e+01
+2.296230000000000e+02 +9.277960000000000e+01
+2.303830000000000e+02 +9.392230000000001e+01
+1.642160000000000e+01 +2.487830000000000e+01
+2.444720000000000e+02 +7.647120000000000e+01
+8.939010000000000e+01 +4.263120000000000e+01
+6.739600000000000e+01 +3.111150000000000e+01
+7.240360000000000e+01 +3.992730000000000e+01
+1.775150000000000e+02 +5.373400000000000e+01
+1.917630000000000e+02 +6.390650000000000e+01
+2.892760000000000e+02 +8.971850000000001e+01
+2.546500000000000e+01 +2.426810000000000e+01
+1.607770000000000e+02 +5.319190000000000e+01
+6.955070000000001e+01 +2.773350000000000e+01
+1.339170000000000e+02 +4.120820000000000e+01
+1.273920000000000e+02 +4.148770000000000e+01
+4.308620000000000e+01 +1.919240000000000e+01
+2.339410000000000e+02 +7.591050000000000e+01
+4.558400000000000e+01 +1.935460000000000e+01
+2.080500000000000e+02 +5.157100000000001e+01
+2.385610000000000e+02 +7.581829999999999e+01
+7.449110000000001e+00 +1.632100000000000e+01
+2.202950000000000e+02 +5.786370000000000e+01
+1.214050000000000e+02 +5.197430000000000e+01
+1.461290000000000e+02 +5.988230000000000e+01
+1.069830000000000e+02 +3.256550000000000e+01
+2.274160000000000e+02 +7.195229999999999e+01
+2.163830000000000e+02 +7.198560000000001e+01
+1.856420000000000e+02 +6.413209999999999e+01
+1.519180000000000e+02 +5.370890000000000e+01
+9.307570000000000e+01 +2.941110000000000e+01
+1.540080000000000e+02 +5.297690000000000e+01
+1.637480000000000e+02 +4.402140000000000e+01
+1.973400000000000e+02 +6.712899999999999e+01
+3.991740000000000e+01 +1.437130000000000e+01
+5.887060000000000e+01 +2.161550000000000e+01
+9.794130000000000e+01 +4.092910000000000e+01
+1.725550000000000e+02 +5.422930000000000e+01
+3.977610000000000e+01 +1.441940000000000e+01
+1.037170000000000e+02 +3.500960000000000e+01
+7.866180000000000e+01 +2.212180000000000e+01
+1.309940000000000e+02 +4.426390000000000e+01
+1.239150000000000e+02 +5.392430000000000e+01
+4.609090000000000e+01 +1.933930000000000e+01
+1.518670000000000e+02 +5.464090000000000e+01
+8.580860000000000e+01 +2.170350000000000e+01
+2.048710000000000e+01 +1.407290000000000e+01
+3.822630000000000e+01 +1.135040000000000e+01
+1.225250000000000e+02 +4.239410000000000e+01
+1.202580000000000e+02 +4.406050000000000e+01
+1.410660000000000e+02 +5.810950000000000e+01
+4.534470000000000e+01 +1.948700000000000e+01
+7.416800000000001e+01 +2.184490000000000e+01
+8.159470000000000e+01 +2.974710000000000e+01
+9.488750000000000e+01 +3.572360000000000e+01
+5.083770000000000e+01 +1.618940000000000e+01
+3.611700000000000e+01 +1.545980000000000e+01
+7.747780000000000e+01 +2.429570000000000e+01
+2.501030000000000e+01 +7.769980000000000e+00
+6.103360000000000e+01 +2.751910000000000e+01
+6.751620000000000e+01 +2.032930000000000e+01
+4.574360000000000e+01 +1.012160000000000e+01
+9.407859999999999e+01 +4.007890000000000e+01
+2.150570000000000e+01 +9.318379999999999e+00
+6.823620000000000e+01 +2.771340000000000e+01
+8.732890000000000e+01 +5.066500000000000e+01
+1.103930000000000e+01 +6.929830000000000e+00
+1.701720000000000e+01 +7.905360000000001e+00
+1.433880000000000e+01 +9.661759999999999e+00
+5.468140000000000e+01 +2.164450000000000e+01
+3.434620000000000e+00 +6.025610000000000e+00
+1.038430000000000e+01 +7.394639999999999e+00
+1.371090000000000e+01 +7.273070000000000e+00
+1.391870000000000e+01 +6.027390000000000e+00
+6.948660000000000e+01 +4.550470000000000e+01
+6.481820000000000e+01 +3.390080000000000e+01
+3.285420000000000e+01 +1.437810000000000e+01
+8.749599999999999e+00 +5.364480000000000e+00
+8.277690000000000e+00 +5.971350000000000e+00
+4.507150000000000e+01 +2.956990000000000e+01
+3.737760000000000e+00 +5.679340000000000e+00
+6.750910000000000e+00 +5.072430000000002e+00
+4.450420000000000e+01 +3.248570000000000e+01
+4.231930000000000e+01 +2.727320000000000e+01
+3.325210000000000e+01 +1.345240000000000e+01
+1.549400000000000e+01 +3.971420000000000e+00
+2.710910000000000e+00 +1.946000000000000e+00
+1.631730000000000e+01 +4.727500000000000e+00
+2.287520000000000e+01 +1.546430000000000e+01
+1.843750000000000e+01 +1.357310000000000e+01
+1.146520000000000e+00 +1.532800000000000e+00
+5.021090000000000e-01 +1.772360000000000e+00
+1.630000000000000e+01 +1.095960000000000e+01
+1.314640000000000e-02 +9.730000000000000e-01
+4.957180000000000e+00 +6.707400000000000e-01
+8.277580000000000e-01 +0.000000000000000e+00
+3.969170000000000e-02 +0.000000000000000e+00
+1.551810000000000e+00 +0.000000000000000e+00
+1.930340000000000e+00 +0.000000000000000e+00
+3.137500000000000e+00 +0.000000000000000e+00
+0.000000000000000e+00 +0.000000000000000e+00
+0.000000000000000e+00 +0.000000000000000e+00
+0.000000000000000e+00 +0.000000000000000e+00
+0.000000000000000e+00 +0.000000000000000e+00
+0.000000000000000e+00 +0.000000000000000e+00
+0.000000000000000e+00 +0.000000000000000e+00
+0.000000000000000e+00 +0.000000000000000e+00
+0.000000000000000e+00 +0.000000000000000e+00
+0.000000000000000e+00 +0.000000000000000e+00
+0.000000000000000e+00 +0.000000000000000e+00
+0.000000000000000e+00 +0.000000000000000e+00
+0.000000000000000e+00 +0.000000000000000e+00
+0.000000000000000e+00 +0.000000000000000e+00
};
\end{groupplot}

\end{tikzpicture}
    \caption[Distancia vs. CO$^{2}$ total]{Gráfico de dispersión, distancia total vs. gramos totales de CO$^{2}$ emitidos en la simulación, para escenarios con factor de demanda 100\%, 15 minutos de tiempo simulado.}
    \label{fig:distvsco2}
\end{figure}

El gráfico \ref{fig:distvsco2} presenta la relación entre la distancia total recorrida vs. la cantidad total de dióxido de carbono emitido por cada vehículo en los escenarios. Este gráfico presenta resultados muy similares al anterior; existen dos grupos de vehículos, en el caso sin comunicación, que tienden a exhibir mayores cantidades totales de emisiones para distancias comparativamente más cortas, mientras que el \emph{run} con comunicación perfecta presenta una dispersión un poco mayor pero sin tendencia a extremos.

Nuevamente se asocian los extremos en el \emph{run} sin comunicación a aquellos vehículos que se ven atrapados en el atochamiento producido por el accidente. Estos vehículos se encontraron la mayor del tiempo desplazándose a muy bajas velocidades o detenidos con el motor encendido, lo cual causa mayores emisiones de dióxido de carbono a la atmósfera. Por otro lado, en el \emph{run} con comunicación perfecta, nuevamente se debe considerar que muchos vehículos son redirigidos y deben extender su trayectoria, lo cual genera una mayor dispersión en los valores obtenidos

\subsection{Conclusiones}

El análisis del modelo vehicular arrojó los resultados esperados, y se evidencia claramente el efecto mitigante en la congestión vehicular producto de un accidente que puede tener la comunicación intervehicular en un sistema de transporte inteligente. 

En particular, los experimentos demostraron que PVEINS es capaz de simular escenarios viales realistas, y que permite analizar los resultados de manera correcta y precisa. La integración permite la obtención de datos claves del sistema, como la distancia total recorrida por cada vehículo, el tiempo que permaneció en el escenario y el total de su emisión de dióxido de carbono. 

%Finalmente, se destaca que el \emph{framework} posibilita la modificación total del comportamiento de vehículos en respuesta a eventos generados de manera dinámica. 

