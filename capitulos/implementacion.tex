\chapter{Comunicando Paramics con OMNeT++ mediante TraCI}
\section{Diseño Arquitectural}

El software desarrollado consiste en un \emph{plugin} que extiende la funcionalidad de Paramics, agregándole la capacidad de comportarse como un servidor TraCI. Específicamente, el \emph{plugin} consiste en una implementación parcial de un servidor TraCI, el cual se ejecuta en un \emph{thread} paralelo a Paramics; este se encuentra a su vez simulando en modo discreto, esperando instrucciones para avanzar la simulación. La comunicación entre ambos se efectúa a través de la API de extensión de Paramics.

La figura \ref{fig:ptraci_arch} ilustra esta arquitectura. A pesar de que se encuentra implementado como un \emph{plugin} de Paramics, el servidor TraCI es prácticamente un programa independiente, y su interacción con el simulador de transporte se limita a un conjunto acotado de llamados a su API.

\begin{figure}[t]
    \centering
    \begin{sequencediagram}
    \newthread{D}{OMNeT++}{}
    \newinst[1]{A}{VEINS}{}
    \newinst[3]{B}{Plugin (TraCIServer)}{}
    \newthread[2]{C}{Paramics}{}
    
    \begin{messcall}{C}{run()}{B}
        \postlevel
        \begin{call}{B}{waitForCommands()}{B}{}
        \end{call}
    \end{messcall}
    
    \begin{call}{D}{Solicitud}{A}{Resultado}
    
        \begin{call}{A}{Comando TraCI}{B}{Respuesta TraCI}
            \begin{call}{B}{parseCommand()}{B}{sendResponse()}
                \postlevel
                \begin{call}{B}{API Paramics}{C}{Datos}
                \end{call}
                \postlevel
            \end{call}
        \end{call}
    \end{call}
\end{sequencediagram}
    \caption{Arquitectura del Framework}
    \label{fig:ptraci_arch}
\end{figure}
\newpage