\chapter{Comunicando Paramics con OMNeT++ mediante TraCI}
\section{Diseño Arquitectural}\label{sec:architecture}

El software desarrollado consiste en un \emph{plugin} que extiende la funcionalidad de Paramics, agregándole la capacidad de comportarse como un servidor TraCI. Específicamente, el \emph{plugin} consiste en una implementación parcial de un servidor TraCI, el cual se ejecuta en un \emph{thread} paralelo a Paramics; este se encuentra a su vez simulando en modo discreto, esperando instrucciones para avanzar la simulación. La comunicación entre ambos se efectúa a través de la API de extensión de Paramics.

La figura \ref{fig:ptraci_arch} ilustra esta arquitectura. A pesar de que se encuentra implementado como un \emph{plugin} de Paramics, el servidor TraCI es prácticamente un programa independiente, y su interacción con el simulador de transporte se limita a un conjunto acotado de llamados a su API.

\begin{figure}[t]
    \centering
    \begin{sequencediagram}
    \newthread{D}{OMNeT++}{}
    \newinst[1]{A}{VEINS}{}
    \newinst[3]{B}{Plugin}{}
    \newthread[3]{C}{Paramics}{}
    
    \begin{messcall}{C}{run()}{B}
        \postlevel
        \begin{call}{B}{waitForCommands()}{B}{}
        \end{call}
    \end{messcall}
    
    \begin{call}{D}{Solicitud}{A}{Resultado}
    
        \begin{call}{A}{Comando TraCI}{B}{Respuesta TraCI}
            \begin{call}{B}{parseCommand()}{B}{sendResponse()}
                \postlevel
                \begin{call}{B}{API Paramics}{C}{Datos}
                \end{call}
                \postlevel
            \end{call}
        \end{call}
    \end{call}
\end{sequencediagram}
    \caption{Arquitectura del Framework}
    \label{fig:ptraci_arch}
\end{figure}

\section{Módulos Principales}
\subsection{plugin.c}

Si bien en estricto rigor no es un módulo del \emph{framework}, merece ser mencionado al ser el archivo principal del \emph{plugin} desarrollado. En este archivo se definen las funciones de extensión (prefijo \texttt{QPX}, ver sección \ref{sec:paramics_api}) a ser invocadas por Paramics al inicializar el \emph{plugin}.

\subsubsection{\texttt{void qpx\_NET\_postOpen(void)}}

Invocada inmediatamente luego de que Paramics carga la red y el \emph{plugin}, esta función cambia el modo de ejecución de Paramics a su modo discreto e inicializa el servidor TraCI. Para esto, crea un \emph{thread} donde corre una función auxiliar \texttt{runner\_fn()}, la cual se encarga de:

\begin{enumerate}
    \item Obtener el puerto en el cual esperar conexiones entrantes desde los parámetros de ejecución de Paramics. De no haberse especificado puerto, utiliza uno por defecto.
    \item Inicializar un objeto \emph{TraCIServer} (ver sección \ref{sec:traciserver}) encargado de las conexiones entrantes en el puerto anteriormente definido.
\end{enumerate}


\subsection{TraCIServer}\label{sec:traciserver}

Implementa el funcionamiento base del servidor TraCI. Es el primer módulo en inicializarse, y tiene como funciones:

\begin{enumerate}
    \item Asociarse a un \emph{socket} TCP, y esperar una conexión de un cliente TraCI.
    \item Mientras exista una conexión abierta, recibir e interpretar comandos TraCI entrantes.
    \begin{itemize}
        \item En el caso de los comandos de obtención de versión y cierre de la conexión, estos son ejecutados por el módulo mismo.
        \item El comando de avance de simulación es ejecutado parcialmente por este módulo y el módulo \emph{Simulation} (ver sección \ref{sec:simulation}).
        \item Demás comandos son delegados a los módulos pertinentes.
    \end{itemize}
    \item Escribir mensajes de estado y respuesta a comandos TraCI.
    \item Al recibir un comando de cierre, finalizar la simulación y cerrar el \emph{socket}.
\end{enumerate}

El módulo en cuestión se implementó como una clase de C++ en los archivos \emph{TraCIServer.h} y \emph{TraCIServer.cpp}. A continuación se detallará la implementación de cada una de las funcionalidades anteriormente enumeradas.

\subsubsection{Inicio de conexión TraCI}


\subsection{Simulation}\label{sec:simulation}
\subsection{VehicleManager}