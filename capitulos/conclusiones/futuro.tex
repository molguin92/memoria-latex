\section{Trabajo futuro}

El \emph{framework} desarrollado cumple con todos los objetivos que se expusieron al principio del presente documento, y se evalúa como altamente aplicable para el estudio de modelos de Sistemas Inteligentes de Transporte; sin embargo, aún así queda trabajo futuro por desarrollar. 

Lo más evidente, en primer lugar, es la implementación de los comandos TraCI restantes, los cuales no fueron considerados dentro del alcance de la presente memoria. Estos comandos incluyen diversas acciones no esencialmente necesarias para la simulación de ITS, pero que en un futuro pudiesen llegar a ser requeridos. Comandos como para, por ejemplo, modificar el comportamiento de peatones, semáforos y hasta detectores de bucle de inducción, se encuentran definidos en el protocolo y, de llegar a necesitarse, deberán ser implementados en el \emph{framework}. 
Se debe también tener en consideración la constante evolución de TraCI -- el protocolo se actualiza periódicamente en conjunto con VEINS, y cada nueva versión puede introducir comandos nuevos o deprecar o modificar totalmente comandos antiguos (reemplazándolos con nuevas versiones). Será necesario entonces mantener actualizado PVEINS con estos cambios constantes si se desea mantener compatibilidad con VEINS en el futuro. 
De esta misma manera, se deberá actualizar el \emph{plugin} de Paramics en el caso de actualizaciones que rompan compatibilidad con el API de extensión actual.

En segundo lugar, se plantea la necesidad a futuro de implementar el inicio automático de la simulación de Paramics al recibir una conexión entrante por el \emph{socket} TCP. Esta funcionalidad estaba presente en la primera versión de la arquitectura del \emph{software}, sin embargo fue necesario descartarla al realizar las modificaciones necesarias para el correcto funcionamiento \emph{single-thread} del \emph{plugin} (ver sección \ref{sec:architecture}). Contar con esta funcionalidad significaría poder ejecutar \emph{batches} (conjuntos) de \emph{runs} de simulaciones sin ninguna intervención del usuario -- permitiendo así realizar análisis mucho más acabados de manera más simple.

El \emph{plugin} todavía admite optimización, a pesar de que se trató de implementar de la manera más óptima posible y se realizaron revisiones y reimplementaciones de funcionalidades al descubrir nuevas maneras de optimizar el funcionamiento de éstas. Por ejemplo, la funcionalidad de cambio de pista utiliza una colección aparte para almacenar los comandos de cambios recibidos, sobra la cual se itera luego de cada paso de simulación. 
Esto significa que Paramics, aparte de actualizar el estado de los $N$ vehículos presentes en la simulación en cualquier instante de acuerdo a su modelo interno, deberá recorrer un máximo de otros $N$ elementos adicionales (máximo un cambio de pista por vehículo) y ejecutar los cambios correspondientes después de cada paso de simulación.
De lograrse implementar entonces una versión del cambio de pista similar a las implementaciones de cambio de ruta y velocidad que se llevaron a cabo en el \emph{framework}, es decir, utilizando el mismo modelo de Paramics para realizar los cambios, podría mejorarse el rendimiento, tal vez de manera considerable para escenarios con una gran cantidad de cambios de pista. 

También existen leves ineficiencias en los procesos de recepción e interpretación de mensajes TraCI, las cuales fueron introducidas intencionalmente para aumentar la modularidad y extensibilidad del código. 
Decisiones de diseño como la encapsulación de cada comando en un objeto \texttt{tcpip::Storage} aparte agregan \emph{overhead} al \emph{software} a cambio de código más legible y flexible, factores que en el futuro pueden ser menos importantes que el rendimiento del \emph{framework}. En ese caso, será necesaria una completa refactorización de la estructura interna de paso de parámetros del código, ya esto fue una decisión de diseño fundamental en la implementación actual.

Finalmente, el \emph{framework} podría ser evaluado y analizado por alguien con más experiencia en el área de transporte. En específico, sería valioso un estudio de la validez de los resultados de PVEINS con un escenario que implemente un modelo de re-enrutamiento más avanzado y realista que el empleado para el escenario de validación del presente trabajo de memoria.