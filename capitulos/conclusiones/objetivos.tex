\section{Cumplimiento de objetivos}\label{sec:conclusiones:obj}

El objetivo general del presente trabajo de memoria, ``\textbf{el desarrollo de un \textit{framework} de integración entre un simulador de redes, OMNeT++ y un microsimulador de tráfico, Quadstone Paramics, de tal manera que exista comunicación bidireccional entre ambos}'', fue logrado en su totalidad. PVEINS, el \emph{framework} desarrollado, permite la integración totalmente transparente de Paramics con el \emph{framework} VEINS, desarrollado por Sommer \emph{et al.} en \autocite{sommer_german_dressler}, y OMNeT++, posibilitando así la simulación de Sistemas de Transporte Inteligentes complejos, además del uso de la gran cantidad de modelos de comunicación ya desarrollados para dicho simulador de redes de comunicaciones. 
Se logró esta integración mediante la implementación de un \emph{plugin} para Paramics, el cual actúa como una interfaz entre el simulador y el protocolo TraCI, permitiendo así el intercambio estandarizado de información y comandos con OMNeT++ y VEINS a través de un \emph{socket} TCP. 

Se demostró además que esta implementación es eficiente en términos de la relación entre tiempo de duración real y tiempo simulado para simulaciones de gran tamaño, del orden de cientos o hasta miles de nodos presentes simultáneamente en la red. En particular, se realizaron experimentos que mostraron la factibilidad de realizar simulaciones de gran envergadura en tiempos razonables: una escenario de 15 minutos de tiempo simulado con un promedio de 869 vehículos presentes en cada instante de la simulación se ejecutó en apenas 11 minutos de tiempo real.
La implementación es también austera en recursos de sistema, utilizando menos de 600 MB de memoria y aumentando la actividad del procesador en un 20\% para una simulación con un promedio de 1400 nodos activos en cualquier instante dado.

En términos de los objetivos particulares enumerados en la sección \ref{sec:obj:part}, se puede concluir que:

\begin{enumerate}
    \item \textbf{Se estableció el estado del arte en cuanto a simulación bidireccional en comunicaciones inalámbricas y sistemas de transporte}, tomando en cuenta herramientas tanto de código abierto como cerrado. Esto se expuso de manera extensa y profunda en la sección \ref{sec:state_of_the_art}.
    
    \item \textbf{Se escogió la opción de adaptar el \emph{framework} VEINS como solución al problema presentado}, dado que correspondía a la opción más madura y flexible en términos de simulación bidireccional. El razonamiento completo detrás de esta decisión se encuentra en la sección \ref{sec:solution}.
    
    \item \textbf{Se adaptó Paramics}, mediante la implementación de un \emph{plugin}, \textbf{para que pudiese comunicarse de manera bidireccional con OMNeT++}. Se implementaron además todas las funcionalidades enumeradas en la sección \ref{sec:functionality}, las cuales incluyen:
    \begin{enumerate}
        
        \item Construcción de la topología del modelo de comunicaciones a partir de la topología del modelo de tráfico.
        
        \item Actualización dinámica de los nodos en OMNeT++, siguiendo los movimientos de los elementos de Paramics.
        \item Modificación del comportamiento de los nodos del modelo de transporte a partir de eventos en OMNeT++.
    \end{enumerate}
    \item \textbf{Se implementó el \emph{framework} PVEINS siguiendo patrones y buenas prácticas de ingeniería de software} -- el detalle de esto se encuentra en el capítulo \ref{cap:diseno}.
    
    \item \textbf{Se probó y validó el funcionamiento de la integración de los simuladores} mediante la simulación de un modelo de transporte simple pero dinámico (ver sección \ref{sec:simpletest}).
    
    \item Finalmente, \textbf{se implementó un modelo avanzado de transporte}, y se utilizó para \textbf{la evaluación del rendimiento del software implementado y su aplicabilidad a modelos realistas de Sistemas de Transporte Inteligentes}, utilizando las métricas de evaluación detalladas en la sección \ref{sec:eval_metrics}. El detalle del modelo desarrollado y los resultados obtenidos se expusieron en el capítulo \ref{cap:validacion}.
\end{enumerate}