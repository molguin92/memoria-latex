\section{Conclusiones generales}

Las conclusiones y resultados obtenidos al final del presente trabajo de memoria son altamente positivos. Se evalúa de muy buena manera el trabajo realizado, el cual cumple a cabalidad con lo propuesto al principio de este documento -- ver sección \ref{sec:conclusiones:obj} para un desglose de cumplimiento de objetivos. El \emph{software} desarrollado es eficiente, y logra su propósito completamente -- se demostró que se puede utilizar para simular no solo sistemas de transportes pequeños y de poco flujo vehicular, sino que también es una herramienta potente para el estudio de sistemas de gran complejidad.

En términos de desarrollo e implementación, el trabajo conllevó la adquisición de experiencia en trabajar con software propietario, el cual no siempre cuenta con documentación completa y actualizada. 

Si bien el API de Paramics era lo suficientemente potente y flexible como para realizar la propuesta de memoria, su documentación en muchas ocasiones no incluía la información necesaria, y en reiterados casos el memorista se vio obligado a reimplementar funcionalidades al descubrir -- por accidente -- maneras más eficientes de realizar la acción en cuestión que no se encontraban documentadas. 
Así mismo, ocurrió también lo inverso. Existieron situaciones en que un método de la API de Paramics parecía ser perfecto para la implementación de alguna funcionalidad, y al utilizarlo se descubría que tenía un desperfecto o que simplemente no funcionaba. El mejor ejemplo de ésto fue la función de avance de simulación de Paramics, que en ningún lugar en la documentación se advertía no funcionaba de manera correcta con \emph{threads} paralelos. Al descubrirse, esto implicó, como se detalló en la sección \ref{sec:architecture}, un rediseño total de la arquitectura del \emph{framework}. De todas maneras, se logró sobrellevar estas dificultades y llevar a cabo el proyecto en su totalidad.

En términos personales, el desarrollo del trabajo presentó grandes oportunidades de aprendizaje, y el acercamiento a un área de investigación un poco alejada de lo común en ciencias de la computación. Implicó la familiarización con temas relacionados con el modelamiento de flujo vehicular, con sistemas de transporte y temas relacionados con las tecnologías de comunicaciones, y los estándares en uso hoy en día. Permitió al memorista además ponerse en contacto con investigadores de nivel mundial como los doctores Falko Dressler y Cristoph Sommer, autores de \autocite{sommer_german_dressler} y \autocite{sommer_dressler2} y reconocidos investigadores en el área de comunicaciones intervehiculares. De hecho, cabe destacar que quien originalmente propuso la idea de resolver la problemática original mediante la adaptación de VEINS fue el mismo profesor Dressler en una visita a la Universidad de Chile.

