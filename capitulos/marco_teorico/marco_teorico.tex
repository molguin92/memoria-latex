\section{Marco Teórico}

En esta sección se detallarán los conceptos esenciales para la comprensión del presente trabajo de memoria.

\subsection{Sistemas Inteligentes de Transporte}

Los Sistemas Inteligentes de Transporte (en adelante \emph{ITS}, por sus siglas en inglés -- \textit{Intelligent Transportation Systems}) surgen como una respuesta a la necesidad de optimización y modernización de sistemas de transporte existentes. La Unión Europea define a los ITS como aplicaciones avanzadas que, sin incorporar inteligencia como tal, pretenden proveer servicios innovadores relacionados con distintos modos de transporte y de administración de tráfico, que además otorgan información a los usuarios, permitiéndoles utilizar el sistema de transporte de manera más segura, coordinada e inteligente \cite{eudirective}. De acuerdo al Departamento de Transportes de los EEUU, estos sistemas se pueden dividir en dos grandes categorías \cite{usdot}:
\begin{description}
    \item [Sistemas de Infraestructura Inteligente] Tienen como enfoque el manejo de los sistemas de transporte a niveles macro, y la transmisión de información oportuna a los usuarios. Esta categoría incluye, entre otros, sistemas de advertencia y señalización dinámica en ruta (ya sea a través de pantallas o sistemas de comunicación inalámbrica), sistemas de pago electrónico y de coordinación del flujo de tráfico.
    
    \item [Sistemas de Vehículos Inteligentes] Engloba todo aquello relacionado con la automatización y optimización de la operación de un vehículo. Dentro de esta categoría se incluyen sistemas de advertencia y prevención de colisiones, de asistencia al conductor --- por ejemplo, sistemas de navegación --- y control autónomo de vehículos.
    
\end{description}

\subsection{Comunicación Inalámbrica}

En el contexto de la presente memoria, se entenderá por \emph{comunicación inalámbrica} todo acto de transmisión de información entre dos o más entidades mediante la interacción con un campo electromagnético, sin otra conexión física entre dichas entidades (\emph{e.g.} cables). Estas entidades denominarán \emph{nodos}, y al establecerse una configuración que permita la comunicación inalámbrica entre múltiples nodos cercanos, se hablará de una \emph{red inalámbrica}.

\subsubsection{802.11p}

\subsubsection{Simulaciones}

\subsection{Simulación de Tráfico}

Se entenderá por \emph{Simulación de Tráfico} aquel entorno virtual que permita, mediante el modelamiento de ésta utilizando herramientas computacionales, la emulación y estudio del comportamiento de un sistema de transporte ficticio o real.

\subsubsection{Microscópica}

Un simulación de tráfico \emph{microscópica} es aquella que simula de manera individual el comportamiento de cada vehículo presente en el sistema de transporte. 

\subsubsection{Macroscópica}
\subsection{Simulación Bidireccional}

