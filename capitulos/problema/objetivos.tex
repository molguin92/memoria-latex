\section{Objetivos}\label{sec:obj}

\subsection[Objetivo general]{Objetivo general: Integración de un simulador de redes de comunicaciones y un simulador de tráfico.}

El principal objetivo de este trabajo de memoria fue el desarrollo de un \textit{framework} de integración entre un simulador de redes, OMNeT++ y un microsimulador de tráfico, Quadstone Paramics, de tal manera que exista comunicación bidireccional entre ambos. Dicho framework debía permitir la comunicación e interacción entre los simuladores. Esto quiere decir que el simulador de redes de comunicaciones debía recibir información periódica desde el simulador de tráfico en base a la cual construir su topología de red interna. A su vez, el simulador de tráfico debía recibir instrucciones del modelo de comunicaciones, y poder modificar su modelo con base en estas.
Además, la implementación de este software debía considerar conceptos y buenas prácticas de ingeniería de software, poniendo especial énfasis en la eficiencia necesaria para simular grandes redes de comunicación vehicular.

Cabe destacar que el simulador de transporte en cuestión, Paramics, fue escogido ya que es el simulador de preferencia del Área de Transporte del Departamento de Ingeniería Civil de la Universidad de Chile, quienes tienen gran interés en el presente proyecto de memoria.


\subsection{Objetivos particulares}\label{sec:obj:part}

El objetivo general discutido anteriormente se desglosó en los siguientes objetivos particulares que debían haberse cumplido al final del desarrollo del trabajo de memoria:

\begin{enumerate}
    \item Establecer el estado del arte en cuanto a simulación bidireccional en comunicaciones inalámbricas y sistemas de transporte, tomando en cuenta herramientas tanto de código abierto como cerrado (ver sección \ref{sec:state_of_the_art}).
    \item Escoger una solución viable a el problema particular presentado, basándose en el estado del arte previamente establecido (ver sección \ref{sec:solution}).
    \item Diseñar el mecanismo que permita la comunicación entre ambos simuladores.\\
    El diseño deberá considerar las siguientes funcionalidades:
    \begin{enumerate}
        \item Construcción de la topología del modelo de comunicaciones a partir de la topología del modelo de tráfico.
        \item Actualización dinámica de los nodos en OMNeT++, siguiendo los movimientos de los elementos de Paramics.
        \item Modificación del comportamiento de los nodos del modelo de transporte a partir de eventos en OMNeT++.
    \end{enumerate}
    \item Implementar dicho mecanismo, siguiendo patrones y buenas prácticas de ingeniería de software.
    \item \label{itm:simple_test} Probar y validar el funcionamiento de la integración de los simuladores mediante la simulación de un modelo de transporte simple pero dinámico.
    
    \item \label{itm:adv_test} Implementar un modelo avanzado de transporte y definir métricas de desempeño e impacto de la red de comunicaciones en la operación del modelo.
\end{enumerate}

Para el punto \ref{itm:adv_test}, se consideraron modelos desarrollados por el Área de Transportes del Departamento de Ingeniería Civil de la Universidad de Chile, de los cuales se escogió uno específico, detallado en el capítulo \ref{cap:validacion}.