\section{Elección de solución a implementar}\label{sec:solution}

Luego de realizar el exhaustivo estudio del estado del arte presentado en la sección \ref{sec:state_of_the_art} y de definir claramente la problemática a abordar, se tomó la decisión de desarrollar el \emph{framework} propuesto para el presente trabajo de memoria basándose en la implementación de VEINS \autocite{sommer_german_dressler, sommer_dressler2}. Esta solución implica reemplazar a SUMO por Paramics, de manera totalmente transparente para OMNeT++ y VEINS, de tal manera de poder reutilizar la amplia oferta de módulos, modelos y esquemas de comunicación inalámbrica ya existentes para OMNeT++ sin necesidad de adaptarlos específicamente para el nuevo simulador de transporte. 

Esto se puede lograr gracias al diseño modular y extensible de VEINS. Dado que ambos simuladores se comunican a través de un protocolo bien definido, el cual abstrae el funcionamiento de ambos extremos de la comunicación, ninguno de los dos simuladores necesita saber detalles de la implementación del otro. TraCI, el protocolo de comunicación, fue justamente diseñado con este tipo de escenarios en mente (\autocite{traci}).

El problema se redujo entonces a la implementación de una capa de interfaz para Paramics, que le permite intepretar y funcionar con TraCI -- para esto se utilizó el API de extensión del \emph{software}, detallado en el apéndice \ref{anex:paramics_api}. 

Se tomó la decisión de optar por esta solución ya que ninguna de las demás opciones de simulación bidireccional presentaba la madurez y flexibilidad necesaria para el tipo de escenarios para los cuales se pretende utilizar el producto final. Como se comentó en la sección \ref{sec:integrated_sim}, si bien existen una serie de alternativas de entornos integrados para la simulación bidireccional, ninguna es tan avanzada y potente como VEINS.

Cabe notar además que los modelos de OMNeT++ y VEINS se implementan mediante módulos autocontenidos, los cuales gobiernan el comportamiento de la simulación. Estos módulos se comunican con el \emph{framework} mediante una serie de APIs, y no existe la necesidad de modificar el código de la integración para cada escenario distinto.

Finalmente, una implementación utilizando VEINS necesariamente cumple con los requisitos de diseño descritos en la sección \ref{sec:obj:part}, ya que éstas son funcionalidades fundamentales del \emph{framework} y del protocolo TraCI.