\section{Metodología}

La metodología seguida en el desarrollo del trabajo de memoria se detalla en los siguientes puntos.

\begin{enumerate}
    \item \label{itm:method_sol} \textbf{Elección de la solución a implementar:} En primera instancia, se debió escoger la solución a elaborar, basándose en una comparación exhaustiva de los puntos a favor y en contra de cada una. 
    
    \item \textbf{Diseño a escala macro de la solución:} Independiente de la solución escogida en el punto \ref{itm:method_sol}, se debió hacer un diseño macroscópico de la implementación a seguir, con el fin de establecer parámetros y guías a seguir durante el proceso de desarrollo. 
    
    \item \textbf{Desarrollo iterativo:} Se buscó seguir una metodología ágil en el desarrollo del software, acumulativamente agregando funcionalidades. Por ejemplo, para la primera iteración se buscó contar con una implementación básica de la comunicación entre los simuladores, la que simplemente permitiese la obtención de las posiciones iniciales de los nodos.
    
    \item \textbf{Validación de la integración:} Se validó el \textit{framework} primero utilizando un modelo básico y simple, y luego con un escenario más complejo, dinámico y realista. 
\end{enumerate}