\section{Organización del documento}

El presente documento de memoria se estructura como sigue; el presente capítulo expone la motivación tras el proyecto desarrollado, y los objetivos principales que se pretendían lograr con la implementación de éste. 

El capítulo \ref{cap:marcoteo} expone el marco téorico que sustenta el trabajo de memoria, además de presentar una extensa revisión del estado del arte en los ámbitos de simulación de sistemas de transporte, de redes de comunicaciones y de simulación bidireccional entre las dos categorías anteriores.

A continuación, en el capítulo \ref{cap:diseno} se detallan las decisiones de diseño macroscópico que se tomaron al elaborar la arquitectura del proyecto, y las evoluciones por las cuales este diseño pasó. Se presenta además la metodología de trabajo utilizada para el desarrollo del \emph{software} y las funcionalidades implementadas en términos generales.

En el capítulo \ref{cap:implementation} Implementación, se describe en detalle la implementación en código del \emph{framework}, además de entregarse una breve descripción de las pruebas que se realizaron durante el desarrollo.

El capítulo siguiente, el capítulo \ref{cap:validacion}, expone el escenario avanzado de prueba que se utilizó para evaluar el rendimiento y la efectividad del proyecto. Se presentan además los resultados obtenidos de las pruebas realizadas, y se realiza un análisis de éstos.

Finalmente, el capítulo \ref{cap:conclusion} concluye la presente memoria, realizando un análisis general de los resultados obtenidos, verificando que se cumplieron los objetivos establecidos en la sección \ref{sec:obj} y presentando trabajo futuro a realizar.

