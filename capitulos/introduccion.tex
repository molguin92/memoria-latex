\chapter{Introducción}
Los sistemas de transporte conforman la columna vertebral de nuestras ciudades, contribuyendo directamente al desarrollo de la sociedad urbana. Un sistema de transporte bien diseñado y eficiente permite el desplazamiento rápido y cómodo de personas y bienes; en cambio, uno ineficiente genera grandes problemas, alargando los tiempos de viaje y aumentando la contaminación atmosférica.
Los Sistemas de Transporte Inteligente (ITS, por sus siglas en inglés: \textit{Intelligent Transportation Systems}) surgen como una respuesta a la necesidad de optimización y modernización de los sistemas de transporte existentes. La Unión Europea define a los ITS como aplicaciones avanzadas que, sin incorporar inteligencia como tal, pretenden proveer servicios innovadores relacionados con distintos modos de transporte y de administración de tráfico, que además otorgan información a los usuarios, permitiéndoles utilizar el sistema de transporte de manera más segura, coordinada e inteligente \cite{eudirective}. De acuerdo al Departamento de Transportes de los EEUU, estos sistemas se pueden dividir en dos grandes categorías: sistemas de infraestructura inteligente y sistemas de vehículos inteligentes \cite{usdot}.

Los sistemas de infraestructura inteligente tienen como enfoque el manejo de los sistemas de transporte a niveles macro, y la transmisión de información oportuna a los usuarios. Esta categoría incluye, entre otros, sistemas de advertencia y señalización dinámica en ruta (ya sea a través de pantallas o sistemas de comunicación inalámbrica), sistemas de pago electrónico y de coordinación del flujo de tráfico.

Por otro lado, la categoría de sistemas de vehículos inteligentes engloba todo aquello relacionado con la automatización y optimización de la operación de un vehículo. Dentro de esta categoría se incluyen sistemas de advertencia y prevención de colisiones, de asistencia al conductor --- por ejemplo, sistemas de navegación --- y control autónomo de vehículos.

El factor común entre ambas categorías es la necesidad de extraer información en tiempo real desde el entorno, la cual debe procesarse y en muchos casos generar una respuesta a transmitir al usuario. Para este fin, se ha propuesto la implementación de tecnologías que posibiliten esta comunicación, principalmente utilizando redes inalámbricas, tanto de área local (los estándares incluídos en la familia WLAN, IEEE 802.11), como de área personal (WPAN, IEEE 802.15) \cite{80211dailey,80215vanet,80211wave}. Sin embargo, estas tecnologías fueron diseñadas originalmente para su uso en redes estáticas o con patrones de movimiento muy limitados, y es necesaria la evaluación de su desempeño en entornos altamente dinámicos como lo son los sistemas de transporte vehicular. Parámetros críticos para el funcionamiento óptimo de la red, como la potencia de transmisión, las condiciones del canal de transmisión y la distancia óptima entre nodos, deben establecerse teniendo en cuenta las particularidades que presentan los sistemas de transporte --- por ejemplo, la alta congestión de nodos en intersecciones con semáforos.

Existe entonces hoy en día la necesidad de modelar de manera realista y precisa el comportamiento de éstas tecnologías en contextos de comunicaciones inalámbricas en redes vehiculares. Por otro lado, existe también la necesidad de modelar el impacto de la comunicación inalámbrica en un sistema de transporte, y cómo esta puede contribuir a optimizar la operación del mismo \cite{bidirectionalsimul}. Un ejemplo de esto son los Sistemas Avanzados de Información al Viajero (\textit{ATIS}, por sus siglas en inglés; \textit{Advanced Traveller Information System}) los cuales proveen información en tiempo real sobre las condiciones del tránsito a conductores, permitiéndoles elegir la ruta más óptima para alcanzar su destino. Este \textit{feedback} inmediato sin duda tiene efectos importantes en el flujo vehicular de un sistema de transportes, los cuales deben ser tomados en consideración al momento de modelar y simular el funcionamiento del mismo.

A raíz de lo anterior, en el presente informe se propone la elaboración de un sistema que integre un simulador de redes de comunicaciones con uno de tráfico vehicular como una herramienta de apoyo para el estudio de las problemáticas previamente señaladas y la evaluación de nuevos modelos y soluciones para transporte inteligente.
