\usepackage{lipsum}
\usepackage{multicol}
\usepackage[]{inputenc}
\usepackage[T1]{fontenc}
\usepackage{bytefield}
\usepackage{subcaption}
\usepackage{pgf-umlsd}
\usepackage{pgf-umlcd}
\usepackage[spanish]{babel}
\usepackage[style=ieee, sorting=none]{biblatex}
\usepackage{csquotes}
\usepackage{appendix}
\usepackage{listings}
\usepackage{float}
\usepackage{sourcecodepro}
\usepackage{tikz}
\usetikzlibrary{shapes.geometric, arrows}
\usepackage{tikz-qtree}
\usepackage{dirtree}
\usepackage[labelfont=bf]{caption}
\usepackage{enumitem}
\usepackage{hyperref}
\usepackage[htt]{hyphenat}
\usepackage{booktabs}
\usepackage{pgfplots}
\usetikzlibrary{plotmarks}
\usepgfplotslibrary{groupplots}
\pgfplotsset{compat=newest}

\newlength\figureheight
\newlength\figurewidth
\setlength\figureheight{9cm}
\setlength\figurewidth{\linewidth}

\inputencoding{utf8}

\renewcommand{\labelenumii}{\roman{enumii}.}

\setlist{  
    listparindent=\parindent
}

% reference description items
\makeatletter
\def\namedlabel#1#2{\begingroup
    #2%
    \def\@currentlabel{#2}%
    \phantomsection\label{#1}\endgroup
}
\makeatother

\captionsetup{format=hang}

% CPP style definition: %
\renewcommand{\lstlistingname}{Código}% Listing -> Código
\renewcommand{\lstlistlistingname}{Lista de \lstlistingname s}% List of Listings -> List of Algorithms
\definecolor{dkgreen}{rgb}{0,0.6,0}
\definecolor{gray}{rgb}{0.5,0.5,0.5}
\definecolor{lightgray}{rgb}{0.95, 0.95, 0.95}
\definecolor{mauve}{rgb}{0.58,0,0.82}
\definecolor{mygreen}{rgb}{0,0.6,0}
\definecolor{mygray}{rgb}{0.5,0.5,0.5}
\definecolor{mymauve}{rgb}{0.58,0,0.82}
\lstdefinestyle{CPP}{ % Estilo de lenguaje C++11
    language=[11]C++,
    frame=Lbtr,
    xleftmargin=\parindent,
    captionpos=b,
    aboveskip=3mm,
    belowskip=3mm,
    showstringspaces=false,
    columns=flexible,
    basicstyle={\small\ttfamily},
    numbers=left,
    numberstyle=\tiny\color{gray},
    keywordstyle=\color{purple},
    commentstyle=\color{gray},
    stringstyle=\color{dkgreen},
    breaklines=true,
    breakatwhitespace=true,
    tabsize=4,
    morekeywords={string,define,\#},
    otherkeywords={\#},
    backgroundcolor=\color{lightgray},
    escapeinside={/l*}{*l/}
}

\lstdefinestyle{myXML}{ % Estilo de lenguaje C++11
    language=XML,
    frame=Lbtr,
    xleftmargin=\parindent,
    captionpos=b,
    aboveskip=3mm,
    belowskip=3mm,
    showstringspaces=false,
    columns=flexible,
    basicstyle={\small\ttfamily},
    numbers=left,
    numberstyle=\tiny\color{gray},
    keywordstyle=\color{purple},
    commentstyle=\color{gray},
    stringstyle=\color{dkgreen},
    breaklines=true,
    breakatwhitespace=true,
    tabsize=4,
    morekeywords={xml,version, launch, basedir, network, seed},
    otherkeywords={\#},
    backgroundcolor=\color{lightgray},
    escapeinside={/l*}{*l/}
}

\lstdefinestyle{MyPython}{ % Estilo de lenguaje C++11
    language=Python,
    frame=Lbtr,
    xleftmargin=\parindent,
    captionpos=b,
    aboveskip=3mm,
    belowskip=3mm,
    showstringspaces=false,
    columns=flexible,
    basicstyle={\small\ttfamily},
    numbers=left,
    numberstyle=\tiny\color{gray},
    keywordstyle=\color{purple},
    commentstyle=\color{gray},
    stringstyle=\color{dkgreen},
    breaklines=true,
    breakatwhitespace=true,
    tabsize=4,
    morekeywords={traci, print},
    otherkeywords={},
    backgroundcolor=\color{lightgray},
    escapeinside={/l*}{*l/}
}

% new an instance thread
% Example:
% \newthread[edge distance]{var}{thread name}
\renewcommand{\newthread}[3][0.2]{
    \newinst[#1]{#2}{#3}
    \stepcounter{threadnum}
    \node[below of=inst\theinstnum,node distance=0.8cm] (thread\thethreadnum) {};
    \tikzstyle{threadcolor\thethreadnum}=[fill=gray!30]
    \tikzstyle{instcolor#2}=[fill=gray!30]
}

\newcommand{\blankpage}{
    \newpage
    %\thispagestyle{empty}
    \mbox{}
    \newpage
}

\newcommand{\minipagelisting}[3]{
    \noindent
    \begin{minipage}{\linewidth}
        \lstinputlisting[style=CPP, label={#3}, caption={#2}]{#1}
    \end{minipage}
}

\newcommand\mcitem[1]{\item\begin{minipage}[t]{\linewidth}#1\end{minipage}}

% footnote spacing
\setlength{\footnotesep}{.7cm}
\addbibresource{bibliografia.bib}
