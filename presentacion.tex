\documentclass[aspectratio=169]{beamer}
\usetheme[titleformat=regular, sectionpage=progressbar, subsectionpage=progressbar, progressbar=head, background=light, numbering=none]{metropolis}           % Use metropolis theme

\usepackage{multicol}
\usepackage[utf8]{inputenc}
\usepackage[T1]{fontenc}
\usepackage{bytefield}
\usepackage{subcaption}
\captionsetup{format=hang}
\usepackage{pgf-umlsd}
\usepackage{pgf-umlcd}
\usepackage[spanish]{babel}
\usepackage[style=ieee, sorting=none]{biblatex}
\usepackage{csquotes}
\usepackage{listings}
\usepackage{sourcecodepro}
\usepackage{tikz}
\usetikzlibrary{shapes.geometric, arrows}
\usepackage{tikz-qtree}
\usepackage[labelfont=bf]{caption}
\usepackage{hyperref}
\usepackage[htt]{hyphenat}
\usepackage{booktabs}
\usepackage{pgfplots}
\usetikzlibrary{plotmarks}
\usepgfplotslibrary{groupplots}
\pgfplotsset{compat=newest}
\usepackage[sfdefault]{roboto}
\usepackage{multirow}
\usepackage{textcomp}

\newlength\figureheight
\newlength\figurewidth
\setlength\figureheight{9cm}
\setlength\figurewidth{\linewidth}

% CPP style definition: %
\renewcommand{\lstlistingname}{Código}% Listing -> Código
\renewcommand{\lstlistlistingname}{Lista de \lstlistingname s}% List of Listings -> List of Algorithms
\definecolor{dkgreen}{rgb}{0,0.6,0}
\definecolor{gray}{rgb}{0.5,0.5,0.5}
\definecolor{lightgray}{rgb}{0.95, 0.95, 0.95}
\definecolor{mauve}{rgb}{0.58,0,0.82}
\definecolor{mygreen}{rgb}{0,0.6,0}
\definecolor{mygray}{rgb}{0.5,0.5,0.5}
\definecolor{mymauve}{rgb}{0.58,0,0.82}
\lstdefinestyle{CPP}{ % Estilo de lenguaje C++11
    language=[11]C++,
    frame=Lbtr,
    xleftmargin=\parindent,
    captionpos=b,
    aboveskip=3mm,
    belowskip=3mm,
    showstringspaces=false,
    columns=flexible,
    basicstyle={\small\ttfamily},
    numbers=left,
    numberstyle=\tiny\color{gray},
    keywordstyle=\color{purple},
    commentstyle=\color{gray},
    stringstyle=\color{dkgreen},
    breaklines=true,
    breakatwhitespace=true,
    tabsize=4,
    morekeywords={string,define,\#},
    otherkeywords={\#},
    backgroundcolor=\color{lightgray},
    escapeinside={/l*}{*l/}
}

\lstdefinestyle{myXML}{ % Estilo de lenguaje C++11
    language=XML,
    frame=Lbtr,
    xleftmargin=\parindent,
    captionpos=b,
    aboveskip=3mm,
    belowskip=3mm,
    showstringspaces=false,
    columns=flexible,
    basicstyle={\small\ttfamily},
    numbers=left,
    numberstyle=\tiny\color{gray},
    keywordstyle=\color{purple},
    commentstyle=\color{gray},
    stringstyle=\color{dkgreen},
    breaklines=true,
    breakatwhitespace=true,
    tabsize=4,
    morekeywords={xml,version, launch, basedir, network, seed},
    otherkeywords={\#},
    backgroundcolor=\color{lightgray},
    escapeinside={/l*}{*l/}
}

\lstdefinestyle{MyPython}{ % Estilo de lenguaje C++11
    language=Python,
    frame=Lbtr,
    xleftmargin=\parindent,
    captionpos=b,
    aboveskip=3mm,
    belowskip=3mm,
    showstringspaces=false,
    columns=flexible,
    basicstyle={\small\ttfamily},
    numbers=left,
    numberstyle=\tiny\color{gray},
    keywordstyle=\color{purple},
    commentstyle=\color{gray},
    stringstyle=\color{dkgreen},
    breaklines=true,
    breakatwhitespace=true,
    tabsize=4,
    morekeywords={traci, print},
    otherkeywords={},
    backgroundcolor=\color{lightgray},
    escapeinside={/l*}{*l/}
}

% new an instance thread
% Example:
% \newthread[edge distance]{var}{thread name}
\renewcommand{\newthread}[3][0.2]{
    \newinst[#1]{#2}{#3}
    \stepcounter{threadnum}
    \node[below of=inst\theinstnum,node distance=0.8cm] (thread\thethreadnum) {};
    \tikzstyle{threadcolor\thethreadnum}=[fill=gray!30]
    \tikzstyle{instcolor#2}=[fill=gray!30]
}

\addbibresource{bibliografia.bib}



\title{Diseño e Implementación de un Framework Integrado para la Simulación de Sistemas Inteligentes de Transporte en OMNeT++ y Paramics}
\subtitle{\emph{Memoria para optar al título de Ingeniero Civil en Computación}}
\date{Agosto 2017}
%\author{Manuel Olguín \\\href{mailto:molguin@dcc.uchile.cl}{\nolinkurl{molguin@dcc.uchile.cl}}}
\institute[Universidad de Chile]{Departamento de Ciencias de la Computación\\Facultad de Ciencias Físicas y Matemáticas, Universidad de Chile}

\author[Olguín]{\texorpdfstring{%
        \begin{minipage}{.5\textwidth}
            %\centering
            Memorista: \\
            Manuel Olguín \\
            \href{mailto:molguin@dcc.uchile.cl}{\nolinkurl{molguin@dcc.uchile.cl}} \\
        \end{minipage}%  
        \begin{minipage}{.5\textwidth}
            %\centering
            Profesora Guía: \\
            Sandra Céspedes \\
            \href{mailto:scespedes@ing.uchile.cl}{\nolinkurl{scespedes@ing.uchile.cl}} \\
    \end{minipage}}{Manuel Olguín}}


%\institute{\includegraphics[width=0.4\linewidth]{figuras/fcfm_dcc_png.png}}

\begin{document}

\maketitle

\begin{frame}{Integrantes}
\begin{minipage}{.5\textwidth}
    Memorista: \\
    Manuel Olguín, DCC\\
    
    Profesora Guía:\\
    Sandra Céspedes, DIE \\
    
    Comisión:\\
    Javier Bustos, DCC \\
    Nancy Hitschfeld, DCC
\end{minipage}%  
\begin{minipage}{.5\textwidth}
    \centering
    %\includegraphics[width=0.8\textwidth]{figuras/logo_escudo_png.png}\\
    \includegraphics[height=.25\textheight]{figuras/fcfm_horizontal_png.png}\\
    \vspace{0.05\textheight}
    \includegraphics[height=.25\textheight]{figuras/dcc_antiguo_png.png}\\
    \vspace{0.05\textheight}
    \includegraphics[height=.25\textheight]{figuras/niclabs.png}    
\end{minipage}    
\end{frame}

\begin{frame}{Organización de la defensa}
\tableofcontents
\end{frame}

\section{Motivación y Marco Teórico}
\subsection{Sistemas Inteligentes de Transporte}

\begin{frame}[standout]
\begin{quote}
    \centering
    
    ``\dots\@
    aplicaciones avanzadas que, sin incorporar inteligencia como tal, pretenden proveer servicios innovadores relacionados con distintos modos de transporte y de administración de tráfico, que además otorgan información a los usuarios, permitiéndoles utilizar el sistema de transporte de manera más segura, coordinada e inteligente\dots''\footnote{\textcite{eudirective}}
\end{quote}
\end{frame}

\begin{frame}{Sistemas Inteligentes de Transporte}
\begin{figure}
    \centering
    \includegraphics[height=0.8\textheight]{figuras/ITS.png}
    \caption{Aplicaciones en un ITS (fuente: ETSI \autocite{etsi})}
    \label{fig:itsetsi}
\end{figure}
\end{frame}

\begin{frame}{Sistemas Inteligentes de Transporte}
Factor común: recopilación y transmisión de información.

Realizado a través de la integración de comunicación inalámbrica en el sistema: LTE, 802.11p (WAVE), etc (\autocite{80211dailey,80211wave,80215vanet,dar2010wireless})
\end{frame}

\begin{frame}{Sistemas Inteligentes de Transporte}
\textbf{Problemática:} la tecnología aún está en su infancia, y existen consecuencias de esta integración que deben estudiarse previo a una implementación a gran escala \autocite{sommer2008need}:
\begin{itemize}
    \item efectos de la comunicación sobre el modelo de transporte;
    \item efectos de la topología de la red sobre las comunicaciones.
\end{itemize}
\end{frame}

\subsection{Simulaciones Bidireccionales}

\begin{frame}{Simulaciones Bidireccionales}
En el contexto de \emph{ITS}, el concepto de ``simulación bidireccional'' se refiere a un entorno de simulación que permite simular la integración de sistemas de transporte con redes de comunicación inalámbrica.
\end{frame}

\begin{frame}{Simulaciones Bidireccionales}
\begin{figure}
    \centering
    \includegraphics[width=\textwidth]{figuras/evolution_bidirectional_sim_sommerdressler.png}
    \caption{Evolución de simulaciones para ITS (fuente: Sommer \emph{et al.} \autocite{sommer_dressler2})}
    \label{fig:bidirsimulation_evol}
\end{figure}
\end{frame}

\begin{frame}{Simulaciones Bidireccionales}
\begin{figure}
    \centering
    \includegraphics[width=\textwidth]{figuras/bidirectionalsimul.png}
    \caption{Arquitectura general de una simulación bidireccional (fuente: Grzybek \emph{et al.} \autocite{trendsvanet2012})}
    \label{fig:bidirsimulation}
\end{figure}
\end{frame}

\begin{frame}{Simulaciones Bidireccionales - Ejemplos}
\begin{itemize}
    \item NCTUns -- Universidad Chiao Tung, Taiwan \autocite{nctuns6}
    \item TraNS -- Escuela Politécnica Federal de Lausanne, Suiza \autocite{piorkowski2008trans}
    \item VEINS -- Universidad de Erlangen, Alemania \autocite{sommer_german_dressler}
\end{itemize}
\end{frame}


    
\section{Especificación del Problema}
\section{Diseño e Implementación}
\section{Validación}
\section{Conclusiones}
\begin{frame}[standout]
Gracias!
\end{frame}

\begin{frame}[c,allowframebreaks]{Referencias}
\printbibliography[heading=none]
\end{frame}

\end{document}