El presente trabajo de memoria presenta el diseño, implementación y validación de un \emph{framework} de integración de un simulador de transporte, \emph{Quadstone Paramics} con un simulador de redes comunicaciones inalámbricas, \emph{OMNeT++}, para la simulación y estudio de Sistemas Inteligentes de Transporte.

Los Sistemas Inteligentes de Transporte surgen como una respuesta a la necesidad de optimización, modernización y mejoramiento de los actuales sistemas de transporte. Los Sistemas de Transporte Inteligente pretenden proveer servicios innovadores que otorguen información a los usuarios y les permitan utilizar el sistema de transporte de manera más segura, coordinada e inteligente. Resulta fundamental la recopilación y transmisión de información en estos sistemas, lo cual se realiza mediante implementación de redes comunicación inalámbrica, tanto entre vehículos como entre vehículos e infraestructura. Es necesario entonces el desarrollo de entornos de software de modelamiento y simulación de estos sistemas, para su estudio previo a su implementación en el mundo real.

Este trabajo de memoria presenta un \emph{framework} que posibilita la simulación y análisis de los Sistemas Inteligentes de Transporte. PVEINS, como se denomina el software desarrollado, permite el estudio de la integración bidireccional de un sistema de transporte con un sistema de comunicaciones inalámbricas. En ese sentido, el \emph{framework} permite determinar tanto el impacto de la comunicación entre vehículos sobre el modelo de transporte, como el impacto del movimiento de los vehículos sobre el medio de comunicación entre estos.

Adicionalmente, el presente trabajo de memoria presenta un análisis de eficiencia del software desarrollado, y un estudio para verificar su validez para la simulación de sistemas de transporte de alta complejidad. Los resultados son positivos y demuestran que PVEINS tiene el potencial para posicionarse como una opción competitiva para la simulación de Sistemas Inteligentes de Transporte en la academia. En particular, se demuestra su eficiencia para simular grandes sistemas de transporte ejecutando un escenario de 15 minutos de tiempo simulado, con aproximadamente 900 vehículos presentes en la red en cada instante de simulación, en apenas 11 minutos de tiempo real. Se demuestra también su austeridad en uso de recursos del sistema al realizar una simulación con un promedio de 1400 nodos utilizando menos de 600 MB de memoria RAM y menos del 20\% de la capacidad total del procesador. Finalmente, se expone su utilidad para el análisis y estudio de Sistemas Inteligentes de Transporte extrayendo información y estadísticas de los escenarios simulados.