\begin{thanks}
    Son muchos, tal vez demasiados, a quienes debo agradecer por su apoyo, amistad y/o cariño en el camino al obtener mi título. 
    
    En primer lugar, quiero agradecer a mi familia, quienes siempre han estado presentes y me inculcaron desde el primer momento el valor del estudio y el esfuerzo. 
    A mi padre, Gabriel Olguín, por ser mi ejemplo a seguir en el ámbito académico - me lleva un doctorado y un postdoc de ventaja, pero eventualmente lo alcanzaré. 
    A mi madre, Valeria Muñoz, por ser una constante de estabilidad emocional en mi vida y que también desde siempre me ha impulsado a alcanzar las estrellas. Me ha tenido que soportar 24 años, pero finalmente llegó este mi primer paso en mi independización.
    A mi hermana, Paola Olguín, por soportar mis estupideces todos los días -- me llena de orgullo saber que ella sigue mis pasos y que se dedicará a la misma carrera que yo. Y a mis abuelos, quienes siempre me han dado apoyo incondicional. 
    En especial, a mi Tata Caupolicán Muñoz, con quien siempre compartí profundas discusiones sobre ciencia y quien me inculcó el amor por la lectura. Yo sé que estaría hinchado de orgullo al verme recibirme de ingeniero civil.
    
    También quiero darle mis profundas gracias a los profesores que me han acompañado en este largo proceso. 
    A la profesora Sandra Céspedes, por ser probablemente la mejor profesora guía que podría haber pedido -- le pido disculpas por todos esos correos y mensajes en fines de semana, que ella respondió de manera muy cordial.
    A Javier Bustos, por haberme dado la oportunidad en su momento de realizar mi práctica en NICLabs a pesar de que no tenía las mejores calificaciones, y por haberme aceptado más tarde como miembro permanente del equipo. Espero haberle demostrado que tomó la decisión correcta.   
    A Cristián Cortés, por su invaluable apoyo durante mi trabajo de memoria, y tremendo entusiasmo con los resultados que le he presentado.
    A la profesora Nancy Hitschfeld, quien fue la primera en confiar en mis habilidades para enseñar, y me aceptó como profesor auxiliar en su curso de computación gráfica. Me abrió el mundo de la docencia, el cual me fascinó y por lo cual le estoy eternamente agradecido.
    Al profesor Jérémy Barbay, por haber sido tanto un tutor como un amigo en estos años -- gracias por los infinitos consejos, los buenos momentos y la oportunidad de desarrollar algo tan novedoso como lo fue Moulinette. 
    
    A mis amigos, quienes me han soportado durante todos estos años, en las buenas y en las malas, gracias. 
    En especial quiero nombrar a JP, Juanjo, Negro y Varas -- gracias amigos por haberme apoyado cuando necesitaba apoyo, haberme corregido cuando necesitaba ser corregido y por haberme insultado cuando necesitaba ser insultado. 
    Un abrazo también para los chiquillos del ``\emph{lolcito}''; Diego, George y el resto del equipo -- no sé que hubiese hecho estos años sin esas partidas nocturnas que me alejaban del estrés diario de los estudios. 
    Quiero además agradecer a mi \emph{polola}, Maria Collin, por su paciencia y su enorme corazón -- su cariño y apoyo me han impulsado a trabajar más duro que nunca.
    
    \emph{Last, but not least,} como dicen en inglés, quiero agradecer a todo el resto del mundo que no cupo en los párrafos anteriores -- no me he olvidado de ustedes. 
    Gracias por el apoyo, las palabras de ánimo, la buena onda. 
    A todos, de todo mi corazón: a mis amigos del colegio, a mis amigos de la universidad, a los profesores del gimnasio, al tío del kiosko a quien semana tras semana le compré bebidas energéticas para seguir adelante. 
    Gracias!\\
    
    \begin{flushright}
        \textit{Manuel Osvaldo J. Olguín Muñoz}\\
        Junio 2017
    \end{flushright}
\end{thanks}